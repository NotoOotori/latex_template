% !Mode::"TeX:UTF-8"

% -------------------- Information --------------------

\newcommand{\TITLE}{论新时代下新青年与数学学科发展}
\newcommand{\AUTHOR}{陈旭阳}
\newcommand{\SUBJECT}{毛泽东思想和中国特色社会主义理论体系概论课程论文}
\newcommand{\KEYWORDS}{新时代, 新青年, 数学学科建设}

% -------------------- Packages --------------------

\documentclass[a4paper, 12pt]{ctexart}
\usepackage{authblk} % 作者 (见校赛论文).
\usepackage{array}
\usepackage{boldline} % 长表格表格线加粗.
\usepackage{caption} % 题注.
\usepackage{enumerate}
\usepackage{fancyhdr} % 脚注.
\usepackage{filecontents}
\usepackage{float} % 你们这帮float给我乖乖听话 HHHHHHHHHHH.
\usepackage[T1]{fontenc} % Bera Mono Font
\usepackage{fontspec} % 字体.
\usepackage{graphicx}
\usepackage{hyperref}
\usepackage{lastpage}
\usepackage{makecell} % 表格线加粗 \Xhline{1.2pt}.
\usepackage[square, numbers, sort&compress]{natbib} % 引用.
\usepackage{subcaption} % subcaption and subfigure
\usepackage{titlesec} % Section标题格式.
\usepackage{varioref} % For Cross References.
\usepackage[dvipsnames]{xcolor} % 颜色声明.

% -------------------- Settings --------------------

% Title

\title{\TITLE}
\author{\AUTHOR}
\date{\today}

% Package: caption

\captionsetup{
    margin    =   6pt,
    font      =   small,
    labelfont =   bf
}

% Package: ctex

\setCJKfamilyfont{fzstk}{FZShuTi} % 方正舒体
\newcommand{\fzstk}{\CJKfamily{fzstk}}
\ctexset{
    section/name = {第,节},
    section/number = \chinese{section},
    subsection/number = (\chinese{subsection}),
    subsubsection/number = \arabic{subsubsection}
}

% Package: fancyhdr

\setlength{\headheight}{15pt}
\lhead{\SUBJECT}
\rhead{第\thepage 页\ 共\ \pageref{LastPage}\ 页}

% Package: graphicx

\graphicspath{{resources/}} % 图像文件目录

% Package: hyperref

\hypersetup{
    linktoc             =   all,
    colorlinks          =   true,
    linkcolor           =   black,
    anchorcolor         =   black,
    citecolor           =   black,
    filecolor           =   black,
    menucolor           =   black,
    runcolor            =   black,
    urlcolor            =   black,
	pdftitle           	=   {\TITLE},
	pdfauthor          	=   {\AUTHOR},
	pdfsubject         	=   {\SUBJECT},
	pdfcreator			=	{Visual Studio Code},
	pdfproducer			=	{XeLaTeX with documentclass ctexart},
	pdfkeywords        	=   {\KEYWORDS},
    bookmarksnumbered   =   true,
    pdfstartview        =   FitH,
    pdfpagelayout       =   OneColumn
}

% Package: varioref

\renewcommand{\reftextbefore}
    {on the \reftextvario{preceding page}{page before}}
\renewcommand{\reftextafter}
    {on the \reftextvario{following}{next} page}
\renewcommand{\reftextfacebefore}
    {on the \reftextvario{facing}{preceding} page}
\renewcommand{\reftextfaceafter}
    {on the \reftextvario{facing}{next} page}
\renewcommand{\reftextfaraway}[1]
    {on page \pageref{#1}}

%% Label formats

\labelformat{lstlisting}{代码#1}
\labelformat{equation}{式(#1)}
\labelformat{figure}{图#1}
\labelformat{table}{表#1}

% -------------------- General new commands --------------------



% -------------------- Specific new commands --------------------



% -------------------- Document --------------------

\begin{document}

    % -------------------- Title Page --------------------

    % \maketitle
    % \thispagestyle{empty}
    \pagenumbering{roman}

    % -------------------- Abstract Page --------------------

    % \newpage

    \begin{abstract}
        % !TeX root = ../main.tex

% 中英文摘要和关键字

\begin{abstract}{交换幺环, 代数簇, 素谱, 概型}
  代数几何历史悠久而又充满活力, 与其它数学分支有着深刻的联系, 是一门重要的学科. 代数几何包含代数与几何, 可以认为是将代数理论应用到几何的研究之中.

  代数几何的理论根基在于代数. 本文从基本定义开始建立了以环论为主模论为辅的交换代数理论, 研究了商环与分式环的基本性质, 证明了Noether环上准素分解的存在性及其满足的唯一性, 以域论为基础利用Noether正规化引理证明了域的有限生成整环上的维数定理和Hilbert零点定理.

  代数几何的研究对象在于几何. 本文研究了固定代数闭域上的仿射与射影空间中的代数集, 建立了根式理想与代数集之间的对应, 并逐步建立了代数簇与素谱之间的对应, 最终引出了概型的概念. 在此过程中, 本文将代数中的准素分解, 维数理论与几何相联系, 并将Noether模论应用到几何中.
\end{abstract}

\begin{abstract*}{commutative ring, algebraic variety, prime spectrum, scheme}
  This dissertation provides a general introduction to algebraic geometry, which is one of the oldest and the most active subjects in mathematics with deep connections to almost all other branches.

  Algebraic geometry lays its foundation on algebra. The author establishes the theory of commutative algebra with emphasis on ideals, where modules are also mentioned. Quotient rings and fraction rings have been studied, and existence and uniqueness of primary decomposition in Noetherian rings have been proved. Moreover, dimension theory of finitely generated domains has been carried out by means of Noetherian normalization lemma, with Hilbert's Nullstellensatz becoming a direct corollary.

  Algebraic geometry marks geometry as its main subject. The author researches on algebraic sets in affine and projective spaces over a fixed algebraically closed field, and establishes the order-reversing correspondence between algebraic sets and radical ideals. Then the author builds up the connection between algebraic varieties and prime spectrums step by step, with the structure of sheaves involved, which eventually leads to the concept of schemes. During this process, algebraic theories, e.g. primary decomposition, dimension theory and Noetherian module theory, have been applied to geometry.
\end{abstract*}

    \end{abstract}

    \textbf{关键词}:
        \KEYWORDS

    % -------------------- Contents --------------------

    % \newpage
    \tableofcontents

    % -------------------- Body --------------------

    \newpage
    \pagestyle{fancy}
    \pagenumbering{arabic}
    
    \section*{引言\quad 新时代下新青年当奋斗}
    \addcontentsline{toc}{section}{引言\ 新时代下新青年当奋斗}

    \subsection{问题重述}
    某学科大型国际学术会议及其附属卫星会议今年7-8月在中国召开,
    具体日程, 参会基本要求以及相关费用如下表(按照会议开始时间排序).
    \begin{table}[htb]\scriptsize
        \begin{center}
        \caption{会议日程, 基本要求和相关费用}
            \begin{tabular}{ccccccccc}
                \Xhline{1.2pt}
                \multicolumn{1}{m{1.2cm}}{\centering 会议时间}
                    & \multicolumn{1}{m{0.6cm}}{\centering 地点}
                    & \multicolumn{1}{m{1.5cm}}{\centering 住宿费用 \\ (元/人/天)}
                    & \multicolumn{1}{m{1.2cm}}{\centering 注册费用 \\ (元/人)}
                    & \multicolumn{1}{m{1.0cm}}{\centering 会场 \\ 交通费 \\ (元/人)}
                    & \multicolumn{1}{m{1.0cm}}{\centering 最低 \\ 总人数}
                    & \multicolumn{1}{m{1.2cm}}{\centering 最低(副)教授人数}
                    & \multicolumn{1}{m{1.2cm}}{\centering 最低 \\ 教授人数}
                    & \multicolumn{1}{m{1.0cm}}{\centering 会议 \\ 影响力} \\
                \hline
                7.20-7.25 & 北京 & 650 & 1000 & 200 & 3 & 2 & 2 & 5\\
                7.21-7.26 & 上海 &   0 &  900 &  80 & 3 & 1 & 1 & 5\\
                7.22-7.26 & 广州 & 550 &  500 & 150 & 2 & 1 & 0 & 4\\
                7.26-7.28 & 兰州 & 400 &  400 & 100 & 2 & 1 & 0 & 3\\
                7.26-7.28 & 成都 & 460 &  400 & 100 & 2 & 1 & 0 & 4\\
                7.29-7.31 & 昆明 & 480 &  400 & 100 & 2 & 1 & 0 & 3\\
                8.01-8.03 & 南京 & 490 &  400 & 150 & 2 & 1 & 0 & 3\\
                8.02-8.04 & 厦门 & 500 &  400 & 150 & 2 & 1 & 0 & 4\\
                8.03-8.06 & 杭州 & 500 &  400 & 200 & 2 & 1 & 0 & 3\\
                8.06-8.08 & 济南 & 450 &  400 & 100 & 2 & 1 & 0 & 4\\
                8.07-8.09 & 天津 & 480 &  400 & 100 & 2 & 1 & 0 & 3\\
                8.07-8.10 & 咸阳 & 320 &  300 & 100 & 2 & 1 & 1 & 3\\
                8.08-8.10 & 大连 & 490 &  500 & 150 & 2 & 1 & 1 & 4\\
                \Xhline{1.2pt}
            \end{tabular}
        \end{center}
    \end{table}
    
    为了了解国际最新的研究动态以及提升同济大学影响力,
    学院要求教研室组织教师积极报名参加这次会议.
    教研室中包含5名教授(包括主任和副主任), 8名副教授和5名讲师,
    其中主任和副主任至多参加三个会议, 其余老师没有限制.
    城市之间可用的交通方式为火车和飞机, 价格仅与里程相关, 并且与里程数成正比.
    
    在论文中, 我们需要一步步地给出不但总费用较小, 而且较能彰显我校在该学科影响力的参会安排.
    建模过程分为如下四个步骤(每步内的假设相互独立):
    \begin{enumerate}
        \item 在外加各教师最低参会数量的约束下, 仅考虑总费用最低, 给出此时最佳的参会安排.
        \item 在不考虑每个会议最低人员要求, 但是总费用不超过50000元的前提下,
                给出学校影响力尽可能大的参会安排.
                即多派职称高的教师去影响力大的会议.
        \item 以被选为大会报告的期望作为衡量在会议中我校学科影响力的关键因素, 简单考虑经费因素,
                给出最优方案.(参加同一会议的教师, 有至少一人的报告被选为大会报告的概率见下表)
            \begin{table}[htb]\footnotesize
                \begin{center}
                    \caption{被选为大会报告的概率}
                    \begin{tabular}{cc}
                        \Xhline{1.2pt}
                        情况 & 概率\\
                        \hline
                        有两位教授 & 0.75\\
                        有一位教授和一位副教授 & 0.50\\
                        有一位教授或两位副教授 & 0.35\\
                        其它 & 0.10\\
                        \Xhline{1.2pt}
                    \end{tabular}
                \end{center}
            \end{table}
        \item 根据之前三个步骤的结果, 全面考虑经费以及影响力因素, 得出科学的参会安排,
                并且给每一位教师打印出行日程和经费预算.
    \end{enumerate}

\subsection{解题途径}
    经检索, 题目给出的火车费用约为高铁一等座和二等座的平均值,
    因此, 我们假定题目中的火车即为高铁,
    并且统计了城市之间是否有高铁以及高铁的实际里程
    (见下表, 其中$+\infty$的里程表示高铁不可达) \citep{Huochepiao}.
    
    \begin{table}[htb]\scriptsize
        \begin{center}
            \caption{高铁里程表}
            \scalebox{0.95}{
            \begin{tabular}{cccccccccccccc}
                \Xhline{1.2pt}
                & 北京& 上海& 广州& 兰州& 成都& 昆明& 南京& 厦门& 杭州& 济南& 天津& 咸阳& 大连\\
                \hline
                北京\\
                上海& 1318\\
                广州& 2298& 1790\\
                兰州& 1784& 2185& 2687\\
                成都& 1874& 1976& $+\infty$& $+\infty$\\
                昆明& 2760& 2252& 1330& $+\infty$& 1112\\
                南京& 1023&  295& 1581& 1782& 1872& 2349\\
                厦门& 2053& 1085& $+\infty$& $+\infty$& $+\infty$& 2337& 1182\\
                杭州& 1279&  159& 1631& 2038& $+\infty$& 2093& 256& 993\\
                济南&  406&  912& 2251& 1737& 1830& 2713& 608& 1647& 873\\
                天津&  120& 1196& 2309& $+\infty$& 1885& $+\infty$& 901& 1931& 1157& 284\\
                咸阳& 1246& 1539& $+\infty$& 538& $+\infty$& $+\infty$& 1244& $+\infty$& $+\infty$
                    & 1199& $+\infty$\\
                大连&  963& 2054& $+\infty$& $+\infty$& $+\infty$& $+\infty$& 1759& $+\infty$& $+\infty$
                    & 1142&  836& $+\infty$\\
                \Xhline{1.2pt}
            \end{tabular}}
        \end{center}
    \end{table}
    
    飞机方面, 因为未能检索到航班的具体里程, 并且考虑到飞机飞行的高度相对于地球半径而言可以忽略不计,
    所以我们在百度地图上$1:20,000,000$的比例尺下,
    用工具箱中的测距工具测得了两个城市之间的球面距离,
    记录在下表(并假设任意两城市间都存在密集的航班) \citep{Baidu_Map}.
    
    \begin{table}[htb]\scriptsize
        \begin{center}
            \caption{飞机里程表}
            \scalebox{0.95}{
            \begin{tabular}{cccccccccccccc}
                \Xhline{1.2pt}
                & 北京& 上海& 广州& 兰州& 成都& 昆明& 南京& 厦门& 杭州& 济南& 天津& 咸阳& 大连\\
                \hline
                北京\\
                上海& 1070\\
                广州& 1893& 1216\\
                兰州& 1182& 1716& 1697\\
                成都& 1528& 1669& 1232&  620\\
                昆明& 2106& 1955& 1068& 1245& 645\\
                南京&  898&  270& 1130& 1451& 1402& 1746\\
                厦门& 1726&  828&  513& 1878& 1536& 1545&  845\\
                杭州& 1121&  174& 1043& 1650& 1540& 1807&  235& 678\\
                济南&  360&  733& 1550& 1196& 1380& 1889&  535& 1357& 762\\
                天津&  107&  963& 1811& 1222& 1518& 2080&  793& 1629& 1014&  271\\
                咸阳&  921& 1250& 1321&  481&  598& 1191&  968& 1417& 1167&  801&  922\\
                大连&  461&  853& 1928& 1598& 1842& 2346&  799& 1631&  967&  472&  380& 1259\\
                \Xhline{1.2pt}
            \end{tabular}}  
        \end{center}
    \end{table}
    \clearpage
    
    本题的各个任务均为规划问题, 要求我们在满足一定约束的情况下,
    寻求既影响力高, 又能照顾到经费的最佳方案.
    考虑到路费是一个离散量, 因此零一整数规划模型在本题并不适用,
    于是我们想到了另一个常用的优化方法---动态规划模型.
    现代意义的动态规划由Richard E. Bellman在1953年提出,
    指将大决策分解成嵌套在其中的小决策, 并从小到大递推求解的方法 \citep{R.Bellman_2002}.
    动态规划中, 有如下的一些基本概念:\citep{H.W.Zhang_2008}
    \begin{itemize}
        \item \textbf{阶段}为系统需要做出决策的步骤.
                我们需要把系统顺序地向前发展划分为若干个相互联系的阶段,
                使能按阶段的次序求解.
                描述阶段的变量称为\textbf{阶段变量}.
                阶段变量的主要作用是按顺序编出所研究过程划分的编号.
        \item \textbf{状态}为每个阶段开始时面临的自然状况或客观条件.
                过程的状态通常可以用一个或一组变数描述,称为\textbf{状态变量}.
                状态变量取值的集合称为\textbf{状态集合}.
                此外, 状态具有\textbf{无后效性}: 未来的收益仅取决于当前的状态,
                并不依赖于过去的状态和决策的历史.
        \item \textbf{决策}为从当前阶段的状态到下一阶段的状态时所做的决定.
                描述决策的变量称为\textbf{决策变量}.
                所以, 系统的状态必须包含在某个给定的阶段上, 
                并且确定全部允许决策所需的全部信息.
        \item \textbf{策略}是由每一阶段的决策组成的决策函数序列.
                对于每个实际的多阶段决策过程,
                可供选取的策略有一定的范围限制,
                这个范围称为\textbf{允许策略集合},
                允许策略集合中达到最优效果的策略称为\textbf{最优策略}.
        \item 利用动态规划解决优化问题时, 所研究的是逐阶段的决策过程.
                给定第$k$阶段状态变量的值后, 如果这一阶段的决策变量一经确定,
                第$k+1$阶段的状态变量也就完全确定,
                即$k+1$阶段的状态变量随$k$阶段的状态变量和策略的变化而变化,
                可以把这一关系看成$k$阶段的状态变量和策略与$k+1$阶段的状态变量确定的对应关系.
                这是从$k$阶段到$k+1$阶段的状态转移规律,
                称为\textbf{状态转移方程}.
    \end{itemize}
    
    任务一和任务二中, 我们均采用了动态规划, 分别建立了费用最优化模型和影响力最优化模型.
    并在任务二中, 我们通过对问题的抽象,
    将问题转化为了动态规划中的分组背包模型(不同于常见的分组01背包模型).
    
    任务三中, 我们定义了星级期望, 建立了影响力期望最优化模型,
    并采用贪心算法, 从被选为大会报告的概率高的老师组合先考虑,
    再以概率从大到小考虑其余的老师组合.

    \section{新时代下数学学科建设的必要性}

    \subsection{国家发展需要数学}

中国的现代化发展离不开数学, 各个产业的现代化都以数学作为基础. 经济, 金融, 大数据等方向与数学的关系比较明显, 在这里我们以农业现代化举例.

农业现代化首先是监测手段的现代化. 以前农民只能通过经验进行进行灌溉, 而现在有天气预报能较为准确地预测未来的下雨情况, 有摄像头能拍摄并自动判断作物状态. 这一切都离不开数学的支持. 我们知道大气的运动是按照著名的偏微分方程Navier-Stokes方程运作的, 而天气预报实质上就是对该方程的数值模拟. 通过从数学理论上研究方程解的性质和各种数值计算方法的效率和稳定性都可以提升天气预报的即时性和准确率. 而智能识别作物状态属于计算机视觉方面, 现在常用深度学习的方法来实现, 而到现在为止深度学习算法仍需要数学理论证明其稳定性. 因此数学上的研究能指导智能识别作物状态的实践.

农业现代化其次是农业设备的现代化, 其中包括自动化与信息化. 自动化离不开数学上的系统控制理论, 正如2019年5月10日下午中国科学院数学与系统科学研究所郭雷院士在同济大学高等讲堂第78讲中所展现的那样, 控制论可以很好地应用在农业产业中. 可是实际操作中, 许多参数的选取都是仅凭借经验得到的, 没有经过理论上的推导. 所以从理论上的进一步研究可以增强自动化的效果和稳定性.信息化涉及到电气与通信方面, 比如现在比较热门的5G概念就是一种信息化的解决方案. 通信的底层是编码, 而编码实际上也需要深刻的数学理论. 任正非曾说华为的技术是以土耳其科学家艾利坎的极化码编制理论的基础上发展起来的. 艾利坎根据信息传输的极化现象, 构建了这套编码理论, 在理论上这样编码信号传输可以达到香农上限. 相反Turbo编码以及高通的LPDC编码由于其理论基础没有极化码编制理论那么完美, 更多的是实践中的产物, 从而限制了未来的应用空间. 可见一项技术背后的数学理论背景有多么重要.

\subsection{中国需要建设好数学学科}

数学是一门重视传承的学科, 历史上常发生一整个学派在某个方向均有结出成果的情况. 比如近代俄罗斯的圣彼得堡数学学派, 包括切比雪夫, 马尔科夫和李雅普诺夫, 他们在概率论等方向均作出了杰出的贡献. 中国也有解析数论学派, 以华罗庚为首, 包括陈景润, 王元, 潘承洞等人, 他们在哥德巴赫猜想等问题上作出了杰出的成果.

并且当前中美展开了贸易战, 美国有可能通过拒绝中国数学学生乃至研究人员的签证来阻断学术交流, 从而打击中国的数学学科.

因此想要发展数学, 就需要中国自己建立优秀的数学系, 建立优秀的数学研究所, 招募优秀的教师, 吸引并培养优秀的毕业生, 以此良性循环, 使学科越来越强. 进行学科建设是有必要的, 这是因为现在中国的数学学科乃至基础学科存在以下问题.

第一, 毕业生待遇不高. 一般数学博士毕业后只能在大学获得博士后或者助理教授的工作, 工资收入相比起统计, 金融, 软件等热门行业差了许多, 而且工作也不稳定, 如果不能及时发表优秀的文章就可能丢掉工作. 所以目前数学学科的吸引力并不是很高, 只有真正热爱数学的人才会走上这一条道路. 哪怕是大学本科是数学专业的同学, 也只有很少的一部分会继续修读数学, 这之中又只有很少的一部分最后会从事数学相关的工作, 这其中有大量的人才流失.

第二, 教授的琐事很多, 难以专心学术. 实际上一个教授在大学数学系的整个数学生涯, 做出最好成果的时间段一般是刚进入大学获得教职的时候. 那时候他们可以专心搞学术, 不用把过多的精力放在评职称申请经费拉项目等等琐事上, 因此可以做出较好的成果.

我觉得提高数学从业人员的待遇, 减少数学工作以外过多的琐事是建设数学学科的好方法.


    \section{新青年应立下志向努力学习}

    数学家一生中的大成果多出于青年时代. 习总书记在五四运动100周年大会上的讲话中提到牛顿和莱布尼茨发现微积分时分别是22岁和28岁. 我们也知道年仅26岁的阿贝尔提出了群的理论证明了5次代数方程没有求根公式, 结论很震撼, 所用方法也统领了新的数学分支的发展, 现在交换群就以阿贝尔的名字命名, 称为阿贝尔群. 年仅20岁的伽罗瓦提出了伽罗瓦理论, 将代数方程的解与循环群联系起来, 被推广到数学的其他领域.

中国近现代的一些著名数学家也是如此. 陈省身出生于1911年, 他最大的贡献有两个, 第一个是在1945年(34岁)推广了Gauss-Bornet定理到高维情形, 现称为Chern-Gauss-Bornet定理, 该定理陈省身的名字写在赫赫有名的高斯之前也足以表明他作出的贡献之大. 第二个是在1946年(35岁)提出了陈氏类, 该概念广泛应用在物理学, Calabi-Yau流形等领域. 陈景润出生于1933年, 他最大的贡献是在1966年(33岁)证明了"$1 + 2$", 是用筛法做哥德巴赫猜想的最好结果, 现称为陈氏定理.

数学大家们能在青年时代就做出大成果离不开他们早年立下坚定志向, 并努力学习积淀, 最后他们能捕捉到不易遇见的微光, 想出别人想不到的思路, 从而做出举世瞩目的成果.

诚然, 能成为数学大家的人是极少数人, 但是数学学科发展也需要默默付出, 给大牛们补充细节计算例子, 完善数学大厦的普通人. 中国需要有青年人立下坚定志向从事数学事业, 并且能在本科阶段就多学习学校课程以外的内容, 多接触现代数学, 这样能更早地进入研究状态, 从而能对新时代中国的数学学科作出更多的贡献.


    \section*{结语}
    \addcontentsline{toc}{section}{结语}

    针对问题一, 本文设计了一个基于快速聚类算法的静态推荐模型,
实现了推送给电视频道与其相似度高的广告.

针对问题二, 本文通过分析设置保留价对竞价带来的变化, 退出保留价应与卖方估价相近,
从而得到合理保留价.

针对问题三, 本文设计了一个基于协同过滤和深度学习的动态推荐模型,
比传统算法能更有效地去刻画广告和用户的特征.

针对问题四, 本文从经济机制理论出发, 推广了格罗夫斯机制,
结合本问题的经济环境得到了具体的经济机制.


    % -------------------- Bibliography --------------------

    \newpage
    \bibliography{bibliography}
    \bibliographystyle{plain}

\end{document}

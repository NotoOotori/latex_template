\subsection{国家发展需要数学}

中国的现代化发展离不开数学, 各个产业的现代化都以数学作为基础. 经济, 金融, 大数据等方向与数学的关系比较明显, 在这里我们以农业现代化举例.

农业现代化首先是监测手段的现代化. 以前农民只能通过经验进行进行灌溉, 而现在有天气预报能较为准确地预测未来的下雨情况, 有摄像头能拍摄并自动判断作物状态. 这一切都离不开数学的支持. 我们知道大气的运动是按照著名的偏微分方程Navier-Stokes方程运作的, 而天气预报实质上就是对该方程的数值模拟. 通过从数学理论上研究方程解的性质和各种数值计算方法的效率和稳定性都可以提升天气预报的即时性和准确率. 而智能识别作物状态属于计算机视觉方面, 现在常用深度学习的方法来实现, 而到现在为止深度学习算法仍需要数学理论证明其稳定性. 因此数学上的研究能指导智能识别作物状态的实践.

农业现代化其次是农业设备的现代化, 其中包括自动化与信息化. 自动化离不开数学上的系统控制理论, 正如2019年5月10日下午中国科学院数学与系统科学研究所郭雷院士在同济大学高等讲堂第78讲中所展现的那样, 控制论可以很好地应用在农业产业中. 可是实际操作中, 许多参数的选取都是仅凭借经验得到的, 没有经过理论上的推导. 所以从理论上的进一步研究可以增强自动化的效果和稳定性.信息化涉及到电气与通信方面, 比如现在比较热门的5G概念就是一种信息化的解决方案. 通信的底层是编码, 而编码实际上也需要深刻的数学理论. 任正非曾说华为的技术是以土耳其科学家艾利坎的极化码编制理论的基础上发展起来的. 艾利坎根据信息传输的极化现象, 构建了这套编码理论, 在理论上这样编码信号传输可以达到香农上限. 相反Turbo编码以及高通的LPDC编码由于其理论基础没有极化码编制理论那么完美, 更多的是实践中的产物, 从而限制了未来的应用空间. 可见一项技术背后的数学理论背景有多么重要.

\subsection{中国需要建设好数学学科}

数学是一门重视传承的学科, 历史上常发生一整个学派在某个方向均有结出成果的情况. 比如近代俄罗斯的圣彼得堡数学学派, 包括切比雪夫, 马尔科夫和李雅普诺夫, 他们在概率论等方向均作出了杰出的贡献. 中国也有解析数论学派, 以华罗庚为首, 包括陈景润, 王元, 潘承洞等人, 他们在哥德巴赫猜想等问题上作出了杰出的成果.

并且当前中美展开了贸易战, 美国有可能通过拒绝中国数学学生乃至研究人员的签证来阻断学术交流, 从而打击中国的数学学科.

因此想要发展数学, 就需要中国自己建立优秀的数学系, 建立优秀的数学研究所, 招募优秀的教师, 吸引并培养优秀的毕业生, 以此良性循环, 使学科越来越强. 进行学科建设是有必要的, 这是因为现在中国的数学学科乃至基础学科存在以下问题.

第一, 毕业生待遇不高. 一般数学博士毕业后只能在大学获得博士后或者助理教授的工作, 工资收入相比起统计, 金融, 软件等热门行业差了许多, 而且工作也不稳定, 如果不能及时发表优秀的文章就可能丢掉工作. 所以目前数学学科的吸引力并不是很高, 只有真正热爱数学的人才会走上这一条道路. 哪怕是大学本科是数学专业的同学, 也只有很少的一部分会继续修读数学, 这之中又只有很少的一部分最后会从事数学相关的工作, 这其中有大量的人才流失.

第二, 教授的琐事很多, 难以专心学术. 实际上一个教授在大学数学系的整个数学生涯, 做出最好成果的时间段一般是刚进入大学获得教职的时候. 那时候他们可以专心搞学术, 不用把过多的精力放在评职称申请经费拉项目等等琐事上, 因此可以做出较好的成果.

我觉得提高数学从业人员的待遇, 减少数学工作以外过多的琐事是建设数学学科的好方法.

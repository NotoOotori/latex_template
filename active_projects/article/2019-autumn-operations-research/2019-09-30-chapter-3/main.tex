% -------------------- Packages --------------------

\documentclass{assignment}[2019/09/15]
\usepackage[lineno, pgfplots]{packages}[2019/09/15]
\pgfplotsset{compat=1.3}
\usetikzlibrary{plotmarks}
\usetikzlibrary{arrows.meta}
\usepgfplotslibrary{patchplots}
\usepackage{grffile}

% -------------------- Settings --------------------

% Title

\title{Homework of Chapter 3}
\author{Chen Xuyang}
\date{\today}
\institute{School of Mathematical Science}
\professor{Chen Xiongda}
\course{Operations Research}
\subject{Operations Research}
\keywords{}

% -------------------- New commands --------------------

\newcommand{\diag}{\mathop{}\!\symup{diag}}
\newcommand{\pr}{\mathop{}\!\symup{Pr}}
\newcommand{\expect}{\mathop{}\!\symup{E}}
\newcommand{\cov}{\mathop{}\!\symup{Cov}}
\newcommand{\var}{\mathop{}\!\symup{Var}}

\newcommand{\dfd}{\nabla f_k^Td_k}
\newcommand{\ak}{\alpha_k}
\newcommand{\nbi}{\norm{\matr{B}^{-1}}}
\newcommand{\nbki}{\norm{\matr{B}_k^{-1}}}
\newcommand{\nbk}{\norm{\matr{B}_k}}
\newcommand{\lxpx}{\frac{1}{2}\left(x_1+x_2\right)}
\newcommand{\blxpx}{\left(\lxpx\right)}
\newcommand{\lxmx}{\frac{1}{2}\left(x_1-x_2\right)}
\newcommand{\blxmx}{\left(\lxmx\right)}
\newcommand{\df}{\nabla f}
\newcommand{\nqs}[1]{\norm{#1}_{\matr{Q}}^2}
\newcommand{\dfdf}{\nabla f_k^T\nabla f_k}
\newcommand{\dfqdf}{\nabla f_k^T\matr{Q}\nabla f_k}
\newcommand{\dfqidf}{\nabla f_k^T\matr{Q}^{-1}\nabla f_k}

% -------------------- Document --------------------

\begin{document}
    \maketitle

    \begin{problem}\label{prob:sd}
        Program the steepest descent and Newton algorithms using the backtracking line search, Algorithm 3.1. Use them to minimize the Rosenbrock function
        \begin{equation}
            f(x) = 100(x_2-{x_1}^2)^2+(1-x_1)^2.
        \end{equation}
        Set the initial step length $\alpha_0=1$ and print the step length used by each method at each iteration. First try the initial point $x_0=(1.2, 1.2)^T$ and then the more diffcult starting point $x_0=(-1.2, 1)$.
    \end{problem}
    \begin{solution}
        The results of steepest descent method and Newton method are listed below, while the codes \ref{code:sd} and \ref{code:nt} are put in the appendix. From this example we can see if the contour lines of the target function (like the one in \ref{fig:rosenbrock}) are extremely flat, the performance of steepest descent method is not satisfying. Newton method is always reliable though.
        \begin{table}[htb]
            \begin{center}
                \caption{Total iterations of steepest descent method and Newton method with different initial points and with $\rho=0.9$, $c=0.5$ and $\varepsilon=0.001$.}
                \pgfplotstabletypeset[
                    string type,
                ]{
                    {Method} {Initial Point} {Total Iterations}
                    {Steepest Descent} {$(1.2, 1.2)^T$} 680
                    {Steepest Descent} {$(-1.2, 1)^T$} 848
                    {Newton} {$(1.2, 1.2)^T$} 7
                    {Newton} {$(-1.2, 1)^T$} 21
                }
            \end{center}
        \end{table}
        \begin{figure}[htb]
            \centering
            % This file was created by matlab2tikz.
%
%The latest updates can be retrieved from
%  http://www.mathworks.com/matlabcentral/fileexchange/22022-matlab2tikz-matlab2tikz
%where you can also make suggestions and rate matlab2tikz.
%
\begin{tikzpicture}

\begin{axis}[%
width=4.069in,
height=3.566in,
at={(0.758in,0.481in)},
scale only axis,
point meta min=0,
point meta max=1023.9990234375,
colormap={mymap}{[1pt] rgb(0pt)=(0,0,0.666667); rgb(1pt)=(0,0,1); rgb(4pt)=(0,1,1); rgb(7pt)=(1,1,0); rgb(9pt)=(1,0.333333,0)},
xmin=-2,
xmax=2,
ymin=-2,
ymax=2,
axis background/.style={fill=white},
legend style={legend cell align=left, align=left, draw=white!15!black}
]
\addplot[contour prepared, contour prepared format=matlab, contour/labels=false] table[row sep=crcr] {%
%
0.0009765625	5\\
0.980139843457652	0.96\\
0.98	0.959480960648148\\
0.979848353171919	0.96\\
0.98	0.960609307065217\\
0.980139843457652	0.96\\
0.0009765625	5\\
0.990219444722926	0.98\\
0.99	0.979141605392157\\
0.989773844117267	0.98\\
0.99	0.980893431122449\\
0.990219444722926	0.98\\
0.0009765625	5\\
1.00024112058962	1\\
1	0.9990234375\\
0.99975402067958	1\\
1	1.0009765625\\
1.00024112058962	1\\
0.0009765625	5\\
1.01020889001551	1.02\\
1.01	1.01914160539216\\
1.00978055527707	1.02\\
1.01	1.02089343112245\\
1.01020889001551	1.02\\
0.0009765625	5\\
1.02012692460093	1.04\\
1.02	1.03948096064815\\
1.01985730876925	1.04\\
1.02	1.04060930706522\\
1.02012692460093	1.04\\
0.001953125	5\\
0.960022732592159	0.92\\
0.96	0.919926420454545\\
0.959968938885158	0.92\\
0.96	0.920142830882353\\
0.960022732592159	0.92\\
0.001953125	5\\
0.970236901425612	0.94\\
0.97	0.939176165254237\\
0.969718020304569	0.94\\
0.97	0.941185518292683\\
0.970236901425612	0.94\\
0.001953125	5\\
0.980383466383934	0.96\\
0.98	0.958576736111111\\
0.979584167455701	0.96\\
0.98	0.961670788043478\\
0.980383466383934	0.96\\
0.001953125	5\\
0.990464203363493	0.98\\
0.99	0.978184191176471\\
0.989521600154979	0.98\\
0.99	0.981889923469388\\
0.990464203363493	0.98\\
0.001953125	5\\
1.00048224117923	1\\
1	0.998046875\\
0.99950804135916	1\\
1	1.001953125\\
1.00048224117923	1\\
0.001953125	5\\
1.01044187641656	1.02\\
1.01	1.01818419117647\\
1.00953579663651	1.02\\
1.01	1.02188992346939\\
1.01044187641656	1.02\\
0.001953125	5\\
1.02034804143553	1.04\\
1.02	1.03857673611111\\
1.01960872470409	1.04\\
1.02	1.04167078804348\\
1.02034804143553	1.04\\
0.001953125	5\\
1.03020563194077	1.06\\
1.03	1.05917616525424\\
1.02974306197965	1.06\\
1.03	1.06118551829268\\
1.03020563194077	1.06\\
0.001953125	5\\
1.0400189441963	1.08\\
1.04	1.07992642045455\\
1.03997266009852	1.08\\
1.04	1.08014283088235\\
1.0400189441963	1.08\\
0.00390625	5\\
0.950173107176885	0.9\\
0.95	0.899479166666667\\
0.949714570165504	0.9\\
0.95	0.9015625\\
0.950173107176885	0.9\\
0.00390625	5\\
0.960479871269748	0.92\\
0.96	0.91844678030303\\
0.959344318654258	0.92\\
0.96	0.923015073529412\\
0.960479871269748	0.92\\
0.00390625	5\\
0.970712867064701	0.94\\
0.97	0.937520974576271\\
0.969151486584482	0.94\\
0.97	0.943567378048781\\
0.970712867064701	0.94\\
0.00390625	5\\
0.980870712236497	0.96\\
0.98	0.956768287037037\\
0.979055796023265	0.96\\
0.98	0.96379375\\
0.980870712236497	0.96\\
0.00390625	5\\
0.990953720644628	0.98\\
0.99	0.976269362745098\\
0.989017112230402	0.98\\
0.99	0.983882908163266\\
0.990953720644628	0.98\\
0.00390625	5\\
1.00096448235846	1\\
1	0.99609375\\
0.999016082718319	1\\
1	1.00390625\\
1.00096448235846	1\\
0.00390625	5\\
1.01090784921866	1.02\\
1.01	1.0162693627451\\
1.00904627935537	1.02\\
1.01	1.02388290816326\\
1.01090784921866	1.02\\
0.00390625	5\\
1.02079027510472	1.04\\
1.02	1.03676828703704\\
1.01911155657376	1.04\\
1.02	1.04379375\\
1.02079027510472	1.04\\
0.00390625	5\\
1.03061877313591	1.06\\
1.03	1.05752097457627\\
1.0292268402273	1.06\\
1.03	1.06356737804878\\
1.03061877313591	1.06\\
0.00390625	5\\
1.04039990052468	1.08\\
1.04	1.07844678030303\\
1.03942287121745	1.08\\
1.04	1.08301507352941\\
1.04039990052468	1.08\\
0.00390625	5\\
1.05013908422496	1.1\\
1.05	1.09947916666667\\
1.04975835885064	1.1\\
1.05	1.1015625\\
1.05013908422496	1.1\\
0.0078125	5\\
0.920060036073177	0.85\\
0.92	0.849583928571429\\
0.919975912333299	0.85\\
0.92	0.850067732558139\\
0.920060036073177	0.85\\
0.0078125	7\\
0.930098374843735	0.86\\
0.93	0.859741666666667\\
0.929709292412618	0.86\\
0.929942943492994	0.87\\
0.93	0.870154207920792\\
0.930213429256595	0.87\\
0.930098374843735	0.86\\
0.0078125	7\\
0.940604719152377	0.88\\
0.94	0.878304360465116\\
0.938721113790835	0.88\\
0.93998064944772	0.89\\
0.94	0.890051096491228\\
0.940111707738038	0.89\\
0.940604719152377	0.88\\
0.0078125	5\\
0.95103864306131	0.9\\
0.95	0.896875\\
0.948287420993022	0.9\\
0.95	0.909375000000001\\
0.95103864306131	0.9\\
0.0078125	5\\
0.961394148624927	0.92\\
0.96	0.9154875\\
0.958095078192459	0.92\\
0.96	0.928759558823529\\
0.961394148624927	0.92\\
0.0078125	5\\
0.971664798342878	0.94\\
0.97	0.934210593220339\\
0.968018419144308	0.94\\
0.97	0.948331097560976\\
0.971664798342878	0.94\\
0.0078125	5\\
0.981845203941624	0.96\\
0.98	0.953151388888889\\
0.977999053158393	0.96\\
0.98	0.968039673913043\\
0.981845203941624	0.96\\
0.0078125	5\\
0.991932755206898	0.98\\
0.99	0.972439705882353\\
0.988008136381248	0.98\\
0.99	0.987868877551021\\
0.991932755206898	0.98\\
0.0078125	5\\
1.00192896471692	1\\
1	0.9921875\\
0.998032165436639	1\\
1	1.0078125\\
1.00192896471692	1\\
0.0078125	5\\
1.01183979482286	1.02\\
1.01	1.01243970588235\\
1.0080672447931	1.02\\
1.01	1.02786887755102\\
1.01183979482286	1.02\\
0.0078125	5\\
1.02167474244311	1.04\\
1.02	1.03315138888889\\
1.0181172203131	1.04\\
1.02	1.04803967391304\\
1.02167474244311	1.04\\
0.0078125	5\\
1.03144505552618	1.06\\
1.03	1.05421059322034\\
1.02819439672261	1.06\\
1.03	1.06833109756097\\
1.03144505552618	1.06\\
0.0078125	5\\
1.04116181318145	1.08\\
1.04	1.0754875\\
1.03832329345531	1.08\\
1.04	1.08875955882353\\
1.04116181318145	1.08\\
0.0078125	5\\
1.05083450534974	1.1\\
1.05	1.096875\\
1.04855015310383	1.1\\
1.05	1.109375\\
1.05083450534974	1.1\\
0.0078125	7\\
1.06047036529312	1.12\\
1.06	1.11830436046512\\
1.05896684260866	1.12\\
1.05998345851851	1.13\\
1.06	1.13005109649123\\
1.06006003607318	1.13\\
1.06047036529312	1.12\\
0.0078125	7\\
1.07007435133367	1.14\\
1.07	1.13974166666667\\
1.06977947833585	1.14\\
1.0699526559769	1.15\\
1.07	1.15015420792079\\
1.07012075983718	1.15\\
1.07007435133367	1.14\\
0.0078125	5\\
1.08003512739334	1.17\\
1.08	1.16958392857143\\
1.07998064944772	1.17\\
1.08	1.17006773255814\\
1.08003512739334	1.17\\
0.015625	5\\
0.890823300141488	0.79\\
0.89	0.787828169014085\\
0.888822676083222	0.79\\
0.89	0.795317241379311\\
0.890823300141488	0.79\\
0.015625	5\\
0.901822688830563	0.81\\
0.9	0.804375\\
0.898352420842975	0.81\\
0.9	0.815625\\
0.901822688830563	0.81\\
0.015625	7\\
0.910156913811345	0.82\\
0.91	0.819632061068702\\
0.908194418430419	0.82\\
0.908275355689834	0.83\\
0.91	0.835191304347827\\
0.912884638614858	0.83\\
0.910156913811345	0.82\\
0.015625	7\\
0.920909316549951	0.84\\
0.92	0.837750438596491\\
0.915640458988525	0.84\\
0.91836059133671	0.85\\
0.92	0.85460988372093\\
0.924086060293739	0.85\\
0.920909316549951	0.84\\
0.015625	73\\
0.931600923165689	0.86\\
0.93	0.85579595959596\\
0.925269110542768	0.86\\
0.928511951643924	0.87\\
0.93	0.874021782178218\\
0.935295329971497	0.88\\
0.938682999750851	0.89\\
0.94	0.893477631578947\\
0.945433122648058	0.9\\
0.948849181384358	0.91\\
0.95	0.913\\
0.95559659726886	0.92\\
0.958996818724899	0.93\\
0.96	0.932600373134328\\
0.96575228426396	0.94\\
0.969118166267533	0.95\\
0.97	0.952285106382979\\
0.975885567428649	0.96\\
0.979208977818732	0.97\\
0.98	0.972057876712329\\
0.98599018468294	0.98\\
0.989267234205979	0.99\\
0.99	0.991920805369128\\
0.996064330873278	1\\
0.999292461730041	1.01\\
1	1.011875\\
1.00610917566856	1.02\\
1.00928538433688	1.03\\
1.01	1.03192080536913\\
1.01612854779178	1.04\\
1.01924779370345	1.05\\
1.02	1.05205787671233\\
1.02612950971323	1.06\\
1.02918259656244	1.07\\
1.03	1.07228510638298\\
1.03612413793103	1.08\\
1.03909405264868	1.09\\
1.04	1.09260037313433\\
1.0473598056817	1.09\\
1.042685638495	1.08\\
1.04006113743567	1.07\\
1.04	1.06982861445783\\
1.03309762030672	1.06\\
1.03032072173668	1.05\\
1.03	1.04910566037736\\
1.02344367711989	1.04\\
1.02051770093348	1.03\\
1.02	1.02856850649351\\
1.01370368603125	1.02\\
1.01064521602133	1.01\\
1.01	1.00823708609271\\
1.00385792943384	1\\
1.00069701738516	0.99\\
1	0.988125\\
0.993890824331437	0.98\\
0.990668013400419	0.97\\
0.99	0.968237086092716\\
0.983794187351877	0.96\\
0.980554695854564	0.95\\
0.98	0.948568506493506\\
0.973568660899232	0.94\\
0.970355344536765	0.93\\
0.97	0.929105660377359\\
0.963222703335284	0.92\\
0.96006996618506	0.91\\
0.96	0.909828614457831\\
0.952769714830161	0.9\\
0.95	0.891666666666667\\
0.947602838239524	0.89\\
0.942224595160588	0.88\\
0.94	0.873762209302326\\
0.935566289825283	0.87\\
0.931600923165689	0.86\\
0.015625	7\\
1.05222534759929	1.1\\
1.05	1.09166666666667\\
1.04613374161022	1.1\\
1.04898827750873	1.11\\
1.05	1.113\\
1.05536826283015	1.11\\
1.05222534759929	1.1\\
0.015625	7\\
1.06173034432707	1.12\\
1.06	1.11376220930233\\
1.05619929859365	1.12\\
1.05887418534978	1.13\\
1.06	1.13347763157895\\
1.06408606029374	1.13\\
1.06173034432707	1.12\\
0.015625	7\\
1.07120997165492	1.14\\
1.07	1.13579595959596\\
1.06641129553783	1.14\\
1.06876525571852	1.15\\
1.07	1.15402178217822\\
1.07314944756736	1.15\\
1.07120997165492	1.14\\
0.015625	7\\
1.08066988833018	1.16\\
1.08	1.15775043859649\\
1.0770188898576	1.16\\
1.07868299975085	1.17\\
1.08	1.17460988372093\\
1.0823907734057	1.17\\
1.08066988833018	1.16\\
0.015625	7\\
1.09011296124867	1.18\\
1.09	1.1796320610687\\
1.08905813385442	1.18\\
1.0886643050247	1.19\\
1.09	1.19519130434783\\
1.09172464431017	1.19\\
1.09011296124867	1.18\\
0.015625	5\\
1.10110421860584	1.21\\
1.1	1.204375\\
1.09877879333927	1.21\\
1.1	1.215625\\
1.10110421860584	1.21\\
0.015625	5\\
1.11050232103591	1.23\\
1.11	1.22782816901408\\
1.10917669985851	1.23\\
1.11	1.23531724137931\\
1.11050232103591	1.23\\
0.03125	5\\
0.831065742290222	0.69\\
0.83	0.687142307692308\\
0.829351186144666	0.69\\
0.83	0.691827049180328\\
0.831065742290222	0.69\\
0.03125	7\\
0.840566356529771	0.7\\
0.84	0.698814150943396\\
0.837986383660392	0.7\\
0.839190585158548	0.71\\
0.84	0.711975531914893\\
0.84350741335348	0.71\\
0.840566356529771	0.7\\
0.03125	173\\
0.852328680748617	0.72\\
0.85	0.714583333333334\\
0.846499504545259	0.72\\
0.849451860167336	0.73\\
0.85	0.73125\\
0.856528096929183	0.74\\
0.859876825828176	0.75\\
0.86	0.750270779220779\\
0.86691177488179	0.76\\
0.87	0.768264814814815\\
0.871585895627644	0.77\\
0.877398956851588	0.78\\
0.88	0.786468867924528\\
0.882857797289559	0.79\\
0.887904213004302	0.8\\
0.89	0.805003488372093\\
0.89377581207346	0.81\\
0.89839150140833	0.82\\
0.9	0.82375\\
0.904513830376273	0.83\\
0.9088436949279	0.84\\
0.91	0.842659467455621\\
0.915129949343534	0.85\\
0.919252162655211	0.86\\
0.92	0.861708064516129\\
0.925649967945783	0.87\\
0.929612533435231	0.88\\
0.93	0.880882835820896\\
0.936087700357113	0.89\\
0.939922749859126	0.9\\
0.94	0.900176168224299\\
0.946451642601771	0.91\\
0.95	0.91925\\
0.950599635421663	0.92\\
0.956747614043674	0.93\\
0.96	0.938430597014925\\
0.961220014503263	0.94\\
0.966979952104003	0.95\\
0.97	0.957825886524823\\
0.97165859596916	0.96\\
0.977152109524123	0.97\\
0.98	0.977408904109589\\
0.981954281286323	0.98\\
0.987266978173206	0.99\\
0.99	0.997164093959732\\
0.992128661746555	1\\
0.997327077646822	1.01\\
1	1.01708333333333\\
1.00219303741948	1.02\\
1.00733467334174	1.03\\
1.01	1.03716409395973\\
1.01215120274914	1.04\\
1.0172918570445	1.05\\
1.02	1.05740890410959\\
1.02199973569446	1.06\\
1.02720060886662	1.07\\
1.03	1.07782588652482\\
1.03172582688248	1.08\\
1.03706285342866	1.09\\
1.04	1.09843059701493\\
1.04130091862299	1.1\\
1.0468805223186	1.11\\
1.05	1.11925\\
1.05066421056361	1.12\\
1.05665563901234	1.13\\
1.0599330545419	1.14\\
1.06	1.1401761682243\\
1.06639045520176	1.15\\
1.06967258637391	1.16\\
1.07	1.1608828358209\\
1.07608770035711	1.17\\
1.07938433215445	1.18\\
1.08	1.18170806451613\\
1.08575109536683	1.19\\
1.08907363322512	1.2\\
1.09	1.20265946745562\\
1.09538655261501	1.21\\
1.0987480664582	1.22\\
1.1	1.22375\\
1.1050054726501	1.23\\
1.10841915771685	1.24\\
1.11	1.24500348837209\\
1.11463621650782	1.25\\
1.11810593191078	1.26\\
1.12	1.26646886792453\\
1.12438418740849	1.27\\
1.12784309303262	1.28\\
1.13	1.28826481481481\\
1.13333432947329	1.28\\
1.13112079948571	1.27\\
1.13	1.26597100840336\\
1.12491101163832	1.26\\
1.12204511484402	1.25\\
1.12	1.2420670212766\\
1.11768621613575	1.24\\
1.11304731655672	1.23\\
1.11042539742441	1.22\\
1.11	1.21868157894737\\
1.10417149251094	1.21\\
1.1011824555134	1.2\\
1.1	1.19625\\
1.09548616962373	1.19\\
1.09194389435077	1.18\\
1.09	1.17366832061069\\
1.08710206543042	1.17\\
1.08271063801998	1.16\\
1.08	1.15089736842105\\
1.07920682302772	1.15\\
1.07348121229741	1.14\\
1.07037112572119	1.13\\
1.07	1.12895753768844\\
1.06425030239497	1.12\\
1.06087510157258	1.11\\
1.06	1.10753924731183\\
1.05500703209841	1.1\\
1.05133423468299	1.09\\
1.05	1.08625\\
1.04573328912208	1.08\\
1.04173999935532	1.07\\
1.04	1.06512228915663\\
1.03640274986779	1.06\\
1.03208277417536	1.05\\
1.03	1.04419213836478\\
1.02698154647345	1.04\\
1.02235237480186	1.03\\
1.02	1.02349545454545\\
1.01743146844805	1.02\\
1.01253881112525	1.01\\
1.01	1.00306324503311\\
1.00771585886768	1\\
1.00263317678839	0.99\\
1	0.982916666666667\\
0.997806962580516	0.98\\
0.992628514786886	0.97\\
0.99	0.963063245033113\\
0.987692154172384	0.96\\
0.982520475561427	0.95\\
0.98	0.943495454545454\\
0.977376386011941	0.94\\
0.972307615418255	0.93\\
0.97	0.92419213836478\\
0.966879812755997	0.92\\
0.961991269597295	0.91\\
0.96	0.905122289156627\\
0.956231858367862	0.9\\
0.951575044101235	0.89\\
0.95	0.88625\\
0.94546434717701	0.88\\
0.94106405979379	0.87\\
0.94	0.867539247311828\\
0.934606019809597	0.86\\
0.930464119917221	0.85\\
0.93	0.848957537688443\\
0.923679461040688	0.84\\
0.92	0.830897368421052\\
0.919176162673646	0.83\\
0.912700252299178	0.82\\
0.91	0.813668320610687\\
0.906885713359904	0.81\\
0.901677577131268	0.8\\
0.9	0.79625\\
0.894994527349903	0.79\\
0.890615487516892	0.78\\
0.89	0.778681578947369\\
0.883343571348503	0.77\\
0.88	0.762067021276596\\
0.877750506512301	0.76\\
0.871834337637494	0.75\\
0.87	0.745971008403362\\
0.864501450957632	0.74\\
0.860402680122425	0.73\\
0.86	0.729166438356164\\
0.852328680748617	0.72\\
0.03125	9\\
1.14024509359675	1.29\\
1.14	1.28916643835616\\
1.13535938989514	1.29\\
1.13770283344851	1.3\\
1.13991316570358	1.31\\
1.14	1.31027077922078\\
1.14108185238034	1.31\\
1.14217421368368	1.3\\
1.14024509359675	1.29\\
0.03125	7\\
1.1511946596874	1.32\\
1.15	1.31458333333333\\
1.14786695019821	1.32\\
1.14962749281806	1.33\\
1.15	1.33125\\
1.15143276328458	1.33\\
1.1511946596874	1.32\\
0.03125	7\\
1.16030041226026	1.34\\
1.16	1.3388141509434\\
1.15896920742958	1.34\\
1.1594738556999	1.35\\
1.16	1.35197553191489\\
1.16100148307941	1.35\\
1.16030041226026	1.34\\
0.03125	5\\
1.17041620763701	1.37\\
1.17	1.36714230769231\\
1.16960278000535	1.37\\
1.17	1.37182704918033\\
1.17041620763701	1.37\\
0.0625	5\\
0.763804663440387	0.58\\
0.76	0.571684615384615\\
0.758762768605682	0.58\\
0.76	0.582921621621621\\
0.763804663440387	0.58\\
0.0625	269\\
0.773071716640365	0.59\\
0.77	0.584456329113924\\
0.765446321809202	0.59\\
0.769085181097622	0.6\\
0.77	0.601883884297521\\
0.775865611467821	0.61\\
0.7799001318136	0.62\\
0.78	0.620193975903614\\
0.786857503949447	0.63\\
0.79	0.636843577981651\\
0.792341989721248	0.64\\
0.797956882038541	0.65\\
0.8	0.654166666666666\\
0.804129403714104	0.66\\
0.809050805186447	0.67\\
0.81	0.671872751322751\\
0.815581070033609	0.68\\
0.82	0.689652380952381\\
0.820254984718382	0.69\\
0.826840769059461	0.7\\
0.83	0.706608385093168\\
0.832380080636373	0.71\\
0.837961278930441	0.72\\
0.84	0.724165979381443\\
0.843970461840697	0.73\\
0.84896763607933	0.74\\
0.85	0.742083333333334\\
0.855261486656131	0.75\\
0.859874005239969	0.76\\
0.86	0.760252755905512\\
0.866351869668398	0.77\\
0.87	0.777856077348067\\
0.871471977240398	0.78\\
0.877291080079795	0.79\\
0.88	0.795767961165048\\
0.882830749249127	0.8\\
0.888107222021754	0.81\\
0.89	0.814008515283843\\
0.893923449764802	0.82\\
0.89881786629595	0.83\\
0.9	0.8325\\
0.904823833597036	0.84\\
0.909434873046104	0.85\\
0.91	0.851196840148699\\
0.915574177602542	0.86\\
0.919966766750298	0.87\\
0.92	0.870070629370629\\
0.926200775151482	0.88\\
0.93	0.888656467661692\\
0.930897101569637	0.89\\
0.936721069617335	0.9\\
0.94	0.907477570093458\\
0.941656565036596	0.91\\
0.947147255052393	0.92\\
0.95	0.926527777777778\\
0.952249204681225	0.93\\
0.957488227809304	0.94\\
0.96	0.945778632478632\\
0.962703523776942	0.95\\
0.967750705171256	0.96\\
0.97	0.965211410788382\\
0.973038372934904	0.97\\
0.977939894750576	0.98\\
0.98	0.984813821138211\\
0.983266466107662	0.99\\
0.988059907245884	1\\
0.99	1.0045781124498\\
0.993396309480384	1.01\\
0.998114014132321	1.02\\
1	1.0245\\
1.00343325135145	1.03\\
1.00810480552623	1.04\\
1.01	1.0445781124498\\
1.01337998372661	1.05\\
1.01803427812591	1.06\\
1.02	1.06481382113821\\
1.02323663347498	1.07\\
1.02790386781825	1.08\\
1.03	1.08521141078838\\
1.03300045498863	1.09\\
1.0377144305937	1.1\\
1.04	1.10577863247863\\
1.04266501193833	1.11\\
1.04746616521897	1.12\\
1.05	1.12652777777778\\
1.05221854633745	1.13\\
1.05715845830115	1.14\\
1.06	1.14747757009346\\
1.06164085416825	1.15\\
1.06678961206698	1.16\\
1.07	1.16865646766169\\
1.07089710156964	1.17\\
1.07635637808246	1.18\\
1.07997236756609	1.19\\
1.08	1.19007062937063\\
1.08585314577214	1.2\\
1.0895414960658	1.21\\
1.09	1.2111968401487\\
1.09527047328652	1.22\\
1.09906479825828	1.23\\
1.1	1.2325\\
1.10459226784555	1.24\\
1.10854130415783	1.25\\
1.11	1.25400851528384\\
1.11378993163456	1.26\\
1.1179689756848	1.27\\
1.12	1.27576796116505\\
1.1228088602497	1.28\\
1.1273440112071	1.29\\
1.13	1.29785607734807\\
1.13153243163195	1.3\\
1.13665948253423	1.31\\
1.13990922266605	1.32\\
1.14	1.32025275590551\\
1.14590242099868	1.33\\
1.14927712794257	1.34\\
1.15	1.34208333333333\\
1.15504682032611	1.35\\
1.1586159655447	1.36\\
1.16	1.36416597938144\\
1.16403385903947	1.37\\
1.1679283454218	1.38\\
1.17	1.38660838509317\\
1.17270021016195	1.39\\
1.17721836582995	1.4\\
1.18	1.40965238095238\\
1.18037189556357	1.41\\
1.18649369318582	1.42\\
1.18940188416205	1.43\\
1.19	1.43187275132275\\
1.19577234550272	1.44\\
1.19876263351184	1.45\\
1.2	1.45416666666667\\
1.20511411788772	1.46\\
1.20818985913442	1.47\\
1.21	1.47684357798165\\
1.21430252343187	1.47\\
1.21200768537977	1.46\\
1.21	1.45081373626374\\
1.20894070524283	1.45\\
1.20361846866406	1.44\\
1.2011324412716	1.43\\
1.2	1.42583333333333\\
1.1958705962859	1.42\\
1.19251488705796	1.41\\
1.19003471542771	1.4\\
1.19	1.399886492891\\
1.18414401010597	1.39\\
1.18124101310633	1.38\\
1.18	1.37576896551724\\
1.17625133040799	1.37\\
1.17255372681617	1.36\\
1.17	1.35076294964029\\
1.16942807064851	1.35\\
1.16403465376113	1.34\\
1.16096434739226	1.33\\
1.16	1.32695048543689\\
1.15578950463896	1.32\\
1.15213420773831	1.31\\
1.15	1.30303571428571\\
1.14801435272571	1.3\\
1.1433918375978	1.29\\
1.14030900908972	1.28\\
1.14	1.27908861788618\\
1.13477342060663	1.27\\
1.13130113397732	1.26\\
1.13	1.2561098173516\\
1.12633033938978	1.25\\
1.12232045791817	1.24\\
1.12	1.23294742268041\\
1.11813730759834	1.23\\
1.11337364970046	1.22\\
1.11010352269332	1.21\\
1.11	1.20971236162362\\
1.10446705416172	1.2\\
1.10089707982575	1.19\\
1.1	1.1875\\
1.09560576055496	1.18\\
1.09167563632985	1.17\\
1.09	1.16531623376623\\
1.0867921373996	1.16\\
1.0824343766932	1.15\\
1.08	1.14317663551402\\
1.07802369358238	1.14\\
1.07316642068071	1.13\\
1.07	1.12110577889447\\
1.06929021853076	1.12\\
1.06386253047177	1.11\\
1.06021765565028	1.1\\
1.06	1.09943986013986\\
1.05451098392819	1.09\\
1.05066692276501	1.08\\
1.05	1.07829545454545\\
1.04509772319462	1.07\\
1.04105613595415	1.06\\
1.04	1.05732255639098\\
1.03560687905272	1.05\\
1.03137758985201	1.04\\
1.03	1.03654073359073\\
1.0260217225386	1.03\\
1.02162358815836	1.02\\
1.02	1.01596771653543\\
1.01632600133309	1.01\\
1.01178686593835	1\\
1.01	0.995617729083665\\
1.00650549559485	0.99\\
1.00186102679051	0.98\\
1	0.9755\\
0.996549517559819	0.97\\
0.991840935907414	0.96\\
0.99	0.955617729083666\\
0.986452034975153	0.95\\
0.981723009631156	0.94\\
0.98	0.935967716535433\\
0.976212157181233	0.93\\
0.97150535556769	0.92\\
0.97	0.916540733590734\\
0.965833876421765	0.91\\
0.961187742338962	0.9\\
0.96	0.897322556390977\\
0.955325149104175	0.89\\
0.950771408118093	0.88\\
0.95	0.878295454545455\\
0.944696555812575	0.87\\
0.940258743912977	0.86\\
0.94	0.85943986013986\\
0.933959841154427	0.85\\
0.93	0.841105778894473\\
0.929219750022161	0.84\\
0.923126599218457	0.83\\
0.92	0.823176635514019\\
0.917786929274843	0.82\\
0.912207272912735	0.81\\
0.91	0.805316233766234\\
0.906337513607014	0.8\\
0.901210524786705	0.79\\
0.9	0.7875\\
0.894881175009214	0.78\\
0.890142950146251	0.77\\
0.89	0.769712361623616\\
0.88342028623211	0.76\\
0.88	0.752947422680412\\
0.877812338593974	0.75\\
0.871952334574621	0.74\\
0.87	0.736109817351598\\
0.865572669368848	0.73\\
0.860471776526908	0.72\\
0.86	0.719088617886179\\
0.853527955884269	0.71\\
0.85	0.703035714285714\\
0.84760638897024	0.7\\
0.841606875647469	0.69\\
0.84	0.686950485436893\\
0.834727791586118	0.68\\
0.83	0.670762949640288\\
0.829329751935535	0.67\\
0.822277670353469	0.66\\
0.82	0.655768965517241\\
0.815304408841071	0.65\\
0.810063565788601	0.64\\
0.81	0.639886492890995\\
0.802305379834381	0.63\\
0.8	0.625833333333333\\
0.794885882112277	0.62\\
0.79	0.610813736263736\\
0.789033605220228	0.61\\
0.781507544141252	0.6\\
0.78	0.597371641791045\\
0.773071716640365	0.59\\
0.0625	9\\
1.22066650896532	1.48\\
1.22	1.47737164179105\\
1.21494147217235	1.48\\
1.21778584566174	1.49\\
1.21994204724409	1.5\\
1.22	1.50019397590361\\
1.22083799609629	1.5\\
1.22244226867778	1.49\\
1.22066650896532	1.48\\
0.0625	7\\
1.23109439620166	1.51\\
1.23	1.50445632911392\\
1.22787994675057	1.51\\
1.22949521120523	1.52\\
1.23	1.52188388429752\\
1.2314881671291	1.52\\
1.23109439620166	1.51\\
0.0625	5\\
1.24078691149975	1.54\\
1.24	1.53168461538462\\
1.23936565686203	1.54\\
1.24	1.54292162162162\\
1.24078691149975	1.54\\
0.125	5\\
0.651115341145678	0.42\\
0.65	0.41875\\
0.648915620843213	0.42\\
0.65	0.42375\\
0.651115341145678	0.42\\
0.125	401\\
0.662419934324899	0.43\\
0.66	0.427045283018868\\
0.653328362977953	0.43\\
0.657902722751412	0.44\\
0.66	0.443970212765958\\
0.664150100677284	0.45\\
0.669299081888157	0.46\\
0.67	0.461173602484472\\
0.676281578074926	0.47\\
0.68	0.476676190476191\\
0.682342609425255	0.48\\
0.688483255482543	0.49\\
0.69	0.492534126984127\\
0.695302606114128	0.5\\
0.7	0.508333333333334\\
0.701178717084325	0.51\\
0.707788272277999	0.52\\
0.71	0.523736602870813\\
0.714413147914033	0.53\\
0.72	0.539983132530121\\
0.720011885811313	0.54\\
0.726996781186363	0.55\\
0.73	0.555171719457013\\
0.733367947605145	0.56\\
0.739221258017571	0.57\\
0.74	0.571318248175183\\
0.745988093290204	0.58\\
0.75	0.587083333333333\\
0.75201461265714	0.59\\
0.75819358357193	0.6\\
0.76	0.60314306569343\\
0.76467159818487	0.61\\
0.77	0.619696606334842\\
0.770207955338451	0.62\\
0.776864010053935	0.63\\
0.78	0.635628571428571\\
0.782948462757528	0.64\\
0.788746860122421	0.65\\
0.79	0.652236084142395\\
0.795160301566987	0.66\\
0.8	0.669\\
0.800670430013811	0.67\\
0.807023853004804	0.68\\
0.81	0.685498096885813\\
0.81296890866564	0.69\\
0.818634122000915	0.7\\
0.82	0.702519018404908\\
0.824860345772508	0.71\\
0.83	0.719881034482759\\
0.830078325029955	0.72\\
0.836462350800905	0.73\\
0.84	0.736779591836734\\
0.842085209941527	0.74\\
0.847843288313231	0.75\\
0.85	0.754134615384616\\
0.853741082730671	0.76\\
0.859046058673563	0.77\\
0.86	0.771833898305085\\
0.865136688729954	0.78\\
0.87	0.789740035587189\\
0.870166543174694	0.79\\
0.876328050519454	0.8\\
0.88	0.80736339869281\\
0.881663590906748	0.81\\
0.887351939272631	0.82\\
0.89	0.82532811550152\\
0.892907197775697	0.83\\
0.89823372732989	0.84\\
0.9	0.843571428571428\\
0.903949481740229	0.85\\
0.908991667721245	0.86\\
0.91	0.86205135501355\\
0.914825484308806	0.87\\
0.919639363955363	0.88\\
0.92	0.880738860103627\\
0.925559780706871	0.89\\
0.93	0.899484883720931\\
0.930317709133753	0.9\\
0.936170215917929	0.91\\
0.94	0.918233121019108\\
0.941077584951103	0.92\\
0.946670123818116	0.93\\
0.95	0.937211538461539\\
0.951683400050154	0.94\\
0.957069699524916	0.95\\
0.96	0.95639880239521\\
0.962154107902395	0.96\\
0.967376875727832	0.97\\
0.97	0.975779912023461\\
0.972503457574044	0.98\\
0.977597895625629	0.99\\
0.98	0.99534450867052\\
0.982741437263328	1\\
0.987737692381225	1.01\\
0.99	1.01508581661891\\
0.992875164499878	1.02\\
0.997800139534007	1.03\\
1	1.035\\
1.00290941736839	1.04\\
1.00778821051845	1.05\\
1.01	1.05508581661891\\
1.01284691040378	1.06\\
1.01770406928019	1.07\\
1.02	1.07534450867052\\
1.02268836316602	1.08\\
1.0275491031181	1.09\\
1.03	1.09577991202346\\
1.03243236847665	1.1\\
1.03732390045391	1.11\\
1.04	1.11639880239521\\
1.04207502738698	1.12\\
1.04702816821043	1.13\\
1.05	1.13721153846154\\
1.05160926581964	1.14\\
1.05666057392349	1.15\\
1.06	1.15823312101911\\
1.06102366345311	1.16\\
1.06621848405576	1.17\\
1.07	1.17948488372093\\
1.07030046993847	1.18\\
1.07569754796348	1.19\\
1.07969852805158	1.2\\
1.08	1.20073886010363\\
1.08509103855876	1.21\\
1.08917648978704	1.22\\
1.09	1.22205135501355\\
1.09438878954968	1.23\\
1.09859123976536	1.24\\
1.1	1.24357142857143\\
1.10357542964746	1.25\\
1.10793842137141	1.26\\
1.11	1.26532811550152\\
1.11262732361865	1.27\\
1.11721147241731	1.28\\
1.12	1.28736339869281\\
1.12150698108802	1.29\\
1.12640052605622	1.3\\
1.13	1.30974003558719\\
1.13015211619553	1.31\\
1.13549054402771	1.32\\
1.13930509245631	1.33\\
1.14	1.33183389830509\\
1.14445798089305	1.34\\
1.14846858243443	1.35\\
1.15	1.35413461538462\\
1.15326443415048	1.36\\
1.15755419077361	1.37\\
1.16	1.37677959183674\\
1.16184354768047	1.38\\
1.16654562502093	1.39\\
1.17	1.39988103448276\\
1.17007101606724	1.4\\
1.17541641563465	1.41\\
1.17909482212241	1.42\\
1.18	1.42251901840491\\
1.1841211609142	1.43\\
1.18808277276704	1.44\\
1.19	1.44549809688581\\
1.19257580107106	1.45\\
1.1969762331929	1.46\\
1.2	1.469\\
1.20060665623218	1.47\\
1.2057418905366	1.48\\
1.20922720739966	1.49\\
1.21	1.49223608414239\\
1.21431775028121	1.5\\
1.21812527782126	1.51\\
1.22	1.51562857142857\\
1.2225749875436	1.52\\
1.22692359042458	1.53\\
1.23	1.53969660633484\\
1.23019672852637	1.54\\
1.23557487347425	1.55\\
1.23896840655227	1.56\\
1.24	1.56314306569343\\
1.24397835914919	1.57\\
1.24779030994586	1.58\\
1.25	1.58708333333333\\
1.25187636706743	1.59\\
1.2565023429242	1.6\\
1.25957592941549	1.61\\
1.26	1.61131824817518\\
1.26502731547525	1.62\\
1.26842606809653	1.63\\
1.27	1.63517171945701\\
1.27315298810667	1.64\\
1.2772073977335	1.65\\
1.28	1.65998313253012\\
1.28001425843412	1.66\\
1.28587328370554	1.67\\
1.28890065106458	1.68\\
1.29	1.68373660287081\\
1.29427926970792	1.69\\
1.29778723856224	1.7\\
1.3	1.70833333333333\\
1.30165678120547	1.71\\
1.30667645880625	1.72\\
1.30928822476018	1.73\\
1.31	1.73253412698413\\
1.31561667827645	1.74\\
1.318364140211	1.75\\
1.32	1.75667619047619\\
1.32461627109343	1.75\\
1.32246229526525	1.74\\
1.32050714310787	1.73\\
1.32	1.72792413793103\\
1.3149820762442	1.72\\
1.31234103481939	1.71\\
1.31018600730542	1.7\\
1.31	1.69929407582938\\
1.30471564651514	1.69\\
1.30197751955767	1.68\\
1.3	1.67166666666667\\
1.29882128291567	1.67\\
1.29416253621881	1.66\\
1.29143947020593	1.65\\
1.29	1.64449764397906\\
1.28729539551358	1.64\\
1.28344085273004	1.63\\
1.280766210732	1.62\\
1.28	1.61727692307692\\
1.27600820488777	1.61\\
1.27260125444742	1.6\\
1.27	1.59009525139665\\
1.26994034811511	1.59\\
1.26479042360692	1.58\\
1.26166855050043	1.57\\
1.26	1.56415398230088\\
1.25767194455121	1.56\\
1.25358231882064	1.55\\
1.25065589253857	1.54\\
1.25	1.53784090909091\\
1.24588301558965	1.53\\
1.24235679299122	1.52\\
1.24	1.51194159292035\\
1.23890347972762	1.51\\
1.23428469888361	1.5\\
1.23109928146056	1.49\\
1.23	1.48647688172043\\
1.22658030941004	1.48\\
1.22276329314413	1.47\\
1.22	1.46086666666667\\
1.21951293905734	1.46\\
1.21462703317536	1.45\\
1.21126438564898	1.44\\
1.21	1.43607920962199\\
1.20679464762957	1.43\\
1.20283766655526	1.42\\
1.2	1.411\\
1.19944554718948	1.41\\
1.1945643902333	1.4\\
1.19111847948369	1.39\\
1.19	1.38665932475884\\
1.1865088288592	1.38\\
1.18255497003824	1.37\\
1.18	1.36225839416058\\
1.17876923076923	1.36\\
1.1740919657077	1.35\\
1.17064655238878	1.34\\
1.17	1.33814469026549\\
1.16576152281537	1.33\\
1.1619144641999	1.32\\
1.16	1.31446666666667\\
1.15760653527243	1.31\\
1.15323388046954	1.3\\
1.15	1.29056818181818\\
1.1496853255999	1.29\\
1.14461611616095	1.28\\
1.14093154330073	1.27\\
1.14	1.26743005780347\\
1.13607376579741	1.26\\
1.13202653738199	1.25\\
1.13	1.24439827586207\\
1.12762043671825	1.24\\
1.12313420285407	1.23\\
1.12	1.2213156462585\\
1.11927015425256	1.22\\
1.11425372688336	1.21\\
1.11050508465893	1.2\\
1.11	1.19867129380054\\
1.10538247895451	1.19\\
1.1013620934832	1.18\\
1.1	1.17642857142857\\
1.0965153826497	1.17\\
1.09219377930947	1.16\\
1.09	1.15426903323263\\
1.08764422956696	1.15\\
1.08299354206837	1.14\\
1.08	1.1322127388535\\
1.07875701059976	1.13\\
1.07375328660486	1.12\\
1.07	1.11028444816053\\
1.06983738827016	1.11\\
1.06446343534527	1.1\\
1.06044803877664	1.09\\
1.06	1.08889844559586\\
1.05511307453173	1.08\\
1.05092328651701	1.07\\
1.05	1.06775\\
1.04569026240278	1.06\\
1.04133266164409	1.05\\
1.04	1.04678360655738\\
1.03618251009033	1.04\\
1.03166898964864	1.03\\
1.03	1.02601406685237\\
1.02657741845995	1.02\\
1.02192528340263	1.01\\
1.02	1.00545423728814\\
1.01686342037932	1\\
1.01209504321534	0.99\\
1.01	0.985114102564102\\
1.00703054565306	0.98\\
1.00217255013935	0.97\\
1	0.965\\
0.997071105197533	0.96\\
0.992153120860771	0.95\\
0.99	0.945114102564103\\
0.986980190940825	0.94\\
0.982033294374072	0.93\\
0.98	0.925454237288136\\
0.976755912126685	0.92\\
0.971810927958996	0.91\\
0.97	0.906014066852368\\
0.966399332916406	0.9\\
0.961485191610156	0.89\\
0.96	0.886783606557377\\
0.955914128905382	0.88\\
0.951056463679561	0.87\\
0.95	0.86775\\
0.945306026859621	0.86\\
0.940526143204499	0.85\\
0.94	0.848898445595855\\
0.934582120746086	0.84\\
0.93	0.830284448160535\\
0.929817889834591	0.83\\
0.923750162953874	0.82\\
0.92	0.812212738853503\\
0.918582547049523	0.81\\
0.912817915103799	0.8\\
0.91	0.794269033232629\\
0.907263148720233	0.79\\
0.901792615856763	0.78\\
0.9	0.776428571428571\\
0.895873793084477	0.77\\
0.89068059285236	0.76\\
0.89	0.758671293800539\\
0.884424764958191	0.75\\
0.88	0.741315646258503\\
0.8791136064165	0.74\\
0.872922837865467	0.73\\
0.87	0.724398275862069\\
0.867047609027303	0.72\\
0.861371745921555	0.71\\
0.86	0.707430057803468\\
0.855022218779952	0.7\\
0.85	0.690568181818182\\
0.849600327411784	0.69\\
0.843023868415648	0.68\\
0.84	0.674466666666666\\
0.836883543293621	0.67\\
0.831038599677992	0.66\\
0.83	0.658144690265487\\
0.824344718111526	0.65\\
0.82	0.642258394160584\\
0.818357640501626	0.64\\
0.811924783474596	0.63\\
0.81	0.626659324758843\\
0.805207067726594	0.62\\
0.8	0.611\\
0.79924264226965	0.61\\
0.792332777885687	0.6\\
0.79	0.596079209621993\\
0.785471232	0.59\\
0.78	0.580866666666666\\
0.779318491136688	0.58\\
0.772171905209081	0.57\\
0.77	0.56647688172043\\
0.765019499758837	0.56\\
0.76	0.551941592920354\\
0.758423029235773	0.55\\
0.751383210446005	0.54\\
0.75	0.537840909090909\\
0.74379933572014	0.53\\
0.74	0.524153982300885\\
0.736538028948096	0.52\\
0.73	0.510095251396648\\
0.729914504199574	0.51\\
0.721850039921609	0.5\\
0.72	0.497276923076923\\
0.713816432137994	0.49\\
0.71	0.484497643979058\\
0.705926839098172	0.48\\
0.7	0.471666666666667\\
0.698343218794526	0.47\\
0.690513806723124	0.46\\
0.69	0.459294075829384\\
0.681555388093444	0.45\\
0.68	0.447924137931035\\
0.672287512236051	0.44\\
0.67	0.437057913669065\\
0.662419934324899	0.43\\
0.125	9\\
1.33065163526272	1.76\\
1.33	1.75705791366906\\
1.32505949864089	1.76\\
1.32770268133132	1.77\\
1.32969164864755	1.78\\
1.33	1.78117360248447\\
1.33201923590703	1.78\\
1.33221576648409	1.77\\
1.33065163526272	1.76\\
0.125	7\\
1.34057263527411	1.79\\
1.34	1.78704528301887\\
1.33819403200231	1.79\\
1.33915263665777	1.8\\
1.34	1.80397021276596\\
1.34134271168756	1.8\\
1.34057263527411	1.79\\
0.125	5\\
1.35019927516978	1.82\\
1.35	1.81875\\
1.34963949933668	1.82\\
1.35	1.82375\\
1.35019927516978	1.82\\
0.25	532\\
1.42254927540187	2\\
1.4201625278408	1.99\\
1.42	1.98937834394904\\
1.41561247066711	1.98\\
1.41257494694369	1.97\\
1.41011877866915	1.96\\
1.41	1.95955604229607\\
1.40558421342864	1.95\\
1.40250675570663	1.94\\
1.4	1.93\\
1.4	1.93\\
1.39543256839729	1.92\\
1.39234665812043	1.91\\
1.39	1.90094852398524\\
1.38952750333165	1.9\\
1.38516817392893	1.89\\
1.38209758429908	1.88\\
1.38	1.87216598639456\\
1.37894809002684	1.87\\
1.37480130626154	1.86\\
1.37176302066429	1.85\\
1.37	1.84361457680251\\
1.36828217065812	1.84\\
1.36434133813445	1.83\\
1.36134666967898	1.82\\
1.36	1.81526242774566\\
1.35754428792037	1.81\\
1.35379646502808	1.8\\
1.35085219193505	1.79\\
1.35	1.78708333333333\\
1.34674457224445	1.78\\
1.3431736170958	1.77\\
1.34028302479052	1.76\\
1.34	1.75905615763547\\
1.33588989184808	1.75\\
1.33247848825948	1.74\\
1.33	1.73150752212389\\
1.3292824598968	1.73\\
1.32498472716109	1.72\\
1.32171563122623	1.71\\
1.32	1.70430160427807\\
1.3179909462708	1.7\\
1.31403182063876	1.69\\
1.31088858321871	1.68\\
1.31	1.67713150851582\\
1.30671813504571	1.67\\
1.30303265300818	1.66\\
1.3	1.65\\
1.3	1.65\\
1.29545038646439	1.64\\
1.29198778939103	1.63\\
1.29	1.62365485933504\\
1.2882816173155	1.62\\
1.28417701228614	1.61\\
1.28089712958662	1.6\\
1.28	1.59721843317972\\
1.27665504631236	1.59\\
1.27288941107358	1.58\\
1.27	1.57091569920844\\
1.26956170871089	1.57\\
1.2650833393152	1.56\\
1.26158049043592	1.55\\
1.26	1.54518497652582\\
1.25756002452348	1.54\\
1.25354052523016	1.53\\
1.25024426987291	1.52\\
1.25	1.51927631578947\\
1.24568752477379	1.51\\
1.24200764118686	1.5\\
1.24	1.49401126760563\\
1.23808896468812	1.49\\
1.23389331171819	1.48\\
1.230470365768	1.47\\
1.23	1.46863684759916\\
1.22593779610673	1.46\\
1.22214274689831	1.45\\
1.22	1.44376221198157\\
1.21819055270813	1.44\\
1.21391221215759	1.43\\
1.21041123615738	1.42\\
1.21	1.41883716904277\\
1.20580324111016	1.41\\
1.20196093042227	1.4\\
1.2	1.39444444444444\\
1.19784690411726	1.39\\
1.19357818982015	1.38\\
1.19004900863249	1.37\\
1.19	1.36986506849315\\
1.18527648808827	1.36\\
1.18145005654699	1.35\\
1.18	1.34601012658228\\
1.17707147079232	1.34\\
1.17288874169444	1.33\\
1.17	1.32205250569476\\
1.16898120247618	1.32\\
1.16437004720675	1.31\\
1.16060958587111	1.3\\
1.16	1.29837312252964\\
1.15589876628059	1.29\\
1.15185415191586	1.28\\
1.15	1.27506578947368\\
1.14747916546627	1.27\\
1.1431086270484	1.26\\
1.14	1.25175067264574\\
1.13911451333957	1.25\\
1.13437144596514	1.24\\
1.13049181186951	1.23\\
1.13	1.22873420038536\\
1.12563972586269	1.22\\
1.1215207670258	1.21\\
1.12	1.20610688259109\\
1.11690893721663	1.2\\
1.1125296190496	1.19\\
1.11	1.18356061571125\\
1.1081725608992	1.18\\
1.10351276230455	1.17\\
1.1	1.16111111111111\\
1.09942176143033	1.16\\
1.09446338683855	1.15\\
1.09040450757852	1.14\\
1.09	1.13900762711864\\
1.08537344328102	1.13\\
1.08113141326724	1.12\\
1.08	1.117246692607\\
1.076233676281	1.11\\
1.07180647438693	1.1\\
1.07	1.09564138276553\\
1.06703374515951	1.09\\
1.06242266732643	1.08\\
1.06	1.07420740740741\\
1.05776244590225	1.07\\
1.05297260535802	1.06\\
1.05	1.05296052631579\\
1.04840803989381	1.05\\
1.04344872727939	1.04\\
1.04	1.03191587982833\\
1.03895868202362	1.03\\
1.03384352319436	1.02\\
1.03	1.01108725490196\\
1.02940292525349	1.01\\
1.0241497850263	1\\
1.02	0.990486343612335\\
1.01973026295697	0.99\\
1.01436086435161	0.98\\
1.01	0.970122062084257\\
1.0099316577799	0.97\\
1.00447091668695	0.96\\
1	0.95\\
1	0.95\\
0.994475110372227	0.94\\
0.99	0.930122062084258\\
0.989930441924377	0.93\\
0.984369780205883	0.92\\
0.98	0.910486343612335\\
0.979720568228557	0.91\\
0.974152510975207	0.9\\
0.97	0.891087254901961\\
0.969370383220312	0.89\\
0.963822143310082	0.88\\
0.96	0.871915879828326\\
0.9588821205652	0.87\\
0.953378702679216	0.86\\
0.95	0.85296052631579\\
0.948259903853888	0.85\\
0.942823260408466	0.84\\
0.94	0.834207407407407\\
0.93750930296294	0.83\\
0.93215774200749	0.82\\
0.93	0.815641382765531\\
0.926636838804098	0.81\\
0.921384701939776	0.8\\
0.92	0.797246692607004\\
0.915649487546354	0.79\\
0.910507084947194	0.78\\
0.91	0.779007627118644\\
0.90455422660693	0.77\\
0.9	0.761111111111111\\
0.899309673544619	0.76\\
0.893357651708467	0.75\\
0.89	0.743560615711253\\
0.887775869500348	0.74\\
0.882065680298596	0.73\\
0.88	0.726106882591093\\
0.876163662374821	0.72\\
0.870683344168509	0.71\\
0.87	0.708734200385357\\
0.864480136381625	0.7\\
0.86	0.69175067264574\\
0.858864216566902	0.69\\
0.852730195525683	0.68\\
0.85	0.675065789473684\\
0.846698996494934	0.67\\
0.840916759935185	0.66\\
0.84	0.658373122529644\\
0.834516001682684	0.65\\
0.83	0.642052505694761\\
0.828620613111868	0.64\\
0.822311344679031	0.63\\
0.82	0.626010126582278\\
0.815946355519192	0.62\\
0.810079639629234	0.61\\
0.81	0.609865068493151\\
0.803315188203234	0.6\\
0.8	0.594444444444444\\
0.796914032664964	0.59\\
0.79070741362541	0.58\\
0.79	0.57883716904277\\
0.783842043640234	0.57\\
0.78	0.563762211981567\\
0.777312595152862	0.56\\
0.770856243648166	0.55\\
0.77	0.548636847599165\\
0.763814450715808	0.54\\
0.76	0.534011267605633\\
0.757055087849308	0.53\\
0.750470082255848	0.52\\
0.75	0.519276315789474\\
0.743176486074224	0.51\\
0.74	0.505184976525822\\
0.73609182996417	0.5\\
0.73	0.490915699208443\\
0.729290902589774	0.49\\
0.721902825393072	0.48\\
0.72	0.477218433179724\\
0.714419416611	0.47\\
0.71	0.463654859335038\\
0.707100759781297	0.46\\
0.7	0.45\\
0.7	0.45\\
0.692075999964782	0.44\\
0.69	0.437131508515815\\
0.684209985678305	0.43\\
0.68	0.424301604278075\\
0.676415130076319	0.42\\
0.67	0.411507522123894\\
0.668707593096036	0.41\\
0.660725173865733	0.4\\
0.66	0.399056157635468\\
0.652292712580311	0.39\\
0.65	0.387083333333333\\
0.643816131022361	0.38\\
0.64	0.375262427745665\\
0.635280150346705	0.37\\
0.63	0.363614576802508\\
0.626663869801085	0.36\\
0.62	0.352165986394558\\
0.617938991827818	0.35\\
0.61	0.34094852398524\\
0.609067662900564	0.34\\
0.6	0.33\\
0.6	0.33\\
0.590402939442548	0.32\\
0.59	0.319556042296072\\
0.580581255117993	0.31\\
0.58	0.309378343949045\\
0.570499879022502	0.3\\
0.57	0.299481772575251\\
0.560122637400923	0.29\\
0.56	0.289876923076923\\
0.55	0.280892857142857\\
0.548886672129395	0.28\\
0.54	0.272486746987952\\
0.536759175662414	0.27\\
0.53	0.264585220125786\\
0.523759469291675	0.26\\
0.52	0.257148051948052\\
0.51	0.250295098039215\\
0.505025378159706	0.26\\
0.509975311105015	0.27\\
0.51	0.27003322147651\\
0.517409504802295	0.28\\
0.52	0.283556164383562\\
0.524747319288508	0.29\\
0.53	0.297382624113475\\
0.531906003873467	0.3\\
0.539244241164241	0.31\\
0.54	0.31097094017094\\
0.547146166985168	0.32\\
0.55	0.32375\\
0.554880355376633	0.33\\
0.56	0.336893457943925\\
0.562391581976796	0.34\\
0.569731568514611	0.35\\
0.57	0.350348671096345\\
0.577644365502269	0.36\\
0.58	0.36313006993007\\
0.58536574042466	0.37\\
0.59	0.376308364312267\\
0.592842930962912	0.38\\
0.6	0.39\\
0.6	0.39\\
0.60777268235268	0.4\\
0.61	0.403048480243161\\
0.615330808227959	0.41\\
0.62	0.416546405228758\\
0.622613415765069	0.42\\
0.629665737898465	0.43\\
0.63	0.430464435695538\\
0.637219534005663	0.44\\
0.64	0.443952542372882\\
0.644521654645109	0.45\\
0.65	0.457980769230769\\
0.651491663730694	0.46\\
0.658593209503542	0.47\\
0.66	0.472038578680203\\
0.665856882789525	0.48\\
0.67	0.486146675900277\\
0.672803264648093	0.49\\
0.679506990375856	0.5\\
0.68	0.500730516431925\\
0.686695434699673	0.51\\
0.69	0.515010154241645\\
0.693567418053501	0.52\\
0.7	0.53\\
0.7	0.53\\
0.707085554129697	0.54\\
0.71	0.544525550122249\\
0.713843037974683	0.55\\
0.72	0.559801092896175\\
0.720138514959806	0.56\\
0.727057823509674	0.57\\
0.73	0.574686342042755\\
0.733662341914624	0.58\\
0.739826872958796	0.59\\
0.74	0.590276793248945\\
0.74662893470741	0.6\\
0.75	0.605514705882353\\
0.753036662976208	0.61\\
0.7591894651009	0.62\\
0.76	0.62133164556962\\
0.765801955731281	0.63\\
0.77	0.637061638954869\\
0.771956042218445	0.64\\
0.77818750992158	0.65\\
0.78	0.653062660944206\\
0.784563684489148	0.66\\
0.79	0.669415525672372\\
0.790383151281044	0.67\\
0.796806764549977	0.68\\
0.8	0.685555555555555\\
0.802878132667525	0.69\\
0.808762757113898	0.7\\
0.81	0.702165541922291\\
0.815012987012987	0.71\\
0.82	0.718946478873239\\
0.820673090622774	0.72\\
0.826863394547804	0.73\\
0.83	0.735659327548807\\
0.832739569428757	0.74\\
0.838480987269131	0.75\\
0.84	0.75275951417004\\
0.844515848015605	0.76\\
0.849902450952339	0.77\\
0.85	0.770178571428572\\
0.856055170308087	0.78\\
0.86	0.787399118942731\\
0.861604838435663	0.79\\
0.867395449918847	0.8\\
0.87	0.804920893970894\\
0.873097860199715	0.81\\
0.878564586846543	0.82\\
0.88	0.822733596837945\\
0.884382490539051	0.83\\
0.889583700386328	0.84\\
0.89	0.840799527410208\\
0.895488916832543	0.85\\
0.9	0.858888888888889\\
0.900666049460833	0.86\\
0.906439875710399	0.87\\
0.91	0.877072388059702\\
0.911736224828502	0.88\\
0.917252894513901	0.89\\
0.92	0.895504526748971\\
0.922638727014705	0.9\\
0.927941755719539	0.91\\
0.93	0.914161576846308\\
0.933393231054784	0.92\\
0.938517480517624	0.93\\
0.94	0.933025680933852\\
0.944014985635966	0.94\\
0.948989004996049	0.95\\
0.95	0.952083333333334\\
0.954515969554895	0.96\\
0.959363653950843	0.97\\
0.96	0.971324344569288\\
0.964905678358703	0.98\\
0.969647477744807	0.99\\
0.97	0.99074112754159\\
0.975191666882595	1\\
0.979845493932274	1.01\\
0.98	1.0103282051282\\
0.985379925120342	1.02\\
0.989961860721634	1.03\\
0.99	1.03008187613843\\
0.995475135871615	1.04\\
1	1.05\\
1	1.05\\
1.00548084444865	1.06\\
1.00996268145571	1.07\\
1.01	1.07008187613843\\
1.0153995587178	1.08\\
1.0198520750356	1.09\\
1.02	1.09032820512821\\
1.02523278915946	1.1\\
1.02966977577367	1.11\\
1.03	1.11074112754159\\
1.03498103169671	1.12\\
1.03941680239151	1.13\\
1.04	1.13132434456929\\
1.04464368945176	1.14\\
1.04909356858423	1.15\\
1.05	1.15208333333333\\
1.05421892208761	1.16\\
1.05869982318197	1.17\\
1.06	1.17302568093385\\
1.06370340143082	1.18\\
1.06823455195919	1.19\\
1.07	1.19416157684631\\
1.07309193731667	1.2\\
1.07769582911651	1.21\\
1.08	1.21550452674897\\
1.08237691408056	1.22\\
1.0870805993795	1.23\\
1.09	1.2370723880597\\
1.09154743859214	1.24\\
1.09638436055157	1.25\\
1.1	1.25888888888889\\
1.1005880312127	1.26\\
1.10560069822038	1.27\\
1.10967271404749	1.28\\
1.11	1.28079952741021\\
1.11472059358405	1.29\\
1.11889699168677	1.3\\
1.12	1.30273359683794\\
1.12373137130534	1.31\\
1.12804444903439	1.32\\
1.13	1.32492089397089\\
1.13261505525971	1.33\\
1.13710719283515	1.34\\
1.14	1.34739911894273\\
1.14134571003641	1.35\\
1.14607417382542	1.36\\
1.14992979920701	1.37\\
1.15	1.37017857142857\\
1.15492975252045	1.38\\
1.15893245206332	1.39\\
1.16	1.39275951417004\\
1.16365137942475	1.4\\
1.16784842174711	1.41\\
1.17	1.41565932754881\\
1.17220568216264	1.42\\
1.17666335441791	1.43\\
1.18	1.43894647887324\\
1.18054152212603	1.44\\
1.18535647768056	1.45\\
1.18918715818157	1.46\\
1.19	1.46216554192229\\
1.19389688792143	1.47\\
1.19795434925804	1.48\\
1.2	1.48555555555556\\
1.20223689876355	1.49\\
1.2066073764751	1.5\\
1.21	1.50941552567237\\
1.21029932696823	1.51\\
1.21511609264371	1.52\\
1.21888901430379	1.53\\
1.22	1.53306266094421\\
1.22343339617131	1.54\\
1.22749451151423	1.55\\
1.23	1.55706163895487\\
1.23148180756446	1.56\\
1.23595942042382	1.57\\
1.23952499388558	1.58\\
1.24	1.58133164556962\\
1.24423476302621	1.59\\
1.24807771959106	1.6\\
1.25	1.60551470588235\\
1.25223805246876	1.61\\
1.25648948759686	1.62\\
1.2599030836679	1.63\\
1.26	1.63027679324894\\
1.26470681951643	1.64\\
1.26839880698764	1.65\\
1.27	1.65468634204276\\
1.27263670919645	1.66\\
1.27674903142488	1.67\\
1.28	1.67980109289617\\
1.28010248108393	1.68\\
1.2848924150407	1.69\\
1.28848824910669	1.7\\
1.29	1.70452555012225\\
1.29271760700566	1.71\\
1.29676795286752	1.72\\
1.3	1.73\\
1.3	1.73\\
1.3048185288496	1.74\\
1.30836893618268	1.75\\
1.31	1.75501015424165\\
1.31249741065972	1.76\\
1.31656689658487	1.77\\
1.31976120777302	1.78\\
1.32	1.78073051643192\\
1.32449864714762	1.79\\
1.32805827918882	1.8\\
1.33	1.80614667590028\\
1.33196635685762	1.81\\
1.33615855504528	1.82\\
1.33935300159091	1.83\\
1.34	1.8320385786802\\
1.34393055616468	1.84\\
1.34756924432199	1.85\\
1.35	1.85798076923077\\
1.35106960369655	1.86\\
1.35554663697155	1.87\\
1.35878914278544	1.88\\
1.36	1.88395254237288\\
1.36308572664048	1.89\\
1.36691039797264	1.9\\
1.36985727248896	1.91\\
1.37	1.91046443569554\\
1.37472054572032	1.92\\
1.37808237405768	1.93\\
1.38	1.93654640522876\\
1.38187384192562	1.94\\
1.38608436636047	1.95\\
1.38910288691608	1.96\\
1.39	1.96304848024316\\
1.39363771562179	1.97\\
1.39724580122397	1.98\\
1.4	1.99\\
1.4	1.99\\
1.40508207551964	2\\
0.5	610\\
1.43325403173784	2\\
1.43033725181255	1.99\\
1.43	1.98885651085142\\
1.42643003987147	1.98\\
1.42305766434692	1.97\\
1.42013385051252	1.96\\
1.42	1.95955179153094\\
1.41614202312234	1.95\\
1.4127878325731	1.94\\
1.41	1.93053305084746\\
1.409778573971	1.93\\
1.4057718021811	1.92\\
1.40244524684767	1.91\\
1.4	1.90181818181818\\
1.3992448812387	1.9\\
1.39532213348136	1.89\\
1.39203100470436	1.88\\
1.39	1.87330481611208\\
1.38862762495227	1.87\\
1.38479605332849	1.86\\
1.38154648073509	1.85\\
1.38	1.84497777777778\\
1.37793284977716	1.84\\
1.37419670160522	1.83\\
1.37099322098087	1.82\\
1.37	1.81682237479806\\
1.36716616618296	1.81\\
1.36352717321953	1.8\\
1.36037284681212	1.79\\
1.36	1.78882476780186\\
1.35633260531826	1.78\\
1.35279040020159	1.77\\
1.35	1.76114130434783\\
1.34951530527459	1.76\\
1.34543653666637	1.75\\
1.34198906416413	1.74\\
1.34	1.7337900990099\\
1.33839224401939	1.73\\
1.33448163602318	1.72\\
1.3311255366037	1.71\\
1.33	1.706543114241\\
1.32722599479175	1.7\\
1.32347089111146	1.69\\
1.32020184319356	1.68\\
1.32	1.67938991097923\\
1.31601905930084	1.67\\
1.3124066332933	1.66\\
1.31	1.65270220949264\\
1.30883326678939	1.65\\
1.30477297066711	1.64\\
1.3012905856936	1.63\\
1.3	1.62615384615385\\
1.29734588728647	1.62\\
1.29348845900136	1.61\\
1.29012392011239	1.6\\
1.29	1.59963683068017\\
1.28584491069339	1.59\\
1.2821655979028	1.58\\
1.28	1.57364794952681\\
1.27840118901543	1.57\\
1.27432713809265	1.56\\
1.27080393033745	1.55\\
1.27	1.54768343151694\\
1.26663595183209	1.54\\
1.26278918409491	1.53\\
1.26	1.52196293929712\\
1.25912681070595	1.52\\
1.25488088382064	1.51\\
1.2512275730586	1.5\\
1.25	1.49652777777778\\
1.24710066491418	1.49\\
1.24312718905212	1.48\\
1.24	1.4711642172524\\
1.23947499414699	1.47\\
1.23511440671488	1.46\\
1.23136700891256	1.45\\
1.23	1.44621067746686\\
1.22720433564811	1.44\\
1.22315223063433	1.43\\
1.22	1.42128201892745\\
1.21941456738993	1.42\\
1.21500249789288	1.41\\
1.21120090696899	1.4\\
1.21	1.39674247467438\\
1.20692655542399	1.39\\
1.20284591369482	1.38\\
1.2	1.37230769230769\\
1.19893382235348	1.37\\
1.19453185045219	1.36\\
1.1907156632067	1.35\\
1.19	1.34810274261603\\
1.18626203182533	1.34\\
1.18219949668911	1.33\\
1.18	1.32419703264095\\
1.17803933392248	1.32\\
1.17370070480249	1.31\\
1.17	1.30028333333333\\
1.16986593109577	1.3\\
1.16521856103068	1.29\\
1.16121293199554	1.28\\
1.16	1.27687932011331\\
1.15675142386066	1.27\\
1.15251657681824	1.26\\
1.15	1.25356481481481\\
1.14829651444733	1.25\\
1.14381299324722	1.24\\
1.14	1.23031083591331\\
1.13984967528214	1.23\\
1.13509795076067	1.22\\
1.13099283429481	1.21\\
1.13	1.207528581363\\
1.12636616598137	1.2\\
1.12206645768025	1.19\\
1.12	1.18489452449568\\
1.11761120976107	1.18\\
1.11310822352923	1.17\\
1.11	1.16238010432191\\
1.10882544375434	1.16\\
1.10411208922601	1.15\\
1.1	1.14\\
1.1	1.14\\
1.09507128563871	1.13\\
1.0908082691109	1.12\\
1.09	1.11807530779754\\
1.08597836935602	1.11\\
1.08156407935542	1.1\\
1.08	1.09630924369748\\
1.07682531404711	1.09\\
1.07226132590768	1.08\\
1.07	1.07471394849785\\
1.06760364357012	1.07\\
1.06289365721572	1.06\\
1.06	1.05330145772595\\
1.05830460641312	1.05\\
1.05345459011995	1.04\\
1.05	1.03208333333333\\
1.04891938724935	1.03\\
1.04393764338611	1.02\\
1.04	1.01107027027027\\
1.0394393471635	1.01\\
1.03433648431555	1\\
1.03	0.990271699544764\\
1.02985627997512	0.99\\
1.02464508124688	0.98\\
1.02012030256821	0.97\\
1.02	0.969735809018568\\
1.01485785335271	0.96\\
1.01026061358723	0.95\\
1.01	0.949434154460719\\
1.00496980841389	0.94\\
1.000310654518	0.93\\
1	0.929333333333333\\
0.994976659406744	0.92\\
0.990266845421807	0.91\\
0.99	0.909434154460719\\
0.984874911839702	0.9\\
0.980126121848141	0.89\\
0.98	0.889735809018568\\
0.974661915756176	0.88\\
0.97	0.870271699544765\\
0.969848085693075	0.87\\
0.964335878975667	0.86\\
0.96	0.85107027027027\\
0.959396267326188	0.85\\
0.953895841155392	0.84\\
0.95	0.832083333333334\\
0.948814436683542	0.83\\
0.943341611248268	0.82\\
0.94	0.813301457725947\\
0.938104730224482	0.81\\
0.932673673546916	0.8\\
0.93	0.794713948497854\\
0.927270106252977	0.79\\
0.921893069449184	0.78\\
0.92	0.776309243697479\\
0.916314112136151	0.77\\
0.911001263187859	0.76\\
0.91	0.758075307797538\\
0.905240646453896	0.75\\
0.9	0.74\\
0.9	0.74\\
0.894053730491281	0.73\\
0.89	0.722380104321908\\
0.888558651654972	0.72\\
0.882757300441626	0.71\\
0.88	0.704894524495677\\
0.877010845891541	0.7\\
0.87135502811934	0.69\\
0.87	0.687528581363004\\
0.865362462250209	0.68\\
0.86	0.670310835913313\\
0.859804945335153	0.67\\
0.853618419682537	0.66\\
0.85	0.653564814814815\\
0.847744729109748	0.65\\
0.84178260359482	0.64\\
0.84	0.636879320113314\\
0.83561096175135	0.63\\
0.83	0.620283333333333\\
0.82981586106944	0.62\\
0.82340630102985	0.61\\
0.82	0.604197032640949\\
0.817250321984885	0.6\\
0.811131883115521	0.59\\
0.81	0.588102742616034\\
0.804646386737354	0.58\\
0.8	0.572307692307692\\
0.798447357661514	0.57\\
0.792000675498513	0.56\\
0.79	0.556742474674385\\
0.785426138176463	0.55\\
0.78	0.541282018927445\\
0.779114186851211	0.54\\
0.772398853227047	0.53\\
0.77	0.526210677466863\\
0.76567404405919	0.52\\
0.76	0.511164217252396\\
0.759173934973449	0.51\\
0.752268317114362	0.5\\
0.75	0.496527777777778\\
0.745331101091926	0.49\\
0.74	0.481962939297125\\
0.738569790787674	0.48\\
0.731563296643228	0.47\\
0.73	0.467683431516937\\
0.724356437785651	0.46\\
0.72	0.453647949526814\\
0.717270629886414	0.45\\
0.710253388867853	0.44\\
0.71	0.439636830680174\\
0.702729540727477	0.43\\
0.7	0.426153846153846\\
0.695271280714509	0.42\\
0.69	0.412702209492635\\
0.687883879650101	0.41\\
0.680448308757393	0.4\\
0.68	0.399389910979229\\
0.672587274164739	0.39\\
0.67	0.386543114241002\\
0.664737011045725	0.38\\
0.66	0.373790099009901\\
0.656892247532322	0.37\\
0.65	0.361141304347826\\
0.649045517020704	0.36\\
0.640929771169991	0.35\\
0.64	0.348824767801858\\
0.63256522447915	0.34\\
0.63	0.336822374798061\\
0.624139739809193	0.33\\
0.62	0.324977777777778\\
0.615638777241049	0.32\\
0.61	0.313304816112084\\
0.607045506141333	0.31\\
0.6	0.301818181818182\\
0.598340785969686	0.3\\
0.59	0.290533050847457\\
0.589503268547361	0.29\\
0.58040068430823	0.28\\
0.58	0.279551791530945\\
0.571045765105538	0.27\\
0.57	0.268856510851419\\
0.561525817659039	0.26\\
0.56	0.258369965870307\\
0.55182335731135	0.25\\
0.55	0.248097826086956\\
0.54192107158039	0.24\\
0.54	0.238043816254417\\
0.531802295746117	0.23\\
0.53	0.22820939177102\\
0.521451497229079	0.22\\
0.52	0.218593501805054\\
0.510854727945061	0.21\\
0.51	0.209192468239564\\
0.5	0.2\\
0.5	0.2\\
0.49	0.191230709534368\\
0.488567335613344	0.19\\
0.48	0.182688986784141\\
0.476797271594412	0.18\\
0.47	0.17435522875817\\
0.464692200464653	0.17\\
0.46	0.166207725321888\\
0.452258099048588	0.16\\
0.45	0.158223684210526\\
0.44	0.150478756476684\\
0.439351522063339	0.15\\
0.43	0.143320927318296\\
0.425399531290687	0.14\\
0.42	0.13624347826087\\
0.411153247197741	0.13\\
0.41	0.129218213457077\\
0.4	0.122857142857143\\
0.395541585857911	0.12\\
0.39	0.116649730458221\\
0.38	0.110465306122449\\
0.379182673637042	0.11\\
0.37	0.10518197492163\\
0.360776677360859	0.1\\
0.36	0.0995976878612716\\
0.35	0.0951136363636364\\
0.34	0.0902757281553398\\
0.339312140478353	0.0899999999999999\\
0.33	0.0867826359832636\\
0.32	0.0836137931034482\\
0.31	0.0809103603603603\\
0.3	0.0800000000000001\\
0.3	0.0800000000000001\\
0.3	0.0800000000000001\\
0.294312041408338	0.0899999999999999\\
0.3	0.0999999999999994\\
0.3	0.1\\
0.308675785881604	0.11\\
0.31	0.111211375661376\\
0.318337558037395	0.12\\
0.32	0.121465486725664\\
0.328543018491861	0.13\\
0.33	0.131260344827586\\
0.339036545508567	0.14\\
0.34	0.140827210884354\\
0.349664044722366	0.15\\
0.35	0.150288461538462\\
0.36	0.159603149606299\\
0.360389527581876	0.16\\
0.37	0.168850355871886\\
0.371143034055727	0.17\\
0.38	0.178180392156863\\
0.381819756516055	0.18\\
0.39	0.187607750759879\\
0.392394396191114	0.19\\
0.4	0.197142857142857\\
0.402851480631318	0.2\\
0.41	0.206793902439025\\
0.413181738721173	0.21\\
0.42	0.216567875647668\\
0.423379807385675	0.22\\
0.43	0.226471197007481\\
0.433442734626847	0.23\\
0.44	0.236510144927536\\
0.443368970864278	0.24\\
0.45	0.246691176470588\\
0.453157664282747	0.25\\
0.46	0.257021198156682\\
0.462808145533532	0.26\\
0.47	0.267507823129252\\
0.472319527251622	0.27\\
0.48	0.278159641255605\\
0.481690367090563	0.28\\
0.49	0.288986525612472\\
0.490918356020625	0.29\\
0.5	0.3\\
0.5	0.3\\
0.509098503709708	0.31\\
0.51	0.310992622950819\\
0.51807453973956	0.32\\
0.52	0.322159706959707\\
0.526924899906984	0.33\\
0.53	0.333513770794824\\
0.535644962799115	0.34\\
0.54	0.345069662921348\\
0.5442283447495	0.35\\
0.55	0.356845238095238\\
0.552666446642893	0.36\\
0.56	0.368862256809339\\
0.560947850399125	0.37\\
0.569212370286654	0.38\\
0.57	0.380956655574043\\
0.577496362194407	0.39\\
0.58	0.393097610921502\\
0.585639906577892	0.4\\
0.59	0.405495518453427\\
0.59362881474717	0.41\\
0.6	0.418181818181818\\
0.601444470926411	0.42\\
0.609221746106886	0.43\\
0.61	0.431006279809221\\
0.617088813007111	0.44\\
0.62	0.443833663366337\\
0.62479461106656	0.45\\
0.63	0.456982702237521\\
0.632315097890323	0.46\\
0.639684243263875	0.47\\
0.64	0.470426911314985\\
0.647276150408795	0.48\\
0.65	0.48375\\
0.654689685393393	0.49\\
0.66	0.497446464646465\\
0.661892274584412	0.5\\
0.66904881898189	0.51\\
0.67	0.511344856278366\\
0.676348081777581	0.52\\
0.68	0.525257507987221\\
0.6834370061648	0.53\\
0.69	0.539624702886248\\
0.69026888292858	0.54\\
0.697398827391388	0.55\\
0.7	0.553846153846154\\
0.7043615983077	0.56\\
0.71	0.568490886699507\\
0.711058051518204	0.57\\
0.717917014704065	0.58\\
0.72	0.583163963963964\\
0.724739661140463	0.59\\
0.73	0.598136795491143\\
0.731278755560468	0.6\\
0.737954792860029	0.61\\
0.74	0.613191691394659\\
0.74461996164127	0.62\\
0.75	0.62855\\
0.750974729630167	0.63\\
0.757546877385507	0.64\\
0.76	0.64393353115727\\
0.764031651342224	0.65\\
0.77	0.659747101449275\\
0.770166644560575	0.66\\
0.776713032781283	0.67\\
0.78	0.675416216216216\\
0.782985988507153	0.68\\
0.789083455219926	0.69\\
0.79	0.69152461212976\\
0.795458282049768	0.7\\
0.8	0.707692307692307\\
0.801475078547933	0.71\\
0.807640748963193	0.72\\
0.81	0.724033309143687\\
0.813770827122842	0.73\\
0.819576133575376	0.74\\
0.82	0.740731680440771\\
0.825791827161025	0.75\\
0.83	0.757396293494705\\
0.831616882354323	0.76\\
0.83757744076984	0.77\\
0.84	0.77429855907781\\
0.843501422497135	0.78\\
0.849158041807314	0.79\\
0.85	0.791508620689655\\
0.855157953826626	0.8\\
0.86	0.808836697247706\\
0.860703288576645	0.81\\
0.866615369289315	0.82\\
0.87	0.826236343612335\\
0.872251004501043	0.83\\
0.877896507755257	0.84\\
0.88	0.84391388101983\\
0.883601751922714	0.85\\
0.889019726637347	0.86\\
0.89	0.861842181069959\\
0.89477787909585	0.87\\
0.9	0.88\\
0.9	0.88\\
0.905797347906578	0.89\\
0.91	0.898126980568012\\
0.911083180256305	0.9\\
0.916674789618047	0.91\\
0.92	0.916494460641399\\
0.922007387466349	0.92\\
0.927422259983008	0.93\\
0.93	0.935085520684736\\
0.93278720090613	0.94\\
0.938049788303087	0.95\\
0.94	0.953886834733893\\
0.943434538165251	0.96\\
0.948565782218097	0.97\\
0.95	0.972887931034483\\
0.953959204358292	0.98\\
0.958977329418185	0.99\\
0.96	0.992080653950954\\
0.964369294371888	1\\
0.969290424222532	1.01\\
0.97	1.01145877192982\\
0.974671488937686	1.02\\
0.979510138241414	1.03\\
0.98	1.03101769436997\\
0.984871272013522	1.04\\
0.989640748453969	1.05\\
0.99	1.05075427236315\\
0.994973088067906	1.06\\
0.999685831964084	1.07\\
1	1.07066666666667\\
1.00498045172697	1.08\\
1.009648333795	1.09\\
1.01	1.09075427236315\\
1.01489601781643	1.1\\
1.01953061192946	1.11\\
1.02	1.11101769436997\\
1.02472161643208	1.12\\
1.02933446211153	1.13\\
1.03	1.13145877192982\\
1.03445825482082	1.14\\
1.03906112349189	1.15\\
1.04	1.15208065395095\\
1.04410608516906	1.16\\
1.04871126485315	1.17\\
1.05	1.17288793103448\\
1.05366433452132	1.18\\
1.05828494974183	1.19\\
1.06	1.19388683473389\\
1.06313118958374	1.2\\
1.06778157718695	1.21\\
1.07	1.21508552068474\\
1.07250362457077	1.22\\
1.07719979258654	1.23\\
1.08	1.2364944606414\\
1.08177715373104	1.24\\
1.0865373604853	1.25\\
1.09	1.25812698056801\\
1.09094548047431	1.26\\
1.09579098688407	1.27\\
1.1	1.28\\
1.1	1.28\\
1.10495607267585	1.29\\
1.10922565076876	1.3\\
1.11	1.30184218106996\\
1.11402637057599	1.31\\
1.11837461213806	1.32\\
1.12	1.32391388101983\\
1.12299350343987	1.33\\
1.12744217420763	1.34\\
1.13	1.34623634361233\\
1.13184627852113	1.35\\
1.13642204646411	1.36\\
1.14	1.36883669724771\\
1.14056969347526	1.37\\
1.14530587329134	1.38\\
1.1493901436055	1.39\\
1.15	1.39150862068965\\
1.15408254279831	1.4\\
1.15828447708237	1.41\\
1.16	1.41429855907781\\
1.16273718070664	1.42\\
1.16708744023949	1.43\\
1.17	1.4373962934947\\
1.17124965056545	1.44\\
1.17578768472143	1.45\\
1.17971243351514	1.46\\
1.18	1.46073168044077\\
1.18436998912179	1.47\\
1.18843595840779	1.48\\
1.19	1.48403330914369\\
1.19281365815598	1.49\\
1.19705900517909	1.5\\
1.2	1.50769230769231\\
1.20109004102188	1.51\\
1.20556553572919	1.52\\
1.20941824825964	1.53\\
1.21	1.53152461212976\\
1.21393358972261	1.54\\
1.21796275326509	1.55\\
1.22	1.55541621621622\\
1.22213247646823	1.56\\
1.22638969327349	1.57\\
1.23	1.57974710144928\\
1.2301181871202	1.58\\
1.23467504323132	1.59\\
1.23854555429073	1.6\\
1.24	1.60393353115727\\
1.24278480236744	1.61\\
1.24688820239277	1.62\\
1.25	1.62855\\
1.25066943922231	1.63\\
1.25508258321416	1.64\\
1.25884078900927	1.65\\
1.26	1.65319169139466\\
1.26308973689969	1.66\\
1.26709295669308	1.67\\
1.27	1.67813679549114\\
1.27085175847029	1.68\\
1.2751858933635	1.69\\
1.27887220519957	1.7\\
1.28	1.70316396396396\\
1.28307336089781	1.71\\
1.28702541166319	1.72\\
1.29	1.72849088669951\\
1.29068488454846	1.73\\
1.29500282652625	1.74\\
1.29865584532543	1.75\\
1.3	1.75384615384615\\
1.30274745088073	1.76\\
1.3066981382088	1.77\\
1.31	1.77962470288625\\
1.31016961767922	1.78\\
1.31454024804865	1.79\\
1.31820076808484	1.8\\
1.32	1.80525750798722\\
1.32210885972545	1.81\\
1.3261148690051	1.82\\
1.32954125220936	1.83\\
1.33	1.83134485627837\\
1.33379273082851	1.84\\
1.33750865285777	1.85\\
1.34	1.85744646464646\\
1.34113528260438	1.86\\
1.34526885371529	1.87\\
1.34874936968232	1.88\\
1.35	1.88375\\
1.3527382592207	1.89\\
1.35657199072779	1.9\\
1.35985816935859	1.91\\
1.36	1.91042691131498\\
1.36413861942457	1.92\\
1.3677281852417	1.93\\
1.37	1.93698270223752\\
1.37132844558113	1.94\\
1.37536994591205	1.95\\
1.37875739787391	1.96\\
1.38	1.96383366336634\\
1.38267875768383	1.97\\
1.38645751282166	1.98\\
1.38967523954899	1.99\\
1.39	1.99100627980922\\
1.3938596102217	2\\
1	690\\
1.44511180331731	2\\
1.44186759179958	1.99\\
1.44	1.98393318284424\\
1.43849661881599	1.98\\
1.43487718692118	1.97\\
1.43163035641733	1.96\\
1.43	1.95475533926585\\
1.42817385546929	1.95\\
1.42456781637646	1.94\\
1.42132434451782	1.93\\
1.42	1.92578205689278\\
1.41776935009022	1.92\\
1.41418409418685	1.91\\
1.41094950045568	1.9\\
1.41	1.89700649838883\\
1.40728467034785	1.89\\
1.40372674742528	1.88\\
1.4005060208042	1.87\\
1.4	1.86842105263158\\
1.39672170782631	1.86\\
1.39319676607356	1.85\\
1.39	1.84001957520092\\
1.38999227116646	1.84\\
1.38608258072311	1.83\\
1.38259533881034	1.82\\
1.38	1.81198747203579\\
1.37921219309427	1.81\\
1.37536954008793	1.8\\
1.37192378915475	1.79\\
1.37	1.78412736670294\\
1.3683574359363	1.78\\
1.36458488319182	1.77\\
1.36118351443406	1.76\\
1.36	1.75642790697674\\
1.35743150344991	1.75\\
1.35373087659255	1.74\\
1.35037592951257	1.73\\
1.35	1.72887820512821\\
1.34643767657293	1.72\\
1.34280969045312	1.71\\
1.34	1.70163002207505\\
1.33933763161141	1.7\\
1.33537894916035	1.69\\
1.33182334476177	1.68\\
1.33	1.67463370607029\\
1.32811006743476	1.67\\
1.32425798284139	1.66\\
1.32077366734634	1.65\\
1.32	1.64775030800821\\
1.31682660192746	1.64\\
1.31307707885796	1.63\\
1.31	1.6210768935236\\
1.30955368374122	1.62\\
1.30548980218413	1.61\\
1.30183816433557	1.6\\
1.3	1.59473684210526\\
1.29802985427489	1.59\\
1.29410167532082	1.58\\
1.29054279030511	1.57\\
1.29	1.56846523713421\\
1.28646572226026	1.56\\
1.28266370159183	1.55\\
1.28	1.54249764453961\\
1.27894518612301	1.54\\
1.27486218744615	1.53\\
1.27117687368527	1.52\\
1.27	1.51672834525026\\
1.26714824716158	1.51\\
1.26321957215182	1.5\\
1.26	1.49108941684665\\
1.25953301809277	1.49\\
1.25532658304493	1.48\\
1.25153770018274	1.47\\
1.25	1.46580128205128\\
1.2475046818235	1.46\\
1.24347857195383	1.45\\
1.24	1.4405494600432\\
1.23976103420339	1.44\\
1.2354680105632	1.43\\
1.23160223774624	1.42\\
1.23	1.4157068947906\\
1.22750958913269	1.41\\
1.22341870862091	1.4\\
1.22	1.39089164882227\\
1.219606682803	1.39\\
1.21526651485463	1.38\\
1.21135245594661	1.37\\
1.21	1.36644707366297\\
1.20714635055976	1.36\\
1.20302430939895	1.35\\
1.2	1.3421052631579\\
1.19905833386302	1.34\\
1.1947098791948	1.33\\
1.19077595367991	1.32\\
1.19	1.31800301681503\\
1.18640753033112	1.31\\
1.18228633409997	1.3\\
1.18	1.2941568788501\\
1.17811488633586	1.29\\
1.17379345511011	1.28\\
1.17	1.27037385516507\\
1.16982870805226	1.27\\
1.16529401842339	1.26\\
1.16120186449606	1.25\\
1.16	1.24699483101392\\
1.15678399308975	1.24\\
1.15251906219577	1.23\\
1.15	1.22375\\
1.14825855125753	1.22\\
1.14381319665285	1.21\\
1.14	1.20061395348837\\
1.13971201547024	1.2\\
1.13507936401823	1.19\\
1.1308932217624	1.18\\
1.13	1.17783420019627\\
1.12631204019172	1.17\\
1.12197654443689	1.16\\
1.12	1.15524828973843\\
1.11750508733964	1.15\\
1.11301772923217	1.14\\
1.11	1.13280849639547\\
1.10865178066006	1.13\\
1.1040112225092	1.12\\
1.1	1.11052631578947\\
1.099744858256	1.11\\
1.09495110605202	1.1\\
1.09061598743794	1.09\\
1.09	1.08856745877789\\
1.08583115641924	1.08\\
1.08137647298064	1.07\\
1.08	1.0668291913215\\
1.07664492060224	1.06\\
1.07207327855725	1.05\\
1.07	1.04527032032032\\
1.06738580702821	1.04\\
1.06270100245844	1.03\\
1.06	1.02389939148073\\
1.05804718986416	1.02\\
1.05325423021256	1.01\\
1.05	1.00272435897436\\
1.04862252345924	1\\
1.04372761533585	0.99\\
1.04	0.981752380952381\\
1.0391054627122	0.98\\
1.03411596308425	0.97\\
1.03	0.960989624608967\\
1.02948998428117	0.96\\
1.02441431389396	0.95\\
1.02	0.94044109014675\\
1.01977050297915	0.94\\
1.0146180229061	0.93\\
1.01	0.920110462670872\\
1.00994197751456	0.92\\
1.00472283190944	0.91\\
1	0.9\\
1	0.9\\
0.994724930230923	0.89\\
0.99	0.880110462670873\\
0.989940864376486	0.88\\
0.984621001558205	0.87\\
0.98	0.86044109014675\\
0.979761610038665	0.86\\
0.974408254365044	0.85\\
0.97	0.840989624608968\\
0.969460038404096	0.84\\
0.964084434476129	0.83\\
0.96	0.821752380952381\\
0.959034701808508	0.82\\
0.953647819278998	0.81\\
0.95	0.802724358974359\\
0.948484865769802	0.8\\
0.943097194081801	0.79\\
0.94	0.78389939148073\\
0.93781044718746	0.78\\
0.932431812040814	0.77\\
0.93	0.765270320320321\\
0.92701193229381	0.76\\
0.921651339864665	0.75\\
0.92	0.746829191321499\\
0.916090279052042	0.74\\
0.910755792086175	0.73\\
0.91	0.728567458777886\\
0.905046809154912	0.72\\
0.9	0.710526315789474\\
0.899691223649798	0.71\\
0.893883094820372	0.7\\
0.89	0.692808496395469\\
0.888336175470173	0.69\\
0.882600845252681	0.68\\
0.88	0.67524828973843\\
0.876859950342337	0.67\\
0.871201797018583	0.66\\
0.87	0.657834200196271\\
0.865265726697208	0.65\\
0.86	0.640613953488372\\
0.859623457410427	0.64\\
0.853556410877438	0.63\\
0.85	0.62375\\
0.847677015397616	0.62\\
0.841734535513169	0.61\\
0.84	0.606994831013916\\
0.835622723667636	0.6\\
0.83	0.590373855165069\\
0.829762519237599	0.59\\
0.823463433187786	0.58\\
0.82	0.574156878850102\\
0.817332410930474	0.57\\
0.811201201826538	0.56\\
0.81	0.558003016815035\\
0.804810452150432	0.55\\
0.8	0.542105263157894\\
0.798614089855484	0.54\\
0.792197716115461	0.53\\
0.79	0.526447073662967\\
0.785710967525384	0.52\\
0.78	0.51089164882227\\
0.77939764569734	0.51\\
0.772733492675154	0.5\\
0.77	0.495706894790603\\
0.766103394765749	0.49\\
0.76	0.480549460043196\\
0.759618949189481	0.48\\
0.752754406208894	0.47\\
0.75	0.465801282051282\\
0.745932935225365	0.46\\
0.74	0.451089416846653\\
0.739224104447478	0.45\\
0.732213243042479	0.44\\
0.73	0.436728345250255\\
0.72515453224709	0.43\\
0.72	0.422497644539615\\
0.718172323964352	0.42\\
0.711071563188023	0.41\\
0.71	0.408465237134208\\
0.703734394897323	0.4\\
0.7	0.394736842105263\\
0.696436758400342	0.39\\
0.69	0.3810768935236\\
0.689176436021742	0.38\\
0.681649720491634	0.37\\
0.68	0.367750308008214\\
0.674003058529125	0.36\\
0.67	0.354633706070287\\
0.666354148773488	0.35\\
0.66	0.341630022075055\\
0.658695469743078	0.34\\
0.650865955955996	0.33\\
0.65	0.328878205128205\\
0.642807705569796	0.32\\
0.64	0.316427906976744\\
0.634702097536538	0.31\\
0.63	0.304127366702938\\
0.626538003422704	0.3\\
0.62	0.291987472035794\\
0.618302920318059	0.29\\
0.61	0.280019575200918\\
0.609982979020769	0.28\\
0.601336058893476	0.27\\
0.6	0.268421052631579\\
0.592583631146895	0.26\\
0.59	0.257006498388829\\
0.583714691783297	0.25\\
0.58	0.245782056892779\\
0.574715067876697	0.24\\
0.57	0.234755339265851\\
0.565570011139608	0.23\\
0.56	0.223933182844244\\
0.556264371204528	0.22\\
0.55	0.213321428571428\\
0.546782790628165	0.21\\
0.54	0.202924711316397\\
0.537109912993867	0.2\\
0.53	0.192746274738067\\
0.527230593138262	0.19\\
0.52	0.182787822014052\\
0.517130096736281	0.18\\
0.51	0.173049412455934\\
0.506794275514049	0.17\\
0.5	0.163529411764706\\
0.496209704420117	0.16\\
0.49	0.154224500587544\\
0.485363768256117	0.15\\
0.48	0.145129742388759\\
0.474244687489736	0.14\\
0.47	0.136238707799767\\
0.462841476039405	0.13\\
0.46	0.127543648960739\\
0.451143827398987	0.12\\
0.45	0.119035714285714\\
0.44	0.110794910941476\\
0.439012915202022	0.11\\
0.43	0.102859887359199\\
0.426361530193862	0.1\\
0.42	0.0950918918918918\\
0.413359554612088	0.0899999999999999\\
0.41	0.0874765944645007\\
0.4	0.0800000000000001\\
0.4	0.0799999999999999\\
0.39	0.0729922827496758\\
0.385735779307795	0.0699999999999998\\
0.38	0.0660916876574306\\
0.371082184835216	0.0600000000000001\\
0.37	0.0592833333333334\\
0.36	0.0528965147453082\\
0.355455300711949	0.0499999999999998\\
0.35	0.0466532258064517\\
0.34	0.0405053824362605\\
0.339145542716877	0.04\\
0.33	0.0348322733423546\\
0.321582179268077	0.0299999999999998\\
0.32	0.0291328165374676\\
0.31	0.0238833333333334\\
0.302658831975113	0.02\\
0.3	0.0186666666666666\\
0.29	0.0138481186685963\\
0.28205364213672	0.00999999999999979\\
0.28	0.00906376021798355\\
0.27	0.0047379234167894\\
0.26	0.000365495207667636\\
0.259110610095042	0\\
0.25	-0.00347222222222222\\
0.24	-0.00723833865814689\\
0.232489193005322	-0.01\\
0.23	-0.0108438144329897\\
0.22	-0.013986119873817\\
0.21	-0.017040524534687\\
0.2	-0.0199999999999999\\
0.2	-0.02\\
0.19	-0.0223878068739771\\
0.18	-0.0246188153310104\\
0.17	-0.0266770871985158\\
0.16	-0.0285438735177865\\
0.151128770092107	-0.03\\
0.15	-0.0301630434782609\\
0.14	-0.0313172161172161\\
0.13	-0.0322291907514451\\
0.12	-0.032880971659919\\
0.11	-0.0332546709129512\\
0.1	-0.0333333333333333\\
0.0899999999999999	-0.0331019721577726\\
0.0800000000000001	-0.0325487922705314\\
0.0699999999999998	-0.0316665413533834\\
0.0600000000000001	-0.0304538860103627\\
0.0569868432367359	-0.03\\
0.0499999999999998	-0.0285227272727272\\
0.04	-0.0259669172932331\\
0.0299999999999998	-0.0229766409266409\\
0.0211562948092917	-0.02\\
0.02	-0.0193454545454546\\
0.00999999999999979	-0.0132115894039734\\
0.00507639981724961	-0.01\\
0	0\\
0.00497537190904999	0.00999999999999979\\
0.00999999999999979	0.0133889261744965\\
0.0194330859468518	0.02\\
0.02	0.0202406504065041\\
0.0299999999999998	0.0246927385892115\\
0.04	0.0295179487179488\\
0.0409348970204297	0.0299999999999998\\
0.0499999999999998	0.0333653846153845\\
0.0600000000000001	0.0374369426751592\\
0.0657495981425256	0.04\\
0.0699999999999998	0.0414836658354114\\
0.0800000000000001	0.0452725388601037\\
0.0899999999999999	0.0495039295392953\\
0.0910876088054421	0.0499999999999998\\
0.1	0.0533333333333334\\
0.11	0.0574893939393939\\
0.115494452238235	0.0600000000000001\\
0.12	0.0617454545454546\\
0.13	0.065960395010395\\
0.138589015360813	0.0699999999999998\\
0.14	0.0705761732851986\\
0.15	0.0749404761904761\\
0.16	0.0798445344129555\\
0.16029816558284	0.0800000000000001\\
0.17	0.0844544563279857\\
0.18	0.0896030418250952\\
0.180725566849101	0.0899999999999999\\
0.19	0.094531324278438\\
0.2	0.1\\
0.2	0.1\\
0.21	0.105206814449918\\
0.218278057426444	0.11\\
0.22	0.110911711711712\\
0.23	0.116526489533011\\
0.235664150200223	0.12\\
0.24	0.122449851632048\\
0.25	0.12855\\
0.252233545699885	0.13\\
0.26	0.134675370919881\\
0.268097929711739	0.14\\
0.27	0.141169140083218\\
0.28	0.147668468468468\\
0.283351065551659	0.15\\
0.29	0.154345486600846\\
0.298003613459638	0.16\\
0.3	0.161333333333333\\
0.31	0.168387445573295\\
0.312160630074384	0.17\\
0.32	0.17555261707989\\
0.325862284962099	0.18\\
0.33	0.182994678055191\\
0.33907528620675	0.19\\
0.34	0.190684130982368\\
0.35	0.198405172413793\\
0.351961258258488	0.2\\
0.36	0.206285411140583\\
0.364490080557893	0.21\\
0.37	0.214400704225352\\
0.376649762402585	0.22\\
0.38	0.222733498759305\\
0.388476138789392	0.23\\
0.39	0.231270144752714\\
0.4	0.24\\
0.4	0.24\\
0.41	0.24877366710013\\
0.411339892729016	0.25\\
0.42	0.257754707379135\\
0.422411011305031	0.26\\
0.43	0.266935018726592\\
0.433234478442739	0.27\\
0.44	0.276308599508599\\
0.443828477871709	0.28\\
0.45	0.285871212121212\\
0.454208650204795	0.29\\
0.46	0.295620143884892\\
0.464388406667668	0.3\\
0.47	0.305554042806183\\
0.474379176060668	0.31\\
0.48	0.315672813238771\\
0.484190596113671	0.32\\
0.49	0.325977561837456\\
0.493830657147184	0.33\\
0.5	0.336470588235294\\
0.503305803337759	0.34\\
0.51	0.347155418138987\\
0.512620994655018	0.35\\
0.52	0.358036879432624\\
0.521779730489994	0.36\\
0.53	0.369121224732461\\
0.530784033947752	0.37\\
0.539669215195903	0.38\\
0.54	0.380371734475375\\
0.548494552049535	0.39\\
0.55	0.391722972972973\\
0.55718798848983	0.4\\
0.56	0.403277024070022\\
0.565748310297247	0.41\\
0.57	0.415044339622641\\
0.574172846428822	0.42\\
0.58	0.427037471783296\\
0.582457275259204	0.43\\
0.59	0.439271518987342\\
0.590595360691429	0.44\\
0.598731291840938	0.45\\
0.6	0.451578947368421\\
0.606801015051496	0.46\\
0.61	0.464050538213132\\
0.614736016794842	0.47\\
0.62	0.476780573951435\\
0.622526735246345	0.48\\
0.63	0.489792224744608\\
0.630160820575897	0.49\\
0.637898239443577	0.5\\
0.64	0.502787421383648\\
0.645515870311955	0.51\\
0.65	0.516047297297297\\
0.652982393174245	0.52\\
0.66	0.52962326621924\\
0.660280532244966	0.53\\
0.667706203250661	0.54\\
0.67	0.543183090530697\\
0.675015253880457	0.55\\
0.68	0.557031533477322\\
0.682156302092526	0.56\\
0.689214499477608	0.57\\
0.69	0.571121284125379\\
0.696371829432445	0.58\\
0.7	0.585263157894737\\
0.703357970890123	0.59\\
0.71	0.59979202420242\\
0.710145689239958	0.6\\
0.717135132321041	0.61\\
0.72	0.614272463768116\\
0.723964530748974	0.62\\
0.73	0.629145439739414\\
0.730584808574666	0.63\\
0.737366603587159	0.64\\
0.74	0.64403613963039\\
0.744033174680872	0.65\\
0.75	0.65929054054054\\
0.750474599438074	0.66\\
0.757110862627099	0.67\\
0.76	0.674549486652977\\
0.763603645188114	0.68\\
0.769869048534137	0.69\\
0.77	0.690208570029383\\
0.776398419124118	0.7\\
0.78	0.705825258799172\\
0.782700460366619	0.71\\
0.788902703930925	0.72\\
0.79	0.721794796828543\\
0.795246862136821	0.73\\
0.8	0.737894736842105\\
0.801333426673201	0.74\\
0.807526062755322	0.75\\
0.81	0.754154651162791\\
0.813660687995694	0.76\\
0.819568827218069	0.77\\
0.82	0.77073216374269\\
0.825739343000709	0.78\\
0.83	0.787345421436004\\
0.831629621348835	0.79\\
0.837596894526458	0.8\\
0.84	0.804188329979879\\
0.843528775758909	0.81\\
0.84925597503952	0.82\\
0.85	0.821310975609756\\
0.855218877908288	0.83\\
0.86	0.838552620545073\\
0.860862582186753	0.84\\
0.866720581809982	0.85\\
0.87	0.855929612640163\\
0.872400210383951	0.86\\
0.878051152355701	0.87\\
0.88	0.873561829025845\\
0.883756899026958	0.88\\
0.889225160772761	0.89\\
0.89	0.891431438289602\\
0.894948658728577	0.9\\
0.9	0.909473684210526\\
0.900301987986918	0.91\\
0.905989018318229	0.92\\
0.91	0.927602631578948\\
0.911361833716823	0.93\\
0.916889494791736	0.94\\
0.92	0.945958620689655\\
0.922273164192302	0.95\\
0.927659954333776	0.96\\
0.93	0.96453041958042\\
0.933046730012103	0.97\\
0.938308892277063	0.98\\
0.94	0.983308875739645\\
0.943691694937849	0.99\\
0.948843651864588	1\\
0.95	1.00228658536585\\
0.954215890942733	1.01\\
0.959270596149239	1.02\\
0.96	1.02145764023211\\
0.964626016366612	1.03\\
0.969595243474642	1.04\\
0.97	1.0408174351585\\
0.974927789877953	1.05\\
0.97982237419929	1.06\\
0.98	1.06036252390057\\
0.98512606943544	1.07\\
0.989956114301561	1.08\\
0.99	1.08009051477598\\
0.995224942887804	1.09\\
1	1.1\\
1	1.1\\
1.00522779494099	1.11\\
1.00995702637933	1.12\\
1.01	1.12009051477598\\
1.01513735374427	1.13\\
1.0198296821137	1.14\\
1.02	1.14036252390057\\
1.02495571914904	1.15\\
1.02961997164134	1.16\\
1.03	1.1608174351585\\
1.03468437366003	1.17\\
1.0393294255047	1.18\\
1.04	1.18145764023211\\
1.04432417613351	1.19\\
1.04895909897387	1.2\\
1.05	1.20228658536585\\
1.0538753372922	1.21\\
1.05850955846913	1.22\\
1.06	1.22330887573965\\
1.06333737502921	1.23\\
1.06798085463997	1.24\\
1.07	1.24453041958042\\
1.07270904615232	1.25\\
1.07737248018247	1.26\\
1.08	1.26595862068966\\
1.08198824952911	1.27\\
1.08668330950961	1.28\\
1.09	1.28760263157895\\
1.0911718933262	1.29\\
1.09591151612339	1.3\\
1.1	1.30947368421053\\
1.10025571588942	1.31\\
1.10505446182618	1.32\\
1.10938557832891	1.33\\
1.11	1.3314314382896\\
1.11410854952762	1.34\\
1.11848747902618	1.35\\
1.12	1.35356182902585\\
1.12306902801635	1.36\\
1.1275092008778	1.37\\
1.13	1.37592961264016\\
1.13192973214609	1.38\\
1.13644668967545	1.39\\
1.14	1.39855262054507\\
1.14068273459731	1.4\\
1.14529470942124	1.41\\
1.14945826471353	1.42\\
1.15	1.42131097560976\\
1.15404655727426	1.43\\
1.15828811918033	1.44\\
1.16	1.44418832997988\\
1.16269368056608	1.45\\
1.16703102951537	1.46\\
1.17	1.467345421436\\
1.17122515578288	1.47\\
1.17567965993807	1.48\\
1.1797056685833	1.49\\
1.18	1.49073216374269\\
1.18422467816641	1.5\\
1.18834834339922	1.51\\
1.19	1.51415465116279\\
1.19265416497368	1.52\\
1.19689711452337	1.53\\
1.2	1.53789473684211\\
1.20095279156017	1.54\\
1.20534166347947	1.55\\
1.20929837917628	1.56\\
1.21	1.56179479682854\\
1.21366877527281	1.57\\
1.21774878231735	1.58\\
1.22	1.58582525879917\\
1.22186137105801	1.59\\
1.22609261582161	1.6\\
1.229919840396	1.61\\
1.23	1.61020857002938\\
1.23431497847269	1.62\\
1.23827164490556	1.63\\
1.24	1.63454948665298\\
1.24239647354368	1.64\\
1.24651263769375	1.65\\
1.25	1.65929054054054\\
1.25031149949329	1.66\\
1.25462571228017	1.67\\
1.25849306850612	1.68\\
1.26	1.68403613963039\\
1.26258838994285	1.69\\
1.26662700851963	1.7\\
1.27	1.70914543973941\\
1.27037065118853	1.71\\
1.27462366902467	1.72\\
1.27843276341646	1.73\\
1.28	1.73427246376812\\
1.2824564293246	1.74\\
1.28645292154421	1.75\\
1.29	1.75979202420242\\
1.29008922061159	1.76\\
1.29432273366682	1.77\\
1.29810383764299	1.78\\
1.3	1.78526315789474\\
1.30200992007218	1.79\\
1.30600025201058	1.8\\
1.30959833677123	1.81\\
1.31	1.81112128412538\\
1.31372832744218	1.82\\
1.31751312434951	1.83\\
1.32	1.83703153347732\\
1.32124788051435	1.84\\
1.32527146923848	1.85\\
1.32888064330214	1.86\\
1.33	1.8631830905307\\
1.33283665847559	1.87\\
1.33666079936346	1.88\\
1.34	1.88962326621924\\
1.34015715882234	1.89\\
1.34426059936444	1.9\\
1.34791360117566	1.91\\
1.35	1.9160472972973\\
1.35163258608333	1.92\\
1.35553913592467	1.93\\
1.35904398454816	1.94\\
1.36	1.94278742138365\\
1.36295088505538	1.95\\
1.36668802487739	1.96\\
1.37	1.96979222474461\\
1.37008535291718	1.97\\
1.37412999948339	1.98\\
1.37772020225449	1.99\\
1.38	1.99678057395143\\
1.38131034715125	2\\
2	804\\
1.46065268807895	2\\
1.46	1.99793030746706\\
1.45710558413746	1.99\\
1.45367997510942	1.98\\
1.45048414266572	1.97\\
1.45	1.96847727272727\\
1.44688763719429	1.96\\
1.44345431311963	1.95\\
1.44024870782729	1.94\\
1.44	1.93922424242424\\
1.43659342255144	1.93\\
1.43315718537862	1.92\\
1.43	1.91018094688222\\
1.42993247678485	1.91\\
1.42622279459111	1.9\\
1.4227882326078	1.89\\
1.42	1.88140395738204\\
1.41947309117653	1.88\\
1.41577588245787	1.87\\
1.41234731724108	1.86\\
1.41	1.85282723516153\\
1.40893286315279	1.85\\
1.40525305957889	1.84\\
1.4018345010952	1.83\\
1.4	1.82444444444444\\
1.39831285087429	1.82\\
1.39465490920516	1.81\\
1.39125001974822	1.8\\
1.39	1.79624876002918\\
1.38761431624675	1.79\\
1.38398218729196	1.78\\
1.38059425455818	1.77\\
1.38	1.76823299856528\\
1.37683867369073	1.76\\
1.37323578401562	1.75\\
1.37	1.74041929492039\\
1.36983739682026	1.74\\
1.36598743952806	1.73\\
1.36241668513956	1.72\\
1.36	1.71291292719168\\
1.35886401532107	1.71\\
1.35506218342465	1.7\\
1.35152593431001	1.69\\
1.35	1.68556818181818\\
1.34781620932445	1.68\\
1.34406448305869	1.67\\
1.34056459719408	1.66\\
1.34	1.65837610241821\\
1.33669615592322	1.65\\
1.33299588289411	1.64\\
1.33	1.63142722180732\\
1.32943161150726	1.63\\
1.32550591705781	1.62\\
1.32185785765211	1.61\\
1.32	1.60473711790393\\
1.31810281933382	1.6\\
1.31424740272485	1.59\\
1.31065178080634	1.58\\
1.31	1.57817225372077\\
1.30670833446328	1.57\\
1.30292234143309	1.56\\
1.3	1.55185185185185\\
1.29924700949822	1.55\\
1.29525019592942	1.54\\
1.29153225753922	1.53\\
1.29	1.52577214234364\\
1.28763964281777	1.52\\
1.28373014452151	1.51\\
1.28007845490822	1.5\\
1.28	1.49978577405858\\
1.27597519055233	1.49\\
1.27214961613759	1.48\\
1.27	1.47416754894851\\
1.26826991187073	1.47\\
1.26425508940997	1.46\\
1.26050974117155	1.45\\
1.26	1.4486316970547\\
1.25639593529619	1.44\\
1.25248039211675	1.43\\
1.25	1.42338636363636\\
1.24857290410375	1.42\\
1.24447390115171	1.41\\
1.2406517857499	1.4\\
1.24	1.39828106591865\\
1.23648983777598	1.39\\
1.23250415723414	1.38\\
1.23	1.3734423857868\\
1.22852732729938	1.37\\
1.22436716953661	1.36\\
1.22048666662178	1.35\\
1.22	1.34873974895398\\
1.21623907873514	1.34\\
1.21220423032549	1.33\\
1.21	1.32433432782171\\
1.20811763071337	1.32\\
1.2039199687564	1.31\\
1.2	1.3\\
1.2	1.3\\
1.19563107600572	1.29\\
1.19156804796094	1.28\\
1.19	1.27604610205528\\
1.18733423431165	1.27\\
1.18312205340064	1.26\\
1.18	1.25218981077147\\
1.17902557666928	1.25\\
1.17465844061063	1.24\\
1.17058715067704	1.23\\
1.17	1.22854833217512\\
1.16617317249586	1.22\\
1.16196761350863	1.21\\
1.16	1.20517553342817\\
1.15766173798342	1.2\\
1.15331844675005	1.19\\
1.15	1.18193181818182\\
1.14911914761677	1.18\\
1.14463522620275	1.17\\
1.14045463640172	1.16\\
1.14	1.15890788381743\\
1.13591316004474	1.15\\
1.13161210793853	1.14\\
1.13	1.13616141649049\\
1.12714710608965	1.13\\
1.12272468746438	1.12\\
1.12	1.11357015781923\\
1.11833159848809	1.11\\
1.11378759971616	1.1\\
1.11	1.09114299781182\\
1.10946088440347	1.09\\
1.10479584871335	1.08\\
1.10044793435074	1.07\\
1.1	1.06896551724138\\
1.09574425597318	1.06\\
1.09129424091126	1.05\\
1.09	1.0470391684137\\
1.08662750579629	1.04\\
1.08207726701677	1.03\\
1.08	1.02529335219236\\
1.07744019699211	1.02\\
1.07279229684849	1.01\\
1.07	1.0037348463188\\
1.06817690006193	1\\
1.06343459374315	0.99\\
1.06	0.982369985569986\\
1.05883221851109	0.98\\
1.05399944772484	0.97\\
1.05	0.961204545454545\\
1.04940085264321	0.96\\
1.04448222456526	0.95\\
1.04	0.940243631039532\\
1.03987766392501	0.94\\
1.03487841496221	0.93\\
1.03022044267198	0.92\\
1.03	0.919526422206991\\
1.02518368230922	0.91\\
1.02045916543318	0.9\\
1.02	0.899023933975241\\
1.01539390748006	0.89\\
1.01060786642963	0.88\\
1.01	0.878721605789111\\
1.00550522907969	0.87\\
1.00066301895737	0.86\\
1	0.858620689655172\\
0.995514077721295	0.85\\
0.990621320752505	0.84\\
0.99	0.838721605789111\\
0.9854172030745	0.83\\
0.980479714713065	0.82\\
0.98	0.819023933975241\\
0.975211692683861	0.81\\
0.970235403340516	0.8\\
0.97	0.799526422206991\\
0.964894981864928	0.79\\
0.96	0.780243631039531\\
0.959867813501055	0.78\\
0.954464854328071	0.77\\
0.95	0.761204545454546\\
0.949339914592421	0.76\\
0.943919433536745	0.75\\
0.94	0.742369985569985\\
0.938688165816352	0.74\\
0.933257165153894	0.73\\
0.93	0.723734846318799\\
0.927911671546048	0.72\\
0.922476791246825	0.71\\
0.92	0.705293352192362\\
0.917009877455631	0.7\\
0.91157731718894	0.69\\
0.91	0.687039168413697\\
0.905982519079251	0.68\\
0.900557972402685	0.67\\
0.9	0.668965517241379\\
0.894829564549904	0.66\\
0.89	0.651142997811816\\
0.889332324964583	0.65\\
0.883551153415675	0.64\\
0.88	0.633570157819225\\
0.877892414377217	0.63\\
0.872147533389187	0.62\\
0.87	0.616161416490487\\
0.866323707807293	0.61\\
0.86061899675646	0.6\\
0.86	0.598907883817427\\
0.85462730270224	0.59\\
0.85	0.581931818181818\\
0.848819415585714	0.58\\
0.842804197697757	0.57\\
0.84	0.565175533428165\\
0.836802214824825	0.56\\
0.830855224506924	0.55\\
0.83	0.548548332175122\\
0.824659206113167	0.54\\
0.82	0.53218981077147\\
0.81861331244023	0.53\\
0.812391550835379	0.52\\
0.81	0.51604610205528\\
0.806127715201953	0.51\\
0.8	0.5\\
0.8	0.5\\
0.793521275906519	0.49\\
0.79	0.484334327821711\\
0.787152918954017	0.48\\
0.780794504588669	0.47\\
0.78	0.468739748953975\\
0.774191256387379	0.46\\
0.77	0.453442385786802\\
0.767679640242934	0.45\\
0.761114777210556	0.44\\
0.76	0.438281065918653\\
0.754350545746811	0.43\\
0.75	0.423386363636364\\
0.747656342409318	0.42\\
0.740913592212554	0.41\\
0.74	0.408631697054699\\
0.733951969732333	0.4\\
0.73	0.394167548948513\\
0.727036724802454	0.39\\
0.720147378773523	0.38\\
0.72	0.379785774058578\\
0.712953210721235	0.37\\
0.71	0.365772142343638\\
0.705780635396932	0.36\\
0.7	0.351851851851852\\
0.698625539132275	0.35\\
0.691317205468117	0.34\\
0.69	0.338172253720765\\
0.683854121973644	0.33\\
0.68	0.32473711790393\\
0.676381537435199	0.32\\
0.67	0.311427221807319\\
0.668892417810105	0.31\\
0.661228567200721	0.3\\
0.66	0.298376102418208\\
0.653409939817185	0.29\\
0.65	0.285568181818182\\
0.645547514022605	0.28\\
0.64	0.272912927191679\\
0.637632045416799	0.27\\
0.63	0.260419294920394\\
0.6296533960047	0.26\\
0.62143157282964	0.25\\
0.62	0.24823299856528\\
0.613094480963914	0.24\\
0.61	0.236248760029176\\
0.604667938420556	0.23\\
0.6	0.224444444444445\\
0.59614072213521	0.22\\
0.59	0.212827235161533\\
0.587501004432786	0.21\\
0.58	0.20140395738204\\
0.578736416719465	0.2\\
0.57	0.190180946882217\\
0.569834124309733	0.19\\
0.560702733615464	0.18\\
0.56	0.179224242424242\\
0.551407227182756	0.17\\
0.55	0.168477272727273\\
0.541952021099945	0.16\\
0.54	0.157930307467057\\
0.532324647949694	0.15\\
0.53	0.147585761589404\\
0.522512447115647	0.14\\
0.52	0.137445199409158\\
0.512502612349627	0.13\\
0.51	0.127509289415248\\
0.502282227910887	0.12\\
0.5	0.117777777777778\\
0.491838289305438	0.11\\
0.49	0.108249481865285\\
0.481157704742448	0.1\\
0.48	0.0989223042836041\\
0.47022727364653	0.0899999999999999\\
0.47	0.0897932671081679\\
0.46	0.080926382306477\\
0.458935419890982	0.0800000000000001\\
0.45	0.072279411764706\\
0.447319316143136	0.0699999999999998\\
0.44	0.0638295489891135\\
0.435388656101501	0.0600000000000001\\
0.43	0.0555697074672827\\
0.423129270435175	0.0499999999999998\\
0.42	0.0474922374429223\\
0.410526330983713	0.04\\
0.41	0.039589068369647\\
0.4	0.0319999999999999\\
0.397327046546811	0.0299999999999998\\
0.39	0.0246082218725414\\
0.383675062768284	0.02\\
0.38	0.0173700154559505\\
0.37	0.0102978260869566\\
0.369565846512601	0.00999999999999979\\
0.36	0.00358009630818611\\
0.354628486793384	0\\
0.35	-0.00301470588235289\\
0.34	-0.00945538971807638\\
0.339127263063482	-0.01\\
0.33	-0.0155439467312348\\
0.32263098535153	-0.02\\
0.32	-0.0215472527472528\\
0.31	-0.0272658546655656\\
0.305138708422343	-0.03\\
0.3	-0.0328000000000001\\
0.29	-0.0381452141057934\\
0.28643485979423	-0.04\\
0.28	-0.0432311183144247\\
0.27	-0.0481619592875318\\
0.26615652019687	-0.05\\
0.26	-0.0528313213703101\\
0.25	-0.057313829787234\\
0.243844406088796	-0.0600000000000001\\
0.24	-0.0616078303425774\\
0.23	-0.0656174300254453\\
0.22	-0.069547795414462\\
0.218783031741323	-0.0700000000000001\\
0.21	-0.0731074307304786\\
0.2	-0.0765217391304347\\
0.19	-0.0798190369036904\\
0.189410799325958	-0.0800000000000001\\
0.18	-0.0827352640545144\\
0.17	-0.0854951273046532\\
0.16	-0.0881041591320072\\
0.152206797866607	-0.0900000000000001\\
0.15	-0.0905053191489362\\
0.14	-0.0925821989528796\\
0.13	-0.0944834226988383\\
0.12	-0.096200365630713\\
0.11	-0.0977245098039216\\
0.1	-0.099047619047619\\
0.0914306525558071	-0.1\\
0.0899999999999999	-0.100147612732096\\
0.0800000000000001	-0.100965170556553\\
0.0699999999999998	-0.101578662420382\\
0.0600000000000001	-0.10198453038674\\
0.0499999999999998	-0.10218023255814\\
0.04	-0.10216435272045\\
0.0299999999999998	-0.101936685552408\\
0.02	-0.101498292220114\\
0.00999999999999979	-0.100851522359657\\
9.92433565785923e-16	-0.1\\
0	-0.1\\
-0.01	-0.0988380126182965\\
-0.02	-0.0974624737945493\\
-0.03	-0.0958821167883212\\
-0.04	-0.0941068322981366\\
-0.05	-0.0921474358974359\\
-0.0600000000000001	-0.090015415821501\\
-0.0600663683004036	-0.0900000000000001\\
-0.0700000000000001	-0.0874693548387097\\
-0.0800000000000001	-0.0847649890590809\\
-0.0900000000000001	-0.081919387755102\\
-0.0964118480776474	-0.0800000000000001\\
-0.1	-0.0788235294117647\\
-0.11	-0.0753880022962112\\
-0.12	-0.0718604026845637\\
-0.125092078071182	-0.0700000000000001\\
-0.13	-0.0680426739926741\\
-0.14	-0.0639470449172577\\
-0.149552990315982	-0.0600000000000001\\
-0.15	-0.0597983870967742\\
-0.16	-0.0551404466501241\\
-0.17	-0.05051126340882\\
-0.171072978550435	-0.05\\
-0.18	-0.0453891472868217\\
-0.19	-0.0402946362515413\\
-0.1905637009165	-0.04\\
-0.2	-0.0346666666666667\\
-0.20841538332071	-0.03\\
-0.21	-0.0290462373371925\\
-0.22	-0.0229798365122616\\
-0.224966112277913	-0.02\\
-0.23	-0.0167348306332843\\
-0.24	-0.0103735537190083\\
-0.240579803097842	-0.01\\
-0.25	-0.00347222222222222\\
-0.255125695727767	0\\
-0.26	0.00356038338658148\\
-0.26904211293669	0.00999999999999979\\
-0.27	0.0107375647668394\\
-0.28	0.0183798107255521\\
-0.282150641193405	0.02\\
-0.29	0.0263435702199662\\
-0.294636286593934	0.0299999999999998\\
-0.3	0.0345454545454546\\
-0.306606401603154	0.04\\
-0.31	0.0430157534246576\\
-0.318078659951332	0.0499999999999998\\
-0.32	0.0517907172995781\\
-0.329065641563283	0.0600000000000001\\
-0.33	0.0609135535307518\\
-0.339575586629058	0.0699999999999998\\
-0.34	0.0704354679802957\\
-0.349613189914469	0.0800000000000001\\
-0.35	0.0804166666666668\\
-0.359180493287862	0.0899999999999999\\
-0.36	0.0909271676300579\\
-0.368277935262707	0.1\\
-0.37	0.102047178683386\\
-0.376905614154474	0.11\\
-0.38	0.113866666666667\\
-0.385064812864265	0.12\\
-0.39	0.126483579335793\\
-0.392759814072025	0.13\\
-0.4	0.14\\
-0.4	0.14\\
-0.405898429051729	0.15\\
-0.41	0.157962213740458\\
-0.41109776909633	0.16\\
-0.41360199781489	0.17\\
-0.41	0.178361594202899\\
-0.4	0.18\\
-0.39	0.177829475982533\\
-0.38	0.174912418300654\\
-0.37	0.171776771653543\\
-0.364862834376778	0.17\\
-0.36	0.168189830508475\\
-0.35	0.164338235294118\\
-0.34	0.160735222672065\\
-0.338051763443744	0.16\\
-0.33	0.156743926247289\\
-0.32	0.152949049429658\\
-0.311758799309337	0.15\\
-0.31	0.149323006134969\\
-0.3	0.145454545454545\\
-0.29	0.141922742200328\\
-0.284326105390673	0.14\\
-0.28	0.138422614840989\\
-0.27	0.134916183574879\\
-0.26	0.131708011869436\\
-0.254404724294636	0.13\\
-0.25	0.12855\\
-0.24	0.125417210682493\\
-0.23	0.122556102635229\\
-0.220278517737044	0.12\\
-0.22	0.119920720720721\\
-0.21	0.117166361071932\\
-0.2	0.114666666666667\\
-0.19	0.112394106463878\\
-0.18	0.110328329297821\\
-0.178307380246528	0.11\\
-0.17	0.108250919842313\\
-0.16	0.106351637279597\\
-0.15	0.104659090909091\\
-0.14	0.103160655737705\\
-0.13	0.101846708286039\\
-0.12	0.100709933774834\\
-0.112692984947459	0.1\\
-0.11	0.0997140530759952\\
-0.1	0.0988235294117647\\
-0.0900000000000001	0.0981207710011507\\
-0.0800000000000001	0.0976018058690745\\
-0.0700000000000001	0.0972640954495006\\
-0.0600000000000001	0.0971063457330416\\
-0.05	0.0971283783783784\\
-0.04	0.0973310492505353\\
-0.03	0.0977162061636557\\
-0.02	0.0982866807610994\\
-0.01	0.0990463119072708\\
0	0.1\\
8.11158883693437e-16	0.1\\
0.00999999999999979	0.101043803622498\\
0.02	0.102274569789675\\
0.0299999999999998	0.103699279538905\\
0.04	0.105326112185687\\
0.0499999999999998	0.107164634146341\\
0.0600000000000001	0.109226035502959\\
0.0633977703214628	0.11\\
0.0699999999999998	0.111385059037239\\
0.0800000000000001	0.113697237569061\\
0.0899999999999999	0.116245977549111\\
0.1	0.119047619047619\\
0.103141739581206	0.12\\
0.11	0.121933525243578\\
0.12	0.124993851717902\\
0.13	0.128332053654024\\
0.134644342842424	0.13\\
0.14	0.131801733102253\\
0.15	0.135416666666667\\
0.16	0.13934478976234\\
0.161573396257477	0.14\\
0.17	0.143306589147287\\
0.18	0.147541030195382\\
0.185431415820705	0.15\\
0.19	0.151958746846089\\
0.2	0.156521739130435\\
0.207103597083974	0.16\\
0.21	0.161349007444169\\
0.22	0.166266895368782\\
0.227082618069397	0.17\\
0.23	0.171468427518428\\
0.24	0.176772742759796\\
0.245712433377811	0.18\\
0.25	0.182321428571429\\
0.26	0.188050425894378\\
0.263231969499064	0.19\\
0.27	0.19392542997543\\
0.279805400403213	0.2\\
0.28	0.200116271721959\\
0.29	0.206311786600496\\
0.295602490246952	0.21\\
0.3	0.2128\\
0.31	0.219528132884777\\
0.310676595616389	0.22\\
0.32	0.22630603588907\\
0.325171191230624	0.23\\
0.33	0.233353647898493\\
0.339085887741551	0.24\\
0.34	0.240651622874807\\
0.35	0.248035714285714\\
0.352547522881387	0.25\\
0.36	0.255613397129187\\
0.365563247452656	0.26\\
0.37	0.263424629195941\\
0.378156930153734	0.27\\
0.38	0.271457274119449\\
0.39	0.279676932465419\\
0.390381887169918	0.28\\
0.4	0.288\\
0.402315468720333	0.29\\
0.41	0.296538179669031\\
0.413926485375282	0.3\\
0.42	0.30528398133748\\
0.425241945590873	0.31\\
0.43	0.314231322059954\\
0.436285504909864	0.32\\
0.44	0.323375342465753\\
0.447077907558208	0.33\\
0.45	0.332712264150944\\
0.45763735372861	0.34\\
0.46	0.34223928035982\\
0.467979805376241	0.35\\
0.47	0.351954474272931\\
0.478119240682558	0.36\\
0.48	0.36185676077266\\
0.488067865226432	0.37\\
0.49	0.371945848776872\\
0.497836286201636	0.38\\
0.5	0.382222222222222\\
0.507433654632406	0.39\\
0.51	0.392687138621201\\
0.516867779382843	0.4\\
0.52	0.4033426448737\\
0.526145215766283	0.41\\
0.53	0.414191610738255\\
0.535271330682683	0.42\\
0.54	0.425237781109445\\
0.544250345397433	0.43\\
0.55	0.436485849056604\\
0.553085356277892	0.44\\
0.56	0.447941552511416\\
0.561778332977202	0.45\\
0.57	0.459611798616449\\
0.570330092645545	0.46\\
0.578823604504491	0.47\\
0.58	0.471396248196248\\
0.587213271432244	0.48\\
0.59	0.48336519357195\\
0.595477309662577	0.49\\
0.6	0.495555555555556\\
0.603613815617104	0.5\\
0.61	0.507978141459744\\
0.611619656123325	0.51\\
0.619526830431672	0.52\\
0.62	0.520599715504979\\
0.627427776606955	0.53\\
0.63	0.533314228819696\\
0.635206925235094	0.54\\
0.64	0.546277104874446\\
0.642858479518899	0.55\\
0.65	0.559504716981132\\
0.650375011786085	0.56\\
0.657924872869754	0.57\\
0.66	0.5728\\
0.66538000348418	0.58\\
0.67	0.586332806759735\\
0.672703381223849	0.59\\
0.679893399285796	0.6\\
0.68	0.600148106591866\\
0.687168578025539	0.61\\
0.69	0.613994924406047\\
0.694312614878717	0.62\\
0.7	0.628148148148148\\
0.701312959700015	0.63\\
0.708312572276594	0.64\\
0.71	0.642449219304471\\
0.71528615613937	0.65\\
0.72	0.656945241581259\\
0.722112195183681	0.66\\
0.728880987468022	0.67\\
0.73	0.671669915552428\\
0.73568863311427	0.68\\
0.74	0.686529257641922\\
0.742341946367228	0.69\\
0.74892747070708	0.7\\
0.75	0.701644736842105\\
0.75556953705891	0.71\\
0.76	0.716893158660844\\
0.762047214262592	0.72\\
0.76849299094128	0.73\\
0.77	0.732373645320197\\
0.774965600192447	0.74\\
0.78	0.748043338213763\\
0.78125993754051	0.75\\
0.787607527294673	0.76\\
0.79	0.763868665720369\\
0.793902177374368	0.77\\
0.8	0.78\\
0.8	0.78\\
0.806290959024164	0.79\\
0.81	0.796154751619871\\
0.812393774720165	0.8\\
0.818449936827157	0.81\\
0.82	0.812602524544179\\
0.824553225928284	0.82\\
0.83	0.829271822189567\\
0.830443735518318	0.83\\
0.836499286584667	0.84\\
0.84	0.846028120516499\\
0.842393717399716	0.85\\
0.848249938055361	0.86\\
0.85	0.863048245614035\\
0.854143929522348	0.87\\
0.859820733190321	0.88\\
0.86	0.880315818431912\\
0.865710496699821	0.89\\
0.87	0.897658906589428\\
0.871367835422286	0.9\\
0.877107410507657	0.91\\
0.88	0.915222759601707\\
0.882761870496202	0.92\\
0.888346879521651	0.93\\
0.89	0.933018159552135\\
0.89399448693901	0.94\\
0.899439609519914	0.95\\
0.9	0.951034482758621\\
0.905076746742584	0.96\\
0.91	0.969208875091308\\
0.910444268969266	0.97\\
0.916018352999533	0.98\\
0.92	0.987529004329004\\
0.921373780322707	0.99\\
0.926827845666969	1\\
0.93	1.00606349036403\\
0.932166992465693	1.01\\
0.93751276068118	1.02\\
0.94	1.02480565770863\\
0.942831615491228	1.03\\
0.94807975981501	1.04\\
0.95	1.04375\\
0.953374416758703	1.05\\
0.958534736931363	1.06\\
0.96	1.0628920502092\\
0.963801333271176	1.07\\
0.968882904979343	1.08\\
0.97	1.08222827897294\\
0.974117562710836	1.09\\
0.979128867084645	1.1\\
0.98	1.10175601659751\\
0.984327636749771	1.11\\
0.989276674317949	1.12\\
0.99	1.12147339544514\\
0.994435479383451	1.13\\
0.999329872122848	1.14\\
1	1.14137931034483\\
1.00444445234569	1.15\\
1.00929153690316	1.16\\
1.01	1.16147339544513\\
1.01435738910393	1.17\\
1.01916430387454	1.18\\
1.02	1.18175601659751\\
1.02417661846243	1.19\\
1.02895038695073	1.2\\
1.03	1.20222827897294\\
1.03390397838686	1.21\\
1.03865159114001	1.22\\
1.04	1.22289205020921\\
1.04354082027891	1.23\\
1.04826931765369	1.24\\
1.05	1.24375\\
1.05308800355045	1.25\\
1.05780456166028	1.26\\
1.06	1.26480565770863\\
1.06254587994697	1.27\\
1.06725790233943	1.28\\
1.07	1.28606349036403\\
1.07191426662386	1.29\\
1.07662948458104	1.3\\
1.08	1.307529004329\\
1.08119240645156	1.31\\
1.08591899131645	1.32\\
1.09	1.32920887509131\\
1.09037891337698	1.33\\
1.09512560503428	1.34\\
1.09954494502177	1.35\\
1.1	1.35103448275862\\
1.10424795648785	1.36\\
1.10868559516898	1.37\\
1.11	1.37301815955213\\
1.11328405789946	1.38\\
1.11774819266948	1.39\\
1.12	1.39522275960171\\
1.12223121704107	1.4\\
1.12673084263803	1.41\\
1.13	1.41765890658943\\
1.13108592733166	1.42\\
1.13563101842099	1.43\\
1.13986617024532	1.44\\
1.14	1.44031581843191\\
1.14444546173505	1.45\\
1.14872109302199	1.46\\
1.15	1.46304824561403\\
1.15317005579298	1.47\\
1.1574960257096	1.48\\
1.16	1.4860281205165\\
1.16179966358694	1.49\\
1.16618713883203	1.5\\
1.17	1.50927182218957\\
1.17032792065448	1.51\\
1.17478967526038	1.52\\
1.17893729642277	1.53\\
1.18	1.53260252454418\\
1.18329776242468	1.54\\
1.18751178409529	1.55\\
1.19	1.55615475161987\\
1.19170417072817	1.56\\
1.19599775281232	1.57\\
1.2	1.58\\
1.2	1.58\\
1.20438849367225	1.59\\
1.20846239663535	1.6\\
1.21	1.60386866572037\\
1.2126757675481	1.61\\
1.21683514204443	1.62\\
1.22	1.62804333821376\\
1.22084943152475	1.63\\
1.22511054598527	1.64\\
1.22907249266136	1.65\\
1.23	1.6523736453202\\
1.23327916001604	1.66\\
1.2373334271818	1.67\\
1.24	1.67689315866084\\
1.24132932451643	1.68\\
1.24549346796696	1.69\\
1.2493681345715	1.7\\
1.25	1.70164473684211\\
1.25354168742721	1.71\\
1.25751679124876	1.72\\
1.26	1.72652925764192\\
1.26146454045007	1.73\\
1.26555936936744	1.74\\
1.26936926113219	1.75\\
1.27	1.75166991555243\\
1.27348309878841	1.76\\
1.27740320384297	1.77\\
1.28	1.77694524158126\\
1.28127204968944	1.78\\
1.28532394784268	1.79\\
1.2890905089771	1.8\\
1.29	1.80244921930447\\
1.29311638724569	1.81\\
1.29700509542174	1.82\\
1.3	1.82814814814815\\
1.30076152298505	1.83\\
1.30479698185189	1.84\\
1.30854162861262	1.85\\
1.31	1.85399492440605\\
1.31244804789822	1.86\\
1.31632959832163	1.87\\
1.31994630360582	1.88\\
1.32	1.88014810659186\\
1.32398234014358	1.89\\
1.32772753458879	1.9\\
1.33	1.90633280675974\\
1.33147776782608	1.91\\
1.33537836558579	1.92\\
1.33900210281419	1.93\\
1.34	1.9328\\
1.34287747756271	1.94\\
1.3466479729469	1.95\\
1.35	1.95950471698113\\
1.3501974548261	1.96\\
1.35414690871863	1.97\\
1.35780129321997	1.98\\
1.36	1.98627710487445\\
1.36147229300991	1.99\\
1.36529672484801	2\\
4	1084\\
1.48117491872714	2\\
1.48	1.9963054247697\\
1.47776965791272	1.99\\
1.47435301395816	1.98\\
1.47110251178614	1.97\\
1.47	1.9665589841756\\
1.46766534783892	1.96\\
1.46423128975913	1.95\\
1.4609634747624	1.94\\
1.46	1.93701566632757\\
1.45748703461959	1.93\\
1.45403895215199	1.92\\
1.45075671406133	1.91\\
1.45	1.90767405063291\\
1.44723382936571	1.9\\
1.44377507846977	1.89\\
1.4404812828486	1.88\\
1.44	1.87853212487412\\
1.43690504522852	1.87\\
1.43343891634379	1.86\\
1.43013637847735	1.85\\
1.43	1.84958731865933\\
1.42650018946225	1.84\\
1.42302987749576	1.83\\
1.42	1.8208802507837\\
1.419675948588	1.82\\
1.41601895233188	1.81\\
1.41254752891283	1.8\\
1.41	1.79239412221647\\
1.40911305075202	1.79\\
1.40546119321611	1.78\\
1.40199158167087	1.77\\
1.4	1.76410256410256\\
1.39847044956355	1.76\\
1.39482692431609	1.75\\
1.39136187772047	1.74\\
1.39	1.73600053272451\\
1.38774852584939	1.73\\
1.38411629242182	1.72\\
1.38065837498547	1.71\\
1.38	1.70808264794383\\
1.37694779253367	1.7\\
1.37332955921067	1.69\\
1.37	1.68036115164148\\
1.36986246893123	1.68\\
1.36606886944736	1.67\\
1.36246708055629	1.66\\
1.36	1.65291921891059\\
1.35888117176814	1.65\\
1.35511245770975	1.64\\
1.35152928531326	1.63\\
1.35	1.62564873417722\\
1.34782102151488	1.62\\
1.3440793142257	1.61\\
1.34051665401484	1.6\\
1.34	1.59854277168494\\
1.3366831411525	1.59\\
1.33297022678439	1.58\\
1.33	1.57167666322847\\
1.32934351730788	1.57\\
1.32546865431336	1.56\\
1.32178599017735	1.55\\
1.32	1.5450399189463\\
1.31801372591103	1.54\\
1.31417865899057	1.53\\
1.31052738367865	1.52\\
1.31	1.51854850820487\\
1.30660866168683	1.51\\
1.30281420347254	1.5\\
1.3	1.49230769230769\\
1.29907675168416	1.49\\
1.2951296440062	1.48\\
1.2913762646986	1.47\\
1.29	1.46627275238574\\
1.28747370603555	1.46\\
1.28357787765894	1.45\\
1.28	1.44037890382627\\
1.27984610134252	1.44\\
1.27579936510468	1.43\\
1.27195443487003	1.42\\
1.27	1.41479386053562\\
1.26803979482729	1.41\\
1.26405495967366	1.4\\
1.26026024046803	1.39\\
1.26	1.38931332675222\\
1.2561658560672	1.38\\
1.25224152714822	1.37\\
1.25	1.36412974683544\\
1.24828532956885	1.36\\
1.2442253850326	1.35\\
1.24035990328741	1.34\\
1.24	1.33906653504442\\
1.23620962924368	1.33\\
1.2322192345639	1.32\\
1.23	1.31428855482567\\
1.22819171500557	1.31\\
1.2240715815265	1.3\\
1.22014801024419	1.29\\
1.22	1.28962281219272\\
1.21591414033348	1.28\\
1.21187169642635	1.27\\
1.21	1.2652682320442\\
1.20774375901422	1.26\\
1.20357880163316	1.25\\
1.2	1.24102564102564\\
1.19955692367678	1.24\\
1.19526604311688	1.23\\
1.19118571203922	1.22\\
1.19	1.21705671307807\\
1.18692982275852	1.21\\
1.18273525099167	1.2\\
1.18	1.19326686930091\\
1.17856622441096	1.19\\
1.17425520968142	1.18\\
1.17015076165889	1.17\\
1.17	1.16963268759196\\
1.16574170853727	1.16\\
1.16153200183678	1.15\\
1.16	1.14629950149551\\
1.15719060794054	1.14\\
1.15287451721327	1.13\\
1.15	1.12311708860759\\
1.1485975197885	1.12\\
1.14417430164695	1.11\\
1.14	1.10009290853032\\
1.13995782377195	1.1\\
1.13542715354664	1.09\\
1.13111942667602	1.08\\
1.13	1.07737149578999\\
1.12662869439647	1.07\\
1.122225007409	1.06\\
1.12	1.05482286860582\\
1.11777439119643	1.05\\
1.11327531511507	1.04\\
1.11	1.03244903602232\\
1.10885958270184	1.03\\
1.10426611102442	1.02\\
1.1	1.01025641025641\\
1.09987950921694	1.01\\
1.09519307253081	1\\
1.09073608860223	0.99\\
1.09	0.988337296898079\\
1.08605182072013	0.98\\
1.08151253100154	0.97\\
1.08	0.966616087388282\\
1.07683794991665	0.96\\
1.07221833494206	0.95\\
1.07	0.945085067533766\\
1.06754705871707	0.94\\
1.06284946723969	0.93\\
1.06	0.923748640483384\\
1.05817478189345	0.92\\
1.05340192151017	0.91\\
1.05	0.90261075949367\\
1.04871682247293	0.9\\
1.04387174455835	0.89\\
1.04	0.88167487283825\\
1.03916898324868	0.88\\
1.03425506220956	0.87\\
1.03	0.860943874425727\\
1.02952719694694	0.86\\
1.02454810396458	0.85\\
1.02	0.840420061412487\\
1.01978755427292	0.84\\
1.0147472258596	0.83\\
1.01	0.820105099948744\\
1.00994632908477	0.82\\
1.00484893093506	0.81\\
1	0.8\\
1	0.8\\
0.994849886760315	0.79\\
0.99	0.780105099948745\\
0.98994526782199	0.78\\
0.98474693952592	0.77\\
0.98	0.760420061412487\\
0.979779068280767	0.76\\
0.974537124297506	0.75\\
0.97	0.740943874425728\\
0.969498579709163	0.74\\
0.964217671122564	0.73\\
0.96	0.72167487283825\\
0.959101225415603	0.72\\
0.953786006789194	0.71\\
0.95	0.702610759493671\\
0.948584670662299	0.7\\
0.943239752149253	0.69\\
0.94	0.683748640483384\\
0.937946814305118	0.68\\
0.932576715032225	0.67\\
0.93	0.665085067533767\\
0.927185775292253	0.66\\
0.921794878885443	0.65\\
0.92	0.646616087388282\\
0.916299874346568	0.64\\
0.910892387376377	0.63\\
0.91	0.62833729689808\\
0.905287611265952	0.62\\
0.9	0.61025641025641\\
0.899853342113252	0.61\\
0.894147638363067	0.6\\
0.89	0.592449036022324\\
0.888584037958067	0.59\\
0.882878730627924	0.58\\
0.88	0.574822868605817\\
0.877180960756293	0.57\\
0.871479753232721	0.56\\
0.87	0.557371495789995\\
0.865643422198479	0.55\\
0.86	0.540092908530318\\
0.859944411221677	0.54\\
0.85397076929593	0.53\\
0.85	0.523117088607595\\
0.848113862212752	0.52\\
0.842162341662916	0.51\\
0.84	0.506299501495513\\
0.836144466564457	0.5\\
0.830217428855519	0.49\\
0.83	0.489632687591957\\
0.824035922234284	0.48\\
0.82	0.473266869300912\\
0.817951314817609	0.47\\
0.81178778382705	0.46\\
0.81	0.457056713078071\\
0.805522063800348	0.45\\
0.8	0.441025641025641\\
0.799340607531251	0.44\\
0.792951337601168	0.43\\
0.79	0.425268232044199\\
0.786571785758059	0.42\\
0.78023851086154	0.41\\
0.78	0.409622812192723\\
0.773660785567962	0.4\\
0.77	0.394288554825669\\
0.767136453737991	0.39\\
0.760606771313757	0.38\\
0.76	0.379066535044422\\
0.753869413651375	0.37\\
0.75	0.364129746835443\\
0.747168701151959	0.36\\
0.7404591805581	0.35\\
0.74	0.349313326752221\\
0.733531824453452	0.34\\
0.73	0.334793860535624\\
0.726623134627193	0.33\\
0.72	0.320378903826267\\
0.719729231906147	0.32\\
0.712604413248519	0.31\\
0.71	0.306272752385736\\
0.705456718220528	0.3\\
0.7	0.292307692307693\\
0.698303739876625	0.29\\
0.691045508516617	0.28\\
0.69	0.278548508204873\\
0.683628755941406	0.27\\
0.68	0.265039918946302\\
0.676186409330049	0.26\\
0.67	0.251676663228468\\
0.668711651822424	0.25\\
0.661100426324854	0.24\\
0.66	0.238542771684945\\
0.653339901401253	0.23\\
0.65	0.225648734177215\\
0.64552594799773	0.22\\
0.64	0.212919218910586\\
0.637650401712318	0.21\\
0.63	0.20036115164148\\
0.629704535038636	0.2\\
0.621547462686567	0.19\\
0.62	0.188082647943832\\
0.613284596714549	0.18\\
0.61	0.176000532724505\\
0.60492989897994	0.17\\
0.6	0.164102564102564\\
0.596473743716101	0.16\\
0.59	0.152394122216468\\
0.587906061395264	0.15\\
0.58	0.140880250783699\\
0.579216362980965	0.14\\
0.570364220355633	0.13\\
0.57	0.129587318659329\\
0.561320274906308	0.12\\
0.56	0.118532124874119\\
0.55213280374027	0.11\\
0.55	0.107674050632911\\
0.542790788381545	0.1\\
0.54	0.0970156663275687\\
0.533282863578255	0.0899999999999999\\
0.53	0.0865589841755996\\
0.523597324131402	0.0800000000000001\\
0.52	0.0763054247697032\\
0.513722125269098	0.0699999999999998\\
0.51	0.0662557919015887\\
0.503644874655364	0.0600000000000001\\
0.5	0.0564102564102564\\
0.493352814019713	0.0499999999999998\\
0.49	0.0467683495643262\\
0.482832788308886	0.04\\
0.48	0.037328966223132\\
0.472071200188243	0.0299999999999998\\
0.47	0.028090377743747\\
0.461053947646934	0.02\\
0.46	0.0190502543234994\\
0.45	0.0102166666666668\\
0.449748940916914	0.00999999999999979\\
0.44	0.00163563096500527\\
0.438059843708234	0\\
0.43	-0.00674826224328579\\
0.426049295436733	-0.01\\
0.42	-0.0149400208986416\\
0.41369961129995	-0.02\\
0.41	-0.0229450802692904\\
0.400991863413796	-0.03\\
0.4	-0.0307692307692308\\
0.39	-0.0383340192410475\\
0.387754327191513	-0.04\\
0.38	-0.0456827877507921\\
0.374020074961148	-0.05\\
0.37	-0.052864486711829\\
0.36	-0.0598803900325028\\
0.359824267482719	-0.0600000000000001\\
0.35	-0.0665833333333333\\
0.34482927006445	-0.0700000000000001\\
0.34	-0.0731391395592866\\
0.33	-0.0795317835780315\\
0.329244179156883	-0.0800000000000001\\
0.32	-0.085619637139808\\
0.312695334920246	-0.0900000000000001\\
0.31	-0.0915850078492935\\
0.3	-0.0972972972972974\\
0.295160174040141	-0.1\\
0.29	-0.102819116869381\\
0.28	-0.108150054525627\\
0.276423266478665	-0.11\\
0.27	-0.113242655667908\\
0.26	-0.118166046002191\\
0.256147505622631	-0.12\\
0.25	-0.12285\\
0.24	-0.127344578313253\\
0.233894544653257	-0.13\\
0.23	-0.131646061734965\\
0.22	-0.135696401308615\\
0.21	-0.139626437744277\\
0.208996783649845	-0.14\\
0.2	-0.143243243243243\\
0.19	-0.146702898950856\\
0.18005198050731	-0.15\\
0.18	-0.150016648879402\\
0.17	-0.153006498096792\\
0.16	-0.155837873754153\\
0.15	-0.158503521126761\\
0.143956818983267	-0.16\\
0.14	-0.160942145178765\\
0.13	-0.163126415612974\\
0.12	-0.165129988851728\\
0.11	-0.166946894409938\\
0.1	-0.168571428571429\\
0.0899999999999999	-0.169998238012709\\
0.089985464077208	-0.17\\
0.0800000000000001	-0.171155016538037\\
0.0699999999999998	-0.172115036131184\\
0.0600000000000001	-0.172875251959686\\
0.0499999999999998	-0.173433098591549\\
0.04	-0.173786636466591\\
0.0299999999999998	-0.173934593519045\\
0.02	-0.173876396807298\\
0.00999999999999979	-0.173612193032553\\
0	-0.173142857142857\\
-0.01	-0.172469988577955\\
-0.02	-0.171595895096921\\
-0.03	-0.170523564525298\\
-0.0341228875209855	-0.17\\
-0.04	-0.169212004801921\\
-0.05	-0.167667910447761\\
-0.0600000000000001	-0.165928351126928\\
-0.0700000000000001	-0.163999381989406\\
-0.0800000000000001	-0.161887514585764\\
-0.0882357283777766	-0.16\\
-0.0900000000000001	-0.159575076640098\\
-0.1	-0.156969696969697\\
-0.11	-0.154196858168761\\
-0.12	-0.151265171192444\\
-0.124069689057679	-0.15\\
-0.13	-0.148071000617665\\
-0.14	-0.144653219927096\\
-0.15	-0.141100746268657\\
-0.152954467154338	-0.14\\
-0.16	-0.137262266500623\\
-0.17	-0.133239139719341\\
-0.177809863557795	-0.13\\
-0.18	-0.129054129606099\\
-0.19	-0.124567939168219\\
-0.2	-0.12\\
-0.2	-0.12\\
-0.21	-0.115066593337524\\
-0.22	-0.110077845777234\\
-0.220150456870804	-0.11\\
-0.23	-0.104726377454085\\
-0.23874726755941	-0.1\\
-0.24	-0.0992996068152032\\
-0.25	-0.0935515873015873\\
-0.256105014433755	-0.0900000000000001\\
-0.26	-0.0876613368283093\\
-0.27	-0.0815598163394553\\
-0.272504079505884	-0.0800000000000001\\
-0.28	-0.0751937418513689\\
-0.288042599369085	-0.0700000000000001\\
-0.29	-0.0686994969818913\\
-0.3	-0.0619354838709677\\
-0.302819270654682	-0.0600000000000001\\
-0.31	-0.0549430509596294\\
-0.316968766765412	-0.05\\
-0.32	-0.0477959294436906\\
-0.33	-0.0404541585445094\\
-0.330605899052944	-0.04\\
-0.34	-0.0328042496679946\\
-0.343631470103917	-0.03\\
-0.35	-0.024978813559322\\
-0.356263566618761	-0.02\\
-0.36	-0.0169704011065006\\
-0.368525619065745	-0.01\\
-0.37	-0.008771634954193\\
-0.38	-0.000350200803212936\\
-0.380408742016757	0\\
-0.39	0.00834605710401079\\
-0.391881202664512	0.00999999999999979\\
-0.4	0.0172413793103447\\
-0.40305972974937	0.02\\
-0.41	0.0263403563941299\\
-0.413960859946679	0.0299999999999998\\
-0.42	0.0356469589816124\\
-0.424600213000504	0.04\\
-0.43	0.0451644388849177\\
-0.434992675141608	0.0499999999999998\\
-0.44	0.0548952380952381\\
-0.445152551260064	0.0600000000000001\\
-0.45	0.0648409090909091\\
-0.455093687955739	0.0699999999999998\\
-0.46	0.0750020497803806\\
-0.464829569369529	0.0800000000000001\\
-0.47	0.0853782560706401\\
-0.474373387644263	0.0899999999999999\\
-0.48	0.095968094534712\\
-0.483738089939572	0.1\\
-0.49	0.106769096965211\\
-0.492936404091504	0.11\\
-0.5	0.117777777777778\\
-0.501980845226658	0.12\\
-0.51	0.128989674315322\\
-0.510883705873784	0.13\\
-0.519624854759551	0.14\\
-0.52	0.140431259968102\\
-0.528152599663683	0.15\\
-0.53	0.152159690230342\\
-0.536561495637723	0.16\\
-0.54	0.164085939968404\\
-0.54486388691294	0.17\\
-0.55	0.176200980392157\\
-0.553071778952409	0.18\\
-0.56	0.188495178849145\\
-0.561196765665342	0.19\\
-0.569172187962301	0.2\\
-0.57	0.201038407005838\\
-0.57695477273095	0.21\\
-0.58	0.213875453047776\\
-0.584679684672636	0.22\\
-0.59	0.226866815597076\\
-0.592357988500301	0.23\\
-0.6	0.24\\
-0.6	0.24\\
-0.607349628204678	0.25\\
-0.61	0.25354146029035\\
-0.614680115719612	0.26\\
-0.62	0.2672\\
-0.622001167227834	0.27\\
-0.62924105135463	0.28\\
-0.63	0.281048302055406\\
-0.636253656100264	0.29\\
-0.64	0.29523630017452\\
-0.643285616700119	0.3\\
-0.65	0.309494680851064\\
-0.650344191924315	0.31\\
-0.657116285197186	0.32\\
-0.66	0.324156238698011\\
-0.66389484309324	0.33\\
-0.67	0.33889468832309\\
-0.670728232418909	0.34\\
-0.677309030446581	0.35\\
-0.68	0.353993296089385\\
-0.683874071071477	0.36\\
-0.69	0.369181863186319\\
-0.690522116497549	0.37\\
-0.696878696793485	0.38\\
-0.7	0.384761904761905\\
-0.703273994660412	0.39\\
-0.709747190372375	0.4\\
-0.71	0.400392583249243\\
-0.715879397915863	0.41\\
-0.72	0.416453384912959\\
-0.722151576044707	0.42\\
-0.728341023133527	0.43\\
-0.73	0.432642543411644\\
-0.734372911196442	0.44\\
-0.74	0.449033918128655\\
-0.740569738668598	0.45\\
-0.746444276109622	0.46\\
-0.75	0.465801282051282\\
-0.75242771704161	0.47\\
-0.758365874073037	0.48\\
-0.76	0.482709287257019\\
-0.76413214389546	0.49\\
-0.77	0.499792696629213\\
-0.770116800713638	0.5\\
-0.775682479503341	0.51\\
-0.78	0.517315631691649\\
-0.781483879559074	0.52\\
-0.787077530380882	0.53\\
-0.79	0.535026992143659\\
-0.792696709563354	0.54\\
-0.798315471718455	0.55\\
-0.8	0.552941176470588\\
-0.803754446672778	0.56\\
-0.809393885624224	0.57\\
-0.81	0.571074044389643\\
-0.814655947001978	0.58\\
-0.82	0.589506636155607\\
-0.82025693813404	0.59\\
-0.825399857919345	0.6\\
-0.83	0.60829684147795\\
-0.830871005592551	0.61\\
-0.835984731011316	0.62\\
-0.84	0.62734094292804\\
-0.841335372863244	0.63\\
-0.846409159125175	0.64\\
-0.85	0.646653225806452\\
-0.851650540422729	0.65\\
-0.856671938762759	0.66\\
-0.86	0.666247721179625\\
-0.861817396808908	0.67\\
-0.866772257675218	0.68\\
-0.87	0.68613776077886\\
-0.871837366636341	0.69\\
-0.876709905528014	0.7\\
-0.88	0.706335446685879\\
-0.881712563761553	0.71\\
-0.886485502952925	0.72\\
-0.89	0.726851043219076\\
-0.891445943495324	0.73\\
-0.896100741249089	0.74\\
-0.9	0.747692307692308\\
-0.901041446082601	0.75\\
-0.905558621620656	0.76\\
-0.91	0.768863787638669\\
-0.910504122207183	0.77\\
-0.91486367946416	0.78\\
-0.919787156484484	0.79\\
-0.92	0.790437354085603\\
-0.924022176305722	0.8\\
-0.928731901352426	0.81\\
-0.93	0.812635370741483\\
-0.933042240112585	0.82\\
-0.93750748059453	0.83\\
-0.94	0.83523621399177\\
-0.941933934485797	0.84\\
-0.946125309355301	0.85\\
-0.95	0.858223684210526\\
-0.950709239233223	0.86\\
-0.954599936915151	0.87\\
-0.959092681900069	0.88\\
-0.96	0.88200218579235\\
-0.962948948495152	0.89\\
-0.96698727528529	0.9\\
-0.97	0.906710445682451\\
-0.971192667962734	0.91\\
-0.974742289433144	0.92\\
-0.978932311409172	0.93\\
-0.98	0.932522834645669\\
-0.982390659911257	0.94\\
-0.985756705862661	0.95\\
-0.98993744146844	0.96\\
-0.99	0.960165894039735\\
-0.992481140193318	0.97\\
-0.994993679520395	0.98\\
-0.997506172223746	0.99\\
-1	1\\
-0.99	1.00006004016064\\
-0.989961426820615	1\\
-0.9821031090122	0.99\\
-0.98	0.988370731707317\\
-0.97450623624358	0.98\\
-0.97	0.975046774193548\\
-0.966878802992519	0.97\\
-0.96	0.961260829493088\\
-0.959240363992254	0.96\\
-0.952656585024889	0.95\\
-0.95	0.946691176470588\\
-0.945895605862698	0.94\\
-0.94	0.932052918287938\\
-0.93875716262772	0.93\\
-0.932152562214418	0.92\\
-0.93	0.917155588822355\\
-0.925590331806021	0.91\\
-0.92	0.902244368600682\\
-0.918625453975387	0.9\\
-0.912019901140439	0.89\\
-0.91	0.887252987697715\\
-0.905478281353574	0.88\\
-0.9	0.872307692307692\\
-0.89856418797651	0.87\\
-0.891914104689242	0.86\\
-0.89	0.857365580286168\\
-0.88533427326972	0.85\\
-0.88	0.842497450424929\\
-0.878416702211247	0.84\\
-0.871665255027301	0.83\\
-0.87	0.827704772393539\\
-0.865025526771112	0.82\\
-0.86	0.81296976127321\\
-0.858077188945039	0.81\\
-0.851163762084254	0.8\\
-0.85	0.798405172413793\\
-0.844455464012265	0.79\\
-0.84	0.78383274559194\\
-0.837460667125602	0.78\\
-0.830320963715248	0.77\\
-0.83	0.769564980289093\\
-0.823540014643562	0.76\\
-0.82	0.755170944309927\\
-0.816488149807807	0.75\\
-0.81	0.741123678290214\\
-0.809231514853123	0.74\\
-0.802195110172579	0.73\\
-0.8	0.727058823529412\\
-0.795077549111703	0.72\\
-0.79	0.713191364136414\\
-0.787757069682033	0.71\\
-0.780328060719393	0.7\\
-0.78	0.699569515011547\\
-0.773136468103609	0.69\\
-0.77	0.685888110749186\\
-0.765740609107153	0.68\\
-0.76	0.672496098562628\\
-0.758177162800985	0.67\\
-0.750554853328486	0.66\\
-0.75	0.65929054054054\\
-0.743070981370766	0.65\\
-0.74	0.646089527720739\\
-0.73541221743082	0.64\\
-0.73	0.633146865817826\\
-0.727603304552151	0.63\\
-0.72	0.620438273921201\\
-0.719662336254404	0.62\\
-0.711837017632242	0.61\\
-0.71	0.607741278493558\\
-0.70390291118145	0.6\\
-0.7	0.595238095238095\\
-0.695824174989845	0.59\\
-0.69	0.582955876951331\\
-0.687613426998573	0.58\\
-0.68	0.570880284191829\\
-0.679279768933132	0.57\\
-0.670935406554009	0.56\\
-0.67	0.558905702167766\\
-0.662539762745988	0.55\\
-0.66	0.547059597806216\\
-0.654003470645097	0.54\\
-0.65	0.535416666666667\\
-0.645332087279914	0.53\\
-0.64	0.523968110918544\\
-0.636528871117512	0.52\\
-0.63	0.512706985605419\\
-0.62759514809591	0.51\\
-0.62	0.501627860696518\\
-0.618530566419042	0.5\\
-0.61	0.490726566314077\\
-0.609333268622671	0.49\\
-0.6	0.48\\
-0.6	0.48\\
-0.590576126460188	0.47\\
-0.59	0.469398588537211\\
-0.580988209239855	0.46\\
-0.58	0.458984148397976\\
-0.571230468508879	0.45\\
-0.57	0.448755162364696\\
-0.561295391047	0.44\\
-0.56	0.438711037891269\\
-0.551173689328087	0.43\\
-0.55	0.428852040816327\\
-0.540854211385268	0.42\\
-0.54	0.41917925445705\\
-0.530323801388147	0.41\\
-0.53	0.409694560838034\\
-0.52	0.40037087667162\\
-0.519596255332915	0.4\\
-0.51	0.391204558932543\\
-0.50865982952507	0.39\\
-0.5	0.382222222222222\\
-0.497471565648691	0.38\\
-0.49	0.37342842846553\\
-0.486008070258723	0.37\\
-0.48	0.36482852897474\\
-0.474242508803402	0.36\\
-0.47	0.356428747203579\\
-0.462144025515435	0.35\\
-0.46	0.34823628185907\\
-0.45	0.340241228070175\\
-0.449692766265512	0.34\\
-0.44	0.332330410183875\\
-0.436951781873685	0.33\\
-0.43	0.324635938615275\\
-0.423766693058854	0.32\\
-0.42	0.317168253968254\\
-0.410082898709855	0.31\\
-0.41	0.309939335281227\\
-0.4	0.302758620689655\\
-0.395990798883437	0.3\\
-0.39	0.295817319804059\\
-0.381276843568824	0.29\\
-0.38	0.289134566145092\\
-0.37	0.282543517893315\\
-0.365970206734241	0.28\\
-0.36	0.276161760660248\\
-0.35	0.270061475409836\\
-0.349896643533193	0.27\\
-0.34	0.263991432396252\\
-0.333037101906179	0.26\\
-0.33	0.258219678302533\\
-0.32	0.252589252948886\\
-0.315171010418895	0.25\\
-0.31	0.247158462055071\\
-0.3	0.241935483870968\\
-0.296118539794671	0.24\\
-0.29	0.236866103379722\\
-0.28	0.232022477650064\\
-0.275609800048515	0.23\\
-0.27	0.227339217619987\\
-0.26	0.222853367217281\\
-0.253243725037421	0.22\\
-0.25	0.21858606557377\\
-0.24	0.2144416772554\\
-0.23	0.210587877853177\\
-0.22839725854602	0.21\\
-0.22	0.206811749680715\\
-0.21	0.203275916718459\\
-0.200000000000001	0.2\\
-0.2	0.2\\
-0.19	0.196777186910006\\
-0.18	0.193807626076261\\
-0.17	0.191077031908489\\
-0.165746418593592	0.19\\
-0.16	0.188483814303639\\
-0.15	0.186057692307692\\
-0.14	0.183856831922612\\
-0.13	0.181872070196312\\
-0.12	0.180095662368113\\
-0.119400055141992	0.18\\
-0.11	0.1784302946593\\
-0.1	0.176969696969697\\
-0.0900000000000001	0.175714170161774\\
-0.0800000000000001	0.174659074733096\\
-0.0700000000000001	0.173800676072898\\
-0.0600000000000001	0.173136056009335\\
-0.05	0.172663043478261\\
-0.04	0.172380161476355\\
-0.03	0.17228658816772\\
-0.02	0.172382130584192\\
-0.01	0.172667209834191\\
0	0.173142857142857\\
0.00999999999999979	0.173810720411664\\
0.02	0.174673081328751\\
0.0299999999999998	0.175732883400345\\
0.04	0.176993771626298\\
0.0499999999999998	0.178460144927536\\
0.0591865672932275	0.18\\
0.0600000000000001	0.180129658213892\\
0.0699999999999998	0.181918350916158\\
0.0800000000000001	0.183916685330347\\
0.0899999999999999	0.186131684567552\\
0.1	0.188571428571429\\
0.105372935449553	0.19\\
0.11	0.191177118644068\\
0.12	0.193933111849391\\
0.13	0.196926979225154\\
0.139481657932349	0.2\\
0.14	0.200161380798274\\
0.15	0.20347602739726\\
0.16	0.207047491638796\\
0.167714816298292	0.21\\
0.17	0.21084306824288\\
0.18	0.21474874041621\\
0.19	0.21893736724427\\
0.192406072610603	0.22\\
0.2	0.223243243243243\\
0.21	0.227789358761747\\
0.214603885451144	0.23\\
0.22	0.232511897106109\\
0.23	0.237431603514552\\
0.234950175958511	0.24\\
0.24	0.242545997865528\\
0.25	0.247859589041096\\
0.253838142167242	0.25\\
0.26	0.253346424759872\\
0.27	0.259079049972543\\
0.271543640840371	0.26\\
0.28	0.26492347266881\\
0.288256062751934	0.27\\
0.29	0.271048166579361\\
0.3	0.277297297297297\\
0.304132590018142	0.28\\
0.31	0.283758046585495\\
0.319282646775178	0.29\\
0.32	0.290473104880581\\
0.33	0.29729121440086\\
0.333806217792627	0.3\\
0.34	0.304331151003168\\
0.347763104142641	0.31\\
0.35	0.311607142857143\\
0.36	0.319061057173678\\
0.361220094268542	0.32\\
0.37	0.326659728867624\\
0.374237787707613	0.33\\
0.38	0.334482266526757\\
0.386838082543604	0.34\\
0.39	0.342520969414204\\
0.399058206008018	0.35\\
0.4	0.350769230769231\\
0.41	0.359179748528625\\
0.410948185486988	0.36\\
0.42	0.367759915164369\\
0.42253441113394	0.37\\
0.43	0.376546528143083\\
0.433825730977113	0.38\\
0.44	0.385535632183908\\
0.444844389136295	0.39\\
0.45	0.394724025974026\\
0.455610118513322	0.4\\
0.46	0.404109203722854\\
0.466140429832244	0.41\\
0.47	0.413689309634209\\
0.476450857767303	0.42\\
0.48	0.423463103802672\\
0.486555170503588	0.43\\
0.49	0.433429938429964\\
0.496465547928432	0.44\\
0.5	0.443589743589744\\
0.506192732704317	0.45\\
0.51	0.453943022062596\\
0.515746157684023	0.46\\
0.52	0.46449085303186\\
0.525134052462078	0.47\\
0.53	0.475234904688305\\
0.534363531285015	0.48\\
0.54	0.486177456049638\\
0.54344066404332	0.49\\
0.55	0.497321428571428\\
0.552370531619731	0.5\\
0.56	0.508670428422153\\
0.561157266453688	0.51\\
0.569811870803101	0.52\\
0.57	0.520217366316841\\
0.578383229301613	0.53\\
0.58	0.531900906344411\\
0.586829556190092	0.54\\
0.59	0.543792254951752\\
0.595152393979695	0.55\\
0.6	0.555897435897436\\
0.603352477376016	0.56\\
0.61	0.568223405909798\\
0.61142970974437	0.57\\
0.619411434410373	0.58\\
0.62	0.580739381854437\\
0.627344423029404	0.59\\
0.63	0.593390686521958\\
0.635164452859214	0.6\\
0.64	0.606273899692938\\
0.642869331672877	0.61\\
0.65	0.61939935064935\\
0.650455944430972	0.62\\
0.658026505983438	0.63\\
0.66	0.632639518555667\\
0.665510188374647	0.64\\
0.67	0.646098393676696\\
0.672880422403975	0.65\\
0.68	0.659818899273105\\
0.68013188267433	0.66\\
0.687408507158199	0.67\\
0.69	0.67362440925088\\
0.694581130412322	0.68\\
0.7	0.687692307692308\\
0.701636307109391	0.69\\
0.708647242653735	0.7\\
0.71	0.701946714783474\\
0.71563607228391	0.71\\
0.72	0.716371922685656\\
0.722506628713002	0.72\\
0.729293463631576	0.73\\
0.73	0.731045497278575\\
0.736108038780556	0.74\\
0.74	0.745841540020264\\
0.742801394417892	0.75\\
0.749401235038995	0.76\\
0.75	0.76091049382716\\
0.756047237859939	0.77\\
0.76	0.776094832826747\\
0.762566883381754	0.78\\
0.76901369537494	0.79\\
0.77	0.791540301830777\\
0.775493160477762	0.8\\
0.78	0.80713489318413\\
0.781838756273718	0.81\\
0.788164515002135	0.82\\
0.79	0.822942234942758\\
0.794475844446093	0.83\\
0.8	0.838974358974359\\
0.800643108529388	0.84\\
0.806878851201277	0.85\\
0.81	0.855132704876823\\
0.813016631741308	0.86\\
0.819065542779619	0.87\\
0.82	0.87155538005923\\
0.825173862751472	0.88\\
0.83	0.888138169301377\\
0.831128460712271	0.89\\
0.837130517204641	0.9\\
0.84	0.9048962888666\\
0.843058801252158	0.91\\
0.848900546473096	0.92\\
0.85	0.921898148148148\\
0.854802000266351	0.93\\
0.86	0.939088638689867\\
0.860534650344139	0.94\\
0.866370500388866	0.95\\
0.87	0.956419459868753\\
0.872077864440294	0.96\\
0.87777541537497	0.97\\
0.88	0.973979661016949\\
0.883456464342189	0.98\\
0.889026702715219	0.99\\
0.89	0.991760941350419\\
0.894680411736132	1\\
0.9	1.00974358974359\\
0.900144271206785	1.01\\
0.905758638986809	1.02\\
0.91	1.02785523108177\\
0.911194360580628	1.03\\
0.916699172453665	1.04\\
0.92	1.04618086606244\\
0.922105149216967	1.05\\
0.927509236465394	1.06\\
0.93	1.06471511744128\\
0.932883822717325	1.07\\
0.938195341234164	1.08\\
0.94	1.08345342601787\\
0.943536832843552	1.09\\
0.948763357336435	1.1\\
0.95	1.10239197530864\\
0.954069972485995	1.11\\
0.959218578870797	1.12\\
0.96	1.12152763028515\\
0.964488438873515	1.13\\
0.969565776992467	1.14\\
0.97	1.14085788829005\\
0.974796886749404	1.15\\
0.979809245194103	1.16\\
0.98	1.16038084066471\\
0.984999472892044	1.17\\
0.989952837434908	1.18\\
0.99	1.18009514397267\\
0.995099893077154	1.19\\
1	1.2\\
1	1.2\\
1.00510141234467	1.21\\
1.0099537977886	1.22\\
1.01	1.22009514397267\\
1.01500688923565	1.23\\
1.01981693557387	1.24\\
1.02	1.24038084066471\\
1.02481879449338	1.25\\
1.02959177464198	1.26\\
1.03	1.26085788829005\\
1.03453922457047	1.27\\
1.03928034509713	1.28\\
1.04	1.28152763028515\\
1.04416991014251	1.29\\
1.04888435400786	1.3\\
1.05	1.30239197530864\\
1.05371221969353	1.31\\
1.05840518946286	1.32\\
1.06	1.32345342601787\\
1.06316715810208	1.33\\
1.06784392049772	1.34\\
1.07	1.34471511744128\\
1.07253536001343	1.35\\
1.07720129274229	1.36\\
1.08	1.36618086606244\\
1.08181707762612	1.37\\
1.08647771951699	1.38\\
1.09	1.38785523108177\\
1.09101216234192	1.39\\
1.09567326796932	1.4\\
1.1	1.40974358974359\\
1.10012003951701	1.41\\
1.10478763968224	1.42\\
1.10922450157847	1.43\\
1.11	1.43176094135042\\
1.1138201449946	1.44\\
1.11826415381521	1.45\\
1.12	1.45397966101695\\
1.12276967003985	1.46\\
1.12722662813837	1.47\\
1.13	1.47641945986875\\
1.13163463521712	1.48\\
1.1361107479373	1.49\\
1.14	1.49908863868987\\
1.14041294343941	1.5\\
1.14491490062106	1.51\\
1.14919425924781	1.52\\
1.15	1.52189814814815\\
1.15363698737183	1.53\\
1.15794109800936	1.54\\
1.16	1.5448962888666\\
1.16227436125219	1.55\\
1.16660994248631	1.56\\
1.17	1.56813816930138\\
1.17082375105054	1.57\\
1.17519817450412	1.58\\
1.17935718984747	1.59\\
1.18	1.59155538005923\\
1.18370258235902	1.6\\
1.18789843086598	1.61\\
1.19	1.61513270487682\\
1.19211927046335	1.62\\
1.19635955528327	1.63\\
1.2	1.63897435897436\\
1.20044354798468	1.64\\
1.20473674737032	1.65\\
1.20881570076306	1.66\\
1.21	1.66294223494276\\
1.2130254287467	1.67\\
1.21715450972437	1.68\\
1.22	1.68713489318413\\
1.22122012530935	1.69\\
1.22540828846619	1.7\\
1.22939040354837	1.71\\
1.23	1.71154030183078\\
1.23357168895876	1.72\\
1.23760992883146	1.73\\
1.24	1.73609483282675\\
1.24163833841099	1.74\\
1.24574229626856	1.75\\
1.24964565449896	1.76\\
1.25	1.76091049382716\\
1.25378126832856	1.77\\
1.25774726031478	1.78\\
1.26	1.78584154002026\\
1.26171941749098	1.79\\
1.26575846275493	1.8\\
1.26959983333886	1.81\\
1.27	1.81104549727858\\
1.27367197431228	1.82\\
1.27758321958936	1.83\\
1.28	1.83637192268566\\
1.28147910521268	1.84\\
1.28547169681309	1.85\\
1.28926711484856	1.86\\
1.29	1.86194671478347\\
1.29325671485264	1.87\\
1.29713023290073	1.88\\
1.3	1.88769230769231\\
1.30092804649307	1.89\\
1.30489246843624	1.9\\
1.30865777491461	1.91\\
1.31	1.91362440925088\\
1.31254372671244	1.92\\
1.31639664839065	1.93\\
1.32	1.9398188992731\\
1.32007187640062	1.94\\
1.3240269062239	1.95\\
1.32777829279507	1.96\\
1.33	1.9660983936767\\
1.33153653283261	1.97\\
1.33538672524624	1.98\\
1.33904538467468	1.99\\
1.34	1.99263951855567\\
1.34287664480293	2\\
8	615\\
-1.35799902318335	2\\
-1.35416767749384	1.99\\
-1.35004262667625	1.98\\
-1.35	1.97989754098361\\
-1.34631415206878	1.97\\
-1.34233054673809	1.96\\
-1.34	1.95440602258469\\
-1.33834570154931	1.95\\
-1.33447942338991	1.94\\
-1.33032630323932	1.93\\
-1.33	1.92922418321589\\
-1.3265074616005	1.92\\
-1.32248039349107	1.91\\
-1.32	1.90411512915129\\
-1.31842900997247	1.9\\
-1.31450701599635	1.89\\
-1.31030028996377	1.88\\
-1.31	1.87929449339207\\
-1.30642129252137	1.87\\
-1.30232805081359	1.86\\
-1.3	1.85454545454545\\
-1.2982352546704	1.85\\
-1.2942368961426	1.84\\
-1.29	1.83010588063195\\
-1.28995854880738	1.83\\
-1.28603956552165	1.82\\
-1.28185720684171	1.81\\
-1.28	1.80570660264106\\
-1.27774626786064	1.8\\
-1.27365013845348	1.79\\
-1.27	1.78159381173736\\
-1.26936523052179	1.78\\
-1.26534116649358	1.77\\
-1.26104550905266	1.76\\
-1.26	1.75761720430108\\
-1.25693903491452	1.75\\
-1.25272208023767	1.74\\
-1.25	1.73382246376812\\
-1.24845080945573	1.73\\
-1.2442994019597	1.72\\
-1.24	1.71028816234498\\
-1.23988220723186	1.71\\
-1.23578501401843	1.7\\
-1.23142155454291	1.69\\
-1.23	1.68682332945962\\
-1.2271850080605	1.68\\
-1.22288108593367	1.67\\
-1.22	1.66359411098528\\
-1.21850429762291	1.66\\
-1.21424909857835	1.65\\
-1.21	1.64060306799337\\
-1.20974682618972	1.64\\
-1.20553084428953	1.63\\
-1.20104517917637	1.62\\
-1.2	1.61771428571429\\
-1.19673052634893	1.61\\
-1.19228812589892	1.6\\
-1.19	1.59502456679709\\
-1.18785147384838	1.59\\
-1.18344301359971	1.58\\
-1.18	1.57255815991238\\
-1.17889627796254	1.57\\
-1.17451339366686	1.56\\
-1.17	1.55030572272972\\
-1.16986689904844	1.55\\
-1.16550201344773	1.54\\
-1.16086562559487	1.53\\
-1.16	1.52816231884058\\
-1.15641093201011	1.52\\
-1.1517943228599	1.51\\
-1.15	1.50621575342466\\
-1.14724161115441	1.5\\
-1.14263703326033	1.49\\
-1.14	1.48447637540453\\
-1.13799498698428	1.48\\
-1.13339526739899	1.47\\
-1.13	1.46293830409357\\
-1.12867152600427	1.46\\
-1.12406995521393	1.45\\
-1.12	1.44159664218258\\
-1.11927126873043	1.44\\
-1.11466150606181	1.43\\
-1.11	1.42044735614308\\
-1.10979386306985	1.42\\
-1.10516985495405	1.41\\
-1.10026707297368	1.4\\
-1.1	1.39945945945946\\
-1.09559449740138	1.39\\
-1.09067488520118	1.38\\
-1.09	1.37864470304976\\
-1.0859345147106	1.37\\
-1.08099138315357	1.36\\
-1.08	1.35802502651113\\
-1.07618859110683	1.35\\
-1.07121539393375	1.34\\
-1.07	1.33759860599684\\
-1.06635502368801	1.33\\
-1.06134528896859	1.32\\
-1.06	1.31736426332288\\
-1.05643172592399	1.31\\
-1.05137899172137	1.3\\
-1.05	1.29732142857143\\
-1.04641622517104	1.29\\
-1.0413139777987	1.28\\
-1.04	1.27747011375388\\
-1.036305654467	1.27\\
-1.03114726787944	1.26\\
-1.03	1.25781089644513\\
-1.02609673868968	1.25\\
-1.02087541364572	1.24\\
-1.02	1.23834491264132\\
-1.01578577499238	1.23\\
-1.01049447667052	1.22\\
-1.01	1.21907385838892\\
-1.00536860726543	1.21\\
-1	1.2\\
-1	1.2\\
-0.994840594204797	1.19\\
-0.99	1.18107122986823\\
-0.989441996148542	1.18\\
-0.984196570388658	1.17\\
-0.98	1.16233587487781\\
-0.978769777274395	1.16\\
-0.973430799563665	1.15\\
-0.97	1.14379762371387\\
-0.967977507420333	1.14\\
-0.962536919113859	1.13\\
-0.96	1.12546076696165\\
-0.957058743308675	1.12\\
-0.951507874416548	1.11\\
-0.95	1.10733024691358\\
-0.946006368287168	1.1\\
-0.940335841467407	1.09\\
-0.94	1.08941171797418\\
-0.934812514966373	1.08\\
-0.93	1.07163015230842\\
-0.929092241533538	1.07\\
-0.923468474557376	1.06\\
-0.92	1.05403413231064\\
-0.917726742890331	1.05\\
-0.911964590452779	1.04\\
-0.91	1.03665874818753\\
-0.906202285473283	1.03\\
-0.900290133011479	1.02\\
-0.9	1.01951219512195\\
-0.894508132891919	1.01\\
-0.89	1.00248142320338\\
-0.888551812992139	1\\
-0.882632343937943	0.99\\
-0.88	0.985661538461538\\
-0.876654193412607	0.98\\
-0.870561599468811	0.97\\
-0.87	0.969085511773186\\
-0.864561928798959	0.96\\
-0.86	0.952636583101207\\
-0.858404066830685	0.95\\
-0.852260117577774	0.94\\
-0.85	0.93639705882353\\
-0.846076902098235	0.93\\
-0.84	0.920402552415679\\
-0.83975025626162	0.92\\
-0.833523167899568	0.91\\
-0.83	0.904506686719112\\
-0.827166253246139	0.9\\
-0.820724010666993	0.89\\
-0.82	0.888885794920038\\
-0.814335854402626	0.88\\
-0.81	0.873393764275925\\
-0.807810439167731	0.87\\
-0.801237663971138	0.86\\
-0.8	0.858139534883721\\
-0.794674222170148	0.85\\
-0.79	0.843047057492078\\
-0.787981345829428	0.84\\
-0.781244045792517	0.83\\
-0.78	0.828175069252078\\
-0.774505007874504	0.82\\
-0.77	0.813463732552904\\
-0.767641772743645	0.81\\
-0.760703980959579	0.8\\
-0.76	0.798993192272309\\
-0.753784874489191	0.79\\
-0.75	0.784648876404494\\
-0.746744438202159	0.78\\
-0.74	0.770576605101143\\
-0.739590739194557	0.77\\
-0.73245910380869	0.76\\
-0.73	0.756615466006303\\
-0.725230452107155	0.75\\
-0.72	0.74288932038835\\
-0.71788774448522	0.74\\
-0.710459662622077	0.73\\
-0.71	0.729384766862834\\
-0.703027148476588	0.72\\
-0.7	0.716\\
-0.695477546015969	0.71\\
-0.69	0.702856684141546\\
-0.687815088003609	0.7\\
-0.680045101592387	0.69\\
-0.68	0.68994213836478\\
-0.672265131193807	0.68\\
-0.67	0.677133547103052\\
-0.664365579941137	0.67\\
-0.66	0.664552397558849\\
-0.656348461246632	0.66\\
-0.65	0.652190860215054\\
-0.648214883360698	0.65\\
-0.64	0.640042141036533\\
-0.639965110042294	0.64\\
-0.631665197102596	0.63\\
-0.63	0.628017075843928\\
-0.623238012479778	0.62\\
-0.62	0.616202601908066\\
-0.61468089735127	0.61\\
-0.61	0.604595513095749\\
-0.605991847342709	0.6\\
-0.6	0.593191489361702\\
-0.597167999909376	0.59\\
-0.59	0.581986893203884\\
-0.588205628675174	0.58\\
-0.58	0.57097870913663\\
-0.579100124276259	0.57\\
-0.57	0.56016449396085\\
-0.569845961836601	0.56\\
-0.560450759391671	0.55\\
-0.56	0.549522558340536\\
-0.550891327307789	0.54\\
-0.55	0.539072580645161\\
-0.541154380822383	0.53\\
-0.54	0.528820565552699\\
-0.531230626697194	0.52\\
-0.53	0.518765890645024\\
-0.521109447940472	0.51\\
-0.52	0.508908439897698\\
-0.510778781561269	0.5\\
-0.51	0.499248595146871\\
-0.50022497238464	0.49\\
-0.5	0.489787234042553\\
-0.49	0.480504267047775\\
-0.489445109690754	0.48\\
-0.48	0.471406050695012\\
-0.478417903925109	0.47\\
-0.47	0.462503461696026\\
-0.46711798620527	0.46\\
-0.46	0.453799178307313\\
-0.455522685071986	0.45\\
-0.45	0.445296391752577\\
-0.443606432520018	0.44\\
-0.44	0.43699884009942\\
-0.431340346740816	0.43\\
-0.43	0.428910849645981\\
-0.42	0.420995655671762\\
-0.41870734923557	0.42\\
-0.41	0.413247043337383\\
-0.405670480289681	0.41\\
-0.4	0.405714285714286\\
-0.392168730299462	0.4\\
-0.39	0.398403849320708\\
-0.38	0.391270231444533\\
-0.378162787677657	0.39\\
-0.37	0.384299455864571\\
-0.363592557552822	0.38\\
-0.36	0.377562836185819\\
-0.35	0.371027227722772\\
-0.348375916922085	0.37\\
-0.34	0.364636407377707\\
-0.332424652374011	0.36\\
-0.33	0.358496119463633\\
-0.32	0.352514330958036\\
-0.31561449360424	0.35\\
-0.31	0.34673320610687\\
-0.3	0.341176470588235\\
-0.297798819433489	0.34\\
-0.29	0.335763630928657\\
-0.28	0.33061075604053\\
-0.278767367324066	0.33\\
-0.27	0.325578322094407\\
-0.26	0.320812432012432\\
-0.258219868992845	0.32\\
-0.25	0.316175742574257\\
-0.24	0.311783682983683\\
-0.235724442995833	0.31\\
-0.23	0.307561662038873\\
-0.22	0.303533593141076\\
-0.210648755962356	0.3\\
-0.21	0.299749282582702\\
-0.2	0.296078431372549\\
-0.19	0.292653128621089\\
-0.181701911751382	0.29\\
-0.18	0.289442280285036\\
-0.17	0.286364291292464\\
-0.16	0.283517039321511\\
-0.15	0.280892857142857\\
-0.146317492988506	0.28\\
-0.14	0.278425685199687\\
-0.13	0.276143723363038\\
-0.12	0.274076438986953\\
-0.11	0.272218695321415\\
-0.1	0.270566037735849\\
-0.0961170577652375	0.27\\
-0.0900000000000001	0.269080167380304\\
-0.0800000000000001	0.267778499613302\\
-0.0700000000000001	0.266680103806228\\
-0.0600000000000001	0.265782402448355\\
-0.05	0.265083333333333\\
-0.04	0.26458132118451\\
-0.03	0.264275255585006\\
-0.02	0.26416447467876\\
-0.01	0.264248754246886\\
0	0.264528301886792\\
0.00999999999999979	0.26500375613439\\
0.02	0.265676190476191\\
0.0299999999999998	0.266547122302158\\
0.04	0.267618526955201\\
0.0499999999999998	0.268892857142857\\
0.0574875368741064	0.27\\
0.0600000000000001	0.270359322033898\\
0.0699999999999998	0.271986282858201\\
0.0800000000000001	0.273818019359643\\
0.0899999999999999	0.275858729861371\\
0.1	0.27811320754717\\
0.107641812943702	0.28\\
0.11	0.28056539025284\\
0.12	0.283164523281596\\
0.13	0.285985434539351\\
0.14	0.289035116804823\\
0.142958906008978	0.29\\
0.15	0.29223623853211\\
0.16	0.295635931700074\\
0.17	0.299278072153326\\
0.171873783803555	0.3\\
0.18	0.303056199559795\\
0.19	0.307057995537374\\
0.196932797595313	0.31\\
0.2	0.311272727272727\\
0.21	0.31564080841639\\
0.219404917869586	0.32\\
0.22	0.320270137382502\\
0.23	0.32501394707828\\
0.239918703050463	0.33\\
0.24	0.330040086517664\\
0.25	0.335172018348624\\
0.258908951782467	0.34\\
0.26	0.340580821917808\\
0.27	0.346116482910695\\
0.276681979859937	0.35\\
0.28	0.351897035430224\\
0.29	0.357855647840532\\
0.29345590619339	0.36\\
0.3	0.364\\
0.309380889706321	0.37\\
0.31	0.370390444603801\\
0.32	0.376907997065297\\
0.324554200267455	0.38\\
0.33	0.383650470843897\\
0.339095580891604	0.39\\
0.34	0.390623908375089\\
0.35	0.397740825688073\\
0.353063504425131	0.4\\
0.36	0.405061437908497\\
0.366527000275942	0.41\\
0.37	0.412602283351313\\
0.379535044319564	0.42\\
0.38	0.420357519600855\\
0.39	0.428260516672774\\
0.392133788524771	0.43\\
0.4	0.436363636363636\\
0.404357877626438	0.44\\
0.41	0.444675677139762\\
0.416236946474464	0.45\\
0.42	0.453192821249103\\
0.427798106477246	0.46\\
0.43	0.461911799357373\\
0.439065555566671	0.47\\
0.44	0.470829850746269\\
0.45	0.479942660550459\\
0.450061555856276	0.48\\
0.46	0.489227212874909\\
0.46081281077182	0.49\\
0.47	0.498712495439621\\
0.471327296832247	0.5\\
0.48	0.50839737800437\\
0.481619981042305	0.51\\
0.49	0.518281174972718\\
0.491704253013069	0.52\\
0.5	0.528363636363636\\
0.501592077820764	0.53\\
0.51	0.53864494361586\\
0.511294127865065	0.54\\
0.52	0.549125710123817\\
0.520819896351823	0.55\\
0.53	0.559806986501277\\
0.53017779458097	0.56\\
0.539387794493329	0.57\\
0.54	0.570665913902611\\
0.548452305445808	0.58\\
0.55	0.581714601769911\\
0.557374033007314	0.59\\
0.56	0.592962046908316\\
0.566158256827245	0.6\\
0.57	0.604410906818993\\
0.574809502231166	0.61\\
0.58	0.61606432160804\\
0.583331576108837	0.62\\
0.59	0.627925947995666\\
0.59172759418886	0.63\\
0.6	0.64\\
0.6	0.64\\
0.608200742670766	0.65\\
0.61	0.652210303994344\\
0.616289067572292	0.66\\
0.62	0.664634069850321\\
0.624266137772268	0.67\\
0.63	0.677276860841424\\
0.63213251765236	0.68\\
0.639891469390807	0.69\\
0.64	0.690139873861248\\
0.647608798809233	0.7\\
0.65	0.703130530973451\\
0.655223832385136	0.71\\
0.66	0.716350465282749\\
0.662735504986486	0.72\\
0.67	0.729807660268019\\
0.670142135444198	0.73\\
0.677527604311297	0.74\\
0.68	0.743386836518047\\
0.68481946477039	0.75\\
0.69	0.757202922194335\\
0.692010525488357	0.76\\
0.699129159352642	0.77\\
0.7	0.771228070175439\\
0.706220305815115	0.78\\
0.71	0.785407956568174\\
0.71321301491422	0.79\\
0.72	0.799850759219089\\
0.720103056365191	0.8\\
0.72700452346964	0.81\\
0.73	0.814407284650833\\
0.733811963540377	0.82\\
0.74	0.829232588320116\\
0.740516416378899	0.83\\
0.747227103041144	0.84\\
0.75	0.844192477876106\\
0.753858709776809	0.85\\
0.76	0.859412833453496\\
0.760385316434735	0.86\\
0.766933238968763	0.87\\
0.77	0.874761768876285\\
0.773395130300924	0.88\\
0.779758797983328	0.89\\
0.78	0.890379344033496\\
0.786159626930273	0.9\\
0.79	0.906119953720185\\
0.792454793514726	0.91\\
0.798695341608492	0.92\\
0.8	0.922105263157894\\
0.804935406496531	0.93\\
0.81	0.938278576550735\\
0.81106377368733	0.94\\
0.817204753834557	0.95\\
0.82	0.954625336164189\\
0.823282804977549	0.96\\
0.829276061584695	0.97\\
0.83	0.971212495630898\\
0.83530578198452	0.98\\
0.84	0.987961059413028\\
0.841217510800792	0.99\\
0.84714447340222	1\\
0.85	1.00490044247788\\
0.85301099686595	1.01\\
0.85880957593968	1.02\\
0.86	1.02206699369306\\
0.864632128115553	1.03\\
0.87	1.03943435814455\\
0.870326791139734	1.04\\
0.876090465809717	1.05\\
0.88	1.05695053456878\\
0.881742737997648	1.06\\
0.887394728694362	1.07\\
0.89	1.07468467656416\\
0.893005183159279	1.08\\
0.898552879239702	1.09\\
0.9	1.09263157894737\\
0.904121934679112	1.1\\
0.909572198544457	1.11\\
0.91	1.11078666782851\\
0.915100123967142	1.12\\
0.92	1.12911529073941\\
0.920484912870907	1.13\\
0.925946269676678	1.14\\
0.93	1.14762404498393\\
0.931288982718464	1.15\\
0.936666333334891	1.16\\
0.94	1.16633802416489\\
0.941966619305305	1.17\\
0.947265767827654	1.18\\
0.95	1.18525442477876\\
0.95252312808598	1.19\\
0.957749559666046	1.2\\
0.96	1.20437092448836\\
0.962963345484942	1.21\\
0.968122265808881	1.22\\
0.97	1.22368565645899\\
0.97329167450331	1.23\\
0.978388045687865	1.24\\
0.98	1.24319718903725\\
0.983512115481422	1.25\\
0.988550688974223	1.26\\
0.99	1.26290451035451\\
0.993628292534961	1.27\\
0.998613639534148	1.28\\
1	1.28280701754386\\
1.00364347608947	1.29\\
1.00858001594167	1.3\\
1.01	1.30290451035451\\
1.01356060185736	1.31\\
1.01845262884911	1.32\\
1.02	1.32319718903725\\
1.02338228653037	1.33\\
1.02823399545468	1.34\\
1.03	1.34368565645899\\
1.03311084039625	1.35\\
1.03792635125238	1.36\\
1.04	1.36437092448836\\
1.04274827702938	1.37\\
1.04753165919894	1.38\\
1.05	1.38525442477876\\
1.05229632014964	1.39\\
1.05705161638546	1.4\\
1.06	1.40633802416489\\
1.0617564076888	1.41\\
1.06648765825517	1.42\\
1.07	1.42762404498393\\
1.07112969305002	1.43\\
1.07584096036292	1.44\\
1.08	1.44911529073941\\
1.08041704348856	1.45\\
1.08511243762419	1.46\\
1.08964424846414	1.47\\
1.09	1.47078666782851\\
1.09430274095133	1.48\\
1.09882118194585	1.49\\
1.1	1.49263157894737\\
1.10341225111893	1.5\\
1.10792120207771	1.51\\
1.11	1.51468467656416\\
1.11244106963852	1.52\\
1.11694452081889	1.53\\
1.12	1.53695053456878\\
1.12138900635107	1.54\\
1.1258910870544	1.55\\
1.13	1.55943435814455\\
1.13025556336283	1.56\\
1.13476057331103	1.57\\
1.13910742938847	1.58\\
1.14	1.58206699369306\\
1.14355235896386	1.59\\
1.14790324205163	1.6\\
1.15	1.60490044247788\\
1.15226550945165	1.61\\
1.15662467592673	1.62\\
1.16	1.62796105941303\\
1.16089875090221	1.63\\
1.16527069994636	1.64\\
1.16949020328235	1.65\\
1.17	1.6512124956309\\
1.17383995509548	1.66\\
1.17807272687061	1.67\\
1.18	1.67462533616419\\
1.1823307234023	1.68\\
1.18658132673949	1.69\\
1.19	1.69827857655073\\
1.1907408907726	1.7\\
1.19501421692324	1.71\\
1.19913769625494	1.72\\
1.2	1.7221052631579\\
1.20336922430723	1.73\\
1.20751543373808	1.74\\
1.21	1.74611995372019\\
1.21164374399095	1.75\\
1.21581775264144	1.76\\
1.21984723095568	1.77\\
1.22	1.7703793440335\\
1.22404199909845	1.78\\
1.22809908619098	1.79\\
1.23	1.79476176887628\\
1.23218504101466	1.8\\
1.23627505679528	1.81\\
1.24	1.8194128334535\\
1.2402432041187	1.82\\
1.24437196734068	1.83\\
1.2483539499295	1.84\\
1.25	1.84419247787611\\
1.25238610506934	1.85\\
1.25640632574899	1.86\\
1.26	1.86923258832011\\
1.26031314286555	1.87\\
1.26437782355423	1.88\\
1.26829782966591	1.89\\
1.27	1.89440728465083\\
1.27226407688968	1.9\\
1.27622793896247	1.91\\
1.28	1.91985075921909\\
1.28006001417486	1.92\\
1.28407445999685	1.93\\
1.28794505148887	1.94\\
1.29	1.94540795656817\\
1.29183227614577	1.95\\
1.29575278757567	1.96\\
1.29953690870684	1.97\\
1.3	1.97122807017544\\
1.30347323766184	1.98\\
1.30730671232133	1.99\\
1.31	1.99720292219434\\
1.31110040942416	2\\
8	751\\
1.50934171430422	2\\
1.5059427300902	1.99\\
1.50266496572333	1.98\\
1.5	1.97163636363636\\
1.49943570094407	1.97\\
1.49601135916562	1.96\\
1.49270924042534	1.95\\
1.49	1.94155763358779\\
1.48945939689078	1.94\\
1.48601199621812	1.93\\
1.48268756546768	1.92\\
1.48	1.91168511256354\\
1.47941137785501	1.91\\
1.47594325833596	1.9\\
1.47259859514348	1.89\\
1.47	1.88201850308083\\
1.46929035217661	1.88\\
1.46580387869482	1.87\\
1.46244108553558	1.86\\
1.46	1.85255704989154\\
1.4590951628286	1.85\\
1.45559270837252	1.84\\
1.45221389615361	1.83\\
1.45	1.82329954954955\\
1.44882478762749	1.82\\
1.44530871661338	1.81\\
1.44191599022361	1.8\\
1.44	1.79424436468055\\
1.43847833766033	1.79\\
1.43495098954762	1.78\\
1.43154643363222	1.77\\
1.43	1.76538944265809\\
1.42805505397417	1.76\\
1.42451872740105	1.75\\
1.42110439255227	1.74\\
1.42	1.73673233830846\\
1.41755430260742	1.73\\
1.41401124025818	1.72\\
1.41058912980056	1.71\\
1.41	1.70827024019781\\
1.40697556807283	1.7\\
1.40342794246791	1.69\\
1.4	1.68\\
1.4	1.68\\
1.39631844542806	1.67\\
1.39276834580274	1.66\\
1.39	1.65198738722483\\
1.38926268154143	1.65\\
1.38558263109143	1.64\\
1.38203205150177	1.63\\
1.38	1.62416492483894\\
1.37844435606504	1.62\\
1.37476791257676	1.61\\
1.37121874134147	1.6\\
1.37	1.59652821922668\\
1.36754502432891	1.59\\
1.3638741573323	1.58\\
1.36032816788886	1.57\\
1.36	1.56907266338721\\
1.35656476357538	1.56\\
1.35290130087429	1.55\\
1.35	1.54185810810811\\
1.34929093935666	1.54\\
1.34550371339401	1.53\\
1.34184933440632	1.52\\
1.34	1.51485274411974\\
1.33813534267022	1.51\\
1.33436206127821	1.5\\
1.33071829211103	1.49\\
1.33	1.48801727721029\\
1.32689786670029	1.48\\
1.32314002808388	1.47\\
1.32	1.46139466474405\\
1.31945607726942	1.46\\
1.31557893238338	1.45\\
1.31183785358254	1.44\\
1.31	1.43500570971185\\
1.3080339746308	1.43\\
1.30417896484345	1.42\\
1.30045578228192	1.41\\
1.3	1.40877192982456\\
1.29653023311976	1.4\\
1.29269837803511	1.39\\
1.29	1.3827836080258\\
1.28889007625408	1.38\\
1.28494543951217	1.37\\
1.28113756009915	1.36\\
1.28	1.35698405081157\\
1.27719512653093	1.35\\
1.27328013458113	1.34\\
1.27	1.3313699352285\\
1.26944534570739	1.33\\
1.26541972017016	1.32\\
1.26153479899455	1.31\\
1.26	1.3059974522293\\
1.25755402276291	1.3\\
1.25356446254346	1.29\\
1.25	1.28077702702703\\
1.24968049615151	1.28\\
1.24558339865199	1.27\\
1.24162986278667	1.26\\
1.24	1.25582052370842\\
1.23758892142169	1.25\\
1.23353406868553	1.24\\
1.23	1.23101009355883\\
1.22957810451008	1.23\\
1.22541967188163	1.22\\
1.22140649618355	1.21\\
1.22	1.20645476358504\\
1.21728366774942	1.2\\
1.21317324925205	1.19\\
1.21	1.18206701898961\\
1.20912283209199	1.18\\
1.20491353282277	1.17\\
1.2008499901897	1.16\\
1.2	1.15789473684211\\
1.19662410285533	1.15\\
1.19246801986415	1.14\\
1.19	1.13393847385272\\
1.18830152380441	1.13\\
1.18405185615579	1.12\\
1.18	1.11013294881038\\
1.17994217419084	1.11\\
1.17559809610942	1.1\\
1.17140614287122	1.09\\
1.17	1.08660833039803\\
1.16710317806506	1.08\\
1.1628233716004	1.07\\
1.16	1.06324903777619\\
1.15856339095706	1.06\\
1.15419545678119	1.05\\
1.15	1.04005630630631\\
1.14997488565943	1.04\\
1.14551878949763	1.03\\
1.14121852611218	1.02\\
1.14	1.01714012649332\\
1.13678964892117	1.01\\
1.1324077309348	1\\
1.13	0.994399804895353\\
1.12800421586178	0.99\\
1.1235410757614	0.98\\
1.12	0.971838511095204\\
1.11915858787301	0.97\\
1.11461489655063	0.96\\
1.11022789737406	0.95\\
1.11	0.949479989550679\\
1.105625467617	0.94\\
1.10116410731207	0.93\\
1.1	0.927368421052632\\
1.09656901708058	0.92\\
1.09203431710832	0.91\\
1.09	0.905444383610032\\
1.08744174312117	0.9\\
1.08283494010165	0.89\\
1.08	0.883711727078891\\
1.07823983080481	0.88\\
1.07356237421684	0.87\\
1.07	0.86217400857449\\
1.06895946922252	0.86\\
1.06421301677522	0.85\\
1.06	0.840834458004307\\
1.05959686865718	0.84\\
1.05478327928663	0.83\\
1.05013633378064	0.82\\
1.05	0.81970652173913\\
1.04526960196429	0.81\\
1.04056106850546	0.8\\
1.04	0.798804187020237\\
1.03566846769521	0.79\\
1.03090030046209	0.78\\
1.03	0.778100402238544\\
1.02597641519678	0.77\\
1.02115071319826	0.76\\
1.02	0.75759663629993\\
1.01619005109326	0.75\\
1.01130904417098	0.74\\
1.01	0.737293949491406\\
1.0063060606572	0.73\\
1.00137209384093	0.72\\
1	0.71719298245614\\
0.996321216977314	0.71\\
0.991336733376708	0.7\\
0.99	0.697293949491407\\
0.986232388337553	0.69\\
0.981199910783401	0.68\\
0.98	0.67759663629993\\
0.976036543620634	0.67\\
0.970958655293085	0.66\\
0.97	0.658100402238545\\
0.965730755582183	0.65\\
0.960610079882513	0.64\\
0.96	0.638804187020237\\
0.955312201878069	0.63\\
0.950151381813191	0.62\\
0.95	0.619706521739131\\
0.94477816376614	0.61\\
0.94	0.600834458004307\\
0.939546112583525	0.6\\
0.934126022443127	0.59\\
0.93	0.582174008574491\\
0.92880498426461	0.58\\
0.923353253016614	0.57\\
0.92	0.563711727078891\\
0.917937919222929	0.56\\
0.912457416149423	0.55\\
0.91	0.545444383610032\\
0.906942646794828	0.54\\
0.901436147448301	0.53\\
0.9	0.527368421052631\\
0.895816978882715	0.52\\
0.890287144699425	0.51\\
0.89	0.509479989550679\\
0.884558793688556	0.5\\
0.88	0.491838511095204\\
0.878932363409628	0.49\\
0.873166017397408	0.48\\
0.87	0.474399804895353\\
0.867416152367858	0.47\\
0.861636604014498	0.46\\
0.86	0.457140126493324\\
0.855758950047533	0.45\\
0.85	0.440056306306306\\
0.849966149943294	0.44\\
0.843958916978147	0.43\\
0.84	0.423249037776194\\
0.838023918544729	0.42\\
0.832014182394377	0.41\\
0.83	0.406608330398028\\
0.825933202911195	0.4\\
0.82	0.390132948810382\\
0.819917156639767	0.39\\
0.813692260186033	0.38\\
0.81	0.373938473852722\\
0.807516089479606	0.37\\
0.801299234209703	0.36\\
0.8	0.357894736842105\\
0.794959784682002	0.35\\
0.79	0.34206701898961\\
0.788663217958001	0.34\\
0.78224637656857	0.33\\
0.78	0.326454763585039\\
0.775772967539557	0.32\\
0.77	0.311010093558834\\
0.769329711723866	0.31\\
0.762720957421688	0.3\\
0.76	0.295820523708421\\
0.756087582698956	0.29\\
0.75	0.280777027027027\\
0.749470565920528	0.28\\
0.742679271249211	0.27\\
0.74	0.2659974522293\\
0.73585925351347	0.26\\
0.73	0.251369935228499\\
0.729040906428513	0.25\\
0.722077374116675	0.24\\
0.72	0.236984050811574\\
0.715043590947507	0.23\\
0.71	0.222783608025797\\
0.70799606902929	0.22\\
0.700871100735333	0.21\\
0.7	0.208771929824562\\
0.693596139240425	0.2\\
0.69	0.195005709711846\\
0.686291456094459	0.19\\
0.68	0.181394664744052\\
0.678951196535435	0.18\\
0.671472092369452	0.17\\
0.67	0.168017277210285\\
0.663882137121431	0.16\\
0.66	0.154852744119744\\
0.656239548971225	0.15\\
0.65	0.141858108108108\\
0.648537460165733	0.14\\
0.640722169244051	0.13\\
0.64	0.12907266338721\\
0.632748544909465	0.12\\
0.63	0.116528219226676\\
0.624697482441923	0.11\\
0.62	0.104164924838941\\
0.616561203909603	0.1\\
0.61	0.0919873872248285\\
0.608331553410559	0.0899999999999999\\
0.6	0.0800000000000001\\
0.6	0.0800000000000001\\
0.591466286592039	0.0699999999999998\\
0.59	0.0682702401978098\\
0.582819899871965	0.0600000000000001\\
0.58	0.0567323383084578\\
0.574051821442868	0.0499999999999998\\
0.57	0.045389442658092\\
0.565152657835332	0.04\\
0.56	0.0342443646805456\\
0.556112635217655	0.0299999999999998\\
0.55	0.0232995495495494\\
0.546921591140234	0.02\\
0.54	0.0125570498915402\\
0.537568962406015	0.00999999999999979\\
0.53	0.00201850308082619\\
0.528043768057397	0\\
0.52	-0.00831488743645602\\
0.518334586364982	-0.01\\
0.51	-0.018442366412214\\
0.508429524593138	-0.02\\
0.5	-0.0283636363636364\\
0.498316180198653	-0.03\\
0.49	-0.0380788622319155\\
0.48798159198755	-0.04\\
0.48	-0.0475886710239651\\
0.477412179605834	-0.05\\
0.47	-0.0568941464298657\\
0.466593669565217	-0.0600000000000001\\
0.46	-0.0659968185104845\\
0.455511005795298	-0.0700000000000001\\
0.45	-0.0748986486486485\\
0.444148242457562	-0.0800000000000001\\
0.44	-0.0836020100502513\\
0.43248841643928	-0.0900000000000001\\
0.43	-0.0921096641657733\\
0.420513396548576	-0.1\\
0.42	-0.10042473347548\\
0.41	-0.108497601611131\\
0.408100708001879	-0.11\\
0.4	-0.116363636363636\\
0.395283675148977	-0.12\\
0.39	-0.124042926741249\\
0.382067827152485	-0.13\\
0.38	-0.131540157480315\\
0.37	-0.138818297168076\\
0.368336263252599	-0.14\\
0.36	-0.145862782228696\\
0.354006294003287	-0.15\\
0.35	-0.152736486486486\\
0.34	-0.159424685883223\\
0.339114234100963	-0.16\\
0.33	-0.165848466593647\\
0.323405124041236	-0.17\\
0.32	-0.172116510454218\\
0.31	-0.178170029509406\\
0.306891777170048	-0.18\\
0.3	-0.184\\
0.29	-0.189680769230769\\
0.289417445113164	-0.19\\
0.28	-0.19507944403804\\
0.270680913703826	-0.2\\
0.27	-0.200353706369198\\
0.26	-0.205345267791636\\
0.250447015068643	-0.21\\
0.25	-0.210213963963964\\
0.24	-0.214795011005136\\
0.23	-0.219247088465845\\
0.22823131727207	-0.22\\
0.22	-0.223433504023409\\
0.21	-0.227451114827202\\
0.203378688200173	-0.23\\
0.2	-0.231272727272727\\
0.19	-0.234850221320546\\
0.18	-0.238268960359013\\
0.174647393830765	-0.24\\
0.17	-0.24146730558598\\
0.16	-0.244435772357724\\
0.15	-0.247231308411215\\
0.14	-0.249848526077097\\
0.139371565207165	-0.25\\
0.13	-0.252198216255976\\
0.12	-0.254359020044543\\
0.11	-0.256332066641707\\
0.1	-0.25811320754717\\
0.0899999999999999	-0.259698574686431\\
0.0878142355129881	-0.26\\
0.0800000000000001	-0.261044657332351\\
0.0699999999999998	-0.262184123749537\\
0.0600000000000001	-0.263125539836188\\
0.0499999999999998	-0.263866822429907\\
0.04	-0.264406301575394\\
0.0299999999999998	-0.264742741632193\\
0.02	-0.264875357950264\\
0.00999999999999979	-0.26480382874387\\
0	-0.264528301886792\\
-0.01	-0.264049396454168\\
-0.02	-0.263368198944989\\
-0.03	-0.262486254230914\\
-0.04	-0.261405551387847\\
-0.05	-0.260128504672897\\
-0.0508705727418926	-0.26\\
-0.0600000000000001	-0.258606032482599\\
-0.0700000000000001	-0.256881473643709\\
-0.0800000000000001	-0.254963733741392\\
-0.0900000000000001	-0.252857069555302\\
-0.1	-0.250566037735849\\
-0.102277195406138	-0.25\\
-0.11	-0.248021373006612\\
-0.12	-0.245274942174248\\
-0.13	-0.242354696449026\\
-0.137606240942308	-0.24\\
-0.14	-0.239237706205813\\
-0.15	-0.235861650485437\\
-0.16	-0.23232663085188\\
-0.166280342854183	-0.23\\
-0.17	-0.228585643954313\\
-0.18	-0.224604195804196\\
-0.19	-0.220482171581769\\
-0.191118177094667	-0.22\\
-0.2	-0.216078431372549\\
-0.21	-0.211528734079506\\
-0.21323772351078	-0.21\\
-0.22	-0.206735280189424\\
-0.23	-0.201769271035285\\
-0.233448485650046	-0.2\\
-0.24	-0.196568171021378\\
-0.25	-0.19120145631068\\
-0.252169142264181	-0.19\\
-0.26	-0.185578463974663\\
-0.269699172302239	-0.18\\
-0.27	-0.179823699072206\\
-0.28	-0.173775532754538\\
-0.286130364437325	-0.17\\
-0.29	-0.167575652348454\\
-0.3	-0.161176470588235\\
-0.301792002293763	-0.16\\
-0.31	-0.154527658303465\\
-0.316680642270571	-0.15\\
-0.32	-0.147716734033953\\
-0.33	-0.140708526979126\\
-0.330986496305334	-0.14\\
-0.34	-0.133441021548284\\
-0.344653780873151	-0.13\\
-0.35	-0.125997474747475\\
-0.357885314655596	-0.12\\
-0.36	-0.118372526573998\\
-0.37	-0.110538685986503\\
-0.370672631513724	-0.11\\
-0.38	-0.102455172413793\\
-0.382986890403775	-0.1\\
-0.39	-0.0941816875758803\\
-0.394959907262294	-0.0900000000000001\\
-0.4	-0.0857142857142858\\
-0.406612179161717	-0.0800000000000001\\
-0.41	-0.077049341834636\\
-0.417962619210798	-0.0700000000000001\\
-0.42	-0.0681835956917979\\
-0.429028781663829	-0.0600000000000001\\
-0.43	-0.0591141934139225\\
-0.439827054236564	-0.05\\
-0.44	-0.0498387259010898\\
-0.45	-0.0403409090909091\\
-0.450352494407089	-0.04\\
-0.46	-0.0306355231143553\\
-0.460643184732981	-0.03\\
-0.47	-0.0207283245221635\\
-0.470721815268975	-0.02\\
-0.48	-0.0106190709046455\\
-0.480601086085641	-0.01\\
-0.49	-0.000308017135862936\\
-0.490293128117119	0\\
-0.499798101752678	0.00999999999999979\\
-0.5	0.0102127659574468\\
-0.509110696948511	0.02\\
-0.51	0.0209549340706083\\
-0.51825959624404	0.0299999999999998\\
-0.52	0.0319034834324554\\
-0.527255603247266	0.04\\
-0.53	0.0430559771089445\\
-0.536109160412102	0.0499999999999998\\
-0.54	0.0544094674556213\\
-0.544830362865161	0.0600000000000001\\
-0.55	0.0659605263157895\\
-0.553428966118687	0.0699999999999998\\
-0.56	0.0777052808046941\\
-0.561914388860237	0.0800000000000001\\
-0.57	0.0896394539391414\\
-0.570295711929162	0.0899999999999999\\
-0.578489381578128	0.1\\
-0.58	0.101834399308557\\
-0.586570358071162	0.11\\
-0.59	0.114231252681253\\
-0.594566500095935	0.12\\
-0.6	0.126808510638298\\
-0.602486156253263	0.13\\
-0.61	0.139560122311261\\
-0.61033730867811	0.14\\
-0.618000387339065	0.15\\
-0.62	0.152587968613775\\
-0.625582827826801	0.16\\
-0.63	0.165801228978008\\
-0.633115051467236	0.17\\
-0.64	0.179173401534527\\
-0.640604197791276	0.18\\
-0.647920416679516	0.19\\
-0.65	0.192815934065934\\
-0.655162181326604	0.2\\
-0.66	0.206642150910668\\
-0.662378552723531	0.21\\
-0.669544888988527	0.22\\
-0.67	0.220634680661009\\
-0.676524710037261	0.23\\
-0.68	0.234911521547933\\
-0.683496649517636	0.24\\
-0.69	0.249312440501947\\
-0.690466118786293	0.25\\
-0.697247618042781	0.26\\
-0.7	0.264\\
-0.704003671515063	0.27\\
-0.71	0.27882542557835\\
-0.710773622740504	0.28\\
-0.717375767342503	0.29\\
-0.72	0.293917994628469\\
-0.723944693236171	0.3\\
-0.73	0.309151842913559\\
-0.730543241970954	0.31\\
-0.736954480494258	0.32\\
-0.74	0.324667026055705\\
-0.743365095833174	0.33\\
-0.749805771077723	0.34\\
-0.75	0.340301724137931\\
-0.756029791996295	0.35\\
-0.76	0.35623935309973\\
-0.762310682961501	0.36\\
-0.768539681396036	0.37\\
-0.77	0.372325401560349\\
-0.774648395099477	0.38\\
-0.78	0.388618084153984\\
-0.780827528899313	0.39\\
-0.786835412221484	0.4\\
-0.79	0.405161592879964\\
-0.79285732575581	0.41\\
-0.79887358687954	0.42\\
-0.8	0.421860465116279\\
-0.804741127656894	0.43\\
-0.81	0.438779308005428\\
-0.810703442344156	0.44\\
-0.816480786112378	0.45\\
-0.82	0.455947010119595\\
-0.822304651177998	0.46\\
-0.828078015977694	0.47\\
-0.83	0.473287073398785\\
-0.833767195975417	0.48\\
-0.839534431312293	0.49\\
-0.84	0.490805698005698\\
-0.845092972149781	0.5\\
-0.85	0.508577586206897\\
-0.850780175288838	0.51\\
-0.856283840498833	0.52\\
-0.86	0.526570736253495\\
-0.861857340257052	0.53\\
-0.867341659574187	0.54\\
-0.87	0.544753680981595\\
-0.872806336183774	0.55\\
-0.878268317321977	0.56\\
-0.88	0.56313123209169\\
-0.883629348792904	0.57\\
-0.889065761583719	0.58\\
-0.89	0.581707894736842\\
-0.894328629456594	0.59\\
-0.899736029005133	0.6\\
-0.9	0.600487804878049\\
-0.904906518246586	0.61\\
-0.91	0.619499319568278\\
-0.910256036763515	0.62\\
-0.9153654646194	0.63\\
-0.92	0.638734531693472\\
-0.920639690102283	0.64\\
-0.925708045358736	0.65\\
-0.93	0.658172963315865\\
-0.930913088768386	0.66\\
-0.935936979422997	0.67\\
-0.94	0.677816299137105\\
-0.941079138480199	0.68\\
-0.946055139386108	0.69\\
-0.95	0.697665662650602\\
-0.95114087225646	0.7\\
-0.956065559215272	0.71\\
-0.96	0.717721587608906\\
-0.961101453683934	0.72\\
-0.965971438198805	0.73\\
-0.97	0.737983997085964\\
-0.970964176833694	0.74\\
-0.975776140918574	0.75\\
-0.98	0.758452190847127\\
-0.980732462834861	0.76\\
-0.985483193251959	0.77\\
-0.99	0.779124841540712\\
-0.990409853190836	0.78\\
-0.995096274484242	0.79\\
-1	0.8\\
-1	0.8\\
-1.00461920570981	0.81\\
-1.00945154899695	0.82\\
-1.01	0.821130215274218\\
-1.01405593479547	0.83\\
-1.01881365546348	0.84\\
-1.02	0.842467144319345\\
-1.02341051826708	0.85\\
-1.0280906387404	0.86\\
-1.03	0.864006661562021\\
-1.03268710055786	0.87\\
-1.03728693354155	0.88\\
-1.04	0.885744048830112\\
-1.04188989111985	0.89\\
-1.04640706248541	0.9\\
-1.05	0.907674050632911\\
-1.0510231399455	0.91\\
-1.05545560669852	0.92\\
-1.06	0.929790936555891\\
-1.06009111207422	0.93\\
-1.06443717520042	0.94\\
-1.06898934283864	0.95\\
-1.07	0.952198551869404\\
-1.07335637379006	0.96\\
-1.07781260902958	0.97\\
-1.08	0.974798746081505\\
-1.08221777413911	0.98\\
-1.0865768611776	0.99\\
-1.09	0.99757278612118\\
-1.09102588375893	1\\
-1.09528702587193	1.01\\
-1.09975790998851	1.02\\
-1.1	1.02054054054054\\
-1.10394792795821	1.03\\
-1.10831017367076	1.04\\
-1.11	1.04380387493319\\
-1.11256425881585	1.05\\
-1.1168178502667	1.06\\
-1.12	1.06722111932418\\
-1.12114054771514	1.07\\
-1.12528591850101	1.08\\
-1.12963909115467	1.09\\
-1.13	1.09082630566245\\
-1.13371914125402	1.1\\
-1.13795275572067	1.11\\
-1.14	1.1147241603467\\
-1.14212204042013	1.12\\
-1.14623756892774	1.13\\
-1.15	1.13875\\
-1.15049887666279	1.14\\
-1.15449820967058	1.15\\
-1.15869765619191	1.16\\
-1.16	1.16305470653378\\
-1.16273903383067	1.17\\
-1.16681093082246	1.18\\
-1.17	1.18753727569331\\
-1.1709640627744	1.19\\
-1.17491150685392	1.2\\
-1.179061255625	1.21\\
-1.18	1.21223720405862\\
-1.18300338587528	1.22\\
-1.18701693219647	1.23\\
-1.19	1.23716236885699\\
-1.19109017065505	1.24\\
-1.19497194906249	1.25\\
-1.1990556895296	1.26\\
-1.2	1.26228571428571\\
-1.20292974790062	1.27\\
-1.20686985404995	1.28\\
-1.21	1.28763207705193\\
-1.2108933252587	1.29\\
-1.21469564589982	1.3\\
-1.21869694640814	1.31\\
-1.22	1.31320230680507\\
-1.22253574060031	1.32\\
-1.22638757330941	1.33\\
-1.23	1.3389415682968\\
-1.23039234868493	1.34\\
-1.23410198421634	1.35\\
-1.23800500560372	1.36\\
-1.24	1.36497612977984\\
-1.24184190095258	1.37\\
-1.24559152855792	1.38\\
-1.24954718159889	1.39\\
-1.25	1.39113805970149\\
-1.25321345108681	1.4\\
-1.25700365477748	1.41\\
-1.26	1.41758331402086\\
-1.26087139178892	1.42\\
-1.26450639360805	1.43\\
-1.26833745831703	1.44\\
-1.27	1.44423886241811\\
-1.27205519937534	1.45\\
-1.27571998674626	1.46\\
-1.27959185764582	1.47\\
-1.28	1.4710482252142\\
-1.28315947835948	1.48\\
-1.28685336611303	1.49\\
-1.29	1.49813663512714\\
-1.29065438999646	1.5\\
-1.2941836155743	1.51\\
-1.29790556178069	1.52\\
-1.3	1.52545454545455\\
-1.30157991028216	1.53\\
-1.30512694196859	1.54\\
-1.30887551906994	1.55\\
-1.31	1.55294913097455\\
-1.31242568775993	1.56\\
-1.31598875508369	1.57\\
-1.31976212328067	1.58\\
-1.32	1.58062820838628\\
-1.32319136346673	1.59\\
-1.32676834453121	1.6\\
-1.33	1.60859124466138\\
-1.33047778875651	1.61\\
-1.33387660918942	1.62\\
-1.33746502038466	1.63\\
-1.34	1.63678505603985\\
-1.34107972801862	1.64\\
-1.34448115365909	1.65\\
-1.34807814427378	1.66\\
-1.35	1.66517857142857\\
-1.35160359458595	1.67\\
-1.3550048102176	1.68\\
-1.35860716282874	1.69\\
-1.36	1.69377800776197\\
-1.36204959276404	1.7\\
-1.36544750550029	1.71\\
-1.36905164296608	1.72\\
-1.37	1.7225892363397\\
-1.37241805439948	1.73\\
-1.37580930856376	1.74\\
-1.37941130834577	1.75\\
-1.38	1.75161767068273\\
-1.38270946176189	1.76\\
-1.38609045977871	1.77\\
-1.38968607616718	1.78\\
-1.39	1.78086815091774\\
-1.39292446925355	1.79\\
-1.39629139871057	1.8\\
-1.39987609332155	1.81\\
-1.4	1.81034482758621\\
-1.40306392328811	1.82\\
-1.40641279013503	1.83\\
-1.40998177079109	1.84\\
-1.41	1.84005104821803\\
-1.41312887965847	1.85\\
-1.41645554728831	1.86\\
-1.42	1.86998996036988\\
-1.4200031528565	1.87\\
-1.42312061772146	1.88\\
-1.42642085144171	1.89\\
-1.42994325552663	1.9\\
-1.43	1.9001608649035\\
-1.43304065076569	1.91\\
-1.43631016692315	1.92\\
-1.43980146598448	1.93\\
-1.44	1.93056623376623\\
-1.44289073200174	1.94\\
-1.44612525078853	1.95\\
-1.44958018019616	1.96\\
-1.45	1.96120454545455\\
-1.45267285572049	1.97\\
-1.45586815647472	1.98\\
-1.45928149939571	1.99\\
-1.46	1.9920737920937\\
-1.46238925330318	2\\
16	580\\
-1.29323901139595	2\\
-1.29	1.99203715624056\\
-1.28920483050473	1.99\\
-1.28527625564527	1.98\\
-1.2812196838652	1.97\\
-1.28	1.96702608695652\\
-1.27723475237771	1.96\\
-1.27321095180225	1.95\\
-1.27	1.94222311050888\\
-1.26911784016943	1.94\\
-1.26512172297911	1.93\\
-1.26099562835623	1.92\\
-1.26	1.91760879657911\\
-1.25695527054436	1.91\\
-1.2528567618805	1.9\\
-1.25	1.89318609022556\\
-1.2487143905134	1.89\\
-1.24463867064685	1.88\\
-1.24042995225551	1.87\\
-1.24	1.86898326206475\\
-1.23634408455125	1.86\\
-1.23215791582909	1.85\\
-1.23	1.84493313760915\\
-1.22797530765024	1.84\\
-1.22380711329465	1.83\\
-1.22	1.82112513368984\\
-1.21953426283184	1.82\\
-1.21537977784468	1.81\\
-1.21109200359683	1.8\\
-1.21	1.7974768661227\\
-1.20687775701831	1.79\\
-1.20260311762382	1.78\\
-1.2	1.77402985074627\\
-1.19830254975887	1.77\\
-1.19403689734584	1.76\\
-1.19	1.75081113897905\\
-1.18965533756908	1.75\\
-1.18539475562993	1.74\\
-1.18099827040934	1.73\\
-1.18	1.72774840649429\\
-1.17667778103639	1.72\\
-1.17228536769373	1.71\\
-1.17	1.70488513835168\\
-1.16788676343301	1.7\\
-1.16349453166724	1.69\\
-1.16	1.68224018856806\\
-1.15902221530631	1.68\\
-1.15462644194225	1.67\\
-1.1500897258593	1.66\\
-1.15	1.65980263157895\\
-1.14568149131065	1.65\\
-1.1411396941021	1.64\\
-1.14	1.63751317829457\\
-1.13665980269676	1.63\\
-1.13210913973535	1.62\\
-1.13	1.61543535936114\\
-1.12756124280477	1.61\\
-1.1229980404562	1.6\\
-1.12	1.59356523781562\\
-1.11838543283661	1.59\\
-1.11380610507637	1.58\\
-1.11	1.57189937299504\\
-1.10913175658336	1.57\\
-1.10453278335562	1.56\\
-1.1	1.5504347826087\\
-1.09979936613073	1.55\\
-1.0951772731062	1.54\\
-1.09040962277139	1.53\\
-1.09	1.52914424161472\\
-1.08573852479709	1.52\\
-1.080944756798	1.51\\
-1.08	1.50804347312463\\
-1.07621524383253	1.5\\
-1.07139163661497	1.49\\
-1.07	1.48714347250809\\
-1.06660589063248	1.48\\
-1.06174871759798	1.47\\
-1.06	1.46644264792033\\
-1.05690867860037	1.46\\
-1.05201418857802	1.45\\
-1.05	1.4459397810219\\
-1.0471215700257	1.44\\
-1.04218596794853	1.43\\
-1.04	1.42563401281305\\
-1.03724226993318	1.42\\
-1.0322616974793	1.41\\
-1.03	1.40552483289741\\
-1.02726821785628	1.4\\
-1.02223873382019	1.39\\
-1.02	1.3856120719675\\
-1.01719657747964	1.38\\
-1.01211413763876	1.37\\
-1.01	1.36589589736155\\
-1.00702422406104	1.36\\
-1.00188466029752	1.35\\
-1	1.3463768115942\\
-0.996747729508663	1.34\\
-0.991546727936389	1.33\\
-0.99	1.3270556538127\\
-0.986363344952422	1.32\\
-0.981096422783631	1.31\\
-0.98	1.30793360417876\\
-0.975866980607953	1.3\\
-0.970529461472859	1.29\\
-0.97	1.28901219122348\\
-0.965254182687709	1.28\\
-0.96	1.27028500282965\\
-0.959848343972494	1.27\\
-0.954520107064041	1.26\\
-0.95	1.25172872340426\\
-0.949069919892916	1.25\\
-0.943659489333012	1.24\\
-0.94	1.23337370517928\\
-0.938164552699517	1.23\\
-0.932666610863318	1.22\\
-0.93	1.21522277920594\\
-0.927126352060161	1.21\\
-0.92153526034016	1.2\\
-0.92	1.19727917383821\\
-0.91594892896912	1.19\\
-0.910258690227066	1.18\\
-0.91	1.17954654078985\\
-0.904625350654325	1.17\\
-0.9	1.16197183098592\\
-0.898877432463531	1.16\\
-0.893148089596571	1.15\\
-0.89	1.1445970388212\\
-0.887351776655961	1.14\\
-0.881508965830248	1.13\\
-0.88	1.12743958927553\\
-0.87566264940032	1.12\\
-0.87	1.11049063110863\\
-0.86971063651074	1.11\\
-0.863800924768278	1.1\\
-0.86	1.09369138998312\\
-0.857796035141271	1.09\\
-0.851756667119256	1.08\\
-0.85	1.07711879432624\\
-0.845696533474646	1.07\\
-0.84	1.06075770728993\\
-0.839536431225184	1.06\\
-0.833401017096151	1.05\\
-0.83	1.04454898904802\\
-0.827181210331012	1.04\\
-0.820897348961576	1.03\\
-0.82	1.02857946681792\\
-0.814614565020985	1.02\\
-0.81	1.01277764001115\\
-0.80823461388958	1.01\\
-0.801822856616833	1\\
-0.8	0.997183098591549\\
-0.795373650307834	0.99\\
-0.79	0.981793003602106\\
-0.788830348629657	0.98\\
-0.782268269655605	0.97\\
-0.78	0.966580818844644\\
-0.775648117498157	0.96\\
-0.77	0.951588773819387\\
-0.768935461934523	0.95\\
-0.762199279208939	0.94\\
-0.76	0.936768662562955\\
-0.755400689577156	0.93\\
-0.75	0.922163793103448\\
-0.748509714995758	0.92\\
-0.741573797161072	0.91\\
-0.74	0.907747957470621\\
-0.734586090433201	0.9\\
-0.73	0.893521941452637\\
-0.727504595862501	0.89\\
-0.720340907145698	0.88\\
-0.72	0.87952529444756\\
-0.713149875217424	0.87\\
-0.71	0.865672194513716\\
-0.705862026098144	0.86\\
-0.7	0.852054794520548\\
-0.698479042729774	0.85\\
-0.691026907799391	0.84\\
-0.69	0.838628852047924\\
-0.683512691737671	0.83\\
-0.68	0.825375399889686\\
-0.675897647448816	0.82\\
-0.67	0.812346066648457\\
-0.668182069286655	0.81\\
-0.660373902201314	0.8\\
-0.66	0.799522314969393\\
-0.652499756188164	0.79\\
-0.65	0.786853448275862\\
-0.644516279269963	0.78\\
-0.64	0.7744004378763\\
-0.636422253002464	0.77\\
-0.63	0.762158910622113\\
-0.628215922548056	0.76\\
-0.62	0.750124986508365\\
-0.619894990393389	0.75\\
-0.611482512345427	0.74\\
-0.61	0.738248263984569\\
-0.602945823941065	0.73\\
-0.6	0.726575342465754\\
-0.594279319382024	0.72\\
-0.59	0.715106827473426\\
-0.585478726163832	0.71\\
-0.58	0.703840260444927\\
-0.57653909205934	0.7\\
-0.57	0.692773561199676\\
-0.567454744946349	0.69\\
-0.56	0.681905008077545\\
-0.558219244452443	0.68\\
-0.55	0.67123322147651\\
-0.54882532465789	0.67\\
-0.54	0.660757150508838\\
-0.539264826913701	0.66\\
-0.53	0.65047606255012\\
-0.529528621622873	0.65\\
-0.52	0.640389535504538\\
-0.5196065175917	0.64\\
-0.51	0.630497452654041\\
-0.509487157276443	0.63\\
-0.5	0.6208\\
-0.499157895918743	0.62\\
-0.49	0.611297666044278\\
-0.48860466216846	0.61\\
-0.48	0.601991243993593\\
-0.477811797320191	0.6\\
-0.47	0.592881836407378\\
-0.466761869720293	0.59\\
-0.46	0.58397086234601\\
-0.455435460211277	0.58\\
-0.45	0.575260067114094\\
-0.443810913639443	0.57\\
-0.44	0.566751534733441\\
-0.431864050421395	0.56\\
-0.43	0.558447703323426\\
-0.42	0.550342102482831\\
-0.419567376182642	0.55\\
-0.41	0.542400358185195\\
-0.406885404491694	0.54\\
-0.4	0.534666666666667\\
-0.39378265217847	0.53\\
-0.39	0.527144797532851\\
-0.380219091584908	0.52\\
-0.38	0.519838963842418\\
-0.37	0.512681023538746\\
-0.366121720500043	0.51\\
-0.36	0.505737666489078\\
-0.351451154871217	0.5\\
-0.35	0.499018456375839\\
-0.34	0.492462625197681\\
-0.336102818649176	0.49\\
-0.33	0.486109797926084\\
-0.32001251752127	0.48\\
-0.32	0.479992270531401\\
-0.31	0.474008696225917\\
-0.302996917785324	0.47\\
-0.3	0.468266666666667\\
-0.29	0.462701745865056\\
-0.284931502801207	0.46\\
-0.28	0.45734126394052\\
-0.27	0.452180829625752\\
-0.265582090739195	0.45\\
-0.26	0.447210174880763\\
-0.25	0.442442810457516\\
-0.244621543757249	0.44\\
-0.24	0.437872602013778\\
-0.23	0.43348938759487\\
-0.221585787987316	0.43\\
-0.22	0.429332766861391\\
-0.21	0.425327106852192\\
-0.2	0.421558441558442\\
-0.195620773590345	0.42\\
-0.19	0.417967524412774\\
-0.18	0.414566440146367\\
-0.17	0.411390559440559\\
-0.165317401827181	0.41\\
-0.16	0.408393041644702\\
-0.15	0.405580065359477\\
-0.14	0.402983705241308\\
-0.13	0.400599574851842\\
-0.127257168115515	0.4\\
-0.12	0.398382343667893\\
-0.11	0.396360133194045\\
-0.1	0.394545454545455\\
-0.0900000000000001	0.392935370896873\\
-0.0800000000000001	0.391527328872877\\
-0.0700000000000001	0.390319136118944\\
-0.0668480283562264	0.39\\
-0.0600000000000001	0.389290823282643\\
-0.05	0.388455882352941\\
-0.04	0.387821387584768\\
-0.03	0.387386344701901\\
-0.02	0.38715007800312\\
-0.01	0.387112223954274\\
0	0.387272727272727\\
0.00999999999999979	0.387631839438815\\
0.02	0.388190119604784\\
0.0299999999999998	0.388948437906795\\
0.04	0.389907981220657\\
0.0407934241153268	0.39\\
0.0499999999999998	0.391042993630573\\
0.0600000000000001	0.392374859478794\\
0.0699999999999998	0.393907959497565\\
0.0800000000000001	0.395644673185795\\
0.0899999999999999	0.397587735849057\\
0.1	0.39974025974026\\
0.101103393486668	0.4\\
0.11	0.402052163400356\\
0.12	0.4045681515617\\
0.13	0.407298879154857\\
0.13916170333632	0.41\\
0.14	0.410242589782499\\
0.15	0.413335987261146\\
0.16	0.416652080123267\\
0.169449866806806	0.42\\
0.17	0.420191605655138\\
0.18	0.423874477840041\\
0.19	0.427790935973258\\
0.195339023867306	0.43\\
0.2	0.431898734177215\\
0.21	0.436191084676388\\
0.218404176956184	0.44\\
0.22	0.440712859304085\\
0.23	0.445387133384341\\
0.239362675190559	0.45\\
0.24	0.450309813789633\\
0.25	0.455374203821656\\
0.258692420568332	0.46\\
0.26	0.460687267237041\\
0.27	0.466152244325427\\
0.276731008175603	0.47\\
0.28	0.471847503782148\\
0.29	0.477726004093118\\
0.293720972579487	0.48\\
0.3	0.48379746835443\\
0.309835822063957	0.49\\
0.31	0.490102519428428\\
0.32	0.496548955680081\\
0.325150117360854	0.5\\
0.33	0.503221143650593\\
0.339830418188308	0.51\\
0.34	0.510115973960941\\
0.35	0.517157643312102\\
0.353898680607853	0.52\\
0.36	0.524415579160344\\
0.367460296813102	0.53\\
0.37	0.531888206480784\\
0.38	0.539560471070148\\
0.380557769312616	0.54\\
0.39	0.547397034868924\\
0.393223448130223	0.55\\
0.4	0.555443037974683\\
0.405510156289461	0.56\\
0.41	0.563695376669186\\
0.417447724211826	0.57\\
0.42	0.572151329653788\\
0.429062869806415	0.58\\
0.43	0.580808535366159\\
0.44	0.589656412876852\\
0.440379473765151	0.59\\
0.45	0.598686305732484\\
0.451420319556406	0.6\\
0.46	0.60791743772242\\
0.462205107156074	0.61\\
0.47	0.617348629789394\\
0.472750600514633	0.62\\
0.48	0.626979016725798\\
0.483071915888753	0.63\\
0.49	0.636808040010129\\
0.493182687561309	0.64\\
0.5	0.646835443037975\\
0.503095212449192	0.65\\
0.51	0.657061268675614\\
0.512820576386737	0.66\\
0.52	0.667485859097821\\
0.522368764437971	0.67\\
0.53	0.678109857904085\\
0.531748757226282	0.68\\
0.54	0.688934214539909\\
0.540968614962384	0.69\\
0.55	0.699960191082802\\
0.550035550590481	0.7\\
0.558966921836736	0.71\\
0.56	0.711159740906826\\
0.567761372273471	0.72\\
0.57	0.722558097975506\\
0.576423924358552	0.73\\
0.58	0.73415835423984\\
0.584959303966306	0.74\\
0.59	0.745962950365331\\
0.593371637552724	0.75\\
0.6	0.757974683544304\\
0.601664483765561	0.76\\
0.609843128937886	0.77\\
0.61	0.770191846612062\\
0.617935110767516	0.78\\
0.62	0.782566949575637\\
0.625920576612402	0.79\\
0.63	0.79515371765888\\
0.633801417671138	0.8\\
0.64	0.807956297420334\\
0.641579066156908	0.81\\
0.649267362438065	0.82\\
0.65	0.820954968944099\\
0.656886065582298	0.83\\
0.66	0.834121982974462\\
0.664409968864907	0.84\\
0.67	0.847512989144155\\
0.671839177779126	0.85\\
0.679189397629855	0.86\\
0.68	0.861105613512171\\
0.686485016140643	0.87\\
0.69	0.874865617949361\\
0.693691539723924	0.88\\
0.7	0.888860759493671\\
0.700807885682381	0.89\\
0.707879399659785	0.9\\
0.71	0.903020940384136\\
0.714882373214091	0.91\\
0.72	0.917394654563793\\
0.721798392676103	0.92\\
0.728656676130938	0.93\\
0.73	0.931967905993534\\
0.735471456463642	0.94\\
0.74	0.946726522395572\\
0.742201055622812	0.95\\
0.74887052946278	0.96\\
0.75	0.961700310559006\\
0.755509177909032	0.97\\
0.76	0.976852340211374\\
0.762063077513019	0.98\\
0.768566168101767	0.99\\
0.77	0.992216600348172\\
0.775037815345859	1\\
0.78	1.00777286938981\\
0.781423870070727	1.01\\
0.787781427981841	1.02\\
0.79	1.0235198179097\\
0.794092539192064	1.03\\
0.8	1.03949367088608\\
0.800315961007252	1.04\\
0.806547633261669	1.05\\
0.81	1.05561768613688\\
0.812702186119667	1.06\\
0.818801126318845	1.07\\
0.82	1.07197496274218\\
0.82489025280867	1.08\\
0.83	1.08852283514264\\
0.830889836018204	1.09\\
0.836890851585298	1.1\\
0.84	1.10524867300951\\
0.842829378915187	1.11\\
0.848713833731697	1.12\\
0.85	1.12219720496894\\
0.854593877462958	1.13\\
0.86	1.13934729387962\\
0.860380044987501	1.14\\
0.866192158707023	1.15\\
0.87	1.15666087666416\\
0.871922732285885	1.16\\
0.877632375813363	1.17\\
0.88	1.17418951572641\\
0.883309195725035	1.18\\
0.888922066710812	1.19\\
0.89	1.19192925043435\\
0.894546754934663	1.2\\
0.9	1.20987341772152\\
0.90007056470819	1.21\\
0.905642179835537	1.22\\
0.91	1.22797857142857\\
0.911114959021377	1.23\\
0.916601735371147	1.24\\
0.92	1.24629081786252\\
0.922024490019478	1.25\\
0.927431221391631	1.26\\
0.93	1.2648075356161\\
0.932804776546589	1.27\\
0.938136008243323	1.28\\
0.94	1.2835264573991\\
0.943461015163452	1.29\\
0.948721068540332	1.3\\
0.95	1.30244565217391\\
0.953998010609644	1.31\\
0.959191005528873	1.32\\
0.96	1.32156351016361\\
0.96442020293855	1.33\\
0.969550078398454	1.34\\
0.97	1.34087873051225\\
0.974731691661879	1.35\\
0.979802224845253	1.36\\
0.98	1.36039031141869\\
0.984936257193006	1.37\\
0.989951081150544	1.38\\
0.99	1.38009754260311\\
0.995037379837063	1.39\\
1	1.4\\
1	1.4\\
1.00503825653932	1.41\\
1.00995206623707	1.42\\
1.01	1.42009754260311\\
1.01494181557111	1.43\\
1.01981011071464	1.44\\
1.02	1.44039031141869\\
1.02475072930381	1.45\\
1.02957672238331	1.46\\
1.03	1.46087873051225\\
1.03446742519506	1.47\\
1.03925425873103	1.48\\
1.04	1.48156351016361\\
1.04409409508755	1.49\\
1.04884485466732	1.5\\
1.05	1.50244565217391\\
1.05363270289822	1.51\\
1.05835042992514	1.52\\
1.06	1.5235264573991\\
1.06308499075445	1.53\\
1.06777269503443	1.54\\
1.07	1.5448075356161\\
1.07245248361326	1.55\\
1.07711315590273	1.56\\
1.08	1.56629081786252\\
1.08173649237889	1.57\\
1.08637311702033	1.58\\
1.09	1.58797857142857\\
1.09093811551389	1.59\\
1.09555368328886	1.6\\
1.1	1.60987341772152\\
1.100058239117	1.61\\
1.10465576045318	1.62\\
1.1091389502614	1.63\\
1.11	1.63192925043435\\
1.11368005409672	1.64\\
1.11814735029432	1.65\\
1.12	1.65418951572641\\
1.12262706713881	1.66\\
1.12708135877716	1.67\\
1.13	1.67666087666415\\
1.13149709574878	1.68\\
1.13594132585127	1.69\\
1.14	1.69934729387962\\
1.14029022356593	1.7\\
1.14472740368724	1.71\\
1.14905419041105	1.72\\
1.15	1.72219720496894\\
1.15343953993809	1.73\\
1.1577608071827	1.74\\
1.16	1.74524867300951\\
1.16207746939513	1.75\\
1.1663961911167	1.76\\
1.17	1.76852283514264\\
1.17064070366203	1.77\\
1.17495995879249	1.78\\
1.17917194191971	1.79\\
1.18	1.79197496274218\\
1.18345150908979	1.8\\
1.18766519323489	1.81\\
1.19	1.81561768613688\\
1.19187001001807	1.82\\
1.19608849595368	1.83\\
1.2	1.83949367088608\\
1.20021438306717	1.84\\
1.20444091725317	1.85\\
1.20856173546378	1.86\\
1.21	1.8635198179097\\
1.21272128108101	1.87\\
1.21685103361978	1.88\\
1.22	1.88777286938981\\
1.22092814904522	1.89\\
1.225070165419	1.9\\
1.2291092906414	1.91\\
1.23	1.91221660034817\\
1.23321760183822	1.92\\
1.23726973352751	1.93\\
1.24	1.93685234021137\\
1.24129153026947	1.94\\
1.24536012010966	1.95\\
1.24932794082049	1.96\\
1.25	1.96170031055901\\
1.25337855092339	1.97\\
1.257363460665	1.98\\
1.26	1.98672652239557\\
1.26132281558951	1.99\\
1.26532842941656	2\\
16	785\\
1.54790614885789	2\\
1.54465191877692	1.99\\
1.54147773038181	1.98\\
1.54	1.97529319213313\\
1.53823758382219	1.97\\
1.53495409168568	1.96\\
1.53175172000838	1.95\\
1.53	1.94445795655469\\
1.52850619454795	1.94\\
1.52519472847174	1.93\\
1.52196535564887	1.92\\
1.52	1.91382417804755\\
1.51871033278392	1.91\\
1.51537223422012	1.9\\
1.51211709105212	1.89\\
1.51	1.88339282460137\\
1.50884841336351	1.88\\
1.50548507076788	1.87\\
1.50220543099635	1.86\\
1.5	1.85316455696203\\
1.49891891845142	1.85\\
1.49553176054115	1.84\\
1.49222893475965	1.83\\
1.49	1.82313972412048\\
1.48892040111658	1.82\\
1.48551088978297	1.81\\
1.48218621903814	1.8\\
1.48	1.79331836115326\\
1.47885148818186	1.79\\
1.4754211111277	1.78\\
1.47207596027193	1.77\\
1.47	1.76370018944178\\
1.46871088231088	1.76\\
1.46526114548594	1.75\\
1.46189689634612	1.74\\
1.46	1.73428461926374\\
1.45849736330271	1.73\\
1.45502978321338	1.72\\
1.4516478276428	1.71\\
1.45	1.70507075471698\\
1.44820978857631	1.7\\
1.44472588454683	1.69\\
1.44132761742802	1.68\\
1.44	1.67605740090316\\
1.43784709283792	1.67\\
1.43434837930008	1.66\\
1.4309351915667	1.65\\
1.43	1.64724307326832\\
1.42740828693565	1.64\\
1.42389626582265	1.63\\
1.42046953756696	1.62\\
1.42	1.61862600896861\\
1.41689245591637	1.61\\
1.41336860923356	1.6\\
1.41	1.59020937420504\\
1.40992401894149	1.59\\
1.40629875631057	1.58\\
1.4027645389517	1.57\\
1.4	1.5620253164557\\
1.39925996469107	1.56\\
1.39562641267998	1.55\\
1.39208324555265	1.54\\
1.39	1.53403602367162\\
1.38851505486182	1.53\\
1.38487471347174	1.52\\
1.38132397698956	1.51\\
1.38	1.50623855783675\\
1.37768868893707	1.5\\
1.37404300623	1.49\\
1.37048603422217	1.48\\
1.37	1.47862977108734\\
1.36678032625572	1.47\\
1.3631306922222	1.46\\
1.36	1.45123689812468\\
1.35953504920955	1.45\\
1.35578947954929	1.44\\
1.35213722054214	1.43\\
1.35	1.42406446540881\\
1.34846130146396	1.42\\
1.34471570799182	1.41\\
1.34106208175525	1.4\\
1.34	1.39707358961558\\
1.33730294601238	1.39\\
1.33355860983935	1.38\\
1.33	1.37026683168317\\
1.32989748079076	1.37\\
1.32605969730609	1.36\\
1.32231781473675	1.35\\
1.32	1.34371132360342\\
1.31856372352106	1.34\\
1.31473129648136	1.33\\
1.31099297576837	1.32\\
1.31	1.31732761156819\\
1.30714347846581	1.31\\
1.30331750252723	1.3\\
1.3	1.29113924050633\\
1.29955242117764	1.29\\
1.29563661539043	1.28\\
1.29181808336002	1.27\\
1.29	1.26518392633425\\
1.28794825273649	1.26\\
1.2840430051375	1.25\\
1.2802328068789	1.24\\
1.28	1.2393883986118\\
1.27625636146621	1.23\\
1.27236251003343	1.22\\
1.27	1.21384193264639\\
1.26845582623935	1.21\\
1.26447670475951	1.2\\
1.26059497444941	1.19\\
1.26	1.18846219572777\\
1.25657298801811	1.18\\
1.25260921160234	1.17\\
1.25	1.16330974842767\\
1.24864880409213	1.16\\
1.2446016796944	1.15\\
1.24065377300558	1.14\\
1.24	1.13833800298063\\
1.23656977317491	1.13\\
1.23254186237158	1.12\\
1.23	1.1135906132194\\
1.22851074564089	1.11\\
1.22440184277463	1.1\\
1.22039343774561	1.09\\
1.22	1.08901655924641\\
1.21623094439335	1.08\\
1.21214496569674	1.07\\
1.21	1.06468279879729\\
1.20802627125312	1.06\\
1.20386203572538	1.05\\
1.2	1.04050632911392\\
1.19978480497144	1.04\\
1.19554175305613	1.03\\
1.19140391909655	1.02\\
1.19	1.01657966841187\\
1.18718111346395	1.01\\
1.18296807294339	1\\
1.18	0.992830598892804\\
1.17877700857001	0.99\\
1.17448849357692	0.98\\
1.17030502881997	0.97\\
1.17	0.969269881158702\\
1.16596210807852	0.96\\
1.16170673063014	0.95\\
1.16	0.945950274588118\\
1.15738575825656	0.94\\
1.15305846419772	0.93\\
1.15	0.922806603773585\\
1.1487562078121	0.92\\
1.14435712387122	0.91\\
1.14006567987742	0.9\\
1.14	0.899846959960455\\
1.13559953720962	0.89\\
1.13123986951521	0.88\\
1.13	0.877136862403583\\
1.12678247262224	0.87\\
1.12235493250544	0.86\\
1.12	0.854611116675013\\
1.11790264748505	0.85\\
1.11340771159405	0.84\\
1.11	0.832273243515487\\
1.10895673665056	0.83\\
1.10439500749687	0.82\\
1.1	0.810126582278481\\
1.09994138125689	0.81\\
1.09531358676923	0.8\\
1.09080031156278	0.79\\
1.09	0.788219560903001\\
1.08616018982886	0.78\\
1.08158438585434	0.77\\
1.08	0.766508420528152\\
1.07693153902848	0.76\\
1.07229407805613	0.75\\
1.07	0.744992510627656\\
1.06762434666722	0.74\\
1.06292620713065	0.73\\
1.06	0.723674059207225\\
1.05823532282465	0.72\\
1.05347758569329	0.71\\
1.05	0.702555031446541\\
1.04876118289728	0.7\\
1.0439450266404	0.69\\
1.04	0.681637115481594\\
1.03919865471544	0.68\\
1.03432534937947	0.67\\
1.03	0.660921710027784\\
1.02954448511857	0.66\\
1.02461538555166	0.65\\
1.02	0.640409914011128\\
1.01979544586882	0.64\\
1.01481198413846	0.63\\
1.01	0.620102518349784\\
1.00994833878685	0.62\\
1.00491201584778	0.61\\
1	0.6\\
1	0.6\\
0.994912376680093	0.59\\
0.99	0.580102518349785\\
0.989947303200122	0.58\\
0.984809990581505	0.57\\
0.98	0.560409914011128\\
0.979787161818343	0.56\\
0.974601811098983	0.55\\
0.97	0.540921710027785\\
0.969516530034282	0.54\\
0.964284821962051	0.53\\
0.96	0.521637115481594\\
0.959132402549289	0.52\\
0.95385603652542	0.51\\
0.95	0.502555031446541\\
0.948631813065756	0.5\\
0.943312496017877	0.49\\
0.94	0.483674059207225\\
0.938011831427611	0.48\\
0.932651266553908	0.47\\
0.93	0.464992510627657\\
0.927269559390183	0.46\\
0.921869434875834	0.45\\
0.92	0.446508420528151\\
0.91640212500043	0.44\\
0.910964102805214	0.43\\
0.91	0.428219560903002\\
0.90540667558075	0.42\\
0.9	0.410126582278481\\
0.899928462026571	0.41\\
0.894280369320738	0.4\\
0.89	0.392273243515488\\
0.888700961645699	0.39\\
0.88302036549128	0.38\\
0.88	0.374611116675012\\
0.877335291457923	0.37\\
0.871623813303699	0.36\\
0.87	0.357136862403583\\
0.865828663256365	0.35\\
0.860087839443343	0.34\\
0.86	0.339846959960455\\
0.854178262295873	0.33\\
0.85	0.322806603773585\\
0.84832083950123	0.32\\
0.842381231611293	0.31\\
0.84	0.305950274588118\\
0.836397955444404	0.3\\
0.830434655081564	0.29\\
0.83	0.289269881158703\\
0.824321284348706	0.28\\
0.82	0.272830598892803\\
0.818244507648911	0.27\\
0.812087873957029	0.26\\
0.81	0.256579668411867\\
0.805869366679684	0.25\\
0.8	0.240506329113924\\
0.799678099240395	0.24\\
0.793330294039872	0.23\\
0.79	0.224682798797294\\
0.786985454765409	0.22\\
0.78062410477258	0.21\\
0.78	0.209016559246405\\
0.774121638929629	0.2\\
0.77	0.193590613219402\\
0.76762824285234	0.19\\
0.761083274983507	0.18\\
0.76	0.178338002980626\\
0.754420316293109	0.17\\
0.75	0.163309748427673\\
0.747755191532817	0.16\\
0.741030193460787	0.15\\
0.74	0.14846219572777\\
0.734183407530081	0.14\\
0.73	0.133841932646394\\
0.727322488691943	0.13\\
0.720421424819688	0.12\\
0.72	0.1193883986118\\
0.713366559386973	0.11\\
0.71	0.105183926334252\\
0.706284898018551	0.1\\
0.7	0.0911392405063294\\
0.699171712319123	0.0899999999999999\\
0.691923753049905	0.0800000000000001\\
0.69	0.0773276115681874\\
0.684595470029283	0.0699999999999998\\
0.68	0.0637113236034224\\
0.677221807290759	0.0600000000000001\\
0.67	0.0502668316831682\\
0.669797228666374	0.0499999999999998\\
0.662205090827013	0.04\\
0.66	0.0370735896155769\\
0.654543327578853	0.0299999999999998\\
0.65	0.0240644654088049\\
0.646815589494131	0.02\\
0.64	0.0112368981246834\\
0.639015526118067	0.00999999999999979\\
0.631083027272593	0\\
0.63	-0.00137022891266496\\
0.623022396579907	-0.01\\
0.62	-0.0137614421632447\\
0.614873060046605	-0.02\\
0.61	-0.0259639763283809\\
0.606627804587072	-0.03\\
0.6	-0.0379746835443037\\
0.598279086926546	-0.04\\
0.59	-0.0497906257949633\\
0.589819023379308	-0.05\\
0.581182295155767	-0.0600000000000001\\
0.58	-0.0613739910313901\\
0.572415117278846	-0.0700000000000001\\
0.57	-0.0727569267316831\\
0.563517258744076	-0.0800000000000001\\
0.56	-0.0839425990968389\\
0.554479743841531	-0.0900000000000001\\
0.55	-0.094929245283019\\
0.545293174650805	-0.1\\
0.54	-0.105715380736258\\
0.535947703222345	-0.11\\
0.53	-0.116299810558222\\
0.526432998727728	-0.12\\
0.52	-0.126681638846737\\
0.516738208734455	-0.13\\
0.51	-0.136860275879524\\
0.506851913650662	-0.14\\
0.5	-0.146835443037975\\
0.496762073257445	-0.15\\
0.49	-0.156607175398633\\
0.486455964097107	-0.16\\
0.48	-0.166175821952453\\
0.475920106309065	-0.17\\
0.47	-0.175542043445314\\
0.465140178294841	-0.18\\
0.46	-0.184706807866868\\
0.454100917340962	-0.19\\
0.45	-0.193671383647799\\
0.442786004023349	-0.2\\
0.44	-0.202437330657301\\
0.431177927845536	-0.21\\
0.43	-0.21100648912228\\
0.42	-0.21936515074093\\
0.419222813890388	-0.22\\
0.41	-0.227501132027474\\
0.406863954819215	-0.23\\
0.4	-0.235443037974684\\
0.394133549740165	-0.24\\
0.39	-0.243193893225888\\
0.381006048202594	-0.25\\
0.38	-0.250756935403105\\
0.37	-0.258088019903036\\
0.367330798208195	-0.26\\
0.36	-0.265215205271161\\
0.35312789472439	-0.27\\
0.35	-0.272161949685535\\
0.34	-0.278905069124424\\
0.338330402193963	-0.28\\
0.33	-0.285417352119827\\
0.322791761922351	-0.29\\
0.32	-0.29175923502768\\
0.31	-0.297882625926873\\
0.306443008369468	-0.3\\
0.3	-0.30379746835443\\
0.29	-0.309542778206117\\
0.289175841102554	-0.31\\
0.28	-0.31503487544484\\
0.270714474168974	-0.32\\
0.27	-0.320377720532797\\
0.26	-0.325463881813551\\
0.250821412649368	-0.33\\
0.25	-0.330400943396226\\
0.24	-0.335081813550688\\
0.23	-0.33961406290281\\
0.229109124179895	-0.34\\
0.22	-0.343891001525165\\
0.21	-0.348000758159856\\
0.204908682149741	-0.35\\
0.2	-0.351898734177215\\
0.19	-0.355581424188187\\
0.18	-0.3590942694889\\
0.177268938140919	-0.36\\
0.17	-0.362370893627824\\
0.16	-0.365448847926267\\
0.15	-0.368346774193548\\
0.143890343285219	-0.37\\
0.14	-0.371033755701977\\
0.13	-0.373495011482521\\
0.12	-0.375768669748331\\
0.11	-0.377851446654611\\
0.1	-0.37974025974026\\
0.0984584434903929	-0.38\\
0.0899999999999999	-0.381395815314169\\
0.0800000000000001	-0.382850076647931\\
0.0699999999999998	-0.384107450628366\\
0.0600000000000001	-0.385166031909418\\
0.0499999999999998	-0.386024193548387\\
0.04	-0.386680600103466\\
0.0299999999999998	-0.38713421870951\\
0.02	-0.387384327970939\\
0.00999999999999979	-0.387430524539081\\
0	-0.387272727272727\\
-0.01	-0.386911178914568\\
-0.02	-0.386346445251687\\
-0.03	-0.385579411764706\\
-0.04	-0.384611277806518\\
-0.05	-0.383443548387097\\
-0.0600000000000001	-0.38207802367473\\
-0.0700000000000001	-0.380516786355476\\
-0.0729373519443134	-0.38\\
-0.0800000000000001	-0.378729732564237\\
-0.0900000000000001	-0.376733737927434\\
-0.1	-0.374545454545455\\
-0.11	-0.372168160681994\\
-0.1184523276014	-0.37\\
-0.12	-0.369594939377965\\
-0.13	-0.366778462948416\\
-0.14	-0.363780343213729\\
-0.15	-0.360604838709677\\
-0.151794526264208	-0.36\\
-0.16	-0.357184235417761\\
-0.17	-0.353576178692368\\
-0.179469710817195	-0.35\\
-0.18	-0.349796290408055\\
-0.19	-0.345756743636841\\
-0.2	-0.341558441558442\\
-0.203556809660295	-0.34\\
-0.21	-0.337132933262991\\
-0.22	-0.332516223265519\\
-0.225256112344552	-0.33\\
-0.23	-0.327695938078857\\
-0.24	-0.322668897020387\\
-0.24513966311731	-0.32\\
-0.25	-0.317442052980132\\
-0.26	-0.312015473078934\\
-0.263600294141678	-0.31\\
-0.27	-0.306372836729293\\
-0.28	-0.300560041731873\\
-0.280932906098952	-0.3\\
-0.29	-0.29449510683197\\
-0.297238853996971	-0.29\\
-0.3	-0.288266666666667\\
-0.31	-0.28182076882708\\
-0.312752705021273	-0.28\\
-0.32	-0.275159300476947\\
-0.327563907583836	-0.27\\
-0.33	-0.268322800213961\\
-0.34	-0.261272517078297\\
-0.341761198860341	-0.26\\
-0.35	-0.253998344370861\\
-0.355380143327909	-0.25\\
-0.36	-0.246540096102509\\
-0.368562486347631	-0.24\\
-0.37	-0.238894044097876\\
-0.38	-0.231028782287823\\
-0.381279307773617	-0.23\\
-0.39	-0.222944298594537\\
-0.393569741776905	-0.22\\
-0.4	-0.214666666666667\\
-0.405519973630297	-0.21\\
-0.41	-0.206193232377379\\
-0.41715125302715	-0.2\\
-0.42	-0.197521593968767\\
-0.428483142399466	-0.19\\
-0.43	-0.18864962151933\\
-0.439533727158277	-0.18\\
-0.44	-0.179575474769398\\
-0.45	-0.170289735099338\\
-0.450305802488638	-0.17\\
-0.46	-0.160793202336697\\
-0.460819393108178	-0.16\\
-0.47	-0.151096288906624\\
-0.471108922480777	-0.15\\
-0.48	-0.141198508257858\\
-0.48118761754555	-0.14\\
-0.49	-0.131099693415089\\
-0.491067949602524	-0.13\\
-0.5	-0.1208\\
-0.500761710344407	-0.12\\
-0.51	-0.110299906691549\\
-0.510280076931216	-0.11\\
-0.519616514312361	-0.1\\
-0.52	-0.0995892720306513\\
-0.528777739505114	-0.0900000000000001\\
-0.53	-0.088666561902159\\
-0.537786844286197	-0.0800000000000001\\
-0.54	-0.0775415166393889\\
-0.546653406636513	-0.0700000000000001\\
-0.55	-0.0662159863945578\\
-0.555386590675701	-0.0600000000000001\\
-0.56	-0.0546921323928377\\
-0.563995172123053	-0.05\\
-0.57	-0.0429724114625574\\
-0.572487559318774	-0.04\\
-0.58	-0.0310595584275713\\
-0.580871810493318	-0.03\\
-0.5891145769959	-0.02\\
-0.59	-0.0189278297989534\\
-0.597217086200198	-0.01\\
-0.6	-0.00657534246575333\\
-0.605229094244619	0\\
-0.61	0.0059648733315174\\
-0.613157755050922	0.00999999999999979\\
-0.62	0.0186889009204116\\
-0.621009920500456	0.02\\
-0.628731850662384	0.0299999999999998\\
-0.63	0.03163665377176\\
-0.63633627290379	0.04\\
-0.64	0.0447999999999998\\
-0.643879878318935	0.0499999999999998\\
-0.65	0.0581377551020407\\
-0.651368734943916	0.0600000000000001\\
-0.658747847374901	0.0699999999999998\\
-0.66	0.0716907376594564\\
-0.666011365403231	0.0800000000000001\\
-0.67	0.0854633278373178\\
-0.673234654740853	0.0899999999999999\\
-0.68	0.0993981491562329\\
-0.680423016036233	0.1\\
-0.687455701859937	0.11\\
-0.69	0.113586554970922\\
-0.694440072125685	0.12\\
-0.7	0.127945205479452\\
-0.701402943000296	0.13\\
-0.708261438008955	0.14\\
-0.71	0.14251992481203\\
-0.71503261434694	0.15\\
-0.72	0.15729576224546\\
-0.721795110902058	0.16\\
-0.728475060630368	0.17\\
-0.73	0.172270759988824\\
-0.735057974923092	0.18\\
-0.74	0.18745416436845\\
-0.741644392468842	0.19\\
-0.748141496080787	0.2\\
-0.75	0.202840909090909\\
-0.754560332700492	0.21\\
-0.76	0.218419415333701\\
-0.760994167212361	0.22\\
-0.767304433049855	0.23\\
-0.77	0.234226613579212\\
-0.773582606207982	0.24\\
-0.779880126323479	0.25\\
-0.78	0.250190379173741\\
-0.786006488640547	0.26\\
-0.79	0.266418560289613\\
-0.79216656741511	0.27\\
-0.798269026214732	0.28\\
-0.8	0.282816901408451\\
-0.804289242999072	0.29\\
-0.81	0.299402118526724\\
-0.810352840233543	0.3\\
-0.816257615507638	0.31\\
-0.82	0.316236373810856\\
-0.822193169172825	0.32\\
-0.828074628553716	0.33\\
-0.83	0.333246976547047\\
-0.833886276693918	0.34\\
-0.83974312064621	0.35\\
-0.84	0.350438334284085\\
-0.845435067919704	0.36\\
-0.85	0.367875874125874\\
-0.85119408817265	0.37\\
-0.856842378267273	0.38\\
-0.86	0.385510659898477\\
-0.862493521139997	0.39\\
-0.868110992598078	0.4\\
-0.87	0.403334200056835\\
-0.873658645766314	0.41\\
-0.879243663185745	0.42\\
-0.88	0.421349971379508\\
-0.884692296645332	0.43\\
-0.89	0.439573522822739\\
-0.890228936028844	0.44\\
-0.895597296276471	0.45\\
-0.9	0.458028169014084\\
-0.901046446529199	0.46\\
-0.90637646822022	0.47\\
-0.91	0.47667885868026\\
-0.911742702485817	0.48\\
-0.917032648836916	0.49\\
-0.92	0.495527831531019\\
-0.922320605727238	0.5\\
-0.927568697553144	0.51\\
-0.93	0.514577036296085\\
-0.932783077086672	0.52\\
-0.937987505608149	0.53\\
-0.94	0.533828112449799\\
-0.9431330633036	0.54\\
-0.948292003243416	0.55\\
-0.95	0.553282374100719\\
-0.953373542467074	0.56\\
-0.958485165311641	0.57\\
-0.96	0.572940796306982\\
-0.963507527999283	0.58\\
-0.968570015296437	0.59\\
-0.97	0.592804004047413\\
-0.973538071192259	0.6\\
-0.978549627750752	0.61\\
-0.98	0.612872264041691\\
-0.983468262325798	0.62\\
-0.988427129179843	0.63\\
-0.99	0.633145479571139\\
-0.99330123041046	0.64\\
-0.998205697413118	0.65\\
-1	0.653623188405797\\
-1.00304014161532	0.66\\
-1.00788855952771	0.67\\
-1.01	0.674304563894523\\
-1.01268819645562	0.68\\
-1.01747898840485	0.69\\
-1.02	0.695188419224088\\
-1.02224862583	0.7\\
-1.02698029801728	0.71\\
-1.03	0.716273214801966\\
-1.03172468601054	0.72\\
-1.03639583756171	0.73\\
-1.04	0.737557068667052\\
-1.04111965270032	0.74\\
-1.04572898456417	0.75\\
-1.05	0.759037769784173\\
-1.05043681428322	0.76\\
-1.05498313709765	0.77\\
-1.05965645654448	0.78\\
-1.06	0.780733845245127\\
-1.0641617052603	0.79\\
-1.068768083647	0.8\\
-1.07	0.802655207413945\\
-1.07326810206867	0.81\\
-1.07780645813058	0.82\\
-1.08	0.824770005858231\\
-1.08230573392348	0.83\\
-1.08677512762685	0.84\\
-1.09	0.847074628388225\\
-1.0912779908056	0.85\\
-1.0956776269386	0.86\\
-1.1	0.869565217391305\\
-1.1001882363571	0.87\\
-1.10451746678533	0.88\\
-1.10896806770028	0.89\\
-1.11	0.892304078908336\\
-1.11329812261036	0.9\\
-1.11767409941809	0.91\\
-1.12	0.915237713612257\\
-1.12202302360874	0.92\\
-1.12632396284431	0.93\\
-1.13	0.938348362094179\\
-1.13069554212304	0.94\\
-1.13492119326767	0.95\\
-1.13926667123301	0.96\\
-1.14	0.961679856545129\\
-1.14346925588927	0.97\\
-1.14773508458998	0.98\\
-1.15	0.985231481481481\\
-1.15197153516667	0.99\\
-1.15615776743858	1\\
-1.16	1.00894797416324\\
-1.16043132498939	1.01\\
-1.16453818073733	1.02\\
-1.16876157076042	1.03\\
-1.17	1.03290837076969\\
-1.17287967869728	1.04\\
-1.17701934518686	1.05\\
-1.18	1.05705655008892\\
-1.18118549997165	1.06\\
-1.18524201726611	1.07\\
-1.18941529565462	1.08\\
-1.19	1.08139566596195\\
-1.19343286063988	1.09\\
-1.19751857809562	1.1\\
-1.2	1.10597014925373\\
-1.20159500845059	1.11\\
-1.20559413005501	1.12\\
-1.20970946158843	1.13\\
-1.21	1.13070466423579\\
-1.21364508854764	1.14\\
-1.21766941636629	1.15\\
-1.22	1.15569670065987\\
-1.22167442674704	1.16\\
-1.22560913325945	1.17\\
-1.22965871441525	1.18\\
-1.23	1.18084082037206\\
-1.23353155361925	1.19\\
-1.23748717933191	1.2\\
-1.24	1.20623836440168\\
-1.24143943477403	1.21\\
-1.2453029198442	1.22\\
-1.24927917576151	1.23\\
-1.25	1.2318034351145\\
-1.2531086227332	1.24\\
-1.25698854579208	1.25\\
-1.26	1.25759134095009\\
-1.26090677517664	1.26\\
-1.26469261491583	1.27\\
-1.26858839031916	1.28\\
-1.27	1.28358555962184\\
-1.27239375979833	1.29\\
-1.27619146523356	1.3\\
-1.28	1.30974589082184\\
-1.28009414771701	1.31\\
-1.28379645660847	1.32\\
-1.28760519174389	1.33\\
-1.29	1.33617412640535\\
-1.2914053819052	1.34\\
-1.29511499587073	1.35\\
-1.29893378761879	1.36\\
-1.3	1.36276923076923\\
-1.30263361240286	1.37\\
-1.30634949587889	1.38\\
-1.31	1.38955030202356\\
-1.31016265966773	1.39\\
-1.31377908000146	1.4\\
-1.31750007388927	1.41\\
-1.32	1.41658790470373\\
-1.32122394265508	1.42\\
-1.32484203459064	1.43\\
-1.32856685732896	1.44\\
-1.33	1.44380396727385\\
-1.33220427092973	1.45\\
-1.33582274549297	1.46\\
-1.33954999518702	1.47\\
-1.34	1.47120299438553\\
-1.34310404696485	1.48\\
-1.34672151181617	1.49\\
-1.35	1.49882633587786\\
-1.35041142825949	1.5\\
-1.35392370904116	1.51\\
-1.35753867271635	1.52\\
-1.36	1.5266730745533\\
-1.36115694661314	1.53\\
-1.36466374027794	1.54\\
-1.36827461745688	1.55\\
-1.37	1.55471048462255\\
-1.37182489449781	1.56\\
-1.37532467733252	1.57\\
-1.37892979513973	1.58\\
-1.38	1.58294201628053\\
-1.38241594037097	1.59\\
-1.38590711865717	1.6\\
-1.38950472398108	1.61\\
-1.39	1.61137087669505\\
-1.39293081463257	1.62\\
-1.3964117322004	1.63\\
-1.4	1.64\\
-1.4	1.64\\
-1.40337031586279	1.65\\
-1.4068392624407	1.66\\
-1.41	1.66886816774992\\
-1.41037918743698	1.67\\
-1.41373531634651	1.68\\
-1.4171905366442	1.69\\
-1.42	1.69793553204729\\
-1.42068644556578	1.7\\
-1.42402676679787	1.71\\
-1.42746647024387	1.72\\
-1.43	1.72720382932166\\
-1.43092282813068	1.73\\
-1.43424570020652	1.74\\
-1.43766807124704	1.75\\
-1.44	1.75667445072191\\
-1.44108946447585	1.76\\
-1.44439323473728	1.77\\
-1.44779644358978	1.78\\
-1.45	1.78634842519685\\
-1.4511875611706	1.79\\
-1.45447057562802	1.8\\
-1.45785278937069	1.81\\
-1.46	1.81622640555906\\
-1.46121840270253	1.82\\
-1.46447901604447	1.83\\
-1.46783840991364	1.84\\
-1.47	1.8463086578031\\
-1.47118335121915	1.85\\
-1.47441993686694	1.86\\
-1.47775470562678	1.87\\
-1.48	1.87659505389981\\
-1.48108384531346	1.88\\
-1.48429480540086	1.89\\
-1.48760317464578	1.9\\
-1.49	1.90708506823231\\
-1.49092139786397	1.91\\
-1.49410517302093	1.92\\
-1.49738541026915	1.93\\
-1.5	1.93777777777778\\
-1.50069759295551	1.94\\
-1.50385267177651	1.95\\
-1.50710309721543	1.96\\
-1.51	1.96867186607426\\
-1.51041408192192	1.97\\
-1.513539010004	1.98\\
-1.51675800675222	1.99\\
-1.52	1.99976563094483\\
-1.52007257856664	2\\
32	531\\
-1.21640903993203	2\\
-1.21220662786312	1.99\\
-1.21	1.9848053273181\\
-1.20799801969032	1.98\\
-1.20379217417578	1.97\\
-1.2	1.96114285714286\\
-1.19951977352519	1.96\\
-1.19530807773577	1.95\\
-1.19101206256104	1.94\\
-1.19	1.9376563788784\\
-1.18675464138166	1.93\\
-1.18245142453732	1.92\\
-1.18	1.91436877152698\\
-1.17813199315182	1.91\\
-1.17381915330696	1.9\\
-1.17	1.8912980326934\\
-1.16944009281294	1.89\\
-1.16511526027964	1.88\\
-1.1607038700149	1.87\\
-1.16	1.86841031959954\\
-1.15633958426991	1.86\\
-1.15191458218423	1.85\\
-1.15	1.84571172248804\\
-1.14749179620379	1.84\\
-1.1430507680093	1.83\\
-1.14	1.82322405786068\\
-1.13857140266799	1.82\\
-1.13411196258026	1.81\\
-1.13	1.80094431925772\\
-1.12957774837801	1.8\\
-1.12509753029142	1.79\\
-1.12052786989866	1.78\\
-1.12	1.77884809834806\\
-1.11600666665222	1.77\\
-1.11141416324663	1.76\\
-1.11	1.75694070567986\\
-1.10683839904803	1.75\\
-1.10222058491095	1.74\\
-1.1	1.7352380952381\\
-1.09759158649204	1.73\\
-1.09294599084497	1.72\\
-1.09	1.71373827101917\\
-1.08826491839669	1.71\\
-1.08358905899548	1.7\\
-1.08	1.69243950056754\\
-1.07885691238163	1.69\\
-1.07414828721244	1.68\\
-1.07	1.67134030371628\\
-1.0693659111243	1.67\\
-1.06462199013371	1.66\\
-1.06	1.65043944298081\\
-1.05979007824871	1.65\\
-1.05500829504235	1.64\\
-1.05013139889246	1.63\\
-1.05	1.62973086124402\\
-1.04530513668256	1.62\\
-1.04038724541942	1.61\\
-1.04	1.60921421474971\\
-1.03551025100951	1.6\\
-1.03054857671725	1.59\\
-1.03	1.58889733829422\\
-1.02562116783695	1.58\\
-1.02061285836363	1.57\\
-1.02	1.56877987037743\\
-1.01563520233501	1.56\\
-1.01057733159089	1.55\\
-1.01	1.54886167841494\\
-1.00554944531859	1.54\\
-1.0004390031896	1.53\\
-1	1.52914285714286\\
-0.995360752254022	1.52\\
-0.990194634050439	1.51\\
-0.99	1.50962372832921\\
-0.985065730898186	1.5\\
-0.98	1.49029913954358\\
-0.979845036131911	1.49\\
-0.974660727469364	1.48\\
-0.97	1.47116475379142\\
-0.96939018514661	1.47\\
-0.964141811233092	1.46\\
-0.96	1.45222857142857\\
-0.958820637650562	1.45\\
-0.953504757368397	1.44\\
-0.95	1.43349178403756\\
-0.948132005137708	1.43\\
-0.942745027959965	1.42\\
-0.94	1.41495581482875\\
-0.937319579339493	1.41\\
-0.931857750939638	1.4\\
-0.93	1.39662232597623\\
-0.926378310064763	1.39\\
-0.920837696775602	1.38\\
-0.92	1.37849322739311\\
-0.915302780705861	1.37\\
-0.91	1.36056005773887\\
-0.909686798716135	1.36\\
-0.904087181166694	1.35\\
-0.9	1.34280373831776\\
-0.898413991555034	1.34\\
-0.892725277934798	1.33\\
-0.89	1.32525482266842\\
-0.886992802278786	1.32\\
-0.881210380980791	1.31\\
-0.88	1.30791617037316\\
-0.87541631177618	1.3\\
-0.87	1.29077624047575\\
-0.869545280610509	1.29\\
-0.863677091683749	1.28\\
-0.86	1.27381008591707\\
-0.85774108871613	1.27\\
-0.851767161857635	1.26\\
-0.85	1.25705985915493\\
-0.84576315598179	1.25\\
-0.84	1.24051968854283\\
-0.839684339431953	1.24\\
-0.833602607470656	1.23\\
-0.83	1.22414455325499\\
-0.827451012559117	1.22\\
-0.821249892566163	1.21\\
-0.82	1.20799309049944\\
-0.815021625556572	1.2\\
-0.81	1.19203172202635\\
-0.808718655992847	1.19\\
-0.802385526059189	1.18\\
-0.8	1.17626168224299\\
-0.795999476946538	1.17\\
-0.79	1.16071195230172\\
-0.789539270492171	1.16\\
-0.783057145905412	1.15\\
-0.78	1.14532746925084\\
-0.776506268007345	1.14\\
-0.77	1.1301794779561\\
-0.769880739258474	1.13\\
-0.763231509617216	1.12\\
-0.76	1.11518630442873\\
-0.756506381189171	1.11\\
-0.75	1.10043202764977\\
-0.749705012614447	1.1\\
-0.74286917244241	1.09\\
-0.74	1.08583758838854\\
-0.735957749337216	1.08\\
-0.73	1.07147075262867\\
-0.728967311372534	1.07\\
-0.721923655960109	1.06\\
-0.72	1.05728423406634\\
-0.714810945719454	1.05\\
-0.71	1.04330023109632\\
-0.707615133809257	1.04\\
-0.700340409539909	1.03\\
-0.7	1.02953271028037\\
-0.693008052392994	1.02\\
-0.69	1.0159285488959\\
-0.685587024458633	1.01\\
-0.68	1.00254537412459\\
-0.678076762669322	1\\
-0.670481257694723	0.99\\
-0.67	0.989367459429211\\
-0.662811049913572	0.98\\
-0.66	0.976359510567297\\
-0.655043567027335	0.97\\
-0.65	0.963565668202765\\
-0.647177106599022	0.96\\
-0.64	0.950982618261826\\
-0.639209583752752	0.95\\
-0.631147270684518	0.94\\
-0.63	0.938581481137335\\
-0.622981559798613	0.93\\
-0.62	0.926371291157973\\
-0.614702696467639	0.92\\
-0.61	0.914367830171302\\
-0.606307138980533	0.91\\
-0.6	0.90256880733945\\
-0.597790859048128	0.9\\
-0.59	0.890972197842384\\
-0.589149311819269	0.89\\
-0.580379053263603	0.88\\
-0.58	0.879568362421092\\
-0.571474774711436	0.87\\
-0.57	0.868349370486947\\
-0.56242607669666	0.86\\
-0.56	0.857331954192833\\
-0.55322624269536	0.85\\
-0.55	0.846514976958525\\
-0.543867823455776	0.84\\
-0.54	0.835897534044903\\
-0.534342573427191	0.83\\
-0.53	0.825478946884764\\
-0.524641378349593	0.82\\
-0.52	0.815258758721998\\
-0.514754172829556	0.81\\
-0.51	0.805236731510369\\
-0.504669846520824	0.8\\
-0.5	0.795412844036697\\
-0.494376137283255	0.79\\
-0.49	0.7857872912461\\
-0.48385950940255	0.78\\
-0.48	0.776360484759456\\
-0.473105014605405	0.77\\
-0.47	0.767133054585554\\
-0.462096133187384	0.76\\
-0.46	0.758105852042694\\
-0.450814592067933	0.75\\
-0.45	0.749279953917051\\
-0.44	0.740644758795793\\
-0.439235115990125	0.74\\
-0.43	0.732196864206508\\
-0.427330457976905	0.73\\
-0.42	0.72395100255195\\
-0.415079214270715	0.72\\
-0.41	0.715909115011885\\
-0.402452299848541	0.71\\
-0.4	0.708073394495413\\
-0.39	0.700438225718936\\
-0.389410470866611	0.7\\
-0.38	0.692975517617145\\
-0.375886331251373	0.69\\
-0.37	0.685722851669403\\
-0.361862832461039	0.68\\
-0.36	0.678683388338834\\
-0.35	0.671826923076923\\
-0.347247450610913	0.67\\
-0.34	0.665162577357117\\
-0.331979697010399	0.66\\
-0.33	0.658717991210401\\
-0.32	0.652452262034021\\
-0.315940597173647	0.65\\
-0.31	0.646387128803061\\
-0.3	0.640540540540541\\
-0.299042422288579	0.64\\
-0.29	0.634857678344527\\
-0.28107782454678	0.63\\
-0.28	0.629408562019759\\
-0.27	0.624122975910161\\
-0.261832967072394	0.62\\
-0.26	0.619066715381805\\
-0.25	0.614179864253394\\
-0.241014996566488	0.61\\
-0.24	0.609523419802704\\
-0.23	0.605028608947654\\
-0.22	0.600766654689184\\
-0.218111545057255	0.6\\
-0.21	0.596672889816664\\
-0.2	0.592792792792793\\
-0.192376193064303	0.59\\
-0.19	0.589119867006741\\
-0.18	0.585619109663409\\
-0.17	0.582337520230174\\
-0.162387479320089	0.58\\
-0.16	0.57925795413178\\
-0.15	0.576351809954751\\
-0.14	0.573658480374505\\
-0.13	0.571174870094965\\
-0.124850731324236	0.57\\
-0.12	0.568878169269887\\
-0.11	0.566768484355218\\
-0.1	0.564864864864865\\
-0.0900000000000001	0.563165191237206\\
-0.0800000000000001	0.561667597565342\\
-0.0700000000000001	0.56037046063203\\
-0.0666314379642033	0.56\\
-0.0600000000000001	0.559259194776932\\
-0.05	0.558342760180995\\
-0.04	0.557626888326708\\
-0.03	0.557110801299404\\
-0.02	0.556793941579517\\
-0.01	0.556675968642999\\
0	0.556756756756757\\
0.00999999999999979	0.557036393944855\\
0.02	0.557515182113235\\
0.0299999999999998	0.558193638332431\\
0.04	0.559072497289483\\
0.0485882323414017	0.56\\
0.0499999999999998	0.56015\\
0.0600000000000001	0.561409903811899\\
0.0699999999999998	0.562870014283164\\
0.0800000000000001	0.564531901181525\\
0.0899999999999999	0.566397369366134\\
0.1	0.568468468468468\\
0.10672843635097	0.57\\
0.11	0.57073422455143\\
0.12	0.57317930788441\\
0.13	0.57583353341695\\
0.14	0.578699891969752\\
0.1442314623005	0.58\\
0.15	0.58175\\
0.16	0.584991633893457\\
0.17	0.588451528502068\\
0.174220637429695	0.59\\
0.18	0.592095840739424\\
0.19	0.595934684201109\\
0.2	0.6\\
0.2	0.6\\
0.21	0.604220172936352\\
0.22	0.608671793029105\\
0.22284932985171	0.61\\
0.23	0.613300649350649\\
0.24	0.618143379978472\\
0.243670756990766	0.62\\
0.25	0.623172222222222\\
0.26	0.628412486544672\\
0.262910621921435	0.63\\
0.27	0.633834362213129\\
0.28	0.639480273086597\\
0.280888349775441	0.64\\
0.29	0.645289882331966\\
0.297788853166583	0.65\\
0.3	0.651327433628318\\
0.31	0.657544990159241\\
0.313810106283279	0.66\\
0.32	0.663962175613224\\
0.329092838835315	0.67\\
0.33	0.670598648648649\\
0.34	0.677404933857705\\
0.343688159845328	0.68\\
0.35	0.684416666666667\\
0.357719282937758	0.69\\
0.36	0.691641174389812\\
0.37	0.699058761870633\\
0.371233798185609	0.7\\
0.38	0.706657723867285\\
0.384271006787137	0.71\\
0.39	0.714464904956476\\
0.396899708904811	0.72\\
0.4	0.722477876106195\\
0.40915219453474	0.73\\
0.41	0.730694469924149\\
0.42	0.739096885069817\\
0.421048384499088	0.74\\
0.43	0.747690582038922\\
0.432621070815244	0.75\\
0.44	0.756486498040613\\
0.443900606206822	0.76\\
0.45	0.765483333333333\\
0.454907196642295	0.77\\
0.46	0.774680014199503\\
0.465659115904746	0.78\\
0.47	0.784075686934941\\
0.476172909172351	0.79\\
0.48	0.793669713071201\\
0.486463570889264	0.8\\
0.49	0.803461665781554\\
0.496544700500841	0.81\\
0.5	0.813451327433628\\
0.506428639065797	0.82\\
0.51	0.823638688263409\\
0.516126589292336	0.83\\
0.52	0.834023946156571\\
0.525648721157508	0.84\\
0.53	0.844607507534125\\
0.535004264944786	0.85\\
0.54	0.855389989350373\\
0.544201593262646	0.86\\
0.55	0.866372222222222\\
0.553248293377501	0.87\\
0.56	0.877555254720342\\
0.562151231000121	0.88\\
0.57	0.888940358864488\\
0.570916606499589	0.89\\
0.5795515646754	0.9\\
0.58	0.90051973267675\\
0.588064540500511	0.91\\
0.59	0.912282051508202\\
0.596457724043419	0.92\\
0.6	0.92424778761062\\
0.604735334850098	0.93\\
0.61	0.936419070882928\\
0.612901119382846	0.94\\
0.62	0.948798287549055\\
0.620958378234196	0.95\\
0.628917173058443	0.96\\
0.63	0.961363659566977\\
0.636782074099161	0.97\\
0.64	0.974117297488504\\
0.64454975784681	0.98\\
0.65	0.987083333333333\\
0.652222182340516	0.99\\
0.659802607019349	1\\
0.66	1.00026048472076\\
0.667311787410633	1.01\\
0.67	1.01360165165165\\
0.674734019715797	1.02\\
0.68	1.02716160682545\\
0.682070166682331	1.03\\
0.689327588677551	1.04\\
0.69	1.0409279223062\\
0.696523873719284	1.05\\
0.7	1.05486725663717\\
0.703639971658428	1.06\\
0.71	1.06903386521662\\
0.710675807884768	1.07\\
0.717658897584733	1.08\\
0.72	1.08337225555948\\
0.724573711341776	1.09\\
0.73	1.09792616082548\\
0.731412149863868	1.1\\
0.738196677507565	1.11\\
0.74	1.11267028551287\\
0.744925683282099	1.12\\
0.75	1.12761666666667\\
0.751580973248606	1.13\\
0.758186576014006	1.14\\
0.76	1.14275840676771\\
0.764742378818526	1.15\\
0.77	1.15810406511297\\
0.771226015755097	1.16\\
0.777670825127085	1.17\\
0.78	1.17363699258736\\
0.784063489201031	1.18\\
0.79	1.18939043501515\\
0.79038451265505	1.19\\
0.796685414331495	1.2\\
0.8	1.20530973451327\\
0.802922720512369	1.21\\
0.80910283439506	1.22\\
0.81	1.22145525575672\\
0.815260808043094	1.23\\
0.82	1.23778371845005\\
0.82134844851313	1.24\\
0.82740867689676	1.25\\
0.83	1.25430824059353\\
0.833422519166067	1.26\\
0.839375541738311	1.27\\
0.84	1.27105079030558\\
0.845319050882276	1.28\\
0.85	1.28797222222222\\
0.851191558775644	1.29\\
0.857046331699602	1.3\\
0.86	1.30509006013442\\
0.862852593694367	1.31\\
0.868612088613825	1.32\\
0.87	1.32241981165288\\
0.874354806659167	1.33\\
0.88	1.33995775954335\\
0.88002400168255	1.34\\
0.885705162066021	1.35\\
0.89	1.35766263101794\\
0.891313641925356	1.36\\
0.896910160715783	1.37\\
0.9	1.37557522123894\\
0.902459977397728	1.38\\
0.907975873225187	1.39\\
0.91	1.39369323513847\\
0.913468871354661	1.4\\
0.918907970342689	1.41\\
0.92	1.41201463243053\\
0.924345799066326	1.42\\
0.92971175047435	1.43\\
0.93	1.43053761620768\\
0.935095874041484	1.44\\
0.94	1.44924745279658\\
0.940401144907521	1.45\\
0.945723871836714	1.46\\
0.95	1.46815\\
0.950976072906374	1.47\\
0.95623425167556	1.48\\
0.96	1.48725367412141\\
0.961434343387525	1.49\\
0.966631176075498	1.5\\
0.97	1.5065575163978\\
0.971779959806871	1.51\\
0.976918528657588	1.52\\
0.98	1.52606078639745\\
0.98201665219668	1.53\\
0.987099930293103	1.54\\
0.99	1.54576295804567\\
0.992147892427024	1.55\\
0.997178753723274	1.56\\
1	1.56566371681416\\
1.0021769079821	1.57\\
1.00715813677192	1.58\\
1.01	1.58576295804567\\
1.01210669436272	1.59\\
1.01704099425609	1.6\\
1.02	1.60606078639745\\
1.02194002621209	1.61\\
1.02683002868669	1.62\\
1.03	1.6265575163978\\
1.031679467249	1.63\\
1.03652773983878	1.64\\
1.04	1.64725367412141\\
1.04132737908088	1.65\\
1.04613643326053	1.66\\
1.05	1.66815\\
1.05088592895797	1.67\\
1.05565822777929	1.68\\
1.06	1.68924745279658\\
1.06035709651994	1.69\\
1.06509506205385	1.7\\
1.06974994237975	1.71\\
1.07	1.71053761620768\\
1.07444870021297	1.72\\
1.07907177888384	1.73\\
1.08	1.73201463243053\\
1.08372073661158	1.74\\
1.0883142789856	1.75\\
1.09	1.75369323513847\\
1.09291259972766	1.76\\
1.09747885136468	1.77\\
1.1	1.77557522123894\\
1.10202555521481	1.78\\
1.10656674861173	1.79\\
1.11	1.79766263101794\\
1.11106070811692	1.8\\
1.11557906949578	1.81\\
1.12	1.81995775954335\\
1.12001900424345	1.82\\
1.12451676044264	1.83\\
1.12893520134001	1.84\\
1.13	1.84241981165288\\
1.13338061621615	1.85\\
1.13778027762727	1.86\\
1.14	1.86509006013442\\
1.14217127978629	1.87\\
1.14655431210911	1.88\\
1.15	1.88797222222222\\
1.15088924135872	1.89\\
1.15525781430386	1.9\\
1.15954976109842	1.91\\
1.16	1.91105079030559\\
1.16389114643687	1.92\\
1.16817017573146	1.93\\
1.17	1.93430824059353\\
1.1724545210431	1.94\\
1.17672276136646	1.95\\
1.18	1.95778371845005\\
1.18094799766589	1.96\\
1.18520761551258	1.97\\
1.18939244979094	1.98\\
1.19	1.98145525575672\\
1.19362468640727	1.99\\
1.19780221470824	2\\
32	831\\
1.60075464761864	2\\
1.6	1.99752212389381\\
1.59760343233601	1.99\\
1.59446066085978	1.98\\
1.59137259790128	1.97\\
1.59	1.96552176345232\\
1.58822981156943	1.96\\
1.58505626243406	1.95\\
1.58193832294118	1.94\\
1.58	1.93371656572854\\
1.57880107855122	1.93\\
1.57559742013407	1.92\\
1.57245023075782	1.91\\
1.57	1.90210842114663\\
1.56931557091319	1.9\\
1.56608251511515	1.89\\
1.56290674316501	1.88\\
1.56	1.87069903329753\\
1.55977164718648	1.87\\
1.55650994738627	1.86\\
1.55330629891735	1.85\\
1.55015769826797	1.84\\
1.55	1.83949889867841\\
1.5468781390538	1.83\\
1.54364735671722	1.82\\
1.54047229729144	1.81\\
1.54	1.80850914225203\\
1.53718553738608	1.8\\
1.53392839805303	1.79\\
1.5307276080448	1.78\\
1.53	1.77771851033752\\
1.52743061766812	1.77\\
1.52414792984038	1.76\\
1.52092216702892	1.75\\
1.52	1.74712783869827\\
1.51761188581722	1.74\\
1.51430448683898	1.73\\
1.51105453599004	1.72\\
1.51	1.71673775437976\\
1.50772788073165	1.71\\
1.5043966338199	1.7\\
1.50112330391909	1.69\\
1.5	1.68654867256637\\
1.49777717634711	1.68\\
1.49442296745934	1.67\\
1.49112708879493	1.66\\
1.49	1.65656079454964\\
1.4877583833779	1.65\\
1.48438211793775	1.64\\
1.48106453905159	1.63\\
1.48	1.62677410682702\\
1.47767015072228	1.62\\
1.47427275022521	1.61\\
1.47093433475198	1.6\\
1.47	1.59718838133946\\
1.46751116651499	1.59\\
1.46409356503708	1.58\\
1.46073518845288	1.57\\
1.46	1.56780317684433\\
1.45728015881243	1.56\\
1.45384329944632	1.55\\
1.45046584574871	1.54\\
1.45	1.53861784140969\\
1.44697589590002	1.53\\
1.44352072714261	1.52\\
1.44012508548433	1.51\\
1.44	1.50963151600422\\
1.4365971862141	1.5\\
1.43312465833078	1.49\\
1.43	1.48085819789248\\
1.42969464651969	1.48\\
1.42614287787454	1.47\\
1.42265393926478	1.46\\
1.42	1.45229155682223\\
1.41917902639888	1.45\\
1.41561185782767	1.44\\
1.41210745141636	1.43\\
1.41	1.42392346830048\\
1.40858462308219	1.42\\
1.40500305060224	1.41\\
1.40148411028052	1.4\\
1.4	1.39575221238938\\
1.3979104106236	1.39\\
1.39431541668469	1.38\\
1.39078286382317	1.37\\
1.39	1.36777588608711\\
1.3871554006654	1.36\\
1.38354795052293	1.35\\
1.38000269057877	1.34\\
1.38	1.33999241306639\\
1.37631863995536	1.33\\
1.37269967817042	1.32\\
1.37	1.31244225840897\\
1.3690936032985	1.31\\
1.36539920749876	1.3\\
1.36176965458547	1.29\\
1.36	1.2850833864683\\
1.35809984537674	1.28\\
1.35439621136415	1.27\\
1.35075696060226	1.26\\
1.35	1.25791299559471\\
1.34702073815037	1.25\\
1.34330878516381	1.24\\
1.34	1.23094484480913\\
1.33964158926013	1.23\\
1.33585547245619	1.22\\
1.33213608423142	1.21\\
1.33	1.20419951232488\\
1.32839529112011	1.2\\
1.32460325950384	1.19\\
1.32087728152101	1.18\\
1.32	1.177636376454\\
1.31706041645819	1.17\\
1.31326332628464	1.16\\
1.31	1.15127446979148\\
1.3095056344216	1.15\\
1.30563624936804	1.14\\
1.30183491074494	1.13\\
1.3	1.12513274336283\\
1.29799406450604	1.12\\
1.29412208197895	1.11\\
1.29031725675196	1.1\\
1.29	1.0991651818661\\
1.28639103277086	1.09\\
1.28251720897032	1.08\\
1.28	1.07342790202343\\
1.27863961554107	1.07\\
1.27469588608854	1.06\\
1.27082092193458	1.05\\
1.27	1.04787410635675\\
1.26685109491878	1.04\\
1.26290796380889	1.03\\
1.26	1.02252911482403\\
1.25898054452684	1.02\\
1.25496850246564	1.01\\
1.251026591619	1\\
1.25	0.99738436123348\\
1.24700021367605	0.99\\
1.24299120632658	0.98\\
1.24	0.972440241734803\\
1.23900068240505	0.97\\
1.23492399630323	0.96\\
1.23091854899198	0.95\\
1.23	0.947698019017433\\
1.22682252510734	0.94\\
1.22275126977864	0.93\\
1.22	0.923161661341853\\
1.218684273408	0.92\\
1.21454678506317	0.91\\
1.21048137261019	0.9\\
1.21	0.898813749780355\\
1.20630256514401	0.89\\
1.2021728258433	0.88\\
1.2	0.874690265486726\\
1.19801600596546	0.87\\
1.1938216645539	0.86\\
1.19	0.850739805738727\\
1.18968443218814	0.85\\
1.18542528409462	0.84\\
1.18124089527433	0.83\\
1.18	0.827019527670074\\
1.17698101422938	0.82\\
1.17273380735081	0.81\\
1.17	0.803490166696223\\
1.16848612256907	0.8\\
1.16417603152623	0.79\\
1.16	0.78014213342847\\
1.15993781827844	0.78\\
1.15556484813594	0.77\\
1.1512687315622	0.76\\
1.15	0.757031938325991\\
1.14689748412931	0.75\\
1.14254050992454	0.74\\
1.14	0.734109245483529\\
1.13817111686121	0.73\\
1.13375341086637	0.72\\
1.13	0.711374452749599\\
1.1293828780327	0.71\\
1.12490463416309	0.7\\
1.12050469643356	0.69\\
1.12	0.68885086055497\\
1.11599133787651	0.68\\
1.11153286348742	0.67\\
1.11	0.666541535884324\\
1.10701064207409	0.66\\
1.10249392254827	0.65\\
1.1	0.644424778761062\\
1.09795963255431	0.64\\
1.09338502104165	0.63\\
1.09	0.622502761498845\\
1.08883536453093	0.62\\
1.0842032745229	0.61\\
1.08	0.600777484859281\\
1.07963486622651	0.6\\
1.07494576987791	0.59\\
1.07033874018641	0.58\\
1.07	0.579263914721881\\
1.06560956837928	0.57\\
1.06094747314211	0.56\\
1.06	0.557960745691171\\
1.05619170854768	0.55\\
1.05147507347984	0.54\\
1.05	0.536855726872246\\
1.04668920876717	0.53\\
1.04191860686293	0.52\\
1.04	0.515950017649135\\
1.03709906960235	0.51\\
1.03227511809526	0.5\\
1.03	0.495244575013253\\
1.02741827576533	0.49\\
1.02254163274812	0.48\\
1.02	0.474740148567386\\
1.01764379768035	0.47\\
1.01271515839889	0.46\\
1.01	0.454437276588214\\
1.0077725925949	0.45\\
1.00279268543322	0.44\\
1	0.434336283185841\\
0.997801605187135	0.43\\
0.99277118736333	0.42\\
0.99	0.414437276588215\\
0.987727767621093	0.41\\
0.982647620616207	0.4\\
0.98	0.394740148567386\\
0.977547999002761	0.39\\
0.972418923747267	0.38\\
0.97	0.375244575013254\\
0.967259204192591	0.37\\
0.962082016036655	0.36\\
0.96	0.355950017649135\\
0.956858271932142	0.35\\
0.951633795427524	0.34\\
0.95	0.336855726872247\\
0.94634207224516	0.33\\
0.941071135767696	0.32\\
0.94	0.317960745691171\\
0.935707453075991	0.31\\
0.930390883318308	0.3\\
0.93	0.299263914721881\\
0.924951236130735	0.29\\
0.92	0.28077748485928\\
0.919571623806638	0.28\\
0.91407021188902	0.27\\
0.91	0.262502761498846\\
0.908605932850972	0.26\\
0.903061133756482	0.25\\
0.9	0.244424778761062\\
0.89750804775976	0.24\\
0.891920711329954	0.23\\
0.89	0.226541535884324\\
0.886274657876436	0.22\\
0.880645602748883	0.21\\
0.88	0.20885086055497\\
0.87490239850299	0.2\\
0.87	0.1913744527496\\
0.869199204705407	0.19\\
0.86338784126879	0.18\\
0.86	0.174109245483528\\
0.85757806681069	0.17\\
0.851727483486668	0.16\\
0.85	0.157031938325991\\
0.845806821411703	0.15\\
0.84	0.140142133428469\\
0.839914226431242	0.14\\
0.833881844113752	0.13\\
0.83	0.123490166696223\\
0.827868453330229	0.12\\
0.821799403610788	0.11\\
0.82	0.107019527670074\\
0.815660771726075	0.1\\
0.81	0.0907398057387276\\
0.809536976979076	0.0899999999999999\\
0.803287273157607	0.0800000000000001\\
0.8	0.0746902654867253\\
0.797027846832049	0.0699999999999998\\
0.79074389854872	0.0600000000000001\\
0.79	0.0588137497803551\\
0.784344213449097	0.0499999999999998\\
0.78	0.0431616613418528\\
0.777944892314171	0.04\\
0.771481761277178	0.0299999999999998\\
0.77	0.0276980190174326\\
0.764931253677611	0.02\\
0.76	0.0124402417348024\\
0.758371870869655	0.00999999999999979\\
0.751729384714773	0\\
0.75	-0.00261563876651981\\
0.745007443042408	-0.01\\
0.74	-0.0174708851759684\\
0.738266719859182	-0.02\\
0.731444793820139	-0.03\\
0.73	-0.0321258936432471\\
0.724529639699428	-0.04\\
0.72	-0.0465720979765705\\
0.717585107613436	-0.05\\
0.71058362790412	-0.0600000000000001\\
0.71	-0.0608348181338956\\
0.703452212899852	-0.0700000000000001\\
0.7	-0.0748672566371679\\
0.69628009356206	-0.0800000000000001\\
0.69	-0.088725530208519\\
0.689062799589789	-0.0900000000000001\\
0.681726639823103	-0.1\\
0.68	-0.102363623545999\\
0.674301672524503	-0.11\\
0.67	-0.11580048767512\\
0.666818955124136	-0.12\\
0.66	-0.129055155190867\\
0.659273321438787	-0.13\\
0.651596180890842	-0.14\\
0.65	-0.142087004405286\\
0.643820137355511	-0.15\\
0.64	-0.154916613531704\\
0.63596716209238	-0.16\\
0.63	-0.167557741591031\\
0.628031302788488	-0.17\\
0.62000608825861	-0.18\\
0.62	-0.180007586933614\\
0.611814558389291	-0.19\\
0.61	-0.19222411391289\\
0.603524381746346	-0.2\\
0.6	-0.204247787610619\\
0.595128674197487	-0.21\\
0.59	-0.216076531699521\\
0.586620217001373	-0.22\\
0.58	-0.22770844317777\\
0.577991436344404	-0.23\\
0.57	-0.239141802107519\\
0.569234380041507	-0.24\\
0.560327937361334	-0.25\\
0.56	-0.250368483995779\\
0.551252824095112	-0.26\\
0.55	-0.261382158590309\\
0.542028769368308	-0.27\\
0.54	-0.272196823155665\\
0.532646366868334	-0.28\\
0.53	-0.282811618660541\\
0.523095679182596	-0.29\\
0.52	-0.293225893172975\\
0.513366189925653	-0.3\\
0.51	-0.303439205450363\\
0.503446749052807	-0.31\\
0.5	-0.313451327433628\\
0.49332551035042	-0.32\\
0.49	-0.323262245620244\\
0.482989859935468	-0.33\\
0.48	-0.332872161301733\\
0.472426334409223	-0.34\\
0.47	-0.342281489662484\\
0.461620527085681	-0.35\\
0.46	-0.35149085774797\\
0.450556980446191	-0.36\\
0.45	-0.360501101321586\\
0.44	-0.36930096670247\\
0.439188018236734	-0.37\\
0.43	-0.377891578853367\\
0.427492568539887	-0.38\\
0.42	-0.386283434271464\\
0.41547222056454	-0.39\\
0.41	-0.394478236547682\\
0.403104691299435	-0.4\\
0.4	-0.402477876106195\\
0.390365640928571	-0.41\\
0.39	-0.410284420737083\\
0.38	-0.41786256703611\\
0.377113539937838	-0.42\\
0.37	-0.425244162662395\\
0.363398822940919	-0.43\\
0.36	-0.432436981934113\\
0.35	-0.439433856502242\\
0.349167643518036	-0.44\\
0.34	-0.446201070281841\\
0.334245002465833	-0.45\\
0.33	-0.452785768753325\\
0.32	-0.45917674919268\\
0.318671751416518	-0.46\\
0.31	-0.46533932454108\\
0.302228756606406	-0.47\\
0.3	-0.471327433628319\\
0.29	-0.477093713110356\\
0.284804668078814	-0.48\\
0.28	-0.482667234646787\\
0.27	-0.488041396307582\\
0.266226619357178	-0.49\\
0.26	-0.493204976892997\\
0.25	-0.498178251121076\\
0.246194372261119	-0.5\\
0.24	-0.502938357625311\\
0.23	-0.507503665531457\\
0.224304854984307	-0.51\\
0.22	-0.511868512602059\\
0.21	-0.516020559828295\\
0.2	-0.52\\
0.2	-0.52\\
0.19	-0.523735332382819\\
0.18	-0.527293003229279\\
0.1719730818848	-0.53\\
0.17	-0.530657731867352\\
0.16	-0.53379293613985\\
0.15	-0.536743273542601\\
0.14	-0.539505805986296\\
0.138070912002252	-0.54\\
0.13	-0.542040745684286\\
0.12	-0.544376689309975\\
0.11	-0.54652000538503\\
0.1	-0.548468468468469\\
0.0912523991581717	-0.55\\
0.0899999999999999	-0.550216116142781\\
0.0800000000000001	-0.551741218382615\\
0.0699999999999998	-0.553069289158778\\
0.0600000000000001	-0.554198925886144\\
0.0499999999999998	-0.555128923766816\\
0.04	-0.555858282429033\\
0.0299999999999998	-0.55638621154884\\
0.02	-0.556712135397911\\
0.00999999999999979	-0.556835696270942\\
0	-0.556756756756757\\
-0.01	-0.55647540082868\\
-0.02	-0.555991933741448\\
-0.03	-0.555306880733945\\
-0.04	-0.554420984549048\\
-0.05	-0.553335201793722\\
-0.0600000000000001	-0.552050698174006\\
-0.0700000000000001	-0.550568842650473\\
-0.0733847339196045	-0.55\\
-0.0800000000000001	-0.548871091766413\\
-0.0900000000000001	-0.546965639124932\\
-0.1	-0.544864864864865\\
-0.11	-0.54257098366541\\
-0.12	-0.540086378262424\\
-0.120321784733205	-0.54\\
-0.13	-0.537366724044211\\
-0.14	-0.534457122250271\\
-0.15	-0.531362107623318\\
-0.154142674387839	-0.53\\
-0.16	-0.528049981837995\\
-0.17	-0.524531314316664\\
-0.18	-0.520834445640474\\
-0.182144175915361	-0.52\\
-0.19	-0.516907811649428\\
-0.2	-0.512792792792793\\
-0.206501936947991	-0.51\\
-0.21	-0.508481323984702\\
-0.22	-0.503951427538851\\
-0.228409086976167	-0.5\\
-0.23	-0.499244926081402\\
-0.24	-0.494305320304017\\
-0.248437358697329	-0.49\\
-0.25	-0.489195205479452\\
-0.26	-0.483852913499819\\
-0.266997181895114	-0.48\\
-0.27	-0.478332350793941\\
-0.28	-0.472596169136249\\
-0.284392799541372	-0.47\\
-0.29	-0.466660162083409\\
-0.3	-0.46054054054054\\
-0.30085613872872	-0.46\\
-0.31	-0.454185982580294\\
-0.316422715788155	-0.45\\
-0.32	-0.447652758494702\\
-0.33	-0.44092055425167\\
-0.331330077016884	-0.44\\
-0.34	-0.433963530693789\\
-0.345562963538872	-0.43\\
-0.35	-0.426820776255708\\
-0.359306957959653	-0.42\\
-0.36	-0.419489386705839\\
-0.37	-0.411930956695053\\
-0.372497308064091	-0.41\\
-0.38	-0.404172406261376\\
-0.385257584532199	-0.4\\
-0.39	-0.396220608663864\\
-0.397642195273388	-0.39\\
-0.4	-0.388073394495413\\
-0.409676060875618	-0.38\\
-0.41	-0.379728770023937\\
-0.42	-0.371163438520131\\
-0.421330119582627	-0.37\\
-0.43	-0.362395881069285\\
-0.432676760853113	-0.36\\
-0.44	-0.353429675537732\\
-0.443746446986298	-0.35\\
-0.45	-0.344263698630137\\
-0.454556073740328	-0.34\\
-0.46	-0.33489703622393\\
-0.465121316337024	-0.33\\
-0.47	-0.325328988825792\\
-0.475456761408132	-0.32\\
-0.48	-0.315559075907591\\
-0.485576021738231	-0.31\\
-0.49	-0.305587039075399\\
-0.495491836166113	-0.3\\
-0.5	-0.295412844036697\\
-0.505216156647373	-0.29\\
-0.51	-0.285036681342873\\
-0.514760224184341	-0.28\\
-0.52	-0.27445896589659\\
-0.524134635083499	-0.27\\
-0.53	-0.263680335226232\\
-0.533349398795507	-0.26\\
-0.54	-0.252701646542261\\
-0.542413988421448	-0.25\\
-0.55	-0.24152397260274\\
-0.551337384824619	-0.24\\
-0.56	-0.230148596427269\\
-0.56012811516535	-0.23\\
-0.568747997008012	-0.22\\
-0.57	-0.218550648268198\\
-0.577241589182703	-0.21\\
-0.58	-0.206751976357591\\
-0.585621433761481	-0.2\\
-0.59	-0.194757309887682\\
-0.593894805232981	-0.19\\
-0.6	-0.182568807339449\\
-0.602068648086759	-0.18\\
-0.61	-0.170188804606105\\
-0.61014959503836	-0.17\\
-0.618071830108486	-0.16\\
-0.62	-0.15757567667779\\
-0.625904553418225	-0.15\\
-0.63	-0.144769689979701\\
-0.633659602634942	-0.14\\
-0.64	-0.131777304443629\\
-0.641342739762324	-0.13\\
-0.648918539256407	-0.12\\
-0.65	-0.118575581395349\\
-0.656377943712895	-0.11\\
-0.66	-0.10515782463929\\
-0.663779017879832	-0.1\\
-0.67	-0.0915607556536129\\
-0.67112678570836	-0.0900000000000001\\
-0.678363404752814	-0.0800000000000001\\
-0.68	-0.0777467807964273\\
-0.68550952360471	-0.0700000000000001\\
-0.69	-0.0637292459804103\\
-0.692614566536601	-0.0600000000000001\\
-0.699670136609624	-0.05\\
-0.7	-0.0495327102803738\\
-0.706586675156159	-0.04\\
-0.71	-0.0350975607493971\\
-0.713473599721919	-0.03\\
-0.72	-0.0204981965403018\\
-0.720334866116019	-0.02\\
-0.727059352534976	-0.01\\
-0.73	-0.00565959286112665\\
-0.733751360876383	0\\
-0.74	0.00934552156284556\\
-0.740428074874529	0.00999999999999979\\
-0.746973484502093	0.02\\
-0.75	0.0245872093023256\\
-0.753492540144329	0.0299999999999998\\
-0.76	0.0399905639513454\\
-0.760006010218545	0.04\\
-0.766372497631734	0.0499999999999998\\
-0.77	0.0556417642684514\\
-0.772739438450729	0.0600000000000001\\
-0.779072570015739	0.0699999999999998\\
-0.78	0.0714609673790777\\
-0.78529764906006	0.0800000000000001\\
-0.79	0.0874993600445187\\
-0.791532257835361	0.0899999999999999\\
-0.79768494269365	0.1\\
-0.8	0.103738317757009\\
-0.803788262154526	0.11\\
-0.809905431920024	0.12\\
-0.81	0.120154594238373\\
-0.815881920521048	0.13\\
-0.82	0.136812206922218\\
-0.821881882519657	0.14\\
-0.827817122527124	0.15\\
-0.83	0.153654439033527\\
-0.833700362893647	0.16\\
-0.839597622537499	0.17\\
-0.84	0.170681643422541\\
-0.845368457683515	0.18\\
-0.85	0.187936046511628\\
-0.851175502293786	0.19\\
-0.856889770683611	0.2\\
-0.86	0.205391096146652\\
-0.862594084684244	0.21\\
-0.868267811232437	0.22\\
-0.87	0.223036858432036\\
-0.873873600815782	0.23\\
-0.879506007377715	0.24\\
-0.88	0.240875859463543\\
-0.885017458664541	0.25\\
-0.89	0.258930748463973\\
-0.890581606079219	0.26\\
-0.896029001696184	0.27\\
-0.9	0.277196261682243\\
-0.90150794612209	0.28\\
-0.906911518737127	0.29\\
-0.91	0.295658047270681\\
-0.912309343097372	0.3\\
-0.917668252851194	0.31\\
-0.92	0.314317802032367\\
-0.922989024477766	0.32\\
-0.928302409240675	0.33\\
-0.93	0.333177023966786\\
-0.933550180553271	0.34\\
-0.938817162191563	0.35\\
-0.94	0.352237003405221\\
-0.943995970453236	0.36\\
-0.94921566108424	0.37\\
-0.95	0.371498815165877\\
-0.95432952729165	0.38\\
-0.959501035492992	0.39\\
-0.96	0.390963311811622\\
-0.964553962462768	0.4\\
-0.969676399400388	0.41\\
-0.97	0.410631118083286\\
-0.974672369116681	0.42\\
-0.979744854555599	0.43\\
-0.98	0.430502626570232\\
-0.984687824847279	0.44\\
-0.989709493009043	0.45\\
-0.99	0.450577994667682\\
-0.994603393628004	0.46\\
-0.999573398859269	0.47\\
-1	0.470857142857143\\
-1.00442212703389	0.48\\
-1.00933964925158	0.49\\
-1.01	0.491339754332508\\
-1.0141470647914	0.5\\
-1.01901131467144	0.51\\
-1.02	0.512025275980206\\
-1.02378123470048	0.52\\
-1.028591458579	0.53\\
-1.03	0.532912920707359\\
-1.03332765197594	0.54\\
-1.03808313643444	0.55\\
-1.04	0.554001671097607\\
-1.04278931805769	0.56\\
-1.04748939416635	0.57\\
-1.05	0.57529028436019\\
-1.05216921894148	0.58\\
-1.05681326613802	0.59\\
-1.06	0.596777298524404\\
-1.0614703230832	0.6\\
-1.06605777266822	0.61\\
-1.07	0.618461039818833\\
-1.07069557893121	0.62\\
-1.07522591716465	0.63\\
-1.07984019798167	0.64\\
-1.08	0.640346144994246\\
-1.08432068292871	0.65\\
-1.08887577654692	0.66\\
-1.09	0.662457092334162\\
-1.09334502969092	0.67\\
-1.09784048901694	0.68\\
-1.1	0.684761904761905\\
-1.10230188993534	0.69\\
-1.10673733477006	0.7\\
-1.11	0.707258025042686\\
-1.11119416507098	0.71\\
-1.11556928360539	0.72\\
-1.12	0.729942727616169\\
-1.12002472150596	0.73\\
-1.12433927136435	0.74\\
-1.12873351460584	0.75\\
-1.13	0.752867033914543\\
-1.13305019551774	0.76\\
-1.13738175078426	0.77\\
-1.14	0.775978040411743\\
-1.14170491077363	0.78\\
-1.14597364929698	0.79\\
-1.15	0.799271327014218\\
-1.15030622475762	0.8\\
-1.15451209332318	0.81\\
-1.15879596908368	0.82\\
-1.16	0.822796158278909\\
-1.1629999140098	0.83\\
-1.16721879132815	0.84\\
-1.17	0.846512893682\\
-1.17143988603203	0.85\\
-1.17559384956365	0.86\\
-1.17982576205569	0.87\\
-1.18	0.870411441824507\\
-1.18392393554651	0.88\\
-1.18808877801265	0.89\\
-1.19	0.894550191901746\\
-1.19221177431747	0.9\\
-1.19630978593187	0.91\\
-1.2	0.918857142857143\\
-1.20046002021789	0.92\\
-1.20449151903564	0.93\\
-1.20859810658547	0.94\\
-1.21	0.943392226931227\\
-1.21263663319721	0.95\\
-1.21667463943367	0.96\\
-1.22	0.968110966755827\\
-1.22074770341269	0.97\\
-1.22471761404022	0.98\\
-1.22876175891404	0.99\\
-1.23	0.993044419772157\\
-1.23272959553844	1\\
-1.23670361211335	1.01\\
-1.24	1.01816854190586\\
-1.24071305832004	1.02\\
-1.24461760873616	1.03\\
-1.24859523340693	1.04\\
-1.25	1.04350845410628\\
-1.2525062004565	1.05\\
-1.2564123861356	1.06\\
-1.26	1.06902962112514\\
-1.26037173911848	1.07\\
-1.26420732962004	1.08\\
-1.26811454172083	1.09\\
-1.27	1.09478220698977\\
-1.27198237914548	1.1\\
-1.27581710564924	1.11\\
-1.27972466719887	1.12\\
-1.28	1.12070370081808\\
-1.28350301017459	1.13\\
-1.28733617228191	1.14\\
-1.29	1.14685976690426\\
-1.2911744119907	1.15\\
-1.29493432796702	1.16\\
-1.29876516369626	1.17\\
-1.3	1.17320388349515\\
-1.3025212314703	1.18\\
-1.30627702378038	1.19\\
-1.31	1.19973153905201\\
-1.31009886030977	1.2\\
-1.313780947632	1.21\\
-1.31753179043324	1.22\\
-1.32	1.22649957479706\\
-1.32127883860057	1.23\\
-1.32495431282144	1.24\\
-1.32869932824612	1.25\\
-1.33	1.25345029188558\\
-1.33237409782703	1.26\\
-1.33604209028024	1.27\\
-1.33978035082447	1.28\\
-1.34	1.28058691735213\\
-1.34338546074525	1.29\\
-1.34704505941113	1.3\\
-1.35	1.30795289855072\\
-1.35073103425432	1.31\\
-1.35431376844131	1.32\\
-1.35796402081658	1.33\\
-1.36	1.33551900505247\\
-1.36158796520266	1.34\\
-1.36515988482776	1.35\\
-1.36879980107655	1.36\\
-1.37	1.3632766262942\\
-1.37236455261373	1.37\\
-1.37592470085695	1.38\\
-1.3795532572316	1.39\\
-1.38	1.39122826855124\\
-1.38306174615982	1.4\\
-1.38660913842087	1.41\\
-1.39	1.4193883291433\\
-1.39021204662418	1.42\\
-1.39368052635408	1.43\\
-1.39721415390894	1.44\\
-1.4	1.44776699029126\\
-1.40076839273288	1.45\\
-1.40422190754635	1.46\\
-1.40774074139641	1.47\\
-1.41	1.47634243812122\\
-1.41124936631122	1.48\\
-1.41468694054725	1.49\\
-1.41818993543835	1.5\\
-1.42	1.50511630817364\\
-1.42165606780706	1.51\\
-1.42507671486092	1.52\\
-1.42856281344702	1.53\\
-1.43	1.53409002745636\\
-1.43198963420975	1.54\\
-1.43539236050939	1.55\\
-1.4388604976333	1.56\\
-1.44	1.56326480534801\\
-1.44225124037659	1.57\\
-1.44563504943511	1.58\\
-1.44908415649661	1.59\\
-1.45	1.59264162561576\\
-1.45244209995583	1.6\\
-1.45580599647149	1.61\\
-1.4592350058516	1.62\\
-1.46	1.62222123963679\\
-1.46256346590124	1.63\\
-1.46590645987549	1.64\\
-1.46931430938433	1.65\\
-1.47	1.65200416090136\\
-1.47261663057699	1.66\\
-1.47593774142016	1.67\\
-1.47932337873487	1.68\\
-1.48	1.68199066086268\\
-1.48260292545538	1.69\\
-1.48590118604929	1.7\\
-1.48926357310834	1.71\\
-1.49	1.71218076618491\\
-1.49252372041398	1.72\\
-1.495798181101	1.73\\
-1.49913629842085	1.74\\
-1.5	1.74257425742574\\
-1.50238042264245	1.75\\
-1.50563015511077	1.76\\
-1.50894300599181	1.77\\
-1.51	1.77317066917442\\
-1.51217447517347	1.78\\
-1.51539857620943	1.79\\
-1.51868519079867	1.8\\
-1.52	1.80396929165018\\
-1.52190735505552	1.81\\
-1.52510495013501	1.82\\
-1.52836438931478	1.83\\
-1.53	1.83496917374975\\
-1.53158057118903	1.84\\
-1.53475081788182	1.85\\
-1.5379821769553	1.86\\
-1.54	1.86616912751678\\
-1.54119566185057	1.87\\
-1.54433775301314	1.88\\
-1.54754016516	1.89\\
-1.55	1.89756773399015\\
-1.55075419193241	1.9\\
-1.5538673586676	1.91\\
-1.55703999814543	1.92\\
-1.56	1.92916335037357\\
-1.56025774992769	1.93\\
-1.5633412642917	1.94\\
-1.56648334936198	1.95\\
-1.5696875374317	1.96\\
-1.57	1.96097320464093\\
-1.57276112213352	1.97\\
-1.57587191769391	1.98\\
-1.57904379660758	1.99\\
-1.58	1.99299656960511\\
-1.58212860353432	2\\
64	462\\
-1.10829202269191	2\\
-1.10372394097822	1.99\\
-1.1	1.98193548387097\\
-1.09911207428031	1.98\\
-1.09451251491441	1.97\\
-1.09	1.96031496331574\\
-1.08985415005804	1.96\\
-1.08522124917158	1.95\\
-1.08052650724232	1.94\\
-1.08	1.93888045797554\\
-1.0758484826828	1.93\\
-1.07111826833744	1.92\\
-1.07	1.91764380599922\\
-1.06639241035792	1.91\\
-1.0616247417175	1.9\\
-1.06	1.89660814104226\\
-1.0568510783663	1.89\\
-1.05204393525877	1.88\\
-1.05	1.87577265372168\\
-1.0472223788054	1.87\\
-1.04237369802637	1.86\\
-1.04	1.85513669511249\\
-1.03750404372786	1.85\\
-1.03261171415613	1.84\\
-1.03	1.83469977393102\\
-1.02769363849456	1.83\\
-1.02275549603624	1.82\\
-1.02	1.81446155435063\\
-1.01778855441635	1.81\\
-1.01280237674986	1.8\\
-1.01	1.79442185443283\\
-1.00778600064162	1.79\\
-1.00274950173464	1.78\\
-1	1.77458064516129\\
-0.997682995241003	1.77\\
-0.99259381960859	1.76\\
-0.99	1.75493805007098\\
-0.987476355434351	1.75\\
-0.982332072105482	1.74\\
-0.98	1.73549434546863\\
-0.977162686898014	1.73\\
-0.971960783055524	1.72\\
-0.97	1.71624996124532\\
-0.966738372083224	1.71\\
-0.961476246339318	1.7\\
-0.96	1.69720548228601\\
-0.956199557467958	1.69\\
-0.95087451273446	1.68\\
-0.95	1.67836165048544\\
-0.945542139655642	1.67\\
-0.940151375564608	1.66\\
-0.94	1.65971936738398\\
-0.934761750224009	1.65\\
-0.93	1.64126329316754\\
-0.92931170783578	1.64\\
-0.923853739216265	1.63\\
-0.92	1.62300477780632\\
-0.918344619706251	1.62\\
-0.912813157154301	1.61\\
-0.91	1.60494876431973\\
-0.907242897747136	1.6\\
-0.901634735439865	1.59\\
-0.9	1.58709677419355\\
-0.896001077527837	1.58\\
-0.89031286499412	1.57\\
-0.89	1.56945050459309\\
-0.884613341838757	1.56\\
-0.88	1.55198606200359\\
-0.878854891098616	1.55\\
-0.873073496150932	1.54\\
-0.87	1.53472136614831\\
-0.867245646294078	1.53\\
-0.86137494184291	1.52\\
-0.86	1.51766561774568\\
-0.855474590098096	1.51\\
-0.85	1.5008107028754\\
-0.849515672365095	1.5\\
-0.843534426916739	1.49\\
-0.84	1.48413692584039\\
-0.837498282373525	1.48\\
-0.831417376591066	1.47\\
-0.83	1.46767671047545\\
-0.825300149288758	1.46\\
-0.82	1.45141466905188\\
-0.819123158347012	1.45\\
-0.8129126530252	1.44\\
-0.81	1.43533944665554\\
-0.806648456611165	1.43\\
-0.800326586731842	1.42\\
-0.8	1.41948387096774\\
-0.793970420606231	1.41\\
-0.79	1.40380214496094\\
-0.787551539516137	1.4\\
-0.781078819904966	1.39\\
-0.78	1.38833726500129\\
-0.774560167360853	1.38\\
-0.77	1.37306008822401\\
-0.767976957230134	1.37\\
-0.761338405907251	1.36\\
-0.76	1.35798960638024\\
-0.754646391555753	1.35\\
-0.75	1.34311102236422\\
-0.747887057442961	1.34\\
-0.741066525504396	1.33\\
-0.74	1.32843982505789\\
-0.734187951358977	1.32\\
-0.73	1.31395511443549\\
-0.727238271249467	1.31\\
-0.720218242832236	1.3\\
-0.72	1.29968931238733\\
-0.71313727555662	1.29\\
-0.71	1.28559494813677\\
-0.705980258705991	1.28\\
-0.7	1.27171974522293\\
-0.698746413931334	1.27\\
-0.69143940723069	1.26\\
-0.69	1.25803555655411\\
-0.684054873893107	1.25\\
-0.68	1.24454525939177\\
-0.67658665847825	1.24\\
-0.67	1.23126815290675\\
-0.669033133457639	1.23\\
-0.66139490988767	1.22\\
-0.66	1.21817849627919\\
-0.653666810896632	1.21\\
-0.65	1.20528354632588\\
-0.645844203949257	1.2\\
-0.64	1.19259730073848\\
-0.63792442684208	1.19\\
-0.63	1.18011761832255\\
-0.629904481226251	1.18\\
-0.621780575455079	1.17\\
-0.62	1.16781491160646\\
-0.613547383257396	1.16\\
-0.61	1.15571630476434\\
-0.605200907853317	1.15\\
-0.6	1.14382165605096\\
-0.596736755715165	1.14\\
-0.59	1.13212948913458\\
-0.588150074380325	1.13\\
-0.58	1.12063849860512\\
-0.579435517277138	1.12\\
-0.570585466821502	1.11\\
-0.57	1.10933918087425\\
-0.561592988671211	1.1\\
-0.56	1.09823332480164\\
-0.552452592019493	1.09\\
-0.55	1.08732827476038\\
-0.543156788674474	1.08\\
-0.54	1.07662333418432\\
-0.533697368700748	1.07\\
-0.53	1.06611796326999\\
-0.524065333238486	1.06\\
-0.52	1.0558117767015\\
-0.51425081873686	1.05\\
-0.51	1.04570454197987\\
-0.504243011406629	1.04\\
-0.5	1.03579617834395\\
-0.494030050495276	1.03\\
-0.49	1.02608675627468\\
-0.483598918748723	1.02\\
-0.48	1.01657649757838\\
-0.472935318138099	1.01\\
-0.47	1.00726577604897\\
-0.46202352858818	1\\
-0.46	0.998155118713301\\
-0.45084624703321	0.99\\
-0.45	0.989245207667731\\
-0.44	0.980530098559515\\
-0.439377199461106	0.98\\
-0.43	0.972005436020757\\
-0.427587176744239	0.97\\
-0.42	0.963681866599036\\
-0.41546099822959	0.96\\
-0.41	0.955560674799847\\
-0.402971555786141	0.95\\
-0.4	0.947643312101911\\
-0.390088714266718	0.94\\
-0.39	0.939931402477966\\
-0.38	0.932396053630154\\
-0.376724117507188	0.93\\
-0.37	0.925066228904961\\
-0.362878802876877	0.92\\
-0.36	0.917944894321365\\
-0.35	0.911021293375394\\
-0.348479491570161	0.91\\
-0.34	0.904282138332911\\
-0.33342915874952	0.9\\
-0.33	0.897755877114871\\
-0.32	0.891427100681302\\
-0.317667902004546	0.89\\
-0.31	0.885285707947776\\
-0.301070687884845	0.88\\
-0.3	0.879363057324841\\
-0.29	0.873616253635099\\
-0.283451226423195	0.87\\
-0.28	0.868083803712179\\
-0.27	0.862742450448176\\
-0.264648432255323	0.86\\
-0.26	0.857603657607315\\
-0.25	0.85266167192429\\
-0.244369376884382	0.85\\
-0.24	0.847921158242317\\
-0.23	0.843373683878298\\
-0.222208087007865	0.84\\
-0.22	0.839037274345284\\
-0.21	0.834880635984322\\
-0.2	0.830943396226415\\
-0.197473986268589	0.83\\
-0.19	0.827187089618456\\
-0.18	0.823635023971738\\
-0.17	0.820296313277226\\
-0.169056259564126	0.82\\
-0.16	0.81713240435774\\
-0.15	0.814175867507886\\
-0.14	0.811428111641941\\
-0.134388370857983	0.81\\
-0.13	0.808872852429895\\
-0.12	0.806506855552745\\
-0.11	0.80434644343549\\
-0.1	0.802389937106918\\
-0.0900000000000001	0.80063583260133\\
-0.0859110295156042	0.8\\
-0.0800000000000001	0.799071164088258\\
-0.0700000000000001	0.797700917605366\\
-0.0600000000000001	0.796532120293151\\
-0.05	0.795563880126183\\
-0.04	0.794795462566171\\
-0.03	0.794226287621206\\
-0.02	0.793855927510697\\
-0.01	0.793684104918858\\
0	0.793710691823899\\
0.00999999999999979	0.793935708894201\\
0.02	0.794359325446766\\
0.0299999999999998	0.794981859967259\\
0.04	0.795803781194858\\
0.0499999999999998	0.79682570977918\\
0.0600000000000001	0.798048420520596\\
0.0699999999999998	0.799472845209467\\
0.0732437452541802	0.8\\
0.0800000000000001	0.801086301026797\\
0.0899999999999999	0.802894585267913\\
0.1	0.804905660377358\\
0.11	0.807121068230546\\
0.12	0.809542524664812\\
0.12174440224272	0.81\\
0.13	0.81214471244205\\
0.14	0.814948353029922\\
0.15	0.817961356466877\\
0.156330955408521	0.82\\
0.16	0.821171278458844\\
0.17	0.824567133525939\\
0.18	0.82817703759778\\
0.184775239401476	0.83\\
0.19	0.831978839654525\\
0.2	0.835974842767296\\
0.209549117527256	0.84\\
0.21	0.840188656511425\\
0.22	0.844572834546824\\
0.23	0.849181031435425\\
0.231701822617775	0.85\\
0.24	0.853966792074241\\
0.25	0.858970820189274\\
0.251975479231582	0.86\\
0.26	0.864154903436167\\
0.27	0.869559771493498\\
0.270785119640656	0.87\\
0.28	0.875137735375345\\
0.288375791323152	0.88\\
0.29	0.880937813709577\\
0.3	0.886918238993711\\
0.304968672015932	0.89\\
0.31	0.893105388659407\\
0.32	0.899501791572041\\
0.32075491681443	0.9\\
0.33	0.906074481849014\\
0.335775013051091	0.91\\
0.34	0.912860045033775\\
0.35	0.919854100946372\\
0.350202795908828	0.92\\
0.36	0.927022780990696\\
0.364027501052336	0.93\\
0.37	0.93440006891367\\
0.377377982295041	0.94\\
0.38	0.94198391206595\\
0.39	0.949769573716736\\
0.390288510202197	0.95\\
0.4	0.957735849056604\\
0.402766381396392	0.96\\
0.41	0.965906255490024\\
0.414883818064096	0.97\\
0.42	0.974279388930628\\
0.426667391518504	0.98\\
0.43	0.9828540119985\\
0.438141085928072	0.99\\
0.44	0.991629049163963\\
0.449326588677236	1\\
0.45	1.00060358255452\\
0.46	1.00977403579531\\
0.460241117630247	1.01\\
0.47	1.01913750787055\\
0.470901330139857	1.02\\
0.48	1.02870113264536\\
0.481329901457044	1.03\\
0.49	1.03846458045037\\
0.491540943412068	1.04\\
0.5	1.04842767295597\\
0.501547317910081	1.05\\
0.51	1.05859038243804\\
0.511360756822729	1.06\\
0.52	1.06895283161339\\
0.520991967859516	1.07\\
0.53	1.07951529404357\\
0.530450728222844	1.08\\
0.539744732771457	1.09\\
0.54	1.09027473238735\\
0.548881093882404	1.1\\
0.55	1.10122663551402\\
0.557870009931652	1.11\\
0.56	1.11237773895683\\
0.566718540251156	1.12\\
0.57	1.12372890263717\\
0.575433125519422	1.13\\
0.58	1.1352811419985\\
0.584019636966777	1.14\\
0.59	1.14703563182332\\
0.592483420159467	1.15\\
0.6	1.15899371069182\\
0.600829333956716	1.16\\
0.609060850522074	1.17\\
0.61	1.1711424772699\\
0.617183522263833	1.18\\
0.62	1.18348278790907\\
0.625202397956004	1.19\\
0.63	1.19602893747651\\
0.633120861428477	1.2\\
0.64	1.20878290168469\\
0.640941945257848	1.21\\
0.648669695876262	1.22\\
0.65	1.22172507788162\\
0.656308216676348	1.23\\
0.66	1.23486154615962\\
0.663859304791858	1.24\\
0.67	1.24820989197337\\
0.671324911853369	1.25\\
0.678709860461059	1.26\\
0.68	1.26175071019188\\
0.686018010730087	1.27\\
0.69	1.27548365878082\\
0.693248173293284	1.28\\
0.7	1.28943396226415\\
0.700401423895159	1.29\\
0.707488264348065	1.3\\
0.71	1.30355986390311\\
0.714504474646581	1.31\\
0.72	1.31789947275923\\
0.721449282517328	1.32\\
0.728330881928391	1.33\\
0.73	1.33243273282633\\
0.735151315580092	1.34\\
0.74	1.34716468522699\\
0.74190466000941	1.35\\
0.748598602825217	1.36\\
0.75	1.36209890965732\\
0.755238564593025	1.37\\
0.76	1.3772273890143\\
0.761814699370238	1.38\\
0.768337170962636	1.39\\
0.77	1.3925574055604\\
0.774809459070246	1.4\\
0.78	1.4080877730354\\
0.781220246928415	1.41\\
0.787586230630025	1.42\\
0.79	1.42380958296916\\
0.793901433792588	1.43\\
0.8	1.43974842767296\\
0.800156618359039	1.44\\
0.806380068819676	1.45\\
0.81	1.45585917511578\\
0.812546807065087	1.46\\
0.818665533778505	1.47\\
0.82	1.472186792923\\
0.824748229601852	1.48\\
0.83	1.4887123414144\\
0.830773347516206	1.49\\
0.836769346820712	1.5\\
0.84	1.50542446835126\\
0.842716020689007	1.51\\
0.848618093722855	1.52\\
0.85	1.52234813084112\\
0.854489894925948	1.53\\
0.86	1.53947437767161\\
0.860304928058407	1.54\\
0.866102144409571	1.55\\
0.87	1.55678072296705\\
0.87184661078619	1.56\\
0.877559506339334	1.57\\
0.88	1.57429467899076\\
0.883236281770373	1.58\\
0.888868310244752	1.59\\
0.89	1.59201431685141\\
0.894480042734252	1.6\\
0.9	1.60993710691824\\
0.90003490232062	1.61\\
0.905583628800357	1.62\\
0.91	1.62803952189735\\
0.911076277097581	1.63\\
0.916552431912075	1.64\\
0.92	1.64634550463311\\
0.92198495882636	1.65\\
0.927391522352771	1.66\\
0.93	1.66485376202975\\
0.932765834678185	1.67\\
0.938105668525825	1.68\\
0.94	1.68356316446219\\
0.943423499894311	1.69\\
0.948699355142328	1.7\\
0.95	1.70247274143302\\
0.953962275041988	1.71\\
0.959176799948434	1.72\\
0.96	1.72158167786906\\
0.96438622187994	1.73\\
0.969541969111042	1.74\\
0.97	1.74088931103097\\
0.974699157944364	1.75\\
0.979798591368609	1.76\\
0.98	1.76039512801392\\
0.984904669955241	1.77\\
0.989950171043167	1.78\\
0.99	1.78009876382159\\
0.995006126132691	1.79\\
1	1.8\\
1	1.8\\
1.00500668750402	1.81\\
1.00995116863269	1.82\\
1.01	1.82009876382159\\
1.01490931827387	1.83\\
1.01980657594337	1.84\\
1.02	1.84039512801392\\
1.02471679532232	1.85\\
1.02956893878086	1.86\\
1.03	1.86088931103097\\
1.03443171688911	1.87\\
1.03924080029275	1.88\\
1.04	1.88158167786906\\
1.04405651049578	1.89\\
1.0488245376416	1.9\\
1.05	1.90247274143302\\
1.05359344015177	1.91\\
1.05832236902982	1.92\\
1.06	1.92356316446219\\
1.06304461288528	1.93\\
1.0677363600728	1.94\\
1.07	1.94485376202975\\
1.07241198463478	1.95\\
1.07706842955509	1.96\\
1.08	1.96634550463311\\
1.08169736553222	1.97\\
1.08632035459985	1.98\\
1.09	1.98803952189735\\
1.09090242460493	1.99\\
1.095493775278	2\\
64	893\\
1.67242966687641	2\\
1.67	1.99171949237939\\
1.66947728018977	1.99\\
1.66644436105364	1.98\\
1.66344952518397	1.97\\
1.66049136944634	1.96\\
1.66	1.95833584811391\\
1.65745114540683	1.95\\
1.65442571873672	1.94\\
1.65143764838627	1.93\\
1.65	1.92516065830721\\
1.64841252131826	1.92\\
1.64535675632543	1.91\\
1.64233901084299	1.9\\
1.64	1.8921747797634\\
1.63932696091939	1.89\\
1.6362411396476	1.88\\
1.63319398665579	1.87\\
1.63018407890677	1.86\\
1.63	1.8593880845492\\
1.6270773653485	1.85\\
1.62400110030719	1.84\\
1.62096268548628	1.83\\
1.62	1.82681971478609\\
1.61786392706677	1.82\\
1.61475887285633	1.81\\
1.61169225742535	1.8\\
1.61	1.79444449253544\\
1.60859931747118	1.79\\
1.60546582386265	1.78\\
1.60237133987065	1.77\\
1.6	1.7622641509434\\
1.59928203027479	1.76\\
1.5961204732865	1.75\\
1.59299847775237	1.74\\
1.59	1.73028029882739\\
1.58991056221228	1.73\\
1.5867213433534	1.72\\
1.58357221754171	1.71\\
1.58046170731967	1.7\\
1.58	1.69851320189668\\
1.57726696036831	1.69\\
1.57409110895948	1.68\\
1.57095436149959	1.67\\
1.57	1.66694649956245\\
1.56775585646622	1.66\\
1.56455370662295	1.65\\
1.56139112759875	1.64\\
1.56	1.6355778612572\\
1.55818657128533	1.63\\
1.5549585716178	1.62\\
1.55177058710137	1.61\\
1.55	1.60440830721003\\
1.54855765354925	1.6\\
1.54530427298245	1.59\\
1.54209132836475	1.58\\
1.54	1.57343871453678\\
1.53886766254502	1.57\\
1.53558938909213	1.56\\
1.53235194791451	1.55\\
1.53	1.54266981404699\\
1.52911516948385	1.54\\
1.52581250893047	1.53\\
1.52255105163184	1.52\\
1.52	1.51210218757858\\
1.51929875873233	1.51\\
1.51597223323667	1.5\\
1.51268725582118	1.49\\
1.51	1.48173626587851\\
1.50941702890204	1.48\\
1.50606717551713	1.47\\
1.50275918814801	1.46\\
1.5	1.45157232704403\\
1.4994685937865	1.45\\
1.49609596291081	1.44\\
1.49276548843629	1.43\\
1.49	1.42161049553515\\
1.48945208313499	1.42\\
1.4860572368984	1.41\\
1.48270480931631	1.4\\
1.48	1.39185074176515\\
1.47936614325374	1.39\\
1.47594965384628	1.38\\
1.4725758167142	1.37\\
1.47	1.36229288227164\\
1.46920943742588	1.36\\
1.46577188537695	1.35\\
1.46237719017517	1.34\\
1.46	1.33293658046698\\
1.45898064614244	1.33\\
1.45552261855867	1.32\\
1.45210762301347	1.31\\
1.45	1.30378134796238\\
1.44867846713801	1.3\\
1.44520055590788	1.29\\
1.44176582228302	1.28\\
1.44	1.2748265464563\\
1.4383016152254	1.27\\
1.4348044151993	1.26\\
1.43135050856344	1.25\\
1.43	1.24607139017377\\
1.42784882192503	1.24\\
1.42433292907908	1.23\\
1.42086041555732	1.22\\
1.42	1.21751494883953\\
1.41731883488576	1.21\\
1.413784844478	1.2\\
1.41029428949537	1.19\\
1.41	1.18915615116424\\
1.40671041709492	1.18\\
1.40315892182243	1.17\\
1.4	1.16100628930818\\
1.39963587198078	1.16\\
1.39602234587658	1.15\\
1.39245393404161	1.14\\
1.39	1.13306449002635\\
1.38888323402913	1.13\\
1.38525341167786	1.12\\
1.38166866537135	1.11\\
1.38	1.10531858894171\\
1.37804793076336	1.1\\
1.37440241664447	1.09\\
1.37080190995449	1.08\\
1.37	1.0777669347799\\
1.36712878462214	1.07\\
1.36346817298707	1.06\\
1.36	1.05041288698716\\
1.35984624220649	1.05\\
1.35612462806804	1.04\\
1.3524495011416	1.03\\
1.35	1.02327978056426\\
1.34876969343053	1.02\\
1.34503430149503	1.01\\
1.34134522772674	1\\
1.34	0.996337346989757\\
1.33760523822279	0.99\\
1.33385665093114	0.98\\
1.33015418330316	0.97\\
1.33	0.969583412115935\\
1.3263517436595	0.96\\
1.32259052554166	0.95\\
1.32	0.943053523952846\\
1.31882875253592	0.94\\
1.31500807889121	0.93\\
1.31123477493898	0.92\\
1.31	0.916714648608164\\
1.30740504715471	0.91\\
1.30357311218786	0.9\\
1.3	0.890566037735849\\
1.2997795640708	0.89\\
1.2958883756883	0.88\\
1.29204570779943	0.87\\
1.29	0.864641352771868\\
1.28817866170684	0.86\\
1.28427762010123	0.85\\
1.28042472263967	0.84\\
1.28	0.838896290764251\\
1.27648204483285	0.83\\
1.27257165287539	0.82\\
1.27	0.813369726782805\\
1.26865697876979	0.81\\
1.26468860161324	0.8\\
1.26076933326675	0.79\\
1.26	0.788032494393221\\
1.25677353599661	0.78\\
1.25279720426616	0.77\\
1.25	0.762903605015674\\
1.24882436158868	0.76\\
1.24479055557965	0.75\\
1.24080670491259	0.74\\
1.24	0.737970196860205\\
1.23674726783424	0.73\\
1.23270689882452	0.72\\
1.23	0.71324439779421\\
1.22866516357992	0.71\\
1.22456794269398	0.7\\
1.22052139910503	0.69\\
1.22	0.688709584266866\\
1.21638763833947	0.68\\
1.21228522373125	0.67\\
1.21	0.664391071205106\\
1.20816373308675	0.66\\
1.20400518762237	0.65\\
1.2	0.640251572327044\\
1.19989392112346	0.64\\
1.19567902091328	0.63\\
1.19151661024437	0.62\\
1.19	0.616340163525153\\
1.18730440274437	0.61\\
1.1830866662561	0.6\\
1.18	0.592614597441686\\
1.17887896267161	0.59\\
1.17460574450658	0.58\\
1.1703854120368	0.57\\
1.17	0.56908583779077\\
1.16607146189913	0.56\\
1.16179667656905	0.55\\
1.16	0.545775268548589\\
1.15748138895978	0.54\\
1.15315205641169	0.53\\
1.15	0.522652821316614\\
1.14883305124315	0.52\\
1.14444911075469	0.51\\
1.14011939177551	0.5\\
1.14	0.499724185930898\\
1.13568535509016	0.49\\
1.13130206649774	0.48\\
1.13	0.477018711809452\\
1.12685826245316	0.47\\
1.12242139515012	0.46\\
1.12	0.454505479109332\\
1.11796526462525	0.45\\
1.11347484026283	0.44\\
1.11	0.432186306611466\\
1.10900375327887	0.43\\
1.10445982362244	0.42\\
1.1	0.410062893081761\\
1.09997108103989	0.41\\
1.09537372716813	0.4\\
1.09083355168283	0.39\\
1.09	0.388160011206574\\
1.08621389377962	0.38\\
1.08162133308988	0.37\\
1.08	0.366454304966309\\
1.07697762793273	0.36\\
1.07233279521726	0.35\\
1.07	0.344946249531191\\
1.06766219619742	0.34\\
1.06296522879555	0.33\\
1.06	0.323636964688205\\
1.05826482755242	0.32\\
1.05351588553412	0.31\\
1.05	0.302527429467084\\
1.04878271348978	0.3\\
1.0439819780651	0.29\\
1.04	0.281618478533769\\
1.03921300788195	0.28\\
1.03436067965189	0.27\\
1.03	0.260910799095363\\
1.02955282658347	0.26\\
1.02464912363527	0.25\\
1.02	0.240404928337943\\
1.01979924673896	0.24\\
1.01484440258947	0.23\\
1.01	0.220101251414916\\
1.0099493057683	0.22\\
1.00494356715991	0.21\\
1	0.2\\
1	0.2\\
0.994943624554387	0.19\\
0.99	0.180101251414917\\
0.989948282923022	0.18\\
0.984841536658436	0.17\\
0.98	0.160404928337943\\
0.979791063027179	0.16\\
0.974634217745797	0.15\\
0.97	0.140910799095364\\
0.96952520120189	0.14\\
0.964318531754053	0.13\\
0.96	0.121618478533768\\
0.959147507662665	0.12\\
0.953891289095314	0.11\\
0.95	0.102527429467085\\
0.948654738374353	0.1\\
0.943349242971248	0.0899999999999999\\
0.94	0.0836369646882044\\
0.938043590939632	0.0800000000000001\\
0.932689085161121	0.0699999999999998\\
0.93	0.0649462495311918\\
0.927310699920643	0.0600000000000001\\
0.921907441250854	0.0499999999999998\\
0.92	0.0464543049663088\\
0.9164526315609	0.04\\
0.911000865270156	0.0299999999999998\\
0.91	0.0281600112065748\\
0.905465877873607	0.02\\
0.9	0.0100628930817609\\
0.899964662550325	0.00999999999999979\\
0.894346850061365	0\\
0.89	-0.00781369338853312\\
0.88875778608991	-0.01\\
0.883091871230693	-0.02\\
0.88	-0.0254945208906682\\
0.877410982020109	-0.03\\
0.871697168363046	-0.04\\
0.87	-0.0429812881905472\\
0.865920402138991	-0.05\\
0.860158863501733	-0.0600000000000001\\
0.86	-0.0602758140691029\\
0.854282087382171	-0.0700000000000001\\
0.85	-0.0773471786833855\\
0.848421640047016	-0.0800000000000001\\
0.842491957346029	-0.0900000000000001\\
0.84	-0.0942247314514117\\
0.836522914892377	-0.1\\
0.830545798883366	-0.11\\
0.83	-0.11091416220923\\
0.824463745596698	-0.12\\
0.82	-0.127385402558315\\
0.818387286288631	-0.13\\
0.812239664954902	-0.14\\
0.81	-0.143659836474847\\
0.806040951414771	-0.15\\
0.8	-0.159748427672956\\
0.799840930176699	-0.16\\
0.793520312250262	-0.17\\
0.79	-0.175608928794894\\
0.787188292917455	-0.18\\
0.780820408930391	-0.19\\
0.78	-0.191290415733134\\
0.774351372241944	-0.2\\
0.77	-0.20675560220579\\
0.767868295203529	-0.21\\
0.761324829783298	-0.22\\
0.76	-0.222029803139796\\
0.754694140095743	-0.23\\
0.75	-0.237096394984326\\
0.748041049866146	-0.24\\
0.741319259516389	-0.25\\
0.74	-0.251967505606778\\
0.734507298391649	-0.26\\
0.73	-0.266630273217195\\
0.727663910230094	-0.27\\
0.720760844057086	-0.28\\
0.72	-0.281103709235748\\
0.713746569694318	-0.29\\
0.71	-0.295358647228132\\
0.706691108542717	-0.3\\
0.7	-0.309433962264151\\
0.699590644684589	-0.31\\
0.692364282990452	-0.32\\
0.69	-0.323285351391836\\
0.685073278852561	-0.33\\
0.68	-0.336946476047153\\
0.677726275881228	-0.34\\
0.670308836175594	-0.35\\
0.67	-0.350416587884065\\
0.662756866146285	-0.36\\
0.66	-0.363662653010242\\
0.655136918189351	-0.37\\
0.65	-0.376720219435737\\
0.647444067524988	-0.38\\
0.64	-0.389587113012836\\
0.639673154658278	-0.39\\
0.631761667129868	-0.4\\
0.63	-0.402233065220102\\
0.623753345452266	-0.41\\
0.62	-0.414681411058294\\
0.615652667823793	-0.42\\
0.61	-0.426935509973655\\
0.607453580611517	-0.43\\
0.6	-0.438993710691824\\
0.599149722676737	-0.44\\
0.590711289561641	-0.45\\
0.59	-0.450843848835762\\
0.58213114745327	-0.46\\
0.58	-0.462485051160469\\
0.573428809134777	-0.47\\
0.57	-0.473928609826229\\
0.564596693934429	-0.48\\
0.56	-0.485173453543701\\
0.555626794404945	-0.49\\
0.55	-0.496218652037618\\
0.54651064034125	-0.5\\
0.54	-0.507063419533015\\
0.537239258264859	-0.51\\
0.53	-0.517707117728358\\
0.527803125742988	-0.52\\
0.52	-0.52814925823485\\
0.518192119814678	-0.53\\
0.51	-0.538389504464847\\
0.508395458683379	-0.54\\
0.5	-0.548427672955975\\
0.498401635702088	-0.55\\
0.49	-0.558263734121494\\
0.48819834451887	-0.56\\
0.48	-0.567897812421423\\
0.477772394062222	-0.57\\
0.47	-0.577330185953009\\
0.467109611820516	-0.58\\
0.46	-0.586561285463219\\
0.45619473359954	-0.59\\
0.45	-0.595591692789968\\
0.44501127761668	-0.6\\
0.44	-0.6044221387428\\
0.433541400396551	-0.61\\
0.43	-0.613053500437555\\
0.421765731454759	-0.62\\
0.42	-0.621486798103319\\
0.41	-0.629719701172614\\
0.409651138795146	-0.63\\
0.4	-0.637735849056604\\
0.397109963481178	-0.64\\
0.39	-0.645555507464559\\
0.384177994338561	-0.65\\
0.38	-0.653180285213911\\
0.370825755695324	-0.66\\
0.37	-0.660611915450804\\
0.36	-0.667825220236597\\
0.356907214465262	-0.67\\
0.35	-0.67483934169279\\
0.342445545981099	-0.68\\
0.34	-0.681664151886086\\
0.33	-0.688280507620607\\
0.327325539136696	-0.69\\
0.32	-0.694689139704038\\
0.311473581952072	-0.7\\
0.31	-0.700913612532767\\
0.3	-0.706918238993711\\
0.294710816595946	-0.71\\
0.29	-0.712730690777124\\
0.28	-0.718344744139148\\
0.276948397553197	-0.72\\
0.27	-0.723747706479509\\
0.26	-0.728964319959627\\
0.257937619574625	-0.73\\
0.25	-0.733961598746081\\
0.24	-0.738775069391875\\
0.237348173742036	-0.74\\
0.23	-0.743371719513723\\
0.22	-0.747777564910512\\
0.214729200762246	-0.75\\
0.21	-0.751979846076836\\
0.2	-0.755974842767296\\
0.19	-0.759787504740235\\
0.189411649415079	-0.76\\
0.18	-0.763372360170554\\
0.17	-0.766768982239577\\
0.16	-0.769977637237541\\
0.159925622251063	-0.77\\
0.15	-0.772958463949843\\
0.14	-0.775748703750315\\
0.13	-0.778346628362167\\
0.123114809702228	-0.78\\
0.12	-0.780740955716787\\
0.11	-0.782920957219922\\
0.1	-0.784905660377358\\
0.0899999999999999	-0.786693601059135\\
0.0800000000000001	-0.788283447055851\\
0.0699999999999998	-0.789674003038359\\
0.0672577407417258	-0.79\\
0.0600000000000001	-0.790853393438517\\
0.0499999999999998	-0.791829937304075\\
0.04	-0.79260697966357\\
0.0299999999999998	-0.793183936424174\\
0.02	-0.793560372139804\\
0.00999999999999979	-0.793736001760785\\
0	-0.793710691823899\\
-0.01	-0.793484461074079\\
-0.02	-0.79305748051295\\
-0.03	-0.792430072873477\\
-0.04	-0.791602711523977\\
-0.05	-0.790576018808777\\
-0.0546977823910494	-0.79\\
-0.0600000000000001	-0.789342531067715\\
-0.0700000000000001	-0.787901626788201\\
-0.0800000000000001	-0.786261713419257\\
-0.0900000000000001	-0.784424025974026\\
-0.1	-0.782389937106918\\
-0.11	-0.780160952201731\\
-0.110662689342579	-0.78\\
-0.12	-0.777710058272105\\
-0.13	-0.775063385528476\\
-0.14	-0.772224918197835\\
-0.147340199165886	-0.77\\
-0.15	-0.769186507936508\\
-0.16	-0.765930078421452\\
-0.17	-0.762486326993324\\
-0.176842457384025	-0.76\\
-0.18	-0.75884318008636\\
-0.19	-0.754984066489698\\
-0.2	-0.750943396226415\\
-0.202226903668601	-0.75\\
-0.21	-0.746682416677227\\
-0.22	-0.742231812452735\\
-0.224811183491037	-0.74\\
-0.23	-0.73757620890976\\
-0.24	-0.73271905122382\\
-0.245393537375383	-0.73\\
-0.25	-0.727662698412698\\
-0.26	-0.722403633610901\\
-0.264415289148459	-0.72\\
-0.27	-0.716941610610484\\
-0.28	-0.711286513738341\\
-0.282200479668005	-0.71\\
-0.29	-0.705415150171081\\
-0.298952461662716	-0.7\\
-0.3	-0.699363057324841\\
-0.31	-0.693087972443433\\
-0.314786447286567	-0.69\\
-0.32	-0.686620675641351\\
-0.329950173928396	-0.68\\
-0.33	-0.67996669855849\\
-0.34	-0.673084138628889\\
-0.344370575440339	-0.67\\
-0.35	-0.666011904761905\\
-0.358282012191122	-0.66\\
-0.36	-0.658748304868723\\
-0.37	-0.651275028412678\\
-0.371664138115742	-0.65\\
-0.38	-0.643593007347353\\
-0.384568796034166	-0.64\\
-0.39	-0.635716293990598\\
-0.397086290682667	-0.63\\
-0.4	-0.627643312101911\\
-0.409243133699618	-0.62\\
-0.41	-0.619372615247095\\
-0.42	-0.610891483447056\\
-0.421028001500796	-0.61\\
-0.43	-0.602204703126978\\
-0.432484309858452	-0.6\\
-0.44	-0.593319198579762\\
-0.443656701561648	-0.59\\
-0.45	-0.584234126984127\\
-0.45456285224193	-0.58\\
-0.46	-0.574948792270531\\
-0.465219085287167	-0.57\\
-0.47	-0.565462647919583\\
-0.47564051542633	-0.56\\
-0.48	-0.555775299210593\\
-0.485841174179134	-0.55\\
-0.49	-0.545886504903834\\
-0.49583411969584	-0.54\\
-0.5	-0.535796178343949\\
-0.505631533119913	-0.53\\
-0.51	-0.525504387976054\\
-0.515244803281723	-0.52\\
-0.52	-0.515011357270181\\
-0.524684601262226	-0.51\\
-0.53	-0.504317464053951\\
-0.533960946141162	-0.5\\
-0.54	-0.493423239257564\\
-0.54308326305671	-0.49\\
-0.55	-0.482329365079365\\
-0.552060434545997	-0.48\\
-0.56	-0.471036672584327\\
-0.560900846003179	-0.47\\
-0.569599973220338	-0.46\\
-0.57	-0.459540319271701\\
-0.578145063341717	-0.45\\
-0.58	-0.447831481955464\\
-0.586572478808269	-0.44\\
-0.59	-0.435924779721619\\
-0.594889192961418	-0.43\\
-0.6	-0.423821656050955\\
-0.603101831579851	-0.42\\
-0.61	-0.411523688222589\\
-0.611216696806069	-0.41\\
-0.619215448981531	-0.4\\
-0.62	-0.399020169361047\\
-0.62708656964657	-0.39\\
-0.63	-0.386303485100397\\
-0.634874695559935	-0.38\\
-0.64	-0.373395258730564\\
-0.642585189350913	-0.37\\
-0.65	-0.360297619047619\\
-0.650223170608038	-0.36\\
-0.65772281604909	-0.35\\
-0.66	-0.346974532410966\\
-0.665150521383956	-0.34\\
-0.67	-0.333460766679424\\
-0.672518054256748	-0.33\\
-0.679824340678321	-0.32\\
-0.68	-0.31975960895292\\
-0.686996461250112	-0.31\\
-0.69	-0.305830104980156\\
-0.694119719791078	-0.3\\
-0.7	-0.29171974522293\\
-0.701198093168162	-0.29\\
-0.708178374849071	-0.28\\
-0.71	-0.27739840200231\\
-0.715081463850045	-0.27\\
-0.72	-0.262878121010978\\
-0.721949703508821	-0.26\\
-0.728747130437829	-0.25\\
-0.73	-0.248160811158247\\
-0.735452374459716	-0.24\\
-0.74	-0.233234244824943\\
-0.742132053880874	-0.23\\
-0.748749759548638	-0.22\\
-0.75	-0.218114951768489\\
-0.755277932418206	-0.21\\
-0.76	-0.202787017633529\\
-0.761789148806582	-0.2\\
-0.768229991713619	-0.19\\
-0.77	-0.187260952564597\\
-0.774600497248031	-0.18\\
-0.78	-0.171537809548124\\
-0.780962034717162	-0.17\\
-0.787228684356836	-0.16\\
-0.79	-0.155601456809139\\
-0.793459697267224	-0.15\\
-0.799678846287403	-0.14\\
-0.8	-0.139483870967742\\
-0.805784165630173	-0.13\\
-0.81	-0.123141588785047\\
-0.811892740136254	-0.12\\
-0.81794022313312	-0.11\\
-0.82	-0.106608078209416\\
-0.823932505181531	-0.1\\
-0.829932467451917	-0.0900000000000001\\
-0.83	-0.0898874466985399\\
-0.835812930757792	-0.0800000000000001\\
-0.84	-0.0729391749935945\\
-0.841707725636391	-0.0700000000000001\\
-0.847538339711193	-0.0600000000000001\\
-0.85	-0.0557998392282957\\
-0.853330052051275	-0.05\\
-0.859112909770086	-0.04\\
-0.86	-0.0384687838884585\\
-0.864805722982588	-0.03\\
-0.87	-0.0209319542140936\\
-0.870522872698245	-0.02\\
-0.87613869980289	-0.01\\
-0.88	-0.00318234539389288\\
-0.881765097912934	0\\
-0.887332830030937	0.00999999999999979\\
-0.89	0.0147621992021617\\
-0.892872413735917	0.02\\
-0.898391857222652	0.0299999999999998\\
-0.9	0.0329032258064514\\
-0.903848493891915	0.04\\
-0.909319430019408	0.0499999999999998\\
-0.91	0.0512421484930797\\
-0.914696921820235	0.0600000000000001\\
-0.92	0.0697830560532376\\
-0.920115108109071	0.0699999999999998\\
-0.925421198216892	0.0800000000000001\\
-0.93	0.0885376394409538\\
-0.930767546213652	0.0899999999999999\\
-0.936024747861144	0.1\\
-0.94	0.107490983817108\\
-0.941302855214485	0.11\\
-0.946510925762632	0.12\\
-0.95	0.126643890675241\\
-0.951724333899107	0.13\\
-0.956883022663821	0.14\\
-0.96	0.145997012619109\\
-0.962035218540553	0.15\\
-0.967144269931125	0.16\\
-0.97	0.165550850625081\\
-0.972238687304132	0.17\\
-0.977297843867148	0.18\\
-0.98	0.185305751870002\\
-0.982337864115085	0.19\\
-0.987346869475701	0.2\\
-0.99	0.205261908140885\\
-0.99233582201807	0.21\\
-0.99729442371064	0.22\\
-1	0.22541935483871\\
-1.00223558605889	0.23\\
-1.00714353823914	0.24\\
-1.01	0.245777970584441\\
-1.01204013571833	0.25\\
-1.01689720174964	0.26\\
-1.02	0.266337477431003\\
-1.02175240692776	0.27\\
-1.02655836183447	0.28\\
-1.03	0.287097441680629\\
-1.03137529369556	0.29\\
-1.03612992647684	0.3\\
-1.04	0.308057275302601\\
-1.04091164937343	0.31\\
-1.04561476517173	0.32\\
-1.05	0.329216237942122\\
-1.05036428759097	0.33\\
-1.05501570970987	0.34\\
-1.05972618424486	0.35\\
-1.06	0.350580900338278\\
-1.06433555465392	0.36\\
-1.0689933257608	0.37\\
-1.07	0.372155481231328\\
-1.07357705753533	0.38\\
-1.07818196379801	0.39\\
-1.08	0.393928545501685\\
-1.08274293880025	0.4\\
-1.08729484378916	0.41\\
-1.09	0.415898725908679\\
-1.09183588153227	0.42\\
-1.09633467523317	0.43\\
-1.1	0.438064516129032\\
-1.10085853097909	0.44\\
-1.10530413104856	0.45\\
-1.10980639418005	0.46\\
-1.11	0.460429807065571\\
-1.11420584675716	0.47\\
-1.11865385134322	0.48\\
-1.12	0.483014920717442\\
-1.1230424195249	0.49\\
-1.12743612319646	0.5\\
-1.13	0.505792596191216\\
-1.13181640708566	0.51\\
-1.13615579986716	0.52\\
-1.14	0.528760753937516\\
-1.14053032657304	0.53\\
-1.14481543145822	0.54\\
-1.14915509638774	0.55\\
-1.15	0.551942182410423\\
-1.15341752663771	0.56\\
-1.15770183746985	0.57\\
-1.16	0.5753279003374\\
-1.1619645511877	0.58\\
-1.1661935868969	0.59\\
-1.17	0.598899218245251\\
-1.17045892653442	0.6\\
-1.17463280185402	0.61\\
-1.17885983217058	0.62\\
-1.18	0.622688141777431\\
-1.18302189358434	0.63\\
-1.18719271079811	0.64\\
-1.19	0.64667371936195\\
-1.19136322578468	0.65\\
-1.19547801061537	0.66\\
-1.19964552341059	0.67\\
-1.2	0.670849673202614\\
-1.2037180832431	0.68\\
-1.20782851305869	0.69\\
-1.21	0.69524756858666\\
-1.21191522908138	0.7\\
-1.21596882923884	0.71\\
-1.22	0.719820636152056\\
-1.22007169575818	0.72\\
-1.22406875730399	0.73\\
-1.22811664812942	0.74\\
-1.23	0.74462521812736\\
-1.23213052741884	0.75\\
-1.23612088255402	0.76\\
-1.24	0.769602485115196\\
-1.24015631310362	0.77\\
-1.24408950166622	0.78\\
-1.2480727550158	0.79\\
-1.25	0.794808631921824\\
-1.25202465743749	0.8\\
-1.25594977521141	0.81\\
-1.25992549507216	0.82\\
-1.26	0.820187359034881\\
-1.26379587421518	0.83\\
-1.26771247878751	0.84\\
-1.27	0.845797245735122\\
-1.27161311718772	0.85\\
-1.2754711054688	0.86\\
-1.27937872164726	0.87\\
-1.28	0.87158669111868\\
-1.28320341153513	0.88\\
-1.28705147498121	0.89\\
-1.29	0.897587966454297\\
-1.29091136641711	0.9\\
-1.29470046295064	0.91\\
-1.2985380923971	0.92\\
-1.3	0.923790849673203\\
-1.30232762069877	0.93\\
-1.30610539853734	0.94\\
-1.30993205718552	0.95\\
-1.31	0.950177512810406\\
-1.31365341211053	0.96\\
-1.31741933759163	0.97\\
-1.32	0.976792911128485\\
-1.3211839579357	0.98\\
-1.32488987914322	0.99\\
-1.32864339513433	1\\
-1.33	1.00359746694594\\
-1.33234546333007	1.01\\
-1.33603815778767	1.02\\
-1.33977868526705	1.03\\
-1.34	1.0305912174599\\
-1.34342048504573	1.04\\
-1.34709938461561	1.05\\
-1.35	1.05780537459283\\
-1.35079203374516	1.06\\
-1.35441018067958	1.07\\
-1.35807469967244	1.08\\
-1.36	1.08521851948731\\
-1.36171267753432	1.09\\
-1.36531571223674	1.1\\
-1.36896524917399	1.11\\
-1.37	1.11282491796824\\
-1.37255090963177	1.12\\
-1.37613824855385	1.13\\
-1.3797721879995	1.14\\
-1.38	1.14062638925467\\
-1.38330791938761	1.15\\
-1.3868789675184	1.16\\
-1.39	1.16864255638118\\
-1.39047566624085	1.17\\
-1.39398490535133	1.18\\
-1.39753905807799	1.19\\
-1.4	1.19686274509804\\
-1.40109134041841	1.2\\
-1.40458307702264	1.21\\
-1.40811972203485	1.22\\
-1.41	1.22528017953086\\
-1.41163000836118	1.23\\
-1.41510365639115	1.24\\
-1.41862217562096	1.25\\
-1.42	1.25389608615708\\
-1.42209290946939	1.26\\
-1.42554787926203	1.27\\
-1.42904765085084	1.28\\
-1.43	1.28271154757205\\
-1.43248129617994	1.29\\
-1.43591699636584	1.3\\
-1.43939739664942	1.31\\
-1.44	1.31172749802267\\
-1.44279643485214	1.32\\
-1.44621227425155	1.33\\
-1.44967267975399	1.34\\
-1.45	1.34094471947195\\
-1.45303960644588	1.35\\
-1.45643499596308	1.36\\
-1.45987478539008	1.37\\
-1.46	1.37036383822363\\
-1.46321210699357	1.38\\
-1.46658646150069	1.39\\
-1.47	1.39998551377464\\
-1.47000479697179	1.4\\
-1.4733152478686	1.41\\
-1.47666798806966	1.42\\
-1.48	1.42981196759864\\
-1.48006183877104	1.43\\
-1.48335035585388	1.44\\
-1.48668091012008	1.45\\
-1.49	1.45983858972683\\
-1.49005272270428	1.46\\
-1.49331877301537	1.47\\
-1.49662657918262	1.48\\
-1.4999778039149	1.49\\
-1.5	1.49006622516556\\
-1.5032218563864	1.5\\
-1.50650636350639	1.51\\
-1.50983396691862	1.52\\
-1.51	1.52049861607734\\
-1.51306097746983	1.53\\
-1.51632164750606	1.54\\
-1.51962506200023	1.55\\
-1.52	1.55113328038125\\
-1.52283752156604	1.56\\
-1.52607383102663	1.57\\
-1.52935250431347	1.58\\
-1.53	1.58196971160206\\
-1.53255288693558	1.59\\
-1.53576432843531	1.6\\
-1.53901772518038	1.61\\
-1.54	1.61300724292889\\
-1.54220848380669	1.62\\
-1.54539456755082	1.63\\
-1.548622170938	1.64\\
-1.55	1.64424504950495\\
-1.55180573323798	1.65\\
-1.55496598842131	1.66\\
-1.55816730162665	1.67\\
-1.56	1.6756821513314\\
-1.56134606584813	1.68\\
-1.56448004196324	1.69\\
-1.56765458953245	1.7\\
-1.57	1.70731741676536\\
-1.57083092042445	1.71\\
-1.57393818847364	1.72\\
-1.5770855175977	1.73\\
-1.58	1.73914956658786\\
-1.58026174242309	1.74\\
-1.58334189602947	1.75\\
-1.58646157771318	1.76\\
-1.58962236774484	1.77\\
-1.59	1.77119280971983\\
-1.5926926387876	1.78\\
-1.59578426890689	1.79\\
-1.59891642715192	1.8\\
-1.6	1.80344370860927\\
-1.60199189519975	1.81\\
-1.60505509544448	1.82\\
-1.60815822242454	1.83\\
-1.61	1.8358891890107\\
-1.61124114615683	1.84\\
-1.61427556485635	1.85\\
-1.61734928914172	1.86\\
-1.62	1.86852736370819\\
-1.62044187307687	1.87\\
-1.62344718590677	1.88\\
-1.62649116485611	1.89\\
-1.62957535919946	1.9\\
-1.63	1.90137425854502\\
-1.6325714665205	1.91\\
-1.63558538689343	1.92\\
-1.63863883481512	1.93\\
-1.64	1.9344315928969\\
-1.64164991168171	1.94\\
-1.64463349012443	1.95\\
-1.64765589211733	1.96\\
-1.65	1.96767739273927\\
-1.65068402132028	1.97\\
-1.65363700472488	1.98\\
-1.65662809311844	1.99\\
-1.65965881667604	2\\
128	364\\
-0.940961303584167	2\\
-0.94	1.99820723411913\\
-0.935587291873528	1.99\\
-0.930170135019851	1.98\\
-0.93	1.97968605981443\\
-0.924735098583825	1.97\\
-0.92	1.96135483640265\\
-0.919255449260802	1.96\\
-0.913752910309305	1.95\\
-0.91	1.94322203867848\\
-0.908209488227967	1.94\\
-0.902635847170067	1.93\\
-0.9	1.92529147982063\\
-0.897026165459617	1.92\\
-0.891378742522204	1.91\\
-0.89	1.90756428699793\\
-0.885700119649759	1.9\\
-0.88	1.89004133499911\\
-0.879976232289348	1.89\\
-0.874225672290934	1.88\\
-0.87	1.87270073785887\\
-0.868428860379479	1.87\\
-0.862596807071013	1.86\\
-0.86	1.8555656446118\\
-0.856723801220621	1.85\\
-0.850807147579729	1.84\\
-0.85	1.83863764044944\\
-0.844854432824964	1.83\\
-0.84	1.82190130427014\\
-0.838853672691431	1.82\\
-0.832813731950863	1.81\\
-0.83	1.80536143266732\\
-0.8267277472381	1.8\\
-0.820594246424536	1.79\\
-0.82	1.78903210497933\\
-0.814418775660954	1.78\\
-0.81	1.77288935114845\\
-0.808192185884892	1.77\\
-0.80191853072146	1.76\\
-0.8	1.75695067264574\\
-0.79559556612746	1.75\\
-0.79	1.7412160719065\\
-0.789219499695809	1.74\\
-0.782793768552423	1.73\\
-0.78	1.72566963997851\\
-0.776313438550467	1.72\\
-0.77	1.71033802245789\\
-0.769777306773797	1.71\\
-0.763186696032241	1.7\\
-0.76	1.69518625380347\\
-0.75653727993585	1.69\\
-0.75	1.68025334075724\\
-0.749828558130927	1.68\\
-0.743059254160493	1.67\\
-0.74	1.66549948093789\\
-0.7362269380312	1.66\\
-0.73	1.65096185277604\\
-0.72933090109931	1.65\\
-0.722367934841255	1.64\\
-0.72	1.63660999462654\\
-0.715336516787376	1.63\\
-0.71	1.62246506824873\\
-0.708235920167757	1.62\\
-0.701063000838482	1.61\\
-0.7	1.6085201793722\\
-0.693813535398462	1.6\\
-0.69	1.59476619447672\\
-0.686488229938874	1.59\\
-0.68	1.58122316052022\\
-0.679085271188727	1.58\\
-0.671597922187921	1.57\\
-0.67	1.56787016844369\\
-0.664024388460211	1.56\\
-0.66	1.55471531177416\\
-0.656364529240933	1.55\\
-0.65	1.54176781737194\\
-0.648615660658183	1.54\\
-0.640771590913491	1.53\\
-0.64	1.5290173211404\\
-0.63282562338503	1.52\\
-0.63	1.51645711027636\\
-0.624779449541284	1.51\\
-0.62	1.50410165982509\\
-0.61662928162716	1.5\\
-0.61	1.49194967940155\\
-0.60837097481676	1.49\\
-0.6	1.48\\
-0.6	1.48\\
-0.591501721949299	1.47\\
-0.59	1.46823591637568\\
-0.58287981237192	1.46\\
-0.58	1.45667398533882\\
-0.574128584096374	1.45\\
-0.57	1.44531336041425\\
-0.565241839706267	1.44\\
-0.56	1.43415330836454\\
-0.556212827219065	1.43\\
-0.55	1.42319320712695\\
-0.54703419053415	1.42\\
-0.54	1.41243254406267\\
-0.537697913968304	1.41\\
-0.53	1.40187091450939\\
-0.528195260090061	1.4\\
-0.52	1.39150802062956\\
-0.518516699938995	1.39\\
-0.51	1.38134367054849\\
-0.508651834568004	1.38\\
-0.5	1.37137777777778\\
-0.498589306671533	1.37\\
-0.49	1.36161036092097\\
-0.488316700854661	1.36\\
-0.48	1.35204154365997\\
-0.477820430850142	1.35\\
-0.47	1.3426715550218\\
-0.467085611694062	1.34\\
-0.46	1.33350072992701\\
-0.45609591451499	1.33\\
-0.45	1.32452951002227\\
-0.444833401162805	1.32\\
-0.44	1.31575844480114\\
-0.433278335384784	1.31\\
-0.43	1.30718819301848\\
-0.421408966626565	1.3\\
-0.42	1.29881952440551\\
-0.41	1.29064752418138\\
-0.409187394449366	1.29\\
-0.4	1.28266666666667\\
-0.396567389860413	1.28\\
-0.39	1.27488849853059\\
-0.383541227030319	1.27\\
-0.38	1.26731422452258\\
-0.370074240474713	1.26\\
-0.37	1.25994517037832\\
-0.36	1.25275805935667\\
-0.356043212193951	1.25\\
-0.35	1.24577672605791\\
-0.341467989376393	1.24\\
-0.34	1.23900332320886\\
-0.33	1.2324179868573\\
-0.32620543425623	1.23\\
-0.32	1.22603342241226\\
-0.310225284934463	1.22\\
-0.31	1.21986048351059\\
-0.3	1.21386666666667\\
-0.293303473495995	1.21\\
-0.29	1.20808555178874\\
-0.28	1.20249842002485\\
-0.275352505809946	1.2\\
-0.27	1.1971110373407\\
-0.26	1.19192577612205\\
-0.25613111429689	1.19\\
-0.25	1.18693485523385\\
-0.24	1.1821475252794\\
-0.235310474401127	1.18\\
-0.23	1.1775566304251\\
-0.22	1.17316413988993\\
-0.212432596523815	1.17\\
-0.21	1.16897769203319\\
-0.2	1.16497777777778\\
-0.19	1.1611904464523\\
-0.186679096317438	1.16\\
-0.18	1.15759230358097\\
-0.17	1.15419402806145\\
-0.16	1.15100435629538\\
-0.156638997079952	1.15\\
-0.15	1.14800389755011\\
-0.14	1.14520163497423\\
-0.13	1.14260499512455\\
-0.12	1.14021255970281\\
-0.119030977510308	1.14\\
-0.11	1.13800542790988\\
-0.1	1.136\\
-0.0900000000000001	1.13419708492324\\
-0.0800000000000001	1.13259571150097\\
-0.0700000000000001	1.13119502256437\\
-0.060047753274746	1.13\\
-0.0600000000000001	1.12999422150883\\
-0.05	1.12898385300668\\
-0.04	1.1281740430835\\
-0.03	1.12756435815319\\
-0.02	1.1271544727014\\
-0.01	1.12694416837052\\
0	1.12693333333333\\
0.00999999999999979	1.12712196195217\\
0.02	1.12751015472168\\
0.0299999999999998	1.1280981184948\\
0.04	1.12888616699306\\
0.0499999999999998	1.12987472160356\\
0.0510540458495884	1.13\\
0.0600000000000001	1.1310549054269\\
0.0699999999999998	1.13243385098664\\
0.0800000000000001	1.13401339712919\\
0.0899999999999999	1.13579438725708\\
0.1	1.13777777777778\\
0.11	1.13996464066257\\
0.110148151603064	1.14\\
0.12	1.14233532637538\\
0.13	1.14490975534084\\
0.14	1.14768963923938\\
0.14773958871305	1.15\\
0.15	1.15067052980132\\
0.16	1.15383771914291\\
0.17	1.15721329810852\\
0.177776778246567	1.16\\
0.18	1.16079208899876\\
0.19	1.16455655505359\\
0.2	1.16853333333333\\
0.203507582223573	1.17\\
0.21	1.17270058802723\\
0.22	1.17706969643174\\
0.226398356453778	1.18\\
0.23	1.18164145835174\\
0.24	1.18640510910059\\
0.247221482274059	1.19\\
0.25	1.19137693156733\\
0.26	1.196538158595\\
0.266439147525026	1.2\\
0.27	1.20190645261019\\
0.28	1.20746912835079\\
0.284382960244413	1.21\\
0.29	1.21323113891591\\
0.3	1.2192\\
0.301296102547634	1.22\\
0.31	1.22535379130127\\
0.317292743205336	1.23\\
0.32	1.23171915945612\\
0.33	1.23827892283101\\
0.332542635709765	1.24\\
0.34	1.24503304409421\\
0.347128784841382	1.25\\
0.35	1.25199503311258\\
0.36	1.25915578460992\\
0.361146224864349	1.26\\
0.37	1.2665053585675\\
0.374620272190563	1.27\\
0.38	1.27406007429683\\
0.387652346320494	1.28\\
0.39	1.28181860270103\\
0.4	1.28977777777778\\
0.400272340203024	1.29\\
0.41	1.29792412370219\\
0.412483648438101	1.3\\
0.42	1.30627283359915\\
0.424355898197935	1.31\\
0.43	1.31482302008672\\
0.435913169817137	1.32\\
0.44	1.32357390843203\\
0.44717727943786	1.33\\
0.45	1.33252483443709\\
0.458168027019556	1.34\\
0.46	1.34167524263279\\
0.468903412401656	1.35\\
0.47	1.35102468477207\\
0.479399824080637	1.36\\
0.48	1.36057281861449\\
0.489672204601407	1.37\\
0.49	1.37031940699621\\
0.499734195838304	1.38\\
0.5	1.38026431718062\\
0.5095982669265	1.39\\
0.51	1.39040752048639\\
0.51927582717906	1.4\\
0.52	1.40074909219108\\
0.528777325971563	1.41\\
0.53	1.41128921170973\\
0.538112341281654	1.42\\
0.54	1.42202816304923\\
0.547289658324496	1.43\\
0.55	1.43296633554084\\
0.556317339518262	1.44\\
0.56	1.44410422485416\\
0.565202786839535	1.45\\
0.57	1.45544243429785\\
0.573952797481202	1.46\\
0.58	1.46698167641326\\
0.582573613600606	1.47\\
0.59	1.47872277486911\\
0.591070966839431	1.48\\
0.599446833509056	1.49\\
0.6	1.49066079295154\\
0.607703323220905	1.5\\
0.61	1.50278956218554\\
0.615851916591596	1.51\\
0.62	1.51512145763312\\
0.62389680222546	1.52\\
0.63	1.52765773867565\\
0.631841812044632	1.53\\
0.639689136318961	1.54\\
0.64	1.54039626563326\\
0.647436228365007	1.55\\
0.65	1.55331953642384\\
0.655094715839015	1.56\\
0.66	1.56644972551797\\
0.662667327453879	1.57\\
0.67	1.57978855785454\\
0.670156532429177	1.58\\
0.677558278990436	1.59\\
0.68	1.5933084230973\\
0.684882425983756	1.6\\
0.69	1.60703684560191\\
0.692131304632878	1.61\\
0.699305297015524	1.62\\
0.7	1.62096916299559\\
0.706404610819918	1.63\\
0.71	1.63508806702626\\
0.713435395069459	1.64\\
0.72	1.64942190662169\\
0.720398810207234	1.65\\
0.727293997326345	1.66\\
0.73	1.66393807525837\\
0.734125752795884	1.67\\
0.74	1.67866695050559\\
0.740895250593382	1.68\\
0.74760314485109	1.69\\
0.75	1.69358443708609\\
0.754252199585117	1.7\\
0.76	1.70871130033706\\
0.760843120499879	1.71\\
0.767378056303719	1.72\\
0.77	1.72402640667786\\
0.773858356261687	1.73\\
0.78	1.73955505059471\\
0.780283780207338	1.74\\
0.786659014806658	1.75\\
0.79	1.75526491732249\\
0.792982419010293	1.76\\
0.799254716197308	1.77\\
0.8	1.77118942731278\\
0.805481285657476	1.78\\
0.81	1.78730259101781\\
0.811657811247412	1.79\\
0.817789072228091	1.8\\
0.82	1.80361744658308\\
0.823875706199544	1.81\\
0.829913962408542	1.82\\
0.83	1.82014250066015\\
0.835914991608755	1.83\\
0.84	1.83684816716841\\
0.84186918174727	1.84\\
0.847783082908991	1.85\\
0.85	1.85376103752759\\
0.853657241927799	1.86\\
0.859486971759897	1.87\\
0.86	1.8708806764136\\
0.865284641915196	1.88\\
0.87	1.88818960641787\\
0.871037301995441	1.89\\
0.876757731783599	1.9\\
0.88	1.90569637360693\\
0.882438329288811	1.91\\
0.888082505566394	1.92\\
0.89	1.92340744549387\\
0.893693958144204	1.93\\
0.899264623152458	1.94\\
0.9	1.94132158590308\\
0.904809641940468	1.95\\
0.91	1.95943268701748\\
0.910311024297215	1.96\\
0.915790532517272	1.97\\
0.92	1.97773482190324\\
0.921228605859052	1.98\\
0.926641497929665	1.99\\
0.93	1.99623882399788\\
0.932018472767508	2\\
128	978\\
1.76880955106278	2\\
1.76596090866775	1.99\\
1.76313744985315	1.98\\
1.76033851617718	1.97\\
1.76	1.96878940490906\\
1.75747269147385	1.96\\
1.75461984327982	1.95\\
1.75179200551342	1.94\\
1.75	1.93362804878049\\
1.74895084592115	1.93\\
1.74606860234103	1.92\\
1.7432118529284	1.91\\
1.7403799309859	1.9\\
1.74	1.89865696627229\\
1.73748245366843	1.89\\
1.7345968021164	1.88\\
1.73173644415041	1.87\\
1.73	1.86389724709637\\
1.72886010669713	1.86\\
1.72594557979614	1.85\\
1.72305680986589	1.84\\
1.72019312366903	1.83\\
1.72	1.82932528674784\\
1.71725689666919	1.82\\
1.71433975568111	1.81\\
1.71144814379197	1.8\\
1.71	1.7949702462138\\
1.70852944841197	1.79\\
1.70558399424667	1.78\\
1.70266451068707	1.77\\
1.7	1.7608\\
1.69976191669756	1.76\\
1.69678822429311	1.75\\
1.69384093953479	1.74\\
1.69091937057971	1.73\\
1.69	1.72684475731589\\
1.68795113163245	1.72\\
1.68497613262958	1.71\\
1.68202726772223	1.7\\
1.68	1.69308398084087\\
1.67907139017848	1.69\\
1.67606878035513	1.68\\
1.6730927174082	1.67\\
1.67014250740147	1.66\\
1.67	1.65951680483288\\
1.66711756217208	1.65\\
1.6641144149159	1.64\\
1.66113751454102	1.63\\
1.66	1.62616635414824\\
1.65812114761324	1.62\\
1.65509104548956	1.61\\
1.65208757771614	1.6\\
1.65	1.59300720620843\\
1.6490781972808	1.59\\
1.64602128529006	1.58\\
1.64299138807942	1.57\\
1.64	1.56004097456874\\
1.6399873638401	1.56\\
1.63690380234156	1.55\\
1.63384762845065	1.54\\
1.63081812470834	1.53\\
1.63	1.5272933165474\\
1.62773725746825	1.52\\
1.62465497420781	1.51\\
1.62159971454443	1.5\\
1.62	1.49474028687799\\
1.61852030521566	1.49\\
1.61541209416235	1.48\\
1.61233125175228	1.47\\
1.61	1.46238284535534\\
1.60925159475061	1.46\\
1.60611765141291	1.45\\
1.60301141284403	1.44\\
1.6	1.43022222222222\\
1.59992977073375	1.43\\
1.59677030417192	1.42\\
1.59363886907647	1.41\\
1.59053473158509	1.4\\
1.59	1.39827491395287\\
1.58736870655901	1.39\\
1.58421228720064	1.38\\
1.58108347274754	1.37\\
1.58	1.36652685168817\\
1.57791150935539	1.36\\
1.57473033017291	1.35\\
1.57157705316845	1.34\\
1.57	1.33497699353925\\
1.56839736071321	1.33\\
1.56519165782183	1.32\\
1.56201414390722	1.31\\
1.56	1.30362615629984\\
1.55882490681401	1.3\\
1.55559492746498	1.29\\
1.55239341298894	1.28\\
1.55	1.27247505543237\\
1.54919279247018	1.27\\
1.54593879447007	1.26\\
1.54271352592667	1.25\\
1.54	1.24152430321321\\
1.53949966166386	1.24\\
1.53622191275471	1.23\\
1.5329731461836	1.22\\
1.53	1.21077440714095\\
1.52974415801724	1.21\\
1.52644293521924	1.2\\
1.52317093556972	1.19\\
1.52	1.1802257686156\\
1.5199249251889	1.18\\
1.5166005141072	1.17\\
1.5133055545678	1.16\\
1.51003927622629	1.15\\
1.51	1.14987975068276\\
1.50669330128805	1.14\\
1.50337566258338	1.13\\
1.50008691494113	1.12\\
1.5	1.11973568281938\\
1.49671994845718	1.11\\
1.493379918114	1.1\\
1.49006897613769	1.09\\
1.49	1.08979165271782\\
1.4866791072481	1.08\\
1.48331697883264	1.07\\
1.48	1.06004805402524\\
1.47998359182231	1.06\\
1.47656942925212	1.05\\
1.47318550158089	1.04\\
1.47	1.03050795363709\\
1.46982538651482	1.03\\
1.46638956594103	1.02\\
1.46298414226742	1.01\\
1.46	1.0011692526185\\
1.45959532625914	1\\
1.45613816848851	0.99\\
1.45271155566749	0.98\\
1.45	0.972031596452329\\
1.44929205874186	0.97\\
1.44581388748618	0.96\\
1.44236639511972	0.95\\
1.44	0.943094524189261\\
1.43891423157545	0.94\\
1.43541537255059	0.93\\
1.43194731211615	0.92\\
1.43	0.914357469687584\\
1.42846049180725	0.91\\
1.42494127181765	0.9\\
1.4214529557825	0.89\\
1.42	0.885819763125331\\
1.4179294853099	0.88\\
1.4143902313212	0.87\\
1.41088197224503	0.86\\
1.41	0.857480632777337\\
1.40731985605021	0.85\\
1.4037608942529	0.84\\
1.40023300388138	0.83\\
1.4	0.829339207048458\\
1.3966302452339	0.82\\
1.3930519001005	0.81\\
1.39	0.80140688936208\\
1.38948871357394	0.8\\
1.38585929032366	0.79\\
1.38226188366235	0.78\\
1.38	0.773677775810164\\
1.37865382287527	0.77\\
1.3750056239284	0.76\\
1.37138947393559	0.75\\
1.37	0.746144805194805\\
1.36773451873663	0.74\\
1.36406787256178	0.73\\
1.36043329287641	0.72\\
1.36	0.71880669839591\\
1.35672942581622	0.71\\
1.35304465526838	0.7\\
1.35	0.691676829268293\\
1.34937265539025	0.69\\
1.34563716075919	0.68\\
1.3419345821158	0.67\\
1.34	0.664751176366531\\
1.33820929525137	0.66\\
1.33445633054487	0.65\\
1.33073625255166	0.64\\
1.33	0.638017554458065\\
1.32695565049155	0.63\\
1.32318553069524	0.62\\
1.32	0.611487386907929\\
1.31943094448088	0.61\\
1.31561031288286	0.6\\
1.31182334334374	0.59\\
1.31	0.585164976571479\\
1.30800896868169	0.58\\
1.30417185967486	0.57\\
1.30036833516357	0.56\\
1.3	0.559030837004405\\
1.29649214008373	0.55\\
1.2926388513623	0.54\\
1.29	0.533110357807103\\
1.28878247507118	0.53\\
1.28487901276991	0.52\\
1.2810098293091	0.51\\
1.28	0.507384224457384\\
1.27708707928472	0.5\\
1.27316812006938	0.49\\
1.27	0.481858059225109\\
1.26926126478626	0.48\\
1.26529215729605	0.47\\
1.2613579718665	0.46\\
1.26	0.456537948084054\\
1.25738012553312	0.45\\
1.25339620845053	0.44\\
1.25	0.431410753880266\\
1.24943016424346	0.43\\
1.24539617853302	0.42\\
1.24139770366093	0.41\\
1.24	0.4064938018718\\
1.23735599319485	0.4\\
1.23330788095499	0.39\\
1.23	0.38176939888288\\
1.22927371929068	0.38\\
1.2251756664265	0.37\\
1.22111365584583	0.36\\
1.22	0.357251879301218\\
1.21699909910321	0.35\\
1.21288758928612	0.34\\
1.21	0.332933225577894\\
1.20877617447848	0.33\\
1.20461489354338	0.32\\
1.20049012658154	0.31\\
1.2	0.308810572687225\\
1.19629353635364	0.3\\
1.19211944496645	0.29\\
1.19	0.284899748032888\\
1.18792144236115	0.28\\
1.18369778319698	0.27\\
1.18	0.261176938087636\\
1.17949649325356	0.26\\
1.17522303765546	0.25\\
1.1709871816022	0.24\\
1.17	0.237664789663991\\
1.16669306173467	0.23\\
1.16240786666126	0.22\\
1.16	0.214353157615426\\
1.15810566583997	0.21\\
1.15377092715546	0.2\\
1.15	0.191233370288248\\
1.14945861734227	0.19\\
1.14507414372339	0.18\\
1.14072822839334	0.17\\
1.14	0.168321946060286\\
1.13631525349979	0.16\\
1.13192005964774	0.15\\
1.13	0.14561472303207\\
1.12749194986114	0.14\\
1.12304730462504	0.13\\
1.12	0.12310224898176\\
1.11860188208488	0.12\\
1.11410762380614	0.11\\
1.11	0.100785826457279\\
1.10964265490935	0.1\\
1.10509863254675	0.0899999999999999\\
1.10059458440631	0.0800000000000001\\
1.1	0.0786784140969165\\
1.09601790044711	0.0699999999999998\\
1.09146453202494	0.0600000000000001\\
1.09	0.0567746050657486\\
1.08686295059323	0.0499999999999998\\
1.08226012052988	0.04\\
1.08	0.0350684815273114\\
1.07763125865511	0.0299999999999998\\
1.07297883329802	0.02\\
1.07	0.0135609390211519\\
1.06832025182623	0.00999999999999979\\
1.06361810434277	0\\
1.06	-0.00774722665248965\\
1.05892730758837	-0.01\\
1.054175317	-0.02\\
1.05	-0.0288553215077609\\
1.04944975228461	-0.03\\
1.04464780242174	-0.04\\
1.04	-0.0497627551926148\\
1.03988485948295	-0.05\\
1.0350328378602	-0.0600000000000001\\
1.03022350980589	-0.0700000000000001\\
1.03	-0.0704649132846206\\
1.0253276447236	-0.0800000000000001\\
1.02046862575941	-0.0900000000000001\\
1.02	-0.0909652281134403\\
1.01552938638496	-0.1\\
1.01062055439698	-0.11\\
1.01	-0.111265522861422\\
1.0056351657234	-0.12\\
1.00067639792359	-0.13\\
1	-0.13136563876652\\
0.995642022377326	-0.14\\
0.99063319388914	-0.15\\
0.99	-0.151265522861421\\
0.985546929687158	-0.16\\
0.980487912053665	-0.17\\
0.98	-0.17096522811344\\
0.975346791304673	-0.18\\
0.970237450919338	-0.19\\
0.97	-0.19046491328462\\
0.965038437444528	-0.2\\
0.96	-0.209762755192615\\
0.959875258177256	-0.21\\
0.954618620752343	-0.22\\
0.95	-0.22885532150776\\
0.949391792563152	-0.23\\
0.944084011762263	-0.24\\
0.94	-0.24774722665249\\
0.938790273100672	-0.25\\
0.933431193915292	-0.26\\
0.93	-0.266439060978847\\
0.928067204607767	-0.27\\
0.922656658107968	-0.28\\
0.92	-0.284931518472689\\
0.917218995474093	-0.29\\
0.911756796738959	-0.3\\
0.91	-0.303225394934251\\
0.906241951248647	-0.31\\
0.900727897218972	-0.32\\
0.9	-0.321321585903084\\
0.895132267654116	-0.33\\
0.89	-0.33921417354272\\
0.88955425068665	-0.34\\
0.883886022988448	-0.35\\
0.88	-0.35689775101824\\
0.878220251315948	-0.36\\
0.872499169871169	-0.37\\
0.87	-0.37438527696793\\
0.866741742276749	-0.38\\
0.860967526288575	-0.39\\
0.86	-0.391678053939715\\
0.85511442734842	-0.4\\
0.85	-0.408766629711751\\
0.849267357341797	-0.41\\
0.843333858139018	-0.42\\
0.84	-0.425646842384575\\
0.837383278637536	-0.43\\
0.8313954228001	-0.44\\
0.83	-0.442335210336008\\
0.825336945980639	-0.45\\
0.82	-0.458823061912365\\
0.819275197297176	-0.46\\
0.813123424622119	-0.47\\
0.81	-0.475100251967111\\
0.806945169813536	-0.48\\
0.800737579771301	-0.49\\
0.8	-0.491189427312776\\
0.79443798478177	-0.5\\
0.79	-0.507066774422106\\
0.788124695473688	-0.51\\
0.78174811177135	-0.52\\
0.78	-0.522748120698783\\
0.775303959560236	-0.53\\
0.77	-0.53823060111712\\
0.76883922217553	-0.54\\
0.762289380874072	-0.55\\
0.76	-0.553506198128201\\
0.755683567151263	-0.56\\
0.75	-0.568589246119734\\
0.74904967783892	-0.57\\
0.742322010547326	-0.58\\
0.74	-0.583462051915945\\
0.735536049022873	-0.59\\
0.73	-0.598141940774891\\
0.7287138515167	-0.6\\
0.72180342074008	-0.61\\
0.72	-0.612615775542615\\
0.714817261798788	-0.62\\
0.71	-0.626889642192897\\
0.707785950048245	-0.63\\
0.700687375198441	-0.64\\
0.7	-0.640969162995594\\
0.693479185400755	-0.65\\
0.69	-0.654835023428521\\
0.686216063306286	-0.66\\
0.68	-0.66851261309207\\
0.678894153995295	-0.67\\
0.67146933006074	-0.68\\
0.67	-0.681982445541935\\
0.6639495188988	-0.69\\
0.66	-0.695248823633469\\
0.656359677434111	-0.7\\
0.65	-0.708323170731707\\
0.648695269915659	-0.71\\
0.640926102693346	-0.72\\
0.64	-0.721193301604089\\
0.633039679153511	-0.73\\
0.63	-0.733855194805195\\
0.625065628519488	-0.74\\
0.62	-0.746322224189835\\
0.616998563838724	-0.75\\
0.61	-0.75859311063792\\
0.608832813608127	-0.76\\
0.60054717745804	-0.77\\
0.6	-0.770660792951542\\
0.592121709187759	-0.78\\
0.59	-0.782519367222664\\
0.583581389273294	-0.79\\
0.58	-0.794180236874669\\
0.574919372915457	-0.8\\
0.57	-0.805642530312417\\
0.566128414468006	-0.81\\
0.56	-0.816905475810739\\
0.557200833062498	-0.82\\
0.55	-0.827968403547672\\
0.54812847419531	-0.83\\
0.54	-0.838830747381502\\
0.538902666736086	-0.84\\
0.53	-0.84949204636291\\
0.529514174735013	-0.85\\
0.52	-0.859951945974765\\
0.519953143310164	-0.86\\
0.510203094574083	-0.87\\
0.51	-0.870208347282178\\
0.500262823486432	-0.88\\
0.5	-0.880264317180617\\
0.490122017665051	-0.89\\
0.49	-0.890120249317241\\
0.48	-0.899774231384397\\
0.479761229375516	-0.9\\
0.47	-0.909225592859046\\
0.469163289150123	-0.91\\
0.46	-0.918475696786792\\
0.458316665286172	-0.92\\
0.45	-0.927524944567627\\
0.447204943437821	-0.93\\
0.44	-0.936373843700159\\
0.435810308687317	-0.94\\
0.43	-0.945023006460749\\
0.424113377699734	-0.95\\
0.42	-0.953473148311826\\
0.412093006073877	-0.96\\
0.41	-0.961725086047127\\
0.4	-0.969777777777778\\
0.399717043969669	-0.97\\
0.39	-0.977617154644663\\
0.386886509139494	-0.98\\
0.38	-0.985259713122012\\
0.373639727349679	-0.99\\
0.37	-0.992706683452602\\
0.36	-0.999959025431265\\
0.359941894884972	-1\\
0.35	-1.00699279379157\\
0.345610810174053	-1.01\\
0.34	-1.01383364585176\\
0.330728624955533	-1.02\\
0.33	-1.02048319516712\\
0.32	-1.02691601915913\\
0.315065297997736	-1.03\\
0.31	-1.03315524268411\\
0.3	-1.0392\\
0.298632030697231	-1.04\\
0.29	-1.0450297537862\\
0.281199255091839	-1.05\\
0.28	-1.05067471325216\\
0.27	-1.05610275290363\\
0.262570838245174	-1.06\\
0.26	-1.06134303372771\\
0.25	-1.06637195121951\\
0.242510406472657	-1.07\\
0.24	-1.07121059509094\\
0.23	-1.07583677187694\\
0.220628130348224	-1.08\\
0.22	-1.08027767778366\\
0.21	-1.08449835709857\\
0.2	-1.08853333333333\\
0.19617957834122	-1.09\\
0.19	-1.09235955706834\\
0.18	-1.09598467269824\\
0.17	-1.09941978378859\\
0.168206741511237	-1.1\\
0.16	-1.10263958959844\\
0.15	-1.10566241685144\\
0.14	-1.10849183709763\\
0.134269564409457	-1.11\\
0.13	-1.1111164369644\\
0.12	-1.1135331326368\\
0.11	-1.11575396593026\\
0.1	-1.11777777777778\\
0.0899999999999999	-1.11960350369513\\
0.0875597386050713	-1.12\\
0.0800000000000001	-1.12121930351777\\
0.0699999999999998	-1.12263341446146\\
0.0600000000000001	-1.12384859117491\\
0.0499999999999998	-1.12486419068736\\
0.04	-1.12567967335345\\
0.0299999999999998	-1.12629460431655\\
0.02	-1.12670865470055\\
0.00999999999999979	-1.12692160252422\\
0	-1.12693333333333\\
-0.01	-1.12674384054751\\
-0.02	-1.12635322551982\\
-0.03	-1.12576169730882\\
-0.04	-1.12496957216403\\
-0.05	-1.12397727272727\\
-0.0600000000000001	-1.12278532695375\\
-0.0700000000000001	-1.12139436675812\\
-0.07877236598505	-1.12\\
-0.0800000000000001	-1.11980338862137\\
-0.0900000000000001	-1.11800079690143\\
-0.1	-1.116\\
-0.11	-1.11380205394375\\
-0.12	-1.11140811050115\\
-0.125433972579337	-1.11\\
-0.13	-1.10880889116677\\
-0.14	-1.10600206295572\\
-0.15	-1.10300166297118\\
-0.159400898041835	-1.1\\
-0.16	-1.09980753167946\\
-0.17	-1.09639460361242\\
-0.18	-1.09279148483236\\
-0.187356134845643	-1.09\\
-0.19	-1.08899107572919\\
-0.2	-1.08497777777778\\
-0.21	-1.08077858028518\\
-0.211770806999874	-1.08\\
-0.22	-1.07636346804344\\
-0.23	-1.07175839613441\\
-0.23366450641378	-1.07\\
-0.24	-1.06694559059327\\
-0.25	-1.06193736141907\\
-0.25372553300172	-1.06\\
-0.26	-1.05672289328345\\
-0.27	-1.05131509442326\\
-0.27234751506471	-1.05\\
-0.28	-1.04569585187823\\
-0.289814552217344	-1.04\\
-0.29	-1.03989196229113\\
-0.3	-1.03386666666667\\
-0.306231105247665	-1.03\\
-0.31	-1.02765310409419\\
-0.32	-1.02123924073089\\
-0.321875523963902	-1.02\\
-0.33	-1.01461508586173\\
-0.336777973427245	-1.01\\
-0.34	-1.00779967874353\\
-0.35	-1.00078436807095\\
-0.351087731252433	-1\\
-0.36	-0.993556749066335\\
-0.364798239011063	-0.99\\
-0.37	-0.986134856047776\\
-0.378057121116786	-0.98\\
-0.38	-0.978517348579596\\
-0.39	-0.97069673941975\\
-0.390869141516686	-0.97\\
-0.4	-0.962666666666667\\
-0.403244890244292	-0.96\\
-0.41	-0.954439226248776\\
-0.415272422517561	-0.95\\
-0.42	-0.946013483146067\\
-0.426975075054109	-0.94\\
-0.43	-0.937388601660863\\
-0.43837416854106	-0.93\\
-0.44	-0.928563847666726\\
-0.449489234676669	-0.92\\
-0.45	-0.919538590604027\\
-0.46	-0.910309533108468\\
-0.460328342258577	-0.91\\
-0.47	-0.900876716404654\\
-0.470910505549445	-0.9\\
-0.48	-0.891243931046739\\
-0.481265358602784	-0.89\\
-0.49	-0.881410981246112\\
-0.49140641922765	-0.88\\
-0.5	-0.871377777777778\\
-0.501346245126918	-0.87\\
-0.51	-0.861144338281042\\
-0.511096526658476	-0.86\\
-0.52	-0.850710787275635\\
-0.520668168804673	-0.85\\
-0.53	-0.840077355893063\\
-0.530071363752752	-0.84\\
-0.539296310149929	-0.83\\
-0.54	-0.829237614185921\\
-0.548365313379309	-0.82\\
-0.55	-0.818196308724832\\
-0.557289098115488	-0.81\\
-0.56	-0.806954693366708\\
-0.566075839433303	-0.8\\
-0.57	-0.795513434235199\\
-0.574733236361128	-0.79\\
-0.58	-0.783873301230605\\
-0.583268550831244	-0.78\\
-0.59	-0.772035166058232\\
-0.591688642511522	-0.77\\
-0.6	-0.76\\
-0.6	-0.76\\
-0.608159707801961	-0.75\\
-0.61	-0.747748898934742\\
-0.616220401373242	-0.74\\
-0.62	-0.735301340003573\\
-0.624187761654832	-0.73\\
-0.63	-0.722658610393083\\
-0.632067185190704	-0.72\\
-0.639860101281135	-0.71\\
-0.64	-0.7098204916562\\
-0.647517004548624	-0.7\\
-0.65	-0.696764541387025\\
-0.655099053595417	-0.69\\
-0.66	-0.683516259146886\\
-0.662610793162795	-0.68\\
-0.67	-0.670077315597473\\
-0.670056558365376	-0.67\\
-0.677371502304027	-0.66\\
-0.68	-0.656417683909075\\
-0.684625864996799	-0.65\\
-0.69	-0.642568811881188\\
-0.691825040660917	-0.64\\
-0.69894507315598	-0.63\\
-0.7	-0.628520179372197\\
-0.705966997479101	-0.62\\
-0.71	-0.614262438566705\\
-0.712943526453414	-0.61\\
-0.719874688518977	-0.6\\
-0.72	-0.599819220405964\\
-0.726686914065785	-0.59\\
-0.73	-0.585154571070758\\
-0.733462673158596	-0.58\\
-0.74	-0.570309210760734\\
-0.74020493873299	-0.57\\
-0.746833816345415	-0.56\\
-0.75	-0.555243288590604\\
-0.753428890193879	-0.55\\
-0.7599982996064	-0.54\\
-0.76	-0.539997411468632\\
-0.766451991247011	-0.53\\
-0.77	-0.524528396994364\\
-0.772884897200895	-0.52\\
-0.779280566563605	-0.51\\
-0.78	-0.508876163104006\\
-0.785582314356523	-0.5\\
-0.79	-0.493011433294612\\
-0.791870187143852	-0.49\\
-0.798096421300038	-0.48\\
-0.8	-0.47695067264574\\
-0.804262667203121	-0.47\\
-0.81	-0.460695651592186\\
-0.810421410537492	-0.46\\
-0.8164822507908	-0.45\\
-0.82	-0.44422509396814\\
-0.822528283143378	-0.44\\
-0.828534108484201	-0.43\\
-0.83	-0.427564319059161\\
-0.8344719695673	-0.42\\
-0.84	-0.410705265036587\\
-0.840412033922208	-0.41\\
-0.846257287194971	-0.4\\
-0.85	-0.39363255033557\\
-0.852097276969206	-0.39\\
-0.857888874091028	-0.38\\
-0.86	-0.376366337699623\\
-0.86363292968415	-0.37\\
-0.869371201584774	-0.36\\
-0.87	-0.358905202806008\\
-0.8750233405838	-0.35\\
-0.88	-0.341236662497767\\
-0.880689745030213	-0.34\\
-0.886272711163838	-0.33\\
-0.89	-0.323362541401844\\
-0.891854384032223	-0.32\\
-0.897385105911927	-0.31\\
-0.9	-0.305291479820628\\
-0.902885748097179	-0.3\\
-0.908364461549508	-0.29\\
-0.91	-0.287022455305004\\
-0.91378767685015	-0.28\\
-0.919214595553231	-0.27\\
-0.92	-0.268554543818607\\
-0.924563896212682	-0.26\\
-0.929939214002986	-0.25\\
-0.93	-0.249886922245247\\
-0.935218025699032	-0.24\\
-0.94	-0.231009762202754\\
-0.940527316691834	-0.23\\
-0.945753585122234	-0.22\\
-0.95	-0.211932326621924\\
-0.950998478497946	-0.21\\
-0.956174000747819	-0.2\\
-0.96	-0.192655131649651\\
-0.961357682334152	-0.19\\
-0.96648261093047	-0.18\\
-0.97	-0.173177789228425\\
-0.971608193585765	-0.17\\
-0.976682671266645	-0.16\\
-0.98	-0.153500017930787\\
-0.981753197776573	-0.15\\
-0.986777359294169	-0.14\\
-0.99	-0.133621643798762\\
-0.991795804921583	-0.13\\
-0.99676977876798	-0.12\\
-1	-0.113542600896861\\
-1.00173905348382	-0.11\\
-1.0066629635396	-0.1\\
-1.01	-0.0932629315756434\\
-1.01158591396127	-0.0900000000000001\\
-1.01645988106638	-0.0800000000000001\\
-1.02	-0.0727827864443249\\
-1.02133929212876	-0.0700000000000001\\
-1.02616343557534	-0.0600000000000001\\
-1.03	-0.0521024240523344\\
-1.03100203195823	-0.05\\
-1.03577647090509	-0.04\\
-1.04	-0.0312222102812107\\
-1.04057691823983	-0.03\\
-1.04530177304821	-0.02\\
-1.05	-0.0101426174496644\\
-1.05006667892513	-0.01\\
-1.05474207241571	0\\
-1.05945890636889	0.00999999999999979\\
-1.06	0.0111460220097422\\
-1.06410004584373	0.02\\
-1.06876701567095	0.0299999999999998\\
-1.07	0.0326358275520313\\
-1.07337831836232	0.04\\
-1.07799554401304	0.0499999999999998\\
-1.08	0.0543247075760303\\
-1.08257946474487	0.0600000000000001\\
-1.08714707256716	0.0699999999999998\\
-1.09	0.0762117554577304\\
-1.09170601085622	0.0800000000000001\\
-1.09622413455633	0.0899999999999999\\
-1.1	0.0982959641255607\\
-1.10076043481663	0.1\\
-1.10522921633833	0.11\\
-1.10973773112087	0.12\\
-1.11	0.120581433474844\\
-1.11416475832244	0.13\\
-1.11862348535452	0.14\\
-1.12	0.143078853434289\\
-1.12303315573507	0.15\\
-1.12744224754256	0.16\\
-1.13	0.165771476751506\\
-1.13183675924958	0.17\\
-1.1361963794957	0.18\\
-1.14	0.188657886237215\\
-1.14057787549478	0.19\\
-1.144888199553	0.2\\
-1.14923683207199	0.21\\
-1.15	0.211752257336343\\
-1.15351998290562	0.22\\
-1.15781886231823	0.23\\
-1.16	0.235051035476319\\
-1.16209396182819	0.24\\
-1.16634328287929	0.25\\
-1.17	0.258540358200915\\
-1.17061232583035	0.26\\
-1.17481229701219	0.27\\
-1.17904961032712	0.28\\
-1.18	0.282238468484739\\
-1.18322806556318	0.29\\
-1.18741557207534	0.3\\
-1.19	0.306138155881559\\
-1.1915927069238	0.31\\
-1.19573064009127	0.32\\
-1.19990552809887	0.33\\
-1.2	0.330226244343891\\
-1.20399690535902	0.34\\
-1.20812176144762	0.35\\
-1.21	0.354535123072762\\
-1.21221641625261	0.36\\
-1.21629150244204	0.37\\
-1.22	0.379027555236213\\
-1.22039117832752	0.38\\
-1.22441677260495	0.39\\
-1.22847813729556	0.4\\
-1.23	0.403734727863526\\
-1.23249955032692	0.41\\
-1.23651098392536	0.42\\
-1.24	0.428631026424591\\
-1.24054177058233	0.43\\
-1.24450358329136	0.44\\
-1.24850062889848	0.45\\
-1.25	0.453738713318284\\
-1.2524578429536	0.46\\
-1.25640483930928	0.47\\
-1.26	0.479035484450836\\
-1.26037562622894	0.48\\
-1.26427291805907	0.49\\
-1.26820484751624	0.5\\
-1.27	0.504547075548335\\
-1.27210670032945	0.51\\
-1.27598851224818	0.52\\
-1.27990510738349	0.53\\
-1.28	0.530242233097698\\
-1.28374005149782	0.54\\
-1.28760611448224	0.55\\
-1.29	0.556158060589667\\
-1.29146122599306	0.56\\
-1.29527715897412	0.57\\
-1.29912718929292	0.58\\
-1.3	0.582262443438914\\
-1.30291997177019	0.59\\
-1.30671947947811	0.6\\
-1.31	0.608568180181802\\
-1.31053623670801	0.61\\
-1.31428566278776	0.62\\
-1.31806844984563	0.63\\
-1.32	0.63508296911685\\
-1.32182739164469	0.64\\
-1.32555973009145	0.65\\
-1.32932548951942	0.66\\
-1.33	0.661788300570704\\
-1.33302864044543	0.67\\
-1.33674358872793	0.68\\
-1.34	0.688697712947956\\
-1.34047675202628	0.69\\
-1.34414139376668	0.7\\
-1.347838639925	0.71\\
-1.35	0.715815462753951\\
-1.35152044154197	0.72\\
-1.35516704952142	0.73\\
-1.35884627312932	0.74\\
-1.36	0.743126996197718\\
-1.36247873325993	0.75\\
-1.36610699475955	0.76\\
-1.36976786790206	0.77\\
-1.37	0.770633728105999\\
-1.37335301145616	0.78\\
-1.37696260740935	0.79\\
-1.38	0.798351974040021\\
-1.38058573670009	0.8\\
-1.38414465191789	0.81\\
-1.38773525788058	0.82\\
-1.39	0.826271976334568\\
-1.39131537629359	0.83\\
-1.39485502340241	0.84\\
-1.39842631053046	0.85\\
-1.4	0.854389140271493\\
-1.40196542718481	0.86\\
-1.40548548896257	0.87\\
-1.40903712499361	0.88\\
-1.41	0.88270447375578\\
-1.41253725033045	0.89\\
-1.41603740714025	0.9\\
-1.41956905737721	0.91\\
-1.42	0.911218885055384\\
-1.42303220228853	0.92\\
-1.42651213302954	0.93\\
-1.43	0.939933782322732\\
-1.43002270748834	0.94\\
-1.43345163615733	0.95\\
-1.43691101921161	0.96\\
-1.44	0.96885845210175\\
-1.44038871924179	0.97\\
-1.44379690239096	0.98\\
-1.44723541656341	0.99\\
-1.45	0.997982505643341\\
-1.45068221743298	1\\
-1.45406934949403	1.01\\
-1.45748667494258	1.02\\
-1.46	1.02730641604916\\
-1.46090454705809	1.03\\
-1.46427032459803	1.04\\
-1.46766614375147	1.05\\
-1.47	1.05683054977846\\
-1.47105705132893	1.06\\
-1.47440117392267	1.07\\
-1.47777517238309	1.08\\
-1.48	1.08655516555093\\
-1.48114107203993	1.09\\
-1.48446324312516	1.1\\
-1.48781511055246	1.11\\
-1.49	1.11648041353724\\
-1.49115794982925	1.12\\
-1.49445787754132	1.13\\
-1.49778730851694	1.14\\
-1.5	1.14660633484163\\
-1.50110902433806	1.15\\
-1.50438642232206	1.16\\
-1.50769311718936	1.17\\
-1.51	1.17693286127952\\
-1.51099563427167	1.18\\
-1.51425022246953	1.19\\
-1.51753388814829	1.2\\
-1.52	1.20745981545142\\
-1.52081911736714	1.21\\
-1.52405062277731	1.22\\
-1.52731097354976	1.23\\
-1.53	1.23818691111312\\
-1.5305808102716	1.24\\
-1.533788967679	1.25\\
-1.53702572594529	1.26\\
-1.54	1.26911375384059\\
-1.54028204833607	1.27\\
-1.54346660101047	1.28\\
-1.5466794980111	1.29\\
-1.54992151569191	1.3\\
-1.55	1.30024202733485\\
-1.55308486569033	1.31\\
-1.55627364219381	1.32\\
-1.55949126316165	1.33\\
-1.56	1.33157880939377\\
-1.56264510332435	1.34\\
-1.56580951027788	1.35\\
-1.56900247379733	1.36\\
-1.57	1.36311528775343\\
-1.57214865373879	1.37\\
-1.57528845288053	1.38\\
-1.5784565094448	1.39\\
-1.58	1.3948506264754\\
-1.58159685444856	1.4\\
-1.58471181887978	1.41\\
-1.58785473080234	1.42\\
-1.59	1.42678388632037\\
-1.59099104006557	1.43\\
-1.59408095478145	1.44\\
-1.59719849674464	1.45\\
-1.6	1.45891402714932\\
-1.60033254165327	1.46\\
-1.60339720403118	1.47\\
-1.60648916359054	1.48\\
-1.60960916876989	1.49\\
-1.61	1.49125121228694\\
-1.61266190627732	1.5\\
-1.61572808432075	1.51\\
-1.61882195782641	1.52\\
-1.62	1.52379450609423\\
-1.62187639659091	1.53\\
-1.62491660775179	1.54\\
-1.62798415508858	1.55\\
-1.63	1.55653263000272\\
-1.63104200464862	1.56\\
-1.63405607767247	1.57\\
-1.63709711900846	1.58\\
-1.64	1.58946413181242\\
-1.64016005388488	1.59\\
-1.64314783194913	1.6\\
-1.64616220244466	1.61\\
-1.6492038945375	1.62\\
-1.65	1.62261104783599\\
-1.65219320160579	1.63\\
-1.65518075169356	1.64\\
-1.65819522758413	1.65\\
-1.66	1.65595464292204\\
-1.66119350988522	1.66\\
-1.66415410550175	1.67\\
-1.66714122400185	1.68\\
-1.67	1.68948827339433\\
-1.67015007129712	1.69\\
-1.67308359406569	1.7\\
-1.67604323027499	1.71\\
-1.67902969625465	1.72\\
-1.68	1.72323936577365\\
-1.68197053802467	1.73\\
-1.68490258346739	1.74\\
-1.68786102962106	1.75\\
-1.69	1.757182912542\\
-1.69081624745292	1.76\\
-1.69372061016552	1.77\\
-1.69665093902981	1.78\\
-1.69960794805555	1.79\\
-1.7	1.79132420091324\\
-1.70249862542725	1.8\\
-1.70540075671635	1.81\\
-1.70832911409695	1.82\\
-1.71	1.82567637612592\\
-1.71123793174188	1.83\\
-1.71411180233609	1.84\\
-1.71701144056417	1.85\\
-1.71993755617047	1.86\\
-1.72	1.86021335284434\\
-1.72278538189376	1.87\\
-1.72565625121069	1.88\\
-1.72855312017717	1.89\\
-1.73	1.89497195099736\\
-1.73142278669585	1.9\\
-1.73426485497894	1.91\\
-1.73713244198459	1.92\\
-1.74	1.92991010339198\\
-1.74002529233094	1.93\\
-1.74283854493522	1.94\\
-1.74567683278583	1.95\\
-1.74854084308825	1.96\\
-1.75	1.96507118451025\\
-1.75137859723884	1.97\\
-1.75418758665275	1.98\\
-1.75702179674699	1.99\\
-1.75988192300929	2\\
256	209\\
-0.639072384226376	2\\
-0.631134820625644	1.99\\
-0.63	1.9885716862918\\
-0.623091536778	1.98\\
-0.62	1.97617019992456\\
-0.614945507872602	1.97\\
-0.61	1.96397155816435\\
-0.606692357301819	1.96\\
-0.6	1.95197492163009\\
-0.598327342298248	1.95\\
-0.59	1.94017953221867\\
-0.589845323823487	1.94\\
-0.581228395321469	1.93\\
-0.58	1.92857580259348\\
-0.572481511476943	1.92\\
-0.57	1.91717218728382\\
-0.563599803018973	1.91\\
-0.56	1.90596928490637\\
-0.554576396598487	1.9\\
-0.55	1.89496664050235\\
-0.545403838940347	1.89\\
-0.54	1.88416387598845\\
-0.536074043063086	1.88\\
-0.53	1.87356068941723\\
-0.526578228165546	1.87\\
-0.52	1.86315685438354\\
-0.516906852315827	1.86\\
-0.51	1.85295221957489\\
-0.507049536939417	1.85\\
-0.5	1.84294670846395\\
-0.496994981939522	1.84\\
-0.49	1.83314031914227\\
-0.486730870088101	1.83\\
-0.48	1.82353312429449\\
-0.476243759094462	1.82\\
-0.47	1.81412527131297\\
-0.465518959481444	1.81\\
-0.46	1.80491698255303\\
-0.454540396067133	1.8\\
-0.45	1.79590855572998\\
-0.443290450450322	1.79\\
-0.44	1.78710036445897\\
-0.431749781414681	1.78\\
-0.43	1.77849285893969\\
-0.42	1.77008602527211\\
-0.419895181528717	1.77\\
-0.41	1.76187030809694\\
-0.407664216593889	1.76\\
-0.4	1.75385579937304\\
-0.395061313987604	1.75\\
-0.39	1.74604325130266\\
-0.382056185129942	1.74\\
-0.38	1.73843349679366\\
-0.37	1.73102102183843\\
-0.368583570425638	1.73\\
-0.36	1.72380213112699\\
-0.354575060959823	1.72\\
-0.35	1.71678767660911\\
-0.340031061356785	1.71\\
-0.34	1.70997880961369\\
-0.33	1.70335561368335\\
-0.324764917613769	1.7\\
-0.32	1.69693904307422\\
-0.31	1.69072605854025\\
-0.308792105209764	1.69\\
-0.3	1.68470219435737\\
-0.291908429312848	1.68\\
-0.29	1.67888811050349\\
-0.28	1.67326488788676\\
-0.273967754160917	1.67\\
-0.27	1.66784642610389\\
-0.26	1.66262496556905\\
-0.254764323249149	1.66\\
-0.25	1.65760400313972\\
-0.24	1.65278146988857\\
-0.233972835763193	1.65\\
-0.23	1.64816047672885\\
-0.22	1.64373463610172\\
-0.211142227916737	1.64\\
-0.21	1.63951668552392\\
-0.2	1.63548589341693\\
-0.19	1.63166420351492\\
-0.185400970483643	1.63\\
-0.18	1.62803787517267\\
-0.17	1.62460866800326\\
-0.16	1.62138571964487\\
-0.155413192824008	1.62\\
-0.15	1.61835753532182\\
-0.14	1.61552583678074\\
-0.13	1.61289825104812\\
-0.12	1.61047377233537\\
-0.117870500768739	1.61\\
-0.11	1.60824050160085\\
-0.1	1.60620689655172\\
-0.0900000000000001	1.60437516124992\\
-0.0800000000000001	1.60274460152634\\
-0.0700000000000001	1.60131460221236\\
-0.0600000000000001	1.60008462595229\\
-0.0591792362269818	1.6\\
-0.05	1.59904827315542\\
-0.04	1.59821182377306\\
-0.03	1.59757549400916\\
-0.02	1.59713904427443\\
-0.01	1.59690231048969\\
0	1.59686520376176\\
0.00999999999999979	1.59702771020127\\
0.02	1.59738989088173\\
0.0299999999999998	1.59795188193965\\
0.04	1.59871389481612\\
0.0499999999999998	1.5996762166405\\
0.052785438201762	1.6\\
0.0600000000000001	1.60083397027601\\
0.0699999999999998	1.60218954752828\\
0.0800000000000001	1.60374547729263\\
0.0899999999999999	1.60550234516876\\
0.1	1.60746081504702\\
0.11	1.60962163035972\\
0.111602501977519	1.61\\
0.12	1.61197321004623\\
0.13	1.61452518302985\\
0.14	1.61728088253729\\
0.149186069533749	1.62\\
0.15	1.62023985959438\\
0.16	1.62338646992622\\
0.17	1.6267388603471\\
0.179163554260615	1.63\\
0.18	1.63029646823911\\
0.19	1.63404083745075\\
0.2	1.63799373040752\\
0.204825748990175	1.64\\
0.21	1.6421432288088\\
0.22	1.64649049229613\\
0.227702553070111	1.65\\
0.23	1.65104319018788\\
0.24	1.65578635282334\\
0.248503680551037	1.66\\
0.25	1.66073907956318\\
0.26	1.66588025541505\\
0.267694978178066	1.67\\
0.27	1.6712304444167\\
0.28	1.6767723412251\\
0.285610920721334	1.68\\
0.29	1.68251801798988\\
0.3	1.68846394984326\\
0.302497223749719	1.69\\
0.31	1.69460372443555\\
0.318497895988713	1.7\\
0.32	1.70095165356296\\
0.33	1.70749069293904\\
0.333718268474227	1.71\\
0.34	1.71423053645117\\
0.348304206557653	1.72\\
0.35	1.72117589703588\\
0.36	1.72831510592955\\
0.362293627650702	1.73\\
0.37	1.73565152055566\\
0.37576341111952	1.74\\
0.38	1.74319150318631\\
0.388792014505878	1.75\\
0.39	1.75093411628923\\
0.4	1.75887147335423\\
0.401386033937041	1.76\\
0.41	1.76700525706056\\
0.413590366239797	1.77\\
0.42	1.77534062304516\\
0.425455972487497	1.78\\
0.43	1.78387694206612\\
0.437006713497258	1.79\\
0.44	1.79261366304484\\
0.448264235192769	1.8\\
0.45	1.80155031201248\\
0.459248211144536	1.81\\
0.46	1.81068649120619\\
0.469976553901142	1.82\\
0.47	1.82002187831183\\
0.48	1.82955344286968\\
0.480458797435018	1.83\\
0.49	1.83928490500972\\
0.490719992828511	1.84\\
0.5	1.84921630094044\\
0.500773586896552	1.85\\
0.51	1.85934760486551\\
0.51063158805239	1.86\\
0.52	1.86967886617334\\
0.520305024220918	1.87\\
0.529801638827514	1.88\\
0.53	1.88020889907113\\
0.539127824311857	1.89\\
0.54	1.8909359610827\\
0.548296361584927	1.9\\
0.55	1.90186232449298\\
0.557315448741161	1.91\\
0.56	1.91298833520669\\
0.566192640349438	1.92\\
0.57	1.92431441472408\\
0.574934902447922	1.93\\
0.58	1.93584106092831\\
0.583548661808603	1.94\\
0.59	1.94756884901998\\
0.592039850139014	1.95\\
0.6	1.95949843260188\\
0.600413943800402	1.96\\
0.608664444078213	1.97\\
0.61	1.97162037244993\\
0.616804299893237	1.98\\
0.62	1.98394122204173\\
0.624841728710956	1.99\\
0.63	1.99646498654652\\
0.632780823249558	2\\
256	1099\\
1.89667165063526	2\\
1.89403481643155	1.99\\
1.89141431498622	1.98\\
1.89	1.9745850009392\\
1.88877499674833	1.97\\
1.8861110324704	1.96\\
1.88346373120231	1.95\\
1.88083278352894	1.94\\
1.88	1.93682861072902\\
1.87816594548641	1.93\\
1.87549170433567	1.92\\
1.87283414131573	1.91\\
1.87019294718734	1.9\\
1.87	1.89926918346963\\
1.86749726121083	1.89\\
1.86481295316805	1.88\\
1.86214533271133	1.87\\
1.86	1.86191777248213\\
1.85947940888749	1.86\\
1.85676823491369	1.85\\
1.8540740685949	1.84\\
1.85139659367722	1.83\\
1.85	1.8247672143975\\
1.84869898316325	1.82\\
1.84597816023765	1.81\\
1.84327434254629	1.8\\
1.84058721446434	1.79\\
1.84	1.78781187054854\\
1.83785659436802	1.78\\
1.83512633304416	1.77\\
1.83241306882604	1.76\\
1.83	1.75105722128113\\
1.82970833937316	1.75\\
1.82695154255866	1.74\\
1.82421205093988	1.73\\
1.82148954264159	1.72\\
1.82	1.71451037936647\\
1.81874893069126	1.71\\
1.81598312936697	1.7\\
1.8132346127613	1.69\\
1.81050306009285	1.68\\
1.81	1.67815633314596\\
1.80772525542769	1.67\\
1.80495065743779	1.66\\
1.80219331801795	1.65\\
1.8	1.64200626959248\\
1.79943736357715	1.64\\
1.79663662018243	1.63\\
1.79385342982522	1.62\\
1.79108746629436	1.61\\
1.79	1.60605897379776\\
1.78829142818095	1.6\\
1.78548233165468	1.59\\
1.78269074934748	1.58\\
1.78	1.57030330111711\\
1.77991399269361	1.57\\
1.77707894384309	1.56\\
1.77426169594947	1.55\\
1.77146191790057	1.54\\
1.77	1.53476169034358\\
1.76864217132522	1.53\\
1.76579921996956	1.52\\
1.7629740178431	1.51\\
1.76016623573152	1.5\\
1.76	1.49940776238612\\
1.75730222037601	1.49\\
1.75445156418108	1.48\\
1.75161860000808	1.47\\
1.75	1.46426643192488\\
1.74876958164828	1.46\\
1.74589345023196	1.45\\
1.74303528148895	1.44\\
1.74019474219807	1.43\\
1.74	1.42931416448272\\
1.73729855529369	1.42\\
1.73441516811168	1.41\\
1.73154967343129	1.4\\
1.73	1.3945739439264\\
1.7286657450583	1.39\\
1.72575713423348	1.38\\
1.7228666771533	1.37\\
1.72	1.36002088615539\\
1.71999387203171	1.36\\
1.71706004103282	1.35\\
1.7141446229728	1.34\\
1.71124727546256	1.33\\
1.71	1.32568376274154\\
1.7083227369145	1.32\\
1.70538236771181	1.31\\
1.70246032027356	1.3\\
1.7	1.29153605015674\\
1.69954405791618	1.29\\
1.69657875579335	1.28\\
1.69363202417368	1.27\\
1.69070351811567	1.26\\
1.69	1.25759421959903\\
1.6877326196278	1.25\\
1.6847612276694	1.24\\
1.6818083018993	1.23\\
1.68	1.22385306122449\\
1.67884277999559	1.22\\
1.67584675956655	1.21\\
1.67286944307378	1.2\\
1.67	1.19030435096305\\
1.66990804642366	1.19\\
1.66688743732744	1.18\\
1.66388576680611	1.17\\
1.66090268289541	1.16\\
1.66	1.15696843683619\\
1.65788206741899	1.15\\
1.65485608715921	1.14\\
1.65184891979952	1.13\\
1.65	1.12382824726135\\
1.64882944564939	1.12\\
1.64577920740921	1.11\\
1.64274800488323	1.1\\
1.64	1.09088303022702\\
1.63972835749056	1.09\\
1.63665392034837	1.08\\
1.63359873803993	1.07\\
1.6305624529328	1.06\\
1.63	1.05814551782259\\
1.62747900857104	1.05\\
1.62439990880051	1.04\\
1.62133991685529	1.03\\
1.62	1.02560940352632\\
1.61825324473991	1.02\\
1.61515029658039	1.01\\
1.61206666285315	1\\
1.61	0.993270117713355\\
1.6089753918298	0.99\\
1.60584867090416	0.98\\
1.60274146679711	0.97\\
1.6	0.961128526645769\\
1.59964420334609	0.96\\
1.59649379160492	0.95\\
1.59336309464842	0.94\\
1.59025174646509	0.93\\
1.59	0.929190509013785\\
1.58708440899554	0.92\\
1.58393030261166	0.91\\
1.58079573434545	0.9\\
1.58	0.897457524665918\\
1.57761926400896	0.89\\
1.57444183725674	0.88\\
1.57128413328812	0.87\\
1.57	0.865923185824114\\
1.56809708830467	0.86\\
1.56489643560731	0.85\\
1.56171568550168	0.84\\
1.56	0.834588064556487\\
1.5585166043384	0.83\\
1.55529282519264	0.82\\
1.55208912341555	0.81\\
1.55	0.803452660406885\\
1.54887652539208	0.8\\
1.54562972406018	0.79\\
1.54240316967546	0.78\\
1.54	0.772517399473882\\
1.53917555556125	0.77\\
1.53590584074585	0.76\\
1.53265653709711	0.75\\
1.53	0.74178263362366\\
1.52941238969675	0.74\\
1.52611987419906	0.73\\
1.52284792857496	0.72\\
1.52	0.711248639839539\\
1.51958571329787	0.71\\
1.51627051365976	0.7\\
1.51297603694349	0.69\\
1.51	0.680915619710363\\
1.50969420235402	0.68\\
1.5063564384844	0.67\\
1.50303954478785	0.66\\
1.5	0.650783699059561\\
1.49973652313185	0.65\\
1.49637631791799	0.64\\
1.49303712420116	0.63\\
1.49	0.620852927716132\\
1.48971133190503	0.62\\
1.48632881080936	0.61\\
1.48296743648562	0.6\\
1.48	0.591123279428357\\
1.47961727462363	0.59\\
1.47621256526669	0.58\\
1.47282913179445	0.57\\
1.47	0.561594651920547\\
1.46945298652029	0.56\\
1.46602621825046	0.55\\
1.46262084871191	0.54\\
1.46	0.532266867092571\\
1.45921709165019	0.53\\
1.45576839510095	0.52\\
1.45234121376853	0.51\\
1.45	0.503139671361503\\
1.44890820236206	0.5\\
1.44543770899741	0.49\\
1.44198884088867	0.48\\
1.44	0.474212736144126\\
1.43852491869725	0.47\\
1.43503276034607	0.46\\
1.43156233076766	0.45\\
1.43	0.445485658478656\\
1.42806582771408	0.44\\
1.42455213609419	0.43\\
1.42106027017571	0.42\\
1.42	0.416957961783439\\
1.41752950273462	0.41\\
1.41399440896725	0.4\\
1.41048123118561	0.39\\
1.41	0.388629096750047\\
1.40691450251102	0.38\\
1.40335813662651	0.37\\
1.4	0.360501567398119\\
1.3998193735219	0.36\\
1.39621937030864	0.35\\
1.39264186074409	0.34\\
1.39	0.33258136935696\\
1.38906372432964	0.33\\
1.38544263290305	0.32\\
1.38184410599272	0.31\\
1.38	0.304859122170814\\
1.37822484846235	0.3\\
1.37458279948814	0.29\\
1.37096337894728	0.28\\
1.37	0.277333981521943\\
1.36730123948408	0.27\\
1.36363836049238	0.26\\
1.36	0.250005067101467\\
1.35999812169269	0.25\\
1.35629137145227	0.24\\
1.35260778630036	0.23\\
1.35	0.222889280125196\\
1.34892107738801	0.22\\
1.3451936975459	0.21\\
1.34148952587663	0.2\\
1.34	0.195968811695614\\
1.33775453605843	0.19\\
1.33400664859218	0.18\\
1.33028200528893	0.17\\
1.33	0.169242536941206\\
1.32649690994625	0.16\\
1.32272863148884	0.15\\
1.32	0.142726230123951\\
1.31895887938387	0.14\\
1.31514658623924	0.13\\
1.31135802751874	0.12\\
1.31	0.116407535444382\\
1.30753440863484	0.11\\
1.3037019252758	0.1\\
1.3	0.0902821316614414\\
1.29989060170158	0.0899999999999999\\
1.29601382791019	0.0800000000000001\\
1.29216125863537	0.0699999999999998\\
1.29	0.0643705240447753\\
1.28829220774064	0.0600000000000001\\
1.28439544567378	0.0499999999999998\\
1.28052288711045	0.04\\
1.28	0.038648671572907\\
1.27659419290547	0.0299999999999998\\
1.27267753802484	0.02\\
1.27	0.0131345547280807\\
1.26875590492507	0.00999999999999979\\
1.2647948075702	0\\
1.2608583464213	-0.01\\
1.26	-0.0121834019717964\\
1.25687306850472	-0.02\\
1.25289226563982	-0.03\\
1.25	-0.0372985133020344\\
1.24891065633989	-0.04\\
1.24488517648422	-0.05\\
1.24088474200696	-0.0600000000000001\\
1.24	-0.062214601272931\\
1.23683538157075	-0.0700000000000001\\
1.23279037284009	-0.0800000000000001\\
1.23	-0.086928027410977\\
1.22874114546234	-0.0900000000000001\\
1.22465123663858	-0.1\\
1.22058675760778	-0.11\\
1.22	-0.111444879630784\\
1.21646556365405	-0.12\\
1.21235628674858	-0.13\\
1.21	-0.135754546307298\\
1.20823154502583	-0.14\\
1.20407715214417	-0.15\\
1.2	-0.15987460815047\\
1.19994733223969	-0.16\\
1.19574750879252	-0.17\\
1.19157388732077	-0.18\\
1.19	-0.18377983573793\\
1.18736547121234	-0.19\\
1.1831465204667	-0.2\\
1.18	-0.207492875923375\\
1.17892911276139	-0.21\\
1.17466452314505	-0.22\\
1.1704264562679	-0.23\\
1.17	-0.231006855165534\\
1.16612592921358	-0.24\\
1.1618423172832	-0.25\\
1.16	-0.254312332875171\\
1.15752872946197	-0.26\\
1.15319926553294	-0.27\\
1.15	-0.277423708920188\\
1.14887087067891	-0.28\\
1.14449524919813	-0.29\\
1.1401468317657	-0.3\\
1.14	-0.300337729029041\\
1.13572817101555	-0.31\\
1.13133367788608	-0.32\\
1.13	-0.323040573693739\\
1.12689588712119	-0.33\\
1.12245501163889	-0.34\\
1.12	-0.345547280230086\\
1.11799620578783	-0.35\\
1.11350864102774	-0.36\\
1.11	-0.367856925051656\\
1.10902688604701	-0.37\\
1.10449232427066	-0.38\\
1.1	-0.389968652037617\\
1.09998563618555	-0.39\\
1.09540376825343	-0.4\\
1.09085046554108	-0.41\\
1.09	-0.411869936373277\\
1.08624062684559	-0.42\\
1.0816400701431	-0.43\\
1.08	-0.433572823779193\\
1.0770004990678	-0.44\\
1.07235237072632	-0.45\\
1.07	-0.455076876679793\\
1.0676809270986	-0.46\\
1.06298490579563	-0.47\\
1.06	-0.476381533842112\\
1.05827939410786	-0.48\\
1.0535351540821	-0.49\\
1.05	-0.497486306729265\\
1.04879332190348	-0.5\\
1.04400053214132	-0.51\\
1.04	-0.518390780408368\\
1.03922006837698	-0.52\\
1.03437839174947	-0.53\\
1.03	-0.539094614324206\\
1.02955692473281	-0.54\\
1.0246660170817	-0.55\\
1.02	-0.559597542935941\\
1.01980111248522	-0.56\\
1.01486062165679	-0.57\\
1.01	-0.579899376214658\\
1.00994978020529	-0.58\\
1.00495934503065	-0.59\\
1	-0.6\\
1	-0.6\\
0.994959249220215	-0.61\\
0.99	-0.619899376214657\\
0.989948763703487	-0.62\\
0.984857314838093	-0.63\\
0.98	-0.639597542935941\\
0.979792978759591	-0.64\\
0.974650436916807	-0.65\\
0.97	-0.659094614324205\\
0.969529463773249	-0.66\\
0.964335420400073	-0.67\\
0.96	-0.678390780408368\\
0.959154943706604	-0.68\\
0.953908975276891	-0.69\\
0.95	-0.697486306729264\\
0.948666044694001	-0.7\\
0.943367711332426	-0.71\\
0.94	-0.716381533842112\\
0.938059288448038	-0.72\\
0.932708132487691	-0.73\\
0.93	-0.735076876679792\\
0.927331086226716	-0.74\\
0.921926630697856	-0.75\\
0.92	-0.753572823779193\\
0.916477732329381	-0.76\\
0.911019479376653	-0.77\\
0.91	-0.771869936373276\\
0.905495397086506	-0.78\\
0.9	-0.789968652037618\\
0.899982437218765	-0.79\\
0.894380119305529	-0.8\\
0.89	-0.807856925051656\\
0.888785802275886	-0.81\\
0.883127798131744	-0.82\\
0.88	-0.825547280230086\\
0.87744846679893	-0.83\\
0.871734184279744	-0.84\\
0.87	-0.843040573693739\\
0.865966053612604	-0.85\\
0.860194870587008	-0.86\\
0.86	-0.860337729029042\\
0.854334019589266	-0.87\\
0.85	-0.877423708920188\\
0.848471388572514	-0.88\\
0.842547644783541	-0.89\\
0.84	-0.894312332875172\\
0.836584955683328	-0.9\\
0.83060202072167	-0.91\\
0.83	-0.911006855165534\\
0.824534921367131	-0.92\\
0.82	-0.927492875923376\\
0.818457750038269	-0.93\\
0.812316002874006	-0.94\\
0.81	-0.94377983573793\\
0.806126234266031	-0.95\\
0.8	-0.95987460815047\\
0.79992092703562	-0.96\\
0.793615456802279	-0.97\\
0.79	-0.975754546307298\\
0.787288735354463	-0.98\\
0.780919456474826	-0.99\\
0.78	-0.991444879630785\\
0.774466066164197	-1\\
0.77	-1.00692802741098\\
0.767986901152711	-1.01\\
0.7614464463108	-1.02\\
0.76	-1.02221460127293\\
0.754830647919489	-1.03\\
0.75	-1.03729851330203\\
0.74818218636357	-1.04\\
0.74146469541638	-1.05\\
0.74	-1.0521834019718\\
0.73466877188428	-1.06\\
0.73	-1.06686544527192\\
0.727832613749585	-1.07\\
0.720931745248485	-1.08\\
0.72	-1.08135132842709\\
0.71393631002159	-1.09\\
0.71	-1.09562947595522\\
0.706892287552462	-1.1\\
0.7	-1.10971786833856\\
0.699796498830208	-1.11\\
0.692584901196055	-1.12\\
0.69	-1.12359246455562\\
0.685310814748454	-1.13\\
0.68	-1.13727376987605\\
0.677975300778403	-1.14\\
0.670561313818869	-1.15\\
0.67	-1.15075746305879\\
0.663032614736733	-1.16\\
0.66	-1.16403118830439\\
0.655431714782943	-1.17\\
0.65	-1.1771107198748\\
0.647754341161099	-1.18\\
0.64	-1.18999493289853\\
0.639995999559357	-1.19\\
0.632101226212836	-1.2\\
0.63	-1.20266601847806\\
0.624117207147999	-1.21\\
0.62	-1.21514087782919\\
0.616038894035288	-1.22\\
0.61	-1.22741863064304\\
0.607860858141086	-1.23\\
0.6	-1.23949843260188\\
0.599577361249679	-1.24\\
0.591153785274281	-1.25\\
0.59	-1.25137090324995\\
0.582604418913302	-1.26\\
0.58	-1.26304203821656\\
0.573932755149408	-1.27\\
0.57	-1.27451434152135\\
0.56513174349381	-1.28\\
0.56	-1.28578726385587\\
0.556193887651956	-1.29\\
0.55	-1.2968603286385\\
0.547111205312416	-1.3\\
0.54	-1.30773313290743\\
0.537875183279884	-1.31\\
0.53	-1.31840534807945\\
0.528476727325509	-1.32\\
0.52	-1.32887672057164\\
0.51890610603034	-1.33\\
0.51	-1.33914707228387\\
0.509152887782794	-1.34\\
0.5	-1.34921630094044\\
0.49920586995517	-1.35\\
0.49	-1.35908438028964\\
0.48905299912303	-1.36\\
0.48	-1.36875136016046\\
0.478681280999326	-1.37\\
0.47	-1.37821736637634\\
0.468076678525855	-1.38\\
0.46	-1.38748260052612\\
0.45722399628971	-1.39\\
0.45	-1.39654733959311\\
0.44610674910153	-1.4\\
0.44	-1.40541193544351\\
0.43470701217257	-1.41\\
0.43	-1.41407681417589\\
0.423005249842689	-1.42\\
0.42	-1.42254247533408\\
0.410980119220478	-1.43\\
0.41	-1.43080949098621\\
0.4	-1.43887147335423\\
0.398565542171267	-1.44\\
0.39	-1.44672988228664\\
0.385734835167752	-1.45\\
0.38	-1.45439059647368\\
0.372487524466448	-1.46\\
0.37	-1.46185448217741\\
0.36	-1.46911696977298\\
0.358750312293238	-1.47\\
0.35	-1.47617175273865\\
0.344424092206881	-1.48\\
0.34	-1.48303156316381\\
0.33	-1.48969564903695\\
0.329529217384131	-1.49\\
0.32	-1.49614693877551\\
0.313849287107984	-1.5\\
0.31	-1.50240578040097\\
0.3	-1.50846394984326\\
0.297380266132883	-1.51\\
0.29	-1.51431623725846\\
0.28	-1.51997911384461\\
0.279961762838834	-1.52\\
0.27	-1.5254260560736\\
0.26130725841701	-1.53\\
0.26	-1.53068583551728\\
0.25	-1.53573356807512\\
0.241222345357124	-1.54\\
0.24	-1.54059223761388\\
0.23	-1.54523830965642\\
0.22	-1.54969669888289\\
0.219287762282851	-1.55\\
0.21	-1.55394102620224\\
0.2	-1.55799373040752\\
0.194798350752085	-1.56\\
0.19	-1.56184366685404\\
0.18	-1.56548962063353\\
0.17	-1.56894277871887\\
0.166751557853848	-1.57\\
0.16	-1.57218812945146\\
0.15	-1.5752327856025\\
0.14	-1.57808222751787\\
0.132768402196853	-1.58\\
0.13	-1.58073081653037\\
0.12	-1.58317138927098\\
0.11	-1.5854149990608\\
0.1	-1.58746081504702\\
0.0899999999999999	-1.58930807545038\\
0.0857988623479077	-1.59\\
0.0800000000000001	-1.59095011864618\\
0.0699999999999998	-1.59238920870054\\
0.0600000000000001	-1.5936291254848\\
0.0499999999999998	-1.59466940532081\\
0.04	-1.5955096580233\\
0.0299999999999998	-1.59614956764208\\
0.02	-1.59658889306757\\
0.00999999999999979	-1.59682746849727\\
0	-1.59686520376176\\
-0.01	-1.59670208450881\\
-0.02	-1.59633817224521\\
-0.03	-1.59577360423585\\
-0.04	-1.59500859326068\\
-0.05	-1.59404342723005\\
-0.0600000000000001	-1.59287846866008\\
-0.0700000000000001	-1.59151415400963\\
-0.0796862023810716	-1.59\\
-0.0800000000000001	-1.58995068493151\\
-0.0900000000000001	-1.5881782028749\\
-0.1	-1.58620689655172\\
-0.11	-1.58403750234801\\
-0.12	-1.58167082655996\\
-0.12651527138634	-1.58\\
-0.13	-1.57910213895345\\
-0.14	-1.57632630126678\\
-0.15	-1.5733548513302\\
-0.16	-1.57018887917031\\
-0.160561158624018	-1.57\\
-0.17	-1.56680964615095\\
-0.18	-1.56323595843245\\
-0.188590537997166	-1.56\\
-0.19	-1.55946696939224\\
-0.2	-1.55548589341693\\
-0.21	-1.5513145488087\\
-0.213009280103072	-1.55\\
-0.22	-1.5469353081461\\
-0.23	-1.54235953125978\\
-0.234947130855806	-1.54\\
-0.24	-1.53758201682783\\
-0.25	-1.53260367762128\\
-0.255035407502553	-1.53\\
-0.26	-1.52742504081376\\
-0.27	-1.52204662056449\\
-0.273673013368226	-1.52\\
-0.28	-1.51646461654324\\
-0.29	-1.51068919704834\\
-0.291153773072818	-1.51\\
-0.3	-1.50470219435737\\
-0.30761567834624	-1.5\\
-0.31	-1.49852422537867\\
-0.32	-1.49214042819582\\
-0.323254995013911	-1.49\\
-0.33	-1.48555486228747\\
-0.338196855551158	-1.48\\
-0.34	-1.47877550609833\\
-0.35	-1.47178990610329\\
-0.35249346956234	-1.47\\
-0.36	-1.46460173084159\\
-0.366235896738611	-1.46\\
-0.37	-1.45721759846724\\
-0.37952128776971	-1.45\\
-0.38	-1.44963654209136\\
-0.39	-1.44184603030493\\
-0.392312697501901	-1.44\\
-0.4	-1.43385579937304\\
-0.404711811438669	-1.43\\
-0.41	-1.42566737492938\\
-0.41675943453612	-1.42\\
-0.42	-1.41728008043232\\
-0.428479469224421	-1.41\\
-0.43	-1.40869331090006\\
-0.439893714913591	-1.4\\
-0.44	-1.39990653405514\\
-0.45	-1.39091353677621\\
-0.450994215750694	-1.39\\
-0.46	-1.38172035575598\\
-0.461832169628042	-1.38\\
-0.47	-1.37232727301209\\
-0.472426533735567	-1.37\\
-0.48	-1.36273406042372\\
-0.482792088936257	-1.36\\
-0.49	-1.35294056485487\\
-0.492942502726546	-1.35\\
-0.5	-1.34294670846395\\
-0.502890439134578	-1.34\\
-0.51	-1.33275248887217\\
-0.512647655747072	-1.33\\
-0.52	-1.32235797919017\\
-0.522225089571815	-1.32\\
-0.53	-1.31176332790275\\
-0.531632933189233	-1.31\\
-0.54	-1.30096875861205\\
-0.540880702434669	-1.3\\
-0.549976723659562	-1.29\\
-0.55	-1.28997440944882\\
-0.558904292927092	-1.28\\
-0.56	-1.27877346090898\\
-0.567693900505244	-1.27\\
-0.57	-1.26737247311152\\
-0.576353122429705	-1.26\\
-0.58	-1.2557719743622\\
-0.58488908109374	-1.25\\
-0.59	-1.24397256606616\\
-0.593308482361186	-1.24\\
-0.6	-1.23197492163009\\
-0.601617648909076	-1.23\\
-0.609818256324185	-1.22\\
-0.61	-1.21977839770651\\
-0.617879024571032	-1.21\\
-0.62	-1.20737153642884\\
-0.625844826489571	-1.2\\
-0.63	-1.19476769583517\\
-0.633720844552899	-1.19\\
-0.64	-1.18196784146494\\
-0.641511992288546	-1.18\\
-0.649204527905949	-1.17\\
-0.65	-1.16896653543307\\
-0.656784070959684	-1.16\\
-0.66	-1.15575777693952\\
-0.664290932687701	-1.15\\
-0.67	-1.1423551634356\\
-0.671729206277185	-1.14\\
-0.67908181113412	-1.13\\
-0.68	-1.12875212296838\\
-0.686331970981565	-1.12\\
-0.69	-1.11494179812708\\
-0.693523924170306	-1.11\\
-0.7	-1.10094043887147\\
-0.700661116082949	-1.1\\
-0.707694728179846	-1.09\\
-0.71	-1.0867290258637\\
-0.714664101934998	-1.08\\
-0.72	-1.07232253043806\\
-0.721587649630997	-1.07\\
-0.728433069243031	-1.06\\
-0.73	-1.05771414761635\\
-0.735201360499444	-1.05\\
-0.74	-1.04290413160869\\
-0.741931903121645	-1.04\\
-0.748595889036302	-1.03\\
-0.75	-1.02789566929134\\
-0.75518265157841	-1.02\\
-0.76	-1.012684365189\\
-0.761738997084465	-1.01\\
-0.768227769631193	-1\\
-0.77	-0.997273313810693\\
-0.774650864841846	-0.99\\
-0.78	-0.98166356219405\\
-0.781050221625718	-0.98\\
-0.787369531697739	-0.97\\
-0.79	-0.965848024038764\\
-0.793645338080763	-0.96\\
-0.799901318756403	-0.95\\
-0.8	-0.949842271293375\\
-0.806058698300322	-0.94\\
-0.81	-0.933621956508076\\
-0.812202286973195	-0.93\\
-0.818296128307205	-0.92\\
-0.82	-0.917208718659443\\
-0.824329885945802	-0.91\\
-0.83	-0.900598466026727\\
-0.830355167955409	-0.9\\
-0.836291694727446	-0.89\\
-0.84	-0.883777392179052\\
-0.842215135003408	-0.88\\
-0.848092882400613	-0.87\\
-0.85	-0.866761811023622\\
-0.853918734191587	-0.86\\
-0.859738412219332	-0.85\\
-0.86	-0.849550624763347\\
-0.865470772117134	-0.84\\
-0.87	-0.832129339154469\\
-0.871205809036377	-0.83\\
-0.876875872406495	-0.82\\
-0.88	-0.814508820938719\\
-0.882524327623087	-0.81\\
-0.888138487045861	-0.8\\
-0.89	-0.796691005607712\\
-0.893704065621425	-0.79\\
-0.899262906895137	-0.78\\
-0.9	-0.778675078864353\\
-0.904749190090392	-0.77\\
-0.91	-0.760457406942439\\
-0.910247677055726	-0.76\\
-0.915663723881535	-0.75\\
-0.92	-0.742033128063341\\
-0.921088959773021	-0.74\\
-0.926451554127109	-0.73\\
-0.93	-0.723409959745896\\
-0.931806805144642	-0.72\\
-0.937116440119085	-0.71\\
-0.94	-0.704587385118973\\
-0.942404872497558	-0.7\\
-0.947662020621871	-0.69\\
-0.95	-0.685564960629921\\
-0.952886705852618	-0.68\\
-0.95809182065821	-0.67\\
-0.96	-0.666342316904072\\
-0.963255740290624	-0.66\\
-0.96840925780471	-0.65\\
-0.97	-0.64691915946781\\
-0.973515307859078	-0.64\\
-0.978617648030677	-0.63\\
-0.98	-0.627295269332661\\
-0.983668643051028	-0.62\\
-0.988720211111439	-0.61\\
-0.99	-0.607470503438269\\
-0.993718887885132	-0.6\\
-0.998720075645109	-0.59\\
-1	-0.587444794952681\\
-1.003669096614	-0.58\\
-1.00862028369964	-0.57\\
-1.01	-0.567218153428806\\
-1.01352224008595	-0.56\\
-1.01842379511517	-0.55\\
-1.02	-0.54679066481645\\
-1.02328120978366	-0.54\\
-1.02813349148503	-0.53\\
-1.03	-0.526162491329844\\
-1.03294882156152	-0.52\\
-1.03775217983703	-0.51\\
-1.04	-0.505333871171057\\
-1.04252781910217	-0.5\\
-1.04728259603548	-0.49\\
-1.05	-0.484305118110236\\
-1.0520208771113	-0.48\\
-1.05672740792294	-0.47\\
-1.06	-0.463076620924084\\
-1.06143060426867	-0.46\\
-1.06608921821948	-0.45\\
-1.07	-0.44164884269451\\
-1.07075954595204	-0.44\\
-1.07537056719625	-0.43\\
-1.08	-0.420022319969838\\
-1.08001018674991	-0.42\\
-1.08457393513906	-0.41\\
-1.08916616245327	-0.4\\
-1.09	-0.398186276925021\\
-1.09370174461674	-0.39\\
-1.09824665761923	-0.38\\
-1.1	-0.376151419558359\\
-1.10275636256826	-0.37\\
-1.10725427116551	-0.36\\
-1.11	-0.353918653518997\\
-1.11174010222161	-0.35\\
-1.11619131439785	-0.34\\
-1.12	-0.3314888133887\\
-1.12065522485697	-0.33\\
-1.12506004729779	-0.32\\
-1.12949240879443	-0.31\\
-1.13	-0.30885561350275\\
-1.13386268009769	-0.3\\
-1.13824857952082	-0.29\\
-1.14	-0.286016609869998\\
-1.14260137472375	-0.28\\
-1.14694111524141	-0.27\\
-1.15	-0.262982283464567\\
-1.15127824611741	-0.26\\
-1.15557213148403	-0.25\\
-1.15989291354685	-0.24\\
-1.16	-0.239752195368847\\
-1.1641436984577	-0.23\\
-1.1684185365504	-0.22\\
-1.17	-0.216309044131574\\
-1.17265784211965	-0.21\\
-1.17688704064406	-0.2\\
-1.18	-0.192672999874007\\
-1.18111654514966	-0.19\\
-1.18530041062909	-0.18\\
-1.1895104737834	-0.17\\
-1.19	-0.168838084245146\\
-1.19366058925871	-0.16\\
-1.19782522134304	-0.15\\
-1.2	-0.144794952681388\\
-1.20196947801447	-0.14\\
-1.20608898973751	-0.13\\
-1.21	-0.12056201308917\\
-1.21022893781462	-0.12\\
-1.21430364320514	-0.11\\
-1.21840379499992	-0.1\\
-1.22	-0.0961164077302009\\
-1.22247100629761	-0.0900000000000001\\
-1.2265262510615	-0.0800000000000001\\
-1.23	-0.0714794980792242\\
-1.23059286443872	-0.0700000000000001\\
-1.23460352330896	-0.0600000000000001\\
-1.23863925784856	-0.05\\
-1.24	-0.0466353595349425\\
-1.24263736274776	-0.04\\
-1.24662840907411	-0.03\\
-1.25	-0.0215964566929134\\
-1.25062948236462	-0.02\\
-1.25457617053428	-0.01\\
-1.25854753817696	0\\
-1.26	0.00364898268671815\\
-1.26248422231189	0.00999999999999979\\
-1.26641113044862	0.02\\
-1.27	0.0290872882423326\\
-1.27035420731085	0.0299999999999998\\
-1.2742369964764	0.04\\
-1.2781440446264	0.0499999999999998\\
-1.28	0.0547361879499811\\
-1.28202674739885	0.0600000000000001\\
-1.28588957859867	0.0699999999999998\\
-1.28977667732711	0.0800000000000001\\
-1.29	0.0805743176492941\\
-1.29360092437006	0.0899999999999999\\
-1.29744370646008	0.1\\
-1.3	0.106624605678234\\
-1.30127962557438	0.11\\
-1.3050784506297	0.12\\
-1.30890108424894	0.13\\
-1.31	0.132869587628866\\
-1.31268242293524	0.14\\
-1.31646100664958	0.15\\
-1.32	0.15931137709462\\
-1.32025710050456	0.16\\
-1.3239920075215	0.17\\
-1.32775024307989	0.18\\
-1.33	0.185963826630469\\
-1.33149553385271	0.19\\
-1.33521000857471	0.2\\
-1.3389477827401	0.21\\
-1.34	0.2128101227382\\
-1.34264409831143	0.22\\
-1.34633802881205	0.23\\
-1.35	0.239852362204725\\
-1.35005389361383	0.24\\
-1.3537043770856	0.25\\
-1.35737764618756	0.26\\
-1.36	0.267107206866086\\
-1.36104817900301	0.27\\
-1.36467792971458	0.28\\
-1.36833041554338	0.29\\
-1.37	0.294558003034326\\
-1.3719573946334	0.3\\
-1.37556629361949	0.31\\
-1.37919787015424	0.32\\
-1.38	0.322205698366468\\
-1.38278306084651	0.33\\
-1.38637098554749	0.34\\
-1.38998152316816	0.35\\
-1.39	0.350051173039123\\
-1.3935266781216	0.36\\
-1.39709350291594	0.37\\
-1.4	0.378107255520504\\
-1.40066617362417	0.38\\
-1.40418972832796	0.39\\
-1.40773532505961	0.4\\
-1.41	0.406361761733308\\
-1.4112713948061	0.41\\
-1.41477367588367	0.42\\
-1.41829791438361	0.43\\
-1.42	0.434815024661692\\
-1.42179908625204	0.44\\
-1.42527996882385	0.45\\
-1.42878271742506	0.46\\
-1.43	0.463467627697956\\
-1.43225068163269	0.47\\
-1.43571003978123	0.48\\
-1.43919116582627	0.49\\
-1.44	0.492320081084505\\
-1.44262759962284	0.5\\
-1.4460653068818	0.51\\
-1.44952467722223	0.52\\
-1.45	0.521372820919176\\
-1.45293124472376	0.53\\
-1.45634717455839	0.54\\
-1.45978465604524	0.55\\
-1.46	0.550626208296334\\
-1.46316300800719	0.56\\
-1.46655703428502	0.57\\
-1.46997249424954	0.58\\
-1.47	0.580080528586839\\
-1.47332426778354	0.59\\
-1.47669626523473	0.6\\
-1.48	0.609737656112022\\
-1.48008732783685	0.61\\
-1.48341639019721	0.62\\
-1.48676623486382	0.63\\
-1.49	0.639595296826699\\
-1.49013380788396	0.64\\
-1.49344072975187	0.65\\
-1.4967682994253	0.66\\
-1.5	0.669652996845426\\
-1.50011396246346	0.67\\
-1.50339862976839	0.68\\
-1.50670380441425	0.69\\
-1.51	0.699910734338528\\
-1.51002912164272	0.7\\
-1.51329142277839	0.71\\
-1.51657408494823	0.72\\
-1.51987749633876	0.73\\
-1.52	0.730370750285642\\
-1.52312043085623	0.74\\
-1.52638046608531	0.75\\
-1.52966109552529	0.76\\
-1.53	0.761032365632337\\
-1.53288696589217	0.77\\
-1.53612426308283	0.78\\
-1.53938199374707	0.79\\
-1.54	0.791894760877838\\
-1.54259232980984	0.8\\
-1.54580678159968	0.81\\
-1.54904150054816	0.82\\
-1.55	0.822957606973059\\
-1.55223781473054	0.83\\
-1.55542931784507	0.84\\
-1.55864091638311	0.85\\
-1.56	0.854220499176486\\
-1.56182470308765	0.86\\
-1.56499315867674	0.87\\
-1.56818153269862	0.88\\
-1.57	0.885682957782138\\
-1.57135426769368	0.89\\
-1.57449958165141	0.9\\
-1.57766463196702	0.91\\
-1.58	0.917344428986974\\
-1.58082777176315	0.92\\
-1.58394985503064	0.93\\
-1.58709148767461	0.94\\
-1.59	0.949204285894764\\
-1.59024646889401	0.95\\
-1.59334523774483	0.96\\
-1.59646336426783	0.97\\
-1.59960121784084	0.98\\
-1.6	0.981269841269842\\
-1.60268697931837	0.99\\
-1.60578151706024	1\\
-1.60889558140936	1.01\\
-1.61	1.01353858664638\\
-1.61197631975889	1.02\\
-1.61504719210328	1.03\\
-1.61813738615435	1.04\\
-1.62	1.04600460934532\\
-1.62121448941345	1.05\\
-1.62426162602374	1.06\\
-1.62732787518668	1.07\\
-1.63	1.07866698590303\\
-1.63040270879456	1.08\\
-1.63342604583097	1.09\\
-1.63646828222096	1.1\\
-1.63952977879803	1.11\\
-1.64	1.11153440873873\\
-1.64254166869655	1.12\\
-1.64555983133571	1.13\\
-1.64859703235595	1.14\\
-1.65	1.1446057844691\\
-1.65160970170945	1.15\\
-1.65460373670895	1.16\\
-1.65761658456903	1.17\\
-1.66	1.17787149183854\\
-1.66063134160936	1.18\\
-1.66360120233196	1.19\\
-1.66658964690434	1.2\\
-1.66959703037469	1.21\\
-1.67	1.21133877946502\\
-1.67255342170416	1.22\\
-1.67551742048216	1.23\\
-1.67850012069213	1.24\\
-1.68	1.24501260301762\\
-1.68146157751171	1.25\\
-1.6844010957347	1.26\\
-1.68735907480825	1.27\\
-1.69	1.27887806274113\\
-1.69032684129267	1.28\\
-1.69324185205144	1.29\\
-1.69617508019971	1.3\\
-1.6991268736189	1.31\\
-1.7	1.31295238095238\\
-1.70204085741384	1.32\\
-1.70494931302148	1.33\\
-1.70787608222816	1.34\\
-1.71	1.34722367487809\\
-1.710799268022	1.35\\
-1.7136829377137	1.36\\
-1.71658466686513	1.37\\
-1.71950480074504	1.38\\
-1.72	1.38169396212025\\
-1.7223771066033	1.39\\
-1.72525378838666	1.4\\
-1.72814861268517	1.41\\
-1.73	1.41636967171557\\
-1.73103295950288	1.42\\
-1.73388459516215	1.43\\
-1.7367541090928	1.44\\
-1.73964184332483	1.45\\
-1.74	1.45123927254229\\
-1.74247822265297	1.46\\
-1.7453224342412	1.47\\
-1.74818459398983	1.48\\
-1.75	1.48631735340729\\
-1.7510357929942	1.49\\
-1.75385471906426	1.5\\
-1.75669131958171	1.51\\
-1.7595459327274	1.52\\
-1.76	1.52158901182755\\
-1.76235208072434	1.53\\
-1.76516314627974	1.54\\
-1.76799194306843	1.55\\
-1.77	1.5570668008112\\
-1.77081562218728	1.56\\
-1.7736011860138	1.57\\
-1.77640419858278	1.58\\
-1.77922499363557	1.59\\
-1.78	1.59274264649803\\
-1.78200653603098	1.6\\
-1.78478380566086	1.61\\
-1.78757856785356	1.62\\
-1.79	1.62861687353556\\
-1.79038027847396	1.63\\
-1.79313185549669	1.64\\
-1.79590063455532	1.65\\
-1.79868694438562	1.66\\
-1.8	1.66469841269841\\
-1.80144942366315	1.67\\
-1.80419227850424	1.68\\
-1.80695236644579	1.69\\
-1.80973001766175	1.7\\
-1.81	1.70097136082999\\
-1.81245456830382	1.71\\
-1.8151885022274	1.72\\
-1.81793969466619	1.73\\
-1.82	1.7374533282617\\
-1.82068855599107	1.74\\
-1.82339641292616	1.75\\
-1.82612122377193	1.76\\
-1.82886331282106	1.77\\
-1.83	1.7741343827435\\
-1.83157714261797	1.78\\
-1.83427565829144	1.79\\
-1.83699114081234	1.8\\
-1.83972391542899	1.81\\
-1.84	1.81100960142621\\
-1.84240403385014	1.82\\
-1.84509300598138	1.83\\
-1.84779895218701	1.84\\
-1.85	1.84809231378764\\
-1.85050736801625	1.85\\
-1.8531699349815	1.86\\
-1.85584915894611	1.87\\
-1.85854535845124	1.88\\
-1.86	1.88537665438842\\
-1.86122293584262	1.89\\
-1.86387555283578	1.9\\
-1.8665448220104	1.91\\
-1.86923106238783	1.92\\
-1.87	1.92285724600802\\
-1.87187913179008	1.93\\
-1.87452159622733	1.94\\
-1.87718070233795	1.95\\
-1.87985676947515	1.96\\
-1.88	1.9605350324965\\
-1.88247666835759	1.97\\
-1.88510877638055	1.98\\
-1.88775750985364	1.99\\
-1.89	1.99842096569653\\
-1.89041100559091	2\\
512	526\\
-0.507633086682859	-2\\
-0.51	-1.99759367433817\\
-0.517324240314877	-1.99\\
-0.52	-1.98722617717478\\
-0.52683728642552	-1.98\\
-0.53	-1.97665813865863\\
-0.536182285723106	-1.97\\
-0.54	-1.96588971018346\\
-0.545368611562006	-1.96\\
-0.55	-1.95492109634551\\
-0.554405011448321	-1.95\\
-0.56	-1.94375255467989\\
-0.563299661975724	-1.94\\
-0.57	-1.93238439532723\\
-0.572060217989121	-1.93\\
-0.58	-1.92081698063147\\
-0.580693856665224	-1.92\\
-0.589189619320384	-1.91\\
-0.59	-1.90904651147308\\
-0.597553987842949	-1.9\\
-0.6	-1.89707317073171\\
-0.605808722589066	-1.89\\
-0.61	-1.8849011098312\\
-0.613959590843646	-1.88\\
-0.62	-1.87253090200938\\
-0.622012031755107	-1.87\\
-0.629970559052857	-1.86\\
-0.63	-1.85996300679426\\
-0.637795646421836	-1.85\\
-0.64	-1.84718616162512\\
-0.645535797416558	-1.84\\
-0.65	-1.83421234772979\\
-0.6531954559713	-1.83\\
-0.66	-1.82104234274087\\
-0.660778834831599	-1.82\\
-0.668254031806264	-1.81\\
-0.67	-1.80766666444829\\
-0.675643561834201	-1.8\\
-0.68	-1.79409104279259\\
-0.682967562091855	-1.79\\
-0.69	-1.78032103622131\\
-0.690229525520147	-1.78\\
-0.697379862320499	-1.77\\
-0.7	-1.76634146341463\\
-0.704469461622093	-1.76\\
-0.71	-1.75216727679164\\
-0.711506148989268	-1.75\\
-0.718462111503775	-1.74\\
-0.72	-1.73779121327771\\
-0.725339728416783	-1.73\\
-0.73	-1.72321373621507\\
-0.732172597077623	-1.72\\
-0.738942360913234	-1.71\\
-0.74	-1.70843874633757\\
-0.745628205335081	-1.7\\
-0.75	-1.69345930232558\\
-0.7522766360404	-1.69\\
-0.758867719502548	-1.68\\
-0.76	-1.67828337032762\\
-0.765380085967218	-1.67\\
-0.77	-1.66290371362771\\
-0.771861654301865	-1.66\\
-0.778280993338339	-1.65\\
-0.78	-1.64732525073223\\
-0.784636507462791	-1.64\\
-0.79	-1.63154756097561\\
-0.790967212364157	-1.63\\
-0.79722122610897	-1.62\\
-0.8	-1.61556541019956\\
-0.803435052396866	-1.61\\
-0.809622229880295	-1.6\\
-0.81	-1.59938958509173\\
-0.815724157448286	-1.59\\
-0.82	-1.5830057233986\\
-0.821810175084988	-1.58\\
-0.827840437212649	-1.57\\
-0.83	-1.56642437330849\\
-0.833822612846105	-1.56\\
-0.839789544999248	-1.55\\
-0.84	-1.54964734737403\\
-0.845672279428794	-1.54\\
-0.85	-1.53266196013289\\
-0.851546836371321	-1.53\\
-0.857364403516118	-1.52\\
-0.86	-1.51547854164819\\
-0.863146647271727	-1.51\\
-0.868903999298896	-1.5\\
-0.87	-1.49809791509392\\
-0.874597847401912	-1.49\\
-0.88	-1.48051709303355\\
-0.880290280082066	-1.48\\
-0.885905094766106	-1.47\\
-0.89	-1.46273018253511\\
-0.891515481356869	-1.46\\
-0.897072866721941	-1.45\\
-0.9	-1.44474501108647\\
-0.902604717874365	-1.44\\
-0.908105470385631	-1.43\\
-0.91	-1.42656104700191\\
-0.913562165994866	-1.42\\
-0.919007052322728	-1.41\\
-0.92	-1.40817780936306\\
-0.924391848914741	-1.4\\
-0.929781607577445	-1.39\\
-0.93	-1.38959486866083\\
-0.935097645164851	-1.38\\
-0.94	-1.37080825290003\\
-0.940424874049091	-1.37\\
-0.945683296536036	-1.36\\
-0.95	-1.35182032115172\\
-0.950946846309117	-1.35\\
-0.956152415472885	-1.34\\
-0.96	-1.33263259771337\\
-0.96135513374336	-1.33\\
-0.966508491973709	-1.32\\
-0.97	-1.31324486679374\\
-0.971653131452168	-1.31\\
-0.976754900031587	-1.3\\
-0.98	-1.29365696550501\\
-0.981844123337006	-1.29\\
-0.986894903648625	-1.28\\
-0.99	-1.2738687840894\\
-0.991931287776432	-1.27\\
-0.996931662453076	-1.26\\
-1	-1.25388026607539\\
-1.00191770292941	-1.25\\
-1.00686823694669	-1.24\\
-1.01	-1.23369140836327\\
-1.01180635169143	-1.23\\
-1.01670759340759	-1.22\\
-1.02	-1.21330226123969\\
-1.02160012632712	-1.21\\
-1.02645260847212	-1.2\\
-1.03	-1.19271292832129\\
-1.0313018328011	-1.19\\
-1.03610607341728	-1.18\\
-1.04	-1.17192356642737\\
-1.04091419482746	-1.17\\
-1.04567069816401	-1.16\\
-1.05	-1.15093438538206\\
-1.05043985765659	-1.15\\
-1.05514911501983	-1.14\\
-1.05987914298136	-1.13\\
-1.06	-1.1297445165881\\
-1.06454388217832	-1.12\\
-1.06922691514772	-1.11\\
-1.07	-1.10835036890529\\
-1.07385748699156	-1.1\\
-1.07849399167543	-1.09\\
-1.08	-1.08675647152883\\
-1.08309234903054	-1.08\\
-1.0876827815675	-1.07\\
-1.09	-1.0649632484133\\
-1.09225082294751	-1.06\\
-1.09679562960268	-1.05\\
-1.1	-1.0429711751663\\
-1.10133520115353	-1.04\\
-1.10583481897699	-1.03\\
-1.11	-1.0207807784325\\
-1.11034771632321	-1.02\\
-1.11480257377404	-1.01\\
-1.11927707584057	-1\\
-1.12	-0.998385489463857\\
-1.12370106127535	-0.99\\
-1.12813098638392	-0.98\\
-1.13	-0.975788753941116\\
-1.13253239412131	-0.97\\
-1.13691815462906	-0.96\\
-1.14	-0.952994730772642\\
-1.14129863233277	-0.95\\
-1.14564063377035	-0.94\\
-1.15	-0.930004152823921\\
-1.15000178520236	-0.93\\
-1.15430042677447	-0.92\\
-1.15861801002842	-0.91\\
-1.16	-0.90680366124589\\
-1.1628994881219	-0.9\\
-1.16717386216613	-0.89\\
-1.17	-0.883407380540397\\
-1.17143972543611	-0.88\\
-1.17567127788728	-0.87\\
-1.17992153239545	-0.86\\
-1.18	-0.859815395568212\\
-1.18411211458456	-0.85\\
-1.18831968333872	-0.84\\
-1.19	-0.836013457865044\\
-1.19249818613892	-0.83\\
-1.19666344720216	-0.82\\
-1.2	-0.812017738359202\\
-1.2008312642314	-0.81\\
-1.2049545916265	-0.8\\
-1.20909611944062	-0.79\\
-1.21	-0.787819614512471\\
-1.21319484362617	-0.78\\
-1.21729455733128	-0.77\\
-1.22	-0.763420040827194\\
-1.2213858907171	-0.76\\
-1.22544415740175	-0.75\\
-1.22952035911766	-0.74\\
-1.23	-0.738823922327506\\
-1.23354656575152	-0.73\\
-1.23758143034877	-0.72\\
-1.24	-0.714021628340584\\
-1.24160339130734	-0.71\\
-1.2455972805338	-0.7\\
-1.24960882915345	-0.69\\
-1.25	-0.68902530589544\\
-1.25356947999688	-0.68\\
-1.25754015531781	-0.67\\
-1.26	-0.663821859184942\\
-1.26149956388887	-0.66\\
-1.26542972375173	-0.65\\
-1.26937725755241	-0.64\\
-1.27	-0.638423548645403\\
-1.27327903343185	-0.63\\
-1.27718614770947	-0.62\\
-1.28	-0.612820946125854\\
-1.28108954987884	-0.61\\
-1.28495659984721	-0.6\\
-1.28884072896425	-0.59\\
-1.29	-0.58701929438442\\
-1.29269004540095	-0.58\\
-1.29653420156861	-0.57\\
-1.3	-0.561019955654102\\
-1.30038788382475	-0.56\\
-1.30419242076388	-0.55\\
-1.30801373325889	-0.54\\
-1.31	-0.534814044245036\\
-1.31181675370946	-0.53\\
-1.31559853519576	-0.52\\
-1.31939704797266	-0.51\\
-1.32	-0.50841377591884\\
-1.32315114055016	-0.5\\
-1.32691020798104	-0.49\\
-1.33	-0.481810149074937\\
-1.33067285500302	-0.48\\
-1.33439283103833	-0.47\\
-1.33812922040729	-0.46\\
-1.34	-0.455004141117925\\
-1.34184619283544	-0.45\\
-1.34554357534324	-0.44\\
-1.34925731760775	-0.43\\
-1.35	-0.428001946607341\\
-1.35293027041951	-0.42\\
-1.35660508886118	-0.41\\
-1.36	-0.400799219373725\\
-1.36029052342893	-0.4\\
-1.36392677435588	-0.39\\
-1.36757905386914	-0.38\\
-1.37	-0.37339077889782\\
-1.3712235636754	-0.37\\
-1.37483735934316	-0.36\\
-1.37846712108689	-0.35\\
-1.38	-0.345784795945586\\
-1.38207228863857	-0.34\\
-1.38566364965862	-0.33\\
-1.38927091115118	-0.32\\
-1.39	-0.317980639490899\\
-1.39283829689477	-0.31\\
-1.39640724055488	-0.3\\
-1.39999201600567	-0.29\\
-1.4	-0.289977728285078\\
-1.40352315899018	-0.28\\
-1.40706969957568	-0.27\\
-1.41	-0.261767651236074\\
-1.41061974762374	-0.26\\
-1.41412841868594	-0.25\\
-1.41765256779441	-0.24\\
-1.42	-0.233358585768855\\
-1.42116922841412	-0.23\\
-1.42465559412932	-0.22\\
-1.42815736097824	-0.21\\
-1.43	-0.204750208898173\\
-1.43164210852211	-0.2\\
-1.43510617895357	-0.19\\
-1.43858557068112	-0.18\\
-1.44	-0.175942150671529\\
-1.44203985991604	-0.17\\
-1.4454816433098	-0.16\\
-1.44893866526852	-0.15\\
-1.45	-0.146934093437152\\
-1.45236393176526	-0.14\\
-1.45578343483332	-0.13\\
-1.45921809087661	-0.12\\
-1.46	-0.117725772278109\\
-1.46261575135117	-0.11\\
-1.46601297954751	-0.1\\
-1.46942527230879	-0.0900000000000001\\
-1.47	-0.0883169753773545\\
-1.47279672491837	-0.0800000000000001\\
-1.47617168270755	-0.0700000000000001\\
-1.47956161387208	-0.0600000000000001\\
-1.48	-0.0587075443128173\\
-1.4829082384685	-0.05\\
-1.48626092958682	-0.04\\
-1.48962850015589	-0.03\\
-1.49	-0.0288973742817695\\
-1.49295165849914	-0.02\\
-1.49628208620825	-0.01\\
-1.49962729675579	0\\
-1.5	0.00111358574610244\\
-1.5029283326903	0.00999999999999979\\
-1.50623650002311	0.02\\
-1.50955935094451	0.0299999999999998\\
-1.51	0.0313253329473075\\
-1.51283959054072	0.04\\
-1.51612550053949	0.0499999999999998\\
-1.51942599229254	0.0600000000000001\\
-1.52	0.0617378106350763\\
-1.52268674395627	0.0699999999999998\\
-1.52595039990283	0.0800000000000001\\
-1.5292285332407	0.0899999999999999\\
-1.53	0.0923509083218309\\
-1.53247108779261	0.1\\
-1.53571249343064	0.11\\
-1.53896826962672	0.12\\
-1.54	0.123164461853467\\
-1.54219390035436	0.13\\
-1.54541306010352	0.14\\
-1.54864648116813	0.15\\
-1.55	0.154178253615128\\
-1.55185644385273	0.16\\
-1.55505336301464	0.17\\
-1.55826443190344	0.18\\
-1.56	0.18539201280797\\
-1.56145996482367	0.19\\
-1.5646346497797	0.2\\
-1.56782337059373	0.21\\
-1.57	0.216805415796258\\
-1.57100569450862	0.22\\
-1.57415815290928	0.23\\
-1.57732453108661	0.24\\
-1.58	0.248418086523941\\
-1.58049484919967	0.25\\
-1.58362509014571	0.26\\
-1.58676913264457	0.27\\
-1.58992716150509	0.28\\
-1.59	0.280230620569747\\
-1.59303666476613	0.29\\
-1.59615838023949	0.3\\
-1.59929396002374	0.31\\
-1.6	0.312249443207127\\
-1.60239406585373	0.32\\
-1.60549346481533	0.33\\
-1.60860660399995	0.34\\
-1.61	0.344466959636865\\
-1.61169846853902	0.35\\
-1.61477556352071	0.36\\
-1.61786627267744	0.37\\
-1.62	0.376882582021873\\
-1.62095103421268	0.38\\
-1.62400583991323	0.39\\
-1.62707413185922	0.4\\
-1.63	0.409495672543186\\
-1.63015291071204	0.41\\
-1.6331854441372	0.42\\
-1.63623133407659	0.43\\
-1.6392907593077	0.44\\
-1.64	0.442315815735544\\
-1.64231551307654	0.45\\
-1.64533901873319	0.46\\
-1.64837592625112	0.47\\
-1.65	0.475335094549499\\
-1.65139717048448	0.48\\
-1.65439831222601	0.49\\
-1.65741272072791	0.5\\
-1.66	0.508550466542255\\
-1.66043152709164	0.51\\
-1.66341032804468	0.52\\
-1.66640225907197	0.53\\
-1.66940749600287	0.54\\
-1.67	0.541969831543295\\
-1.6723761668509	0.55\\
-1.67534564489675	0.56\\
-1.67832828854827	0.57\\
-1.68	0.575590851650796\\
-1.68129691653941	0.58\\
-1.68424396914743	0.59\\
-1.68720404562319	0.6\\
-1.69	0.609406114788325\\
-1.69017365228199	0.61\\
-1.69309831013624	0.62\\
-1.69603584876966	0.63\\
-1.69898644061572	0.64\\
-1.7	0.643429844097995\\
-1.70190973356239	0.65\\
-1.70482476699854	0.66\\
-1.70775270706308	0.67\\
-1.71	0.677649239695878\\
-1.71067929251806	0.68\\
-1.71357185678682	0.69\\
-1.71647717990638	0.7\\
-1.71939543281789	0.71\\
-1.72	0.712069751270393\\
-1.72227816206245	0.72\\
-1.72516090661524	0.73\\
-1.72805642975656	0.74\\
-1.73	0.746692337292584\\
-1.73094471417812	0.75\\
-1.73380492214019	0.76\\
-1.73667775645933	0.77\\
-1.73956338637971	0.78\\
-1.74	0.781512030678677\\
-1.74241024886818	0.79\\
-1.74526043905289	0.8\\
-1.74812326926382	0.81\\
-1.75	0.816536429365962\\
-1.75097789656968	0.82\\
-1.75380549108666	0.83\\
-1.75664556919912	0.84\\
-1.75949829825505	0.85\\
-1.76	0.851757281726567\\
-1.76231391346445	0.86\\
-1.76513129099474	0.87\\
-1.76796115978837	0.88\\
-1.77	0.887181682904044\\
-1.77078669437043	0.89\\
-1.77358142676245	0.9\\
-1.77638849006036	0.91\\
-1.77920804952333	0.92\\
-1.78	0.922805242043327\\
-1.78199695582984	0.93\\
-1.78478127243665	0.94\\
-1.7875779217071	0.95\\
-1.79	0.958627408741274\\
-1.79037884465019	0.96\\
-1.79314047741414	0.97\\
-1.79591427884638	0.98\\
-1.79870041193211	0.99\\
-1.8	0.994654788418708\\
-1.80146706252089	1\\
-1.804218082607	1.01\\
-1.80698126732327	1.02\\
-1.80975677989343	1.03\\
-1.81	1.03087595600375\\
-1.81249028157389	1.04\\
-1.81523058823314	1.05\\
-1.817983052727	1.06\\
-1.82	1.06730394233336\\
-1.82073181114236	1.07\\
-1.82344931426168	1.08\\
-1.82617880498341	1.09\\
-1.82892044397131	1.1\\
-1.83	1.1039307032399\\
-1.83163837168491	1.11\\
-1.83434496721297	1.12\\
-1.83706353791074	1.13\\
-1.83979424452055	1.14\\
-1.84	1.14075322681425\\
-1.84248245649403	1.15\\
-1.84517804169927	1.16\\
-1.84788558689701	1.17\\
-1.85	1.17778225806452\\
-1.8505921763165	1.18\\
-1.85326486311874	1.19\\
-1.85594933407313	1.2\\
-1.85864574709158	1.21\\
-1.86	1.21501117348303\\
-1.86132489619965	1.22\\
-1.86398638441594	1.23\\
-1.8666596361785	1.24\\
-1.86934480932156	1.25\\
-1.87	1.25243735447611\\
-1.87199762236835	1.26\\
-1.8746478089335	1.27\\
-1.87730973573355	1.28\\
-1.87998356047433	1.29\\
-1.88	1.29006148075377\\
-1.88261114006666	1.3\\
-1.88524992126648	1.31\\
-1.88790041668627	1.32\\
-1.89	1.32789359841574\\
-1.89055049006903	1.33\\
-1.89316623073644	1.34\\
-1.89579350238682	1.35\\
-1.89843245954346	1.36\\
-1.9	1.36592427616926\\
-1.90105951455359	1.37\\
-1.90366367232208	1.38\\
-1.90627932994701	1.39\\
-1.90890664167331	1.4\\
-1.91	1.4041537626499\\
-1.91151176664024	1.41\\
-1.91410423954977	1.42\\
-1.91670817855811	1.43\\
-1.91932373758217	1.44\\
-1.92	1.44258252877666\\
-1.92190801804441	1.45\\
-1.92448870418193	1.46\\
-1.9270808200426	1.47\\
-1.92968451916593	1.48\\
-1.93	1.48121099379437\\
-1.93224903757698	1.49\\
-1.93481783524712	1.5\\
-1.93739802366268	1.51\\
-1.93998975593646	1.52\\
-1.94	1.52003952470294\\
-1.94253559135231	1.53\\
-1.94509239924616	1.54\\
-1.94766055632422	1.55\\
-1.95	1.55907258064516\\
-1.95023488124038	1.56\\
-1.95276844297306	1.57\\
-1.9553131603347	1.58\\
-1.95786918275723	1.59\\
-1.96	1.59830556396332\\
-1.96042693643288	1.6\\
-1.96294835369245	1.61\\
-1.96548088048304	1.62\\
-1.96802466567312	1.63\\
-1.97	1.63773850349526\\
-1.97056690646358	1.64\\
-1.97307608255438	1.65\\
-1.97559631961357	1.66\\
-1.97812776589948	1.67\\
-1.98	1.67737155072593\\
-1.98065554727749	1.68\\
-1.98315238651206	1.69\\
-1.98566023571652	1.7\\
-1.98817924249285	1.71\\
-1.99	1.71720480379493\\
-1.9906936126738	1.72\\
-1.99317802052578	1.73\\
-1.99567338494457	1.74\\
-1.99817985283018	1.75\\
-2	1.75723830734967\\
512	523\\
2	1.7394701986755\\
1.99761339157696	1.73\\
1.99510387619586	1.72\\
1.99260541787412	1.71\\
1.99011787128626	1.7\\
1.99	1.69952607169661\\
1.98758731089635	1.69\\
1.98506531278239	1.68\\
1.98255442510966	1.67\\
1.98005450186305	1.66\\
1.98	1.65978197227863\\
1.97751005231287	1.65\\
1.97497567012247	1.64\\
1.97245244933422	1.63\\
1.97	1.62023888691875\\
1.9699388841557	1.62\\
1.96738085793117	1.61\\
1.96483419150243	1.6\\
1.96229873498468	1.59\\
1.96	1.58089749180183\\
1.95976922425231	1.58\\
1.95719896806212	1.57\\
1.95464011825846	1.56\\
1.95209252439388	1.55\\
1.95	1.54175664451827\\
1.94954600299751	1.54\\
1.94696362112511	1.53\\
1.94439268967581	1.52\\
1.94183305768682	1.51\\
1.94	1.50281611617816\\
1.93926845607287	1.5\\
1.93667405352904	1.49\\
1.93409114286683	1.48\\
1.93151957265585	1.47\\
1.93	1.46407562502766\\
1.92893581717625	1.46\\
1.9263294995314	1.45\\
1.92373471262749	1.44\\
1.92115130461453	1.43\\
1.92	1.42553483682674\\
1.91854731786097	1.42\\
1.91592919107496	1.41\\
1.91332263127152	1.4\\
1.91072748622941	1.39\\
1.91	1.38719336529539\\
1.9081021873516	1.38\\
1.90547235760133	1.37\\
1.9028541284419	1.36\\
1.90024734732932	1.35\\
1.9	1.34905077262693\\
1.89759965233599	1.34\\
1.89495822584106	1.33\\
1.89232843089892	1.32\\
1.89	1.31111147268619\\
1.88970366951427	1.31\\
1.88703893673299	1.3\\
1.88438601957957	1.29\\
1.88174476228443	1.28\\
1.88	1.27337509073205\\
1.87909540491848	1.27\\
1.87641926143535	1.26\\
1.87375495939858	1.25\\
1.87110234286174	1.24\\
1.87	1.23583682081436\\
1.8684272902087	1.23\\
1.86573984402727	1.22\\
1.86306426239245	1.21\\
1.86040038923073	1.2\\
1.86	1.19849601695664\\
1.85769854002039	1.19\\
1.8549998984762	1.18\\
1.85231314185889	1.17\\
1.85	1.16135797342193\\
1.84963015719217	1.16\\
1.8469083652213	1.15\\
1.84419863479845	1.14\\
1.84150080696387	1.13\\
1.84	1.12442346279749\\
1.83878876287246	1.12\\
1.83605597255096	1.11\\
1.83333525869813	1.1\\
1.83062646237998	1.09\\
1.83	1.08768501479747\\
1.82788424591753	1.08\\
1.82514056423281	1.07\\
1.82240897117901	1.06\\
1.82	1.05114684149907\\
1.81968253211078	1.05\\
1.81691580503298	1.04\\
1.81416133755866	1.03\\
1.8114189681289	1.02\\
1.81	1.01481407500774\\
1.80866004912531	1.01\\
1.80588263373601	1\\
1.80311748444506	0.99\\
1.80036443987626	0.98\\
1.8	0.978675496688741\\
1.79757191862727	0.97\\
1.79478391988298	0.96\\
1.79200819096125	0.95\\
1.79	0.942742200433801\\
1.78922823795427	0.94\\
1.78641732430199	0.93\\
1.78361884516777	0.92\\
1.78083263682815	0.91\\
1.78	0.907007722894759\\
1.77801673448425	0.9\\
1.77519544332902	0.89\\
1.77238658459086	0.88\\
1.77	0.871471590858763\\
1.76958117423432	0.87\\
1.76673701330176	0.86\\
1.76390544582025	0.85\\
1.76108630589857	0.84\\
1.76	0.836140228056218\\
1.75824257005698	0.83\\
1.75538823943678	0.82\\
1.7525464942278	0.81\\
1.75	0.801003599114064\\
1.74971111576797	0.8\\
1.74683397154797	0.79\\
1.74396956982294	0.78\\
1.7411177427212	0.77\\
1.74	0.766073932643862\\
1.73824163552483	0.76\\
1.73535452987055	0.75\\
1.73248015264199	0.74\\
1.73	0.731338724035609\\
1.72961021207345	0.73\\
1.72670035886031	0.72\\
1.72380338693407	0.71\\
1.72091912663678	0.7\\
1.72	0.69680890695414\\
1.71800602864236	0.69\\
1.71508642109687	0.68\\
1.71217967467293	0.67\\
1.71	0.662476607941215\\
1.70927049847705	0.66\\
1.70632821799134	0.65\\
1.70339894701139	0.64\\
1.7004825142932	0.63\\
1.7	0.628344370860928\\
1.69752772787269	0.62\\
1.69457589735853	0.61\\
1.69163705004655	0.6\\
1.69	0.594416038653752\\
1.6886838884131	0.59\\
1.68570946677213	0.58\\
1.68274817200571	0.57\\
1.68	0.56068170461593\\
1.67979562471301	0.56\\
1.67679858366408	0.55\\
1.67381481180424	0.54\\
1.67084413435991	0.53\\
1.67	0.527154956049295\\
1.66784216379027	0.52\\
1.66483588833734	0.51\\
1.66184284604646	0.5\\
1.66	0.493826276209856\\
1.65883911022627	0.49\\
1.65581030772893	0.48\\
1.65279487563398	0.47\\
1.65	0.460693521594684\\
1.64978831332724	0.46\\
1.64673696328283	0.45\\
1.6436991192889	0.44\\
1.64067460347905	0.43\\
1.64	0.427767411463393\\
1.63761473541685	0.42\\
1.63455446018406	0.41\\
1.63150764511916	0.4\\
1.63	0.395041029665325\\
1.62844249157801	0.39\\
1.62535976840527	0.38\\
1.62229063565697	0.37\\
1.62	0.36251202974241\\
1.61921908613755	0.36\\
1.6161139008376	0.35\\
1.61302243441018	0.34\\
1.61	0.330181075273581\\
1.60994336026402	0.33\\
1.60681570102969	0.32\\
1.6037018872276	0.31\\
1.60060173651585	0.3\\
1.6	0.298057395143488\\
1.59746399903451	0.29\\
1.59432782632136	0.28\\
1.59120543948733	0.27\\
1.59	0.26613287305024\\
1.58805761122499	0.26\\
1.58489907007381	0.25\\
1.58175443558693	0.24\\
1.58	0.234407216768374\\
1.57859534008264	0.23\\
1.57541442281754	0.22\\
1.57224753092478	0.21\\
1.57	0.202880881897429\\
1.56907597395754	0.2\\
1.56587267458681	0.19\\
1.56268351714749	0.18\\
1.56	0.171554272558222\\
1.55949828679794	0.17\\
1.55627260083845	0.16\\
1.55306117115069	0.15\\
1.55	0.140427740863787\\
1.54986103784743	0.14\\
1.54661296214061	0.13\\
1.54337925475932	0.12\\
1.54015972654824	0.11\\
1.54	0.109503785405453\\
1.53689250382734	0.1\\
1.53363651437384	0.0899999999999999\\
1.53039481227937	0.0800000000000001\\
1.53	0.0787814576989271\\
1.52710995561717	0.0699999999999998\\
1.52383168058049	0.0600000000000001\\
1.52056779872394	0.0499999999999998\\
1.52	0.0482590624172332\\
1.51726403119371	0.04\\
1.51396346772371	0.0299999999999998\\
1.51067740082756	0.02\\
1.51	0.0179367378923664\\
1.50735342774624	0.00999999999999979\\
1.50403057343848	0\\
1.50072231661384	-0.01\\
1.5	-0.0121854304635764\\
1.49737682546827	-0.02\\
1.49403167814087	-0.03\\
1.49070122666685	-0.04\\
1.49	-0.0421074102688614\\
1.48733288701194	-0.05\\
1.48396544447444	-0.0600000000000001\\
1.48061279356968	-0.0700000000000001\\
1.48	-0.0718292222124131\\
1.47722025689611	-0.0800000000000001\\
1.47383051671042	-0.0900000000000001\\
1.47045566129682	-0.1\\
1.47	-0.101350940023831\\
1.46703756086593	-0.11\\
1.46362552009948	-0.12\\
1.46022845455822	-0.13\\
1.46	-0.130672690373247\\
1.45678340520152	-0.14\\
1.45334906017277	-0.15\\
1.45	-0.159793743078627\\
1.44992837986841	-0.16\\
1.44645637597356	-0.17\\
1.44299972198991	-0.18\\
1.44	-0.188711378730187\\
1.43954944020211	-0.19\\
1.43605503824326	-0.2\\
1.43257606933148	-0.21\\
1.43	-0.217428866321518\\
1.42909473985965	-0.22\\
1.42557793520423	-0.23\\
1.42207664383353	-0.24\\
1.42	-0.245946546387194\\
1.41856280105848	-0.25\\
1.41502358726367	-0.26\\
1.41149996406158	-0.27\\
1.41	-0.27426481154169\\
1.40795212254221	-0.28\\
1.40439049106032	-0.29\\
1.40084452452136	-0.3\\
1.4	-0.302384105960265\\
1.39726117853948	-0.31\\
1.39367711841627	-0.32\\
1.39010879460357	-0.33\\
1.39	-0.330304924793789\\
1.38648841765385	-0.34\\
1.38288191521978	-0.35\\
1.38	-0.358019084712756\\
1.3792773045537	-0.36\\
1.37563226168157	-0.37\\
1.37200330023627	-0.38\\
1.37	-0.385533708386754\\
1.36835868918432	-0.39\\
1.36469110435439	-0.4\\
1.3610396638443	-0.41\\
1.36	-0.412850799258147\\
1.35735345741944	-0.42\\
1.35366331000397	-0.43\\
1.35	-0.439970930232558\\
1.34998915897518	-0.44\\
1.34625994678383	-0.45\\
1.34254721214463	-0.46\\
1.34	-0.466881500486597\\
1.33882834519515	-0.47\\
1.33507646282839	-0.48\\
1.33134111197095	-0.49\\
1.33	-0.493595960510623\\
1.32757586060881	-0.5\\
1.32380127754549	-0.51\\
1.32004327676654	-0.52\\
1.32	-0.520115162741467\\
1.31622994704353	-0.53\\
1.31243262769148	-0.54\\
1.31	-0.546424260315775\\
1.30862580257144	-0.55\\
1.30478880968551	-0.56\\
1.30096871301254	-0.57\\
1.3	-0.572538631346579\\
1.29711047368655	-0.58\\
1.2932506152111	-0.59\\
1.29	-0.598452965782834\\
1.28939623956725	-0.6\\
1.28549626371358	-0.61\\
1.28161348981265	-0.62\\
1.28	-0.624163082088893\\
1.27770424564798	-0.63\\
1.27378126288081	-0.64\\
1.27	-0.649679921608574\\
1.26987311576592	-0.65\\
1.26590956693947	-0.66\\
1.26196351704746	-0.67\\
1.26	-0.674986793953859\\
1.25799692215348	-0.68\\
1.25401021005303	-0.69\\
1.25004102539058	-0.7\\
1.25	-0.700103362734289\\
1.24601408052559	-0.71\\
1.24200413289455	-0.72\\
1.24	-0.725008892424644\\
1.23797357799713	-0.73\\
1.23392250582513	-0.74\\
1.23	-0.749724210549625\\
1.22988711632753	-0.75\\
1.22579455561135	-0.76\\
1.22172005945726	-0.77\\
1.22	-0.774229354069099\\
1.21761865688819	-0.78\\
1.21350256188015	-0.79\\
1.21	-0.798541496613696\\
1.20939314655767	-0.8\\
1.20523508319886	-0.81\\
1.20109535213791	-0.82\\
1.2	-0.822649006622516\\
1.19691591820571	-0.83\\
1.19273409871796	-0.84\\
1.19	-0.846556939100438\\
1.18854332156561	-0.85\\
1.18431903705625	-0.86\\
1.18011334201005	-0.87\\
1.18	-0.870269524565581\\
1.17584837637854	-0.88\\
1.17160008449209	-0.89\\
1.17	-0.893772646760016\\
1.16732028300475	-0.9\\
1.16302900881235	-0.91\\
1.16	-0.917080562682473\\
1.15873287899723	-0.92\\
1.15439823201944	-0.93\\
1.15008266557784	-0.94\\
1.15	-0.940191565600882\\
1.14570582503717	-0.95\\
1.14134673431941	-0.96\\
1.14	-0.963093667755895\\
1.13694981109501	-0.97\\
1.13254679383781	-0.98\\
1.13	-0.985798972103099\\
1.12812816405125	-0.99\\
1.12368081153764	-1\\
1.12	-1.0083067717093\\
1.11923880660121	-1.01\\
1.11474670315494	-1.02\\
1.11027439023488	-1.03\\
1.11	-1.03061368929469\\
1.10574233081208	-1.04\\
1.10122508239675	-1.05\\
1.1	-1.05271523178808\\
1.09666550093927	-1.06\\
1.09210288502233	-1.07\\
1.09	-1.07461830895674\\
1.0875139620531	-1.08\\
1.08290553788955	-1.09\\
1.08	-1.09632241973998\\
1.07828540238197	-1.1\\
1.07363071985813	-1.11\\
1.07	-1.11782711403159\\
1.06897744732755	-1.12\\
1.06427604629623	-1.13\\
1.06	-1.13913199327017\\
1.05958765675066	-1.14\\
1.05483906633019	-1.15\\
1.05011140505251	-1.16\\
1.05	-1.16023566703418\\
1.04531725990451	-1.17\\
1.04054215651623	-1.18\\
1.04	-1.18113593929233\\
1.03570803463902	-1.19\\
1.03088499336696	-1.2\\
1.03	-1.20183639834062\\
1.02600872246853	-1.21\\
1.02113723440992	-1.22\\
1.02	-1.22233685883288\\
1.01621657604979	-1.23\\
1.01129611831388	-1.24\\
1.01	-1.24263718820361\\
1.00632876491896	-1.25\\
1.00135879972399	-1.26\\
1	-1.26273730684327\\
0.996342371381824	-1.27\\
0.991322345105244	-1.28\\
0.99	-1.28263718820361\\
0.986254386117595	-1.29\\
0.981183728297381	-1.3\\
0.98	-1.30233685883288\\
0.976061703475542	-1.31\\
0.970939825760147	-1.32\\
0.97	-1.32183639834061\\
0.965761116442143	-1.33\\
0.96058741148657	-1.34\\
0.96	-1.34113593929233\\
0.955349311254656	-1.35\\
0.950123151559903	-1.36\\
0.95	-1.36023566703418\\
0.944822861635062	-1.37\\
0.94	-1.37913199327017\\
0.939534857843111	-1.38\\
0.93417822261617	-1.39\\
0.93	-1.39782711403159\\
0.928823042505727	-1.4\\
0.923411723929341	-1.41\\
0.92	-1.41632241973998\\
0.917986284229649	-1.42\\
0.9125195629207	-1.43\\
0.91	-1.43461830895674\\
0.907020667303419	-1.44\\
0.901497796959862	-1.45\\
0.9	-1.45271523178808\\
0.895922129478815	-1.46\\
0.890342335302062	-1.47\\
0.89	-1.47061368929469\\
0.884686453326974	-1.48\\
0.88	-1.4883067717093\\
0.879030525418843	-1.49\\
0.873309256964089	-1.5\\
0.87	-1.5057989721031\\
0.867566894490929	-1.51\\
0.861785984096504	-1.52\\
0.86	-1.5230936677559\\
0.855953350441811	-1.53\\
0.850111893330462	-1.54\\
0.85	-1.54019156560088\\
0.844184993075918	-1.55\\
0.84	-1.55708056268247\\
0.838248472813188	-1.56\\
0.832256715435941	-1.57\\
0.83	-1.57377264676002\\
0.82621864658972	-1.58\\
0.820163190645217	-1.59\\
0.82	-1.59026952456558\\
0.814019080329992	-1.6\\
0.81	-1.60655693910044\\
0.807857415026047	-1.61\\
0.801644011052686	-1.62\\
0.8	-1.62264900662252\\
0.795368064319643	-1.63\\
0.79	-1.6385414966137\\
0.789069278599499	-1.64\\
0.782692095096478	-1.65\\
0.78	-1.6542293540691\\
0.77627007866764	-1.66\\
0.77	-1.66972421054963\\
0.769819410660543	-1.67\\
0.763272174754745	-1.68\\
0.76	-1.68500889242464\\
0.756688523970042	-1.69\\
0.750068424154142	-1.7\\
0.75	-1.70010336273429\\
0.743345749411876	-1.71\\
0.74	-1.71498679395386\\
0.736583384978693	-1.72\\
0.73	-1.72967992160857\\
0.729778850148067	-1.73\\
0.722870642821618	-1.74\\
0.72	-1.74416308208889\\
0.715910754123478	-1.75\\
0.71	-1.75845296578283\\
0.708900797156022	-1.76\\
0.701800481307794	-1.77\\
0.7	-1.77253863134658\\
0.69462226931007	-1.78\\
0.69	-1.78642426031578\\
0.68738517264994	-1.79\\
0.680084077054534	-1.8\\
0.68	-1.80011516274147\\
0.672664448173752	-1.81\\
0.67	-1.81359596051062\\
0.665176049783019	-1.82\\
0.66	-1.8268815004866\\
0.657615052307998	-1.83\\
0.65	-1.83997093023256\\
0.649977423273964	-1.84\\
0.642211193139462	-1.85\\
0.64	-1.85285079925815\\
0.634360188492089	-1.86\\
0.63	-1.86553370838675\\
0.62642024051558	-1.87\\
0.62	-1.87801908471276\\
0.618386443134367	-1.88\\
0.610248123046663	-1.89\\
0.61	-1.89030492479379\\
0.601972263762521	-1.9\\
0.6	-1.90238410596027\\
0.593587764637135	-1.91\\
0.59	-1.91426481154169\\
0.585088587070146	-1.92\\
0.58	-1.92594654638719\\
0.576468314841375	-1.93\\
0.57	-1.93742886632152\\
0.567720121125205	-1.94\\
0.56	-1.94871137873019\\
0.558836731867551	-1.95\\
0.55	-1.95979374307863\\
0.549810385055268	-1.96\\
0.540618162851564	-1.97\\
0.54	-1.97067269037325\\
0.531264791162024	-1.98\\
0.53	-1.98135094002383\\
0.521745409209275	-1.99\\
0.52	-1.99182922221241\\
0.512050155528093	-2\\
1024	373\\
-1.09230098125488	-2\\
-1.09685223468853	-1.99\\
-1.1	-1.9830985915493\\
-1.10139527597607	-1.98\\
-1.10590250295107	-1.97\\
-1.11	-1.96093468924963\\
-1.11041710346194	-1.96\\
-1.11488082037525	-1.95\\
-1.11935845366389	-1.94\\
-1.12	-1.93856785602308\\
-1.12378928962052	-1.93\\
-1.1282237624692	-1.92\\
-1.13	-1.9159994658354\\
-1.13262995497761	-1.91\\
-1.13702176050665	-1.9\\
-1.14	-1.89323293057034\\
-1.14140480473328	-1.89\\
-1.14575442513663	-1.88\\
-1.15	-1.87026876465989\\
-1.15011577339897	-1.87\\
-1.15442368046025	-1.86\\
-1.15874501157732	-1.85\\
-1.16	-1.84709845170187\\
-1.16303139939575	-1.84\\
-1.16731132389918	-1.83\\
-1.17	-1.82373020288675\\
-1.17157940563634	-1.82\\
-1.1758183762751	-1.81\\
-1.18	-1.80016548445612\\
-1.1800694754969	-1.8\\
-1.18426793603586	-1.79\\
-1.18847947799587	-1.78\\
-1.19	-1.77639368712983\\
-1.19266172530079	-1.77\\
-1.19683303671407	-1.76\\
-1.2	-1.75242566510172\\
-1.20100142260157	-1.75\\
-1.20513293092105	-1.74\\
-1.20927732981415	-1.73\\
-1.21	-1.72825721833746\\
-1.21338078934657	-1.72\\
-1.21748564842018	-1.71\\
-1.22	-1.70388636562911\\
-1.22157820047896	-1.7\\
-1.22564393251092	-1.69\\
-1.22972236100728	-1.68\\
-1.23	-1.67931939364472\\
-1.23375372467463	-1.67\\
-1.23779327477125	-1.66\\
-1.24	-1.65454648875525\\
-1.24181653021319	-1.65\\
-1.24581760151008	-1.64\\
-1.24983117200973	-1.63\\
-1.25	-1.62957941176471\\
-1.25379680409116	-1.62\\
-1.25777213132227	-1.61\\
-1.26	-1.60440553780618\\
-1.26173231070603	-1.6\\
-1.26566978245876	-1.59\\
-1.26961955313027	-1.58\\
-1.27	-1.57903707613162\\
-1.2735255142546	-1.57\\
-1.27743765327395	-1.56\\
-1.28	-1.5534636187136\\
-1.28134068317664	-1.55\\
-1.28521556777197	-1.54\\
-1.28910254812904	-1.53\\
-1.29	-1.5276927957731\\
-1.29295461646949	-1.52\\
-1.29680455582148	-1.51\\
-1.3	-1.50172143974961\\
-1.30065608888358	-1.5\\
-1.30446935503258	-1.49\\
-1.30829451108028	-1.48\\
-1.31	-1.47554758390524\\
-1.3120982010868	-1.47\\
-1.31588688813979	-1.46\\
-1.31968741656929	-1.45\\
-1.32	-1.44917773734078\\
-1.32344489825239	-1.44\\
-1.32720915723125	-1.43\\
-1.33	-1.42260305425968\\
-1.33096973618562	-1.42\\
-1.33469808167696	-1.41\\
-1.33843805793632	-1.4\\
-1.34	-1.3958290979753\\
-1.34215535373622	-1.39\\
-1.34585960848964	-1.38\\
-1.34957544199659	-1.37\\
-1.35	-1.36885784313725\\
-1.35325099335388	-1.36\\
-1.35693129348743	-1.35\\
-1.36	-1.34168344080636\\
-1.36061332111539	-1.34\\
-1.36425843410347	-1.33\\
-1.36791491087969	-1.32\\
-1.37	-1.31430768351139\\
-1.37155794420453	-1.31\\
-1.37517941586405	-1.3\\
-1.3788121959458	-1.29\\
-1.38	-1.28673365523296\\
-1.38241768703769	-1.28\\
-1.38601564091119	-1.27\\
-1.38962484661039	-1.26\\
-1.39	-1.25896090332904\\
-1.39319421990745	-1.25\\
-1.3967687754203	-1.24\\
-1.4	-1.23098591549296\\
-1.40034893649971	-1.23\\
-1.40388917820053	-1.22\\
-1.40744045089467	-1.21\\
-1.41	-1.20280877360559\\
-1.41098703326109	-1.2\\
-1.4145041637607	-1.19\\
-1.41803226552181	-1.18\\
-1.42	-1.17443238704017\\
-1.42154661726129	-1.17\\
-1.425040746183	-1.16\\
-1.42854578546122	-1.15\\
-1.43	-1.1458564516129\\
-1.43202923030926	-1.14\\
-1.43550046404262	-1.13\\
-1.43898254606685	-1.12\\
-1.44	-1.11708069999373\\
-1.44243638415337	-1.11\\
-1.44588482606112	-1.1\\
-1.44934405304715	-1.09\\
-1.45	-1.08810490196078\\
-1.45276956159411	-1.08\\
-1.45619531221309	-1.07\\
-1.45963178356561	-1.06\\
-1.46	-1.0589288646206\\
-1.46303021753977	-1.05\\
-1.46643337477582	-1.04\\
-1.46984718728441	-1.03\\
-1.47	-1.02955243259362\\
-1.4732197800076	-1.02\\
-1.47660043932462	-1.01\\
-1.4799916873538	-1\\
-1.48	-0.999975488164752\\
-1.48333965107306	-0.99\\
-1.48669790567623	-0.98\\
-1.49	-0.970197331851898\\
-1.49006562153952	-0.97\\
-1.49339120776931	-0.96\\
-1.49672714878261	-0.95\\
-1.5	-0.940219092331768\\
-1.5000723719883	-0.94\\
-1.5033758029394	-0.93\\
-1.50668951957753	-0.92\\
-1.51	-0.91004084222716\\
-1.51001340189272	-0.91\\
-1.51329476604314	-0.9\\
-1.51658634577788	-0.89\\
-1.51988823736291	-0.88\\
-1.52	-0.879661555848559\\
-1.52314940392085	-0.87\\
-1.52641893264193	-0.86\\
-1.52969870144447	-0.85\\
-1.53	-0.849081608022851\\
-1.53294100151583	-0.84\\
-1.5361885636864	-0.83\\
-1.53944629319218	-0.82\\
-1.54	-0.81830123642754\\
-1.54267082255754	-0.81\\
-1.54589650136434	-0.8\\
-1.5491322738061	-0.79\\
-1.55	-0.787320588235294\\
-1.55234011020725	-0.78\\
-1.55554398770551	-0.77\\
-1.55875788420871	-0.76\\
-1.56	-0.756139848209245\\
-1.56195008766794	-0.75\\
-1.565132244921	-0.74\\
-1.56832434564426	-0.73\\
-1.57	-0.724759238534123\\
-1.57150195876006	-0.72\\
-1.57466247597394	-0.71\\
-1.57783286024432	-0.7\\
-1.58	-0.693179018612521\\
-1.58099690846476	-0.69\\
-1.58413586511761	-0.68\\
-1.58728461156074	-0.67\\
-1.59	-0.661399484826657\\
-1.59043610343627	-0.66\\
-1.59355357840276	-0.65\\
-1.59668076506794	-0.64\\
-1.59981775498872	-0.63\\
-1.6	-0.629419152276295\\
-1.60291676415546	-0.62\\
-1.60602246863574	-0.61\\
-1.60913789668219	-0.6\\
-1.61	-0.597235196573688\\
-1.61222655342697	-0.59\\
-1.61531085297429	-0.58\\
-1.61840479561713	-0.57\\
-1.62	-0.564852423653351\\
-1.62148406041696	-0.56\\
-1.62454703205241	-0.55\\
-1.62761956556089	-0.54\\
-1.63	-0.53227127886212\\
-1.6306903828715	-0.53\\
-1.63373210349076	-0.52\\
-1.63678330403852	-0.51\\
-1.63984407391128	-0.5\\
-1.64	-0.499490648747095\\
-1.64286714893092	-0.49\\
-1.64589709269719	-0.48\\
-1.64893652229558	-0.47\\
-1.65	-0.466504901960784\\
-1.65195323438186	-0.46\\
-1.65496199764787	-0.45\\
-1.65798016258191	-0.44\\
-1.66	-0.433321732589482\\
-1.66099141054481	-0.43\\
-1.66397906978545	-0.42\\
-1.66697604589053	-0.41\\
-1.66998242674802	-0.4\\
-1.67	-0.399941548101385\\
-1.67294934508864	-0.39\\
-1.67592520851292	-0.38\\
-1.67891039038028	-0.37\\
-1.68	-0.366354119344921\\
-1.68187384490054	-0.36\\
-1.68482867218782	-0.35\\
-1.68779273102473	-0.34\\
-1.69	-0.332570554040927\\
-1.69075357619106	-0.33\\
-1.69368744435908	-0.32\\
-1.69663045663229	-0.31\\
-1.69958269924871	-0.3\\
-1.7	-0.298587127158556\\
-1.7025025184165	-0.29\\
-1.70542456117388	-0.28\\
-1.70835574537599	-0.27\\
-1.71	-0.264400330814336\\
-1.71127487392047	-0.26\\
-1.71417602486071	-0.25\\
-1.71708622785557	-0.24\\
-1.72	-0.230019014216822\\
-1.72000547681161	-0.23\\
-1.72288581434944	-0.22\\
-1.72577511409438	-0.21\\
-1.7286734605013	-0.2\\
-1.73	-0.195429685686502\\
-1.73155488293325	-0.19\\
-1.734423358197	-0.18\\
-1.73730078888131	-0.17\\
-1.74	-0.160646845830984\\
-1.74018417071964	-0.16\\
-1.74303190113914	-0.15\\
-1.74588849534309	-0.14\\
-1.74875403687551	-0.13\\
-1.75	-0.125657843137255\\
-1.75160167092883	-0.12\\
-1.75443750885052	-0.11\\
-1.75728220110853	-0.1\\
-1.76	-0.090474572448788\\
-1.76013358275623	-0.0900000000000001\\
-1.76294874559809	-0.0800000000000001\\
-1.76577266945436	-0.0700000000000001\\
-1.76860543689606	-0.0600000000000001\\
-1.77	-0.0550846309482731\\
-1.77142310914585	-0.05\\
-1.77422634655732	-0.04\\
-1.77703833291774	-0.03\\
-1.77985915080417	-0.02\\
-1.78	-0.0195007601935036\\
-1.78264412459279	-0.01\\
-1.78543541269057	0\\
-1.78823543637593	0.00999999999999979\\
-1.79	0.0162895189865477\\
-1.79102688337929	0.02\\
-1.79379755722994	0.0299999999999998\\
-1.79657687049904	0.04\\
-1.79936490469475	0.0499999999999998\\
-1.8	0.052276295133438\\
-1.80212563533058	0.0600000000000001\\
-1.80488432323368	0.0699999999999998\\
-1.80765163463914	0.0800000000000001\\
-1.81	0.0884635721224656\\
-1.8104205036589	0.0899999999999999\\
-1.81315865254013	0.1\\
-1.81590532733838	0.11\\
-1.81866060843617	0.12\\
-1.82	0.124853761686641\\
-1.82140070436051	0.13\\
-1.82412683008937	0.14\\
-1.82686146332724	0.15\\
-1.82960468437429	0.16\\
-1.83	0.161440404849399\\
-1.83231697826714	0.17\\
-1.83503105246808	0.18\\
-1.83775361449996	0.19\\
-1.84	0.198229699742995\\
-1.84047659539137	0.2\\
-1.84317020080715	0.21\\
-1.8458721940206	0.22\\
-1.84858265411812	0.23\\
-1.85	0.235220588235294\\
-1.85127972151244	0.24\\
-1.85396123758257	0.25\\
-1.85665111937026	0.26\\
-1.85934944582851	0.27\\
-1.86	0.272409118884632\\
-1.8620215481151	0.28\\
-1.86469094477332	0.29\\
-1.86736868382249	0.3\\
-1.87	0.309796455089446\\
-1.87005391687716	0.31\\
-1.87270292311615	0.32\\
-1.87536016952072	0.33\\
-1.8780257337473	0.34\\
-1.88	0.347389377312347\\
-1.88068783556356	0.35\\
-1.88332468565625	0.36\\
-1.885969750291	0.37\\
-1.88862310694148	0.38\\
-1.89	0.385180792883813\\
-1.89126300337487	0.39\\
-1.89388766644509	0.4\\
-1.89652051721912	0.41\\
-1.8991616329713	0.42\\
-1.9	0.423171114599686\\
-1.90178024339897	0.43\\
-1.90439268808741	0.44\\
-1.90701329243358	0.45\\
-1.90964213349596	0.46\\
-1.91	0.461360719110301\\
-1.91224037083029	0.47\\
-1.91484056539069	0.48\\
-1.91744889036272	0.49\\
-1.92	0.499750730087109\\
-1.92006430521698	0.5\\
-1.92264419326086	0.51\\
-1.92523210565477	0.52\\
-1.9278281180229	0.53\\
-1.93	0.538344307031568\\
-1.93042490595683	0.54\\
-1.93299251095393	0.55\\
-1.93556810894403	0.56\\
-1.93815177528927	0.57\\
-1.94	0.577137452173367\\
-1.94073082633959	0.58\\
-1.94328611710076	0.59\\
-1.9458493683426	0.6\\
-1.94842065514932	0.61\\
-1.95	0.616130392156862\\
-1.95098285275295	0.62\\
-1.95352579806083	0.63\\
-1.95607667019301	0.64\\
-1.95863554394008	0.65\\
-1.96	0.655323316387372\\
-1.96118176488955	0.66\\
-1.96371233358622	0.67\\
-1.96625079431893	0.68\\
-1.96879722156906	0.69\\
-1.97	0.694716376847986\\
-1.97132833595749	0.7\\
-1.97384649703045	0.71\\
-1.97637251423225	0.72\\
-1.97890646171971	0.73\\
-1.98	0.734309687951278\\
-1.98142333287593	0.74\\
-1.98392905554231	0.75\\
-1.98644259732522	0.76\\
-1.98896403204171	0.77\\
-1.99	0.774103326426172\\
-1.99146751645619	0.78\\
-1.99396077024517	0.79\\
-1.9964618050479	0.8\\
-1.99897069432662	0.81\\
-2	0.814097331240189\\
1024	372\\
2	0.801564945226917\\
1.99960624804035	0.8\\
1.99709143709267	0.79\\
1.99458450536989	0.78\\
1.99208537929269	0.77\\
1.99	0.761633909736786\\
1.98958683639295	0.76\\
1.98705948039637	0.75\\
1.98454004208074	0.74\\
1.98202844751194	0.73\\
1.98	0.721903386117544\\
1.97951627086487	0.72\\
1.9769764027744	0.71\\
1.97444448971503	0.7\\
1.97192045741282	0.69\\
1.97	0.682373323633406\\
1.96939378774513	0.68\\
1.96684144085107	0.67\\
1.9642970852176	0.66\\
1.96176064624615	0.65\\
1.96	0.643043633860977\\
1.95921861742794	0.64\\
1.95665382527176	0.63\\
1.95409705947106	0.62\\
1.95154824511799	0.61\\
1.95	0.603914190774043\\
1.94898998423478	0.6\\
1.94641278052313	0.59\\
1.94384363711413	0.58\\
1.94128247880623	0.57\\
1.94	0.564984830863502\\
1.93870710622164	0.56\\
1.93611752473861	0.55\\
1.93353603634548	0.54\\
1.93096256556263	0.53\\
1.93	0.526255353292291\\
1.92836919497124	0.52\\
1.92576726948888	0.51\\
1.92317346871236	0.5\\
1.92058771689985	0.49\\
1.92	0.487725520084963\\
1.91797545536943	0.48\\
1.91536121955663	0.47\\
1.91275513888383	0.46\\
1.91015713736296	0.45\\
1.91	0.44939505635166\\
1.90752508536551	0.44\\
1.90489857269529	0.43\\
1.90228024440787	0.42\\
1.9	0.411267605633803\\
1.89966432187902	0.41\\
1.89701727571597	0.4\\
1.89437851937119	0.39\\
1.89174797545199	0.38\\
1.89	0.373341373432173\\
1.88911049378617	0.37\\
1.88645120971097	0.36\\
1.88380024248863	0.35\\
1.88115751452686	0.34\\
1.88	0.335614077639557\\
1.87849755132555	0.33\\
1.87582606288338	0.32\\
1.87316291709745	0.31\\
1.87050803619244	0.3\\
1.87	0.298085294668791\\
1.86782466253908	0.29\\
1.86514100269954	0.28\\
1.86246571008246	0.27\\
1.86	0.260756927314844\\
1.85979525573791	0.26\\
1.85709098717514	0.25\\
1.85439518823154	0.24\\
1.85170777983442	0.23\\
1.85	0.223632720875684\\
1.84901207470627	0.22\\
1.84629567634289	0.21\\
1.84358776981023	0.2\\
1.84088827590182	0.19\\
1.84	0.186706142598262\\
1.8381663684109	0.18\\
1.83543787214682	0.17\\
1.83271788865865	0.16\\
1.83000633862314	0.15\\
1.83	0.149976623802241\\
1.82725727054954	0.14\\
1.82451670730109	0.13\\
1.82178467650526	0.12\\
1.82	0.11345433164446\\
1.81904513689942	0.11\\
1.81628390512939	0.1\\
1.81353130472305	0.0899999999999999\\
1.81078725517632	0.0800000000000001\\
1.81	0.0771285198837899\\
1.8080186435416	0.0699999999999998\\
1.80524538604029	0.0600000000000001\\
1.80248077710537	0.0499999999999998\\
1.8	0.0410015649452264\\
1.79972007196425	0.04\\
1.79692607122821	0.0299999999999998\\
1.79414081660586	0.02\\
1.79136422646629	0.00999999999999979\\
1.79	0.00507921134068946\\
1.78857249898696	0\\
1.78576651366154	-0.01\\
1.78296928911215	-0.02\\
1.78018074367731	-0.03\\
1.78	-0.0306482861959176\\
1.77735699607153	-0.04\\
1.77453905312264	-0.05\\
1.77172988431298	-0.0600000000000001\\
1.77	-0.0661697660965002\\
1.76891137935486	-0.0700000000000001\\
1.76607263520286	-0.0800000000000001\\
1.76324275995375	-0.0900000000000001\\
1.76042167091167	-0.1\\
1.76	-0.101495384999688\\
1.75756913963791	-0.11\\
1.75471847599837	-0.12\\
1.75187669201621	-0.13\\
1.75	-0.136617474589523\\
1.74902765817745	-0.14\\
1.74615612526942	-0.15\\
1.74329356514155	-0.16\\
1.74043989412338	-0.17\\
1.74	-0.171542221944669\\
1.73755478788456	-0.18\\
1.73467137139819	-0.19\\
1.73179693578965	-0.2\\
1.73	-0.20626357766034\\
1.72891353111786	-0.21\\
1.72600917900418	-0.22\\
1.72311389914422	-0.23\\
1.72022760695481	-0.24\\
1.72	-0.240788761940438\\
1.71730604313148	-0.25\\
1.71438984021525	-0.26\\
1.71148271495306	-0.27\\
1.71	-0.275108341408522\\
1.7085610057407	-0.28\\
1.70562380176241	-0.29\\
1.70269576496694	-0.3\\
1.7	-0.30923317683881\\
1.69977309540333	-0.31\\
1.69681481309485	-0.32\\
1.69386578700917	-0.33\\
1.69092593077678	-0.34\\
1.69	-0.343152633469745\\
1.68796188987573	-0.35\\
1.68499179738092	-0.36\\
1.68203096232705	-0.37\\
1.68	-0.376874060173891\\
1.67906403391244	-0.38\\
1.67607279845764	-0.39\\
1.6730909074835	-0.4\\
1.67011827386644	-0.41\\
1.67	-0.410397919722838\\
1.66710777845963	-0.42\\
1.66410475492713	-0.43\\
1.66111107431091	-0.44\\
1.66	-0.443715653315003\\
1.65809571120623	-0.45\\
1.6550714788594	-0.46\\
1.65205667438381	-0.47\\
1.65	-0.47683639562158\\
1.64903555585281	-0.48\\
1.64599003873466	-0.49\\
1.64295403380662	-0.5\\
1.64	-0.509759569273148\\
1.63992625660939	-0.51\\
1.63685937897419	-0.52\\
1.63380209718034	-0.53\\
1.63075432169191	-0.54\\
1.63	-0.54247687154502\\
1.62767842866075	-0.55\\
1.62459979367552	-0.56\\
1.62153074710063	-0.57\\
1.62	-0.574995411639682\\
1.61844610121264	-0.58\\
1.6153460367015	-0.59\\
1.61225564198741	-0.6\\
1.61	-0.60731547183385\\
1.60916129403596	-0.61\\
1.60603972355406	-0.62\\
1.60292790350795	-0.63\\
1.6	-0.63943661971831\\
1.59982288815402	-0.64\\
1.59667973503991	-0.65\\
1.59354641222321	-0.66\\
1.59042282800096	-0.67\\
1.59	-0.671354217789017\\
1.58726493507712	-0.68\\
1.58411003169494	-0.69\\
1.58096494517324	-0.7\\
1.58	-0.703071006434684\\
1.57779417026992	-0.71\\
1.57461760805567	-0.72\\
1.57145094014678	-0.73\\
1.57	-0.734588423075722\\
1.56826626945678	-0.74\\
1.56506796955177	-0.75\\
1.56187964054891	-0.76\\
1.56	-0.76590618395548\\
1.55868004323017	-0.77\\
1.55545992605828	-0.78\\
1.55224985551088	-0.79\\
1.55	-0.797024042220485\\
1.54903428342676	-0.8\\
1.54579226856396	-0.81\\
1.54256037514796	-0.82\\
1.54	-0.827941788149909\\
1.53932776258635	-0.83\\
1.53606376862522	-0.84\\
1.53280997000824	-0.85\\
1.53	-0.858659249350731\\
1.52955923337824	-0.86\\
1.52627317778718	-0.87\\
1.52299739048856	-0.88\\
1.52	-0.889176290918195\\
1.51972742799313	-0.89\\
1.51641922697043	-0.9\\
1.51312136621606	-0.91\\
1.51	-0.919492815561329\\
1.50983105749912	-0.92\\
1.50650062582164	-0.93\\
1.50318060539353	-0.94\\
1.5	-0.94960876369327\\
1.49986881115991	-0.95\\
1.49651606202644	-0.96\\
1.49317379410692	-0.97\\
1.49	-0.979524113486275\\
1.48983935571341	-0.98\\
1.48646420058259	-0.99\\
1.48309959559313	-1\\
1.48	-1.00923888089128\\
1.47974133460889	-1.01\\
1.47634368303158	-1.02\\
1.47295664946611	-1.03\\
1.47	-1.03875311962201\\
1.46957336720068	-1.04\\
1.46615312664677	-1.05\\
1.46274357089945	-1.06\\
1.46	-1.06806692110367\\
1.45933404789627	-1.07\\
1.45589112357582	-1.08\\
1.45245894976318	-1.09\\
1.45	-1.09718041438624\\
1.44902194525671	-1.1\\
1.44555623993539	-1.11\\
1.44210134971274	-1.12\\
1.44	-1.12609376602264\\
1.43863560104712	-1.13\\
1.43514701485571	-1.14\\
1.43166930722778	-1.15\\
1.43	-1.15480717991187\\
1.42817352923476	-1.16\\
1.42466195947284	-1.17\\
1.42116133059846	-1.18\\
1.42	-1.18332089710752\\
1.41763421493231	-1.19\\
1.41409955586585	-1.2\\
1.41057589885668	-1.21\\
1.41	-1.21163519559177\\
1.40701611328372	-1.22\\
1.40345825593658	-1.23\\
1.4	-1.23974960876369\\
1.39991004190417	-1.24\\
1.3963176482899	-1.25\\
1.39273648022906	-1.26\\
1.39	-1.26765953363986\\
1.38915309011774	-1.27\\
1.38553721157136	-1.28\\
1.3819326166858	-1.29\\
1.38	-1.29537048196537\\
1.37831264334426	-1.3\\
1.37467316106484	-1.31\\
1.37104501933791	-1.32\\
1.37	-1.32288288047722\\
1.36738702859605	-1.33\\
1.36372381965069	-1.34\\
1.36007200692585	-1.35\\
1.36	-1.35019719153717\\
1.35637453662619	-1.36\\
1.3526874737077	-1.37\\
1.35	-1.37730551211884\\
1.34899608127225	-1.38\\
1.3452734204142	-1.39\\
1.34156237159187	-1.4\\
1.34	-1.40421555958258\\
1.33782872948749	-1.41\\
1.3340818935722	-1.42\\
1.33034672203537	-1.43\\
1.33	-1.43092852305003\\
1.32656928033231	-1.44\\
1.3227981286675	-1.45\\
1.32	-1.45743701757678\\
1.31902335840322	-1.46\\
1.31521586897746	-1.47\\
1.31142025545754	-1.48\\
1.31	-1.48374617943832\\
1.30759873816566	-1.49\\
1.3037665862072	-1.5\\
1.3	-1.50985915492958\\
1.29994550348993	-1.51\\
1.29607644746342	-1.52\\
1.29221947647366	-1.53\\
1.29	-1.53576477603076\\
1.28834857309145	-1.54\\
1.2844544886506	-1.55\\
1.28057253584996	-1.56\\
1.28	-1.56147553224699\\
1.27665009478134	-1.57\\
1.27273081385555	-1.58\\
1.27	-1.58698279964977\\
1.26880496334987	-1.59\\
1.26484797511384	-1.6\\
1.26090332272458	-1.61\\
1.26	-1.61229161306439\\
1.2569226509857	-1.62\\
1.25294006611347	-1.63\\
1.25	-1.6373993354183\\
1.24895344001318	-1.64\\
1.24493253278042	-1.65\\
1.24092416278734	-1.66\\
1.24	-1.66230722537938\\
1.23687928135847	-1.67\\
1.23283235203034	-1.68\\
1.23	-1.68701407017105\\
1.22877883530489	-1.69\\
1.22469294481627	-1.7\\
1.22061979000024	-1.71\\
1.22	-1.71152235749516\\
1.21650442155801	-1.72\\
1.21239205539347	-1.73\\
1.21	-1.73582729361383\\
1.20826522482679	-1.74\\
1.20411323216423	-1.75\\
1.2	-1.75993740219092\\
1.19997375806603	-1.76\\
1.19578171631042	-1.77\\
1.19160276323854	-1.78\\
1.19	-1.78383989722283\\
1.18739586297154	-1.79\\
1.18317658849205	-1.8\\
1.18	-1.80754648151623\\
1.17895398496822	-1.81\\
1.17469395087447	-1.82\\
1.17044719683335	-1.83\\
1.17	-1.83105337089173\\
1.16615311070603	-1.84\\
1.16186530747534	-1.85\\
1.16	-1.85435615822033\\
1.15755228237384	-1.86\\
1.15322297441187	-1.87\\
1.15	-1.8774618842846\\
1.1488896325733	-1.88\\
1.14451835480369	-1.89\\
1.14016070148908	-1.9\\
1.14	-1.90036881982151\\
1.13574955578332	-1.91\\
1.13134961427854	-1.92\\
1.13	-1.92307026921515\\
1.1269146324548	-1.93\\
1.12247191747947	-1.94\\
1.12	-1.94557364505845\\
1.11801158577231	-1.95\\
1.11352560100414	-1.96\\
1.11	-1.96787848206187\\
1.10903836028974	-1.97\\
1.10450859781831	-1.98\\
1.1	-1.98998435054773\\
1.09999284177261	-1.99\\
1.09541878152576	-2\\
};

\end{axis}

% \begin{axis}[%
% width=0.226in,
% height=3.566in,
% at={(5.053in,0.481in)},
% scale only axis,
% axis on top,
% xmin=-1,
% xmax=2,
% xtick={\empty},
% ymin=0.5,
% ymax=10.5,
% ytick={0.5,1.5,2.5,3.5,4.5,5.5,6.5,7.5,8.5,9.5,10.5},
% yticklabels={{0},{102.4},{204.8},{307.2},{409.6},{512},{614.399},{716.799},{819.199},{921.599},{1024}},
% axis background/.style={fill=white},
% yticklabel pos=right,
% legend style={legend cell align=left, align=left, draw=white!15!black}
% ]
% \addplot [forget plot] graphics [xmin=-1.5, xmax=2.5, ymin=1, ymax=1] {resources/rosenbrock-1.png};
% \end{axis}
\end{tikzpicture}%
            \caption{Contour lines of the Rosenbrock function $f$, where each line represents a value of $f$ from $2^{-5}$ to $2^4$, which increases from inner to outer.}
            \label{fig:rosenbrock}
        \end{figure}
    \end{solution}

    \begin{problem}
        Show that the one-dimensional minimizer of a strongly convex quadratic function always satisfies the Goldstein conditions
        \begin{equation}
            f(x_k)+(1-c)\ak\dfd\leq f(x_k+\ak d_k)\leq f(x_k)+c\ak\dfd,
        \end{equation}
        with $0<c<1/2$.
    \end{problem}
    \begin{proof}
        Let the quadratic function
        \begin{equation}
            f(x)=\frac{1}{2}x^T\matr{A}x+b^Tx+c,
        \end{equation}
        where $\matr{A}$ is a symmetric matrix.

        Since $f$ is strongly convex, by definition we have
        \begin{equation}
            f\blxpx-\left(\lambda f(x_1)+(1-\lambda)f(x_2)\right)<0,\quad\text{for }x_1\neq x_2\text{ and } \lambda\in(0,1).
        \end{equation}
        Hence
        \begin{equation}
            \begin{aligned}
                0
                &>f\blxpx-\left(\frac{1}{2} f(x_1)+f(x_2)\right)\\
                &=\left(\frac{1}{2}\blxpx^T\matr{A}\blxpx+b^T\blxpx+c\right)\\
                &\phantom{{}={}}-\left(\frac{1}{2}\left(\frac{1}{2} {x_1}^T\matr{A}x_1+\frac{1}{2}{x_2}^T\matr{A}x_2\right)+\frac{1}{2}\left(b^Tx_1+b^Tx_2\right)+c\right)\\
                &= -\frac{1}{2}\blxmx^T\matr{A}\blxmx.
            \end{aligned}
        \end{equation}
        Since $x_1, x_2$ are arbitrary, $\matr{A}$ must be positive definite.

        Only for convenience of typography, let's omit the iterator $k$ in the following context. Given a descent direction $d$, we can calculate $\alpha$ explicitly as follows since $f$ is a quadratic function, that is
        \begin{equation}
            \begin{aligned}
                \alpha
                &=\arg\min f(x+\alpha d)\\
                &=\arg\min \left(\frac{1}{2}d^T\matr{A}d\right)\alpha^2+\left(d^T\matr{A}x+d^Tb\right)\alpha+\frac{1}{2}\left(x^T\matr{A}x+b^Tx+c\right)\\
                &= -\frac{d^T(\matr{A}x+b)}{d^T\matr{A}d},
            \end{aligned}
        \end{equation}
        which is greater than 0 because $\matr{A}$ is positive definite and $\matr{A}x+b$ is the steepest descent direction. Thus
        \begin{equation}
            \frac{f(x+\alpha d)-f(x)}{\alpha\nabla f^Td}=\left.\left\{-\frac{1}{2}\frac{\left(d^T(\matr{A}x+b)\right)^2}{d^T\matr{A}d}\right\}\middle/\left(-\frac{d^T(\matr{A}x+b)}{d^T\matr{A}d}(\matr{A}x+b)^Td\right)=\frac{1}{2}\right.,
        \end{equation}
        since $(\matr{A}x+b)^Td$ is a scalar and its transpose equals itself. It directly follows that the step length always satisfies the Goldstein conditions.
    \end{proof}

    \begin{problem}
        Prove that $\norm{\matr{B}x}\geq\norm{x}/\nbi$ for any nonsingular matrix $\matr{B}$. Use this fact to establish
        \begin{equation}
            \cos{\theta_k}\doteq\frac{-\dfd}{\norm{\nabla f_k}\norm{d_k}}\geq\frac{1}{M},
        \end{equation}
        where the constant $M$ satisfies
        \begin{equation}
            \nbk\nbki\leq M,\quad \text{for all } k,
        \end{equation}
        and where
        \begin{equation}
            d_k=-\matr{B}_k^{-1}\nabla f_k.
        \end{equation}
    \end{problem}
    \begin{proof}
        Notice that the matrix norm $\norm{\cdot}$ is compatible with the vector norm $\norm{\cdot}$, that is
        \begin{equation}
            \norm{\matr{B}}=\sup_{x\neq 0}\frac{\norm{\matr{B}x}}{\norm{x}}.
        \end{equation}
        Thus
        \begin{equation}
            \norm{\matr{B}}\norm{x}\geq\norm{\matr{B}x}.
        \end{equation}
        Replace $\matr{A}x$ by $\matr{B}^{-1}$ and $\matr{B}x$ respectively, we have
        \begin{equation}
            \norm{\matr{B}^{-1}}\norm{\matr{B}x}\geq\norm{x},
        \end{equation}
        since that $\matr{B}$ is nonsingular yields that $\norm{\matr{B}^{-1}}\neq 0$.

        It is easy to prove by definition that for any positive definite matrix $\matr{B}$, the 2-norm
        \begin{equation}
            \norm{\matr{B}}=\lambda_{\symup{max}}.
        \end{equation}

        Only for convenience of typography, let's omit the iterator $k$ in the following context. Let $\matr{B}=\matr{Q}^T\matr{D}\matr{Q}$
        denote the orthonormal diagonalization of $\matr{B}$,
        then
        \begin{equation}
            \begin{aligned}
                \cos{\theta}
                &\doteq\frac{-\nabla f^Td}{\norm{\nabla f}\norm{d}}
                =\frac{\df^T\matr{Q}^T\matr{D}^{-1}\matr{Q}\df}{\norm{\df}\norm{\matr{D}^{-1}\df}}\\
                &\geq\frac{\lambda_{\symup{min}}(\matr{D}^{-1})\norm{\matr{Q}\df}^2}{\norm{\df}^2\norm{\matr{D}^{-1}}}
                =\frac{1}{\norm{\matr{B}}\norm{\matr{B}^{-1}}}
                \geq\frac{1}{M},
            \end{aligned}
        \end{equation}
        as desired.
    \end{proof}

    \begin{problem}
        Prove the result
        \begin{equation}
            \nqs{x_{k+1}-x^{*}}=\left\{1-\frac{\left(\dfdf\right)^2}{\left(\dfqdf\right)\left(\dfqidf\right)}\right\}\nqs{x_k-x^{*}}
        \end{equation}
        by working through the following steps. First use
        \begin{equation}
            \ak=\frac{\dfdf}{\dfqdf}\quad\text{and}\quad x_{k+1}=x_k-\ak\df_k
        \end{equation}
        to show that
        \begin{equation}
            \nqs{x_k-x^{*}}-\nqs{x_{k+1}-x^{*}}=2\ak\df_k^T\matr{Q}(x_k-x^*)-\ak^2\df_k^T\matr{Q}\df_k,
        \end{equation}
        where the weight norm is defined as $\nqs{x}=x^T\matr{Q}x$. Second, use the fact that $\df_k=\matr{Q}(x_k-x^*)$ to obtain
        \begin{equation}
            \nqs{x_k-x^*}-\nqs{x_{k+1}-x^{*}}=\frac{2\left(\dfdf\right)^2}{\dfqdf}-\frac{\left(\dfdf\right)^2}{\dfqdf}
        \end{equation}
        and
        \begin{equation}
            \nqs{x_k-x^*}=\dfqidf.
        \end{equation}
    \end{problem}
    \begin{proof}
        First,
        \begin{equation}
            \begin{aligned}
                &\phantom{{}={}}\nqs{x_k-x^{*}}-\nqs{x_{k+1}-x^{*}}\\
                &=\left(x_k-x^{*}\right)^T\matr{Q}\left(x_k-x^{*}\right)-\left(x_{k}-x^{*}-\ak\df_k\right)^T\matr{Q}\left(x_{k}-x^{*}-\ak\df_k\right)\\
                &=2\left(\ak\df_k\right)^T\matr{Q}\left(x_{k}-x^{*}\right)-\left(\ak\df_k\right)^T\matr{Q}\left(\ak\df_k\right)\\
                &=2\ak\df_k^T\matr{Q}(x_k-x^*)-\ak^2\df_k^T\matr{Q}\df_k.
            \end{aligned}
        \end{equation}
        Second, apply the value of $\ak$ the the fact that $\df_k=\matr{Q}(x_k-x^*)$, we have
        \begin{equation}
            \begin{aligned}
                &\phantom{{}={}}\nqs{x_k-x^{*}}-\nqs{x_{k+1}-x^{*}}=2\ak\df_k^T\df_k-\ak^2\df_k^T\matr{Q}\df_k\\
                &=2\left(\frac{\dfdf}{\dfqdf}\df_k\right)^T\df_k-\left(\frac{\dfdf}{\dfqdf}\df_k\right)^T\matr{Q}\left(\frac{\dfdf}{\dfqdf}\df_k\right)\\
                &=\frac{2\left(\dfdf\right)^2}{\dfqdf}-\frac{\left(\dfdf\right)^2}{\dfqdf}
            \end{aligned}
        \end{equation}
        and
        \begin{equation}
            \begin{aligned}
                \nqs{x_k-x^*}
                &=\left(x_k-x^{*}\right)^T\matr{Q}^T\matr{Q}^{-1}\matr{Q}\left(x_k-x^{*}\right)\\
                &=\dfqidf.
            \end{aligned}
        \end{equation}
        It follows the conclusion.
    \end{proof}
    
    \begin{problem}(Kantorovich Inequality)
        Let $\matr{Q}$ be a positive definite symmetric matrix. Prove that for any vector x, we have
        \begin{equation}
            \frac{\left(x^Tx\right)^2}{\left(x^T\matr{Q}x\right)\left(x^T\matr{Q}^{-1}x\right)}\geq\frac{4\lambda_n\lambda_1}{\left(\lambda_n+\lambda_1\right)^2},
        \end{equation}
        where $\lambda_n$ and $\lambda_1$ are, respectively, the largest and smallest eigenvalues of $\matr{Q}$.
    \end{problem}
    \begin{proof}
        Without loss of generality assume that $Q$ is a positive definite diagnoal matrix, since we can write down the orthonormal diagonalization $\matr{Q}=\matr{P}^T\matr{D}\matr{P}$ and denote $y$ by $\matr{P}x$. Let
        \begin{equation}
            \matr{D}=\diag\left(\lambda_1, \lambda_2, \dotsc, \lambda_n\right)\quad\text{and}\quad x=\left(x_1, x_2,\dotsc, x_n\right)^T,
        \end{equation}
        where $\lambda_1\leq\lambda_2\leq\dotsb\leq\lambda_n$.
        Then
        \begin{equation}
            \frac{\left(x^Tx\right)^2}{\left(x^T\matr{Q}x\right)\left(x^T\matr{Q}^{-1}x\right)}
            =\frac{\sum_{1}^{n}x_j^2}{\left(\sum_1^n\lambda_jx_j^2\right)\left(\sum_1^n\lambda_j^{-1}x_j^2\right)}.
        \end{equation}
        Hence it is equivalent to prove that
        \begin{equation}\label{eq:5proof}
            \frac{\left(\sum_1^n\lambda_jp_j\right)\left(\sum_1^n\lambda_j^{-1}p_j\right)}{\left(\sum_{1}^{n}p_j\right)^2}\leq\frac{\left(\lambda_n+\lambda_1\right)^2}{4\lambda_n\lambda_1},
        \end{equation}
        where $p_j>0$ for all $j$. Without loss of generality assume that $\sum_1^np_j=1$.

        Now let $X$ denote a random variable whose value is taking from $\left\{\lambda_1, \lambda_2, \dotsc, \lambda_n\right\}$ and the probability
        \begin{equation}
            \pr(X=j)=p_j,\quad\text{for}\ j=1,2\dotsc,n.
        \end{equation}
        Thus the left hand side of \ref{eq:5proof}
        \begin{equation}
            \symup{l.h.s.}=\expect(X)\expect\left(X^{-1}\right)=\expect\left(XX^{-1}\right)-\cov\left(X,X^{-1}\right)\leq 1+\left(\var(X)\var\left(X^{-1}\right)\right)^{1/2}.
        \end{equation}
        And for such a random variable $X$ with finite sample space, we have
        \begin{equation}
            \begin{aligned}
                \var(X)
                &=\sum_{j=1}^{n}p_j\left(\lambda_j-\sum_{k
                =1}^np_j\lambda_j\right)^2
                =\min_{\mu\in\symbb{R}}\sum_{j=1}^np_j(\lambda_j-\mu)^2\\
                &\leq \sum_{j=1}^np_j\left(\lambda_j-\frac{\lambda_1+\lambda_n}{2}\right)^2
                \leq \frac{\left(\lambda_n-\lambda_1\right)^2}{4}.
            \end{aligned}
        \end{equation}
        Similarly, for $X^{-1}$ we have
        \begin{equation}
            \var\left(X^{-1}\right) \leq \frac{\left(\lambda_n^{-1}-\lambda_1^{-1}\right)^2}{4}
            =\frac{\left(\lambda_n-\lambda_1\right)^2}{4\lambda_n^2\lambda_1^2}.
        \end{equation}
        Hence the left hand side of \ref{eq:5proof}
        \begin{equation}
            \symup{l.h.s.}\leq 1+\frac{\left(\lambda_n-\lambda_1\right)^2}{4\lambda_n\lambda_1}=\frac{\left(\lambda_n+\lambda_1\right)^2}{4\lambda_n\lambda_1}=\symup{r.h.s.},
        \end{equation}
        which completes the proof.
    \end{proof}

    \begin{problem}
        Program the BFGS algorithm using the line search algorithm describe in this chapter that implements the strong Wolfe conditions. Have the code verify that $y_k^Ts_k$ is always positive. Use it to minimize the Rosenbrock function using the starting points given in \ref{prob:sd}.
    \end{problem}
    \begin{solution}
        The result of quasi-Newton method with BFGS formula is listed below,while the codes \ref{code:bfgs} are put in the appendix.
        \begin{table}[htb]
            \begin{center}
                \caption{Total iterations of quasi-Newton method with BGFS formula with $\rho=0.9$, $c_1 = 10^{-4}$, $c_2=0.9$, $\matr{B}_0=\left(y_1^Ts_1\right)/\left(y_1^Ty_1\right)\matr{I}$ and $\varepsilon=0.001$.}
                \pgfplotstabletypeset[
                    string type,
                ]{
                    {Initial Point} {Total Iterations}
                    {$(1.2, 1.2)^T$} 680
                    {$(-1.2, 1)^T$} 848
                    {$(1.2, 1.2)^T$} 7
                    {$(-1.2, 1)^T$} 21
                }
            \end{center}
        \end{table}
        According to the discussions in chapter 6, we choose the parameters
        \begin{equation}
            c_1 = 10^{-4},\quad c_2=0.9,\quad \matr{B}_0=\matr{I},
        \end{equation}
        and we update $\matr{B}_0$ by setting
        \begin{equation}
            \matr{B}_0 \leftarrow \frac{y_0^Ts_0}{y_0^Ty_0}\matr{I}
        \end{equation}
        after the first step has been computed but before the first BFGS update is performed. The update of initial matrix is extremely important for the algorithm. Otherwise the approximation of Hessian matrix by $\matr{B}_k$ won't be accurate enough, then the initial step length in each iteration might be too short to satisfiy the strong Wolfe condition, or might be unnecessarily long so that extra tests for step length will be made, and it will affect the performance.
    \end{solution}

    \clearpage\appendix
    \section{Codes of Algorithms}

    \matlabinputlisting[caption={Steepest Descent Method}, label={code:sd}]{steepest_descent.m}
    \matlabinputlisting[caption={Newton Method}, label={code:nt}]{newton.m}
    \matlabinputlisting[caption={Quasi Newton Method with BFGS Formula}, label={code:bfgs}]{bfgs_quasi_newton.m}

    \section{Other Codes}

    \matlabinputlisting[caption={Rosenbrock Function with Vector Input}]{f.m}
    \matlabinputlisting[caption={Rosenbrock Function with Coodernite Input Used by Contour Plot}]{f_cdn.m}
    \matlabinputlisting[caption={Gradient of Rosenbrock Function}]{df.m}
    \matlabinputlisting[caption={Hessian Matrix of Rosenbrock Function}]{ddf.m}
    \matlabinputlisting[caption={Contour Plot of Rosenbrock Function}]{rosenbrock_contour.m}
    \matlabinputlisting[caption={Tests for the Algorithms}]{test.m}
\end{document}

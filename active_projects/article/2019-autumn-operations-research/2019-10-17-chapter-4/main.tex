% -------------------- Packages --------------------

\documentclass{assignment}[2019/10/15]
\usepackage[lineno, pgfplots]{packages}[2019/10/15]
\pgfplotsset{compat=1.3}

% -------------------- Settings --------------------

% Title

\title{Homework of Chapter 4}
\author{Chen Xuyang}
\date{\today}
\institute{School of Mathematical Science}
\professor{Chen Xiongda}
\course{Operations Research}
\subject{Operations Research}
\keywords{}

% -------------------- New commands --------------------

\newcommand{\BR}{\symbb{R}}
\newcommand{\diag}{\mathop{}\!\symup{diag}}
\newcommand{\pr}{\mathop{}\!\symup{Pr}}
\newcommand{\expect}{\mathop{}\!\symup{E}}
\newcommand{\cov}{\mathop{}\!\symup{Cov}}
\newcommand{\var}{\mathop{}\!\symup{Var}}

\newcommand{\B}{\matr{B}}
\newcommand{\Bi}{\matr{B}^{-1}}
\newcommand{\Bii}{\matr{B}^{-2}}
\newcommand{\pC}{p^{\symup{C}}}
\newcommand{\pB}{p^{\symup{B}}}

% -------------------- Document --------------------

\begin{document}
    \maketitle
    \begin{problem}\label{prob:1}
        Let $f(x)=10\left(x_2-x_1^2\right)^2+(1-x_1)^2$. At $x=(0, -1)^T$ draw the contour lines of the quadratic model
        \begin{equation}
            m(p)=f+g^Tp+\frac{1}{2}p^T\B p
        \end{equation}
        assuming that $\B$ is the Hessian of $f$. Draw the family of solutions of trust region subproblem
        \begin{equation}
            \min_{p\in\BR}m(p)=f+g^Tp+\frac{1}{2}p^T\B p\quad\text{s.t. }\norm{p}\leq\Delta
        \end{equation}
        as the trust region radius varies from $\Delta=0$ to $\Delta=2$. Repeat this at $x=(0, 0.5)^T$.
    \end{problem}
    \begin{solution}
        \begin{figure}[htb]
            \centering
            % This file was created by matlab2tikz.
%
%The latest updates can be retrieved from
%  http://www.mathworks.com/matlabcentral/fileexchange/22022-matlab2tikz-matlab2tikz
%where you can also make suggestions and rate matlab2tikz.
%
\definecolor{mycolor1}{rgb}{0.00000,0.44700,0.74100}%
\definecolor{mycolor2}{rgb}{0.85000,0.32500,0.09800}%
\definecolor{mycolor3}{rgb}{0.92900,0.69400,0.12500}%
\definecolor{mycolor4}{rgb}{0.49400,0.18400,0.55600}%
\definecolor{mycolor5}{rgb}{0.46600,0.67400,0.18800}%
\definecolor{mycolor6}{rgb}{0.30100,0.74500,0.93300}%
\definecolor{mycolor7}{rgb}{0.63500,0.07800,0.18400}%
%
\begin{tikzpicture}

\begin{axis}[%
width=4.069in,
height=3.566in,
at={(0.758in,0.481in)},
scale only axis,
point meta min=23,
point meta max=225.947954545455,
colormap={mymap}{[1pt] rgb(0pt)=(0,0,0.666667); rgb(1pt)=(0,0,1); rgb(4pt)=(0,1,1); rgb(7pt)=(1,1,0); rgb(9pt)=(1,0.333333,0)},
xmin=-2,
xmax=2,
ymin=-3,
ymax=1,
axis background/.style={fill=white},
axis x line*=bottom,
axis y line*=left,
legend style={legend cell align=left, align=left, draw=white!15!black}
]
\addplot[contour prepared, contour prepared format=matlab, contour/labels=false] table[row sep=crcr] {%
%
23.5022727272727	509\\
1.08385294301751	1\\
1.08382999144532	0.99\\
1.08376113672878	0.98\\
1.08364637886786	0.97\\
1.08348571786258	0.96\\
1.08327915371294	0.95\\
1.08302668641893	0.94\\
1.08272831598055	0.93\\
1.08238404239781	0.92\\
1.08199386567071	0.91\\
1.08155778579924	0.9\\
1.0810758027834	0.89\\
1.0805479166232	0.88\\
1.08	0.87045090909091\\
1.0799738754872	0.87\\
1.07934815126936	0.86\\
1.07867607710945	0.85\\
1.07795765300748	0.84\\
1.07719287896345	0.83\\
1.07638175497735	0.82\\
1.07552428104919	0.81\\
1.07462045717897	0.8\\
1.07367028336669	0.79\\
1.07267375961235	0.78\\
1.07163088591594	0.77\\
1.07054166227747	0.76\\
1.07	0.755230055658625\\
1.06940025104781	0.75\\
1.06820671020999	0.74\\
1.06696636384912	0.73\\
1.06567921196519	0.72\\
1.06434525455822	0.71\\
1.06296449162819	0.7\\
1.06153692317512	0.69\\
1.06006254919899	0.68\\
1.06	0.679588811188811\\
1.0585268902688	0.67\\
1.05694334028061	0.66\\
1.05531252014353	0.65\\
1.05363442985754	0.64\\
1.05190906942266	0.63\\
1.05013643883887	0.62\\
1.05	0.619250295159385\\
1.0482996592808	0.61\\
1.04641376765989	0.6\\
1.04448013194731	0.59\\
1.04249875214306	0.58\\
1.04046962824714	0.57\\
1.04	0.567738766980146\\
1.03837648245172	0.56\\
1.03623035271938	0.55\\
1.03403599535261	0.54\\
1.0317934103514	0.53\\
1.03	0.522171291866028\\
1.02949750858155	0.52\\
1.02713453659617	0.51\\
1.02472284353892	0.5\\
1.02226242940981	0.49\\
1.02	0.480983230361871\\
1.01975074396438	0.48\\
1.01716644664713	0.47\\
1.01453292461907	0.46\\
1.01185017788021	0.45\\
1.01	0.443227682227681\\
1.00910899595306	0.44\\
1.00629874974565	0.43\\
1.00343876466731	0.42\\
1.00052904071805	0.41\\
1	0.408212375859434\\
0.997543923781672	0.4\\
0.994502958715072	0.39\\
0.991411729763074	0.38\\
0.99	0.375506181818182\\
0.988251783776295	0.37\\
0.985025977324682	0.36\\
0.981749370771469	0.35\\
0.98	0.344742539902845\\
0.978404948068619	0.34\\
0.974990313922278	0.33\\
0.971524331894037	0.32\\
0.97	0.315666224286662\\
0.967985277115825	0.31\\
0.964377698605573	0.3\\
0.960718212490857	0.29\\
0.96	0.288064844246663\\
0.956973831731113	0.28\\
0.953169056081677	0.27\\
0.95	0.26178416821274\\
0.949304131792855	0.26\\
0.94535082852939	0.25\\
0.94134446079259	0.24\\
0.94	0.236688057040998\\
0.937254433251213	0.23\\
0.933095592360417	0.22\\
0.93	0.21265141864505\\
0.928870358949056	0.21\\
0.924555569261132	0.2\\
0.920186505489083	0.19\\
0.92	0.189578360290017\\
0.915714321363512	0.18\\
0.911185087215831	0.17\\
0.91	0.167414806750137\\
0.906561308727374	0.16\\
0.90186816793315	0.15\\
0.9	0.146065922381712\\
0.897085493882347	0.14\\
0.89222457789471	0.13\\
0.89	0.125475844155844\\
0.887275312120351	0.12\\
0.882242613797917	0.11\\
0.88	0.105593702386998\\
0.877118639633748	0.1\\
0.871910006540222	0.0899999999999999\\
0.87	0.0863730750124194\\
0.866602759314673	0.0800000000000001\\
0.861213886608585	0.0699999999999998\\
0.86	0.0677715119105491\\
0.855714324012761	0.0600000000000001\\
0.850140746896866	0.0499999999999998\\
0.85	0.0497501189909567\\
0.84443931702815	0.04\\
0.84	0.0322967498822418\\
0.83865959259972	0.0299999999999998\\
0.832763009593974	0.02\\
0.83	0.0153617904937702\\
0.826764861478491	0.00999999999999979\\
0.820669915077266	0\\
0.82	-0.00108819538670312\\
0.814442219276225	-0.00999999999999979\\
0.81	-0.017052843708016\\
0.808119230659312	-0.02\\
0.801674716355634	-0.0299999999999998\\
0.8	-0.0325735617039968\\
0.79510266732312	-0.04\\
0.79	-0.0476637668551544\\
0.788423400945196	-0.0499999999999998\\
0.781610356063052	-0.0600000000000001\\
0.78	-0.062341442594964\\
0.774660318404999	-0.0699999999999998\\
0.77	-0.0766219873150107\\
0.767589536251999	-0.0800000000000001\\
0.760387412123224	-0.0899999999999999\\
0.76	-0.0905330012453304\\
0.753021633364952	-0.1\\
0.75	-0.104062114356232\\
0.745519724386232	-0.11\\
0.74	-0.117249898083979\\
0.737875743930472	-0.12\\
0.730082186604528	-0.13\\
0.73	-0.130104525430517\\
0.722104489535069	-0.14\\
0.72	-0.142614529575228\\
0.713966205040054	-0.15\\
0.71	-0.154812672176309\\
0.70566000460966	-0.16\\
0.7	-0.166706984003121\\
0.697178106924337	-0.17\\
0.69	-0.178305222437137\\
0.688512242333684	-0.18\\
0.68	-0.189614883007288\\
0.679653613938244	-0.19\\
0.670583405209591	-0.2\\
0.67	-0.200637872500943\\
0.661298601128914	-0.21\\
0.66	-0.211385708941264\\
0.651794305263533	-0.22\\
0.65	-0.22186827458256\\
0.642059494909236	-0.23\\
0.64	-0.232092013249908\\
0.632082396727715	-0.24\\
0.63	-0.242063161737861\\
0.621850421743205	-0.25\\
0.62	-0.251787758058674\\
0.611350093465075	-0.26\\
0.61	-0.261271649299317\\
0.600566968273077	-0.27\\
0.6	-0.270520499108734\\
0.59	-0.279536185383244\\
0.589475973738269	-0.28\\
0.58	-0.28832189600283\\
0.578052944797866	-0.29\\
0.57	-0.296886977886978\\
0.56628958708112	-0.3\\
0.56	-0.305236502960641\\
0.554165777910499	-0.31\\
0.55	-0.313375388869686\\
0.541659776317508	-0.32\\
0.54	-0.321308404802745\\
0.53	-0.329032933104631\\
0.528721831784176	-0.33\\
0.52	-0.336549540347293\\
0.515306812392905	-0.34\\
0.51	-0.343872592091923\\
0.501419715110501	-0.35\\
0.5	-0.351006373700101\\
0.49	-0.357939953035894\\
0.486960954124808	-0.36\\
0.48	-0.364683982683983\\
0.471914915168397	-0.37\\
0.47	-0.371249917355372\\
0.46	-0.377624462809917\\
0.456181924446098	-0.38\\
0.45	-0.383818510009846\\
0.44	-0.389843780767969\\
0.439734033636922	-0.39\\
0.43	-0.395676441837732\\
0.42238944651563	-0.4\\
0.42	-0.401347783888709\\
0.41	-0.406838887091556\\
0.404082127066804	-0.41\\
0.4	-0.412165114037906\\
0.39	-0.417320591069708\\
0.384648745749663	-0.42\\
0.38	-0.422311323763955\\
0.37	-0.427135885167464\\
0.363876423651367	-0.43\\
0.36	-0.43180044345898\\
0.35	-0.436298701298701\\
0.341495014193173	-0.44\\
0.34	-0.440646115130544\\
0.33	-0.444822585718779\\
0.32	-0.448853727587292\\
0.317050109285194	-0.45\\
0.31	-0.452720712277413\\
0.3	-0.456435488909716\\
0.290016624376586	-0.46\\
0.29	-0.460005895128762\\
0.28	-0.463408625504189\\
0.27	-0.466668011169718\\
0.26	-0.469784052125349\\
0.259273562258636	-0.47\\
0.25	-0.472738058551618\\
0.24	-0.475548228043143\\
0.23	-0.478216024653313\\
0.222935936546676	-0.48\\
0.22	-0.480736455463728\\
0.21	-0.483103458830732\\
0.2	-0.485329048056321\\
0.19	-0.487413223140496\\
0.18	-0.489355984083257\\
0.176424808836022	-0.49\\
0.17	-0.491149589540894\\
0.16	-0.492798418972332\\
0.15	-0.494306780176345\\
0.14	-0.495674673152934\\
0.13	-0.496902097902098\\
0.12	-0.497989054423837\\
0.11	-0.498935542718151\\
0.1	-0.499741562785041\\
0.0961169483782555	-0.5\\
0.0899999999999999	-0.500404409543944\\
0.0800000000000001	-0.50092600422833\\
0.0699999999999998	-0.501308064028994\\
0.0600000000000001	-0.501550588945938\\
0.0499999999999998	-0.501653578979161\\
0.04	-0.501617034128662\\
0.0299999999999998	-0.501440954394443\\
0.02	-0.501125339776503\\
0.00999999999999979	-0.500670190274841\\
0	-0.500075505889459\\
-0.00102838338132531	-0.5\\
-0.01	-0.499336880510794\\
-0.02	-0.498457281848586\\
-0.03	-0.497437214958954\\
-0.04	-0.496276679841897\\
-0.05	-0.494975676497416\\
-0.0600000000000001	-0.493534204925509\\
-0.0700000000000001	-0.491952265126178\\
-0.0800000000000001	-0.490229857099422\\
-0.0812338828137741	-0.49\\
-0.0900000000000001	-0.488355984083257\\
-0.1	-0.486339149066422\\
-0.11	-0.484180899908173\\
-0.12	-0.481881236608509\\
-0.127706583072101	-0.48\\
-0.13	-0.479436363636364\\
-0.14	-0.476836363636364\\
-0.15	-0.474093990755008\\
-0.16	-0.471209244992296\\
-0.163994706301537	-0.47\\
-0.17	-0.468169717654359\\
-0.18	-0.464978591374496\\
-0.19	-0.461644120384735\\
-0.194727451155322	-0.46\\
-0.2	-0.458153701968135\\
-0.21	-0.454507653858169\\
-0.22	-0.450717275851296\\
-0.221822945613338	-0.45\\
-0.23	-0.446760301981755\\
-0.24	-0.442653035545769\\
-0.246238627117391	-0.44\\
-0.25	-0.438389293633196\\
-0.26	-0.433960722204625\\
-0.268657481132729	-0.43\\
-0.27	-0.429381499202552\\
-0.28	-0.42462711323764\\
-0.289439708466194	-0.42\\
-0.29	-0.419723417924831\\
-0.3	-0.414638612271121\\
-0.308863789822601	-0.41\\
-0.31	-0.409401164671627\\
-0.32	-0.403981235846005\\
-0.32714841707813	-0.4\\
-0.33	-0.39840045617465\\
-0.34	-0.392640599543825\\
-0.344467721484095	-0.39\\
-0.35	-0.386706596652445\\
-0.36	-0.380601903511651\\
-0.360962073125951	-0.38\\
-0.37	-0.374304462809917\\
-0.376668886043534	-0.37\\
-0.38	-0.367834165834166\\
-0.39	-0.361178488178488\\
-0.391730646975402	-0.36\\
-0.4	-0.354327406910433\\
-0.40616900196069	-0.35\\
-0.41	-0.347292666441365\\
-0.42	-0.340069618114228\\
-0.420094343943211	-0.34\\
-0.43	-0.332635682669391\\
-0.433471767502355	-0.33\\
-0.44	-0.325006518010292\\
-0.446412408278044	-0.32\\
-0.45	-0.317177670238507\\
-0.458942336678007	-0.31\\
-0.46	-0.309144548937653\\
-0.47	-0.300895506792059\\
-0.471064816732243	-0.3\\
-0.48	-0.292427518427518\\
-0.482810582354614	-0.29\\
-0.49	-0.283742129465865\\
-0.494219952929914	-0.28\\
-0.5	-0.274834224598931\\
-0.505311190317653	-0.27\\
-0.51	-0.265698526769673\\
-0.516101258031009	-0.26\\
-0.52	-0.256329590727997\\
-0.526605934605935	-0.25\\
-0.53	-0.246721796276013\\
-0.536839915295167	-0.24\\
-0.54	-0.236869341185131\\
-0.54681690346616	-0.23\\
-0.55	-0.226766233766234\\
-0.556549692898962	-0.22\\
-0.56	-0.216406285072952\\
-0.566050242023814	-0.21\\
-0.57	-0.205783100716711\\
-0.575329741004693	-0.2\\
-0.58	-0.194890072270825\\
-0.584398672460396	-0.19\\
-0.59	-0.183720368239355\\
-0.593266866516218	-0.18\\
-0.6	-0.172266924564797\\
-0.601943550794335	-0.17\\
-0.61	-0.160522434646898\\
-0.610437395877568	-0.16\\
-0.618737791134485	-0.15\\
-0.62	-0.148466057959507\\
-0.626862751407884	-0.14\\
-0.63	-0.13609571485783\\
-0.634826394393887	-0.13\\
-0.64	-0.123407676767677\\
-0.642635707365855	-0.12\\
-0.65	-0.110393395841827\\
-0.650297279812698	-0.11\\
-0.657786266596839	-0.1\\
-0.66	-0.0970170195101704\\
-0.665134404519637	-0.0899999999999999\\
-0.67	-0.0832890657729368\\
-0.672351654934851	-0.0800000000000001\\
-0.679435674704209	-0.0699999999999998\\
-0.68	-0.0691959026888606\\
-0.686365704493751	-0.0600000000000001\\
-0.69	-0.0547001292546318\\
-0.693179632166487	-0.0499999999999998\\
-0.699880497828558	-0.04\\
-0.7	-0.0398199385155906\\
-0.706430646228179	-0.0299999999999998\\
-0.71	-0.0244962305986696\\
-0.712877699883632	-0.02\\
-0.719215508188346	-0.00999999999999979\\
-0.72	-0.00874988692899145\\
-0.725419806695616	0\\
-0.73	0.00746870717222473\\
-0.73153248333656	0.00999999999999979\\
-0.737525790303399	0.02\\
-0.74	0.0241706293706293\\
-0.743414620825255	0.0299999999999998\\
-0.749212157623092	0.04\\
-0.75	0.0413731556401713\\
-0.754887948005717	0.0499999999999998\\
-0.76	0.059117845117845\\
-0.760488506512532	0.0600000000000001\\
-0.765967557201076	0.0699999999999998\\
-0.77	0.0774393120393118\\
-0.771371135376924	0.0800000000000001\\
-0.776667806867517	0.0899999999999999\\
-0.78	0.0963606228026117\\
-0.781883692515663	0.1\\
-0.787002365664076	0.11\\
-0.79	0.115922444786852\\
-0.792039353695189	0.12\\
-0.796984253384366	0.13\\
-0.8	0.136169206516028\\
-0.801850676550481	0.14\\
-0.806625879007895	0.15\\
-0.81	0.157149542764928\\
-0.811329636413821	0.16\\
-0.815939076104685	0.17\\
-0.82	0.178916804407714\\
-0.82048765968002	0.18\\
-0.824935135805531	0.19\\
-0.829328041671834	0.2\\
-0.83	0.201548885077187\\
-0.833624837530961	0.21\\
-0.837860068175687	0.22\\
-0.84	0.22511788856305\\
-0.842018477654647	0.23\\
-0.846099565944858	0.24\\
-0.85	0.249683925346177\\
-0.850125896261481	0.25\\
-0.854056257643702	0.26\\
-0.857933862497302	0.27\\
-0.86	0.275401880877743\\
-0.861739439793184	0.28\\
-0.865470198989636	0.29\\
-0.869148779735787	0.3\\
-0.87	0.302347285807718\\
-0.872745102639296	0.31\\
-0.876280586510264	0.32\\
-0.879764457478006	0.33\\
-0.88	0.330686261107313\\
-0.883162438786641	0.34\\
-0.886506835008239	0.35\\
-0.889800171745538	0.36\\
-0.89	0.360616320687186\\
-0.893010172449425	0.37\\
-0.896167535419872	0.38\\
-0.899274380582792	0.39\\
-0.9	0.392374154770849\\
-0.902306241621413	0.4\\
-0.905280498057304	0.41\\
-0.908204766990071	0.42\\
-0.91	0.426245849802371\\
-0.911067838913498	0.43\\
-0.913862793155398	0.44\\
-0.916608279180626	0.45\\
-0.919304296989184	0.46\\
-0.92	0.462628717077315\\
-0.921930788917325	0.47\\
-0.92450116835429	0.48\\
-0.927022588182931	0.49\\
-0.929495048403249	0.5\\
-0.93	0.502083562901744\\
-0.931899024164592	0.51\\
-0.934249399740071	0.52\\
-0.936551313963478	0.53\\
-0.938804766834813	0.54\\
-0.94	0.545420579420579\\
-0.940999585686561	0.55\\
-0.943134390196036	0.56\\
-0.945221221570466	0.57\\
-0.947260079809852	0.58\\
-0.949250964914193	0.59\\
-0.95	0.593855218855219\\
-0.951181969301181	0.6\\
-0.953058008246799	0.61\\
-0.954886552535567	0.62\\
-0.956667602167483	0.63\\
-0.958401157142549	0.64\\
-0.96	0.64948271446863\\
-0.960086356155014	0.65\\
-0.961708740354402	0.66\\
-0.963284098924823	0.67\\
-0.964812431866276	0.68\\
-0.966293739178761	0.69\\
-0.967728020862279	0.7\\
-0.969115276916829	0.71\\
-0.97	0.716601275917065\\
-0.970451053021484	0.72\\
-0.971731611810774	0.73\\
-0.972965604825908	0.74\\
-0.974153032066885	0.75\\
-0.975293893533707	0.76\\
-0.976388189226373	0.77\\
-0.977435919144883	0.78\\
-0.978437083289237	0.79\\
-0.979391681659435	0.8\\
-0.98	0.8066993006993\\
-0.980296811788626	0.81\\
-0.981149936068082	0.82\\
-0.981956945521622	0.83\\
-0.982717840149244	0.84\\
-0.983432619950951	0.85\\
-0.98410128492674	0.86\\
-0.984723835076614	0.87\\
-0.98530027040057	0.88\\
-0.98583059089861	0.89\\
-0.986314796570734	0.9\\
-0.986752887416941	0.91\\
-0.987144863437231	0.92\\
-0.987490724631605	0.93\\
-0.987790471000063	0.94\\
-0.988044102542604	0.95\\
-0.988251619259228	0.96\\
-0.988413021149936	0.97\\
-0.988528308214727	0.98\\
-0.988597480453602	0.99\\
-0.98862053786656	1\\
46.0520454545455	719\\
1.51308202911805	1\\
1.51306580325402	0.99\\
1.51301712566194	0.98\\
1.51293599634181	0.97\\
1.51282241529361	0.96\\
1.51267638251737	0.95\\
1.51249789801307	0.94\\
1.51228696178071	0.93\\
1.51204357382031	0.92\\
1.51176773413184	0.91\\
1.51145944271532	0.9\\
1.51111869957075	0.89\\
1.51074550469813	0.88\\
1.51033985809744	0.87\\
1.51	0.862242424242422\\
1.50990108568119	0.86\\
1.50942730688687	0.85\\
1.50892085369295	0.84\\
1.50838172609942	0.83\\
1.50780992410628	0.82\\
1.50720544771353	0.81\\
1.50656829692118	0.8\\
1.50589847172922	0.79\\
1.50519597213765	0.78\\
1.50446079814647	0.77\\
1.50369294975568	0.76\\
1.50289242696529	0.75\\
1.50205922977529	0.74\\
1.50119335818568	0.73\\
1.50029481219646	0.72\\
1.5	0.716834130781498\\
1.49935919484365	0.71\\
1.49838864047615	0.7\\
1.49738518596061	0.69\\
1.49634883129701	0.68\\
1.49527957648537	0.67\\
1.49417742152567	0.66\\
1.49304236641792	0.65\\
1.49187441116212	0.64\\
1.49067355575827	0.63\\
1.49	0.624540606060608\\
1.48943590284157	0.62\\
1.48816043489391	0.61\\
1.48685183790865	0.6\\
1.4855101118858	0.59\\
1.48413525682533	0.58\\
1.48272727272727	0.57\\
1.48128615959161	0.56\\
1.48	0.551275791624105\\
1.47981059974221	0.55\\
1.47829266813253	0.54\\
1.47674137538858	0.53\\
1.47515672151035	0.52\\
1.47353870649784	0.51\\
1.47188733035105	0.5\\
1.47020259306998	0.49\\
1.47	0.488820829655782\\
1.46847380236092	0.48\\
1.46670998579784	0.47\\
1.4649125727288	0.46\\
1.4630815631538	0.45\\
1.46121695707283	0.44\\
1.46	0.433588897827836\\
1.45931391396626	0.43\\
1.45736838867443	0.42\\
1.45538902816013	0.41\\
1.45337583242337	0.4\\
1.45132880146414	0.39\\
1.45	0.383614190687362\\
1.44924255332332	0.38\\
1.44711271859849	0.37\\
1.44494880651807	0.36\\
1.44275081708205	0.35\\
1.44051875029043	0.34\\
1.44	0.337710868079289\\
1.43824001123297	0.33\\
1.43592321013464	0.32\\
1.43357208605707	0.31\\
1.43118663900026	0.3\\
1.43	0.295096067053515\\
1.42875791624106	0.29\\
1.4262860061287	0.28\\
1.42377952384694	0.27\\
1.42123846939577	0.26\\
1.42	0.255191580231849\\
1.41865306380891	0.25\\
1.41602377598024	0.24\\
1.41335966314723	0.23\\
1.41066072530986	0.22\\
1.41	0.217583092067169\\
1.40791169013411	0.21\\
1.4051227057454	0.2\\
1.40229863979206	0.19\\
1.4	0.181960401561628\\
1.39943533229449	0.18\\
1.39651962280519	0.17\\
1.39356857138267	0.16\\
1.39058217802696	0.15\\
1.39	0.14807336523126\\
1.38754220143396	0.14\\
1.38446226552511	0.13\\
1.38134672342078	0.12\\
1.38	0.115726245505906\\
1.37818198124745	0.11\\
1.37497121891561	0.1\\
1.37172458214431	0.0899999999999999\\
1.37	0.0847461500248386\\
1.36843024497642	0.0800000000000001\\
1.36508667006227	0.0699999999999998\\
1.36170694839229	0.0600000000000001\\
1.36	0.0550028860028856\\
1.35827800864224	0.0499999999999998\\
1.35479958941077	0.04\\
1.35128474694128	0.0299999999999998\\
1.35	0.0263822843822846\\
1.34771601127813	0.02\\
1.3441006690135	0.00999999999999979\\
1.34044862276648	0\\
1.34	-0.00121619176843108\\
1.33673470417136	-0.00999999999999979\\
1.33298031171713	-0.02\\
1.33	-0.027860753880266\\
1.32918258069982	-0.0299999999999998\\
1.3253242396001	-0.04\\
1.32142861984241	-0.0499999999999998\\
1.32	-0.0536324859974155\\
1.317475961949	-0.0600000000000001\\
1.31347445903711	-0.0699999999999998\\
1.31	-0.0786021141649048\\
1.3094308929094	-0.0800000000000001\\
1.30532182265755	-0.0899999999999999\\
1.30117488079048	-0.1\\
1.3	-0.102807486631017\\
1.29696593728852	-0.11\\
1.29270939978136	-0.12\\
1.29	-0.126308686868687\\
1.28840187520772	-0.13\\
1.2840339706474	-0.14\\
1.28	-0.149154823342597\\
1.27962454809981	-0.15\\
1.27514346177586	-0.16\\
1.27062357816771	-0.17\\
1.27	-0.171367891682785\\
1.26603257294241	-0.18\\
1.26139732944544	-0.19\\
1.26	-0.192989349562572\\
1.25669582832864	-0.2\\
1.25194331403166	-0.21\\
1.25	-0.214055368499813\\
1.24712756918961	-0.22\\
1.24225582530415	-0.23\\
1.24	-0.234592933382407\\
1.2373219461509	-0.24\\
1.23232896439834	-0.25\\
1.23	-0.254627308946034\\
1.22727291111315	-0.26\\
1.22215663204339	-0.27\\
1.22	-0.274182174688057\\
1.21697420874047	-0.28\\
1.21173251997849	-0.29\\
1.21	-0.293279747279747\\
1.20641936750762	-0.3\\
1.20105010192441	-0.31\\
1.2	-0.311940891807812\\
1.19560169027901	-0.32\\
1.1901026240827	-0.33\\
1.19	-0.330185223016684\\
1.18451424439074	-0.34\\
1.18	-0.34801655964853\\
1.17887318805798	-0.35\\
1.17314985120476	-0.36\\
1.17	-0.365463203463204\\
1.16736086628053	-0.37\\
1.16150107510218	-0.38\\
1.16	-0.382543157203807\\
1.15555892183499	-0.39\\
1.15	-0.399266862170088\\
1.14955620426438	-0.4\\
1.14345943706976	-0.41\\
1.14	-0.415634115001606\\
1.13729462348468	-0.42\\
1.13105420307741	-0.43\\
1.13	-0.431677541970225\\
1.12472144650929	-0.44\\
1.12	-0.44739257628185\\
1.11831910449577	-0.45\\
1.11182791555979	-0.46\\
1.11	-0.462796773192678\\
1.10524758960922	-0.47\\
1.1	-0.477899845916795\\
1.09859162206287	-0.48\\
1.09184008762322	-0.49\\
1.09	-0.492707205837641\\
1.08499530536024	-0.5\\
1.08	-0.507230745998188\\
1.07806826082377	-0.51\\
1.07104624460129	-0.52\\
1.07	-0.521480178837556\\
1.06391868604132	-0.53\\
1.06	-0.535454249333728\\
1.05670161792828	-0.54\\
1.05	-0.549176228302442\\
1.04939234792422	-0.55\\
1.04196814166974	-0.56\\
1.04	-0.562634040081324\\
1.0344404498323	-0.57\\
1.03	-0.575845887445888\\
1.02681231314362	-0.58\\
1.02	-0.588821623171781\\
1.01908061665138	-0.59\\
1.01122927527801	-0.6\\
1.01	-0.60155593316341\\
1.00325902647464	-0.61\\
1	-0.614057134815649\\
0.995175809362791	-0.62\\
0.99	-0.626336783216783\\
0.986976008497472	-0.63\\
0.98	-0.638398943564081\\
0.978655852491539	-0.64\\
0.970209125919011	-0.65\\
0.97	-0.650246086240044\\
0.961620696977562	-0.66\\
0.96	-0.661875238875239\\
0.95290069416283	-0.67\\
0.95	-0.673299864314789\\
0.944044718878892	-0.68\\
0.94	-0.684523603992447\\
0.935048174256653	-0.69\\
0.93	-0.69555001340842\\
0.925906253854693	-0.7\\
0.92	-0.706382564649427\\
0.916613929578519	-0.71\\
0.91	-0.717024648820567\\
0.907165938754386	-0.72\\
0.9	-0.727479578392622\\
0.897556770287875	-0.73\\
0.89	-0.737750589468169\\
0.887780649830692	-0.74\\
0.88	-0.747840843969784\\
0.877831523871812	-0.75\\
0.87	-0.757753431753432\\
0.867703042660806	-0.76\\
0.86	-0.767491372650013\\
0.857388541862148	-0.77\\
0.85	-0.7770576184379\\
0.846881022829067	-0.78\\
0.84	-0.786455054749173\\
0.83617313137422	-0.79\\
0.83	-0.79568650291213\\
0.825257134901852	-0.8\\
0.82	-0.804754721732561\\
0.814124897751953	-0.81\\
0.81	-0.813662409216128\\
0.802767854591181	-0.82\\
0.8	-0.822412204234122\\
0.791176981667584	-0.83\\
0.79	-0.831006688134753\\
0.78	-0.839445380232846\\
0.779333730099688	-0.84\\
0.77	-0.847727519093373\\
0.767216908547808	-0.85\\
0.76	-0.855861063464837\\
0.754831553502035	-0.86\\
0.75	-0.863848403607117\\
0.74216611316304	-0.87\\
0.74	-0.871691878787879\\
0.73	-0.879390545454545\\
0.729196676785429	-0.88\\
0.72	-0.886940197733301\\
0.715884604157883	-0.89\\
0.71	-0.894352122811226\\
0.702247209509071	-0.9\\
0.7	-0.901628489620615\\
0.69	-0.908764972560248\\
0.688242265765613	-0.91\\
0.68	-0.915760977925469\\
0.673838537174401	-0.92\\
0.67	-0.922627154663518\\
0.66	-0.929362219598583\\
0.659037454117815	-0.93\\
0.65	-0.93595724688748\\
0.643764266821262	-0.94\\
0.64	-0.942427903715822\\
0.63	-0.948769806029446\\
0.628026616682287	-0.95\\
0.62	-0.954978144617531\\
0.611760195322931	-0.96\\
0.61	-0.961067314365024\\
0.6	-0.967024057367569\\
0.594912807940211	-0.97\\
0.59	-0.972859378596088\\
0.58	-0.97857330264672\\
0.577455776728914	-0.98\\
0.57	-0.98415983512709\\
0.56	-0.989633386764369\\
0.55931700866004	-0.99\\
0.55	-0.994976076555024\\
0.540404282170677	-1\\
0.54	-1.00021060983904\\
0.53	-1.00531534799365\\
0.520630695987305	-1.01\\
0.52	-1.0103137829912\\
0.51	-1.01518475073314\\
0.5	-1.01995150011279\\
0.499895979486187	-1.02\\
0.49	-1.02459124579125\\
0.48	-1.02912704826038\\
0.478030387439858	-1.03\\
0.47	-1.03354165735984\\
0.46	-1.03784878266696\\
0.454882843632113	-1.04\\
0.45	-1.04204267615026\\
0.44	-1.04612336074683\\
0.430254791305805	-1.05\\
0.43	-1.05010086264101\\
0.42	-1.05395731032957\\
0.41	-1.05771156823712\\
0.403733874386773	-1.06\\
0.4	-1.06135703279771\\
0.39	-1.06488972044904\\
0.38	-1.06832071318512\\
0.374956033057851	-1.07\\
0.37	-1.07164205914567\\
0.36	-1.07485410733844\\
0.35	-1.07796495071194\\
0.343238226945192	-1.08\\
0.34	-1.08096991497711\\
0.33	-1.08386439938958\\
0.32	-1.08665816437759\\
0.31	-1.08935120994114\\
0.307497266840467	-1.09\\
0.3	-1.09193425905836\\
0.29	-1.09441397266218\\
0.28	-1.09679344760252\\
0.27	-1.09907268387937\\
0.265744299512098	-1.1\\
0.26	-1.10124573526236\\
0.25	-1.10331461887281\\
0.24	-1.10528374001296\\
0.23	-1.10715309868279\\
0.22	-1.10892269488231\\
0.213548428811586	-1.11\\
0.21	-1.11058972705781\\
0.2	-1.11215237481195\\
0.19	-1.11361573178594\\
0.18	-1.1149797979798\\
0.17	-1.11624457339351\\
0.16	-1.11741005802708\\
0.15	-1.11847625188051\\
0.14	-1.11944315495379\\
0.133581867723556	-1.12\\
0.13	-1.12030930481283\\
0.12	-1.12107401069519\\
0.11	-1.12173989304813\\
0.1	-1.12230695187166\\
0.0899999999999999	-1.12277518716578\\
0.0800000000000001	-1.12314459893048\\
0.0699999999999998	-1.12341518716578\\
0.0600000000000001	-1.12358695187166\\
0.0499999999999998	-1.12365989304813\\
0.04	-1.12363401069519\\
0.0299999999999998	-1.12350930481283\\
0.02	-1.12328577540107\\
0.00999999999999979	-1.12296342245989\\
0	-1.1225422459893\\
-0.01	-1.1220222459893\\
-0.02	-1.12140342245989\\
-0.03	-1.12068577540107\\
-0.0383992664396108	-1.12\\
-0.04	-1.11986868686869\\
-0.05	-1.11894906511928\\
-0.0600000000000001	-1.11793015258973\\
-0.0700000000000001	-1.11681194928003\\
-0.0800000000000001	-1.1155944551902\\
-0.0900000000000001	-1.11427767032022\\
-0.1	-1.11286159467011\\
-0.11	-1.11134622823985\\
-0.118337548249701	-1.11\\
-0.12	-1.10973029583243\\
-0.13	-1.10800820557115\\
-0.14	-1.10618635283956\\
-0.15	-1.10426473763766\\
-0.16	-1.10224335996545\\
-0.17	-1.10012221982293\\
-0.170550315994166	-1.1\\
-0.18	-1.09789129963116\\
-0.19	-1.09555955738772\\
-0.2	-1.0931275764808\\
-0.21	-1.09059535691039\\
-0.212261600593423	-1.09\\
-0.22	-1.08795312840637\\
-0.23	-1.08520732504905\\
-0.24	-1.08236080226728\\
-0.248010207855611	-1.08\\
-0.25	-1.07941073384447\\
-0.26	-1.07634808324206\\
-0.27	-1.07318422782037\\
-0.279752432069776	-1.07\\
-0.28	-1.06991877613912\\
-0.29	-1.06653620955316\\
-0.3	-1.06305194805195\\
-0.308510834202933	-1.06\\
-0.31	-1.05946339305463\\
-0.32	-1.05575779694758\\
-0.33	-1.0519500110595\\
-0.334987271595859	-1.05\\
-0.34	-1.04803045121138\\
-0.35	-1.0439986663703\\
-0.359671523036396	-1.04\\
-0.36	-1.03986352468171\\
-0.37	-1.03560553942372\\
-0.38	-1.03124436006254\\
-0.382787311752639	-1.03\\
-0.39	-1.02676408529742\\
-0.4	-1.02217396184063\\
-0.404631533642581	-1.02\\
-0.41	-1.01746762914505\\
-0.42	-1.01264628919468\\
-0.42537256697962	-1.01\\
-0.43	-1.00770936295625\\
-0.44	-1.00265450011335\\
-0.445144777890066	-1\\
-0.45	-0.997482342219185\\
-0.46	-0.992191615402142\\
-0.464061563146561	-0.99\\
-0.47	-0.986779482482253\\
-0.48	-0.981250515227845\\
-0.482219287194701	-0.98\\
-0.49	-0.975593555811277\\
-0.49969484223543	-0.97\\
-0.5	-0.969823039555864\\
-0.51	-0.963917187138561\\
-0.516514830916016	-0.96\\
-0.52	-0.957893745640549\\
-0.53	-0.951742850499884\\
-0.532784857153472	-0.95\\
-0.54	-0.945461322738958\\
-0.548535373826656	-0.94\\
-0.55	-0.939058022081278\\
-0.56	-0.932517970401691\\
-0.563787231035579	-0.93\\
-0.57	-0.925847933884298\\
-0.57860976881627	-0.92\\
-0.58	-0.919050795157845\\
-0.59	-0.912113458343223\\
-0.592999090572266	-0.91\\
-0.6	-0.905040563111429\\
-0.607006401114391	-0.9\\
-0.61	-0.897834972415447\\
-0.62	-0.89049196449988\\
-0.620660016090104	-0.89\\
-0.63	-0.883001205690861\\
-0.63394647556838	-0.88\\
-0.64	-0.875371878787879\\
-0.646924903903247	-0.87\\
-0.65	-0.867601754813551\\
-0.659608453220788	-0.86\\
-0.66	-0.859688556726293\\
-0.67	-0.851621171281549\\
-0.671981728321102	-0.85\\
-0.68	-0.843405025868441\\
-0.684083375188348	-0.84\\
-0.69	-0.835039385682438\\
-0.695929639453205	-0.83\\
-0.7	-0.826521793275218\\
-0.707530700871416	-0.82\\
-0.71	-0.81784973703982\\
-0.718896205262112	-0.81\\
-0.72	-0.8090206497104\\
-0.73	-0.80003173004281\\
-0.730034848023896	-0.8\\
-0.74	-0.790875917953912\\
-0.740944591605451	-0.79\\
-0.75	-0.781555895085307\\
-0.751647744127721	-0.78\\
-0.76	-0.772068886043534\\
-0.76215193245079	-0.77\\
-0.77	-0.762412052536699\\
-0.772464412577293	-0.76\\
-0.78	-0.752582491582492\\
-0.782592091922947	-0.75\\
-0.79	-0.742577233654598\\
-0.792541550001284	-0.74\\
-0.8	-0.732393240764999\\
-0.802319057652763	-0.73\\
-0.81	-0.722027404479578\\
-0.811930594936391	-0.72\\
-0.82	-0.711476543864299\\
-0.821381867791145	-0.71\\
-0.83	-0.700737403359104\\
-0.830678323564754	-0.7\\
-0.839823184638399	-0.69\\
-0.84	-0.689805503102239\\
-0.848814229249012	-0.68\\
-0.85	-0.678672998643148\\
-0.85766408479413	-0.67\\
-0.86	-0.667340704340704\\
-0.866377392500533	-0.66\\
-0.87	-0.655804998626751\\
-0.874958592375367	-0.65\\
-0.88	-0.644062171870683\\
-0.883411933994012	-0.64\\
-0.89	-0.632108423686406\\
-0.8917414866013	-0.63\\
-0.899950630324462	-0.62\\
-0.9	-0.619939487756825\\
-0.908024130331054	-0.61\\
-0.91	-0.607537241574625\\
-0.915984305083983	-0.6\\
-0.92	-0.594910515816472\\
-0.923834650050072	-0.59\\
-0.93	-0.582055061657585\\
-0.9315785184043	-0.58\\
-0.939211181355596	-0.57\\
-0.94	-0.568959918675574\\
-0.946726923831745	-0.56\\
-0.95	-0.555612394036831\\
-0.954144988234278	-0.55\\
-0.96	-0.542020888496617\\
-0.96146826838809	-0.54\\
-0.968686702434645	-0.53\\
-0.97	-0.52816870342772\\
-0.975800825484178	-0.52\\
-0.98	-0.514047704770477\\
-0.982827886892909	-0.51\\
-0.989768168193347	-0.5\\
-0.99	-0.499663727576771\\
-0.996598434677905	-0.49\\
-1	-0.484984695439241\\
-1.00334868705147	-0.48\\
-1.01	-0.470032049306626\\
-1.01002118428289	-0.47\\
-1.01658627502903	-0.46\\
-1.02	-0.45476444860981\\
-1.02307759055595	-0.45\\
-1.02949268489557	-0.44\\
-1.03	-0.439203674374406\\
-1.03580922486355	-0.43\\
-1.04	-0.423313875598086\\
-1.0420580835595	-0.42\\
-1.0482250044573	-0.41\\
-1.05	-0.407101261727596\\
-1.05430890638189	-0.4\\
-1.06	-0.390553600521342\\
-1.06033052545571	-0.39\\
-1.06625819504698	-0.38\\
-1.07	-0.373641652892562\\
-1.07212397570616	-0.37\\
-1.07791400751952	-0.36\\
-1.08	-0.356370680979537\\
-1.08362929271695	-0.35\\
-1.08928411718616	-0.34\\
-1.09	-0.338724548859381\\
-1.09485404156169	-0.33\\
-1.1	-0.320680960548885\\
-1.10037276295281	-0.32\\
-1.10580552477888	-0.31\\
-1.11	-0.302220132358064\\
-1.11118667734068	-0.3\\
-1.11649079366261	-0.29\\
-1.12	-0.283332507958967\\
-1.12173899400092	-0.28\\
-1.12691665897554	-0.27\\
-1.13	-0.26399784405318\\
-1.13203634899427	-0.26\\
-1.13708970111829	-0.25\\
-1.14	-0.24419459656809\\
-1.14208515581045	-0.24\\
-1.14701627978729	-0.23\\
-1.15	-0.223899814471243\\
-1.15189161462177	-0.22\\
-1.15670254315077	-0.21\\
-1.16	-0.203089023010185\\
-1.16146172107884	-0.2\\
-1.16615443657069	-0.19\\
-1.17	-0.1817360951285\\
-1.17080127467469	-0.18\\
-1.1753777108967	-0.17\\
-1.17991519872533	-0.16\\
-1.18	-0.159811491538764\\
-1.1843779303563	-0.15\\
-1.18880134159833	-0.14\\
-1.19	-0.137266319583501\\
-1.1931604800641	-0.13\\
-1.19747165078646	-0.12\\
-1.2	-0.114082755809213\\
-1.20173057317136	-0.11\\
-1.20593129546751	-0.1\\
-1.21	-0.0902258198422582\\
-1.21009325767576	-0.0899999999999999\\
-1.21418528105875	-0.0800000000000001\\
-1.21823958994051	-0.0699999999999998\\
-1.22	-0.0656171574903971\\
-1.22223845565099	-0.0600000000000001\\
-1.22618607024407	-0.0499999999999998\\
-1.23	-0.0402461939973903\\
-1.23009551614155	-0.04\\
-1.23393810014007	-0.0299999999999998\\
-1.23774355772314	-0.02\\
-1.24	-0.014012091356919\\
-1.24150019256853	-0.00999999999999979\\
-1.24520253185753	0\\
-1.24886803194963	0.00999999999999979\\
-1.25	0.0131195200738346\\
-1.25247752612951	0.02\\
-1.25604177398182	0.0299999999999998\\
-1.2595694654459	0.04\\
-1.26	0.0412332222751071\\
-1.26303728377554	0.0499999999999998\\
-1.26646554425078	0.0600000000000001\\
-1.26985752683739	0.0699999999999998\\
-1.27	0.0704245700245697\\
-1.27318893707553	0.0800000000000001\\
-1.27648326650847	0.0899999999999999\\
-1.27974159234105	0.1\\
-1.28	0.100801929913662\\
-1.28294167329901	0.11\\
-1.2861040817321	0.12\\
-1.28923075673656	0.13\\
-1.29	0.132488702049396\\
-1.29230440600364	0.14\\
-1.29533685856615	0.15\\
-1.29833384384723	0.16\\
-1.3	0.165626020685901\\
-1.30128578515306	0.17\\
-1.30419020338929	0.18\\
-1.30705941655598	0.19\\
-1.30989342465315	0.2\\
-1.31	0.200380789022299\\
-1.31267246993789	0.21\\
-1.31541578638032	0.22\\
-1.31812415611647	0.23\\
-1.32	0.237016636957813\\
-1.32079176850903	0.24\\
-1.32341102262872	0.25\\
-1.32599558464086	0.26\\
-1.32854545454545	0.27\\
-1.33	0.275783072100313\\
-1.33105296115973	0.28\\
-1.33351550636379	0.29\\
-1.33594361037619	0.3\\
-1.33833727319693	0.31\\
-1.34	0.317047777040477\\
-1.34069149349539	0.32\\
-1.34299958034785	0.33\\
-1.345273473321	0.34\\
-1.34751317241487	0.35\\
-1.34971867762943	0.36\\
-1.35	0.361295633500358\\
-1.35187651425132	0.37\\
-1.3539983950865	0.38\\
-1.35608632582831	0.39\\
-1.35814030647675	0.4\\
-1.36	0.409206264323911\\
-1.36015920199807	0.41\\
-1.36213122289812	0.42\\
-1.3640695340392	0.43\\
-1.3659741354213	0.44\\
-1.36784502704442	0.45\\
-1.36968220890857	0.46\\
-1.37	0.461762107051827\\
-1.37147523773298	0.47\\
-1.37323255990871	0.48\\
-1.3749564092811	0.49\\
-1.37664678585013	0.5\\
-1.37830368961582	0.51\\
-1.37992712057817	0.52\\
-1.38	0.52045837320574\\
-1.38150648918972	0.53\\
-1.38305210993095	0.54\\
-1.38456449151646	0.55\\
-1.38604363394624	0.56\\
-1.3874895372203	0.57\\
-1.38890220133863	0.58\\
-1.39	0.587958378970428\\
-1.39027967411364	0.59\\
-1.39161652838002	0.6\\
-1.39292037389908	0.61\\
-1.39419121067083	0.62\\
-1.39542903869525	0.63\\
-1.39663385797236	0.64\\
-1.39780566850215	0.65\\
-1.39894447028463	0.66\\
-1.4	0.669545454545454\\
-1.40004991730119	0.67\\
-1.40111531641609	0.68\\
-1.40214793401976	0.69\\
-1.4031477701122	0.7\\
-1.40411482469342	0.71\\
-1.40504909776341	0.72\\
-1.40595058932217	0.73\\
-1.4068192993697	0.74\\
-1.40765522790601	0.75\\
-1.40845837493108	0.76\\
-1.40922874044493	0.77\\
-1.40996632444756	0.78\\
-1.41	0.780477801268501\\
-1.41066653841031	0.79\\
-1.41133396474923	0.8\\
-1.41196883370577	0.81\\
-1.41257114527992	0.82\\
-1.41314089947168	0.83\\
-1.41367809628106	0.84\\
-1.41418273570805	0.85\\
-1.41465481775265	0.86\\
-1.41509434241487	0.87\\
-1.4155013096947	0.88\\
-1.41587571959214	0.89\\
-1.4162175721072	0.9\\
-1.41652686723987	0.91\\
-1.41680360499016	0.92\\
-1.41704778535806	0.93\\
-1.41725940834357	0.94\\
-1.41743847394669	0.95\\
-1.41758498216743	0.96\\
-1.41769893300579	0.97\\
-1.41778032646175	0.98\\
-1.41782916253533	0.99\\
-1.41784544122653	1\\
68.6018181818182	881\\
1.84244029913655	1\\
1.8424270523489	0.99\\
1.84238731198593	0.98\\
1.84232107804766	0.97\\
1.84222835053409	0.96\\
1.8421091294452	0.95\\
1.84196341478101	0.94\\
1.8417912065415	0.93\\
1.84159250472669	0.92\\
1.84136730933658	0.91\\
1.84111562037115	0.9\\
1.84083743783042	0.89\\
1.84053276171438	0.88\\
1.84020159202303	0.87\\
1.84	0.86436363636363\\
1.83984305557238	0.86\\
1.83945674945808	0.85\\
1.8390438015428	0.84\\
1.83860421182654	0.83\\
1.83813798030929	0.82\\
1.83764510699105	0.81\\
1.83712559187183	0.8\\
1.83657943495162	0.79\\
1.83600663623043	0.78\\
1.83540719570825	0.77\\
1.83478111338508	0.76\\
1.83412838926093	0.75\\
1.8334490233358	0.74\\
1.83274301560967	0.73\\
1.83201036608257	0.72\\
1.83125107475447	0.71\\
1.83046514162539	0.7\\
1.83	0.694275707898659\\
1.82965061194666	0.69\\
1.82880667356756	0.68\\
1.82793594349388	0.67\\
1.82703842172563	0.66\\
1.8261141082628	0.65\\
1.8251630031054	0.64\\
1.82418510625343	0.63\\
1.82318041770687	0.62\\
1.82214893746575	0.61\\
1.82109066553005	0.6\\
1.82000560189977	0.59\\
1.82	0.589949616648409\\
1.81888748729379	0.58\\
1.81774239770737	0.57\\
1.81657036483656	0.56\\
1.81537138868137	0.55\\
1.81414546924179	0.54\\
1.81289260651783	0.53\\
1.81161280050947	0.52\\
1.81030605121673	0.51\\
1.81	0.50770523415978\\
1.8089665110665	0.5\\
1.80759813279797	0.49\\
1.80620265793007	0.48\\
1.80478008646279	0.47\\
1.80333041839613	0.46\\
1.80185365373009	0.45\\
1.80034979246468	0.44\\
1.8	0.437715205148831\\
1.79881207496686	0.43\\
1.79724510398989	0.42\\
1.79565088134376	0.41\\
1.79402940702845	0.4\\
1.79238068104399	0.39\\
1.79070470339036	0.38\\
1.79	0.375862545454546\\
1.78899572676193	0.37\\
1.7872552854847	0.36\\
1.78548743568341	0.35\\
1.78369217735807	0.34\\
1.78186951050868	0.33\\
1.78001943513524	0.32\\
1.78	0.319896483078964\\
1.77813119478729	0.31\\
1.77621527473216	0.3\\
1.77427178748199	0.29\\
1.77230073303678	0.28\\
1.77030211139653	0.27\\
1.77	0.268508967223253\\
1.76826588356881	0.26\\
1.76620016888698	0.25\\
1.76410672649131	0.24\\
1.76198555638179	0.23\\
1.76	0.220760117302051\\
1.75983570188005	0.22\\
1.75764632801308	0.21\\
1.75542906403316	0.2\\
1.75318390994029	0.19\\
1.75091086573446	0.18\\
1.75	0.176041322314049\\
1.74860174192479	0.17\\
1.74625919738839	0.16\\
1.7438885984264	0.15\\
1.74148994503883	0.14\\
1.74	0.133860220704152\\
1.73905768564722	0.13\\
1.73658837564298	0.12\\
1.73409084495299	0.11\\
1.73156509357723	0.1\\
1.73	0.0938719236564538\\
1.72900522614362	0.0899999999999999\\
1.72640763920253	0.0800000000000001\\
1.72378166333312	0.0699999999999998\\
1.72112729853539	0.0600000000000001\\
1.72	0.0557979797979801\\
1.71843521607623	0.0499999999999998\\
1.71570781353446	0.04\\
1.71295185180377	0.0299999999999998\\
1.71016733088417	0.02\\
1.71	0.0194051684356246\\
1.70733828734116	0.00999999999999979\\
1.70447950268378	0\\
1.70159198652231	-0.00999999999999979\\
1.7	-0.0154590237348853\\
1.69866770027198	-0.02\\
1.69570484436795	-0.0299999999999998\\
1.69271308255265	-0.04\\
1.69	-0.0489817311874732\\
1.68969053630679	-0.0499999999999998\\
1.68662233796449	-0.0600000000000001\\
1.68352505717345	-0.0699999999999998\\
1.68039869393367	-0.0800000000000001\\
1.68	-0.0812635106828654\\
1.67722630843918	-0.0899999999999999\\
1.67402221187737	-0.1\\
1.67078885415974	-0.11\\
1.67	-0.112417855686915\\
1.66751094040657	-0.12\\
1.66419870722535	-0.13\\
1.66085703197141	-0.14\\
1.66	-0.142542278682017\\
1.65747027376416	-0.15\\
1.65404855846272	-0.16\\
1.65059721792058	-0.17\\
1.65	-0.171715667311412\\
1.64709819916259	-0.18\\
1.64356563097061	-0.19\\
1.64000325207659	-0.2\\
1.64	-0.200009053187476\\
1.63638845330461	-0.21\\
1.6327436355455	-0.22\\
1.63	-0.227466048237477\\
1.62906291595197	-0.23\\
1.62533461406518	-0.24\\
1.62157612349914	-0.25\\
1.62	-0.254160086925027\\
1.61777332799359	-0.26\\
1.61393009542485	-0.27\\
1.61005648157099	-0.28\\
1.61	-0.280144676335338\\
1.60612778140679	-0.29\\
1.60216814220789	-0.3\\
1.6	-0.305433646812957\\
1.59816615143589	-0.31\\
1.59411937500874	-0.32\\
1.59004182461637	-0.33\\
1.59	-0.330101804562479\\
1.58590704943191	-0.34\\
1.58174102805975	-0.35\\
1.58	-0.354148272391815\\
1.57752795295118	-0.36\\
1.57327230213279	-0.37\\
1.57	-0.377633388429752\\
1.56897878835428	-0.38\\
1.56463232671926	-0.39\\
1.56025448268976	-0.4\\
1.56	-0.400577159495309\\
1.55581769360003	-0.41\\
1.55134762568027	-0.42\\
1.55	-0.422993620414674\\
1.54682490351118	-0.43\\
1.54226137989852	-0.44\\
1.54	-0.444921044353571\\
1.53765036308337	-0.45\\
1.53299212713011	-0.46\\
1.53	-0.466379460130314\\
1.52829038165702	-0.47\\
1.52353615119771	-0.48\\
1.52	-0.487387817569636\\
1.51874116796719	-0.49\\
1.51388963462304	-0.5\\
1.51	-0.507964059196618\\
1.50899882669201	-0.51\\
1.50404865515142	-0.52\\
1.5	-0.528125186289121\\
1.49905935485801	-0.53\\
1.49400918213223	-0.54\\
1.49	-0.547887319799941\\
1.48891863809538	-0.55\\
1.48376707274835	-0.56\\
1.48	-0.56726575660761\\
1.47857244673592	-0.57\\
1.47331806808704	-0.58\\
1.47	-0.58627502150846\\
1.46801643174564	-0.59\\
1.46265778904448	-0.6\\
1.46	-0.604928915321439\\
1.45724612048415	-0.61\\
1.4517817320558	-0.62\\
1.45	-0.62324055944056\\
1.44625691228179	-0.63\\
1.44068526464165	-0.64\\
1.44	-0.641222437137331\\
1.43504407382561	-0.65\\
1.43	-0.658879703378193\\
1.42935900055001	-0.66\\
1.42360273434431	-0.67\\
1.42	-0.67622143826323\\
1.4177958938154	-0.68\\
1.41192788058189	-0.69\\
1.41	-0.693266023062483\\
1.40599575831991	-0.7\\
1.40001435154917	-0.71\\
1.4	-0.710023853697323\\
1.39395331651914	-0.72\\
1.39	-0.726484584980237\\
1.38784080793694	-0.73\\
1.38166313302151	-0.74\\
1.38	-0.742676738734046\\
1.37541557276804	-0.75\\
1.37	-0.758601657601658\\
1.36911292575128	-0.76\\
1.36273302334752	-0.77\\
1.36	-0.77425966709347\\
1.35628880316551	-0.78\\
1.35	-0.78967277820219\\
1.34978561537563	-0.79\\
1.34319725053805	-0.8\\
1.34	-0.804825988416016\\
1.33654555542479	-0.81\\
1.33	-0.8197498121713\\
1.32983072100314	-0.82\\
1.32302736592392	-0.83\\
1.32	-0.834425563537281\\
1.31615664440763	-0.84\\
1.31	-0.848881251539788\\
1.30921829543303	-0.85\\
1.3021931108089	-0.86\\
1.3	-0.86310504508896\\
1.29509101320516	-0.87\\
1.29	-0.877112484848485\\
1.28791645529764	-0.88\\
1.28066227019084	-0.89\\
1.28	-0.89090813144639\\
1.27331558063663	-0.9\\
1.27	-0.904486041517538\\
1.26589122201874	-0.91\\
1.26	-0.917864704486115\\
1.25838726448074	-0.92\\
1.25079507359137	-0.93\\
1.25	-0.931041813483674\\
1.24310634682478	-0.94\\
1.24	-0.944015891563449\\
1.23533222742768	-0.95\\
1.23	-0.956800976517089\\
1.22747053957165	-0.96\\
1.22	-0.969399953735832\\
1.21951903287412	-0.97\\
1.21146274080804	-0.98\\
1.21	-0.981806503320357\\
1.20330833535935	-0.99\\
1.2	-0.994030530872636\\
1.1950574430757	-1\\
1.19	-1.00607776014509\\
1.18670751898301	-1.01\\
1.18	-1.01795082337018\\
1.17825592781031	-1.02\\
1.17	-1.02965230078563\\
1.16969993994925	-1.03\\
1.16102744900529	-1.04\\
1.16	-1.04117892865081\\
1.15224291941127	-1.05\\
1.15	-1.05253815527538\\
1.143346022604	-1.06\\
1.14	-1.06373409641206\\
1.1343336584589	-1.07\\
1.13	-1.0747691128149\\
1.12520261175289	-1.08\\
1.12	-1.08564551994768\\
1.11594954677266	-1.09\\
1.11	-1.09636558906487\\
1.10657100161716	-1.1\\
1.1	-1.10693154826171\\
1.09706338217363	-1.11\\
1.09	-1.11734558349452\\
1.0874229557452	-1.12\\
1.08	-1.12760983957219\\
1.07764584430633	-1.13\\
1.07	-1.13772642111986\\
1.06772801736059	-1.14\\
1.06	-1.14769739351558\\
1.05766528437292	-1.15\\
1.05	-1.15752478380089\\
1.04745328674667	-1.16\\
1.04	-1.16721058156624\\
1.03708748931319	-1.17\\
1.03	-1.17675673981191\\
1.02656317129886	-1.18\\
1.02	-1.18616517578531\\
1.01587541673193	-1.19\\
1.01	-1.1954377717954\\
1.00501910424815	-1.2\\
1	-1.20457637600495\\
0.993988896250771	-1.21\\
0.99	-1.21358280320131\\
0.982779227376637	-1.22\\
0.98	-1.22245883554648\\
0.971384292216128	-1.23\\
0.97	-1.23120622330689\\
0.96	-1.23982591010779\\
0.959795806397748	-1.24\\
0.95	-1.24831443612067\\
0.94799199208857	-1.25\\
0.94	-1.256678895384\\
0.935981169353855	-1.26\\
0.93	-1.26492093116596\\
0.923756260022203	-1.27\\
0.92	-1.27304215784216\\
0.911309859506401	-1.28\\
0.91	-1.28104416152775\\
0.9	-1.28892381141834\\
0.898618100079185	-1.29\\
0.89	-1.29668231332937\\
0.885670121746323	-1.3\\
0.88	-1.30432577400907\\
0.872472465663833	-1.31\\
0.87	-1.31185568427253\\
0.86	-1.31927037109758\\
0.859003780059516	-1.32\\
0.85	-1.32656598240469\\
0.845231955264801	-1.33\\
0.84	-1.33375199532801\\
0.831176842510377	-1.34\\
0.83	-1.34082981197907\\
0.82	-1.34779143244815\\
0.816786167602178	-1.35\\
0.81	-1.35464369812777\\
0.802069103483753	-1.36\\
0.8	-1.36139150490102\\
0.79	-1.36802786853738\\
0.786987994246632	-1.37\\
0.78	-1.3745561722488\\
0.771535783365571	-1.38\\
0.77	-1.38098360968172\\
0.76	-1.38730017152659\\
0.755665371316667	-1.39\\
0.75	-1.39351394951604\\
0.74	-1.39962877206301\\
0.739384072802847	-1.4\\
0.73	-1.40563239463239\\
0.722615753315227	-1.41\\
0.72	-1.41154074910597\\
0.71	-1.41734406173537\\
0.705353791445787	-1.42\\
0.7	-1.42304779756326\\
0.69	-1.42865398313027\\
0.687561381465005	-1.43\\
0.68	-1.43415643083816\\
0.67	-1.43956710845623\\
0.669186971917401	-1.44\\
0.66	-1.44487153746049\\
0.650169274081466	-1.45\\
0.65	-1.45008794667654\\
0.64	-1.45519792630994\\
0.630442569185983	-1.46\\
0.63	-1.46022146413424\\
0.62	-1.46514032823161\\
0.61	-1.4699739996312\\
0.609945244844861	-1.47\\
0.6	-1.47470339761249\\
0.59	-1.47934784205693\\
0.588569702340194	-1.48\\
0.58	-1.48389171391988\\
0.57	-1.48834845436254\\
0.566222649876585	-1.49\\
0.56	-1.49270978320277\\
0.55	-1.49698032428493\\
0.54278689238	-1.5\\
0.54	-1.50116203955725\\
0.53	-1.50524786790056\\
0.52	-1.50924986390855\\
0.518085490668273	-1.51\\
0.51	-1.51315543105006\\
0.5	-1.51697451653714\\
0.491900914412889	-1.52\\
0.49	-1.52070729072907\\
0.48	-1.52434491449145\\
0.47	-1.52789936993699\\
0.463948555722658	-1.53\\
0.46	-1.53136525013448\\
0.45	-1.53474000358616\\
0.44	-1.53803191680115\\
0.433867128569034	-1.54\\
0.43	-1.54123611359171\\
0.42	-1.54435006251116\\
0.41	-1.54738149669584\\
0.401120465144449	-1.55\\
0.4	-1.55032912293186\\
0.39	-1.55318430884184\\
0.38	-1.555957302971\\
0.37	-1.55864810531934\\
0.364817568028371	-1.56\\
0.36	-1.56125181640971\\
0.35	-1.56376838561049\\
0.34	-1.56620308346624\\
0.33	-1.56855590997696\\
0.323641045649629	-1.57\\
0.32	-1.57082365401589\\
0.31	-1.57300423654016\\
0.3	-1.57510326566637\\
0.29	-1.57712074139453\\
0.28	-1.57905666372462\\
0.274912898619704	-1.58\\
0.27	-1.58090750835238\\
0.26	-1.58267346579919\\
0.25	-1.58435818533497\\
0.24	-1.58596166695973\\
0.23	-1.58748391067347\\
0.22	-1.58892491647617\\
0.212093624725204	-1.59\\
0.21	-1.59028358731827\\
0.2	-1.59155719040112\\
0.19	-1.59274986862848\\
0.18	-1.59386162200035\\
0.17	-1.59489245051673\\
0.16	-1.59584235417761\\
0.15	-1.59671133298301\\
0.14	-1.59749938693291\\
0.13	-1.59820651602733\\
0.12	-1.59883272026625\\
0.11	-1.59937799964968\\
0.1	-1.59984235417761\\
0.095888533576978	-1.6\\
0.0899999999999999	-1.60022491711743\\
0.0800000000000001	-1.60052626068749\\
0.0699999999999998	-1.60074699005409\\
0.0600000000000001	-1.60088710521724\\
0.0499999999999998	-1.60094660617693\\
0.04	-1.60092549293317\\
0.0299999999999998	-1.60082376548595\\
0.02	-1.60064142383528\\
0.00999999999999979	-1.60037846798116\\
0	-1.60003489792357\\
-0.000822706705059097	-1.6\\
-0.01	-1.59960921352251\\
-0.02	-1.59910246978455\\
-0.03	-1.5985148011911\\
-0.04	-1.59784620774216\\
-0.05	-1.59709668943773\\
-0.0600000000000001	-1.59626624627781\\
-0.0700000000000001	-1.59535487826239\\
-0.0800000000000001	-1.59436258539149\\
-0.0900000000000001	-1.59328936766509\\
-0.1	-1.5921352250832\\
-0.11	-1.59090015764582\\
-0.116840143750832	-1.59\\
-0.12	-1.58958255670828\\
-0.13	-1.58818023562511\\
-0.14	-1.58669667663091\\
-0.15	-1.58513187972569\\
-0.16	-1.58348584490944\\
-0.17	-1.58175857218217\\
-0.179723869713175	-1.58\\
-0.18	-1.57994986760812\\
-0.19	-1.57805278022948\\
-0.2	-1.57607413945278\\
-0.21	-1.57401394527802\\
-0.22	-1.57187219770521\\
-0.228420801905518	-1.57\\
-0.23	-1.56964752791069\\
-0.24	-1.56733368775474\\
-0.25	-1.56493797625377\\
-0.26	-1.56246039340776\\
-0.269612961296129	-1.56\\
-0.27	-1.55990055150329\\
-0.28	-1.5572488880982\\
-0.29	-1.55451503291229\\
-0.3	-1.55169898594556\\
-0.305862132465779	-1.55\\
-0.31	-1.54879603500625\\
-0.32	-1.54580389355242\\
-0.33	-1.54272923736382\\
-0.338644566385699	-1.54\\
-0.34	-1.53957037833961\\
-0.35	-1.53631791285637\\
-0.36	-1.53298260713645\\
-0.36872580391334	-1.53\\
-0.37	-1.52956273627363\\
-0.38	-1.52604788478848\\
-0.39	-1.5224498649865\\
-0.39665509315859	-1.52\\
-0.4	-1.51876378095066\\
-0.41	-1.51498445689499\\
-0.42	-1.51112163383336\\
-0.422842225784292	-1.51\\
-0.43	-1.50716403556523\\
-0.44	-1.5031181273816\\
-0.447550419614219	-1.5\\
-0.45	-1.49898433230097\\
-0.46	-1.49475387137912\\
-0.47	-1.4904392421206\\
-0.470998550424518	-1.49\\
-0.48	-1.48602451070057\\
-0.49	-1.48152350466435\\
-0.493322430092944	-1.48\\
-0.5	-1.47692561983471\\
-0.51	-1.47223673094582\\
-0.514685492247913	-1.47\\
-0.52	-1.46745270145676\\
-0.53	-1.46257440531071\\
-0.535186684994614	-1.46\\
-0.54	-1.45760118496575\\
-0.55	-1.4525319385299\\
-0.554911820696096	-1.45\\
-0.56	-1.44736642498606\\
-0.57	-1.44210466629485\\
-0.573935685729185	-1.44\\
-0.58	-1.436743699832\\
-0.59	-1.43128784767594\\
-0.592323756273367	-1.43\\
-0.6	-1.4257282099344\\
-0.61	-1.42007666354264\\
-0.610133603371117	-1.42\\
-0.62	-1.41431507622812\\
-0.627377634754626	-1.41\\
-0.63	-1.40845983745984\\
-0.64	-1.40249933849934\\
-0.644132629144661	-1.4\\
-0.65	-1.39643670525716\\
-0.66	-1.3902759536914\\
-0.66044163654588	-1.39\\
-0.67	-1.38400247760625\\
-0.676290549648046	-1.38\\
-0.68	-1.37762985645933\\
-0.69	-1.37115196172249\\
-0.691754350170509	-1.37\\
-0.7	-1.36456275225831\\
-0.7068275286877	-1.36\\
-0.71	-1.35787087434858\\
-0.72	-1.35107044972013\\
-0.721553719008265	-1.35\\
-0.73	-1.34415603799186\\
-0.735929972066267	-1.34\\
-0.74	-1.33713529297255\\
-0.75	-1.33000681331516\\
-0.750009438796149	-1.33\\
-0.76	-1.32275738025415\\
-0.763756758916443	-1.32\\
-0.77	-1.3153978009032\\
-0.777233521905012	-1.31\\
-0.78	-1.30792664168803\\
-0.79	-1.30034095839085\\
-0.790444141899355	-1.3\\
-0.8	-1.29263220439691\\
-0.803373867127009	-1.29\\
-0.81	-1.28480783767655\\
-0.816064537174115	-1.28\\
-0.82	-1.27686633366633\\
-0.82852437058167	-1.27\\
-0.83	-1.26880614087899\\
-0.84	-1.26062291792093\\
-0.840752685564636	-1.26\\
-0.85	-1.25231304172546\\
-0.852751732572361	-1.25\\
-0.86	-1.24388013767969\\
-0.86454521737068	-1.24\\
-0.87	-1.23532255440309\\
-0.876139824046921	-1.23\\
-0.88	-1.22663861082738\\
-0.887541950008123	-1.22\\
-0.89	-1.21782659552637\\
-0.898757721187628	-1.21\\
-0.9	-1.20888476602762\\
-0.909793006293881	-1.2\\
-0.91	-1.1998113481052\\
-0.92	-1.19060178090702\\
-0.920646711917214	-1.19\\
-0.93	-1.18125670896609\\
-0.931330704672115	-1.18\\
-0.94	-1.17177513061651\\
-0.941852199132123	-1.17\\
-0.95	-1.16215515431451\\
-0.952216057511712	-1.16\\
-0.96	-1.15239485340645\\
-0.962426949960456	-1.15\\
-0.97	-1.14249226531045\\
-0.972489363953858	-1.14\\
-0.98	-1.1324453906749\\
-0.982407613138533	-1.13\\
-0.99	-1.12225219251337\\
-0.992185845668376	-1.12\\
-1	-1.11191059531485\\
-1.00182805206556	-1.11\\
-1.01	-1.1014184841287\\
-1.01133807263765	-1.1\\
-1.02	-1.09077370362335\\
-1.02071960447987	-1.09\\
-1.02997598627787	-1.08\\
-1.03	-1.079973932092\\
-1.03910310494432	-1.07\\
-1.04	-1.06901254677526\\
-1.04811109570316	-1.06\\
-1.05	-1.05789095332891\\
-1.05700319978014	-1.05\\
-1.06	-1.04660680151145\\
-1.06578254187499	-1.04\\
-1.07	-1.03515769488497\\
-1.07445213535139	-1.03\\
-1.08	-1.02354118967452\\
-1.08301488715195	-1.02\\
-1.09	-1.0117547935935\\
-1.09147360245506	-1.01\\
-1.09982950993578	-1\\
-1.1	-0.999794941900205\\
-1.10807308782934	-0.99\\
-1.11	-0.987650332035723\\
-1.11621990951911	-0.98\\
-1.12	-0.975327042577675\\
-1.12427245039225	-0.97\\
-1.13	-0.962822345593338\\
-1.13223310211029	-0.96\\
-1.14	-0.950133457335503\\
-1.14010417611935	-0.95\\
-1.14787019728126	-0.94\\
-1.15	-0.937243363871271\\
-1.1555497561239	-0.93\\
-1.16	-0.924161511216057\\
-1.16314585081127	-0.92\\
-1.17	-0.910885592214574\\
-1.17066052934407	-0.91\\
-1.17808019828273	-0.9\\
-1.18	-0.897398896617894\\
-1.18541661544945	-0.89\\
-1.19	-0.883706052568122\\
-1.19267710637705	-0.88\\
-1.19986239091432	-0.87\\
-1.2	-0.869807457957592\\
-1.20695339634705	-0.86\\
-1.21	-0.855679735358981\\
-1.21397356556323	-0.85\\
-1.22	-0.841339246119734\\
-1.22092456841568	-0.84\\
-1.22779079853729	-0.83\\
-1.23	-0.826764881693649\\
-1.2345835934995	-0.82\\
-1.24	-0.811961933383421\\
-1.24131181031163	-0.81\\
-1.24796128535307	-0.8\\
-1.25	-0.796916940997721\\
-1.25453880792941	-0.79\\
-1.26	-0.781630761395467\\
-1.26105601635803	-0.78\\
-1.2674953415893	-0.77\\
-1.27	-0.766088333762555\\
-1.27386891416414	-0.76\\
-1.28	-0.750296814296814\\
-1.28018613867819	-0.75\\
-1.28642162197281	-0.74\\
-1.29	-0.734228189677757\\
-1.29260184752293	-0.73\\
-1.2987199535701	-0.72\\
-1.3	-0.717895573813941\\
-1.30476708646045	-0.71\\
-1.31	-0.701282058117835\\
-1.31076390322939	-0.7\\
-1.31668736994266	-0.69\\
-1.32	-0.684374426760184\\
-1.3225571237089	-0.68\\
-1.32836805172278	-0.67\\
-1.33	-0.66717472017472\\
-1.33411431344615	-0.66\\
-1.33981433066675	-0.65\\
-1.34	-0.649672285161647\\
-1.34544055704938	-0.64\\
-1.35	-0.631844592716152\\
-1.3510239039521	-0.63\\
-1.35654079412355	-0.62\\
-1.36	-0.613690965381368\\
-1.36200940808728	-0.61\\
-1.3674198244028	-0.6\\
-1.37	-0.5952011969222\\
-1.37277687333587	-0.59\\
-1.37808231266641	-0.58\\
-1.38	-0.576362481962483\\
-1.38333086557783	-0.57\\
-1.38853279344886	-0.56\\
-1.39	-0.557161356328559\\
-1.39367582409339	-0.55\\
-1.39877567555402	-0.54\\
-1.4	-0.537583654130886\\
-1.40381606592064	-0.53\\
-1.40881524638286	-0.52\\
-1.41	-0.517614461446145\\
-1.41375579003448	-0.51\\
-1.41865567608364	-0.5\\
-1.42	-0.497238066281545\\
-1.42349908135518	-0.49\\
-1.42830102153303	-0.48\\
-1.43	-0.476437904468412\\
-1.4330499145948	-0.47\\
-1.43775523015607	-0.46\\
-1.44	-0.455196501093409\\
-1.44241215794893	-0.45\\
-1.44702214359257	-0.44\\
-1.45	-0.433495407031993\\
-1.45158957664107	-0.43\\
-1.45610550121721	-0.42\\
-1.46	-0.411315130099583\\
-1.46058583632643	-0.41\\
-1.46500894351988	-0.4\\
-1.46940056952335	-0.39\\
-1.47	-0.38862520511979\\
-1.4737360153529	-0.38\\
-1.47803624991115	-0.37\\
-1.48	-0.365399933399934\\
-1.48229016905108	-0.36\\
-1.48650020478201	-0.35\\
-1.49	-0.341625211220007\\
-1.49067476743044	-0.34\\
-1.49479577375858	-0.33\\
-1.49888591112545	-0.32\\
-1.5	-0.317255444175596\\
-1.50292621042506	-0.31\\
-1.50692866403647	-0.3\\
-1.51	-0.292267111267112\\
-1.51089468508027	-0.29\\
-1.51481057720278	-0.28\\
-1.51869599545662	-0.27\\
-1.52	-0.266617319439454\\
-1.52253471887688	-0.26\\
-1.52633487027734	-0.25\\
-1.53	-0.240277838627236\\
-1.53010407976257	-0.24\\
-1.53382004185073	-0.23\\
-1.53750591517705	-0.22\\
-1.54	-0.213177702955481\\
-1.54115440546895	-0.21\\
-1.54475733565284	-0.2\\
-1.54833036600117	-0.19\\
-1.55	-0.185287686996548\\
-1.55186180629651	-0.18\\
-1.55535311516592	-0.17\\
-1.55881471076836	-0.16\\
-1.56	-0.156546241637151\\
-1.56223266177201	-0.15\\
-1.56561373367515	-0.14\\
-1.56896527656604	-0.13\\
-1.57	-0.126885252525253\\
-1.57227319415727	-0.12\\
-1.57554538784766	-0.11\\
-1.57878823450944	-0.1\\
-1.58	-0.0962291407222914\\
-1.58198947326554	-0.0899999999999999\\
-1.58515412252907	-0.0800000000000001\\
-1.58828960451828	-0.0699999999999998\\
-1.59	-0.0644938113529662\\
-1.59138742110187	-0.0600000000000001\\
-1.59444583536481	-0.0499999999999998\\
-1.59747525991909	-0.04\\
-1.6	-0.0315854194115062\\
-1.60047281633508	-0.0299999999999998\\
-1.60342628125368	-0.02\\
-1.60635093188039	-0.00999999999999979\\
-1.60924676821522	0\\
-1.61	0.00262722704431137\\
-1.61210107664067	0.00999999999999979\\
-1.61492221368779	0.02\\
-1.61771470974965	0.0299999999999998\\
-1.62	0.0382684879886954\\
-1.62047570365577	0.04\\
-1.62319456486574	0.0499999999999998\\
-1.62588495632481	0.0600000000000001\\
-1.628546878033	0.0699999999999998\\
-1.63	0.0755179361179359\\
-1.63117331515237	0.0800000000000001\\
-1.63376281564763	0.0899999999999999\\
-1.63632401559103	0.1\\
-1.63885691498257	0.11\\
-1.64	0.114563944530046\\
-1.64135346999322	0.12\\
-1.643815138301	0.13\\
-1.64624867325669	0.14\\
-1.64865407486029	0.15\\
-1.65	0.155661646046261\\
-1.65102528571973	0.16\\
-1.65336062343474	0.17\\
-1.6556679930334	0.18\\
-1.65794739451571	0.19\\
-1.66	0.199116883116883\\
-1.66019766692365	0.2\\
-1.66240814933711	0.21\\
-1.66459082694033	0.22\\
-1.66674569973333	0.23\\
-1.66887276771609	0.24\\
-1.67	0.245369656833233\\
-1.67096638813847	0.25\\
-1.67302582144877	0.26\\
-1.67505761135892	0.27\\
-1.67706175786895	0.28\\
-1.67903826097883	0.29\\
-1.68	0.294934880722113\\
-1.68098142341374	0.3\\
-1.68289155121366	0.31\\
-1.68477419516034	0.32\\
-1.68662935525379	0.33\\
-1.688457031494	0.34\\
-1.69	0.348571131158916\\
-1.69025574780459	0.35\\
-1.69201828367015	0.36\\
-1.69375349339825	0.37\\
-1.69546137698891	0.38\\
-1.69714193444211	0.39\\
-1.69879516575787	0.4\\
-1.7	0.407410236822\\
-1.70041866841215	0.41\\
-1.70200812636623	0.42\\
-1.70357041409888	0.43\\
-1.70510553161008	0.44\\
-1.70661347889985	0.45\\
-1.70809425596819	0.46\\
-1.70954786281508	0.47\\
-1.71	0.473169696969696\\
-1.71096877187381	0.48\\
-1.71236009971388	0.49\\
-1.71372441147938	0.5\\
-1.71506170717031	0.51\\
-1.71637198678668	0.52\\
-1.71765525032849	0.53\\
-1.71891149779573	0.54\\
-1.72	0.548855144855143\\
-1.72013993528298	0.55\\
-1.72133536845961	0.56\\
-1.72250393796935	0.57\\
-1.7236456438122	0.58\\
-1.72476048598816	0.59\\
-1.72584846449722	0.6\\
-1.7269095793394	0.61\\
-1.72794383051468	0.62\\
-1.72895121802308	0.63\\
-1.72993174186458	0.64\\
-1.73	0.640715749039693\\
-1.73088043517855	0.65\\
-1.73180203261411	0.66\\
-1.73269691708052	0.67\\
-1.73356508857778	0.68\\
-1.73440654710589	0.69\\
-1.73522129266486	0.7\\
-1.73600932525468	0.71\\
-1.73677064487536	0.72\\
-1.73750525152689	0.73\\
-1.73821314520927	0.74\\
-1.73889432592251	0.75\\
-1.7395487936666	0.76\\
-1.74	0.767187620889747\\
-1.74017556357841	0.77\\
-1.74077325251452	0.78\\
-1.74134437749792	0.79\\
-1.7418889385286	0.8\\
-1.74240693560656	0.81\\
-1.74289836873181	0.82\\
-1.74336323790435	0.83\\
-1.74380154312416	0.84\\
-1.74421328439126	0.85\\
-1.74459846170565	0.86\\
-1.74495707506732	0.87\\
-1.74528912447627	0.88\\
-1.7455946099325	0.89\\
-1.74587353143602	0.9\\
-1.74612588898683	0.91\\
-1.74635168258491	0.92\\
-1.74655091223029	0.93\\
-1.74672357792294	0.94\\
-1.74686967966288	0.95\\
-1.7469892174501	0.96\\
-1.74708219128461	0.97\\
-1.7471486011664	0.98\\
-1.74718844709547	0.99\\
-1.74720172907183	1\\
91.1515909090909	855\\
2	-0.00754183627317985\\
1.99939590303327	-0.00999999999999979\\
1.99691393702275	-0.02\\
1.99440751814514	-0.0299999999999998\\
1.99187664640043	-0.04\\
1.99	-0.0473440626359274\\
1.98931781871809	-0.0499999999999998\\
1.98672472544047	-0.0600000000000001\\
1.98410705307965	-0.0699999999999998\\
1.98146480163563	-0.0800000000000001\\
1.98	-0.085492668621701\\
1.97879173451626	-0.0899999999999999\\
1.97608636082879	-0.1\\
1.97335628053232	-0.11\\
1.97060149362682	-0.12\\
1.97	-0.122164040404039\\
1.96781064086789	-0.13\\
1.96499181558538	-0.14\\
1.96214815483784	-0.15\\
1.96	-0.15748878394333\\
1.95927588202585	-0.16\\
1.95636738121447	-0.17\\
1.95343391473088	-0.18\\
1.9504754825751	-0.19\\
1.95	-0.191593761886648\\
1.94747886697316	-0.2\\
1.94445464812512	-0.21\\
1.94140533202523	-0.22\\
1.94	-0.224571057513915\\
1.93832207530076	-0.23\\
1.93520614241316	-0.24\\
1.93206497929975	-0.25\\
1.93	-0.256521550162984\\
1.92889271920217	-0.26\\
1.92568409523261	-0.27\\
1.92245010664668	-0.28\\
1.92	-0.287517155995755\\
1.91918641985326	-0.29\\
1.91588411206287	-0.3\\
1.91255630382623	-0.31\\
1.91	-0.317623228482544\\
1.90919870412884	-0.32\\
1.90580170374428	-0.33\\
1.90237906562096	-0.34\\
1.9	-0.346899290300777\\
1.8989250020504	-0.35\\
1.89543228391662	-0.36\\
1.89191378926525	-0.37\\
1.89	-0.375399669421488\\
1.88836064415046	-0.38\\
1.88477116637412	-0.39\\
1.88115577179073	-0.4\\
1.88	-0.403174053704303\\
1.87750085875037	-0.41\\
1.87381356233343	-0.42\\
1.87010020728457	-0.43\\
1.87	-0.430267975926512\\
1.86634076914832	-0.44\\
1.86255457761157	-0.45\\
1.86	-0.456700718525461\\
1.85873522472785	-0.46\\
1.85487539071388	-0.47\\
1.85098920971006	-0.48\\
1.85	-0.482528313437404\\
1.84706041739424	-0.49\\
1.84309962788569	-0.5\\
1.84	-0.50777378435518\\
1.83910737856788	-0.51\\
1.83507114571854	-0.52\\
1.83100827106822	-0.53\\
1.83	-0.532465501924785\\
1.82690141874201	-0.54\\
1.82276210193022	-0.55\\
1.82	-0.556629932768197\\
1.81858804942868	-0.56\\
1.81437142542221	-0.57\\
1.81012785813136	-0.58\\
1.81	-0.58029939776312\\
1.80583377467946	-0.59\\
1.80151186707886	-0.6\\
1.8	-0.603476352308128\\
1.79714662636723	-0.61\\
1.79274548179712	-0.62\\
1.79	-0.626199720279721\\
1.78830739905565	-0.63\\
1.78382610537332	-0.64\\
1.78	-0.648486045869025\\
1.77931345153812	-0.65\\
1.77475108075935	-0.66\\
1.77016114278554	-0.67\\
1.77	-0.670348982360922\\
1.76551768902109	-0.68\\
1.76084556923736	-0.69\\
1.76	-0.691799141861089\\
1.75612314752602	-0.7\\
1.7513678832672	-0.71\\
1.75	-0.712859793267957\\
1.74656460806692	-0.72\\
1.7417252196534	-0.73\\
1.74	-0.733544406602043\\
1.73683915491874	-0.74\\
1.73191464525315	-0.75\\
1.73	-0.753865837865838\\
1.72694380282599	-0.76\\
1.72193315697787	-0.77\\
1.72	-0.773836363636363\\
1.71687549491776	-0.78\\
1.71177767969571	-0.79\\
1.71	-0.793467713345151\\
1.7066311005472	-0.8\\
1.70144506405819	-0.81\\
1.7	-0.812771099423992\\
1.69620741305233	-0.82\\
1.69093208424759	-0.83\\
1.69	-0.831757245479317\\
1.6856011474348	-0.84\\
1.68023543564186	-0.85\\
1.68	-0.85043641264396\\
1.67480893795305	-0.86\\
1.67	-0.868812088715574\\
1.66934772426161	-0.87\\
1.66382733562625	-0.88\\
1.66	-0.886896310585966\\
1.65826678875857	-0.89\\
1.65265280564496	-0.9\\
1.65	-0.904700548795037\\
1.64699047412567	-0.91\\
1.64128172468462	-0.92\\
1.64	-0.922233530106257\\
1.63551508788197	-0.93\\
1.63	-0.939501057082452\\
1.62970854202401	-0.94\\
1.62383684391081	-0.95\\
1.62	-0.95650104626831\\
1.61792178200047	-0.96\\
1.61195185947275	-0.97\\
1.61	-0.973252934407365\\
1.60592556045086	-0.98\\
1.6	-0.989762995191207\\
1.59985522248178	-0.99\\
1.59371581641931	-1\\
1.59	-1.00602221718431\\
1.58752967181494	-1.01\\
1.58128838329086	-1.02\\
1.58	-1.02205409652076\\
1.57498349039892	-1.03\\
1.57	-1.03785481349118\\
1.56863001583384	-1.04\\
1.56221231616336	-1.05\\
1.56	-1.0534304357443\\
1.5557351272957	-1.06\\
1.55	-1.06879154743562\\
1.54920640656847	-1.07\\
1.54260758322612	-1.08\\
1.54	-1.08393263570961\\
1.5359499105025	-1.09\\
1.53	-1.09887090475157\\
1.52923756501355	-1.1\\
1.52245271408688	-1.11\\
1.52	-1.11359789383194\\
1.51560588851356	-1.12\\
1.51	-1.12812919786096\\
1.50870104409559	-1.13\\
1.5017250597793	-1.14\\
1.5	-1.14246132655224\\
1.49467989651408	-1.15\\
1.49	-1.15660071714828\\
1.48757314740916	-1.16\\
1.48040071076844	-1.17\\
1.48	-1.17055611285266\\
1.47314747137766	-1.18\\
1.47	-1.18431786977325\\
1.46582884107326	-1.19\\
1.46	-1.19790412093601\\
1.45844342597007	-1.2\\
1.45098275941619	-1.21\\
1.45	-1.21131130720295\\
1.44344165801825	-1.22\\
1.44	-1.2245391215526\\
1.43582960200945	-1.23\\
1.43	-1.23759934919666\\
1.42814504596527	-1.24\\
1.42038359393416	-1.25\\
1.42	-1.25049203789559\\
1.41253336077122	-1.26\\
1.41	-1.26321171984748\\
1.40460604997528	-1.27\\
1.4	-1.27577122877123\\
1.39659994538065	-1.28\\
1.39	-1.28817266759499\\
1.38851327949245	-1.29\\
1.38034165763025	-1.3\\
1.38	-1.30041628870045\\
1.37207517325724	-1.31\\
1.37	-1.31249872373846\\
1.36372295154692	-1.32\\
1.36	-1.32442991202346\\
1.35528302511548	-1.33\\
1.35	-1.33621179676854\\
1.34675336592202	-1.34\\
1.34	-1.34784628804032\\
1.33813188291272	-1.35\\
1.33	-1.35933526346265\\
1.32941641955435	-1.36\\
1.32060001694484	-1.37\\
1.32	-1.37067770334928\\
1.3116812118935	-1.38\\
1.31	-1.38187611968744\\
1.30266220240657	-1.39\\
1.3	-1.39293509204783\\
1.29354057852817	-1.4\\
1.29	-1.40385635985636\\
1.28431385239736	-1.41\\
1.28	-1.41464163372859\\
1.27497945507451	-1.42\\
1.27	-1.42529259606373\\
1.26553473321124	-1.43\\
1.26	-1.43581090162404\\
1.25597694555494	-1.44\\
1.25	-1.44619817810002\\
1.24630325927801	-1.45\\
1.24	-1.45645602666173\\
1.23651074612173	-1.46\\
1.23	-1.46658602249677\\
1.2265963783436	-1.47\\
1.22	-1.47658971533517\\
1.21655702445627	-1.48\\
1.21	-1.48646862996159\\
1.20638944474575	-1.49\\
1.2	-1.49622426671525\\
1.19609028655511	-1.5\\
1.19	-1.50585810197786\\
1.18565607931947	-1.51\\
1.18	-1.51537158864992\\
1.17508322933666	-1.52\\
1.17	-1.52476615661566\\
1.16436801425721	-1.53\\
1.16	-1.53404321319706\\
1.15350657727566	-1.54\\
1.15	-1.54320414359707\\
1.14249492100436	-1.55\\
1.14	-1.5522503113325\\
1.13132890100922	-1.56\\
1.13	-1.56118305865674\\
1.12000421898543	-1.57\\
1.12	-1.57000370697264\\
1.11	-1.57870856134157\\
1.10850238480277	-1.58\\
1.1	-1.58730349920872\\
1.09683089131827	-1.59\\
1.09	-1.59578980557015\\
1.08498487282742	-1.6\\
1.08	-1.60416873146048\\
1.07295923311914	-1.61\\
1.07	-1.61244150877803\\
1.06074867561645	-1.62\\
1.06	-1.62060935064935\\
1.05	-1.6286683982684\\
1.04833112697758	-1.63\\
1.04	-1.63662221838882\\
1.03570752131881	-1.64\\
1.03	-1.6444743083004\\
1.02287919388772	-1.65\\
1.02	-1.652225817497\\
1.01	-1.6598774182503\\
1.00983812258371	-1.66\\
1	-1.66742196827563\\
0.996546642600928	-1.67\\
0.99	-1.67486898895497\\
0.983027455145819	-1.68\\
0.98	-1.68221957000169\\
0.97	-1.68947282884713\\
0.969265271453177	-1.69\\
0.96	-1.69662320796087\\
0.955224112020229	-1.7\\
0.95	-1.70368005377248\\
0.940928387081213	-1.71\\
0.94	-1.7106443997991\\
0.93	-1.71750811987276\\
0.926328111508573	-1.72\\
0.92	-1.7242787322769\\
0.911440870411499	-1.73\\
0.91	-1.73095961442579\\
0.9	-1.737542795413\\
0.896223402896626	-1.74\\
0.89	-1.74403444278854\\
0.880687827952542	-1.75\\
0.88	-1.75043903646263\\
0.87	-1.75674575152615\\
0.864776898021873	-1.76\\
0.86	-1.76296547756041\\
0.85	-1.76909748479369\\
0.848509731534515	-1.77\\
0.84	-1.77513497133497\\
0.831835225554609	-1.78\\
0.83	-1.78108960339481\\
0.82	-1.7869513628203\\
0.814731334442784	-1.79\\
0.81	-1.79272792323955\\
0.8	-1.79841844202309\\
0.79718352689044	-1.8\\
0.79	-1.80401944579485\\
0.78	-1.80953994490358\\
0.779155185240291	-1.81\\
0.77	-1.81496786694655\\
0.760597109495218	-1.82\\
0.76	-1.82031842316975\\
0.75	-1.8255768302494\\
0.741467767466401	-1.83\\
0.74	-1.83075821709155\\
0.73	-1.83584992784993\\
0.721729030196517	-1.84\\
0.72	-1.84086451509826\\
0.71	-1.84579070139\\
0.701325277402786	-1.85\\
0.7	-1.85064082152523\\
0.69	-1.85540264289126\\
0.680193907698577	-1.86\\
0.68	-1.86009059178169\\
0.67	-1.86468919562113\\
0.66	-1.86921450103125\\
0.658235629521399	-1.87\\
0.65	-1.87365375494071\\
0.64	-1.87801723320158\\
0.635378634336884	-1.88\\
0.63	-1.88229966913502\\
0.62	-1.88650244209863\\
0.611531301262733	-1.89\\
0.61	-1.8906302402261\\
0.6	-1.89467341811901\\
0.59	-1.89864405715183\\
0.586521528980546	-1.9\\
0.58	-1.90253340635268\\
0.57	-1.90634579877953\\
0.560229678283061	-1.91\\
0.56	-1.91008560736005\\
0.55	-1.91374083892094\\
0.54	-1.91732402931545\\
0.532378647244304	-1.92\\
0.53	-1.92083232323232\\
0.52	-1.92425967365967\\
0.51	-1.92761522921523\\
0.502737683971417	-1.93\\
0.5	-1.93089592690104\\
0.49	-1.93409694904755\\
0.48	-1.93722642093852\\
0.470929855659661	-1.94\\
0.47	-1.94028337706436\\
0.46	-1.94325960796419\\
0.45	-1.94616453156351\\
0.44	-1.94899814786232\\
0.436373135162318	-1.95\\
0.43	-1.9517544993078\\
0.42	-1.95443639440086\\
0.41	-1.95704722350408\\
0.4	-1.95958698661744\\
0.398326998566889	-1.96\\
0.39	-1.96204875057489\\
0.38	-1.96443829526292\\
0.37	-1.96675701364403\\
0.36	-1.96900490571823\\
0.355429195127104	-1.97\\
0.35	-1.97117799847212\\
0.34	-1.97327715813598\\
0.33	-1.97530572956455\\
0.32	-1.97726371275783\\
0.31	-1.97915110771581\\
0.305327558657809	-1.98\\
0.3	-1.98096467184407\\
0.29	-1.98270504035328\\
0.28	-1.9843750571037\\
0.27	-1.98597472209533\\
0.26	-1.98750403532816\\
0.25	-1.98896299680219\\
0.242532075885516	-1.99\\
0.24	-1.99035043253908\\
0.23	-1.99166428896646\\
0.22	-1.9929080285324\\
0.21	-1.99408165123691\\
0.2	-1.99518515707998\\
0.19	-1.99621854606162\\
0.18	-1.99718181818182\\
0.17	-1.99807497344058\\
0.16	-1.99889801183791\\
0.15	-1.9996509333738\\
0.144887752833965	-2\\
0.14	-2.00033262743912\\
0.13	-2.00094327635759\\
0.12	-2.0014840417486\\
0.11	-2.00195492361216\\
0.1	-2.00235592194827\\
0.0899999999999999	-2.00268703675692\\
0.0800000000000001	-2.00294826803812\\
0.0699999999999998	-2.00313961579186\\
0.0600000000000001	-2.00326108001815\\
0.0499999999999998	-2.00331266071699\\
0.04	-2.00329435788837\\
0.0299999999999998	-2.00320617153229\\
0.02	-2.00304810164877\\
0.00999999999999979	-2.00282014823779\\
0	-2.00252231129935\\
-0.01	-2.00215459083346\\
-0.02	-2.00171698684012\\
-0.03	-2.00120949931932\\
-0.04	-2.00063212827106\\
-0.0497663005375091	-2\\
-0.05	-1.99998482319017\\
-0.0600000000000001	-1.99926529063591\\
-0.0700000000000001	-1.99847564122022\\
-0.0800000000000001	-1.99761587494309\\
-0.0900000000000001	-1.99668599180452\\
-0.1	-1.99568599180452\\
-0.11	-1.99461587494309\\
-0.12	-1.99347564122022\\
-0.13	-1.99226529063591\\
-0.14	-1.99098482319017\\
-0.147291830542759	-1.99\\
-0.15	-1.98963301355261\\
-0.16	-1.9882075529161\\
-0.17	-1.98671174052079\\
-0.18	-1.98514557636668\\
-0.19	-1.98350906045378\\
-0.2	-1.98180219278209\\
-0.21	-1.98002497335161\\
-0.210135168548586	-1.98\\
-0.22	-1.97817127578304\\
-0.23	-1.97624690603514\\
-0.24	-1.97425194805195\\
-0.25	-1.97218640183346\\
-0.26	-1.97005026737968\\
-0.26022779200997	-1.97\\
-0.27	-1.96783627165415\\
-0.28	-1.96555128008585\\
-0.29	-1.96319546221064\\
-0.3	-1.96076881802851\\
-0.30307838683936	-1.96\\
-0.31	-1.95826549761575\\
-0.32	-1.95568850946008\\
-0.33	-1.95304045531457\\
-0.34	-1.9503213351792\\
-0.341151662164398	-1.95\\
-0.35	-1.94752276585893\\
-0.36	-1.94465179811699\\
-0.37	-1.94170952307455\\
-0.375672727272728	-1.94\\
-0.38	-1.9386914975995\\
-0.39	-1.93559609725879\\
-0.4	-1.93242914666254\\
-0.407500836880112	-1.93\\
-0.41	-1.92918787878788\\
-0.42	-1.92586651126651\\
-0.43	-1.92247334887335\\
-0.437138180024219	-1.92\\
-0.44	-1.91900498986434\\
-0.45	-1.91545610478715\\
-0.46	-1.91183517854358\\
-0.464969387324241	-1.91\\
-0.47	-1.90813581599124\\
-0.48	-1.9043578469723\\
-0.49	-1.90050758879675\\
-0.491294028481391	-1.9\\
-0.5	-1.89657324540744\\
-0.51	-1.89256460982886\\
-0.516283999538338	-1.89\\
-0.52	-1.88847817866709\\
-0.53	-1.88431006774854\\
-0.54	-1.88006916653537\\
-0.540160341867856	-1.88\\
-0.55	-1.87574071146245\\
-0.56	-1.87133897233202\\
-0.562992262304349	-1.87\\
-0.57	-1.86685324448675\\
-0.58	-1.86228954466127\\
-0.584937557737709	-1.86\\
-0.59	-1.85764432415221\\
-0.6	-1.85291752905588\\
-0.606077742031773	-1.85\\
-0.61	-1.84811056079246\\
-0.62	-1.8432195238856\\
-0.626484633950121	-1.84\\
-0.63	-1.83824851691518\\
-0.64	-1.83319207952541\\
-0.646221756929904	-1.83\\
-0.65	-1.82805470635559\\
-0.66	-1.82283169750603\\
-0.665345503143699	-1.82\\
-0.67	-1.81752559341192\\
-0.68	-1.81213482964638\\
-0.683906106892782	-1.81\\
-0.69	-1.8066575919624\\
-0.7	-1.8010978771674\\
-0.701948462799459	-1.8\\
-0.71	-1.79544706456334\\
-0.719506428632249	-1.79\\
-0.72	-1.78971617431043\\
-0.73	-1.78389032152766\\
-0.736592361091905	-1.78\\
-0.74	-1.77798181818182\\
-0.75	-1.77198361998362\\
-0.753265823467544	-1.77\\
-0.76	-1.76589495314812\\
-0.769551447673335	-1.76\\
-0.77	-1.75972215805973\\
-0.78	-1.75345174063686\\
-0.785438687706346	-1.75\\
-0.79	-1.74709455207816\\
-0.8	-1.74064828614009\\
-0.800993881851184	-1.74\\
-0.81	-1.73410403855742\\
-0.816196271297017	-1.73\\
-0.82	-1.72747139282736\\
-0.83	-1.72074662218515\\
-0.83109767761238	-1.72\\
-0.84	-1.71392248451364\\
-0.845681272581779	-1.71\\
-0.85	-1.70700722567636\\
-0.859999760197597	-1.7\\
-0.86	-1.69999983133749\\
-0.87	-1.69288851408332\\
-0.874017829912024	-1.69\\
-0.88	-1.68568325715253\\
-0.887791445215494	-1.68\\
-0.89	-1.67838300764656\\
-0.9	-1.67098300764656\\
-0.901314444116244	-1.67\\
-0.91	-1.66348012962647\\
-0.914587952240686	-1.66\\
-0.92	-1.65587930148947\\
-0.927642372315567	-1.65\\
-0.93	-1.64817941227015\\
-0.94	-1.64037789998282\\
-0.940479513290739	-1.64\\
-0.95	-1.63246851819907\\
-0.953089311543361	-1.63\\
-0.96	-1.62445696969697\\
-0.965501784836372	-1.62\\
-0.97	-1.6163420823918\\
-0.977722721981162	-1.61\\
-0.98	-1.60812266620136\\
-0.989757687551093	-1.6\\
-0.99	-1.59979751269925\\
-1	-1.59136013312314\\
-1.00159671814275	-1.59\\
-1.01	-1.58281413750659\\
-1.01325993522498	-1.58\\
-1.02	-1.57415904677846\\
-1.02475451518515	-1.57\\
-1.03	-1.56539358497253\\
-1.03608513105544	-1.56\\
-1.04	-1.55651645614659\\
-1.0472562847917	-1.55\\
-1.05	-1.54752634399\\
-1.05827231503111	-1.54\\
-1.06	-1.53842191142191\\
-1.06913740443165	-1.53\\
-1.07	-1.52920180018002\\
-1.07985558661911	-1.52\\
-1.08	-1.51986463039942\\
-1.09	-1.51040737393819\\
-1.09042698289416	-1.51\\
-1.1	-1.50083015786609\\
-1.10085913879552	-1.5\\
-1.11	-1.49113208234651\\
-1.11115688939363	-1.49\\
-1.12	-1.48131168831169\\
-1.12132367328103	-1.48\\
-1.13	-1.47136749311295\\
-1.13136281274594	-1.47\\
-1.14	-1.46129799004241\\
-1.14127751864825	-1.46\\
-1.15	-1.45110164784299\\
-1.15107089505228	-1.45\\
-1.16	-1.44077691020636\\
-1.1607459436303	-1.44\\
-1.17	-1.43032219525854\\
-1.17030556784987	-1.43\\
-1.17975055324422	-1.42\\
-1.18	-1.41973480143045\\
-1.18908230459901	-1.41\\
-1.19	-1.40901228501229\\
-1.19830601473637	-1.4\\
-1.2	-1.39815429872841\\
-1.20742427122393	-1.39\\
-1.21	-1.38715913855536\\
-1.21643957965474	-1.38\\
-1.22	-1.37602507177033\\
-1.22535436686793	-1.37\\
-1.23	-1.36475033634442\\
-1.23417098401877	-1.36\\
-1.24	-1.3533331403204\\
-1.24289170950619	-1.35\\
-1.25	-1.34177166117465\\
-1.25151875176551	-1.34\\
-1.26	-1.33006404516255\\
-1.26005425193345	-1.33\\
-1.2684888610392	-1.32\\
-1.27	-1.31820066758296\\
-1.27683511987562	-1.31\\
-1.28	-1.30618635377638\\
-1.28509542449689	-1.3\\
-1.29	-1.29401940978412\\
-1.29327169549727	-1.29\\
-1.3	-1.28169783170877\\
-1.30136579667472	-1.28\\
-1.3093749499928	-1.27\\
-1.31	-1.26921613485852\\
-1.31729496608581	-1.26\\
-1.32	-1.25656742592219\\
-1.3251377434361	-1.25\\
-1.33	-1.24375703583721\\
-1.33290495796609	-1.24\\
-1.34	-1.23078279438682\\
-1.34059823746095	-1.23\\
-1.34820637560422	-1.22\\
-1.35	-1.21763184896368\\
-1.35573910896437	-1.21\\
-1.36	-1.20430818387961\\
-1.36320227387647	-1.2\\
-1.37	-1.19081300476289\\
-1.37059733739064	-1.19\\
-1.37791114492202	-1.18\\
-1.38	-1.17713082549634\\
-1.38515524196595	-1.17\\
-1.39	-1.16326768843166\\
-1.39233521883299	-1.16\\
-1.39944860388003	-1.15\\
-1.4	-1.14922123331214\\
-1.40648403391396	-1.14\\
-1.41	-1.1349763678944\\
-1.41345907389046	-1.13\\
-1.42	-1.1205456684492\\
-1.42037495406776	-1.12\\
-1.42721408098773	-1.11\\
-1.43	-1.10590714748434\\
-1.43399375155116	-1.1\\
-1.44	-1.09107376871339\\
-1.44071767282008	-1.09\\
-1.44736945141316	-1.08\\
-1.45	-1.07602628696605\\
-1.4539627778338	-1.07\\
-1.46	-1.06077459828307\\
-1.46050355594351	-1.06\\
-1.46697294048624	-1.05\\
-1.47	-1.04529806623694\\
-1.47338844267539	-1.04\\
-1.47975278982159	-1.03\\
-1.48	-1.02960965207632\\
-1.48604615362888	-1.02\\
-1.49	-1.0136846379427\\
-1.49229188006005	-1.01\\
-1.49848110679257	-1\\
-1.5	-0.997533606744133\\
-1.50460958302547	-0.99\\
-1.51	-0.981145866727731\\
-1.51069314199438	-0.98\\
-1.51671173105426	-0.97\\
-1.52	-0.964508674531575\\
-1.52268267841167	-0.96\\
-1.52860243617094	-0.95\\
-1.53	-0.947627015657864\\
-1.53446353105297	-0.94\\
-1.54	-0.93049072116514\\
-1.54028391252939	-0.93\\
-1.54603963087294	-0.92\\
-1.55	-0.913083313553287\\
-1.55175443335449	-0.91\\
-1.55741481071298	-0.9\\
-1.56	-0.895408731110578\\
-1.56302658953331	-0.89\\
-1.56859280834329	-0.88\\
-1.57	-0.877458424242424\\
-1.57410404855599	-0.87\\
-1.57957726939238	-0.86\\
-1.58	-0.859223474638569\\
-1.58499038805733	-0.85\\
-1.59	-0.840690810544469\\
-1.59036948701393	-0.84\\
-1.59568909855183	-0.83\\
-1.6	-0.821851805728518\\
-1.60097379209179	-0.82\\
-1.6062035860696	-0.81\\
-1.61	-0.802700579199195\\
-1.61139611784464	-0.8\\
-1.61653717469699	-0.79\\
-1.62	-0.783226636108989\\
-1.62163972824329	-0.78\\
-1.62669310902621	-0.77\\
-1.63	-0.763419005923255\\
-1.63170780966592	-0.76\\
-1.63667455651749	-0.75\\
-1.64	-0.743266215160198\\
-1.64160347319019	-0.74\\
-1.64648460977762	-0.73\\
-1.65	-0.722756258234519\\
-1.65132975680452	-0.72\\
-1.65612628875809	-0.71\\
-1.66	-0.701876566249\\
-1.66088962754193	-0.7\\
-1.66560254287628	-0.69\\
-1.67	-0.680613973563529\\
-1.67028598353961	-0.68\\
-1.67491625306276	-0.67\\
-1.67951887918578	-0.66\\
-1.68	-0.658948365833563\\
-1.68407023373768	-0.65\\
-1.68859132759504	-0.64\\
-1.69	-0.636865165415625\\
-1.6930672347191	-0.63\\
-1.6975077320548	-0.62\\
-1.7	-0.614352659724177\\
-1.70190994306604	-0.61\\
-1.70627076360672	-0.6\\
-1.71	-0.591394699344543\\
-1.71060098485872	-0.59\\
-1.71488303267717	-0.58\\
-1.71913806442106	-0.57\\
-1.72	-0.567961370897473\\
-1.72334709078698	-0.56\\
-1.72752439098846	-0.55\\
-1.73	-0.544035304501324\\
-1.7316654322037	-0.54\\
-1.73576587296769	-0.53\\
-1.73983960076253	-0.52\\
-1.74	-0.519603660366037\\
-1.74386493437496	-0.51\\
-1.74786280925874	-0.5\\
-1.75	-0.494618425053208\\
-1.75182394543773	-0.49\\
-1.75574680899605	-0.48\\
-1.75964325596475	-0.47\\
-1.76	-0.469078188023581\\
-1.76349390396809	-0.46\\
-1.76731631300527	-0.45\\
-1.77	-0.442930481283423\\
-1.77110634760406	-0.44\\
-1.77485553114423	-0.43\\
-1.77857858796984	-0.42\\
-1.78	-0.416155155798266\\
-1.78226310132637	-0.41\\
-1.78591387435483	-0.4\\
-1.78953866323361	-0.39\\
-1.79	-0.388718083360683\\
-1.79312044074287	-0.38\\
-1.79667387141867	-0.37\\
-1.8	-0.360571095571096\\
-1.80020037153439	-0.36\\
-1.80368321435665	-0.35\\
-1.80714035354184	-0.34\\
-1.81	-0.33166632618318\\
-1.81056871927765	-0.33\\
-1.81395616654852	-0.32\\
-1.81731804817961	-0.31\\
-1.82	-0.301961337513061\\
-1.82065086979137	-0.3\\
-1.82394394035716	-0.29\\
-1.82721158180662	-0.28\\
-1.83	-0.271399643493761\\
-1.83045138369914	-0.27\\
-1.83365108017062	-0.26\\
-1.83682548259885	-0.25\\
-1.83997459098382	-0.24\\
-1.84	-0.239918660287081\\
-1.84308203433251	-0.23\\
-1.84616418303046	-0.22\\
-1.84922117133087	-0.21\\
-1.85	-0.207431158053564\\
-1.85224115766601	-0.2\\
-1.85523202238882	-0.19\\
-1.85819785895496	-0.18\\
-1.86	-0.173871953578337\\
-1.86113271391872	-0.17\\
-1.86403324921064	-0.16\\
-1.86690888720392	-0.15\\
-1.86975962789855	-0.14\\
-1.87	-0.139149379255106\\
-1.87257202364199	-0.13\\
-1.87535840135097	-0.12\\
-1.87812001125809	-0.11\\
-1.88	-0.103130810366105\\
-1.88085241972516	-0.1\\
-1.88355046087336	-0.0899999999999999\\
-1.88622386237638	-0.0800000000000001\\
-1.88887262423421	-0.0699999999999998\\
-1.89	-0.0657037985488691\\
-1.89148904166063	-0.0600000000000001\\
-1.89407514289853	-0.0499999999999998\\
-1.89663673132848	-0.04\\
-1.8991738069505	-0.0299999999999998\\
-1.9	-0.0267117516629713\\
-1.90167773331412	-0.02\\
-1.9041530412034	-0.00999999999999979\\
-1.90660396182284	0\\
-1.90903049517243	0.00999999999999979\\
-1.91	0.0140359944623906\\
-1.91142534161217	0.02\\
-1.91379098518853	0.0299999999999998\\
-1.91613236575386	0.04\\
-1.91844948330815	0.0499999999999998\\
-1.92	0.0567623857623858\\
-1.92073857464201	0.0600000000000001\\
-1.92299566577056	0.0699999999999998\\
-1.92522861688704	0.0800000000000001\\
-1.92743742799144	0.0899999999999999\\
-1.92962209908378	0.1\\
-1.93	0.101749111223971\\
-1.93177363887464	0.11\\
-1.93389925434239	0.12\\
-1.93600085155627	0.13\\
-1.93807843051628	0.14\\
-1.94	0.149357256778309\\
-1.94013132882173	0.15\\
-1.94215068597313	0.16\\
-1.94414614540675	0.17\\
-1.94611770712261	0.18\\
-1.94806537112069	0.19\\
-1.94998913740101	0.2\\
-1.95	0.200057175528873\\
-1.95187957328606	0.21\\
-1.9537461765437	0.22\\
-1.95558900141589	0.23\\
-1.95740804790264	0.24\\
-1.95920331600393	0.25\\
-1.96	0.254497254423428\\
-1.96096996225116	0.26\\
-1.96270898981534	0.27\\
-1.96442435714055	0.28\\
-1.96611606422679	0.29\\
-1.96778411107407	0.3\\
-1.96942849768237	0.31\\
-1.97	0.313526211015262\\
-1.97104403659907	0.32\\
-1.97263320669913	0.33\\
-1.97419883353844	0.34\\
-1.97574091711702	0.35\\
-1.97725945743485	0.36\\
-1.97875445449195	0.37\\
-1.98	0.378464727272727\\
-1.9802247968735	0.38\\
-1.98166558403527	0.39\\
-1.98308294376351	0.4\\
-1.98447687605823	0.41\\
-1.98584738091942	0.42\\
-1.98719445834709	0.43\\
-1.98851810834123	0.44\\
-1.98981833090185	0.45\\
-1.99	0.451422852376981\\
-1.99108976464729	0.46\\
-1.99233699625936	0.47\\
-1.99356091513103	0.48\\
-1.99476152126228	0.49\\
-1.99593881465312	0.5\\
-1.99709279530354	0.51\\
-1.99822346321356	0.52\\
-1.99933081838316	0.53\\
-2	0.536173020527859\\
113.701363636364	734\\
2	-0.808348023167968\\
1.99927085996288	-0.81\\
1.99483266458447	-0.82\\
1.99037001633896	-0.83\\
1.99	-0.830824622244241\\
1.98586166445083	-0.84\\
1.98132682360039	-0.85\\
1.98	-0.852910071061015\\
1.97675063170307	-0.86\\
1.97214284912123	-0.87\\
1.97	-0.874625696969697\\
1.9674973753429	-0.88\\
1.96281589018209	-0.89\\
1.96	-0.895983209402735\\
1.95809965842421	-0.9\\
1.95334369786997	-0.91\\
1.95	-0.916993828625682\\
1.94855519684231	-0.92\\
1.94372397586101	-0.93\\
1.94	-0.937668311017149\\
1.93886165808457	-0.94\\
1.93395437916442	-0.95\\
1.93	-0.958016972797024\\
1.9290166599412	-0.96\\
1.92403251282643	-0.97\\
1.92	-0.978049712313004\\
1.9190177691746	-0.98\\
1.91395593059242	-0.99\\
1.91	-0.997776030986558\\
1.90886250014567	-1\\
1.90372213352601	-1.01\\
1.9	-1.01720505301151\\
1.89854831339559	-1.02\\
1.89332856858311	-1.03\\
1.89	-1.036345543891\\
1.88807261418121	-1.04\\
1.88277262713961	-1.05\\
1.88	-1.05520592789206\\
1.87743275096239	-1.06\\
1.87205164347054	-1.07\\
1.87	-1.07379430449069\\
1.86662601383943	-1.08\\
1.86116289317914	-1.09\\
1.86	-1.09211846387503\\
1.85564963293853	-1.1\\
1.85010359157375	-1.11\\
1.85	-1.11018590156888\\
1.84450077674346	-1.12\\
1.84	-1.12799443850267\\
1.83886457488163	-1.13\\
1.83317655037117	-1.14\\
1.83	-1.14555859292223\\
1.82744760397004	-1.15\\
1.8216739937892	-1.16\\
1.82	-1.16288599622087\\
1.8158502443266	-1.17\\
1.81	-1.17998307210031\\
1.80999002352478	-1.18\\
1.80406941656095	-1.19\\
1.8	-1.19684199627252\\
1.79811096384199	-1.2\\
1.79210197079117	-1.21\\
1.79	-1.21348224912785\\
1.78604308121644	-1.22\\
1.78	-1.22990929519918\\
1.7799443643882	-1.23\\
1.77378309629722	-1.24\\
1.77	-1.24611277586556\\
1.76758025282634	-1.25\\
1.76132765335316	-1.26\\
1.76	-1.2621139875577\\
1.75501831873328	-1.27\\
1.75	-1.27790909090909\\
1.74866550198294	-1.28\\
1.74225506573662	-1.29\\
1.74	-1.29350247573777\\
1.73579178264941	-1.3\\
1.73	-1.30890376651548\\
1.72928266339764	-1.31\\
1.72271062649203	-1.32\\
1.72	-1.32410674486804\\
1.71608687186011	-1.33\\
1.71	-1.3391277009928\\
1.7094147915007	-1.34\\
1.7026772537906	-1.35\\
1.7	-1.35395676510326\\
1.69588616326584	-1.36\\
1.69	-1.36861022487027\\
1.68904411278702	-1.37\\
1.68213703121076	-1.38\\
1.68	-1.3830809986659\\
1.6751713772694	-1.39\\
1.67	-1.39737919908901\\
1.66815197462629	-1.4\\
1.66107115613667	-1.41\\
1.66	-1.41150649350649\\
1.65392332449065	-1.42\\
1.65	-1.42546110590441\\
1.64671879022751	-1.43\\
1.64	-1.43925592682472\\
1.63945648172137	-1.44\\
1.63212184845303	-1.45\\
1.63	-1.4528811331235\\
1.62472397941681	-1.46\\
1.62	-1.46634814678222\\
1.61726513195145	-1.47\\
1.61	-1.47966189164371\\
1.60974413142277	-1.48\\
1.60214590485191	-1.49\\
1.6	-1.49281289852432\\
1.59448166850844	-1.5\\
1.59	-1.5058136454364\\
1.5867517986118	-1.51\\
1.58	-1.5186672691126\\
1.57895500602282	-1.52\\
1.57108283143201	-1.53\\
1.57	-1.53137009144701\\
1.56313453061923	-1.54\\
1.56	-1.54392462939811\\
1.55511552102928	-1.55\\
1.55	-1.55633784024195\\
1.5470243860131	-1.56\\
1.54	-1.56861137692717\\
1.53885967082381	-1.57\\
1.53061571372441	-1.58\\
1.53	-1.58074397749253\\
1.522287890997	-1.59\\
1.52	-1.59273533018042\\
1.5138822592303	-1.6\\
1.51	-1.60459239225266\\
1.50539721673523	-1.61\\
1.5	-1.61631670432818\\
1.49683111755821	-1.62\\
1.49	-1.62790978354978\\
1.48818226994142	-1.63\\
1.48	-1.63937312402967\\
1.47944893471833	-1.64\\
1.47062491470165	-1.65\\
1.47	-1.65070552987502\\
1.46170945130797	-1.66\\
1.46	-1.66190926147024\\
1.45270451085034	-1.67\\
1.45	-1.67298810535259\\
1.44360817236946	-1.68\\
1.44	-1.68394345691552\\
1.43441845952229	-1.69\\
1.43	-1.69477669084163\\
1.42513333857154	-1.7\\
1.42	-1.70548916148546\\
1.41575071628702	-1.71\\
1.41	-1.71608220324795\\
1.40626843775414	-1.72\\
1.4	-1.72655713094245\\
1.39668428408489	-1.73\\
1.39	-1.7369152401529\\
1.38699597002606	-1.74\\
1.38	-1.74715780758404\\
1.37720114145944	-1.75\\
1.37	-1.75728609140406\\
1.36729737278807	-1.76\\
1.36	-1.76730133157981\\
1.35728216420258	-1.77\\
1.35	-1.77720475020475\\
1.34715293882113	-1.78\\
1.34	-1.78699755181981\\
1.33690703969602	-1.79\\
1.33	-1.79668092372744\\
1.32654172667966	-1.8\\
1.32	-1.80625603629882\\
1.31605417314228	-1.81\\
1.31	-1.81572404327467\\
1.30544146253292	-1.82\\
1.3	-1.82508608205953\\
1.29470058477503	-1.83\\
1.29	-1.83434327400994\\
1.28382843248727	-1.84\\
1.28	-1.84349672471641\\
1.27282179701966	-1.85\\
1.27	-1.85254752427957\\
1.26167736429422	-1.86\\
1.26	-1.86149674758052\\
1.25039171043904	-1.87\\
1.25	-1.87034545454545\\
1.24	-1.87909154150198\\
1.2389525493556	-1.88\\
1.23	-1.88773767134079\\
1.22736023531575	-1.89\\
1.22	-1.8962860731669\\
1.21561418797397	-1.9\\
1.21	-1.90473775621968\\
1.20371042247199	-1.91\\
1.2	-1.9130937158896\\
1.19164481503141	-1.92\\
1.19	-1.92135493395493\\
1.18	-1.92952074592075\\
1.17940789094749	-1.93\\
1.17	-1.93758881833669\\
1.16698409623617	-1.94\\
1.16	-1.94556474764624\\
1.15438322159457	-1.95\\
1.15	-1.95344946931241\\
1.14160042604391	-1.96\\
1.14	-1.96124390617814\\
1.13	-1.96894542388472\\
1.12861798091411	-1.97\\
1.12	-1.97655401069519\\
1.11542635816113	-1.98\\
1.11	-1.98407476777829\\
1.10203475875622	-1.99\\
1.1	-1.99150857489756\\
1.09	-1.99885248140841\\
1.08842239238842	-2\\
1.08	-2.00610588413251\\
1.07457621405246	-2.01\\
1.07	-2.01327468717021\\
1.06050932919175	-2.02\\
1.06	-2.02035972952667\\
1.05	-2.02735311795642\\
1.04617721738753	-2.03\\
1.04	-2.03426299236184\\
1.03160269197887	-2.04\\
1.03	-2.04109135691894\\
1.02	-2.04783191521123\\
1.01675027409214	-2.05\\
1.01	-2.0544887665526\\
1.00162555673623	-2.06\\
1	-2.06106629096841\\
0.99	-2.06755731870087\\
0.986196688757015	-2.07\\
0.98	-2.07396688839616\\
0.970474034309721	-2.08\\
0.97	-2.08029924856343\\
0.96	-2.08654398114042\\
0.954404713628015	-2.09\\
0.95	-2.09271185196064\\
0.94	-2.09880070494933\\
0.938008146937581	-2.1\\
0.93	-2.10480617771922\\
0.921247317133342	-2.11\\
0.92	-2.11073777907486\\
0.91	-2.11658529111338\\
0.904092297594103	-2.12\\
0.9	-2.12235781818182\\
0.89	-2.12805221818182\\
0.88653863054773	-2.13\\
0.88	-2.1336676816007\\
0.87	-2.13920994635349\\
0.868557053200223	-2.14\\
0.86	-2.14467032808209\\
0.85011393796413	-2.15\\
0.85	-2.15006123037026\\
0.84	-2.15536867886472\\
0.831163097561646	-2.16\\
0.83	-2.16060764038489\\
0.82	-2.16576561826799\\
0.811683637491891	-2.17\\
0.81	-2.17085454545455\\
0.8	-2.17586399427344\\
0.791633119986099	-2.18\\
0.79	-2.18080476666191\\
0.78	-2.18566661909519\\
0.770964439815504	-2.19\\
0.77	-2.19046108977095\\
0.76	-2.19517626973965\\
0.75	-2.19982572200882\\
0.749619789565164	-2.2\\
0.74	-2.20439568855482\\
0.73	-2.20889958871082\\
0.727520690206103	-2.21\\
0.72	-2.21332758376926\\
0.71	-2.21768683726849\\
0.704612953146095	-2.22\\
0.7	-2.22197463002114\\
0.69	-2.22619013389711\\
0.680820457092404	-2.23\\
0.68	-2.23033946887734\\
0.67	-2.23441211184488\\
0.66	-2.23841983982015\\
0.655992302483875	-2.24\\
0.65	-2.24235537190083\\
0.64	-2.24622131951254\\
0.63005932859196	-2.25\\
0.63	-2.25002248289345\\
0.62	-2.25374752129591\\
0.61	-2.25740804356933\\
0.602792124577687	-2.26\\
0.6	-2.26100097452318\\
0.59	-2.26452164833635\\
0.58	-2.26797800361966\\
0.574038990355019	-2.27\\
0.57	-2.27136585704372\\
0.56	-2.27468341429563\\
0.55	-2.27793684941013\\
0.543531050089211	-2.28\\
0.54	-2.28112273419123\\
0.53	-2.28423841151238\\
0.52	-2.2872901618929\\
0.510930394109203	-2.29\\
0.51	-2.29027714167471\\
0.5	-2.29319216443647\\
0.49	-2.29604345426955\\
0.48	-2.29883101117396\\
0.475708280577363	-2.3\\
0.47	-2.30155013065603\\
0.46	-2.30420217301609\\
0.45	-2.30679067528538\\
0.44	-2.3093156374639\\
0.437219645750685	-2.31\\
0.43	-2.31177169888935\\
0.42	-2.31416234745647\\
0.41	-2.31648964760729\\
0.4	-2.31875359934183\\
0.394336095706899	-2.32\\
0.39	-2.32095133287765\\
0.38	-2.32308215994532\\
0.37	-2.32514982911825\\
0.36	-2.32715434039645\\
0.35	-2.3290956937799\\
0.345185239100372	-2.33\\
0.34	-2.33097096906092\\
0.33	-2.33278056426332\\
0.32	-2.33452719094998\\
0.31	-2.33621084912089\\
0.3	-2.33783153877607\\
0.29	-2.3393892599155\\
0.285914105954229	-2.34\\
0.28	-2.34088136975133\\
0.27	-2.34230887348825\\
0.26	-2.34367359695611\\
0.25	-2.34497554015491\\
0.24	-2.34621470308466\\
0.23	-2.34739108574535\\
0.22	-2.34850468813698\\
0.21	-2.34955551025955\\
0.205501306560309	-2.35\\
0.2	-2.35054193198754\\
0.19	-2.35146443571332\\
0.18	-2.35232434629454\\
0.17	-2.3531216637312\\
0.16	-2.3538563880233\\
0.15	-2.35452851917084\\
0.14	-2.35513805717382\\
0.13	-2.35568500203225\\
0.12	-2.35616935374611\\
0.11	-2.3565911123154\\
0.1	-2.35695027774014\\
0.0899999999999999	-2.35724685002032\\
0.0800000000000001	-2.35748082915594\\
0.0699999999999998	-2.357652215147\\
0.0600000000000001	-2.3577610079935\\
0.0499999999999998	-2.35780720769543\\
0.04	-2.35779081425281\\
0.0299999999999998	-2.35771182766563\\
0.02	-2.35757024793388\\
0.00999999999999979	-2.35736607505758\\
0	-2.35709930903672\\
-0.01	-2.35676994987129\\
-0.02	-2.35637799756131\\
-0.03	-2.35592345210676\\
-0.04	-2.35540631350765\\
-0.05	-2.35482658176399\\
-0.0600000000000001	-2.35418425687576\\
-0.0700000000000001	-2.35347933884298\\
-0.0800000000000001	-2.35271182766563\\
-0.0900000000000001	-2.35188172334372\\
-0.1	-2.35098902587725\\
-0.11	-2.35003373526622\\
-0.110331425529083	-2.35\\
-0.12	-2.34901290936269\\
-0.13	-2.34792920233727\\
-0.14	-2.3467827150428\\
-0.15	-2.34557344747928\\
-0.16	-2.34430139964669\\
-0.17	-2.34296657154505\\
-0.18	-2.34156896317434\\
-0.19	-2.34010857453458\\
-0.190712820055315	-2.34\\
-0.2	-2.33858116396347\\
-0.21	-2.3369904593158\\
-0.22	-2.33533678615238\\
-0.23	-2.33362014447322\\
-0.24	-2.33184053427832\\
-0.249988904504772	-2.33\\
-0.25	-2.329997949419\\
-0.26	-2.32808667122351\\
-0.27	-2.32611223513329\\
-0.28	-2.32407464114833\\
-0.29	-2.32197388926863\\
-0.299121864931456	-2.32\\
-0.3	-2.31980940627999\\
-0.31	-2.31757562045797\\
-0.32	-2.31527848621966\\
-0.33	-2.31291800356506\\
-0.34	-2.31049417249417\\
-0.341986879100282	-2.31\\
-0.35	-2.30800096272865\\
-0.36	-2.30544271764544\\
-0.37	-2.30282093247146\\
-0.38	-2.30013560720671\\
-0.380493320658429	-2.3\\
-0.39	-2.29737881087047\\
-0.4	-2.29455787005104\\
-0.41	-2.29167319630294\\
-0.415674916951294	-2.29\\
-0.42	-2.28872090770721\\
-0.43	-2.285699598727\\
-0.44	-2.28261436280614\\
-0.448301770728064	-2.28\\
-0.45	-2.27946356696738\\
-0.46	-2.27624066620403\\
-0.47	-2.27295364330326\\
-0.478813833091739	-2.27\\
-0.48	-2.26960128080189\\
-0.49	-2.26617555338995\\
-0.5	-2.26268550744814\\
-0.507555520739495	-2.26\\
-0.51	-2.25912847367686\\
-0.52	-2.25549867336964\\
-0.53	-2.25180435693339\\
-0.53480031207044	-2.25\\
-0.54	-2.24803950133072\\
-0.55	-2.24420437035999\\
-0.56	-2.2403045244432\\
-0.560768116454087	-2.24\\
-0.57	-2.23632864971196\\
-0.58	-2.2322869186455\\
-0.585568823348274	-2.23\\
-0.59	-2.22817463002114\\
-0.6	-2.22399013389711\\
-0.60938940665318	-2.22\\
-0.61	-2.2197397144069\\
-0.62	-2.21541156510674\\
-0.63	-2.21101809698855\\
-0.632283349716207	-2.21\\
-0.64	-2.20654857467026\\
-0.65	-2.2020103531414\\
-0.654366778595854	-2.2\\
-0.66	-2.19739849196187\\
-0.67	-2.19271461089771\\
-0.675715441066348	-2.19\\
-0.68	-2.18795861281576\\
-0.69	-2.18312815755673\\
-0.696388702672769	-2.18\\
-0.7	-2.17822619899785\\
-0.71	-2.17324824624195\\
-0.716439701416286	-2.17\\
-0.72	-2.16819847766767\\
-0.73	-2.1630720953612\\
-0.735916143485356	-2.16\\
-0.74	-2.15787264082985\\
-0.75	-2.15259688805648\\
-0.754860980016721	-2.15\\
-0.76	-2.14724584477526\\
-0.77	-2.1418197716433\\
-0.773312985133535	-2.14\\
-0.78	-2.13631520951138\\
-0.79	-2.13073785703929\\
-0.791307251663286	-2.13\\
-0.8	-2.12507781818182\\
-0.80886243050443	-2.12\\
-0.81	-2.11934612578433\\
-0.82	-2.11353071647454\\
-0.82600173632643	-2.11\\
-0.83	-2.10764031620553\\
-0.84	-2.10167091201874\\
-0.842767768374597	-2.1\\
-0.85	-2.09561976795418\\
-0.859176038943911	-2.09\\
-0.86	-2.08949373802858\\
-0.87	-2.08328142036246\\
-0.875224868035191	-2.08\\
-0.88	-2.07699127864006\\
-0.89	-2.070622172949\\
-0.890966497508554	-2.07\\
-0.9	-2.064165801572\\
-0.90638249528527	-2.06\\
-0.91	-2.05763115607796\\
-0.92	-2.05101413480137\\
-0.921516857683321	-2.05\\
-0.93	-2.04430974772354\\
-0.936359671343922	-2.04\\
-0.94	-2.03752493634866\\
-0.95	-2.0306567320653\\
-0.950946654864964	-2.03\\
-0.96	-2.02369857250188\\
-0.96526131287006	-2.02\\
-0.97	-2.01665777174732\\
-0.979347232511378	-2.01\\
-0.98	-2.00953350476479\\
-0.99	-2.00231719860838\\
-0.993180261163819	-2\\
-1	-1.99501441796934\\
-1.00679402027513	-1.99\\
-1.01	-1.98762585655551\\
-1.02	-1.98015014466271\\
-1.02019897084048	-1.98\\
-1.03	-1.97257922077922\\
-1.03337505248216	-1.97\\
-1.04	-1.9649201287751\\
-1.04635789140038	-1.96\\
-1.05	-1.95717197354253\\
-1.05915274533284	-1.95\\
-1.06	-1.94933384781602\\
-1.07	-1.94140006173792\\
-1.07174896365564	-1.94\\
-1.08	-1.93337246399257\\
-1.08416152273205	-1.93\\
-1.09	-1.92525236985237\\
-1.096402659645	-1.92\\
-1.1	-1.91703882738188\\
-1.10847683611575	-1.91\\
-1.11	-1.9087308715381\\
-1.12	-1.90032639649507\\
-1.12038504845408	-1.9\\
-1.13	-1.8918214790391\\
-1.13212328641763	-1.89\\
-1.14	-1.88321947376713\\
-1.14370859725222	-1.88\\
-1.15	-1.87451936758893\\
-1.15514479581001	-1.87\\
-1.16	-1.86572013326987\\
-1.16643557110472	-1.86\\
-1.17	-1.85682072918325\\
-1.17758449145791	-1.85\\
-1.18	-1.84782009905736\\
-1.1885950093947	-1.84\\
-1.19	-1.83871717171717\\
-1.19947046630319	-1.83\\
-1.2	-1.8295108608206\\
-1.21	-1.82019935639582\\
-1.21021240121372	-1.82\\
-1.22	-1.81078088164056\\
-1.22082251892168	-1.81\\
-1.23	-1.80125603629882\\
-1.23130803111869	-1.8\\
-1.24	-1.79162367864693\\
-1.24167182973593	-1.79\\
-1.25	-1.78188265056308\\
-1.251916717901	-1.78\\
-1.26	-1.77203177723178\\
-1.26204541332058	-1.77\\
-1.27	-1.76206986684202\\
-1.2720605515097	-1.76\\
-1.28	-1.75199571027883\\
-1.28196468887553	-1.75\\
-1.29	-1.74180808080808\\
-1.29176030566348	-1.74\\
-1.3	-1.73150573375436\\
-1.30144980877246	-1.73\\
-1.31	-1.72108740617181\\
-1.31103553444633	-1.72\\
-1.32	-1.71055181650762\\
-1.32051975084759	-1.71\\
-1.32990396593866	-1.7\\
-1.33	-1.69989728453365\\
-1.33918656167322	-1.69\\
-1.34	-1.68912036566785\\
-1.3483734593326	-1.68\\
-1.35	-1.6782217502124\\
-1.35746670576071	-1.67\\
-1.36	-1.66720006822446\\
-1.36646828984264	-1.66\\
-1.37	-1.65605392912173\\
-1.37538014454165	-1.65\\
-1.38	-1.64478192129232\\
-1.38420414885098	-1.64\\
-1.39	-1.6333826116957\\
-1.39294212966436	-1.63\\
-1.4	-1.62185454545455\\
-1.40159586356931	-1.62\\
-1.41	-1.61019624543716\\
-1.41016707856688	-1.61\\
-1.41864827667855	-1.6\\
-1.42	-1.59840007006481\\
-1.427048724921	-1.59\\
-1.43	-1.58646931598382\\
-1.43537118413945	-1.58\\
-1.44	-1.57440317740512\\
-1.44361722182746	-1.57\\
-1.45	-1.56220007088428\\
-1.4517883637516	-1.56\\
-1.45988533728987	-1.55\\
-1.46	-1.54985783175567\\
-1.46789805818296	-1.54\\
-1.47	-1.53736614667384\\
-1.47583993176487	-1.53\\
-1.48	-1.52473195319532\\
-1.48371234482466	-1.52\\
-1.49	-1.51195355141876\\
-1.49151664819206	-1.51\\
-1.49924932298755	-1.5\\
-1.5	-1.49902532337402\\
-1.5069063584782	-1.49\\
-1.51	-1.48594091823669\\
-1.51449891263696	-1.48\\
-1.52	-1.47270633608815\\
-1.52202821553919	-1.47\\
-1.52949225793132	-1.46\\
-1.53	-1.45931697833734\\
-1.53688225720421	-1.45\\
-1.54	-1.44576203755345\\
-1.54421234319575	-1.44\\
-1.55	-1.43205058801568\\
-1.55148363744412	-1.43\\
-1.55868910468524	-1.42\\
-1.56	-1.41817316017316\\
-1.56582849012791	-1.41\\
-1.57	-1.40412606312606\\
-1.57291215900754	-1.4\\
-1.57994077236044	-1.39\\
-1.58	-1.38991538021727\\
-1.58689753006218	-1.38\\
-1.59	-1.37552133971292\\
-1.59380147320429	-1.37\\
-1.6	-1.36095906207957\\
-1.60065356052966	-1.36\\
-1.60743932626947	-1.35\\
-1.61	-1.34621031207598\\
-1.61417103875646	-1.34\\
-1.62	-1.33128401790929\\
-1.62085357489486	-1.33\\
-1.62747279197671	-1.32\\
-1.63	-1.31616552130375\\
-1.63403938922264	-1.31\\
-1.64	-1.30086255176494\\
-1.64055934219108	-1.3\\
-1.64701594649548	-1.29\\
-1.65	-1.28535806644122\\
-1.65342418733553	-1.28\\
-1.65978693380455	-1.27\\
-1.66	-1.26966365643187\\
-1.66608596741782	-1.26\\
-1.67	-1.25375750856682\\
-1.67234227555444	-1.25\\
-1.67854821888547	-1.24\\
-1.68	-1.23765019320724\\
-1.68469923919697	-1.23\\
-1.69	-1.2213317671093\\
-1.69080972313654	-1.22\\
-1.69686246258182	-1.21\\
-1.7	-1.2047928262214\\
-1.7028713984019	-1.2\\
-1.70883526200739	-1.19\\
-1.71	-1.18803806948201\\
-1.71474488229588	-1.18\\
-1.72	-1.1710566353187\\
-1.72061737590818	-1.17\\
-1.72643335978998	-1.16\\
-1.73	-1.15383906348872\\
-1.73221001250653	-1.15\\
-1.73793994438846	-1.14\\
-1.74	-1.13638790717479\\
-1.74362284016953	-1.13\\
-1.74926768012171	-1.12\\
-1.75	-1.11869653986675\\
-1.75485885135865	-1.11\\
-1.76	-1.10075448067372\\
-1.76041722890271	-1.1\\
-1.76592097249919	-1.09\\
-1.77	-1.08255308480488\\
-1.77139077251945	-1.08\\
-1.77681206579182	-1.07\\
-1.78	-1.06409112920977\\
-1.78219518820792	-1.06\\
-1.78753493096484	-1.05\\
-1.79	-1.04536096910425\\
-1.79283322956924	-1.04\\
-1.79809230696942	-1.03\\
-1.8	-1.02635465768799\\
-1.80330759075136	-1.02\\
-1.80848687361989	-1.01\\
-1.81	-1.00706393108139\\
-1.81362090804504	-1\\
-1.81872125318118	-0.99\\
-1.82	-0.987480192351729\\
-1.82377576142865	-0.98\\
-1.82879801190545	-0.97\\
-1.83	-0.96759449456396\\
-1.83377467606379	-0.96\\
-1.83871966151971	-0.95\\
-1.84	-0.947397522785698\\
-1.84362012374341	-0.94\\
-1.84848866066629	-0.93\\
-1.85	-0.926879574970485\\
-1.85331452429438	-0.92\\
-1.85810741629788	-0.91\\
-1.86	-0.906030541636842\\
-1.86286024693593	-0.9\\
-1.8675782850288	-0.89\\
-1.87	-0.884839884253677\\
-1.87225961159583	-0.88\\
-1.87690357444413	-0.87\\
-1.88	-0.863296612234951\\
-1.88151489018558	-0.86\\
-1.88608554436816	-0.85\\
-1.89	-0.841389258438039\\
-1.8906283078363	-0.84\\
-1.89512640809368	-0.83\\
-1.89959999554313	-0.82\\
-1.9	-0.819100926621588\\
-1.90402833357351	-0.81\\
-1.90843023578056	-0.8\\
-1.91	-0.796414028868068\\
-1.91279344457556	-0.79\\
-1.91712439204614	-0.78\\
-1.92	-0.773322919334187\\
-1.92142382180282	-0.77\\
-1.92568453393318	-0.76\\
-1.92992110605146	-0.75\\
-1.93	-0.749812711643658\\
-1.93411268928022	-0.74\\
-1.93827985632717	-0.73\\
-1.94	-0.725848221343874\\
-1.94241084521883	-0.72\\
-1.94650930381603	-0.71\\
-1.95	-0.70143295121301\\
-1.95058094918991	-0.7\\
-1.95461138552329	-0.69\\
-1.95861804347121	-0.68\\
-1.96	-0.6765302578019\\
-1.96258799995698	-0.67\\
-1.96652742974522	-0.66\\
-1.97	-0.65113183191431\\
-1.97044100807962	-0.65\\
-1.97431387447161	-0.64\\
-1.97816319760287	-0.63\\
-1.98	-0.625198881118882\\
-1.9819791921795	-0.62\\
-1.98576272269373	-0.61\\
-1.98952282577444	-0.6\\
-1.99	-0.59872299800513\\
-1.99324354396041	-0.59\\
-1.99693861331581	-0.58\\
-2	-0.571662337662338\\
136.251136363636	683\\
2	-1.35055491217912\\
1.99456090431149	-1.36\\
1.99	-1.36788660388238\\
1.98877149273241	-1.37\\
1.98293396047237	-1.38\\
1.98	-1.3850049552125\\
1.97705665674659	-1.39\\
1.9711394238868	-1.4\\
1.97	-1.4019175959176\\
1.96517243714934	-1.41\\
1.96	-1.41862394127612\\
1.95917034532064	-1.42\\
1.95311616981196	-1.43\\
1.95	-1.43512600336009\\
1.94702141251897	-1.44\\
1.94088513444142	-1.45\\
1.94	-1.45143658581744\\
1.93469580375472	-1.46\\
1.93	-1.46755043333948\\
1.92846843834669	-1.47\\
1.92219069579754	-1.48\\
1.92	-1.48347558075727\\
1.91586556628377	-1.49\\
1.91	-1.49921916560394\\
1.90950053023505	-1.5\\
1.90307827667781	-1.51\\
1.9	-1.51477408277607\\
1.89661214542643	-1.52\\
1.89010357473433	-1.53\\
1.89	-1.53015850815851\\
1.88353422785586	-1.54\\
1.88	-1.54535827826398\\
1.87692164643174	-1.55\\
1.87026366597572	-1.56\\
1.87	-1.56039447102605\\
1.86354429927468	-1.57\\
1.86	-1.57525313327449\\
1.85677955952623	-1.58\\
1.85	-1.58995428169509\\
1.849968689411	-1.59\\
1.84309360661858	-1.6\\
1.84	-1.6044824637934\\
1.83617072550468	-1.61\\
1.83	-1.61885729184773\\
1.82919941545394	-1.62\\
1.82216659562808	-1.63\\
1.82	-1.63306900120752\\
1.81507929898473	-1.64\\
1.81	-1.64712734146761\\
1.80794115111281	-1.65\\
1.80074700397827	-1.66\\
1.8	-1.66103445335153\\
1.79348870913798	-1.67\\
1.79	-1.67478572642311\\
1.78617700923169	-1.68\\
1.78	-1.6883936008126\\
1.77881097675584	-1.69\\
1.77138161769313	-1.7\\
1.77	-1.7018527978491\\
1.76388931600771	-1.71\\
1.76	-1.71516641553658\\
1.75633992976762	-1.72\\
1.75	-1.72834195162636\\
1.74873245004399	-1.73\\
1.74105956464632	-1.74\\
1.74	-1.74137589004802\\
1.73331926574907	-1.75\\
1.73	-1.75426926249794\\
1.72551790438093	-1.76\\
1.72	-1.7670295906625\\
1.71765438188828	-1.77\\
1.71	-1.77965831285831\\
1.70972757310209	-1.78\\
1.70172584921184	-1.79\\
1.7	-1.79214912993983\\
1.69365712333627	-1.8\\
1.69	-1.80451045211473\\
1.6855218316655	-1.81\\
1.68	-1.81674487324399\\
1.67731874709051	-1.82\\
1.67	-1.82885374094932\\
1.66904661215422	-1.83\\
1.66069978453756	-1.84\\
1.66	-1.84083543697076\\
1.65227575712689	-1.85\\
1.65	-1.8526906543544\\
1.64377904849659	-1.86\\
1.64	-1.86442471838807\\
1.63520828504029	-1.87\\
1.63	-1.87603889328063\\
1.62656205831904	-1.88\\
1.62	-1.88753442571293\\
1.61783892395012	-1.89\\
1.61	-1.89891254514052\\
1.60903740045309	-1.9\\
1.60015496657447	-1.91\\
1.6	-1.91017386558553\\
1.59118535718782	-1.92\\
1.59	-1.92131686091686\\
1.58213313247075	-1.93\\
1.58	-1.93234644571783\\
1.57299666973712	-1.94\\
1.57	-1.94326377527396\\
1.56377430352481	-1.95\\
1.56	-1.95406998923243\\
1.55446432417685	-1.96\\
1.55	-1.96476621186571\\
1.54506497636566	-1.97\\
1.54	-1.97535355233002\\
1.53557445755781	-1.98\\
1.53	-1.98583310491853\\
1.52599091641638	-1.99\\
1.52	-1.99620594930946\\
1.51631245113802	-2\\
1.51	-2.00647315080926\\
1.50653710772156	-2.01\\
1.5	-2.01663576059098\\
1.49666287816477	-2.02\\
1.49	-2.02669481592787\\
1.48668769858599	-2.03\\
1.48	-2.03665134042235\\
1.47660944726666	-2.04\\
1.47	-2.04650634423048\\
1.46642594261106	-2.05\\
1.46	-2.0562608242821\\
1.45613494101906	-2.06\\
1.45	-2.06591576449651\\
1.44573413466751	-2.07\\
1.44	-2.07547213599409\\
1.43522114919575	-2.08\\
1.43	-2.08493089730367\\
1.42459354129017	-2.09\\
1.42	-2.09429299456602\\
1.41384879616292	-2.1\\
1.41	-2.10355936173327\\
1.40298432491907	-2.11\\
1.4	-2.11273092076463\\
1.39199746180662	-2.12\\
1.39	-2.12180858181818\\
1.380885461343	-2.13\\
1.38	-2.13079324343918\\
1.37	-2.13968479048862\\
1.36964280433104	-2.14\\
1.36	-2.14848229512935\\
1.35826145262496	-2.15\\
1.35	-2.15718916582625\\
1.34674502410784	-2.16\\
1.34	-2.16580626166882\\
1.33509037106781	-2.17\\
1.33	-2.17433443092341\\
1.32329424722528	-2.18\\
1.32	-2.18277451120308\\
1.31135330384438	-2.19\\
1.31	-2.19112732963437\\
1.3	-2.19939180537772\\
1.29925818598275	-2.2\\
1.29	-2.20756686994753\\
1.28699896793605	-2.21\\
1.28	-2.2156568641312\\
1.27458266466802	-2.22\\
1.27	-2.22366257928118\\
1.26200540511708	-2.23\\
1.26	-2.23158479696501\\
1.25	-2.23942250948433\\
1.24925703646125	-2.24\\
1.24	-2.24717313349209\\
1.23632116229469	-2.25\\
1.23	-2.25484234045524\\
1.22321003768729	-2.26\\
1.22	-2.26243087846304\\
1.21	-2.26993930112766\\
1.2099184604739	-2.27\\
1.2	-2.27736086051353\\
1.19641287328567	-2.28\\
1.19	-2.28470374982704\\
1.18271585437799	-2.29\\
1.18	-2.29196868533591\\
1.17	-2.29915381431922\\
1.1688117699475	-2.3\\
1.16	-2.30625622335305\\
1.15467934559528	-2.31\\
1.15	-2.31328259975319\\
1.14033708751652	-2.32\\
1.14	-2.32023362952837\\
1.13	-2.3271012987013\\
1.12574003013561	-2.33\\
1.12	-2.33389409840534\\
1.11091537728388	-2.34\\
1.11	-2.34061339855959\\
1.1	-2.34725166462835\\
1.09582033106698	-2.35\\
1.09	-2.35381574312424\\
1.08047593214681	-2.36\\
1.08	-2.36030811833041\\
1.07	-2.36671970822639\\
1.0648335212646	-2.37\\
1.06	-2.3730597979798\\
1.05	-2.37932794612795\\
1.04891707719352	-2.38\\
1.04	-2.3855175238351\\
1.03268233334064	-2.39\\
1.03	-2.39163823805061\\
1.02	-2.39768389342616\\
1.01612937148994	-2.4\\
1.01	-2.4036569216393\\
1	-2.40956147376852\\
0.99924946880212	-2.41\\
0.99	-2.41538852655397\\
0.981999907636179	-2.42\\
0.98	-2.42114943596549\\
0.97	-2.42683556735236\\
0.964374159450723	-2.43\\
0.96	-2.43245322217811\\
0.95	-2.43800052931057\\
0.946355435490484	-2.44\\
0.94	-2.44347658002375\\
0.93	-2.44888586884813\\
0.927916861971135	-2.45\\
0.92	-2.45422194448099\\
0.91	-2.45949401394553\\
0.909029057585014	-2.46\\
0.9	-2.464691722419\\
0.89	-2.46982736455464\\
0.889659834052783	-2.47\\
0.88	-2.47488829300196\\
0.87	-2.47988829300196\\
0.869773852712973	-2.48\\
0.86	-2.48481400808661\\
0.85	-2.48967914438503\\
0.849332229430766	-2.49\\
0.84	-2.49447119261282\\
0.83	-2.49920223696189\\
0.828292078518725	-2.5\\
0.82	-2.50386214498768\\
0.81	-2.50845986253404\\
0.806605984395987	-2.51\\
0.8	-2.51298913746282\\
0.79	-2.51745428682271\\
0.784221387266269	-2.52\\
0.78	-2.52185441650548\\
0.77	-2.52618774983881\\
0.761079866035059	-2.53\\
0.76	-2.53046020316317\\
0.75	-2.53466246624662\\
0.74	-2.53880532338948\\
0.737074345813522	-2.54\\
0.73	-2.54288062572125\\
0.72	-2.54689331965637\\
0.712141860993092	-2.55\\
0.71	-2.55084439330009\\
0.7	-2.55472765630993\\
0.69	-2.55855184758982\\
0.686153767786191	-2.56\\
0.68	-2.56231046793319\\
0.67	-2.5660061201071\\
0.66	-2.56964286625016\\
0.659001817469086	-2.57\\
0.65	-2.57321169739352\\
0.64	-2.57672078830261\\
0.630496001768802	-2.58\\
0.63	-2.5801706605807\\
0.62	-2.58355280841892\\
0.61	-2.58687637885127\\
0.600432992893481	-2.59\\
0.6	-2.59014097863194\\
0.59	-2.59333847515489\\
0.58	-2.5964775572133\\
0.57	-2.59955822480718\\
0.568538258795967	-2.6\\
0.56	-2.602573319884\\
0.55	-2.60552893708234\\
0.54	-2.60842630185349\\
0.53445707687525	-2.61\\
0.53	-2.61126191374324\\
0.52	-2.61403508110147\\
0.51	-2.61675015717339\\
0.5	-2.61940714195901\\
0.497718805941264	-2.62\\
0.49	-2.6220005015674\\
0.48	-2.62453429467085\\
0.47	-2.62701015673981\\
0.46	-2.62942808777429\\
0.457576643111418	-2.63\\
0.45	-2.63178316868826\\
0.44	-2.63407890458922\\
0.43	-2.63631686882581\\
0.42	-2.63849706139802\\
0.412918753314086	-2.64\\
0.41	-2.64061778276593\\
0.4	-2.64267676767677\\
0.39	-2.64467813941888\\
0.38	-2.64662189799227\\
0.37	-2.64850804339693\\
0.361840687444589	-2.65\\
0.36	-2.65033565476931\\
0.35	-2.65210172864072\\
0.34	-2.65381034697177\\
0.33	-2.65546150976247\\
0.32	-2.65705521701281\\
0.31	-2.6585914687228\\
0.30047514927256	-2.66\\
0.3	-2.66007007317376\\
0.29	-2.6614875356567\\
0.28	-2.66284769936748\\
0.27	-2.66415056430609\\
0.26	-2.66539613047253\\
0.25	-2.6665843978668\\
0.24	-2.6677153664889\\
0.23	-2.66878903633883\\
0.22	-2.66980540741659\\
0.217971033234191	-2.67\\
0.21	-2.67076239950526\\
0.2	-2.67166171923315\\
0.19	-2.6725038961039\\
0.18	-2.6732889301175\\
0.17	-2.67401682127396\\
0.16	-2.67468756957328\\
0.15	-2.67530117501546\\
0.14	-2.67585763760049\\
0.13	-2.67635695732839\\
0.12	-2.67679913419913\\
0.11	-2.67718416821274\\
0.1	-2.6775120593692\\
0.0899999999999999	-2.67778280766852\\
0.0800000000000001	-2.6779964131107\\
0.0699999999999998	-2.67815287569573\\
0.0600000000000001	-2.67825219542362\\
0.0499999999999998	-2.67829437229437\\
0.04	-2.67827940630798\\
0.0299999999999998	-2.67820729746444\\
0.02	-2.67807804576376\\
0.00999999999999979	-2.67789165120594\\
0	-2.67764811379097\\
-0.01	-2.67734743351886\\
-0.02	-2.67698961038961\\
-0.03	-2.67657464440322\\
-0.04	-2.67610253555968\\
-0.05	-2.675573283859\\
-0.0600000000000001	-2.67498688930118\\
-0.0700000000000001	-2.67434335188621\\
-0.0800000000000001	-2.6736426716141\\
-0.0900000000000001	-2.67288484848485\\
-0.1	-2.67206988249845\\
-0.11	-2.67119777365492\\
-0.12	-2.67026852195424\\
-0.122722257053291	-2.67\\
-0.13	-2.66928016867171\\
-0.14	-2.66823378395138\\
-0.15	-2.66713010045889\\
-0.16	-2.66596911819422\\
-0.17	-2.66475083715739\\
-0.18	-2.66347525734838\\
-0.19	-2.66214237876721\\
-0.2	-2.66075220141387\\
-0.205196641247536	-2.66\\
-0.21	-2.65930282303196\\
-0.22	-2.6577939311031\\
-0.23	-2.65622758363388\\
-0.24	-2.6546037806243\\
-0.25	-2.65292252207437\\
-0.26	-2.65118380798408\\
-0.266590735996677	-2.65\\
-0.27	-2.64938595834892\\
-0.28	-2.64752724778651\\
-0.29	-2.64561092405537\\
-0.3	-2.64363698715551\\
-0.31	-2.64160543708692\\
-0.317684593804094	-2.64\\
-0.32	-2.63951494310366\\
-0.33	-2.63736226084782\\
-0.34	-2.6351518069276\\
-0.35	-2.632883581343\\
-0.36	-2.63055758409403\\
-0.362339086187904	-2.63\\
-0.37	-2.62816877742947\\
-0.38	-2.6257205015674\\
-0.39	-2.62321429467085\\
-0.4	-2.62065015673981\\
-0.402479556214432	-2.62\\
-0.41	-2.61802263296869\\
-0.42	-2.61533521941406\\
-0.43	-2.61258971457312\\
-0.439237117100956	-2.61\\
-0.44	-2.60978552515446\\
-0.45	-2.60691589963435\\
-0.46	-2.60398802168705\\
-0.47	-2.60100189131257\\
-0.473290950507351	-2.6\\
-0.48	-2.59795182703249\\
-0.49	-2.59484056138576\\
-0.5	-2.5916708812745\\
-0.505176060475501	-2.59\\
-0.51	-2.58843844300748\\
-0.52	-2.58514276657791\\
-0.53	-2.58178851274249\\
-0.535240554296541	-2.58\\
-0.54	-2.57837113795296\\
-0.55	-2.57489001907184\\
-0.56	-2.57135015893198\\
-0.563751899091969	-2.57\\
-0.57	-2.56774525054188\\
-0.58	-2.56407764885885\\
-0.59	-2.56035114114497\\
-0.590927616288861	-2.56\\
-0.6	-2.5565560669991\\
-0.61	-2.55270093338448\\
-0.616900336458367	-2.55\\
-0.62	-2.54878330555199\\
-0.63	-2.54479882036159\\
-0.64	-2.54075509680728\\
-0.641840370011562	-2.54\\
-0.65	-2.53664266426643\\
-0.66	-2.53246868972612\\
-0.665831485587582	-2.53\\
-0.67	-2.5282303030303\\
-0.68	-2.5239253384913\\
-0.688993706975507	-2.52\\
-0.69	-2.5195595499806\\
-0.7	-2.51512285012285\\
-0.71	-2.51062640631062\\
-0.711374847444158	-2.51\\
-0.72	-2.50605900661393\\
-0.73	-2.50142990533005\\
-0.733049478662499	-2.5\\
-0.74	-2.49673156457277\\
-0.75	-2.49196904669008\\
-0.754082953534155	-2.49\\
-0.76	-2.48713825485848\\
-0.77	-2.48224155471501\\
-0.774522036574135	-2.48\\
-0.78	-2.4772767822106\\
-0.79	-2.47224512753434\\
-0.794409052377406	-2.47\\
-0.8	-2.4671448248721\\
-0.81	-2.46197743670471\\
-0.81378240032118	-2.46\\
-0.82	-2.45674003420603\\
-0.83	-2.45143612682542\\
-0.832676999288815	-2.45\\
-0.84	-2.44606003430532\\
-0.85	-2.44061881514712\\
-0.851124673269226	-2.44\\
-0.86	-2.43510242159587\\
-0.869145222114176	-2.43\\
-0.87	-2.42952169873922\\
-0.88	-2.42386476443265\\
-0.88675865110126	-2.42\\
-0.89	-2.41814108877945\\
-0.9	-2.41234460268867\\
-0.904002408488786	-2.41\\
-0.91	-2.40647630489921\\
-0.92	-2.4005394473368\\
-0.920899298987426	-2.4\\
-0.93	-2.39452483598875\\
-0.937444544793708	-2.39\\
-0.94	-2.3884421914865\\
-0.95	-2.38228414126494\\
-0.953672200513806	-2.38\\
-0.96	-2.37605239057239\\
-0.969605839728106	-2.37\\
-0.97	-2.36975091179252\\
-0.98	-2.36336903957855\\
-0.985227953969019	-2.36\\
-0.99	-2.35691559409294\\
-1	-2.35038951361604\\
-1.00059118669162	-2.35\\
-1.01	-2.34378230737872\\
-1.01566964740391	-2.34\\
-1.02	-2.33710249420744\\
-1.03	-2.33034837126891\\
-1.03051102625107	-2.33\\
-1.04	-2.32351182501709\\
-1.04508904693041	-2.32\\
-1.05	-2.31660085013026\\
-1.05945014821067	-2.31\\
-1.06	-2.30961477100811\\
-1.07	-2.30254517948013\\
-1.07356810951509	-2.3\\
-1.08	-2.29539812387916\\
-1.0874781661475	-2.29\\
-1.09	-2.28817406946174\\
-1.1	-2.2808696554587\\
-1.10118025952564	-2.28\\
-1.11	-2.27348147120055\\
-1.11467000539907	-2.27\\
-1.12	-2.2660143394125\\
-1.12797434240886	-2.26\\
-1.13	-2.25846753246753\\
-1.14	-2.25083773216031\\
-1.14108876749124	-2.25\\
-1.15	-2.24312228603446\\
-1.15401180684293	-2.24\\
-1.16	-2.23532513699592\\
-1.1667649001303	-2.23\\
-1.17	-2.22744552501762\\
-1.17935221740285	-2.22\\
-1.18	-2.21948268061643\\
-1.19	-2.21143135868797\\
-1.19176348656134	-2.21\\
-1.2	-2.20329385902709\\
-1.20401323633599	-2.2\\
-1.21	-2.1950709916062\\
-1.21611043491677	-2.19\\
-1.22	-2.18676195233338\\
-1.22805867845905	-2.18\\
-1.23	-2.1783659269864\\
-1.23986145096782	-2.17\\
-1.24	-2.16988209105271\\
-1.25	-2.16130547177941\\
-1.25151044349546	-2.16\\
-1.26	-2.15263924506555\\
-1.26302079382616	-2.15\\
-1.27	-2.1438829310594\\
-1.27439669421488	-2.14\\
-1.28	-2.1350356676816\\
-1.28564117140676	-2.13\\
-1.29	-2.12609658181818\\
-1.29675716197263	-2.12\\
-1.3	-2.11706478914344\\
-1.30774751564225	-2.11\\
-1.31	-2.10793939393939\\
-1.31861499849094	-2.1\\
-1.32	-2.09871948891173\\
-1.32936229598675	-2.09\\
-1.33	-2.08940415500221\\
-1.33999201590557	-2.08\\
-1.34	-2.07999246119734\\
-1.35	-2.07048189209165\\
-1.35050307865619	-2.07\\
-1.36	-2.06087335014089\\
-1.3609023489573	-2.06\\
-1.37	-2.05116589793186\\
-1.37119224039559	-2.05\\
-1.38	-2.04135856097925\\
-1.38137504343753	-2.04\\
-1.39	-2.03145035195447\\
-1.39145298504104	-2.03\\
-1.4	-2.02144027047333\\
-1.40142823084144	-2.02\\
-1.41	-2.01132730287954\\
-1.41130288724787	-2.01\\
-1.42	-2.0011104220239\\
-1.4210790034541	-2\\
-1.43	-1.990788587039\\
-1.43075857336818	-1.99\\
-1.44	-1.98036074310949\\
-1.44034353746429	-1.98\\
-1.44983468437767	-1.97\\
-1.45	-1.96982523378813\\
-1.45923207674911	-1.96\\
-1.46	-1.95917997231195\\
-1.46853984516978	-1.95\\
-1.47	-1.94842506559654\\
-1.4777597554908	-1.94\\
-1.48	-1.93755939290692\\
-1.4868935274761	-1.93\\
-1.49	-1.92658181818182\\
-1.49594283629629	-1.92\\
-1.5	-1.91549118977078\\
-1.50490931396467	-1.91\\
-1.51	-1.90428634016586\\
-1.51379455071821	-1.9\\
-1.52	-1.89296608572774\\
-1.52260009634574	-1.89\\
-1.53	-1.88152922640618\\
-1.53132746146587	-1.88\\
-1.53997798049701	-1.87\\
-1.54	-1.86997445660796\\
-1.54854442164204	-1.86\\
-1.55	-1.85829485750677\\
-1.55703664186059	-1.85\\
-1.56	-1.84649448793737\\
-1.56545602190516	-1.84\\
-1.57	-1.8345720699054\\
-1.57380390849063	-1.83\\
-1.58	-1.822526307321\\
-1.58208161533668	-1.82\\
-1.59	-1.81035588567738\\
-1.59029042417215	-1.81\\
-1.59842203744943	-1.8\\
-1.6	-1.79805252886648\\
-1.60648518028579	-1.79\\
-1.61	-1.78562004243512\\
-1.61448283493679	-1.78\\
-1.62	-1.77305831285831\\
-1.62241617599482	-1.77\\
-1.63	-1.76036593785961\\
-1.63028635012928	-1.76\\
-1.63808315216692	-1.75\\
-1.64	-1.74753253849975\\
-1.64581708205987	-1.74\\
-1.65	-1.7345637360811\\
-1.65349092943136	-1.73\\
-1.66	-1.72145938281902\\
-1.66110574674873	-1.72\\
-1.66865475272677	-1.71\\
-1.67	-1.70821139304319\\
-1.67614010177797	-1.7\\
-1.68	-1.69481885646821\\
-1.68356927804915	-1.69\\
-1.69	-1.68128559336381\\
-1.6909432485809	-1.68\\
-1.69825299034891	-1.67\\
-1.7	-1.66760105747911\\
-1.70550382235615	-1.66\\
-1.71	-1.65376716315699\\
-1.71270209865779	-1.65\\
-1.71984785038007	-1.64\\
-1.72	-1.6397862687597\\
-1.72692728493803	-1.63\\
-1.73	-1.6256425974026\\
-1.73395667642095	-1.62\\
-1.74	-1.61134868764123\\
-1.7409368623142	-1.61\\
-1.74785677199676	-1.6\\
-1.75	-1.5968908740585\\
-1.75472376652538	-1.59\\
-1.76	-1.58227343063126\\
-1.7615438902356	-1.58\\
-1.76830863475156	-1.57\\
-1.77	-1.56748998759525\\
-1.77501941689923	-1.56\\
-1.78	-1.55253887208682\\
-1.7816855445451	-1.55\\
-1.78829851063578	-1.54\\
-1.79	-1.53741689080151\\
-1.7948589787264	-1.53\\
-1.8	-1.52212151215122\\
-1.80137689710367	-1.52\\
-1.80784136182543	-1.51\\
-1.81	-1.50664743240791\\
-1.81425714385321	-1.5\\
-1.82	-1.49099672071416\\
-1.82063237588857	-1.49\\
-1.82695151129862	-1.48\\
-1.83	-1.47515629017447\\
-1.83322797981075	-1.47\\
-1.83946296147257	-1.46\\
-1.84	-1.45913516015553\\
-1.84564267660884	-1.45\\
-1.85	-1.44291689905187\\
-1.85178496262841	-1.44\\
-1.85787931877908	-1.43\\
-1.86	-1.42650590440487\\
-1.86392800153911	-1.42\\
-1.86994069916141	-1.41\\
-1.87	-1.40990096390096\\
-1.87589766394596	-1.4\\
-1.88	-1.39308426646422\\
-1.88182008579076	-1.39\\
-1.88769664116837	-1.38\\
-1.89	-1.37606392344498\\
-1.89353029003109	-1.37\\
-1.8993275691094	-1.36\\
-1.9	-1.35883516695619\\
-1.90507421489619	-1.35\\
-1.91	-1.34138670284939\\
-1.91078898900444	-1.34\\
-1.91645440208221	-1.33\\
-1.92	-1.32371476050831\\
-1.92208492895156	-1.32\\
-1.92767334174576	-1.31\\
-1.93	-1.30581857621771\\
-1.93322128453989	-1.3\\
-1.93873347380373	-1.29\\
-1.94	-1.28769226178635\\
-1.94420045840168	-1.28\\
-1.94963718919389	-1.27\\
-1.95	-1.26932972105158\\
-1.95502480517936	-1.26\\
-1.96	-1.25072142713163\\
-1.9603849090694	-1.25\\
-1.96569663271781	-1.24\\
-1.97	-1.23186211104332\\
-1.97097982770614	-1.23\\
-1.97621820322115	-1.22\\
-1.98	-1.21274799917915\\
-1.98142598528331	-1.21\\
-1.9865917343755	-1.2\\
-1.99	-1.19337212673431\\
-1.99172556665854	-1.19\\
-1.9968194004387	-1.18\\
-2	-1.17372727272727\\
158.800909090909	653\\
2	-1.7892802350253\\
1.99950983116407	-1.79\\
1.99267525480999	-1.8\\
1.99	-1.80390034030141\\
1.98579463058755	-1.81\\
1.98	-1.81837493944776\\
1.9788697849402	-1.82\\
1.97189016789264	-1.83\\
1.97	-1.83269857303191\\
1.96485917162435	-1.84\\
1.96	-1.84687713692283\\
1.95778191350529	-1.85\\
1.95065421409199	-1.86\\
1.95	-1.8609146438204\\
1.94346729942162	-1.87\\
1.94	-1.87480537549407\\
1.93623198045804	-1.88\\
1.93	-1.88856168268473\\
1.92894748371476	-1.89\\
1.92160442727849	-1.9\\
1.92	-1.90217743702081\\
1.91420513949903	-1.91\\
1.91	-1.91565710276002\\
1.90675437880924	-1.92\\
1.9	-1.92900699300699\\
1.89925130932993	-1.93\\
1.89168590141654	-1.94\\
1.89	-1.94222086741781\\
1.88406257730866	-1.95\\
1.88	-1.95530472235041\\
1.87638448326917	-1.96\\
1.87	-1.96826322244366\\
1.86865071519598	-1.97\\
1.86085561497326	-1.98\\
1.86	-1.98109395462159\\
1.85299577250572	-1.99\\
1.85	-1.99379647898012\\
1.84507761413312	-2\\
1.84	-2.00637785509\\
1.8371001610618	-2.01\\
1.83	-2.01883928840645\\
1.82906241247032	-2.02\\
1.82095792486147	-2.03\\
1.82	-2.03117807398532\\
1.81278605807502	-2.04\\
1.81	-2.04339587998209\\
1.80455099703169	-2.05\\
1.8	-2.05549769379557\\
1.79625165677761	-2.06\\
1.79	-2.06748465074892\\
1.78788692737987	-2.07\\
1.78	-2.0793578713969\\
1.77945567320343	-2.08\\
1.77095119353424	-2.09\\
1.77	-2.0911148479953\\
1.76237500472632	-2.1\\
1.76	-2.10275860049773\\
1.75372903487532	-2.11\\
1.75	-2.11429228075295\\
1.74501205065099	-2.12\\
1.74	-2.12571694545455\\
1.73622278948651	-2.13\\
1.73	-2.13703363781354\\
1.72735995870701	-2.14\\
1.72	-2.1482433877728\\
1.71842223462672	-2.15\\
1.71	-2.15934721221726\\
1.7094082616134	-2.16\\
1.70031474056758	-2.17\\
1.7	-2.17034502505369\\
1.6911390242941	-2.18\\
1.69	-2.18123719138005\\
1.68188322074901	-2.19\\
1.68	-2.19202674633661\\
1.67254585356122	-2.2\\
1.67	-2.2027146504042\\
1.66312540984704	-2.21\\
1.66	-2.21330185211367\\
1.65362033907435	-2.22\\
1.65	-2.22378928823115\\
1.64402905188417	-2.23\\
1.64	-2.23417788393986\\
1.63434991886769	-2.24\\
1.63	-2.24446855301863\\
1.62458126929674	-2.25\\
1.62	-2.25466219801704\\
1.6147213898057	-2.26\\
1.61	-2.2647597104274\\
1.60476852302261	-2.27\\
1.6	-2.27476197085357\\
1.59472086614724	-2.28\\
1.59	-2.2846698491767\\
1.58457656947358	-2.29\\
1.58	-2.29448420471789\\
1.57433373485439	-2.3\\
1.57	-2.30420588639802\\
1.56399041410496	-2.31\\
1.56	-2.31383573289456\\
1.55354460734338	-2.32\\
1.55	-2.32337457279563\\
1.54299426126426	-2.33\\
1.54	-2.33282322475126\\
1.53233726734287	-2.34\\
1.53	-2.34218249762196\\
1.5215714599663	-2.35\\
1.52	-2.35145319062458\\
1.51069461448822	-2.36\\
1.51	-2.36063609347562\\
1.5	-2.36973119005808\\
1.49970240320627	-2.37\\
1.49	-2.37873791245791\\
1.48858885358471	-2.38\\
1.48	-2.38765892305626\\
1.47735628174994	-2.39\\
1.47	-2.39649497924756\\
1.46600216850175	-2.4\\
1.46	-2.40524682952877\\
1.45452392304025	-2.41\\
1.45	-2.41391521362971\\
1.44291888040397	-2.42\\
1.44	-2.42250086264101\\
1.43118429879714	-2.43\\
1.43	-2.43100449913987\\
1.42	-2.43942516871775\\
1.4193123644595	-2.44\\
1.41	-2.44776210581871\\
1.40729537082808	-2.45\\
1.4	-2.45601894487567\\
1.39513887773298	-2.46\\
1.39	-2.4641963793782\\
1.38283975593572	-2.47\\
1.38	-2.47229509483322\\
1.37039478189971	-2.48\\
1.37	-2.48031576887961\\
1.36	-2.48825407590974\\
1.3577838115263	-2.49\\
1.35	-2.49611457926909\\
1.3450157660288	-2.5\\
1.34	-2.50389884580469\\
1.33209005161659	-2.51\\
1.33	-2.51160752618647\\
1.32	-2.51923910513384\\
1.31899509845781	-2.52\\
1.31	-2.52679161831077\\
1.30571689246183	-2.53\\
1.3	-2.53427028417127\\
1.29226778184595	-2.54\\
1.29	-2.54167572765739\\
1.28	-2.54900576997051\\
1.27863257208359	-2.55\\
1.27	-2.55625891829689\\
1.26479775260921	-2.56\\
1.26	-2.5634405202091\\
1.25077714634015	-2.57\\
1.25	-2.57055117609663\\
1.24	-2.5775847425302\\
1.23653717848224	-2.58\\
1.23	-2.58454684924559\\
1.22209320709624	-2.59\\
1.22	-2.59143962574282\\
1.21	-2.59825881906689\\
1.20742458528922	-2.6\\
1.2	-2.60500567393771\\
1.19252768397819	-2.61\\
1.19	-2.61168477304162\\
1.18	-2.61829196529611\\
1.17739195545743	-2.62\\
1.17	-2.62482771159875\\
1.16200956937799	-2.63\\
1.16	-2.63129723646367\\
1.15	-2.6376947605352\\
1.14636383360619	-2.64\\
1.14	-2.64402344431974\\
1.13046001950753	-2.65\\
1.13	-2.65028740206442\\
1.12	-2.65647755254322\\
1.11425628130513	-2.66\\
1.11	-2.66260312538757\\
1.1	-2.66866178841622\\
1.09777015437393	-2.67\\
1.09	-2.67465046382189\\
1.08097565046842	-2.68\\
1.08	-2.68057678549402\\
1.07	-2.68643160231898\\
1.06384528647108	-2.69\\
1.06	-2.6922233977119\\
1.05	-2.69794870217739\\
1.04638121487011	-2.7\\
1.04	-2.70360741013373\\
1.03	-2.7092039013618\\
1.02856294983944	-2.71\\
1.02	-2.71473094335006\\
1.01036448660864	-2.72\\
1.01	-2.72019877974375\\
1	-2.72559609517999\\
0.991754438072608	-2.73\\
0.99	-2.73093452598272\\
0.98	-2.73620494097603\\
0.972721671140156	-2.74\\
0.97	-2.74141534166768\\
0.96	-2.74655953392402\\
0.953238233820758	-2.75\\
0.95	-2.75164326352742\\
0.94	-2.75666190533834\\
0.933273654169818	-2.76\\
0.93	-2.76162030665218\\
0.92	-2.76651406495231\\
0.912794649764181	-2.77\\
0.91	-2.77134846478025\\
0.9	-2.77611800120409\\
0.89176479603566	-2.78\\
0.89	-2.78082971058004\\
0.88	-2.78547568151795\\
0.870144146500981	-2.79\\
0.87	-2.79006599592766\\
0.86	-2.79458905258115\\
0.85	-2.79905677326626\\
0.8478623198241	-2.8\\
0.84	-2.80346004061641\\
0.83	-2.80780563851392\\
0.824885423917583	-2.81\\
0.82	-2.81209055165019\\
0.81	-2.81631466698439\\
0.801160298362437	-2.82\\
0.8	-2.82048247177659\\
0.79	-2.82458573975045\\
0.78	-2.82863410576352\\
0.776579675643506	-2.83\\
0.77	-2.83262071826479\\
0.76	-2.83654901031172\\
0.751090847893271	-2.84\\
0.75	-2.8404214446152\\
0.74	-2.844230287268\\
0.73	-2.84798451353588\\
0.72455216488257	-2.85\\
0.72	-2.8516797547459\\
0.71	-2.8553152930079\\
0.7	-2.85889635656173\\
0.69687050720519	-2.86\\
0.69	-2.86241667646713\\
0.68	-2.8658798071269\\
0.67	-2.86928860402211\\
0.667879255337797	-2.87\\
0.66	-2.87263624633431\\
0.65	-2.87592785923754\\
0.64	-2.87916527859238\\
0.637377749935514	-2.88\\
0.63	-2.88234245934246\\
0.62	-2.88546343746344\\
0.61	-2.88853036153036\\
0.605122131179372	-2.89\\
0.6	-2.89153926945968\\
0.59	-2.89449048897188\\
0.58	-2.89738779320808\\
0.570813051508312	-2.9\\
0.57	-2.90023059015249\\
0.56	-2.9030129204982\\
0.55	-2.9057414736352\\
0.54	-2.9084162495635\\
0.533957454367812	-2.91\\
0.53	-2.91103459886219\\
0.52	-2.91359526297457\\
0.51	-2.91610228724022\\
0.5	-2.91855567165912\\
0.49398132468915	-2.92\\
0.49	-2.9209529820498\\
0.48	-2.92329310943833\\
0.47	-2.92557973364215\\
0.46	-2.92781285466126\\
0.45	-2.92999247249566\\
0.449964595021514	-2.93\\
0.44	-2.93211320318817\\
0.43	-2.93418054753379\\
0.42	-2.93619452466212\\
0.41	-2.93815513457318\\
0.400327054690813	-2.94\\
0.4	-2.94006221914967\\
0.39	-2.94191139532204\\
0.38	-2.94370733955525\\
0.37	-2.94545005184929\\
0.36	-2.94713953220417\\
0.35	-2.94877578061989\\
0.342266540505131	-2.95\\
0.34	-2.95035788989771\\
0.33	-2.95188380645903\\
0.32	-2.95335662567521\\
0.31	-2.95477634754626\\
0.3	-2.95614297207218\\
0.29	-2.95745649925296\\
0.28	-2.95871692908861\\
0.27	-2.95992426157913\\
0.269343821567264	-2.96\\
0.26	-2.96107577668233\\
0.25	-2.96217413733807\\
0.24	-2.9632195345638\\
0.23	-2.96421196835951\\
0.22	-2.96515143872521\\
0.21	-2.9660379456609\\
0.2	-2.96687148916657\\
0.19	-2.96765206924223\\
0.18	-2.96837968588788\\
0.17	-2.96905433910352\\
0.16	-2.96967602888914\\
0.154303567829062	-2.97\\
0.15	-2.97024413950829\\
0.14	-2.9707586049171\\
0.13	-2.97122024013722\\
0.12	-2.97162904516867\\
0.11	-2.97198502001144\\
0.1	-2.97228816466552\\
0.0899999999999999	-2.97253847913093\\
0.0800000000000001	-2.97273596340766\\
0.0699999999999998	-2.97288061749571\\
0.0600000000000001	-2.97297244139508\\
0.0499999999999998	-2.97301143510578\\
0.04	-2.97299759862779\\
0.0299999999999998	-2.97293093196112\\
0.02	-2.97281143510578\\
0.00999999999999979	-2.97263910806175\\
0	-2.97241395082905\\
-0.01	-2.97213596340766\\
-0.02	-2.9718051457976\\
-0.03	-2.97142149799886\\
-0.04	-2.97098502001143\\
-0.05	-2.97049571183533\\
-0.0591436405821533	-2.97\\
-0.0600000000000001	-2.96995345637969\\
-0.0700000000000001	-2.96935698727502\\
-0.0800000000000001	-2.96870755474034\\
-0.0900000000000001	-2.96800515877565\\
-0.1	-2.96724979938095\\
-0.11	-2.96644147655623\\
-0.12	-2.9655801903015\\
-0.13	-2.96466594061676\\
-0.14	-2.96369872750201\\
-0.15	-2.96267855095724\\
-0.16	-2.96160541098246\\
-0.17	-2.96047930757767\\
-0.174065143412738	-2.96\\
-0.18	-2.95929847144006\\
-0.19	-2.95806332605448\\
-0.2	-2.95677508332376\\
-0.21	-2.9554337432479\\
-0.22	-2.95403930582692\\
-0.23	-2.9525917710608\\
-0.24	-2.95109113894955\\
-0.247022708780235	-2.95\\
-0.25	-2.94953623689365\\
-0.26	-2.94792533702039\\
-0.27	-2.94626120520797\\
-0.28	-2.94454384145639\\
-0.29	-2.94277324576564\\
-0.3	-2.94094941813573\\
-0.305058007488798	-2.94\\
-0.31	-2.93907000115513\\
-0.32	-2.93713480420469\\
-0.33	-2.93514624003696\\
-0.34	-2.93310430865196\\
-0.35	-2.93100901004967\\
-0.354695984086877	-2.93\\
-0.36	-2.92885744064852\\
-0.37	-2.92664979733642\\
-0.38	-2.92438865083961\\
-0.39	-2.92207400115808\\
-0.398757885471174	-2.92\\
-0.4	-2.91970509694648\\
-0.41	-2.91727725531174\\
-0.42	-2.91479577383026\\
-0.43	-2.91226065250203\\
-0.438732564919047	-2.91\\
-0.44	-2.90967105109999\\
-0.45	-2.90702188336631\\
-0.46	-2.90431893842393\\
-0.47	-2.90156221627284\\
-0.475558500724788	-2.9\\
-0.48	-2.89874851207842\\
-0.49	-2.89587688178317\\
-0.5	-2.89295133621193\\
-0.509905604950843	-2.89\\
-0.51	-2.8899718029718\\
-0.52	-2.88693061893062\\
-0.53	-2.88383538083538\\
-0.54	-2.88068608868609\\
-0.542141787501371	-2.88\\
-0.55	-2.87747624633431\\
-0.56	-2.8742104398827\\
-0.57	-2.8708904398827\\
-0.572638970971667	-2.87\\
-0.58	-2.86750982006351\\
-0.59	-2.86407256262496\\
-0.6	-2.86058097142185\\
-0.601638419952905	-2.86\\
-0.61	-2.85702735526471\\
-0.62	-2.85341775733994\\
-0.62932775543041	-2.85\\
-0.63	-2.84975304409505\\
-0.64	-2.84602482562951\\
-0.65	-2.84224199077905\\
-0.655842395489972	-2.84\\
-0.66	-2.83840037928173\\
-0.67	-2.8344981628541\\
-0.68	-2.83054118762593\\
-0.681349011729253	-2.83\\
-0.69	-2.82652038027332\\
-0.7	-2.82244325609032\\
-0.705912973454892	-2.82\\
-0.71	-2.81830680328846\\
-0.72	-2.81410890027404\\
-0.729661297100434	-2.81\\
-0.73	-2.80985557281089\\
-0.74	-2.80553625612233\\
-0.75	-2.80116174889499\\
-0.752622636929965	-2.8\\
-0.76	-2.79672343993293\\
-0.77	-2.79222673374057\\
-0.774891724773056	-2.79\\
-0.78	-2.78766854809655\\
-0.79	-2.7830489972379\\
-0.796521873153691	-2.78\\
-0.8	-2.77836965683323\\
-0.81	-2.77362661047562\\
-0.817557523900334	-2.77\\
-0.82	-2.76882482192442\\
-0.83	-2.76395762404926\\
-0.838039090663856	-2.76\\
-0.84	-2.75903207844087\\
-0.85	-2.7540400677884\\
-0.85800340519412	-2.75\\
-0.86	-2.74898944046608\\
-0.87	-2.74387195047943\\
-0.877484105571847	-2.74\\
-0.88	-2.73869490081538\\
-0.89	-2.73345125958379\\
-0.896511975016648	-2.73\\
-0.9	-2.72814643075046\\
-0.91	-2.7227759609518\\
-0.915115238459291	-2.72\\
-0.92	-2.71734197968922\\
-0.93	-2.71184399853175\\
-0.933319822895786	-2.71\\
-0.94	-2.70627947491105\\
-0.95	-2.70065329407435\\
-0.951149586580601	-2.7\\
-0.96	-2.69495682125723\\
-0.96861295769831	-2.69\\
-0.97	-2.68919958060935\\
-0.98	-2.68337190082645\\
-0.985729976732974	-2.68\\
-0.99	-2.67748039579468\\
-1	-2.67152257266543\\
-1.00253130719089	-2.67\\
-1.01	-2.66549534912564\\
-1.01902552298698	-2.66\\
-1.02	-2.65940504912324\\
-1.03	-2.65324225842557\\
-1.03521242777456	-2.65\\
-1.04	-2.64701384212495\\
-1.05	-2.64071891757077\\
-1.05113170138003	-2.64\\
-1.06	-2.63435088158059\\
-1.06676886563041	-2.63\\
-1.07	-2.62791736677116\\
-1.08	-2.6214139184953\\
-1.08215491046687	-2.62\\
-1.09	-2.61483830001257\\
-1.09728920797894	-2.61\\
-1.1	-2.60819568780734\\
-1.11	-2.60148140209305\\
-1.11218736618696	-2.6\\
-1.12	-2.59469414590972\\
-1.12685297646516	-2.59\\
-1.13	-2.58783834157474\\
-1.14	-2.580910865982\\
-1.14130383491533	-2.58\\
-1.15	-2.57390781945327\\
-1.15553175788773	-2.57\\
-1.16	-2.56683462960602\\
-1.16956821305535	-2.56\\
-1.17	-2.55969070451349\\
-1.18	-2.55246848229127\\
-1.18339016980701	-2.55\\
-1.19	-2.54517361200154\\
-1.1970283405041	-2.54\\
-1.2	-2.53780635206378\\
-1.21	-2.53036505079079\\
-1.21048668849536	-2.53\\
-1.22	-2.52284410058027\\
-1.22375133939961	-2.52\\
-1.23	-2.51524906245959\\
-1.23684999240596	-2.51\\
-1.24	-2.50757930229542\\
-1.24978650010884	-2.5\\
-1.25	-2.49983417869684\\
-1.26	-2.49200728313175\\
-1.26254505878667	-2.49\\
-1.27	-2.48410343028564\\
-1.27514867850421	-2.48\\
-1.28	-2.47612243296272\\
-1.28760244936411	-2.47\\
-1.29	-2.46806362324544\\
-1.29990971964726	-2.46\\
-1.3	-2.45992632548349\\
-1.31	-2.45170490724905\\
-1.31205851984814	-2.45\\
-1.32	-2.44340361525267\\
-1.32406780730111	-2.44\\
-1.33	-2.43502196638878\\
-1.33594126211312	-2.43\\
-1.34	-2.42655925680159\\
-1.34768181041048	-2.42\\
-1.35	-2.41801477439106\\
-1.35929229487199	-2.41\\
-1.36	-2.40938779869176\\
-1.37	-2.40067561073288\\
-1.37077002662609	-2.4\\
-1.38	-2.39187789530058\\
-1.38211914726457	-2.39\\
-1.39	-2.38299570296764\\
-1.39334723701781	-2.38\\
-1.4	-2.37402828282828\\
-1.40445679545827	-2.37\\
-1.41	-2.36497487505066\\
-1.41545025379959	-2.36\\
-1.42	-2.3558347107438\\
-1.42632997721761	-2.35\\
-1.43	-2.34660701182226\\
-1.43709826707739	-2.34\\
-1.44	-2.3372909908682\\
-1.44775736307081	-2.33\\
-1.45	-2.32788585099111\\
-1.45830944526872	-2.32\\
-1.46	-2.3183907856849\\
-1.46875663609175	-2.31\\
-1.47	-2.30880497868244\\
-1.47910100220343	-2.3\\
-1.48	-2.29912760380742\\
-1.48934455632918	-2.29\\
-1.49	-2.28935782482358\\
-1.49948925900462	-2.28\\
-1.5	-2.27949479528106\\
-1.50953702025624	-2.27\\
-1.51	-2.26953765836002\\
-1.51948970121757	-2.26\\
-1.52	-2.25948554671135\\
-1.52934911568371	-2.25\\
-1.53	-2.24933758229444\\
-1.53911703160688	-2.24\\
-1.54	-2.23909287621189\\
-1.54879517253564	-2.23\\
-1.55	-2.22875052854123\\
-1.55838521900028	-2.22\\
-1.56	-2.21830962816344\\
-1.56788880984658	-2.21\\
-1.57	-2.20776925258829\\
-1.57730754352031	-2.2\\
-1.58	-2.19712846777636\\
-1.58664297930449	-2.19\\
-1.59	-2.18638632795776\\
-1.59589663851151	-2.18\\
-1.6	-2.17554187544739\\
-1.60507000563188	-2.17\\
-1.61	-2.1645941404567\\
-1.61416452944163	-2.16\\
-1.62	-2.15354214090189\\
-1.62318162406988	-2.15\\
-1.63	-2.14238488220841\\
-1.63212267002843	-2.14\\
-1.64	-2.13112135711179\\
-1.6409890152048	-2.13\\
-1.64978068773258	-2.12\\
-1.65	-2.11974974463739\\
-1.6584940440625	-2.11\\
-1.66	-2.10826584687454\\
-1.66713623093159	-2.1\\
-1.67	-2.09667219856073\\
-1.67570848778036	-2.09\\
-1.68	-2.08496773242965\\
-1.68421202543506	-2.08\\
-1.69	-2.07315136733186\\
-1.69264802692867	-2.07\\
-1.7	-2.06122200800831\\
-1.70101764829383	-2.06\\
-1.70931815094293	-2.05\\
-1.71	-2.04917584714136\\
-1.71754841403362	-2.04\\
-1.72	-2.03701003444661\\
-1.72571536723854	-2.03\\
-1.73	-2.02472742299023\\
-1.73382007601054	-2.02\\
-1.74	-2.01232685059551\\
-1.74186358202828	-2.01\\
-1.74984604981949	-2\\
-1.75	-1.99980649567461\\
-1.7577586724463	-1.99\\
-1.76	-1.98715760621288\\
-1.76561288704728	-1.98\\
-1.77	-1.97438670741024\\
-1.77340965500861	-1.97\\
-1.78	-1.96149256477081\\
-1.78114991673261	-1.96\\
-1.78882823295971	-1.95\\
-1.79	-1.94846874517673\\
-1.79644539463638	-1.94\\
-1.8	-1.93531361313303\\
-1.80400859903495	-1.93\\
-1.81	-1.92203092463092\\
-1.81151871521039	-1.92\\
-1.8189710992063	-1.91\\
-1.82	-1.90861461430136\\
-1.82636375195053	-1.9\\
-1.83	-1.89506060606061\\
-1.83370569230946	-1.89\\
-1.84	-1.88137450764141\\
-1.84099772412768	-1.88\\
-1.84823133841878	-1.87\\
-1.85	-1.86754640647311\\
-1.85541120123776	-1.86\\
-1.86	-1.8535780926604\\
-1.86254337222593	-1.85\\
-1.86962665368989	-1.84\\
-1.87	-1.83947105980439\\
-1.87665026737968	-1.83\\
-1.88	-1.82521255028158\\
-1.88362829975248	-1.82\\
-1.89	-1.81081366058453\\
-1.89056145472373	-1.81\\
-1.8974372973515	-1.8\\
-1.9	-1.79625955439909\\
-1.90426666371063	-1.79\\
-1.91	-1.78155851150645\\
-1.91105312496554	-1.78\\
-1.91778611052904	-1.77\\
-1.92	-1.76669998356074\\
-1.9244720469633	-1.76\\
-1.93	-1.75168800527966\\
-1.93111695797897	-1.75\\
-1.9377099686671	-1.74\\
-1.94	-1.73651387734752\\
-1.94425748704635	-1.73\\
-1.95	-1.72118181818182\\
-1.95076577209498	-1.72\\
-1.95722160374402	-1.71\\
-1.96	-1.70568038985045\\
-1.96363550326404	-1.7\\
-1.97	-1.6900187215382\\
-1.97001187864519	-1.69\\
-1.97633324415432	-1.68\\
-1.98	-1.67417774001699\\
-1.98261812218471	-1.67\\
-1.98886153321904	-1.66\\
-1.99	-1.65816966272898\\
-1.9950566393625	-1.65\\
-2	-1.64198315861832\\
181.350681818182	204\\
-0.93794237504681	-3\\
-0.94	-2.99893719422005\\
-0.95	-2.99371942200478\\
-0.957057274238467	-2.99\\
-0.96	-2.98844519219802\\
-0.97	-2.98310893121935\\
-0.975769076092709	-2.98\\
-0.98	-2.97771423670669\\
-0.99	-2.97225889079474\\
-0.994100977807303	-2.97\\
-1	-2.96674251977531\\
-1.01	-2.96116748824946\\
-1.01207443016316	-2.96\\
-1.02	-2.95552821514768\\
-1.02970658863889	-2.95\\
-1.03	-2.94983246917848\\
-1.04	-2.94406947805047\\
-1.04699677093445	-2.94\\
-1.05	-2.93824881598706\\
-1.06	-2.93236444495784\\
-1.0639820633037	-2.93\\
-1.07	-2.92641760277939\\
-1.08	-2.92041123335264\\
-1.08067861715749	-2.92\\
-1.09	-2.91433693254383\\
-1.09707609539866	-2.91\\
-1.1	-2.90820335234548\\
-1.11	-2.90200488883716\\
-1.11320667250014	-2.9\\
-1.12	-2.89574174349399\\
-1.12908186432857	-2.89\\
-1.13	-2.88941804141804\\
-1.14	-2.88302550602551\\
-1.14469318862411	-2.88\\
-1.15	-2.87657008797654\\
-1.16	-2.8700526686217\\
-1.16008014565446	-2.87\\
-1.17	-2.86346418910973\\
-1.17521483579711	-2.86\\
-1.18	-2.85681299375074\\
-1.19	-2.85009833746021\\
-1.19014527338919	-2.85\\
-1.2	-2.84331185719352\\
-1.20484093933058	-2.84\\
-1.21	-2.83646130141045\\
-1.21934531054463	-2.83\\
-1.22	-2.82954616755793\\
-1.23	-2.82255923945336\\
-1.23363433855916	-2.82\\
-1.24	-2.81550565947814\\
-1.24773773840821	-2.81\\
-1.25	-2.80838609485127\\
-1.26	-2.80119687014694\\
-1.26165212802797	-2.8\\
-1.27	-2.79393651934363\\
-1.27537861058833	-2.79\\
-1.28	-2.78660874264441\\
-1.28893839232056	-2.78\\
-1.29	-2.77921300421433\\
-1.3	-2.77174413004214\\
-1.30231793378247	-2.77\\
-1.31	-2.76420427381384\\
-1.31553174590568	-2.76\\
-1.32	-2.75659496429004\\
-1.32859118505086	-2.75\\
-1.33	-2.74891564510256\\
-1.34	-2.74116264109722\\
-1.34148883258988	-2.74\\
-1.35	-2.73333576731167\\
-1.35422987299579	-2.73\\
-1.36	-2.72543733984137\\
-1.36682760522808	-2.72\\
-1.37	-2.71746678086382\\
-1.37928505135032	-2.71\\
-1.38	-2.70942350631824\\
-1.39	-2.70130339835603\\
-1.39159402241594	-2.7\\
-1.4	-2.69310800836511\\
-1.40376465855076	-2.69\\
-1.41	-2.68483828789935\\
-1.4158046853033	-2.68\\
-1.42	-2.67649363017934\\
-1.42771676343059	-2.67\\
-1.43	-2.6680734218033\\
-1.43950348190432	-2.66\\
-1.44	-2.65957704265639\\
-1.45	-2.65100111926377\\
-1.45115959147808	-2.65\\
-1.46	-2.6423468013468\\
-1.46269292961092	-2.64\\
-1.47	-2.63361460547705\\
-1.474109176203	-2.63\\
-1.48	-2.62480388714734\\
-1.48541062324346	-2.62\\
-1.49	-2.61591399471897\\
-1.49659950329035	-2.61\\
-1.5	-2.60694426932291\\
-1.50767799138448	-2.6\\
-1.51	-2.59789404475914\\
-1.51864820688987	-2.59\\
-1.52	-2.58876264739445\\
-1.52951221526392	-2.58\\
-1.53	-2.57954939605849\\
-1.54	-2.57025289256198\\
-1.54027032169505	-2.57\\
-1.55	-2.56087211526202\\
-1.55092381248227	-2.56\\
-1.56	-2.55140762050889\\
-1.56147765861777	-2.55\\
-1.57	-2.54185869983331\\
-1.57193370239445	-2.54\\
-1.58	-2.53222463674939\\
-1.58229374096809	-2.53\\
-1.59	-2.52250470664088\\
-1.59255952773129	-2.52\\
-1.6	-2.51269817664555\\
-1.60273277363754	-2.51\\
-1.61	-2.50280430553754\\
-1.61281514847747	-2.5\\
-1.62	-2.49282234360775\\
-1.62280828210935	-2.49\\
-1.63	-2.48275153254206\\
-1.6327137656457	-2.48\\
-1.64	-2.47259110529758\\
-1.64253315259786	-2.47\\
-1.65	-2.46234028597665\\
-1.65226795998017	-2.46\\
-1.66	-2.45199828969872\\
-1.66191966937553	-2.45\\
-1.67	-2.44156432246998\\
-1.67148972796381	-2.44\\
-1.68	-2.43103758105068\\
-1.68097954951466	-2.43\\
-1.69	-2.42041725282017\\
-1.69039051534611	-2.42\\
-1.69972239128545	-2.41\\
-1.7	-2.40970164197036\\
-1.7089754356498	-2.4\\
-1.71	-2.39888927567278\\
-1.71815320570285	-2.39\\
-1.72	-2.38798052907211\\
-1.72725697539532	-2.38\\
-1.73	-2.37697454545455\\
-1.73628799009192	-2.37\\
-1.74	-2.36587045792246\\
-1.7452474673686	-2.36\\
-1.75	-2.35466738924265\\
-1.75413659778341	-2.35\\
-1.76	-2.34336445169181\\
-1.76295654562172	-2.34\\
-1.77	-2.33196074689928\\
-1.77170844961701	-2.33\\
-1.78	-2.32045536568694\\
-1.78039342364794	-2.32\\
-1.78900716926311	-2.31\\
-1.79	-2.30884390042635\\
-1.79755371260088	-2.3\\
-1.8	-2.29712718995724\\
-1.8060362654952	-2.29\\
-1.81	-2.28530565933306\\
-1.81445585859877	-2.28\\
-1.82	-2.27337834836919\\
-1.82281350054904	-2.27\\
-1.83	-2.26134428511764\\
-1.83111017855295	-2.26\\
-1.83934338963174	-2.25\\
-1.84	-2.24920002801513\\
-1.84751140795297	-2.24\\
-1.85	-2.23694253196572\\
-1.85562109646079	-2.23\\
-1.86	-2.2245749119098\\
-1.86367337007571	-2.22\\
-1.87	-2.21209614025166\\
-1.87166912468337	-2.21\\
-1.87960720517872	-2.2\\
-1.88	-2.19950362782757\\
-1.8874816042649	-2.19\\
-1.89	-2.18679092336235\\
-1.89530189751418	-2.18\\
-1.9	-2.17396349319971\\
-1.90306891620756	-2.17\\
-1.91	-2.16102024989229\\
-1.91078347468375	-2.16\\
-1.91843845466677	-2.15\\
-1.92	-2.14795360601243\\
-1.92603840456466	-2.14\\
-1.93	-2.13476526025808\\
-1.93358810878086	-2.13\\
-1.94	-2.12145730909091\\
-1.94108832379235	-2.12\\
-1.94853246287707	-2.11\\
-1.95	-2.10802225150051\\
-1.95592297964786	-2.1\\
-1.96	-2.09446012630342\\
-1.96326608089651	-2.09\\
-1.97	-2.08077442168852\\
-1.97056246990208	-2.08\\
-1.97780202258013	-2.07\\
-1.98	-2.06695402639775\\
-1.98499334447911	-2.06\\
-1.99	-2.05300461240887\\
-1.99213989763588	-2.05\\
-1.99923862709152	-2.04\\
-2	-2.03892391792721\\
181.350681818182	181\\
2	-2.16781487864426\\
1.99830886194134	-2.17\\
1.99054507663751	-2.18\\
1.99	-2.18069987155701\\
1.98271945210989	-2.19\\
1.98	-2.19346293925167\\
1.97483980010107	-2.2\\
1.97	-2.20611204084527\\
1.96690527527855	-2.21\\
1.96	-2.2186482397851\\
1.95891501458222	-2.22\\
1.95086358529749	-2.23\\
1.95	-2.23106927076015\\
1.94274917578343	-2.24\\
1.94	-2.24337568286875\\
1.93457653359634	-2.25\\
1.93	-2.25557268537914\\
1.92634472819508	-2.26\\
1.92	-2.26766128358625\\
1.91805280910598	-2.27\\
1.91	-2.27964247050659\\
1.90969980538626	-2.28\\
1.90127780820641	-2.29\\
1.9	-2.29151262243068\\
1.89279136252329	-2.3\\
1.89	-2.30327657818732\\
1.88424104986629	-2.31\\
1.88	-2.31493637734814\\
1.87562582173527	-2.32\\
1.87	-2.32649295967191\\
1.86694460655289	-2.33\\
1.86	-2.3379472536459\\
1.85819630902624	-2.34\\
1.85	-2.34930017665444\\
1.84937980948711	-2.35\\
1.84049121497128	-2.36\\
1.84	-2.36055099284074\\
1.83152899717839	-2.37\\
1.83	-2.37170047138047\\
1.82249540278877	-2.38\\
1.82	-2.38275157781657\\
1.81338922023686	-2.39\\
1.81	-2.39370518141652\\
1.80420921038046	-2.4\\
1.8	-2.40456214123615\\
1.7949541057117	-2.41\\
1.79	-2.41532330626913\\
1.78562260954066	-2.42\\
1.78	-2.4259895155939\\
1.77621339515068	-2.43\\
1.77	-2.43656159851793\\
1.76672510492419	-2.44\\
1.76	-2.44704037471962\\
1.75715634943776	-2.45\\
1.75	-2.45742665438758\\
1.7475057065252	-2.46\\
1.74	-2.4677212383576\\
1.73777172030735	-2.47\\
1.73	-2.47792491824722\\
1.72795290018711	-2.48\\
1.72	-2.48803847658797\\
1.71804771980839	-2.49\\
1.71	-2.49806268695539\\
1.70805461597733	-2.5\\
1.7	-2.50799831409675\\
1.69797198754418	-2.51\\
1.69	-2.51784611405664\\
1.68779819424432	-2.52\\
1.68	-2.52760683430045\\
1.67753155549644	-2.53\\
1.67	-2.53728121383567\\
1.66717034915622	-2.54\\
1.66	-2.5468699833312\\
1.6567128102234	-2.55\\
1.65	-2.55637386523463\\
1.6461571295004	-2.56\\
1.64	-2.56579357388754\\
1.63550145220012	-2.57\\
1.63	-2.57512981563891\\
1.62474387650086	-2.58\\
1.62	-2.58438328895651\\
1.6138824520459	-2.59\\
1.61	-2.5935546845366\\
1.60291517838529	-2.6\\
1.6	-2.60264468541168\\
1.59184000335716	-2.61\\
1.59	-2.61165396705646\\
1.58065482140594	-2.62\\
1.58	-2.62058319749216\\
1.57	-2.62943147335423\\
1.56935323737929	-2.63\\
1.56	-2.63819919969989\\
1.5579321089588	-2.64\\
1.55	-2.64688863948123\\
1.54639341418639	-2.65\\
1.54	-2.65550043526925\\
1.53473478178617	-2.66\\
1.53	-2.66403522262185\\
1.52295377628013	-2.67\\
1.52	-2.67249363017934\\
1.51104789580045	-2.68\\
1.51	-2.68087627975823\\
1.5	-2.68918157148144\\
1.49900776144402	-2.69\\
1.49	-2.69740976749908\\
1.48682924992847	-2.7\\
1.48	-2.70556385719544\\
1.47451679429828	-2.71\\
1.47	-2.71364443900648\\
1.46206755951927	-2.72\\
1.46	-2.72165210494204\\
1.45	-2.72958633312996\\
1.44947489892966	-2.73\\
1.44	-2.73744420104661\\
1.43672355960497	-2.74\\
1.43	-2.74523073188494\\
1.42382509625206	-2.75\\
1.42	-2.75294649558165\\
1.41077628100613	-2.76\\
1.41	-2.76059205601835\\
1.4	-2.76816310515514\\
1.39755578402866	-2.77\\
1.39	-2.77566333534016\\
1.38417094210756	-2.78\\
1.38	-2.7830948721028\\
1.37062388911537	-2.79\\
1.37	-2.79045825847407\\
1.36	-2.79774811354653\\
1.35688730319035	-2.8\\
1.35	-2.80496953768964\\
1.3429745240995	-2.81\\
1.34	-2.81212427022519\\
1.33	-2.8192107708805\\
1.32887757349826	-2.82\\
1.32	-2.82622590612002\\
1.31457602513962	-2.83\\
1.31	-2.83317577337916\\
1.30008865400837	-2.84\\
1.3	-2.84006088190093\\
1.29	-2.84687362572408\\
1.28537390452534	-2.85\\
1.28	-2.85362233227214\\
1.27046133497928	-2.86\\
1.27	-2.86030765612137\\
1.26	-2.86692214512525\\
1.25530825908463	-2.87\\
1.25	-2.87347331378299\\
1.24	-2.87996234604106\\
1.23994148422262	-2.88\\
1.23	-2.88638071838072\\
1.22431308024635	-2.89\\
1.22	-2.89273789240285\\
1.21	-2.89903185902672\\
1.20844850479699	-2.9\\
1.2	-2.90525840996392\\
1.19231545586599	-2.91\\
1.19	-2.91142505514919\\
1.18	-2.91752594914664\\
1.17590880291831	-2.92\\
1.17	-2.92356409959467\\
1.16	-2.92954244354372\\
1.15922773206153	-2.93\\
1.15	-2.93545339031997\\
1.14223653326496	-2.94\\
1.14	-2.94130648692246\\
1.13	-2.94709482659293\\
1.1249344048217	-2.95\\
1.12	-2.95282277899092\\
1.11	-2.95849028847259\\
1.10731100693947	-2.96\\
1.1	-2.96409434827468\\
1.09	-2.96964163705147\\
1.0893477580487	-2.97\\
1.08	-2.97512304173814\\
1.07101464236806	-2.98\\
1.07	-2.98054933272499\\
1.06	-2.98591068780655\\
1.05229689950796	-2.99\\
1.05	-2.99121629309364\\
1.04	-2.99645909659802\\
1.0331777626762	-3\\
203.900454545455	124\\
-1.38246060404611	-3\\
-1.39	-2.9943223347366\\
-1.39569986046302	-2.99\\
-1.4	-2.9867309227786\\
-1.40879289535248	-2.98\\
-1.41	-2.97907364208119\\
-1.42	-2.97134659805603\\
-1.42173087381495	-2.97\\
-1.43	-2.96355049868165\\
-1.43452151189103	-2.96\\
-1.44	-2.95568716239513\\
-1.44717585811859	-2.95\\
-1.45	-2.94775607788916\\
-1.4596966335835	-2.94\\
-1.46	-2.93975672865889\\
-1.47	-2.93168430172115\\
-1.47207278413533	-2.93\\
-1.48	-2.92354209612044\\
-1.48431962941517	-2.92\\
-1.49	-2.9153301985371\\
-1.49644165064755	-2.91\\
-1.5	-2.90704807356536\\
-1.5084412597064	-2.9\\
-1.51	-2.89869518030109\\
-1.52	-2.89027027657836\\
-1.52031876677448	-2.89\\
-1.53	-2.88177032877033\\
-1.53206942297961	-2.88\\
-1.54	-2.87319812316716\\
-1.54370540507482	-2.87\\
-1.55	-2.86455309890627\\
-1.55522885968585	-2.86\\
-1.56	-2.85583468930551\\
-1.56664188019274	-2.85\\
-1.57	-2.84704232178745\\
-1.57794650837057	-2.84\\
-1.58	-2.83817541780254\\
-1.58914473596988	-2.83\\
-1.59	-2.82923339275104\\
-1.6	-2.82021509209744\\
-1.60023706303781	-2.82\\
-1.61	-2.81111867032051\\
-1.61122231913869	-2.81\\
-1.62	-2.80194552622148\\
-1.62210753801359	-2.8\\
-1.63	-2.7926950532998\\
-1.632894503261	-2.79\\
-1.64	-2.78336663864537\\
-1.64358495632937	-2.78\\
-1.65	-2.7739596628537\\
-1.65418059775492	-2.77\\
-1.66	-2.76447349993964\\
-1.6646830883561	-2.76\\
-1.67	-2.75490751724973\\
-1.67509405038638	-2.75\\
-1.68	-2.74526107537323\\
-1.68541506864717	-2.74\\
-1.69	-2.7355335280516\\
-1.69564769156243	-2.73\\
-1.7	-2.72572422208664\\
-1.70579343221647	-2.72\\
-1.71	-2.71583249724703\\
-1.7158537693564	-2.71\\
-1.72	-2.70585768617348\\
-1.72583014836071	-2.7\\
-1.73	-2.6957991142822\\
-1.73572398217516	-2.69\\
-1.74	-2.68565609966695\\
-1.74553665221749	-2.68\\
-1.75	-2.67542795299938\\
-1.75526950925181	-2.67\\
-1.76	-2.66511397742776\\
-1.76492387423426	-2.66\\
-1.77	-2.65471346847407\\
-1.77450103913069	-2.65\\
-1.78	-2.64422571392942\\
-1.78400226770761	-2.64\\
-1.79	-2.63364999374766\\
-1.79342879629738	-2.63\\
-1.8	-2.6229855799373\\
-1.80278183453868	-2.62\\
-1.81	-2.61223173645165\\
-1.81206256609299	-2.61\\
-1.82	-2.60138771907704\\
-1.82127214933827	-2.6\\
-1.83	-2.59045277531926\\
-1.8304117180404	-2.59\\
-1.83947963254654	-2.58\\
-1.84	-2.57942453909727\\
-1.84847722412196	-2.57\\
-1.85	-2.56830230779039\\
-1.85740776555443	-2.56\\
-1.86	-2.55708656182074\\
-1.86627230854543	-2.55\\
-1.87	-2.54577650980895\\
-1.87507188291585	-2.54\\
-1.88	-2.53437135142086\\
-1.88380749717204	-2.53\\
-1.89	-2.52287027724049\\
-1.89248013905447	-2.52\\
-1.9	-2.51127246864089\\
-1.90109077606944	-2.51\\
-1.90963851415016	-2.5\\
-1.91	-2.49957588763168\\
-1.91812027836291	-2.49\\
-1.92	-2.48777696621886\\
-1.92654260163494	-2.48\\
-1.93	-2.47587848266841\\
-1.93490638339684	-2.47\\
-1.94	-2.46387957497048\\
-1.94321250502395	-2.46\\
-1.95	-2.45177937113538\\
-1.95146183028718	-2.45\\
-1.95965348407389	-2.44\\
-1.96	-2.43957575757576\\
-1.96778249787596	-2.43\\
-1.97	-2.42726357000664\\
-1.97585702819841	-2.42\\
-1.98	-2.41484706508718\\
-1.98387787917834	-2.41\\
-1.99	-2.4023253237218\\
-1.99184583920567	-2.4\\
-1.99976051457576	-2.39\\
-2	-2.38969652208943\\
203.900454545455	103\\
2	-2.50571261833744\\
1.99632540097145	-2.51\\
1.99	-2.5173592396224\\
1.98771850246349	-2.52\\
1.98	-2.52890857511283\\
1.97904946936942	-2.53\\
1.97031568308159	-2.54\\
1.97	-2.54036043082446\\
1.96151315715205	-2.55\\
1.96	-2.5517138473341\\
1.95264579385164	-2.56\\
1.95	-2.5629727145225\\
1.94371256801926	-2.57\\
1.94	-2.57413782581055\\
1.93471243276718	-2.58\\
1.93	-2.58520996576645\\
1.9256443189024	-2.59\\
1.92	-2.59618991022885\\
1.9165071343294	-2.6\\
1.91	-2.60707842642794\\
1.90729976343359	-2.61\\
1.9	-2.61787627310449\\
1.89802106644483	-2.62\\
1.89	-2.62858420062696\\
1.88866987878003	-2.63\\
1.88	-2.63920295110667\\
1.87924501036423	-2.64\\
1.87	-2.64973325851104\\
1.86974524492931	-2.65\\
1.86016840751283	-2.66\\
1.86	-2.66017536896937\\
1.85051316750698	-2.67\\
1.85	-2.67052999381571\\
1.84077963366611	-2.68\\
1.84	-2.68079856913778\\
1.83096649188031	-2.69\\
1.83	-2.69098179357855\\
1.82107239846557	-2.7\\
1.82	-2.70108035823825\\
1.81109597932715	-2.71\\
1.81	-2.71109494677597\\
1.80103582909436	-2.72\\
1.8	-2.72102623550946\\
1.79089051022557	-2.73\\
1.79	-2.73087489351345\\
1.78065855208243	-2.74\\
1.78	-2.74064158271635\\
1.77033844997181	-2.75\\
1.77	-2.7503269579954\\
1.76	-2.75993148529234\\
1.75992824634576	-2.76\\
1.75	-2.76945490764216\\
1.74942424667491	-2.77\\
1.74	-2.77889873570139\\
1.73882678912734	-2.78\\
1.73	-2.78826360033626\\
1.72813420220659	-2.79\\
1.72	-2.79755012576356\\
1.7173447743175	-2.8\\
1.71	-2.80675892963804\\
1.70645675255645	-2.81\\
1.7	-2.81589062313833\\
1.69546834145765	-2.82\\
1.69	-2.82494581105169\\
1.68437770169339	-2.83\\
1.68	-2.8339250918573\\
1.67318294872648	-2.84\\
1.67	-2.84282905780825\\
1.66188215141255	-2.85\\
1.66	-2.85165829501238\\
1.65047333055036	-2.86\\
1.65	-2.8604133835117\\
1.64	-2.86909255556862\\
1.63894787078828	-2.87\\
1.63	-2.87769747800587\\
1.62730648370497	-2.88\\
1.62	-2.88622978822979\\
1.6155499703092	-2.89\\
1.61	-2.8946900455129\\
1.60367611291018	-2.9\\
1.6	-2.90307880339891\\
1.59168263648953	-2.91\\
1.59	-2.91139660977592\\
1.58	-2.9196430976431\\
1.57956437327287	-2.92\\
1.57	-2.92781598147076\\
1.56730981555711	-2.93\\
1.56	-2.93591937160679\\
1.55492741344898	-2.94\\
1.55	-2.94395379652034\\
1.5424146055884	-2.95\\
1.54	-2.95191977933571\\
1.53	-2.95981737731295\\
1.52976719654238	-2.96\\
1.52	-2.96764255416714\\
1.51696664847403	-2.97\\
1.51	-2.97540068610635\\
1.50402637715168	-2.98\\
1.5	-2.98309227786016\\
1.49094348651842	-2.99\\
1.49	-2.99071782910456\\
1.48	-2.99827352372284\\
1.4776989915839	-3\\
226.450227272727	59\\
-1.71843073446883	-3\\
-1.72	-2.99854602343839\\
-1.72917162219916	-2.99\\
-1.73	-2.98922618911828\\
-1.73982138737448	-2.98\\
-1.74	-2.97983178959405\\
-1.75	-2.97036134934248\\
-1.75037943828724	-2.97\\
-1.76	-2.96081497191333\\
-1.7608489067743	-2.96\\
-1.77	-2.95119250660844\\
-1.77123223086515	-2.95\\
-1.78	-2.94149337481277\\
-1.7815308207448	-2.94\\
-1.79	-2.93171699202957\\
-1.79174605598562	-2.93\\
-1.8	-2.92186276780544\\
-1.80187928637357	-2.92\\
-1.81	-2.91193010565424\\
-1.81193183270775	-2.91\\
-1.82	-2.90191840297986\\
-1.82190498757441	-2.9\\
-1.83	-2.89182705099778\\
-1.83180001609621	-2.89\\
-1.84	-2.88165543465544\\
-1.84161815665778	-2.88\\
-1.85	-2.87140293255132\\
-1.8513606216084	-2.87\\
-1.86	-2.86106891685288\\
-1.86102859794258	-2.86\\
-1.87	-2.85065275321306\\
-1.87062324795947	-2.85\\
-1.88	-2.84015380068566\\
-1.88014570990178	-2.84\\
-1.8895950138318	-2.83\\
-1.89	-2.82957029114676\\
-1.89897324761279	-2.82\\
-1.9	-2.81890206124151\\
-1.90828224939309	-2.81\\
-1.91	-2.80814884721061\\
-1.91752307743209	-2.8\\
-1.92	-2.79730997724278\\
-1.9266967685302	-2.79\\
-1.93	-2.78638477242704\\
-1.93580433857004	-2.78\\
-1.94	-2.77537254665864\\
-1.94484678304131	-2.77\\
-1.95	-2.76427260654352\\
-1.95382507754996	-2.76\\
-1.96	-2.7530842513013\\
-1.96274017831217	-2.75\\
-1.97	-2.74180677266659\\
-1.97159302263364	-2.74\\
-1.98	-2.73043945478885\\
-1.98038452937481	-2.73\\
-1.98911124836274	-2.72\\
-1.99	-2.71897883274196\\
-1.99777628246564	-2.71\\
-2	-2.70742546926758\\
226.450227272727	38\\
2	-2.8137906588824\\
1.99420744923251	-2.82\\
1.99	-2.82449839572193\\
1.98482766710984	-2.83\\
1.98	-2.83512160720635\\
1.97537773036105	-2.84\\
1.97	-2.8456609528313\\
1.96585654132283	-2.85\\
1.96	-2.85611708524938\\
1.95626297931254	-2.86\\
1.95	-2.8664906503587\\
1.94659590002168	-2.87\\
1.94	-2.87678228739003\\
1.93685413488996	-2.88\\
1.93	-2.88699262899263\\
1.92703649045945	-2.89\\
1.92	-2.89712230131871\\
1.91714174770786	-2.9\\
1.91	-2.90717192410662\\
1.90716866136043	-2.91\\
1.9	-2.9171421107628\\
1.89711595917937	-2.92\\
1.89	-2.92703346844239\\
1.88698234123011	-2.93\\
1.88	-2.93684659812868\\
1.87676647912348	-2.94\\
1.87	-2.94658209471137\\
1.8664670152329	-2.95\\
1.86	-2.95624054706356\\
1.85608256188549	-2.96\\
1.85	-2.96582253811762\\
1.84561170052626	-2.97\\
1.84	-2.97532864493997\\
1.83505298085423	-2.98\\
1.83	-2.98475943880461\\
1.82440491992937	-2.99\\
1.82	-2.99411548526567\\
1.81366600124919	-3\\
};

\addplot[only marks, mark=o, mark options={}, mark size=1.1180pt, draw=mycolor1] table[row sep=crcr]{%
x	y\\
0.0163484477750558	-0.800669299172575\\
};

\addplot[only marks, mark=o, mark options={}, mark size=1.5811pt, draw=mycolor2] table[row sep=crcr]{%
x	y\\
0.0277313296101314	-0.600962441098888\\
};

\addplot[only marks, mark=o, mark options={}, mark size=1.9365pt, draw=mycolor3] table[row sep=crcr]{%
x	y\\
0.0361050920358772	-0.401087299895024\\
};

\addplot[only marks, mark=o, mark options={}, mark size=2.2361pt, draw=mycolor4] table[row sep=crcr]{%
x	y\\
0.0425217910698668	-0.201109096223154\\
};

\addplot[only marks, mark=o, mark options={}, mark size=2.5000pt, draw=mycolor5] table[row sep=crcr]{%
x	y\\
0.0475933848935667	-0.0011310556937616\\
};

\addplot[only marks, mark=o, mark options={}, mark size=2.7386pt, draw=mycolor6] table[row sep=crcr]{%
x	y\\
0.0476190476190476	0\\
};

\addplot[only marks, mark=o, mark options={}, mark size=2.9580pt, draw=mycolor7] table[row sep=crcr]{%
x	y\\
0.0476190476190476	0\\
};

\addplot[only marks, mark=o, mark options={}, mark size=3.1623pt, draw=mycolor1] table[row sep=crcr]{%
x	y\\
0.0476190476190476	0\\
};

\addplot[only marks, mark=o, mark options={}, mark size=3.3541pt, draw=mycolor2] table[row sep=crcr]{%
x	y\\
0.0476190476190476	0\\
};

\addplot[only marks, mark=o, mark options={}, mark size=3.5355pt, draw=mycolor3] table[row sep=crcr]{%
x	y\\
0.0476190476190476	0\\
};

\end{axis}

\end{tikzpicture}%
            \caption{The contour lines of the quadratic model at point $x = (0, -1)^T$ and the family of solutions of trust region subproblem with $\Delta=0.2, 0.4, \dotsc, 2$, where a larger circle symbolizes a greater $\Delta$.}
            \label{fig:1}
        \end{figure}
        \begin{figure}[htb]
            \centering
            % This file was created by matlab2tikz.
%
%The latest updates can be retrieved from
%  http://www.mathworks.com/matlabcentral/fileexchange/22022-matlab2tikz-matlab2tikz
%where you can also make suggestions and rate matlab2tikz.
%
\definecolor{mycolor1}{rgb}{0.00000,0.44700,0.74100}%
\definecolor{mycolor2}{rgb}{0.85000,0.32500,0.09800}%
\definecolor{mycolor3}{rgb}{0.92900,0.69400,0.12500}%
\definecolor{mycolor4}{rgb}{0.49400,0.18400,0.55600}%
\definecolor{mycolor5}{rgb}{0.46600,0.67400,0.18800}%
\definecolor{mycolor6}{rgb}{0.30100,0.74500,0.93300}%
\definecolor{mycolor7}{rgb}{0.63500,0.07800,0.18400}%
%
\begin{tikzpicture}

\begin{axis}[%
width=4.069in,
height=3.566in,
at={(0.758in,0.481in)},
scale only axis,
point meta min=-28,
point meta max=78.4545363636364,
colormap={mymap}{[1pt] rgb(0pt)=(0,0,0.666667); rgb(1pt)=(0,0,1); rgb(4pt)=(0,1,1); rgb(7pt)=(1,1,0); rgb(9pt)=(1,0.333333,0)},
xmin=-2,
xmax=2,
ymin=-1.5,
ymax=2.5,
axis background/.style={fill=white},
axis x line*=bottom,
axis y line*=left,
legend style={legend cell align=left, align=left, draw=white!15!black}
]
\addplot[contour prepared, contour prepared format=matlab, contour/labels=false] table[row sep=crcr] {%
%
-27.1717181818182	272\\
1.95132342516317	-1.5\\
1.95	-1.49752672453175\\
1.94595293655277	-1.49\\
1.94063005084621	-1.48\\
1.94	-1.4788041958042\\
1.9353385333103	-1.47\\
1.93009824007109	-1.46\\
1.93	-1.45981056639695\\
1.92488726280541	-1.45\\
1.92	-1.44052284752285\\
1.91972904255054	-1.44\\
1.91460154049405	-1.43\\
1.91	-1.42092874692875\\
1.90952654124602	-1.42\\
1.90448382975525	-1.41\\
1.9	-1.40101004520342\\
1.89949371868785	-1.4\\
1.89453664308552	-1.39\\
1.89	-1.3807447354905\\
1.88963312501581	-1.38\\
1.8847625433292	-1.37\\
1.88	-1.36010877561745\\
1.87994736172918	-1.36\\
1.87516414494596	-1.35\\
1.87043687221869	-1.34\\
1.87	-1.33906477953185\\
1.86574411531679	-1.33\\
1.86110537480512	-1.32\\
1.86	-1.31758784160625\\
1.8565051760898	-1.31\\
1.85195586837576	-1.3\\
1.85	-1.29564665523156\\
1.8474501045674	-1.29\\
1.84299114404482	-1.28\\
1.84	-1.27320527859237\\
1.83858173513612	-1.27\\
1.83421405029455	-1.26\\
1.83	-1.25022396146899\\
1.82990296074084	-1.25\\
1.82562749445676	-1.24\\
1.82140941698187	-1.23\\
1.82	-1.22661253918495\\
1.8172344442405	-1.22\\
1.81310984083698	-1.21\\
1.81	-1.20235332043843\\
1.80903792930757	-1.2\\
1.80500777563984	-1.19\\
1.80103560979467	-1.18\\
1.8	-1.17735420875421\\
1.79710632998596	-1.17\\
1.79322990911741	-1.16\\
1.79	-1.15154059680777\\
1.78940867804917	-1.15\\
1.78562901206616	-1.14\\
1.78190794555576	-1.13\\
1.78	-1.12479054545454\\
1.77823617619494	-1.12\\
1.77461320123176	-1.11\\
1.77104913643058	-1.1\\
1.77	-1.09700687547746\\
1.76753089078526	-1.09\\
1.76406627722293	-1.08\\
1.76066088782405	-1.07\\
1.76	-1.06802493966211\\
1.75730033289128	-1.06\\
1.75399577796422	-1.05\\
1.75075076456738	-1.04\\
1.75	-1.03764316057774\\
1.74755217545101	-1.03\\
1.74440940382575	-1.02\\
1.74132649451716	-1.01\\
1.74	-1.00561206120612\\
1.73829425765327	-1\\
1.73531502202282	-0.99\\
1.73239597297075	-0.98\\
1.73	-0.971619138755981\\
1.729534589465	-0.97\\
1.7267206711594	-0.96\\
1.72396726722597	-0.95\\
1.7212743776647	-0.94\\
1.72	-0.935158829676072\\
1.71863456592085	-0.93\\
1.71604862129048	-0.92\\
1.71352352241613	-0.91\\
1.71105926929778	-0.9\\
1.71	-0.895592635212888\\
1.70864846074696	-0.89\\
1.70629300036152	-0.88\\
1.70399872076532	-0.87\\
1.70176562195834	-0.86\\
1.7	-0.851870678617158\\
1.69959145437767	-0.85\\
1.6974690305081	-0.84\\
1.69540812617097	-0.83\\
1.69340874136629	-0.82\\
1.69147087609407	-0.81\\
1.69	-0.802160953800297\\
1.68959227286787	-0.8\\
1.68776734246267	-0.79\\
1.68600427410511	-0.78\\
1.68430306779518	-0.77\\
1.68266372353289	-0.76\\
1.68108624131822	-0.75\\
1.68	-0.74283302411874\\
1.67956821716386	-0.74\\
1.67810631980772	-0.73\\
1.67670663084971	-0.72\\
1.67536915028983	-0.71\\
1.67409387812809	-0.7\\
1.67288081436448	-0.69\\
1.67172995899901	-0.68\\
1.67064131203167	-0.67\\
1.67	-0.663752066115701\\
1.66961270509284	-0.66\\
1.66864304603748	-0.65\\
1.66773594563085	-0.64\\
1.66689140387295	-0.63\\
1.66610942076378	-0.62\\
1.66538999630335	-0.61\\
1.66473313049165	-0.6\\
1.66413882332869	-0.59\\
1.66360707481446	-0.58\\
1.66313788494896	-0.57\\
1.66273125373219	-0.56\\
1.66238718116416	-0.55\\
1.66210566724486	-0.54\\
1.66188671197429	-0.53\\
1.66173031535246	-0.52\\
1.66163647737936	-0.51\\
1.66160519805499	-0.5\\
1.66163647737936	-0.49\\
1.66173031535246	-0.48\\
1.66188671197429	-0.47\\
1.66210566724486	-0.46\\
1.66238718116416	-0.45\\
1.66273125373219	-0.44\\
1.66313788494896	-0.43\\
1.66360707481446	-0.42\\
1.66413882332869	-0.41\\
1.66473313049165	-0.4\\
1.66538999630335	-0.39\\
1.66610942076378	-0.38\\
1.66689140387295	-0.37\\
1.66773594563085	-0.36\\
1.66864304603748	-0.35\\
1.66961270509284	-0.34\\
1.67	-0.336247933884298\\
1.67064131203167	-0.33\\
1.67172995899901	-0.32\\
1.67288081436448	-0.31\\
1.67409387812809	-0.3\\
1.67536915028983	-0.29\\
1.67670663084971	-0.28\\
1.67810631980772	-0.27\\
1.67956821716386	-0.26\\
1.68	-0.25716697588126\\
1.68108624131822	-0.25\\
1.68266372353289	-0.24\\
1.68430306779518	-0.23\\
1.68600427410511	-0.22\\
1.68776734246267	-0.21\\
1.68959227286787	-0.2\\
1.69	-0.197839046199703\\
1.69147087609407	-0.19\\
1.69340874136629	-0.18\\
1.69540812617097	-0.17\\
1.6974690305081	-0.16\\
1.69959145437767	-0.15\\
1.7	-0.148129321382842\\
1.70176562195834	-0.14\\
1.70399872076532	-0.13\\
1.70629300036152	-0.12\\
1.70864846074696	-0.11\\
1.71	-0.104407364787112\\
1.71105926929778	-0.1\\
1.71352352241613	-0.0900000000000001\\
1.71604862129048	-0.0800000000000001\\
1.71863456592085	-0.0700000000000001\\
1.72	-0.0648411703239285\\
1.7212743776647	-0.0600000000000001\\
1.72396726722597	-0.05\\
1.7267206711594	-0.04\\
1.729534589465	-0.03\\
1.73	-0.0283808612440191\\
1.73239597297075	-0.02\\
1.73531502202282	-0.01\\
1.73829425765327	0\\
1.74	0.00561206120612121\\
1.74132649451716	0.01\\
1.74440940382575	0.02\\
1.74755217545101	0.03\\
1.75	0.03764316057774\\
1.75075076456738	0.04\\
1.75399577796422	0.05\\
1.75730033289128	0.0600000000000001\\
1.76	0.0680249396621075\\
1.76066088782405	0.0700000000000001\\
1.76406627722293	0.0800000000000001\\
1.76753089078526	0.0900000000000001\\
1.77	0.0970068754774638\\
1.77104913643058	0.1\\
1.77461320123176	0.11\\
1.77823617619494	0.12\\
1.78	0.124790545454545\\
1.78190794555576	0.13\\
1.78562901206616	0.14\\
1.78940867804917	0.15\\
1.79	0.151540596807772\\
1.79322990911741	0.16\\
1.79710632998596	0.17\\
1.8	0.177354208754208\\
1.80103560979467	0.18\\
1.80500777563984	0.19\\
1.80903792930757	0.2\\
1.81	0.202353320438427\\
1.81310984083698	0.21\\
1.8172344442405	0.22\\
1.82	0.226612539184953\\
1.82140941698187	0.23\\
1.82562749445676	0.24\\
1.82990296074084	0.25\\
1.83	0.250223961468995\\
1.83421405029455	0.26\\
1.83858173513612	0.27\\
1.84	0.273205278592375\\
1.84299114404482	0.28\\
1.8474501045674	0.29\\
1.85	0.295646655231561\\
1.85195586837576	0.3\\
1.8565051760898	0.31\\
1.86	0.317587841606246\\
1.86110537480512	0.32\\
1.86574411531679	0.33\\
1.87	0.339064779531846\\
1.87043687221869	0.34\\
1.87516414494596	0.35\\
1.87994736172918	0.36\\
1.88	0.360108775617446\\
1.8847625433292	0.37\\
1.88963312501581	0.38\\
1.89	0.380744735490498\\
1.89453664308552	0.39\\
1.89949371868785	0.4\\
1.9	0.401010045203415\\
1.90448382975525	0.41\\
1.90952654124602	0.42\\
1.91	0.420928746928747\\
1.91460154049405	0.43\\
1.91972904255054	0.44\\
1.92	0.440522847522847\\
1.92488726280541	0.45\\
1.93	0.459810566396954\\
1.93009824007109	0.46\\
1.9353385333103	0.47\\
1.94	0.478804195804196\\
1.94063005084621	0.48\\
1.94595293655277	0.49\\
1.95	0.49752672453175\\
1.95132342516317	0.5\\
1.95672810384023	0.51\\
1.96	0.515994178235557\\
1.96217602841365	0.52\\
1.96766171211716	0.53\\
1.97	0.534221343873518\\
1.97318557075415	0.54\\
1.97875148287132	0.55\\
1.98	0.552221887117622\\
1.98434980603812	0.56\\
1.98999518107125	0.57\\
1.99	0.57000845665962\\
1.99566653077864	0.58\\
2	0.587570590699623\\
-27.1717181818182	147\\
-2	0.118722099039172\\
-1.99683627784778	0.11\\
-1.99326800890057	0.1\\
-1.99	0.0906875477463713\\
-1.98975743201358	0.0900000000000001\\
-1.98628870980783	0.0800000000000001\\
-1.98287928199876	0.0700000000000001\\
-1.98	0.0614054706355589\\
-1.97952662240889	0.0600000000000001\\
-1.97621812762498	0.05\\
-1.97296924535971	0.04\\
-1.97	0.0306898895497025\\
-1.96977878878688	0.03\\
-1.96663224997957	0.02\\
-1.96354564524478	0.01\\
-1.9605189745825	0\\
-1.96	-0.00174931129476633\\
-1.95753896299543	-0.01\\
-1.95461639595716	-0.02\\
-1.95175408803309	-0.03\\
-1.95	-0.0362600195503428\\
-1.94894632480516	-0.04\\
-1.94618958442345	-0.05\\
-1.94349343174245	-0.0600000000000001\\
-1.94085786676214	-0.0700000000000001\\
-1.94	-0.0733315508021391\\
-1.93827347492593	-0.0800000000000001\\
-1.93574529947664	-0.0900000000000001\\
-1.9332780439177	-0.1\\
-1.93087170824911	-0.11\\
-1.93	-0.113716646989374\\
-1.92851816789642	-0.12\\
-1.9262210775442	-0.13\\
-1.92398524293471	-0.14\\
-1.92181066406794	-0.15\\
-1.92	-0.158567852437418\\
-1.9196956631296	-0.16\\
-1.91763221995128	-0.17\\
-1.91563037209172	-0.18\\
-1.91369011955091	-0.19\\
-1.91181146232886	-0.2\\
-1.91	-0.209969183359013\\
-1.90999436921084	-0.21\\
-1.90822911681072	-0.22\\
-1.9065258030913	-0.23\\
-1.90488442805259	-0.24\\
-1.90330499169459	-0.25\\
-1.90178749401729	-0.26\\
-1.90033193502069	-0.27\\
-1.9	-0.272381818181818\\
-1.89893236318338	-0.28\\
-1.8975932165001	-0.29\\
-1.89631635570907	-0.3\\
-1.89510178081028	-0.31\\
-1.89394949180374	-0.32\\
-1.89285948868945	-0.33\\
-1.8918317714674	-0.34\\
-1.8908663401376	-0.35\\
-1.89	-0.359592476489029\\
-1.88996298721635	-0.36\\
-1.8891173874669	-0.37\\
-1.88833442473593	-0.38\\
-1.88761409902343	-0.39\\
-1.88695641032941	-0.4\\
-1.88636135865387	-0.41\\
-1.88582894399681	-0.42\\
-1.88535916635823	-0.43\\
-1.88495202573812	-0.44\\
-1.88460752213649	-0.45\\
-1.88432565555334	-0.46\\
-1.88410642598867	-0.47\\
-1.88394983344247	-0.48\\
-1.88385587791476	-0.49\\
-1.88382455940552	-0.5\\
-1.88385587791476	-0.51\\
-1.88394983344247	-0.52\\
-1.88410642598867	-0.53\\
-1.88432565555334	-0.54\\
-1.88460752213649	-0.55\\
-1.88495202573812	-0.56\\
-1.88535916635823	-0.57\\
-1.88582894399681	-0.58\\
-1.88636135865387	-0.59\\
-1.88695641032941	-0.6\\
-1.88761409902343	-0.61\\
-1.88833442473593	-0.62\\
-1.8891173874669	-0.63\\
-1.88996298721635	-0.64\\
-1.89	-0.640407523510971\\
-1.8908663401376	-0.65\\
-1.8918317714674	-0.66\\
-1.89285948868945	-0.67\\
-1.89394949180374	-0.68\\
-1.89510178081028	-0.69\\
-1.89631635570907	-0.7\\
-1.8975932165001	-0.71\\
-1.89893236318338	-0.72\\
-1.9	-0.727618181818182\\
-1.90033193502069	-0.73\\
-1.90178749401729	-0.74\\
-1.90330499169459	-0.75\\
-1.90488442805259	-0.76\\
-1.9065258030913	-0.77\\
-1.90822911681072	-0.78\\
-1.90999436921084	-0.79\\
-1.91	-0.790030816640986\\
-1.91181146232886	-0.8\\
-1.91369011955091	-0.81\\
-1.91563037209172	-0.82\\
-1.91763221995128	-0.83\\
-1.9196956631296	-0.84\\
-1.92	-0.841432147562583\\
-1.92181066406794	-0.85\\
-1.92398524293471	-0.86\\
-1.9262210775442	-0.87\\
-1.92851816789642	-0.88\\
-1.93	-0.886283353010626\\
-1.93087170824911	-0.89\\
-1.9332780439177	-0.9\\
-1.93574529947664	-0.91\\
-1.93827347492593	-0.92\\
-1.94	-0.926668449197861\\
-1.94085786676214	-0.93\\
-1.94349343174245	-0.94\\
-1.94618958442345	-0.95\\
-1.94894632480516	-0.96\\
-1.95	-0.963739980449657\\
-1.95175408803309	-0.97\\
-1.95461639595716	-0.98\\
-1.95753896299543	-0.99\\
-1.96	-0.998250688705234\\
-1.9605189745825	-1\\
-1.96354564524478	-1.01\\
-1.96663224997957	-1.02\\
-1.96977878878688	-1.03\\
-1.97	-1.0306898895497\\
-1.97296924535971	-1.04\\
-1.97621812762498	-1.05\\
-1.97952662240889	-1.06\\
-1.98	-1.06140547063556\\
-1.98287928199876	-1.07\\
-1.98628870980783	-1.08\\
-1.98975743201358	-1.09\\
-1.99	-1.09068754774637\\
-1.99326800890057	-1.1\\
-1.99683627784778	-1.11\\
-2	-1.11872209903917\\
-15.3434363636364	366\\
1.60334853880338	-1.5\\
1.6	-1.49480219278209\\
1.59688819158699	-1.49\\
1.59047334300346	-1.48\\
1.59	-1.47925454545455\\
1.58408897358783	-1.47\\
1.58	-1.46353179463024\\
1.57775408117418	-1.46\\
1.57146083570466	-1.45\\
1.57	-1.44765416065416\\
1.56520505017025	-1.44\\
1.56	-1.43160233349538\\
1.55900087908091	-1.43\\
1.55283215617327	-1.42\\
1.55	-1.41535866865375\\
1.54671037784758	-1.41\\
1.54063858985697	-1.4\\
1.54	-1.39893651599797\\
1.53460129475654	-1.39\\
1.53	-1.38229738058552\\
1.52861923136286	-1.38\\
1.52267695632042	-1.37\\
1.52	-1.3654429847609\\
1.51678293008665	-1.36\\
1.51094077087927	-1.35\\
1.51	-1.34837062937063\\
1.50513703553236	-1.34\\
1.5	-1.33105171475231\\
1.49939247193484	-1.33\\
1.493685104242	-1.32\\
1.49	-1.31346402677078\\
1.48803468025187	-1.31\\
1.4824307818878	-1.3\\
1.48	-1.29560777587193\\
1.47687693042509	-1.29\\
1.47137780608183	-1.28\\
1.47	-1.27746217008798\\
1.4659230300894	-1.27\\
1.46053000929279	-1.26\\
1.46	-1.25900421432872\\
1.45517688413041	-1.25\\
1.45	-1.24020561317877\\
1.44989062347863	-1.24\\
1.44464249780922	-1.23\\
1.44	-1.22103197492163\\
1.43946231992944	-1.22\\
1.43432398000849	-1.21\\
1.43	-1.20146550612508\\
1.42925267137925	-1.2\\
1.42422554660529	-1.19\\
1.42	-1.18147179827472\\
1.41926597610617	-1.18\\
1.41435152397657	-1.17\\
1.41	-1.16101230348599\\
1.409506645791	-1.16\\
1.40470635264333	-1.15\\
1.4	-1.14004369274137\\
1.39997920928205	-1.14\\
1.39529459105999	-1.13\\
1.39068374635324	-1.12\\
1.39	-1.11849297856615\\
1.38612091955575	-1.11\\
1.38162778921784	-1.1\\
1.38	-1.09631627196333\\
1.377190144435	-1.09\\
1.3728163126593	-1.08\\
1.37	-1.07344901185771\\
1.3685072022445	-1.07\\
1.36425428542101	-1.06\\
1.36007664146166	-1.05\\
1.36	-1.04981317764804\\
1.35594681180888	-1.04\\
1.35189224568535	-1.03\\
1.35	-1.02524415584416\\
1.34789913634629	-1.02\\
1.34396933855919	-1.01\\
1.34011584752523	-1\\
1.34	-0.999693296602387\\
1.33631334474208	-0.99\\
1.33258687528376	-0.98\\
1.33	-0.972911961722488\\
1.32892983998593	-0.97\\
1.3253321610691	-0.96\\
1.32181185159135	-0.95\\
1.32	-0.944737487231869\\
1.31835747423593	-0.94\\
1.3149683039983	-0.93\\
1.31165704572015	-0.92\\
1.31	-0.914875136911282\\
1.30841256820857	-0.91\\
1.3052348514569	-0.9\\
1.30213559684725	-0.89\\
1.3	-0.882930342384887\\
1.29910850903344	-0.88\\
1.29614525340325	-0.87\\
1.29326101792321	-0.86\\
1.2904558025933	-0.85\\
1.29	-0.848328063241107\\
1.28771334515067	-0.84\\
1.28504720905835	-0.83\\
1.28246065911804	-0.82\\
1.28	-0.810184704184703\\
1.27995336126799	-0.81\\
1.27750847148843	-0.8\\
1.27514374202951	-0.79\\
1.27285917289124	-0.78\\
1.2706547640736	-0.77\\
1.27	-0.766917667238422\\
1.26851983704628	-0.76\\
1.26646089477741	-0.75\\
1.26448269534261	-0.74\\
1.26258523874188	-0.73\\
1.26076852497523	-0.72\\
1.26	-0.715572938689216\\
1.25902547229103	-0.71\\
1.25735812784206	-0.7\\
1.25577211726866	-0.69\\
1.25426744057082	-0.68\\
1.25284409774853	-0.67\\
1.2515020888018	-0.66\\
1.25024141373064	-0.65\\
1.25	-0.647952978056426\\
1.24905515623254	-0.64\\
1.24794905217683	-0.63\\
1.24692488175487	-0.62\\
1.24598264496667	-0.61\\
1.24512234181222	-0.6\\
1.24434397229153	-0.59\\
1.2436475364046	-0.58\\
1.24303303415143	-0.57\\
1.24250046553201	-0.56\\
1.24204983054635	-0.55\\
1.24168112919444	-0.54\\
1.2413943614763	-0.53\\
1.2411895273919	-0.52\\
1.24106662694127	-0.51\\
1.24102566012439	-0.5\\
1.24106662694127	-0.49\\
1.2411895273919	-0.48\\
1.2413943614763	-0.47\\
1.24168112919444	-0.46\\
1.24204983054635	-0.45\\
1.24250046553201	-0.44\\
1.24303303415143	-0.43\\
1.2436475364046	-0.42\\
1.24434397229153	-0.41\\
1.24512234181222	-0.4\\
1.24598264496667	-0.39\\
1.24692488175487	-0.38\\
1.24794905217683	-0.37\\
1.24905515623254	-0.36\\
1.25	-0.352047021943574\\
1.25024141373064	-0.35\\
1.2515020888018	-0.34\\
1.25284409774853	-0.33\\
1.25426744057082	-0.32\\
1.25577211726866	-0.31\\
1.25735812784206	-0.3\\
1.25902547229103	-0.29\\
1.26	-0.284427061310783\\
1.26076852497523	-0.28\\
1.26258523874188	-0.27\\
1.26448269534261	-0.26\\
1.26646089477741	-0.25\\
1.26851983704628	-0.24\\
1.27	-0.233082332761578\\
1.2706547640736	-0.23\\
1.27285917289124	-0.22\\
1.27514374202951	-0.21\\
1.27750847148843	-0.2\\
1.27995336126799	-0.19\\
1.28	-0.189815295815297\\
1.28246065911804	-0.18\\
1.28504720905835	-0.17\\
1.28771334515067	-0.16\\
1.29	-0.151671936758893\\
1.2904558025933	-0.15\\
1.29326101792321	-0.14\\
1.29614525340325	-0.13\\
1.29910850903344	-0.12\\
1.3	-0.117069657615113\\
1.30213559684725	-0.11\\
1.3052348514569	-0.1\\
1.30841256820857	-0.0900000000000001\\
1.31	-0.0851248630887183\\
1.31165704572015	-0.0800000000000001\\
1.3149683039983	-0.0700000000000001\\
1.31835747423593	-0.0600000000000001\\
1.32	-0.0552625127681313\\
1.32181185159135	-0.05\\
1.32533216106911	-0.04\\
1.32892983998593	-0.03\\
1.33	-0.0270880382775119\\
1.33258687528376	-0.02\\
1.33631334474208	-0.01\\
1.34	-0.000306703397613014\\
1.34011584752523	0\\
1.34396933855919	0.01\\
1.34789913634629	0.02\\
1.35	0.0252441558441562\\
1.35189224568535	0.03\\
1.35594681180888	0.04\\
1.36	0.0498131776480396\\
1.36007664146166	0.05\\
1.36425428542101	0.0600000000000001\\
1.3685072022445	0.0700000000000001\\
1.37	0.0734490118577077\\
1.3728163126593	0.0800000000000001\\
1.377190144435	0.0900000000000001\\
1.38	0.0963162719633306\\
1.38162778921784	0.1\\
1.38612091955575	0.11\\
1.39	0.118492978566149\\
1.39068374635324	0.12\\
1.39529459105999	0.13\\
1.39997920928205	0.14\\
1.4	0.140043692741367\\
1.40470635264333	0.15\\
1.409506645791	0.16\\
1.41	0.161012303485988\\
1.41435152397657	0.17\\
1.41926597610617	0.18\\
1.42	0.181471798274718\\
1.42422554660529	0.19\\
1.42925267137925	0.2\\
1.43	0.201465506125081\\
1.43432398000849	0.21\\
1.43946231992944	0.22\\
1.44	0.22103197492163\\
1.44464249780922	0.23\\
1.44989062347863	0.24\\
1.45	0.240205613178768\\
1.45517688413041	0.25\\
1.46	0.259004214328717\\
1.46053000929279	0.26\\
1.4659230300894	0.27\\
1.47	0.277462170087977\\
1.47137780608183	0.28\\
1.47687693042509	0.29\\
1.48	0.295607775871927\\
1.4824307818878	0.3\\
1.48803468025187	0.31\\
1.49	0.313464026770776\\
1.493685104242	0.32\\
1.49939247193484	0.33\\
1.5	0.331051714752314\\
1.50513703553236	0.34\\
1.51	0.348370629370629\\
1.51094077087927	0.35\\
1.51678293008665	0.36\\
1.52	0.365442984760904\\
1.52267695632042	0.37\\
1.52861923136286	0.38\\
1.53	0.382297380585516\\
1.53460129475654	0.39\\
1.54	0.398936515997969\\
1.54063858985697	0.4\\
1.54671037784758	0.41\\
1.55	0.41535866865375\\
1.55283215617327	0.42\\
1.55900087908091	0.43\\
1.56	0.431602333495382\\
1.56520505017025	0.44\\
1.57	0.44765416065416\\
1.57146083570466	0.45\\
1.57775408117418	0.46\\
1.58	0.463531794630241\\
1.58408897358783	0.47\\
1.59	0.479254545454545\\
1.59047334300346	0.48\\
1.59688819158699	0.49\\
1.6	0.494802192782093\\
1.60334853880338	0.5\\
1.60985549898467	0.51\\
1.61	0.510219883564711\\
1.61638996986277	0.52\\
1.62	0.525471396895787\\
1.62297076363636	0.53\\
1.62959476363636	0.54\\
1.63	0.540605915615485\\
1.63624678217106	0.55\\
1.64	0.555590693666523\\
1.64294325731213	0.56\\
1.64968163124443	0.57\\
1.65	0.570468076109936\\
1.65644656695931	0.58\\
1.66	0.585205697528278\\
1.6632541871641	0.59\\
1.67	0.599847654628476\\
1.67010377491871	0.6\\
1.67697780291248	0.61\\
1.68	0.61435711373828\\
1.68389224756067	0.62\\
1.69	0.628776161616161\\
1.69084701210816	0.63\\
1.69782947904141	0.64\\
1.7	0.643081381500596\\
1.70484663088517	0.65\\
1.71	0.65729279811098\\
1.71190253616174	0.66\\
1.71899106673673	0.67\\
1.72	0.671411218568666\\
1.7261070004126	0.68\\
1.73	0.685428845416187\\
1.73326019752141	0.69\\
1.74	0.699370863446178\\
1.7404500557808	0.7\\
1.74766346493973	0.71\\
1.75	0.713212495323606\\
1.7549102817397	0.72\\
1.76	0.726978107606679\\
1.76219237085094	0.73\\
1.76950655503809	0.74\\
1.77	0.740669222343921\\
1.7768434864105	0.75\\
1.78	0.754269467584208\\
1.78421436752524	0.76\\
1.79	0.767804886812792\\
1.79161866405236	0.77\\
1.79905089955221	0.78\\
1.8	0.781267067562787\\
1.80650728801497	0.79\\
1.81	0.794651105651106\\
1.81399585704172	0.8\\
1.82	0.80797561825148\\
1.8215161079953	0.81\\
1.82906273640276	0.82\\
1.83	0.821232590051458\\
1.83663249682091	0.83\\
1.84	0.834418113721484\\
1.84423278510754	0.84\\
1.85	0.847548834065563\\
1.85186313545172	0.85\\
1.85952066582753	0.86\\
1.86	0.860621378621378\\
1.86719809849976	0.87\\
1.87	0.873624132231405\\
1.87490451366815	0.88\\
1.88	0.886576304561864\\
1.8826394757483	0.89\\
1.89	0.899478983382209\\
1.89040255784094	0.9\\
1.8981843357418	0.91\\
1.9	0.912316736267266\\
1.90599163305694	0.92\\
1.91	0.925103987240829\\
1.91382605877807	0.93\\
1.92	0.937845422869813\\
1.92168721319608	0.94\\
1.92957261565174	0.95\\
1.93	0.950538269290847\\
1.93747587194234	0.96\\
1.94	0.963172820353708\\
1.9454049274152	0.97\\
1.95	0.975764869029276\\
1.95335940746535	0.98\\
1.96	0.988315273951637\\
1.96133894470528	0.99\\
1.96934001508259	1\\
1.97	1.00081938991241\\
1.97735915748699	1.01\\
1.98	1.01327272727273\\
1.98540250102402	1.02\\
1.99	1.02568733233979\\
1.9934697009664	1.03\\
2	1.03806396209654\\
-15.3434363636364	316\\
-2	0.751257515392974\\
-1.99906919385539	0.75\\
-1.9917262271789	0.74\\
-1.99	0.73763010673537\\
-1.9844125811929	0.73\\
-1.98	0.723925046382189\\
-1.97713372171792	0.72\\
-1.97	0.710150766928545\\
-1.96989021167625	0.71\\
-1.96266815593756	0.7\\
-1.96	0.696274629136554\\
-1.95548166205593	0.69\\
-1.95	0.682323360184119\\
-1.94833191044036	0.68\\
-1.94121285560849	0.67\\
-1.94	0.668281701131487\\
-1.93412233821615	0.66\\
-1.93	0.654141282959465\\
-1.92707002645134	0.65\\
-1.92005624390923	0.64\\
-1.92	0.639919102923508\\
-1.91306548702299	0.63\\
-1.91	0.625576161616162\\
-1.90611447394352	0.62\\
-1.9	0.611146351406441\\
-1.89920387305003	0.61\\
-1.89232128195691	0.6\\
-1.89	0.596596513075965\\
-1.88547561426985	0.59\\
-1.88	0.581943024717218\\
-1.87867201145311	0.58\\
-1.87190035790981	0.57\\
-1.87	0.567167306871532\\
-1.86516428139488	0.56\\
-1.86	0.552273158121499\\
-1.85847200903588	0.55\\
-1.85181383764372	0.54\\
-1.85	0.53724945103206\\
-1.84519180915208	0.53\\
-1.84	0.52209578713969\\
-1.83861541616617	0.52\\
-1.83207336009141	0.51\\
-1.83	0.506799185888738\\
-1.82557006040961	0.5\\
-1.82	0.491365006852444\\
-1.81911432551798	0.49\\
-1.81269111065003	0.48\\
-1.81	0.475767365967366\\
-1.80631145830228	0.47\\
-1.8	0.460029203956665\\
-1.79998140427702	0.46\\
-1.79367985363366	0.45\\
-1.79	0.444098605098605\\
-1.7874290197025	0.44\\
-1.78122258093715	0.43\\
-1.78	0.428008845208845\\
-1.77505296706632	0.42\\
-1.77	0.411730253353204\\
-1.7689363911198	0.41\\
-1.76285644486518	0.4\\
-1.76	0.39524936515998\\
-1.75682448001475	0.39\\
-1.75084272942333	0.38\\
-1.75	0.378575064935065\\
-1.74489753021545	0.37\\
-1.74	0.361674198633736\\
-1.73900911268	0.36\\
-1.73315895872858	0.35\\
-1.73	0.344536309844002\\
-1.72736097637302	0.34\\
-1.72161226725082	0.33\\
-1.72	0.327161432506887\\
-1.71590704411626	0.32\\
-1.71026104480902	0.31\\
-1.71	0.309531902879729\\
-1.70465097119681	0.3\\
-1.7	0.291607775871927\\
-1.69910334491885	0.29\\
-1.69359650521348	0.28\\
-1.69	0.273384750733138\\
-1.68814812437827	0.27\\
-1.6827474890094	0.26\\
-1.68	0.254845273931366\\
-1.67740093654126	0.25\\
-1.6721078637171	0.24\\
-1.67	0.235963512677798\\
-1.66686579777034	0.23\\
-1.66168167192121	0.22\\
-1.66	0.216710743801653\\
-1.65654682848638	0.21\\
-1.65147306094409	0.2\\
-1.65	0.197054937867887\\
-1.64644825656053	0.19\\
-1.64148628625992	0.18\\
-1.64	0.176960269360269\\
-1.63657442083982	0.17\\
-1.63172571504325	0.16\\
-1.63	0.156386537126995\\
-1.62692977481234	0.15\\
-1.62219582985822	0.14\\
-1.62	0.135288475304224\\
-1.61751889041878	0.13\\
-1.61290123249488	0.12\\
-1.61	0.113614929785661\\
-1.60834646201697	0.11\\
-1.6038466479597	0.1\\
-1.6	0.0913078686019867\\
-1.59941731050679	0.0900000000000001\\
-1.59503692862734	0.0800000000000001\\
-1.59073142507062	0.0700000000000001\\
-1.59	0.068271118262269\\
-1.58647705856149	0.0600000000000001\\
-1.58229311585512	0.05\\
-1.58	0.0444186822351963\\
-1.57817215801277	0.04\\
-1.57411143695015	0.03\\
-1.5701266172158	0.02\\
-1.57	0.0196760812003531\\
-1.56619168374614	0.01\\
-1.56233230277556	0\\
-1.56	-0.00616528925619838\\
-1.55853929833153	-0.01\\
-1.55480709363741	-0.02\\
-1.55115184161741	-0.03\\
-1.55	-0.0332189638318668\\
-1.54755662005565	-0.04\\
-1.54403085484837	-0.05\\
-1.5405825789863	-0.0600000000000001\\
-1.54	-0.0617283176593522\\
-1.53719221083248	-0.0700000000000001\\
-1.53387578476927	-0.0800000000000001\\
-1.53063739226049	-0.0900000000000001\\
-1.53	-0.092016835016835\\
-1.52745918914092	-0.1\\
-1.52435506340418	-0.11\\
-1.52132952312913	-0.12\\
-1.52	-0.124511515151515\\
-1.51837104723531	-0.13\\
-1.51548224628557	-0.14\\
-1.51267259056733	-0.15\\
-1.51	-0.159787878787879\\
-1.50994166455306	-0.16\\
-1.50727127794485	-0.17\\
-1.50468060436972	-0.18\\
-1.50216964382768	-0.19\\
-1.5	-0.198923994038748\\
-1.49973650596694	-0.2\\
-1.49736797927083	-0.21\\
-1.49507974161527	-0.22\\
-1.49287179300026	-0.23\\
-1.49074413342579	-0.24\\
-1.49	-0.24363458110517\\
-1.48868727713855	-0.25\\
-1.48670587802816	-0.26\\
-1.48480535235084	-0.27\\
-1.48298570010661	-0.28\\
-1.48124692129545	-0.29\\
-1.48	-0.297521064301553\\
-1.47958600259211	-0.3\\
-1.47799740788743	-0.31\\
-1.47649027957786	-0.32\\
-1.4750646176634	-0.33\\
-1.47372042214405	-0.34\\
-1.47245769301981	-0.35\\
-1.47127643029069	-0.36\\
-1.47017663395667	-0.37\\
-1.47	-0.371734545454546\\
-1.46915208714142	-0.38\\
-1.46820830380125	-0.39\\
-1.46734658857761	-0.4\\
-1.46656694147051	-0.41\\
-1.46586936247995	-0.42\\
-1.46525385160592	-0.43\\
-1.46472040884844	-0.44\\
-1.46426903420748	-0.45\\
-1.46389972768307	-0.46\\
-1.46361248927519	-0.47\\
-1.46340731898385	-0.48\\
-1.46328421680904	-0.49\\
-1.46324318275077	-0.5\\
-1.46328421680904	-0.51\\
-1.46340731898385	-0.52\\
-1.46361248927519	-0.53\\
-1.46389972768307	-0.54\\
-1.46426903420748	-0.55\\
-1.46472040884844	-0.56\\
-1.46525385160592	-0.57\\
-1.46586936247995	-0.58\\
-1.46656694147051	-0.59\\
-1.46734658857761	-0.6\\
-1.46820830380125	-0.61\\
-1.46915208714142	-0.62\\
-1.47	-0.628265454545452\\
-1.47017663395667	-0.63\\
-1.47127643029069	-0.64\\
-1.47245769301981	-0.65\\
-1.47372042214405	-0.66\\
-1.4750646176634	-0.67\\
-1.47649027957786	-0.68\\
-1.47799740788743	-0.69\\
-1.47958600259211	-0.7\\
-1.48	-0.702478935698448\\
-1.48124692129545	-0.71\\
-1.48298570010661	-0.72\\
-1.48480535235084	-0.73\\
-1.48670587802816	-0.74\\
-1.48868727713855	-0.75\\
-1.49	-0.75636541889483\\
-1.49074413342579	-0.76\\
-1.49287179300026	-0.77\\
-1.49507974161527	-0.78\\
-1.49736797927083	-0.79\\
-1.49973650596694	-0.8\\
-1.5	-0.801076005961252\\
-1.50216964382768	-0.81\\
-1.50468060436972	-0.82\\
-1.50727127794485	-0.83\\
-1.50994166455306	-0.84\\
-1.51	-0.840212121212121\\
-1.51267259056733	-0.85\\
-1.51548224628557	-0.86\\
-1.51837104723531	-0.87\\
-1.52	-0.875488484848485\\
-1.52132952312913	-0.88\\
-1.52435506340418	-0.89\\
-1.52745918914092	-0.9\\
-1.53	-0.907983164983165\\
-1.53063739226049	-0.91\\
-1.53387578476927	-0.92\\
-1.53719221083248	-0.93\\
-1.54	-0.938271682340648\\
-1.5405825789863	-0.94\\
-1.54403085484837	-0.95\\
-1.54755662005565	-0.96\\
-1.55	-0.966781036168133\\
-1.55115184161741	-0.97\\
-1.55480709363741	-0.98\\
-1.55853929833153	-0.99\\
-1.56	-0.993834710743802\\
-1.56233230277556	-1\\
-1.56619168374614	-1.01\\
-1.57	-1.01967608120035\\
-1.5701266172158	-1.02\\
-1.57411143695015	-1.03\\
-1.57817215801277	-1.04\\
-1.58	-1.0444186822352\\
-1.58229311585512	-1.05\\
-1.58647705856149	-1.06\\
-1.59	-1.06827111826227\\
-1.59073142507062	-1.07\\
-1.59503692862734	-1.08\\
-1.59941731050679	-1.09\\
-1.6	-1.09130786860199\\
-1.6038466479597	-1.1\\
-1.60834646201697	-1.11\\
-1.61	-1.11361492978566\\
-1.61290123249488	-1.12\\
-1.61751889041878	-1.13\\
-1.62	-1.13528847530422\\
-1.62219582985822	-1.14\\
-1.62692977481234	-1.15\\
-1.63	-1.156386537127\\
-1.63172571504325	-1.16\\
-1.63657442083982	-1.17\\
-1.64	-1.17696026936027\\
-1.64148628625992	-1.18\\
-1.64644825656053	-1.19\\
-1.65	-1.19705493786789\\
-1.65147306094409	-1.2\\
-1.65654682848638	-1.21\\
-1.66	-1.21671074380165\\
-1.66168167192121	-1.22\\
-1.66686579777034	-1.23\\
-1.67	-1.2359635126778\\
-1.6721078637171	-1.24\\
-1.67740093654126	-1.25\\
-1.68	-1.25484527393137\\
-1.6827474890094	-1.26\\
-1.68814812437827	-1.27\\
-1.69	-1.27338475073314\\
-1.69359650521348	-1.28\\
-1.69910334491885	-1.29\\
-1.7	-1.29160777587193\\
-1.70465097119681	-1.3\\
-1.71	-1.30953190287973\\
-1.71026104480902	-1.31\\
-1.71590704411626	-1.32\\
-1.72	-1.32716143250689\\
-1.72161226725082	-1.33\\
-1.72736097637302	-1.34\\
-1.73	-1.344536309844\\
-1.73315895872858	-1.35\\
-1.73900911268	-1.36\\
-1.74	-1.36167419863374\\
-1.74489753021545	-1.37\\
-1.75	-1.37857506493506\\
-1.75084272942333	-1.38\\
-1.75682448001475	-1.39\\
-1.76	-1.39524936515998\\
-1.76285644486518	-1.4\\
-1.7689363911198	-1.41\\
-1.77	-1.4117302533532\\
-1.77505296706632	-1.42\\
-1.78	-1.42800884520885\\
-1.78122258093715	-1.43\\
-1.7874290197025	-1.44\\
-1.79	-1.44409860509861\\
-1.79367985363366	-1.45\\
-1.79998140427702	-1.46\\
-1.8	-1.46002920395667\\
-1.80631145830227	-1.47\\
-1.81	-1.47576736596737\\
-1.81269111065003	-1.48\\
-1.81911432551798	-1.49\\
-1.82	-1.49136500685244\\
-1.82557006040961	-1.5\\
-3.51515454545454	485\\
1.16368979301065	-1.5\\
1.16	-1.49574097761535\\
1.15498703578124	-1.49\\
1.15	-1.48423073373327\\
1.14631377910016	-1.48\\
1.14	-1.47267925407925\\
1.13767073319013	-1.47\\
1.13	-1.4610852567122\\
1.12905863125638	-1.46\\
1.12047436159346	-1.45\\
1.12	-1.44944155844156\\
1.11191456934547	-1.44\\
1.11	-1.43774039863879\\
1.10338759915279	-1.43\\
1.1	-1.42599164619165\\
1.09489426740924	-1.42\\
1.09	-1.41419374068554\\
1.08643541781024	-1.41\\
1.08	-1.40234505273732\\
1.07801192250373	-1.4\\
1.07	-1.3904438801422\\
1.06962468329969	-1.39\\
1.06126379525057	-1.38\\
1.06	-1.37847116883117\\
1.05293732946425	-1.37\\
1.05	-1.36643615344193\\
1.04464942553842	-1.36\\
1.04	-1.3543418394471\\
1.03640109284801	-1.35\\
1.03	-1.34218612157074\\
1.02819337630585	-1.34\\
1.02002711713714	-1.33\\
1.02	-1.32996639118457\\
1.01188724940575	-1.32\\
1.01	-1.31765309537089\\
1.00379066926105	-1.31\\
1	-1.30526990400903\\
0.995738550751107	-1.3\\
0.99	-1.29281417953116\\
0.987732110768097	-1.29\\
0.98	-1.28028314997105\\
0.979772611020693	-1.28\\
0.97184422227389	-1.27\\
0.97	-1.26764349376114\\
0.963962547519594	-1.26\\
0.96	-1.25491691751957\\
0.956131034156047	-1.25\\
0.95	-1.24210372178157\\
0.948351107072832	-1.24\\
0.940618334847688	-1.23\\
0.94	-1.22918934169279\\
0.932923767682132	-1.22\\
0.93	-1.2161500317864\\
0.925284426029734	-1.21\\
0.92	-1.20301031592521\\
0.917701924496727	-1.2\\
0.910176207117192	-1.19\\
0.91	-1.18976244193762\\
0.902687509319549	-1.18\\
0.9	-1.17635892255892\\
0.895259776115657	-1.17\\
0.89	-1.16283800410116\\
0.887894843583633	-1.16\\
0.880588652841859	-1.15\\
0.88	-1.14918181818182\\
0.873327016645326	-1.14\\
0.87	-1.13535003579098\\
0.866132871112951	-1.13\\
0.86	-1.12137963636364\\
0.859008311987035	-1.12\\
0.85193528150975	-1.11\\
0.85	-1.10721863260706\\
0.844924726638846	-1.1\\
0.84	-1.09287776928953\\
0.837989110126515	-1.09\\
0.831118881118881	-1.08\\
0.83	-1.07834308300395\\
0.824305907742109	-1.07\\
0.82	-1.06357924376508\\
0.817573757975677	-1.06\\
0.810915089709331	-1.05\\
0.81	-1.04860050041701\\
0.804315011852913	-1.04\\
0.8	-1.03335004248088\\
0.797802240677777	-1.03\\
0.791364472437434	-1.02\\
0.79	-1.01783936451898\\
0.784994082173251	-1.01\\
0.78	-1.00202430243024\\
0.778718153320034	-1\\
0.77251125676831	-0.99\\
0.77	-0.985870665417057\\
0.766389000979996	-0.98\\
0.760364904594454	-0.97\\
0.76	-0.969381231671554\\
0.75440375532101	-0.96\\
0.75	-0.952455544455545\\
0.748549938056752	-0.95\\
0.742774467583033	-0.94\\
0.74	-0.935085684430513\\
0.737094848683818	-0.93\\
0.73151375872978	-0.92\\
0.73	-0.917222343921139\\
0.726016913319239	-0.91\\
0.720634853518574	-0.9\\
0.72	-0.898790563866514\\
0.715329828208106	-0.89\\
0.71015161704469	-0.88\\
0.71	-0.879699393939395\\
0.705047960888669	-0.87\\
0.700078593972399	-0.86\\
0.7	-0.85983738796415\\
0.695186391830086	-0.85\\
0.690431050686047	-0.84\\
0.69	-0.839066485753053\\
0.685760959208272	-0.83\\
0.681225020617902	-0.82\\
0.68	-0.817213564213565\\
0.676788307099261	-0.81\\
0.672477353035657	-0.8\\
0.67	-0.794058551617874\\
0.668285937398321	-0.79\\
0.664205765601614	-0.78\\
0.660268757727598	-0.77\\
0.66	-0.769291595197256\\
0.65642890104819	-0.76\\
0.652730568923462	-0.75\\
0.65	-0.742315398886828\\
0.649166388350811	-0.74\\
0.64571304522076	-0.73\\
0.6424066528622	-0.72\\
0.64	-0.712382663847781\\
0.639237121776213	-0.71\\
0.636184255059906	-0.7\\
0.633280308671224	-0.69\\
0.630525282610167	-0.68\\
0.63	-0.677984415584416\\
0.627890909090909	-0.67\\
0.625400343053173	-0.66\\
0.623060720411664	-0.65\\
0.620872041166381	-0.64\\
0.62	-0.63572053872054\\
0.618818251373722	-0.63\\
0.616905474021006	-0.62\\
0.615145718856507	-0.61\\
0.613538985880225	-0.6\\
0.612085275092161	-0.59\\
0.610784586492314	-0.58\\
0.61	-0.573163636363637\\
0.609631849918894	-0.57\\
0.608623316171803	-0.56\\
0.607769941462726	-0.55\\
0.607071725791663	-0.54\\
0.606528669158615	-0.53\\
0.60614077156358	-0.52\\
0.605908033006559	-0.51\\
0.605830453487552	-0.5\\
0.605908033006559	-0.49\\
0.60614077156358	-0.48\\
0.606528669158615	-0.47\\
0.607071725791663	-0.46\\
0.607769941462726	-0.45\\
0.608623316171803	-0.44\\
0.609631849918894	-0.43\\
0.61	-0.426836363636363\\
0.610784586492314	-0.42\\
0.612085275092161	-0.41\\
0.613538985880225	-0.4\\
0.615145718856507	-0.39\\
0.616905474021006	-0.38\\
0.618818251373722	-0.37\\
0.62	-0.36427946127946\\
0.620872041166381	-0.36\\
0.623060720411664	-0.35\\
0.625400343053173	-0.34\\
0.627890909090909	-0.33\\
0.63	-0.322015584415584\\
0.630525282610167	-0.32\\
0.633280308671224	-0.31\\
0.636184255059906	-0.3\\
0.639237121776213	-0.29\\
0.64	-0.287617336152219\\
0.6424066528622	-0.28\\
0.64571304522076	-0.27\\
0.649166388350811	-0.26\\
0.65	-0.257684601113172\\
0.652730568923462	-0.25\\
0.65642890104819	-0.24\\
0.66	-0.230708404802744\\
0.660268757727598	-0.23\\
0.664205765601614	-0.22\\
0.668285937398321	-0.21\\
0.67	-0.205941448382126\\
0.672477353035657	-0.2\\
0.676788307099261	-0.19\\
0.68	-0.182786435786435\\
0.681225020617903	-0.18\\
0.685760959208272	-0.17\\
0.69	-0.160933514246947\\
0.690431050686047	-0.16\\
0.695186391830086	-0.15\\
0.7	-0.140162612035851\\
0.700078593972399	-0.14\\
0.705047960888669	-0.13\\
0.71	-0.120300606060606\\
0.71015161704469	-0.12\\
0.715329828208106	-0.11\\
0.72	-0.101209436133486\\
0.720634853518574	-0.1\\
0.726016913319239	-0.0900000000000001\\
0.73	-0.0827776560788607\\
0.73151375872978	-0.0800000000000001\\
0.737094848683818	-0.0700000000000001\\
0.74	-0.0649143155694873\\
0.742774467583033	-0.0600000000000001\\
0.748549938056752	-0.05\\
0.75	-0.0475444555444552\\
0.75440375532101	-0.04\\
0.76	-0.0306187683284458\\
0.760364904594454	-0.03\\
0.766389000979996	-0.02\\
0.77	-0.0141293345829425\\
0.77251125676831	-0.01\\
0.778718153320034	0\\
0.78	0.00202430243024299\\
0.784994082173251	0.01\\
0.79	0.0178393645189765\\
0.791364472437434	0.02\\
0.797802240677777	0.03\\
0.8	0.0333500424808836\\
0.804315011852913	0.04\\
0.81	0.0486005004170146\\
0.810915089709331	0.05\\
0.817573757975677	0.0600000000000001\\
0.82	0.0635792437650845\\
0.824305907742109	0.0700000000000001\\
0.83	0.0783430830039529\\
0.831118881118881	0.0800000000000001\\
0.837989110126515	0.0900000000000001\\
0.84	0.092877769289534\\
0.844924726638846	0.1\\
0.85	0.107218632607063\\
0.85193528150975	0.11\\
0.859008311987035	0.12\\
0.86	0.121379636363636\\
0.866132871112951	0.13\\
0.87	0.135350035790981\\
0.873327016645326	0.14\\
0.88	0.149181818181818\\
0.880588652841859	0.15\\
0.887894843583633	0.16\\
0.89	0.162838004101162\\
0.895259776115657	0.17\\
0.9	0.176358922558923\\
0.902687509319549	0.18\\
0.91	0.189762441937625\\
0.910176207117192	0.19\\
0.917701924496727	0.2\\
0.92	0.20301031592521\\
0.925284426029734	0.21\\
0.93	0.216150031786396\\
0.932923767682132	0.22\\
0.94	0.22918934169279\\
0.940618334847688	0.23\\
0.948351107072832	0.24\\
0.95	0.242103721781574\\
0.956131034156047	0.25\\
0.96	0.254916917519567\\
0.963962547519594	0.26\\
0.97	0.267643493761141\\
0.97184422227389	0.27\\
0.979772611020693	0.28\\
0.98	0.280283149971048\\
0.987732110768097	0.29\\
0.99	0.292814179531161\\
0.995738550751107	0.3\\
1	0.305269904009035\\
1.00379066926105	0.31\\
1.01	0.317653095370887\\
1.01188724940575	0.32\\
1.02	0.329966391184573\\
1.02002711713714	0.33\\
1.02819337630585	0.34\\
1.03	0.342186121570737\\
1.03640109284801	0.35\\
1.04	0.354341839447103\\
1.04464942553842	0.36\\
1.05	0.366436153441934\\
1.05293732946425	0.37\\
1.06	0.378471168831169\\
1.06126379525057	0.38\\
1.06962468329969	0.39\\
1.07	0.390443880142204\\
1.07801192250373	0.4\\
1.08	0.402345052737318\\
1.08643541781024	0.41\\
1.09	0.414193740685544\\
1.09489426740924	0.42\\
1.1	0.425991646191646\\
1.10338759915279	0.43\\
1.11	0.437740398638794\\
1.11191456934547	0.44\\
1.12	0.449441558441559\\
1.12047436159346	0.45\\
1.12905863125638	0.46\\
1.13	0.4610852567122\\
1.13767073319013	0.47\\
1.14	0.472679254079254\\
1.14631377910016	0.48\\
1.15	0.484230733733272\\
1.15498703578124	0.49\\
1.16	0.49574097761535\\
1.16368979301065	0.5\\
1.17	0.507211216644053\\
1.17242136265462	0.51\\
1.18	0.51864263322884\\
1.18118107781631	0.52\\
1.18996804738339	0.53\\
1.19	0.530036012296882\\
1.19877305595298	0.54\\
1.2	0.541380165289257\\
1.20760466633409	0.55\\
1.21	0.55268935803533\\
1.21646227672621	0.56\\
1.22	0.563964575330773\\
1.2253453033453	0.57\\
1.23	0.575206765327696\\
1.23425317975462	0.58\\
1.24	0.586416841223293\\
1.24318535622509	0.59\\
1.25	0.597595682855957\\
1.25214129912381	0.6\\
1.26	0.608744138214726\\
1.26112049032921	0.61\\
1.27	0.619863024867509\\
1.27012242667152	0.62\\
1.27914046274367	0.63\\
1.28	0.630944733680416\\
1.2881796476504	0.64\\
1.29	0.641997618102422\\
1.29724040084767	0.65\\
1.3	0.653023612750885\\
1.30632226541603	0.66\\
1.31	0.664023410066329\\
1.31542479725183	0.67\\
1.32	0.674997678916828\\
1.32454756462107	0.68\\
1.33	0.685947065592635\\
1.33369014773164	0.69\\
1.34	0.696872194750856\\
1.34285213832333	0.7\\
1.35	0.70777367031309\\
1.35203313927452	0.71\\
1.36	0.718652076318743\\
1.36123276422486	0.72\\
1.37	0.72950797773655\\
1.37045063721325	0.73\\
1.37968428207307	0.74\\
1.38	0.740339174881344\\
1.38893258616615	0.75\\
1.39	0.751145237232887\\
1.39819824955568	0.76\\
1.4	0.761930650377291\\
1.40748092874513	0.77\\
1.41	0.772695900178253\\
1.41678028924116	0.78\\
1.42	0.78344145737531\\
1.42609600526056	0.79\\
1.43	0.794167778167779\\
1.4354277594486	0.8\\
1.44	0.804875304771857\\
1.44477524260816	0.81\\
1.45	0.815564465952299\\
1.45413815343932	0.82\\
1.46	0.826235677530017\\
1.46351619828885	0.83\\
1.47	0.836889342866871\\
1.47290909090909	0.84\\
1.48	0.847525853328827\\
1.48231655222605	0.85\\
1.49	0.858145588728615\\
1.49173831011603	0.86\\
1.5	0.868748917748918\\
1.5011740991906	0.87\\
1.51	0.879336198347107\\
1.5106236605895	0.88\\
1.52	0.889907778142435\\
1.52008674178114	0.89\\
1.52956042599167	0.9\\
1.53	0.900460692332578\\
1.53904672782499	0.91\\
1.54	0.910998072598779\\
1.54854594248422	0.92\\
1.55	0.921520893141946\\
1.55805783746097	0.93\\
1.56	0.932029458346532\\
1.56758218579564	0.94\\
1.57	0.942524064171123\\
1.57711876591284	0.95\\
1.58	0.953004998437988\\
1.58666736146264	0.96\\
1.59	0.963472541110766\\
1.59622776116752	0.97\\
1.6	0.973926964560863\\
1.60579975867448	0.98\\
1.61	0.984368533823079\\
1.61538315241244	0.99\\
1.62	0.994797506840985\\
1.62497774545455	1\\
1.63	1.00521413470251\\
1.63458334538513	1.01\\
1.64	1.01561866186619\\
1.64419976417129	1.02\\
1.65	1.02601132637854\\
1.65382681803883	1.03\\
1.66	1.03639236008291\\
1.66346432735235	1.04\\
1.67	1.04676198882024\\
1.67311211649936	1.05\\
1.68	1.05712043262204\\
1.68277001377837	1.06\\
1.69	1.06746790589602\\
1.69243785129051	1.07\\
1.7	1.07780461760462\\
1.70211546483495	1.08\\
1.71	1.08813077143677\\
1.71180269380756	1.09\\
1.72	1.09844656597321\\
1.72149938110301	1.1\\
1.73	1.10875219484565\\
1.73120537302	1.11\\
1.74	1.11904784688995\\
1.74092051916955	1.12\\
1.75	1.12933370629371\\
1.75064467238626	1.13\\
1.76	1.13960995273839\\
1.76037768864242	1.14\\
1.77	1.14987676153634\\
1.77011942696479	1.15\\
1.77986905877628	1.16\\
1.78	1.1601334971335\\
1.78962656154276	1.17\\
1.79	1.17038046132972\\
1.79939243793222	1.18\\
1.8	1.18061855948206\\
1.80916655684125	1.19\\
1.81	1.19084794851167\\
1.81894878988908	1.2\\
1.82	1.20106878165822\\
1.82873901134733	1.21\\
1.83	1.21128120858733\\
1.83853709807178	1.22\\
1.84	1.22148537549407\\
1.84834292943637	1.23\\
1.85	1.23168142520304\\
1.85815638726913	1.24\\
1.86	1.24186949726491\\
1.86797735579012	1.25\\
1.87	1.25204972804973\\
1.87780572155118	1.26\\
1.88	1.26222225083698\\
1.88764137337753	1.27\\
1.89	1.27238719590269\\
1.89748420231112	1.28\\
1.9	1.28254469060351\\
1.90733410155565	1.29\\
1.91	1.29269485945809\\
1.91719096642321	1.3\\
1.92	1.30283782422564\\
1.92705469428253	1.31\\
1.93	1.31297370398197\\
1.93692518450868	1.32\\
1.94	1.32310261519303\\
1.94680233843433	1.33\\
1.95	1.33322467178598\\
1.95668605930235	1.34\\
1.96	1.34333998521803\\
1.96657625221982	1.35\\
1.97	1.353448664543\\
1.9764728241133	1.36\\
1.98	1.36355081647575\\
1.98637568368552	1.37\\
1.99	1.37364654545455\\
1.99628474137311	1.38\\
2	1.38373595370147\\
-3.51515454545454	440\\
-2	1.15785031584729\\
-1.99233720275596	1.15\\
-1.99	1.14759104725062\\
-1.98259574697464	1.14\\
-1.98	1.13732249096469\\
-1.97286302669015	1.13\\
-1.97	1.12704447552448\\
-1.96313918326205	1.12\\
-1.96	1.11675682521813\\
-1.95342436111644	1.11\\
-1.95	1.10645935995469\\
-1.94371870782958	1.1\\
-1.94	1.09615189512682\\
-1.93402237421427	1.09\\
-1.93	1.08583424146831\\
-1.92433551440902	1.08\\
-1.92	1.0755062049062\\
-1.91465828597027	1.07\\
-1.91	1.0651675864072\\
-1.90499084996762	1.06\\
-1.9	1.05481818181818\\
-1.89533337108236	1.05\\
-1.89	1.0444577817005\\
-1.88568601770919	1.04\\
-1.88	1.03408617115783\\
-1.87604896206156	1.03\\
-1.87	1.02370312965723\\
-1.86642238028047	1.02\\
-1.86	1.01330843084308\\
-1.85680645254713	1.01\\
-1.85	1.0029018423437\\
-1.84720136319944	1\\
-1.84	0.992483125570082\\
-1.83760730085255	0.99\\
-1.83	0.982052035506581\\
-1.82802445852365	0.98\\
-1.82	0.971608320493066\\
-1.81845303376115	0.97\\
-1.81	0.961151721998138\\
-1.80889322877844	0.96\\
-1.8	0.950681974383006\\
-1.79934525059236	0.95\\
-1.79	0.940198804655552\\
-1.78980931116676	0.94\\
-1.78028392119	0.93\\
-1.78	0.929699840510367\\
-1.77076976779481	0.92\\
-1.77	0.919185351750723\\
-1.76126820777476	0.91\\
-1.76	0.908656421869945\\
-1.75177947095149	0.9\\
-1.75	0.898112740306289\\
-1.74230379277302	0.89\\
-1.74	0.887553987528717\\
-1.73284141448698	0.88\\
-1.73	0.876979834710744\\
-1.72339258332029	0.87\\
-1.72	0.866389943389943\\
-1.71395755266555	0.86\\
-1.71	0.855783965112378\\
-1.70453658227447	0.85\\
-1.7	0.845161541061169\\
-1.69512993845859	0.84\\
-1.69	0.834522301668369\\
-1.68573789429773	0.83\\
-1.68	0.823865866209262\\
-1.67636072985629	0.82\\
-1.67	0.813191842378154\\
-1.6669987324081	0.81\\
-1.66	0.802499825844654\\
-1.65765219666983	0.8\\
-1.65	0.7917893997894\\
-1.64832142504363	0.79\\
-1.64	0.781060134418111\\
-1.63900672786929	0.78\\
-1.63	0.770311586452763\\
-1.62970842368641	0.77\\
-1.62042402001668	0.76\\
-1.62	0.759539659543644\\
-1.61115458239581	0.75\\
-1.61	0.748744797371303\\
-1.60190236316305	0.74\\
-1.6	0.737928965771071\\
-1.5926677104251	0.73\\
-1.59	0.727091651205937\\
-1.58345098173594	0.72\\
-1.58	0.716232323232323\\
-1.57425254441953	0.71\\
-1.57	0.705350433798567\\
-1.56507277590579	0.7\\
-1.56	0.694445416508178\\
-1.55591206408059	0.69\\
-1.55	0.6835166858458\\
-1.54677080765031	0.68\\
-1.54	0.672563636363636\\
-1.53764941652183	0.67\\
-1.53	0.661585641825985\\
-1.52854831219861	0.66\\
-1.52	0.650582054309327\\
-1.51946792819369	0.65\\
-1.51040579918696	0.64\\
-1.51	0.639548257909492\\
-1.50136128120584	0.63\\
-1.5	0.62848202020202\\
-1.49233860078099	0.62\\
-1.49	0.617387688544639\\
-1.48333823475352	0.61\\
-1.48	0.606264500205677\\
-1.47436067394927	0.6\\
-1.47	0.595111664591117\\
-1.4654064236953	0.59\\
-1.46	0.58392836196062\\
-1.45647600435943	0.58\\
-1.45	0.572713742071882\\
-1.44756995191397	0.57\\
-1.44	0.561466922748613\\
-1.43868881852516	0.56\\
-1.43	0.550186988367084\\
-1.42983317316933	0.55\\
-1.420995963867	0.54\\
-1.42	0.538862099253404\\
-1.41218383235852	0.53\\
-1.41	0.52749977827051\\
-1.40339864943987	0.52\\
-1.4	0.516100761307658\\
-1.39464104480548	0.51\\
-1.39	0.504663952962461\\
-1.38591166792213	0.5\\
-1.38	0.493188213796254\\
-1.37721118881119	0.49\\
-1.37	0.481672358098754\\
-1.36854029886817	0.48\\
-1.36	0.470115151515152\\
-1.35989971172195	0.47\\
-1.35127979211499	0.46\\
-1.35	0.458499762018087\\
-1.34269043428431	0.45\\
-1.34	0.446838383838384\\
-1.33413320678414	0.44\\
-1.33	0.435130772970345\\
-1.32560890368213	0.43\\
-1.32	0.42337542997543\\
-1.31711834542937	0.42\\
-1.31	0.411570789865872\\
-1.30866237996531	0.41\\
-1.30023985786201	0.4\\
-1.3	0.399712036566785\\
-1.29184207158398	0.39\\
-1.29	0.387782229070365\\
-1.28348104805748	0.38\\
-1.28	0.375796883116883\\
-1.27515773486657	0.37\\
-1.27	0.363754072517078\\
-1.26687311244365	0.36\\
-1.26	0.351651780967571\\
-1.25862819550532	0.35\\
-1.25042032760828	0.34\\
-1.25	0.339481763745237\\
-1.24224177096789	0.33\\
-1.24	0.327226997245179\\
-1.2341055138633	0.32\\
-1.23	0.314904629113218\\
-1.22601269553389	0.31\\
-1.22	0.302512140033879\\
-1.21796449649998	0.3\\
-1.21	0.290046883933677\\
-1.2099621404497	0.29\\
-1.20198854979454	0.28\\
-1.2	0.277473900293255\\
-1.19406225245795	0.27\\
-1.19	0.264819964349376\\
-1.18618492216526	0.26\\
-1.18	0.252082480433474\\
-1.17835793971042	0.25\\
-1.17057726085925	0.24\\
-1.17	0.239247990105133\\
-1.16283366080414	0.23\\
-1.16	0.226292789968652\\
-1.15514393536214	0.22\\
-1.15	0.213240940877305\\
-1.14750964779444	0.21\\
-1.14	0.200088330109607\\
-1.13993242243378	0.2\\
-1.13239037142998	0.19\\
-1.13	0.186784339747843\\
-1.12490709838107	0.18\\
-1.12	0.17336632996633\\
-1.11748503295266	0.17\\
-1.11012476731901	0.16\\
-1.11	0.159827897293546\\
-1.10280349611261	0.15\\
-1.1	0.146112050739958\\
-1.09554796981674	0.14\\
-1.09	0.132263421617752\\
-1.0883602136597	0.13\\
-1.08122958045947	0.12\\
-1.08	0.118247597930525\\
-1.07415299973801	0.11\\
-1.07	0.104045078888054\\
-1.06714936199502	0.1\\
-1.06021866892572	0.0900000000000001\\
-1.06	0.089679098679099\\
-1.05333458183958	0.0800000000000001\\
-1.05	0.0750735177865615\\
-1.0465291222757	0.0700000000000001\\
-1.04	0.0602872083668547\\
-1.03980484338271	0.0600000000000001\\
-1.03313015907724	0.05\\
-1.03	0.0452243536280236\\
-1.0265382702404	0.04\\
-1.02003371096988	0.03\\
-1.02	0.0299471861471864\\
-1.01358048835	0.02\\
-1.01	0.0143442188879085\\
-1.00721905192384	0.01\\
-1.0009418611221	0\\
-1	-0.00153076216712554\\
-0.99473001542769	-0.01\\
-0.99	-0.0177582005623241\\
-0.988617580766341	-0.02\\
-0.982578165636017	-0.03\\
-0.98	-0.0343607038123165\\
-0.976627301958491	-0.04\\
-0.970775211926337	-0.05\\
-0.97	-0.0513544433094992\\
-0.964993789554622	-0.0600000000000001\\
-0.96	-0.0688223615464992\\
-0.959325513196481	-0.0700000000000001\\
-0.953729726494703	-0.0800000000000001\\
-0.95	-0.0868258488499449\\
-0.948244806492641	-0.0900000000000001\\
-0.94284840409424	-0.1\\
-0.94	-0.105411967779056\\
-0.937555936982774	-0.11\\
-0.932363758965242	-0.12\\
-0.93	-0.124673939393939\\
-0.927273347812597	-0.13\\
-0.922290412659013	-0.14\\
-0.92	-0.144725992317541\\
-0.917412200791606	-0.15\\
-0.912643714267764	-0.16\\
-0.91	-0.165709633649932\\
-0.907988421655321	-0.17\\
-0.903439786245944	-0.18\\
-0.9	-0.187802308802309\\
-0.899018748791959	-0.19\\
-0.894695573738805	-0.2\\
-0.890514142130017	-0.21\\
-0.89	-0.211272727272727\\
-0.88642889773543	-0.22\\
-0.882480584741891	-0.23\\
-0.88	-0.236519725557461\\
-0.878658512396694	-0.24\\
-0.874949421487603	-0.25\\
-0.871385785123967	-0.26\\
-0.87	-0.264054158607349\\
-0.867940644469753	-0.27\\
-0.864624505928854	-0.28\\
-0.861455751323106	-0.29\\
-0.86	-0.294818181818181\\
-0.858413334238577	-0.3\\
-0.855500712879353	-0.31\\
-0.852737456718039	-0.32\\
-0.850123565754634	-0.33\\
-0.85	-0.330501377410467\\
-0.847627141972335	-0.34\\
-0.845280434932214	-0.35\\
-0.843085128346294	-0.36\\
-0.841041222214575	-0.37\\
-0.84	-0.375501818181816\\
-0.839136956673411	-0.38\\
-0.83737179934417	-0.39\\
-0.835760133956603	-0.4\\
-0.834301960510709	-0.41\\
-0.832997279006488	-0.42\\
-0.83184608944394	-0.43\\
-0.830848391823065	-0.44\\
-0.830004186143864	-0.45\\
-0.83	-0.450060606060602\\
-0.829303855677396	-0.46\\
-0.828759108595684	-0.47\\
-0.828370003537318	-0.48\\
-0.828136540502299	-0.49\\
-0.828058719490626	-0.5\\
-0.828136540502299	-0.51\\
-0.828370003537318	-0.52\\
-0.828759108595684	-0.53\\
-0.829303855677396	-0.54\\
-0.83	-0.549939393939397\\
-0.830004186143864	-0.55\\
-0.830848391823065	-0.56\\
-0.83184608944394	-0.57\\
-0.832997279006488	-0.58\\
-0.834301960510709	-0.59\\
-0.835760133956603	-0.6\\
-0.837371799344171	-0.61\\
-0.839136956673411	-0.62\\
-0.84	-0.624498181818183\\
-0.841041222214576	-0.63\\
-0.843085128346294	-0.64\\
-0.845280434932214	-0.65\\
-0.847627141972335	-0.66\\
-0.85	-0.669498622589533\\
-0.850123565754634	-0.67\\
-0.852737456718039	-0.68\\
-0.855500712879353	-0.69\\
-0.858413334238577	-0.7\\
-0.86	-0.705181818181819\\
-0.861455751323106	-0.71\\
-0.864624505928854	-0.72\\
-0.867940644469752	-0.73\\
-0.87	-0.735945841392651\\
-0.871385785123967	-0.74\\
-0.874949421487603	-0.75\\
-0.878658512396694	-0.76\\
-0.88	-0.763480274442539\\
-0.882480584741891	-0.77\\
-0.88642889773543	-0.78\\
-0.89	-0.788727272727273\\
-0.890514142130017	-0.79\\
-0.894695573738805	-0.8\\
-0.899018748791959	-0.81\\
-0.9	-0.812197691197691\\
-0.903439786245944	-0.82\\
-0.907988421655321	-0.83\\
-0.91	-0.834290366350068\\
-0.912643714267764	-0.84\\
-0.917412200791606	-0.85\\
-0.92	-0.855274007682459\\
-0.922290412659013	-0.86\\
-0.927273347812597	-0.87\\
-0.93	-0.875326060606061\\
-0.932363758965242	-0.88\\
-0.937555936982774	-0.89\\
-0.94	-0.894588032220944\\
-0.94284840409424	-0.9\\
-0.948244806492641	-0.91\\
-0.95	-0.913174151150055\\
-0.953729726494703	-0.92\\
-0.959325513196481	-0.93\\
-0.96	-0.931177638453501\\
-0.964993789554622	-0.94\\
-0.97	-0.948645556690501\\
-0.970775211926337	-0.95\\
-0.976627301958491	-0.96\\
-0.98	-0.965639296187684\\
-0.982578165636017	-0.97\\
-0.988617580766341	-0.98\\
-0.99	-0.982241799437676\\
-0.99473001542769	-0.99\\
-1	-0.998469237832874\\
-1.0009418611221	-1\\
-1.00721905192384	-1.01\\
-1.01	-1.01434421888791\\
-1.01358048835	-1.02\\
-1.02	-1.02994718614719\\
-1.02003371096988	-1.03\\
-1.0265382702404	-1.04\\
-1.03	-1.04522435362802\\
-1.03313015907724	-1.05\\
-1.03980484338271	-1.06\\
-1.04	-1.06028720836685\\
-1.0465291222757	-1.07\\
-1.05	-1.07507351778656\\
-1.05333458183958	-1.08\\
-1.06	-1.0896790986791\\
-1.06021866892572	-1.09\\
-1.06714936199502	-1.1\\
-1.07	-1.10404507888805\\
-1.07415299973801	-1.11\\
-1.08	-1.11824759793053\\
-1.08122958045947	-1.12\\
-1.0883602136597	-1.13\\
-1.09	-1.13226342161775\\
-1.09554796981674	-1.14\\
-1.1	-1.14611205073996\\
-1.10280349611261	-1.15\\
-1.11	-1.15982789729355\\
-1.11012476731901	-1.16\\
-1.11748503295266	-1.17\\
-1.12	-1.17336632996633\\
-1.12490709838107	-1.18\\
-1.13	-1.18678433974784\\
-1.13239037142998	-1.19\\
-1.13993242243378	-1.2\\
-1.14	-1.20008833010961\\
-1.14750964779444	-1.21\\
-1.15	-1.2132409408773\\
-1.15514393536214	-1.22\\
-1.16	-1.22629278996865\\
-1.16283366080414	-1.23\\
-1.17	-1.23924799010513\\
-1.17057726085925	-1.24\\
-1.17835793971042	-1.25\\
-1.18	-1.25208248043347\\
-1.18618492216526	-1.26\\
-1.19	-1.26481996434938\\
-1.19406225245795	-1.27\\
-1.2	-1.27747390029326\\
-1.20198854979454	-1.28\\
-1.2099621404497	-1.29\\
-1.21	-1.29004688393368\\
-1.21796449649998	-1.3\\
-1.22	-1.30251214003388\\
-1.22601269553389	-1.31\\
-1.23	-1.31490462911322\\
-1.2341055138633	-1.32\\
-1.24	-1.32722699724518\\
-1.24224177096789	-1.33\\
-1.25	-1.33948176374524\\
-1.25042032760828	-1.34\\
-1.25862819550532	-1.35\\
-1.26	-1.35165178096757\\
-1.26687311244365	-1.36\\
-1.27	-1.36375407251708\\
-1.27515773486657	-1.37\\
-1.28	-1.37579688311688\\
-1.28348104805748	-1.38\\
-1.29	-1.38778222907036\\
-1.29184207158398	-1.39\\
-1.3	-1.39971203656679\\
-1.30023985786201	-1.4\\
-1.30866237996531	-1.41\\
-1.31	-1.41157078986587\\
-1.31711834542937	-1.42\\
-1.32	-1.42337542997543\\
-1.32560890368213	-1.43\\
-1.33	-1.43513077297035\\
-1.33413320678414	-1.44\\
-1.34	-1.44683838383838\\
-1.34269043428431	-1.45\\
-1.35	-1.45849976201809\\
-1.35127979211499	-1.46\\
-1.35989971172195	-1.47\\
-1.36	-1.47011515151515\\
-1.36854029886817	-1.48\\
-1.37	-1.48167235809875\\
-1.37721118881119	-1.49\\
-1.38	-1.49318821379625\\
-1.38591166792213	-1.5\\
8.31312727272727	143\\
0.446440831895377	-1.5\\
0.44	-1.49676016445866\\
0.43	-1.49182046596619\\
0.426245878473858	-1.49\\
0.42	-1.48694047069682\\
0.41	-1.48213336409783\\
0.405476073979841	-1.48\\
0.4	-1.47739114219114\\
0.39	-1.47271934731935\\
0.384061895551257	-1.47\\
0.38	-1.46812058407913\\
0.37	-1.46358690532266\\
0.361922138538241	-1.46\\
0.36	-1.45913755354593\\
0.35	-1.45474488338886\\
0.34	-1.45044645406949\\
0.338938073134835	-1.45\\
0.33	-1.4462025012025\\
0.32	-1.44204906204906\\
0.314950811899964	-1.44\\
0.31	-1.43796937287312\\
0.3	-1.43396402527953\\
0.29	-1.43005493437044\\
0.289855922478643	-1.43\\
0.28	-1.42620147420147\\
0.27	-1.42244471744472\\
0.263319457499665	-1.42\\
0.26	-1.41877198211624\\
0.25	-1.41517088922007\\
0.24	-1.41166815697963\\
0.235099956223551	-1.41\\
0.23	-1.40824460070316\\
0.22	-1.4049020592667\\
0.21	-1.40165896534405\\
0.204722799169196	-1.4\\
0.2	-1.39849873031996\\
0.19	-1.39542051802946\\
0.18	-1.39244286439817\\
0.171509267431598	-1.39\\
0.17	-1.38956086286595\\
0.16	-1.38675295326143\\
0.15	-1.38404673857216\\
0.14	-1.38144221879815\\
0.134237635953212	-1.38\\
0.13	-1.37892727272727\\
0.12	-1.3764987012987\\
0.11	-1.37417298701299\\
0.1	-1.37195012987013\\
0.0908012741975009	-1.37\\
0.0899999999999999	-1.3698281660536\\
0.0800000000000001	-1.36778770362585\\
0.0699999999999998	-1.36585128744088\\
0.0600000000000001	-1.36401891749869\\
0.0499999999999998	-1.36229059379926\\
0.04	-1.36066631634262\\
0.0356170065675763	-1.36\\
0.0299999999999998	-1.35913609782031\\
0.02	-1.3577033492823\\
0.00999999999999979	-1.35637586390218\\
0	-1.35515364167996\\
-0.01	-1.35403668261563\\
-0.02	-1.3530249867092\\
-0.03	-1.35211855396066\\
-0.04	-1.35131738437002\\
-0.05	-1.35062147793727\\
-0.0600000000000001	-1.35003083466241\\
-0.0606352683461085	-1.35\\
-0.0700000000000001	-1.34954007530931\\
-0.0800000000000001	-1.34915545992469\\
-0.0900000000000001	-1.34887735341581\\
-0.1	-1.34870575578268\\
-0.11	-1.34864066702528\\
-0.12	-1.34868208714363\\
-0.13	-1.34883001613771\\
-0.14	-1.34908445400753\\
-0.15	-1.34944540075309\\
-0.16	-1.34991285637439\\
-0.161518275538897	-1.35\\
-0.17	-1.35048112706007\\
-0.18	-1.35115364167996\\
-0.19	-1.35193141945774\\
-0.2	-1.35281446039341\\
-0.21	-1.35380276448698\\
-0.22	-1.35489633173844\\
-0.23	-1.35609516214779\\
-0.24	-1.35739925571505\\
-0.25	-1.35880861244019\\
-0.257865917865917	-1.36\\
-0.26	-1.36031949553337\\
-0.27	-1.36192065160273\\
-0.28	-1.36362585391487\\
-0.29	-1.36543510246978\\
-0.3	-1.36734839726747\\
-0.31	-1.36936573830793\\
-0.312989843943523	-1.37\\
-0.32	-1.37147012987013\\
-0.33	-1.37367012987013\\
-0.34	-1.37597298701299\\
-0.35	-1.3783787012987\\
-0.356463035825222	-1.38\\
-0.36	-1.38087724704674\\
-0.37	-1.38345916795069\\
-0.38	-1.3861427837699\\
-0.39	-1.38892809450437\\
-0.393712862479986	-1.39\\
-0.4	-1.39179481970543\\
-0.41	-1.394750126968\\
-0.42	-1.39780599288979\\
-0.426950925181013	-1.4\\
-0.43	-1.40095178302361\\
-0.44	-1.40417277749874\\
-0.45	-1.40749321948769\\
-0.457330004405933	-1.41\\
-0.46	-1.41090312965723\\
-0.47	-1.41438400397417\\
-0.48	-1.41796323894685\\
-0.485538295285696	-1.42\\
-0.49	-1.4216230958231\\
-0.5	-1.42535823095823\\
-0.51	-1.42919066339066\\
-0.51205952232087	-1.43\\
-0.52	-1.43308701993194\\
-0.53	-1.43707097715119\\
-0.537178601215298	-1.44\\
-0.54	-1.44113900913901\\
-0.55	-1.44527128427128\\
-0.56	-1.4494987974988\\
-0.561159452542562	-1.45\\
-0.57	-1.45378153260352\\
-0.58	-1.45815326035221\\
-0.58413513801556	-1.46\\
-0.59	-1.46259208666981\\
-0.6	-1.46710504003768\\
-0.606284896206156	-1.47\\
-0.61	-1.47169370629371\\
-0.62	-1.47634498834499\\
-0.62770515970516	-1.48\\
-0.63	-1.48107752653438\\
-0.64	-1.48586432856484\\
-0.648477911266673	-1.49\\
-0.65	-1.49073503883052\\
-0.66	-1.49565463682047\\
-0.668673292605088	-1.5\\
8.31312727272727	648\\
-2	1.48274879780169\\
-1.99678185571432	1.48\\
-1.99	1.47417790563867\\
-1.98510740371399	1.47\\
-1.98	1.46561647004395\\
-1.97342094567132	1.46\\
-1.97	1.45706463613113\\
-1.96172228730215	1.45\\
-1.96	1.44852255199813\\
-1.95001123010764	1.44\\
-1.95	1.43999036880432\\
-1.94	1.43146065304205\\
-1.938278182372	1.43\\
-1.93	1.42294096812279\\
-1.92653208965613	1.42\\
-1.92	1.41443152148113\\
-1.91477279726741	1.41\\
-1.91	1.4059324743498\\
-1.90300008446184	1.4\\
-1.9	1.39744399136484\\
-1.89121372554571	1.39\\
-1.89	1.3889662406559\\
-1.88	1.38049674463468\\
-1.87941016463851	1.38\\
-1.87	1.37203272727273\\
-1.86758545224177	1.37\\
-1.86	1.36357981964416\\
-1.85574618436676	1.36\\
-1.85	1.35513820142122\\
-1.84389210917247	1.35\\
-1.84	1.34670805617147\\
-1.83202296897431	1.34\\
-1.83	1.33828957146396\\
-1.82013850007367	1.33\\
-1.82	1.32988293897883\\
-1.81	1.3214801992528\\
-1.80822803303617	1.32\\
-1.8	1.31308890558477\\
-1.7963006508503	1.31\\
-1.79	1.30470989675145\\
-1.78435687777208	1.3\\
-1.78	1.2963433780704\\
-1.77239641827288	1.29\\
-1.77	1.28798955946015\\
-1.76041896967661	1.28\\
-1.76	1.27964865556978\\
-1.75	1.27131344430218\\
-1.74841457760193	1.27\\
-1.74	1.26298944115375\\
-1.73638976145305	1.26\\
-1.73	1.25467883967884\\
-1.72434673759975	1.25\\
-1.72	1.24638187027872\\
-1.71228516547533	1.24\\
-1.71	1.23809876866649\\
-1.70020469596629	1.23\\
-1.7	1.22982977602108\\
-1.69	1.22156600790514\\
-1.68809293071912	1.22\\
-1.68	1.21331566392791\\
-1.67595995478766	1.21\\
-1.67	1.20507997867235\\
-1.66380667598401	1.2\\
-1.66	1.19685921158488\\
-1.65163269979391	1.19\\
-1.65	1.18865362827084\\
-1.64	1.18046074993256\\
-1.63943393099791	1.18\\
-1.63	1.17227544097693\\
-1.62720266889074	1.17\\
-1.62	1.16410592410592\\
-1.61494912180542	1.16\\
-1.61	1.15595248558088\\
-1.60267284221914	1.15\\
-1.6	1.14781541862393\\
-1.59037337054559	1.14\\
-1.59	1.1396950236308\\
-1.58	1.13158187378371\\
-1.57803691564602	1.13\\
-1.57	1.12348391608392\\
-1.56567339424046	1.12\\
-1.56	1.11540332113707\\
-1.55328482982965	1.11\\
-1.55	1.1073404134806\\
-1.54087069846078	1.1\\
-1.54	1.09929552579082\\
-1.53	1.0912610430322\\
-1.52841936060011	1.09\\
-1.52	1.08324060797247\\
-1.51593481310933	1.08\\
-1.51	1.07523896103896\\
-1.50342258777492	1.07\\
-1.5	1.06725646238745\\
-1.49088208459545	1.06\\
-1.49	1.05929348143818\\
-1.48	1.05134171294943\\
-1.47830031475653	1.05\\
-1.47	1.04340541335687\\
-1.46568209795949	1.04\\
-1.46	1.03548948771099\\
-1.45303318426096	1.03\\
-1.45	1.02759433681073\\
-1.44035288326834	1.02\\
-1.44	1.0197203720372\\
-1.43	1.01185568556856\\
-1.42762252546608	1.01\\
-1.42	1.00401087284808\\
-1.41485610256808	1\\
-1.41	0.996188203101247\\
-1.40205550567938	0.99\\
-1.4	0.988388123660851\\
-1.39	0.980606978879706\\
-1.38921381278991	0.98\\
-1.38	0.972838212634823\\
-1.37631968031968	0.97\\
-1.37	0.965093080980453\\
-1.36338824666694	0.96\\
-1.36	0.957372071227741\\
-1.35041861220512	0.95\\
-1.35	0.949675684177414\\
-1.34	0.941990563070148\\
-1.33738868485124	0.94\\
-1.33	0.934328476401647\\
-1.32431454129395	0.93\\
-1.32	0.926692185007974\\
-1.31119855686538	0.92\\
-1.31	0.919082235785416\\
-1.3	0.911488596209444\\
-1.29802312188047	0.91\\
-1.29	0.90391556130702\\
-1.28479284085531	0.9\\
-1.28	0.896370153144347\\
-1.27151659796051	0.89\\
-1.27	0.888852970134559\\
-1.26	0.88135477518871\\
-1.25817740297585	0.88\\
-1.25	0.873877355371901\\
-1.24477573382032	0.87\\
-1.24	0.86642957042957\\
-1.23132341706736	0.86\\
-1.23	0.859012076484401\\
-1.22	0.851613552499161\\
-1.21779933202178	0.85\\
-1.21	0.844238932071646\\
-1.20420887390923	0.84\\
-1.2	0.836896152536602\\
-1.19056241554448	0.83\\
-1.19	0.829585934819897\\
-1.18	0.822291595197256\\
-1.17682886304296	0.82\\
-1.17	0.815027652955409\\
-1.16302966406287	0.81\\
-1.16	0.80779797979798\\
-1.15	0.800598746081505\\
-1.1491602755117	0.8\\
-1.14	0.793418041418042\\
-1.13519656686233	0.79\\
-1.13	0.78627343473647\\
-1.12116562889166	0.78\\
-1.12	0.779165775401069\\
-1.11	0.772079500891265\\
-1.10703592662229	0.77\\
-1.1	0.76502479338843\\
-1.09282172373081	0.76\\
-1.09	0.75800905469033\\
-1.08	0.75102499094531\\
-1.07851716007336	0.75\\
-1.07	0.744065352318364\\
-1.06410441044104	0.74\\
-1.06	0.737146853146853\\
-1.05	0.730268310636732\\
-1.04960575415067	0.73\\
-1.04	0.723409276437848\\
-1.03497731372656	0.72\\
-1.03	0.716593714927048\\
-1.02025974025974	0.71\\
-1.02	0.70982270841192\\
-1.01	0.703071671067522\\
-1.00539917509464	0.7\\
-1	0.696365157854698\\
-0.990438260670819	0.69\\
-0.99	0.689705792098197\\
-0.98	0.683068661296509\\
-0.975323004969308	0.68\\
-0.97	0.676477756286267\\
-0.960095818300112	0.67\\
-0.96	0.669936792820913\\
-0.95	0.663417479516192\\
-0.94469505178366	0.66\\
-0.94	0.656949232585596\\
-0.93	0.650529319165683\\
-0.929165373875271	0.65\\
-0.92	0.644136562127828\\
-0.913453540239995	0.64\\
-0.91	0.637798558269924\\
-0.9	0.631503404084902\\
-0.897581341408414	0.63\\
-0.89	0.625245656565656\\
-0.881527116099981	0.62\\
-0.88	0.619046066041582\\
-0.87	0.61288014675907\\
-0.865266965900717	0.61\\
-0.86	0.60676593994241\\
-0.85	0.600707116412999\\
-0.848817011905581	0.6\\
-0.84	0.594681610626816\\
-0.832131444917323	0.59\\
-0.83	0.588720150816925\\
-0.82	0.58279849183075\\
-0.81520700294181	0.58\\
-0.81	0.576931501057082\\
-0.8	0.571122198731501\\
-0.798040026585924	0.57\\
-0.79	0.565353393085787\\
-0.780599565315147	0.56\\
-0.78	0.559655320982335\\
-0.77	0.55399181387333\\
-0.762843902062254	0.55\\
-0.76	0.548398434101783\\
-0.75	0.542852979556329\\
-0.744774121583938	0.54\\
-0.74	0.537368467281511\\
-0.73	0.531943346508564\\
-0.726359522830111	0.53\\
-0.72	0.526572062084257\\
-0.71	0.521269623059867\\
-0.707565269155541	0.52\\
-0.7	0.516016121809225\\
-0.69	0.510838781907747\\
-0.688351667693391	0.51\\
-0.68	0.505707824513795\\
-0.67	0.500658073270013\\
-0.668673292605088	0.5\\
-0.66	0.495654636820466\\
-0.65	0.490735038830516\\
-0.648477911266673	0.49\\
-0.64	0.485864328564836\\
-0.63	0.481077526534379\\
-0.62770515970516	0.48\\
-0.62	0.476344988344988\\
-0.61	0.471693706293706\\
-0.606284896206156	0.47\\
-0.6	0.467105040037683\\
-0.59	0.462592086669807\\
-0.58413513801556	0.46\\
-0.58	0.458153260352213\\
-0.57	0.453781532603522\\
-0.561159452542562	0.45\\
-0.56	0.449498797498797\\
-0.55	0.445271284271284\\
-0.54	0.441139009139009\\
-0.537178601215299	0.44\\
-0.53	0.437070977151191\\
-0.52	0.43308701993194\\
-0.51205952232087	0.43\\
-0.51	0.429190663390663\\
-0.5	0.425358230958231\\
-0.49	0.421623095823096\\
-0.485538295285695	0.42\\
-0.48	0.417963238946845\\
-0.47	0.414384003974168\\
-0.46	0.410903129657228\\
-0.457330004405933	0.41\\
-0.45	0.407493219487695\\
-0.44	0.404172777498744\\
-0.43	0.400951783023606\\
-0.426950925181014	0.4\\
-0.42	0.397805992889792\\
-0.41	0.394750126968004\\
-0.4	0.391794819705434\\
-0.393712862479986	0.39\\
-0.39	0.388928094504366\\
-0.38	0.386142783769902\\
-0.37	0.383459167950693\\
-0.36	0.380877247046739\\
-0.356463035825222	0.38\\
-0.35	0.378378701298701\\
-0.34	0.375972987012987\\
-0.33	0.37367012987013\\
-0.32	0.37147012987013\\
-0.312989843943523	0.37\\
-0.31	0.369365738307935\\
-0.3	0.367348397267472\\
-0.29	0.365435102469784\\
-0.28	0.363625853914871\\
-0.27	0.361920651602732\\
-0.26	0.360319495533368\\
-0.257865917865919	0.36\\
-0.25	0.358808612440191\\
-0.24	0.357399255715045\\
-0.23	0.356095162147794\\
-0.22	0.354896331738437\\
-0.21	0.353802764486975\\
-0.2	0.352814460393408\\
-0.19	0.351931419457735\\
-0.18	0.351153641679957\\
-0.17	0.350481127060074\\
-0.161518275538897	0.35\\
-0.16	0.349912856374395\\
-0.15	0.349445400753093\\
-0.14	0.349084454007531\\
-0.13	0.348830016137708\\
-0.12	0.348682087143626\\
-0.11	0.348640667025282\\
-0.1	0.348705755782679\\
-0.0900000000000001	0.348877353415815\\
-0.0800000000000001	0.349155459924691\\
-0.0700000000000001	0.349540075309306\\
-0.0606352683461085	0.35\\
-0.0600000000000001	0.350030834662414\\
-0.05	0.350621477937267\\
-0.04	0.351317384370016\\
-0.03	0.352118553960659\\
-0.02	0.353024986709197\\
-0.01	0.35403668261563\\
0	0.355153641679957\\
0.00999999999999979	0.35637586390218\\
0.02	0.357703349282297\\
0.0299999999999998	0.359136097820308\\
0.035617006567578	0.36\\
0.04	0.360666316342617\\
0.0499999999999998	0.362290593799264\\
0.0600000000000001	0.364018917498686\\
0.0699999999999998	0.365851287440883\\
0.0800000000000001	0.367787703625854\\
0.0899999999999999	0.369828166053599\\
0.0908012741975014	0.37\\
0.1	0.37195012987013\\
0.11	0.374172987012987\\
0.12	0.376498701298701\\
0.13	0.378927272727273\\
0.134237635953211	0.38\\
0.14	0.381442218798151\\
0.15	0.384046738572162\\
0.16	0.386752953261428\\
0.17	0.389560862865948\\
0.171509267431597	0.39\\
0.18	0.392442864398172\\
0.19	0.395420518029456\\
0.2	0.398498730319959\\
0.204722799169196	0.4\\
0.21	0.401658965344048\\
0.22	0.4049020592667\\
0.23	0.408244600703164\\
0.235099956223552	0.41\\
0.24	0.411668156979632\\
0.25	0.41517088922007\\
0.26	0.418771982116244\\
0.263319457499664	0.42\\
0.27	0.422444717444717\\
0.28	0.426201474201474\\
0.289855922478643	0.43\\
0.29	0.430054934370442\\
0.3	0.433964025279533\\
0.31	0.437969372873116\\
0.314950811899964	0.44\\
0.32	0.442049062049062\\
0.33	0.446202501202501\\
0.338938073134835	0.45\\
0.34	0.450446454069491\\
0.35	0.454744883388862\\
0.36	0.45913755354593\\
0.361922138538241	0.46\\
0.37	0.463586905322657\\
0.38	0.468120584079133\\
0.384061895551257	0.47\\
0.39	0.472719347319347\\
0.4	0.477391142191142\\
0.405476073979842	0.48\\
0.41	0.482133364097831\\
0.42	0.486940470696816\\
0.426245878473858	0.49\\
0.43	0.491820465966195\\
0.44	0.496760164458657\\
0.446440831895377	0.5\\
0.45	0.50177250113071\\
0.46	0.506842152872004\\
0.466120803015692	0.51\\
0.47	0.511981639050605\\
0.48	0.51717868338558\\
0.485337626027281	0.52\\
0.49	0.522440354767184\\
0.5	0.527762305986696\\
0.50413640462333	0.53\\
0.51	0.533141414141414\\
0.52	0.538585858585859\\
0.522556570067487	0.54\\
0.53	0.544077859939104\\
0.54	0.549642453240539\\
0.540632745747056	0.55\\
0.55	0.555242998707454\\
0.558371370081128	0.56\\
0.56	0.560916773367478\\
0.57	0.566630388390952\\
0.575811556864188	0.57\\
0.58	0.572405919661734\\
0.59	0.578233826638478\\
0.592987626063944	0.58\\
0.6	0.584107247591119\\
0.609920304675929	0.59\\
0.61	0.590046907430469\\
0.62	0.596014943960149\\
0.626586620926244	0.6\\
0.63	0.602046482928836\\
0.64	0.608123406005759\\
0.643047224634293	0.61\\
0.65	0.614243375458622\\
0.659309117278661	0.62\\
0.66	0.620423434343434\\
0.67	0.626632323232323\\
0.675354963058143	0.63\\
0.68	0.632895474569483\\
0.69	0.63920824989988\\
0.691238644195226	0.64\\
0.7	0.645551409289401\\
0.706934835076428	0.65\\
0.71	0.651949232585596\\
0.72	0.658386462022826\\
0.722476593174268	0.66\\
0.73	0.664859539602029\\
0.737864263117054	0.67\\
0.74	0.67138413926499\\
0.75	0.677941586073501\\
0.75310280482827	0.68\\
0.76	0.684537015726889\\
0.768210065140947	0.69\\
0.77	0.691181057436288\\
0.78	0.697854697603651\\
0.783178718367807	0.7\\
0.79	0.704565447001132\\
0.798030210133215	0.71\\
0.8	0.711322109988776\\
0.81	0.718108118219229\\
0.812757812073949	0.72\\
0.82	0.724927643784787\\
0.827375236039924	0.73\\
0.83	0.731790577843209\\
0.84	0.738685314685315\\
0.841886852252919	0.74\\
0.85	0.745607520993063\\
0.856289403523446	0.75\\
0.86	0.752570807678377\\
0.87	0.759570807678377\\
0.870606914212548	0.76\\
0.88	0.766590010779734\\
0.88481164123105	0.77\\
0.89	0.773648128342246\\
0.898943828121078	0.78\\
0.9	0.780744251857092\\
0.91	0.787860983374602\\
0.912976325244869	0.79\\
0.92	0.795008775008775\\
0.926931513526687	0.8\\
0.93	0.802192615813306\\
0.94	0.809407175200278\\
0.940813925685046	0.81\\
0.95	0.81663982025579\\
0.954605144724999	0.82\\
0.96	0.823906689536878\\
0.968336227530859	0.83\\
0.97	0.831207013959823\\
0.98	0.8385291113381\\
0.981990508224669	0.84\\
0.99	0.845874619803988\\
0.995573718095064	0.85\\
1	0.853251928882925\\
1.00910267432916	0.86\\
1.01	0.860660339660339\\
1.02	0.868085248085248\\
1.02255612358302	0.87\\
1.03	0.875535537190082\\
1.03595117437095	0.88\\
1.04	0.883015425008205\\
1.0492971036652	0.89\\
1.05	0.890524275008146\\
1.06	0.898047572499185\\
1.06257310946021	0.9\\
1.07	0.905595276609511\\
1.07579731743666	0.91\\
1.08	0.913170575008031\\
1.08897690326394	0.92\\
1.09	0.92077288676236\\
1.1	0.92839043062201\\
1.10209560197683	0.93\\
1.11	0.936028824833703\\
1.11516414713515	0.94\\
1.12	0.943692985215477\\
1.12819203268641	0.95\\
1.13	0.951382380506092\\
1.14	0.959090284286161\\
1.14117084154236	0.96\\
1.15	0.966813217499224\\
1.15409709202601	0.97\\
1.16	0.974560246533128\\
1.16698618751732	0.98\\
1.17	0.982330884603612\\
1.17983899469861	0.99\\
1.18	0.990124657950745\\
1.19	0.997927333536029\\
1.19263601562198	1\\
1.2	1.00575173663546\\
1.20539775125676	1.01\\
1.21	1.01359825982598\\
1.21812621396199	1.02\\
1.22	1.02146646795827\\
1.23	1.02935171385991\\
1.23081604322215	1.03\\
1.24	1.03724844536571\\
1.24346057874939	1.04\\
1.25	1.04516593115622\\
1.25607453140597	1.05\\
1.26	1.05310377082724\\
1.26865856791573	1.06\\
1.27	1.06106157420854\\
1.28	1.06903282021493\\
1.28120464493724	1.07\\
1.29	1.07701673881674\\
1.293712869509	1.08\\
1.3	1.08501978778319\\
1.3061935161739	1.09\\
1.31	1.09304160729553\\
1.31864716506711	1.1\\
1.32	1.10108184650241\\
1.33	1.10913480600396\\
1.3310669507212	1.11\\
1.34	1.11719898677174\\
1.3434518400333	1.12\\
1.35	1.12528083916084\\
1.35581177443246	1.13\\
1.36	1.13338003892132\\
1.36814726109419	1.14\\
1.37	1.14149626968776\\
1.38	1.1496269687759\\
1.38045572696891	1.15\\
1.39	1.15776517440264\\
1.39272861406391	1.16\\
1.4	1.16591973791974\\
1.4049788467304	1.17\\
1.41	1.17409036635007\\
1.4172068703048	1.18\\
1.42	1.18227677367143\\
1.42941311852704	1.19\\
1.43	1.19047868061142\\
1.44	1.19868812013945\\
1.44158774463666	1.2\\
1.45	1.20690989069581\\
1.45373794704763	1.21\\
1.46	1.215146567718\\
1.46586791425001	1.22\\
1.47	1.22339789196311\\
1.47797802897628	1.23\\
1.48	1.23166361016505\\
1.49	1.23994314906995\\
1.4900682368479	1.24\\
1.5	1.24822714248502\\
1.50212694146692	1.25\\
1.51	1.25652499352499\\
1.51416715843091	1.26\\
1.52	1.2648364666495\\
1.52618922673252	1.27\\
1.53	1.27316133162612\\
1.53819347712699	1.28\\
1.54	1.28149936338172\\
1.55	1.28984950343774\\
1.55017915062597	1.29\\
1.56	1.29820410230438\\
1.56213698134812	1.3\\
1.57	1.30657139259632\\
1.57407817882282	1.31\\
1.58	1.31495116453794\\
1.58600303725099	1.32\\
1.59	1.32334321295143\\
1.59791184393594	1.33\\
1.6	1.33174733713153\\
1.60980487948439	1.34\\
1.61	1.34016334072432\\
1.62	1.34858339492486\\
1.62167272727273	1.35\\
1.63	1.35701421220289\\
1.63352442657565	1.36\\
1.64	1.36545649524738\\
1.64536136435535	1.37\\
1.65	1.37391006060606\\
1.65718379135806	1.38\\
1.66	1.38237472871956\\
1.66899195268291	1.39\\
1.67	1.39085032381866\\
1.68	1.39933317342288\\
1.68078171132919	1.4\\
1.69	1.40782223812932\\
1.69255222169402	1.41\\
1.7	1.41632186090672\\
1.70430935231792	1.42\\
1.71	1.4248318772137\\
1.71605332300799	1.43\\
1.72	1.43335212591027\\
1.72778434878284	1.44\\
1.73	1.44188244917037\\
1.73950264000219	1.45\\
1.74	1.45042269239712\\
1.75	1.45896744943037\\
1.75120192698043	1.46\\
1.76	1.46751977793199\\
1.76288637036638	1.47\\
1.77	1.47608170310702\\
1.77455884321864	1.48\\
1.78	1.48465307991756\\
1.78621953493328	1.49\\
1.79	1.49323376623377\\
1.79786863093188	1.5\\
1.8	1.50182362276128\\
1.80950631276523	1.51\\
1.81	1.5104225129709\\
1.82	1.51902549063839\\
1.82112690752576	1.52\\
1.83	1.52763524130191\\
1.83273402019049	1.53\\
1.84	1.53625374134465\\
1.84433038134827	1.54\\
1.85	1.54488086241387\\
1.8559161550515	1.55\\
1.86	1.55351647865516\\
1.86749150203184	1.56\\
1.87	1.56216046665199\\
1.87905657978385	1.57\\
1.88	1.57081270536692\\
1.89	1.57947053669222\\
1.89060849424738	1.58\\
1.9	1.58813254850665\\
1.90214584533681	1.59\\
1.91	1.59680256020829\\
1.91367350499788	1.6\\
1.92	1.6054804577845\\
1.92519161602381	1.61\\
1.93	1.6141661293789\\
1.93670031841631	1.62\\
1.94	1.62285946524064\\
1.94819974945346	1.63\\
1.95	1.63156035767511\\
1.95969004375565	1.64\\
1.96	1.64026870099597\\
1.97	1.64897965670693\\
1.97116571843602	1.65\\
1.98	1.65769668846235\\
1.98263113509867	1.66\\
1.99	1.66642095318077\\
1.99408791156087	1.67\\
2	1.67515235109718\\
20.1414090909091	596\\
-2	1.76144390929159\\
-1.99807109728962	1.76\\
-1.99	1.75393146543036\\
-1.98474328221438	1.75\\
-1.98	1.74643672808261\\
-1.97138572009213	1.74\\
-1.97	1.73895993491967\\
-1.96	1.73149461053488\\
-1.95798707168096	1.73\\
-1.95	1.72404290091931\\
-1.94454986092369	1.72\\
-1.94	1.71660968602504\\
-1.93108105114502	1.71\\
-1.93	1.70919521748093\\
-1.92	1.70179158936302\\
-1.91756670493043	1.7\\
-1.91	1.69440339614827\\
-1.9040133449703	1.69\\
-1.9	1.68703453297275\\
-1.89042637524419	1.68\\
-1.89	1.67968526645768\\
-1.88	1.67234503657262\\
-1.87678711524696	1.67\\
-1.87	1.66502330463993\\
-1.86311026002822	1.66\\
-1.86	1.65772178865218\\
-1.85	1.65043872600717\\
-1.84939413358189	1.65\\
-1.84	1.6431657130748\\
-1.83562329710251	1.64\\
-1.83	1.63591356184799\\
-1.8218149403271	1.63\\
-1.82	1.62868256684492\\
-1.81	1.62146609625668\\
-1.80795640895674	1.62\\
-1.8	1.61426499032882\\
-1.79404786899013	1.61\\
-1.79	1.60708572662492\\
-1.78009926681351	1.6\\
-1.78	1.59992861792146\\
-1.77	1.59278064656108\\
-1.76608635905579	1.59\\
-1.76	1.58565489426641\\
-1.75203078435589	1.58\\
-1.75	1.57855202628697\\
-1.74	1.57146527929901\\
-1.73791963424875	1.57\\
-1.73	1.56439489324235\\
-1.72375183852292	1.56\\
-1.72	1.55734815306348\\
-1.71	1.55032382216324\\
-1.70953610697424	1.55\\
-1.7	1.54331073571905\\
-1.69525047032939	1.54\\
-1.69	1.53632209068573\\
-1.68091743413664	1.53\\
-1.68	1.52935824915825\\
-1.67	1.5224076318743\\
-1.6665137972503	1.52\\
-1.66	1.51547913376946\\
-1.652054368805	1.51\\
-1.65	1.50857628655634\\
-1.64	1.50169099977329\\
-1.63752792231465	1.5\\
-1.63	1.49482478924584\\
-1.62293544620517	1.49\\
-1.62	1.48798511564003\\
-1.61	1.48116647584154\\
-1.60827783224585	1.48\\
-1.6	1.47436478711162\\
-1.59354514822504	1.47\\
-1.59	1.46759056210965\\
-1.58	1.46083992597733\\
-1.57874728307745	1.46\\
-1.57	1.45410509183911\\
-1.56386667593011	1.45\\
-1.56	1.44739869128301\\
-1.55	1.44071745734985\\
-1.54891867140995	1.44\\
-1.54	1.43405191449377\\
-1.53388181463484	1.43\\
-1.53	1.42741582054309\\
-1.52	1.4208054309327\\
-1.5187728891607	1.42\\
-1.51	1.4142117256112\\
-1.50357078155006	1.41\\
-1.5	1.40764853256979\\
-1.49	1.40111047482701\\
-1.4882891592839	1.4\\
-1.48	1.39459126888942\\
-1.47291205332346	1.39\\
-1.47	1.38810368941403\\
-1.46	1.3816394984326\\
-1.45744484948702	1.38\\
-1.45	1.37519757575758\\
-1.44188217030783	1.37\\
-1.44	1.36878844747746\\
-1.43	1.36239970753107\\
-1.42621526042668	1.36\\
-1.42	1.35603798088704\\
-1.41045551380873	1.35\\
-1.41	1.34971027346637\\
-1.4	1.34339862034984\\
-1.39457338420991	1.34\\
-1.39	1.33712013871687\\
-1.38	1.3308721823136\\
-1.37859300699301	1.33\\
-1.37	1.32464408468244\\
-1.36248962822733	1.32\\
-1.36	1.31845204107188\\
-1.35	1.31228399699474\\
-1.34626703777987	1.31\\
-1.34	1.30614429614707\\
-1.33	1.30004180307227\\
-1.3299309340545	1.3\\
-1.32	1.29395568498354\\
-1.3134467424592	1.29\\
-1.31	1.28790781767252\\
-1.3	1.28188820982939\\
-1.29683673904697	1.28\\
-1.29	1.27589603072983\\
-1.28009422191628	1.27\\
-1.28	1.26994360030904\\
-1.27	1.2640087561164\\
-1.26318685166543	1.26\\
-1.26	1.25811421911422\\
-1.25	1.25224812224812\\
-1.24613380250323	1.25\\
-1.24	1.24641286793436\\
-1.23	1.24061630632977\\
-1.22892722738608	1.24\\
-1.22	1.23484176054493\\
-1.21154458525873	1.23\\
-1.21	1.22911040843215\\
-1.2	1.22340316205534\\
-1.19398210707796	1.22\\
-1.19	1.21773495891863\\
-1.18	1.21209939040551\\
-1.17623973415618	1.21\\
-1.17	1.20649586776859\\
-1.16	1.20093281791522\\
-1.15830712661474	1.2\\
-1.15	1.1953955484044\\
-1.14017146206829	1.19\\
-1.14	1.1899053142703\\
-1.13	1.18443647154033\\
-1.12180772104608	1.18\\
-1.12	1.17901519674355\\
-1.11	1.17362116689281\\
-1.10321916764063	1.17\\
-1.1	1.16827054327054\\
-1.09	1.16295222495222\\
-1.08439195145984	1.16\\
-1.08	1.15767399066191\\
-1.07	1.15243229881901\\
-1.06531106051782	1.15\\
-1.06	1.14722823984526\\
-1.05	1.14206410610666\\
-1.04596019685252	1.14\\
-1.04	1.13693605782597\\
-1.03	1.13185043091465\\
-1.026321635811	1.13\\
-1.02	1.12680027972028\\
-1.01	1.12179412587413\\
-1.00637606644443	1.12\\
-1	1.11682381086406\\
-0.99	1.11189811426963\\
-0.986102409986708	1.11\\
-0.98	1.10700962900028\\
-0.97	1.10216539224016\\
-0.965477612823091	1.1\\
-0.96	1.09736078654887\\
-0.95	1.09259903106298\\
-0.944476409666283	1.09\\
-0.94	1.08788041296243\\
-0.93	1.08320217952395\\
-0.92307105181508	1.08\\
-0.92	1.07857171717172\\
-0.91	1.07397806637807\\
-0.901230994338062	1.07\\
-0.9	1.06943799012489\\
-0.89	1.06493000290444\\
-0.88	1.06047952367122\\
-0.878908429752067	1.06\\
-0.87	1.05606138555978\\
-0.86	1.05169804150833\\
-0.856056079842488	1.05\\
-0.85	1.04737569873492\\
-0.84	1.04310061782877\\
-0.832647038303217	1.04\\
-0.83	1.03887651761919\\
-0.82	1.03469084986675\\
-0.81	1.03056381403613\\
-0.808614164058519	1.03\\
-0.8	1.02647242921013\\
-0.79	1.02243636363636\\
-0.783873941392491	1.02\\
-0.78	1.01844914491449\\
-0.77	1.01450525052505\\
-0.76	1.01062076207621\\
-0.758377127617852	1.01\\
-0.75	1.0067743884023\\
-0.74	1.00298369072788\\
-0.732002752367846	1\\
-0.73	0.999247795682578\\
-0.72	0.995552143508665\\
-0.71	0.991916692003648\\
-0.70463899991496	0.99\\
-0.7	0.988330272421181\\
-0.69	0.984791551882461\\
-0.68	0.981313437404346\\
-0.676156739811913	0.98\\
-0.67	0.977881664098613\\
-0.66	0.974502003081664\\
-0.65	0.971183359013867\\
-0.646367420300825	0.97\\
-0.64	0.967911573068569\\
-0.63	0.96469314303444\\
-0.62	0.961536146447409\\
-0.615037586448833	0.96\\
-0.61	0.958429865666979\\
-0.6	0.95537488284911\\
-0.59	0.952381755701343\\
-0.58187466695087	0.95\\
-0.58	0.949446681346335\\
-0.57	0.946557407989934\\
-0.56	0.943730418370557\\
-0.55	0.940965712488204\\
-0.546426492841346	0.94\\
-0.54	0.938251187836554\\
-0.53	0.935592651251188\\
-0.52	0.932996832435857\\
-0.51	0.93046373139056\\
-0.50812283626106	0.93\\
-0.5	0.927979266347687\\
-0.49	0.925554704944178\\
-0.48	0.923193301435406\\
-0.47	0.920895055821372\\
-0.465995433138291	0.92\\
-0.46	0.918650497911982\\
-0.45	0.916463218760038\\
-0.44	0.914339543848378\\
-0.43	0.912279473177\\
-0.42	0.910283006745904\\
-0.418535815190295	0.91\\
-0.41	0.908338401811711\\
-0.4	0.906455839534131\\
-0.39	0.904637334196053\\
-0.38	0.902882885797476\\
-0.37	0.901192494338402\\
-0.36266759498707	0.9\\
-0.36	0.899563049853372\\
-0.35	0.897989573150863\\
-0.34	0.896480612577387\\
-0.33	0.895036168132942\\
-0.32	0.89365623981753\\
-0.31	0.89234082763115\\
-0.3	0.891089931573802\\
-0.290812963471577	0.89\\
-0.29	0.889902855267476\\
-0.28	0.888772891368559\\
-0.27	0.887707909419101\\
-0.26	0.8867079094191\\
-0.25	0.885772891368559\\
-0.24	0.884902855267476\\
-0.23	0.884097801115852\\
-0.22	0.883357728913685\\
-0.21	0.882682638660978\\
-0.2	0.882072530357729\\
-0.19	0.881527404003938\\
-0.18	0.881047259599606\\
-0.17	0.880632097144732\\
-0.16	0.880281916639317\\
-0.150115074798622	0.88\\
-0.15	0.879996694214876\\
-0.14	0.879774876033058\\
-0.13	0.879618512396694\\
-0.12	0.879527603305785\\
-0.11	0.87950214876033\\
-0.1	0.87954214876033\\
-0.0900000000000001	0.879647603305785\\
-0.0800000000000001	0.879818512396694\\
-0.0723216783216753	0.88\\
-0.0700000000000001	0.880054479816212\\
-0.0600000000000001	0.880354118805382\\
-0.05	0.88071873974401\\
-0.04	0.881148342632097\\
-0.03	0.881642927469642\\
-0.02	0.882202494256646\\
-0.01	0.882827042993108\\
0	0.883516573679028\\
0.00999999999999979	0.884271086314407\\
0.02	0.885090580899245\\
0.0299999999999998	0.885975057433541\\
0.04	0.886924515917295\\
0.0499999999999998	0.887938956350508\\
0.0600000000000001	0.88901837873318\\
0.0685775738457155	0.89\\
0.0699999999999998	0.890161616161616\\
0.0800000000000001	0.891362333007494\\
0.0899999999999999	0.892627565982404\\
0.1	0.893957315086347\\
0.11	0.895351580319322\\
0.12	0.896810361681329\\
0.13	0.898333659172369\\
0.14	0.89992147279244\\
0.140475251429699	0.9\\
0.15	0.901562601099968\\
0.16	0.903267227434487\\
0.17	0.905035910708508\\
0.18	0.906868650922032\\
0.19	0.908765448075056\\
0.196295990760602	0.91\\
0.2	0.910721169290074\\
0.21	0.912731769996788\\
0.22	0.914805974943784\\
0.23	0.916943784131063\\
0.24	0.919145197558625\\
0.243773932775493	0.92\\
0.25	0.921400318979266\\
0.26	0.92371259968102\\
0.27	0.926088038277512\\
0.28	0.92852663476874\\
0.285889328063243	0.93\\
0.29	0.931021222679759\\
0.3	0.933568261007285\\
0.31	0.936178017104846\\
0.32	0.938850490972442\\
0.32420266357846	0.94\\
0.33	0.941574709027996\\
0.34	0.944353255740799\\
0.35	0.947194086190626\\
0.359665185827285	0.95\\
0.36	0.950096532333646\\
0.37	0.953041549515776\\
0.38	0.95604842236801\\
0.39	0.959117150890347\\
0.392820077836544	0.96\\
0.4	0.962232392181197\\
0.41	0.965403040645361\\
0.42	0.968635122556624\\
0.42414413565709	0.97\\
0.43	0.971915562403698\\
0.44	0.975247765793528\\
0.45	0.978640986132511\\
0.453934338478009	0.98\\
0.46	0.982081114172023\\
0.47	0.985572696663606\\
0.48	0.989124885215794\\
0.482422265525715	0.99\\
0.49	0.992719367588933\\
0.5	0.996368197020371\\
0.509791786211985	1\\
0.51	1.00007671398369\\
0.52	1.0038209000302\\
0.53	1.00762488674117\\
0.536147111701712	1.01\\
0.54	1.01147884788479\\
0.55	1.01537653765377\\
0.56	1.01933363336334\\
0.561659072234257	1.02\\
0.57	1.02332816691505\\
0.58	1.02737734724292\\
0.586383951244286	1.03\\
0.59	1.03147586615339\\
0.6	1.03561593129997\\
0.61	1.03981462836837\\
0.610435417681019	1.04\\
0.62	1.04404560164754\\
0.63	1.04833362753751\\
0.633834021525756	1.05\\
0.64	1.05266267173341\\
0.65	1.05703887752119\\
0.656678093480125	1.06\\
0.66	1.06146354923032\\
0.67	1.06592680801626\\
0.67900995823964	1.07\\
0.68	1.07044473304473\\
0.69	1.07499393939394\\
0.7	1.07960028860029\\
0.700857107494276	1.08\\
0.71	1.0842368798394\\
0.72	1.08892773157442\\
0.722258532165509	1.09\\
0.73	1.09365232259903\\
0.74	1.09842661726988\\
0.743257034983187	1.1\\
0.75	1.1032370433305\\
0.76	1.10809374114982\\
0.763880209834554	1.11\\
0.77	1.11298789755136\\
0.78	1.11792597804672\\
0.784153187172406	1.12\\
0.79	1.12290181818182\\
0.8	1.12792027972028\\
0.804098902916368	1.13\\
0.81	1.13297581317765\\
0.82	1.13807367250487\\
0.823738332883734	1.14\\
0.83	1.14320696324952\\
0.84	1.14838325504283\\
0.843090697797264	1.15\\
0.85	1.15359241966493\\
0.86	1.15884619609997\\
0.862173643089978	1.16\\
0.87	1.1641294021294\\
0.88	1.16945973245973\\
0.881003397049131	1.17\\
0.89	1.17481519674355\\
0.899590883991768	1.18\\
0.9	1.18021985432965\\
0.91	1.18564715403291\\
0.917942117438598	1.19\\
0.92	1.19112121212121\\
0.93	1.19662268704747\\
0.93608023946314	1.2\\
0.94	1.20216448946947\\
0.95	1.20773926952813\\
0.954017243829646	1.21\\
0.96	1.21334720381659\\
0.97	1.21899443413729\\
0.971764240874216	1.22\\
0.98	1.22466693017128\\
0.989325438879418	1.23\\
0.99	1.23038354728845\\
1	1.23612129944983\\
1.00669939816281	1.24\\
1.01	1.24189997395155\\
1.02	1.24770799687419\\
1.02391153589687	1.25\\
1.03	1.25354726754727\\
1.04	1.25942476042476\\
1.04097025031672	1.26\\
1.05	1.26532320370847\\
1.057865217203	1.27\\
1.06	1.27126222791293\\
1.07	1.27722560819462\\
1.07461315733447	1.28\\
1.08	1.28322154316272\\
1.09	1.28925235548765\\
1.09122942925338	1.29\\
1.1	1.29530387439858\\
1.10770173179949	1.3\\
1.11	1.30139360362629\\
1.12	1.30750717703349\\
1.12404453524004	1.31\\
1.13	1.31365038817931\\
1.14	1.3198294515402\\
1.14027381287443	1.32\\
1.15	1.32602490660025\\
1.15636642865691	1.33\\
1.16	1.33225637849889\\
1.17	1.33851523408472\\
1.17235381896721	1.34\\
1.18	1.34479699433358\\
1.18822935744067	1.35\\
1.19	1.35111345258515\\
1.2	1.35745037980887\\
1.20399286235082	1.36\\
1.21	1.36381525712893\\
1.21966447042693	1.37\\
1.22	1.37021357575758\\
1.23	1.37662690909091\\
1.23522042546805	1.38\\
1.24	1.38307185917531\\
1.25	1.38954666023632\\
1.2506950349366	1.39\\
1.26	1.39603717917966\\
1.26606341982604	1.4\\
1.27	1.40255929372465\\
1.28	1.40910785015509\\
1.28135260282893	1.41\\
1.29	1.41567386660337\\
1.29654646025646	1.42\\
1.3	1.42227036599764\\
1.31	1.42889114521842\\
1.31166306618975	1.43\\
1.32	1.43552995066949\\
1.32669210480042	1.44\\
1.33	1.44219817714419\\
1.34	1.4488896938537\\
1.34164787901911	1.45\\
1.35	1.45559869797721\\
1.35652106514175	1.46\\
1.36	1.46233610918344\\
1.37	1.4690969234328\\
1.37132676295667	1.47\\
1.38	1.47587364787112\\
1.38605239172265	1.48\\
1.39	1.48267781085413\\
1.4	1.48950652621937\\
1.40071787867684	1.49\\
1.41	1.49634859876965\\
1.41530363702552	1.5\\
1.42	1.50321718431195\\
1.4298369225711	1.51\\
1.43	1.51011188811189\\
1.44	1.51701759530792\\
1.44429099996754	1.52\\
1.45	1.52394837261504\\
1.45869425005644	1.53\\
1.46	1.53090440026804\\
1.47	1.53787491623855\\
1.47302945390861	1.54\\
1.48	1.54486574794399\\
1.48730911622314	1.55\\
1.49	1.55188099977881\\
1.5	1.55891506303915\\
1.50153286040189	1.56\\
1.51	1.56596390050627\\
1.51569494052241	1.57\\
1.52	1.57303636363636\\
1.52981293409477	1.58\\
1.53	1.58013211249182\\
1.54	1.58723762807935\\
1.54386417004666	1.59\\
1.55	1.59436537209807\\
1.55787232107672	1.6\\
1.56	1.60151565536601\\
1.57	1.60868192614986\\
1.57182836603265	1.61\\
1.58	1.61586309907586\\
1.58573176904982	1.62\\
1.59	1.62306609625668\\
1.59959592670436	1.63\\
1.6	1.63029061102832\\
1.61	1.63752480306579\\
1.61340170289961	1.64\\
1.62	1.64477876668786\\
1.62716770909091	1.65\\
1.63	1.65205357519511\\
1.64	1.65934591858258\\
1.64089183514998	1.66\\
1.65	1.66664917069074\\
1.65456404243759	1.67\\
1.66	1.67397262277952\\
1.66820115449143	1.68\\
1.67	1.68131599750364\\
1.68	1.68867297690868\\
1.68179371819025	1.69\\
1.69	1.69604348726444\\
1.69534269175918	1.7\\
1.7	1.70343331271903\\
1.70885870018632	1.71\\
1.71	1.71084219166838\\
1.72	1.7182620562282\\
1.72232952826296	1.72\\
1.73	1.72569683350357\\
1.73576259130578	1.73\\
1.74	1.73315009151922\\
1.74916464858099	1.74\\
1.75	1.74062158331646\\
1.76	1.74810265235878\\
1.7625226801626	1.75\\
1.77	1.75559887119532\\
1.77584656580533	1.76\\
1.78	1.76311278346378\\
1.7891412513651	1.77\\
1.79	1.77064415584416\\
1.8	1.77818481518481\\
1.8023946334906	1.78\\
1.81	1.78573980505272\\
1.81561554395993	1.79\\
1.82	1.79331174489998\\
1.82880892135125	1.8\\
1.83	1.8009004141195\\
1.84	1.80849911260106\\
1.84196509256151	1.81\\
1.85	1.81611034753583\\
1.8550887513165	1.82\\
1.86	1.82373782991202\\
1.86818641858563	1.83\\
1.87	1.83138135098306\\
1.88	1.83903659723574\\
1.88125218227361	1.84\\
1.89	1.84670168637333\\
1.89428388006344	1.85\\
1.9	1.85438235861803\\
1.90729101430397	1.86\\
1.91	1.86207841629829\\
1.92	1.86978877570632\\
1.92027260324941	1.87\\
1.93	1.87750545454545\\
1.93321723891097	1.88\\
1.94	1.88523708785973\\
1.94613863574955	1.89\\
1.95	1.89298348832796\\
1.95903713119362	1.9\\
1.96	1.90074447174447\\
1.97	1.9085137025137\\
1.97190388572812	1.91\\
1.98	1.91629418407679\\
1.98474399441004	1.92\\
1.99	1.92408884723524\\
1.99756240857533	1.93\\
2	1.93189751726713\\
31.9696909090909	567\\
-2	2.00937506804573\\
-1.99	2.00260660497187\\
-1.98612834541681	2\\
-1.98	1.99585753324831\\
-1.97128790136838	1.99\\
-1.97	1.98913060179257\\
-1.96	1.98241631607829\\
-1.95638171409789	1.98\\
-1.95	1.97572102846648\\
-1.94142105697998	1.97\\
-1.94	1.9690484971418\\
-1.93	1.96238926793288\\
-1.92639231518864	1.96\\
-1.92	1.95574930568413\\
-1.91130610073634	1.95\\
-1.91	1.94913273842722\\
-1.9	1.94252946644358\\
-1.89614790068231	1.94\\
-1.89	1.93594642523801\\
-1.88093044443812	1.93\\
-1.88	1.92938744142455\\
-1.87	1.92284104967198\\
-1.86563538457108	1.92\\
-1.86	1.91631658196876\\
-1.85028063367025	1.91\\
-1.85	1.90981685881686\\
-1.84	1.90332829332829\\
-1.83484076992939	1.9\\
-1.83	1.89686411083697\\
-1.82	1.89042360979313\\
-1.81933841183271	1.89\\
-1.81	1.88399561654279\\
-1.8037490682489	1.88\\
-1.8	1.87759349282297\\
-1.79	1.87121244019139\\
-1.78808858582506	1.87\\
-1.78	1.86484758793004\\
-1.77234421004705	1.86\\
-1.77	1.8585093611272\\
-1.76	1.85218876664737\\
-1.7565160219976	1.85\\
-1.75	1.84588893196356\\
-1.74060894562765	1.84\\
-1.74	1.83961650768931\\
-1.73	1.83335740704691\\
-1.72460272257863	1.83\\
-1.72	1.82712453567937\\
-1.71	1.82091593352884\\
-1.7085154789442	1.82\\
-1.7	1.8147233457687\\
-1.69232932623322	1.81\\
-1.69	1.80855945572865\\
-1.68	1.80241411950306\\
-1.67604650411755	1.8\\
-1.67	1.7962917409388\\
-1.66	1.79019805902159\\
-1.65967287120939	1.79\\
-1.65	1.7841179629998\\
-1.64318395838135	1.78\\
-1.64	1.77806793206793\\
-1.63	1.77203936063936\\
-1.62659482902419	1.77\\
-1.62	1.76603291190046\\
-1.61	1.76005719446117\\
-1.60990364785828	1.76\\
-1.6	1.75409514210845\\
-1.59308532725231	1.75\\
-1.59	1.74816460822029\\
-1.58	1.74225592225147\\
-1.576155942729	1.74\\
-1.57	1.7363711612772\\
-1.56	1.73051657514745\\
-1.55911154639896	1.73\\
-1.55	1.7246784473953\\
-1.54193371138741	1.72\\
-1.54	1.71887338395239\\
-1.53	1.71308783090499\\
-1.52462511162708	1.71\\
-1.52	1.70733085961657\\
-1.51	1.70160070088642\\
-1.50718649226421	1.7\\
-1.5	1.69589273141437\\
-1.49	1.69021847173328\\
-1.48961217512774	1.69\\
-1.48	1.68456043270231\\
-1.47188224402888	1.68\\
-1.47	1.67893772204807\\
-1.46	1.67333542319749\\
-1.45400202938855	1.67\\
-1.45	1.66776422422843\\
-1.44	1.6622191895864\\
-1.43596764963949	1.66\\
-1.43	1.65670048512972\\
-1.42	1.6512132461506\\
-1.41777201069063	1.65\\
-1.41	1.64574803983895\\
-1.4	1.64031913541004\\
-1.3994075764132	1.64\\
-1.39	1.63490845220353\\
-1.38085953296594	1.63\\
-1.38	1.62953625668449\\
-1.37	1.62418331550802\\
-1.36212268900794	1.62\\
-1.36	1.61886739737804\\
-1.35	1.61357425317\\
-1.34319266505669	1.61\\
-1.34	1.60831569855323\\
-1.33	1.60308291945584\\
-1.32405991262742	1.6\\
-1.32	1.5978828379258\\
-1.31	1.59271100021697\\
-1.30471424341131	1.59\\
-1.3	1.58757052539786\\
-1.29	1.58246021364726\\
-1.2851447747709	1.58\\
-1.28	1.57738050383352\\
-1.27	1.57233231106243\\
-1.26533986956712	1.57\\
-1.26	1.56731454985692\\
-1.25	1.56232907770196\\
-1.24528706961828	1.56\\
-1.24	1.55737447467374\\
-1.23	1.55245233355452\\
-1.22497302199048	1.55\\
-1.22	1.54756212491665\\
-1.21	1.5427039342076\\
-1.20438339720209	1.54\\
-1.2	1.53787938351575\\
-1.19	1.53308577172214\\
-1.18350279828811	1.53\\
-1.18	1.52832817059484\\
-1.17	1.52359977553311\\
-1.16231465951023	1.52\\
-1.16	1.51891044439432\\
-1.15	1.51424791337695\\
-1.14080113331054	1.51\\
-1.14	1.50962820222172\\
-1.13	1.50503219224666\\
-1.12	1.50048107005214\\
-1.118932434472	1.5\\
-1.11	1.49595465937571\\
-1.1	1.49147095010253\\
-1.09668600174529	1.49\\
-1.09	1.48701740325166\\
-1.08	1.4826017861232\\
-1.07404663348179	1.48\\
-1.07	1.47822255466053\\
-1.06	1.47387571921749\\
-1.05098935202526	1.47\\
-1.05	1.46957228776313\\
-1.04	1.46529493407356\\
-1.03	1.46106338191071\\
-1.0274595192042	1.46\\
-1.02	1.45686166007905\\
-1.01	1.45270053475936\\
-1.00343748234364	1.45\\
-1	1.4485781724702\\
-0.99	1.44448819817714\\
-0.98	1.44044449637766\\
-0.978888044431452	1.44\\
-0.97	1.43642870566126\\
-0.96	1.43245712943387\\
-0.953739900652343	1.43\\
-0.95	1.42852443919717\\
-0.94	1.42462573789847\\
-0.93	1.42077378984652\\
-0.927966490847037	1.42\\
-0.92	1.41695276525042\\
-0.91	1.4131746973653\\
-0.901491189006934	1.41\\
-0.9	1.40944070627535\\
-0.89	1.40573729420186\\
-0.88	1.40208112622286\\
-0.874233388429752	1.4\\
-0.87	1.39846414008155\\
-0.86	1.39488366514752\\
-0.85	1.39135068361717\\
-0.846124836556328	1.39\\
-0.84	1.38785387026766\\
-0.83	1.38439763684591\\
-0.82	1.38098914878225\\
-0.817056755399296	1.38\\
-0.81	1.37761575757576\\
-0.8	1.37428509090909\\
-0.79	1.37100242424242\\
-0.786900996777336	1.37\\
-0.78	1.36775578844748\\
-0.77	1.36455203509627\\
-0.76	1.36139653911772\\
-0.755505529845477	1.36\\
-0.75	1.35828007841215\\
-0.74	1.35520460671404\\
-0.73	1.35217765253614\\
-0.722688605512135	1.35\\
-0.72	1.34919487558512\\
-0.71	1.34624907612712\\
-0.7	1.34335205715694\\
-0.69	1.34050381867455\\
-0.688200299216755	1.34\\
-0.68	1.33769185038395\\
-0.67	1.33492618280902\\
-0.66	1.33220956155561\\
-0.651716965363543	1.33\\
-0.65	1.32953947696139\\
-0.64	1.32690660024907\\
-0.63	1.32432303860523\\
-0.62	1.32178879202989\\
-0.612801443319634	1.32\\
-0.61	1.31930002504383\\
-0.6	1.31685098923115\\
-0.59	1.31445154019534\\
-0.58	1.31210167793639\\
-0.570863364180729	1.31\\
-0.57	1.30980030219088\\
-0.56	1.30753714429615\\
-0.55	1.30532384789726\\
-0.54	1.30316041299421\\
-0.53	1.30104683958701\\
-0.524927394752896	1.3\\
-0.52	1.29897746264877\\
-0.51	1.29695239301089\\
-0.5	1.29497746264877\\
-0.49	1.29305267156242\\
-0.48	1.29117801975184\\
-0.473543372657876	1.29\\
-0.47	1.28934988540871\\
-0.46	1.28756557168322\\
-0.45	1.2858316781258\\
-0.44	1.28414820473644\\
-0.43	1.28251515151515\\
-0.42	1.28093251846193\\
-0.413913910586668	1.28\\
-0.41	1.27939692701665\\
-0.4	1.27790678617158\\
-0.39	1.27646734955186\\
-0.38	1.27507861715749\\
-0.37	1.27374058898848\\
-0.36	1.27245326504481\\
-0.35	1.2712166453265\\
-0.34	1.27003072983355\\
-0.339729302955109	1.27\\
-0.33	1.26888926088076\\
-0.32	1.26779860932269\\
-0.31	1.26675894926603\\
-0.3	1.26577028071079\\
-0.29	1.26483260365697\\
-0.28	1.26394591810456\\
-0.27	1.26311022405357\\
-0.26	1.26232552150399\\
-0.25	1.26159181045583\\
-0.24	1.26090909090909\\
-0.23	1.26027736286377\\
-0.225223946784916	1.26\\
-0.22	1.2596948976949\\
-0.21	1.25916213416213\\
-0.2	1.25868065268065\\
-0.19	1.25825045325045\\
-0.18	1.25787153587154\\
-0.17	1.2575439005439\\
-0.16	1.25726754726755\\
-0.15	1.25704247604248\\
-0.14	1.25686868686869\\
-0.13	1.25674617974618\\
-0.12	1.25667495467495\\
-0.11	1.25665501165501\\
-0.1	1.25668635068635\\
-0.0900000000000001	1.25676897176897\\
-0.0800000000000001	1.25690287490288\\
-0.0700000000000001	1.25708806008806\\
-0.0600000000000001	1.25732452732453\\
-0.05	1.25761227661228\\
-0.04	1.25795130795131\\
-0.03	1.25834162134162\\
-0.02	1.25878321678322\\
-0.01	1.25927609427609\\
0	1.25982025382025\\
0.00301870378424739	1.26\\
0.00999999999999979	1.26041334020088\\
0.02	1.26105639969096\\
0.0299999999999998	1.26175045068246\\
0.04	1.26249549317538\\
0.0499999999999998	1.26329152716971\\
0.0600000000000001	1.26413855266546\\
0.0699999999999998	1.26503656966263\\
0.0800000000000001	1.26598557816122\\
0.0899999999999999	1.26698557816122\\
0.1	1.26803656966263\\
0.11	1.26913855266546\\
0.117471521107884	1.27\\
0.12	1.27028988476312\\
0.13	1.27148706786172\\
0.14	1.27273495518566\\
0.15	1.27403354673496\\
0.16	1.2753828425096\\
0.17	1.2767828425096\\
0.18	1.27823354673496\\
0.19	1.27973495518566\\
0.191707639003462	1.28\\
0.2	1.2812798573975\\
0.21	1.28287369493252\\
0.22	1.2845179526356\\
0.23	1.28621263050675\\
0.24	1.28795772854596\\
0.25	1.28975324675325\\
0.251336736101531	1.29\\
0.26	1.29159027601925\\
0.27	1.29347606989111\\
0.28	1.29541200303874\\
0.29	1.29739807546214\\
0.3	1.29943428716131\\
0.302711494113362	1.3\\
0.31	1.30151221354822\\
0.32	1.30363686728784\\
0.33	1.30581138252329\\
0.34	1.3080357592546\\
0.348636917284907	1.31\\
0.35	1.31030828950664\\
0.36	1.31261958427248\\
0.37	1.31498046581518\\
0.38	1.31739093413474\\
0.39	1.31985098923115\\
0.39059375311845	1.32\\
0.4	1.32234769613948\\
0.41	1.32489290161893\\
0.42	1.32748742216687\\
0.429503532736693	1.33\\
0.43	1.33013054248204\\
0.44	1.33280901659648\\
0.45	1.33553653703245\\
0.46	1.33831310378994\\
0.465970018409748	1.34\\
0.47	1.34113254496181\\
0.48	1.3439916235526\\
0.49	1.34689948263119\\
0.5	1.34985612219759\\
0.50047872776457	1.35\\
0.51	1.352846116148\\
0.52	1.35588385199706\\
0.53	1.35897010536633\\
0.5332853904479	1.36\\
0.54	1.36209359005606\\
0.55	1.3652598098952\\
0.56	1.368474287107\\
0.564676178382012	1.37\\
0.57	1.37172775757576\\
0.58	1.37502109090909\\
0.59	1.37836242424242\\
0.594831557113224	1.38\\
0.6	1.38174246443212\\
0.61	1.38516156257536\\
0.62	1.38862840607668\\
0.623902572898799	1.39\\
0.63	1.3921316862557\\
0.64	1.39567522187575\\
0.65	1.3992662508995\\
0.652016612828795	1.4\\
0.66	1.40288952517299\\
0.67	1.40655619183966\\
0.679272727272727	1.41\\
0.68	1.41026869214337\\
0.69	1.41401020650368\\
0.7	1.41779871825303\\
0.705739216535676	1.42\\
0.71	1.42162573789847\\
0.72	1.4254880755608\\
0.73	1.42939716646989\\
0.73152390616606	1.43\\
0.74	1.43333568240545\\
0.75	1.43731759455015\\
0.756658697300134	1.44\\
0.76	1.44133909792007\\
0.77	1.44539308249591\\
0.78	1.44949333956532\\
0.781221890322944	1.45\\
0.79	1.45362125087189\\
0.8	1.45779260637061\\
0.805234026131539	1.46\\
0.81	1.46199976867916\\
0.82	1.46624149895906\\
0.828766118154842	1.47\\
0.83	1.47052635212888\\
0.84	1.47483774453395\\
0.85	1.47919470655926\\
0.851829159914266	1.48\\
0.86	1.48357911609801\\
0.87	1.48800480879322\\
0.874462483994879	1.49\\
0.88	1.49246343130554\\
0.89	1.49695716564138\\
0.896703980723859	1.5\\
0.9	1.5014885513489\\
0.91	1.50604964860576\\
0.918576561500221	1.51\\
0.92	1.51065237987819\\
0.93	1.51528017144146\\
0.94	1.51995262801714\\
0.940100425613313	1.52\\
0.95	1.52464668911336\\
0.96	1.52938496071829\\
0.961285962359787	1.53\\
0.97	1.53414719678356\\
0.98	1.538950636587\\
0.982164677694328	1.54\\
0.99	1.54377972882863\\
1	1.54864769948878\\
1.00275306574958	1.55\\
1.01	1.55354235788542\\
1.02	1.55847423136474\\
1.02306645921316	1.56\\
1.03	1.56343319392472\\
1.04	1.56842835130971\\
1.043119129789	1.57\\
1.05	1.57345038335159\\
1.06	1.57850821467689\\
1.06292437840855	1.58\\
1.07	1.58359210813168\\
1.08	1.58871201220842\\
1.0824946163915	1.59\\
1.09	1.5938565849425\\
1.1	1.59903796919071\\
1.10184143859795	1.6\\
1.11	1.60424206434895\\
1.12	1.60948434463399\\
1.12097568947906	1.61\\
1.13	1.61474683000215\\
1.13990678069144	1.62\\
1.14	1.62004919786096\\
1.15	1.62536919786096\\
1.15863576528781	1.63\\
1.16	1.63072812433468\\
1.17	1.63610751543538\\
1.1771796583546	1.64\\
1.18	1.64152193261284\\
1.19	1.64696016105107\\
1.19554696260779	1.65\\
1.2	1.65242902341278\\
1.21	1.65792554313436\\
1.21374566782191	1.66\\
1.22	1.66344782699979\\
1.23	1.66900209951711\\
1.23178328893558	1.67\\
1.24	1.67457680250784\\
1.24966444452721	1.68\\
1.25	1.68018743499064\\
1.26	1.68581443727897\\
1.2673842991889	1.69\\
1.27	1.69147587492234\\
1.28	1.69715924622075\\
1.28496255833303	1.7\\
1.29	1.70287054215626\\
1.3	1.7086097711812\\
1.3024052212989	1.71\\
1.31	1.71436999794788\\
1.31971597549315	1.72\\
1.32	1.72016384065373\\
1.33	1.72597282941777\\
1.33688471344253	1.73\\
1.34	1.73181411429734\\
1.35	1.73767764897295\\
1.35393365255434	1.74\\
1.36	1.74356549908888\\
1.37	1.74948309374367\\
1.37086762956669	1.75\\
1.38	1.75541664986898\\
1.38767579245856	1.76\\
1.39	1.76138169777243\\
1.4	1.7673662452338\\
1.40437189779806	1.77\\
1.41	1.77337562437562\\
1.42	1.77941298701299\\
1.42096597073812	1.78\\
1.43	1.78546588422518\\
1.43744552967693	1.79\\
1.44	1.79154882154882\\
1.45	1.79765121806298\\
1.45382437356896	1.8\\
1.46	1.80377637546835\\
1.47	1.80993038848353\\
1.47011240248368	1.81\\
1.48	1.81609699587669\\
1.48628990918584	1.82\\
1.49	1.82229227761486\\
1.5	1.82850948191593\\
1.50238257445545	1.83\\
1.51	1.8347449873467\\
1.51838432152064	1.84\\
1.52	1.84100833494863\\
1.53	1.84728765264586\\
1.53429325315252	1.85\\
1.54	1.85359004053272\\
1.55	1.85991912758155\\
1.55012701203431	1.86\\
1.56	1.86625985008649\\
1.56586374182662	1.87\\
1.57	1.87262717703349\\
1.58	1.8790166507177\\
1.58152994074383	1.88\\
1.59	1.8854211930627\\
1.59711198602765	1.89\\
1.6	1.89185158474094\\
1.61	1.89830043651547\\
1.61262018316412	1.9\\
1.62	1.90476696276696\\
1.62805469090909	1.91\\
1.63	1.91125861095426\\
1.64	1.91776585733107\\
1.64341376434385	1.92\\
1.65	1.92429259606373\\
1.65870742657783	1.93\\
1.66	1.93084375583349\\
1.67	1.93740843755833\\
1.67392563268769	1.94\\
1.68	1.94399367912251\\
1.68908472288615	1.95\\
1.69	1.95060266617293\\
1.7	1.95722384743566\\
1.70416974888067	1.96\\
1.71	1.96386594136087\\
1.71920015488011	1.97\\
1.72	1.97053112947658\\
1.73	1.97720789715335\\
1.73415916614231	1.98\\
1.74	1.9839052496799\\
1.74906641996136	1.99\\
1.75	1.99062506831846\\
1.76	1.99735653124431\\
1.76390610278084	2\\
1.77	2.00410760297587\\
1.7786954076851	2.01\\
1.78	2.01088053497199\\
1.79	2.01766582324236\\
1.79342200789592	2.02\\
1.8	2.02446912691269\\
1.80809826300113	2.03\\
1.81	2.03129370629371\\
1.82	2.03813197059351\\
1.822717620973	2.04\\
1.83	2.04498606894088\\
1.83728544364571	2.05\\
1.84	2.05186087884718\\
1.85	2.05875128980608\\
1.85180302602173	2.06\\
1.86	2.06565479354953\\
1.86626677230557	2.07\\
1.87	2.07257846425419\\
1.88	2.07952021182701\\
1.88068770083243	2.08\\
1.89	2.0864717777387\\
1.89505148410161	2.09\\
1.9	2.0934429847609\\
1.90937748941607	2.1\\
1.91	2.1004336067004\\
1.92	2.1074336067004\\
1.92364826987474	2.11\\
1.93	2.11445106900747\\
1.93787969293807	2.12\\
1.94	2.12148744588745\\
1.95	2.12853696969697\\
1.9520653157007	2.13\\
1.96	2.13559944799034\\
1.96620575571071	2.14\\
1.97	2.14268035744973\\
1.98	2.14977865612648\\
1.98031033901164	2.15\\
1.99	2.15688495120699\\
1.99436320471931	2.16\\
2	2.16400921030189\\
43.7979727272727	549\\
-2	2.23491723450224\\
-1.99206804107128	2.23\\
-1.99	2.22871326105088\\
-1.98	2.22252427022519\\
-1.97589947161631	2.22\\
-1.97	2.21635493052068\\
-1.96	2.21020944249121\\
-1.95965734476431	2.21\\
-1.95	2.2040752814653\\
-1.9433210321941	2.2\\
-1.94	2.19796609883623\\
-1.93	2.19187518974532\\
-1.92690435751079	2.19\\
-1.92	2.18580209920433\\
-1.91040428927402	2.18\\
-1.91	2.17975463041631\\
-1.9	2.17371911639762\\
-1.89380340307466	2.17\\
-1.89	2.16770868156234\\
-1.88	2.16171806242538\\
-1.87711581961346	2.16\\
-1.87	2.15574524910118\\
-1.86033662568031	2.15\\
-1.86	2.14979910637567\\
-1.85	2.14386526894655\\
-1.84344848679036	2.14\\
-1.84	2.13795773676039\\
-1.83	2.1320696912196\\
-1.82646441726831	2.13\\
-1.82	2.12620138528139\\
-1.81	2.12035948051948\\
-1.80938101911208	2.12\\
-1.8	2.11453102729011\\
-1.79218169820941	2.11\\
-1.79	2.10873076251963\\
-1.78	2.10294765311464\\
-1.77487236302929	2.1\\
-1.77	2.09718829917674\\
-1.76	2.09145226834822\\
-1.75745276352576	2.09\\
-1.75	2.08573448215228\\
-1.74	2.08004589414454\\
-1.73991882561503	2.08\\
-1.73	2.07437034421889\\
-1.72225222969801	2.07\\
-1.72	2.06872461456672\\
-1.71	2.06309693425483\\
-1.70446243543839	2.06\\
-1.7	2.05749457391923\\
-1.69	2.05191531755915\\
-1.68654526200943	2.05\\
-1.68	2.04635702804072\\
-1.67	2.04082657617432\\
-1.66849579094484	2.04\\
-1.66	2.03531307154384\\
-1.65030684680559	2.03\\
-1.65	2.02983114311431\\
-1.64	2.02436381638164\\
-1.63196599608922	2.02\\
-1.63	2.01892788722212\\
-1.62	2.0135103921923\\
-1.61347718037411	2.01\\
-1.61	2.00812121212121\\
-1.6	2.00275394665215\\
-1.59483441679997	2\\
-1.59	1.99741227910366\\
-1.58	1.99209564583713\\
-1.57603139554942	1.99\\
-1.57	1.98680226815438\\
-1.56	1.98153667459301\\
-1.55706145720382	1.98\\
-1.55	1.97629237832874\\
-1.54	1.9710782369146\\
-1.53791756819069	1.97\\
-1.53	1.96588382813941\\
-1.52	1.96072155633413\\
-1.51859229413246	1.96\\
-1.51	1.9555778559526\\
-1.5	1.9504678763192\\
-1.49907777088427	1.95\\
-1.49	1.94537572039413\\
-1.48	1.94031846068042\\
-1.47936567302351	1.94\\
-1.47	1.93527870076535\\
-1.46	1.93027459398917\\
-1.45944717952573	1.93\\
-1.45	1.92528809746954\\
-1.44	1.92033758200562\\
-1.4393129363293	1.92\\
-1.43	1.91540523244871\\
-1.42	1.91050875211745\\
-1.41895301545493	1.91\\
-1.41	1.90563144963145\\
-1.4	1.90078945378945\\
-1.39835687030408	1.9\\
-1.39	1.89596811539191\\
-1.38	1.89118105902448\\
-1.37751328671329	1.89\\
-1.37	1.88641661902039\\
-1.36	1.88168496283591\\
-1.35641032928661	1.88\\
-1.35	1.87697837320574\\
-1.34	1.87230258373206\\
-1.33503528246606	1.87\\
-1.33	1.86765481453008\\
-1.32	1.86303536421295\\
-1.31337458572807	1.86\\
-1.31	1.85844740397607\\
-1.3	1.85388477127968\\
-1.29141376221151	1.85\\
-1.29	1.84935762744718\\
-1.28	1.84485229695677\\
-1.27	1.84038534599729\\
-1.2691298638771	1.84\\
-1.26	1.83593945882811\\
-1.25	1.83153046525209\\
-1.24649815153	1.83\\
-1.24	1.82714780058651\\
-1.23	1.8227972629521\\
-1.22351258218091	1.82\\
-1.22	1.81847889259768\\
-1.21	1.81418731592382\\
-1.20015374671037	1.81\\
-1.2	1.8099343324788\\
-1.19	1.80570222835733\\
-1.18	1.801509169789\\
-1.17636695941135	1.8\\
-1.17	1.79734363240246\\
-1.16	1.79321073479897\\
-1.1521568532585	1.79\\
-1.15	1.78911318878058\\
-1.14	1.78504097871494\\
-1.13	1.78100815595783\\
-1.12747546699875	1.78\\
-1.12	1.7770015984016\\
-1.11	1.77303016983017\\
-1.10229330758677	1.77\\
-1.1	1.76909432069035\\
-1.09	1.76518482841662\\
-1.08	1.76131507124222\\
-1.07656641341368	1.76\\
-1.07	1.75747389639186\\
-1.06	1.75366680104818\\
-1.05026646690567	1.75\\
-1.05	1.74989916987244\\
-1.04	1.74615529459405\\
-1.03	1.74245150840251\\
-1.02330864879801	1.74\\
-1.02	1.73878238763474\\
-1.01	1.73514256660565\\
-1	1.73154301403295\\
-0.995664819153192	1.73\\
-0.99	1.7279746680286\\
-0.98	1.72443983656793\\
-0.97	1.72094545454545\\
-0.967262672265926	1.72\\
-0.96	1.7174801970039\\
-0.95	1.71405130309871\\
-0.94	1.71066304124769\\
-0.938019371053761	1.71\\
-0.93	1.70730323644609\\
-0.92	1.70398124098124\\
-0.91	1.70070006184292\\
-0.907839557223743	1.7\\
-0.9	1.69744812590598\\
-0.89	1.69423400289915\\
-0.88	1.69106088217022\\
-0.876612892561983	1.69\\
-0.87	1.68791928437695\\
-0.86	1.68481402121906\\
-0.85	1.68174994799251\\
-0.844210997178447	1.68\\
-0.84	1.67872121212121\\
-0.83	1.67572580982236\\
-0.82	1.67277178683386\\
-0.810483604792997	1.67\\
-0.81	1.66985849254671\\
-0.8	1.66697396598782\\
-0.79	1.66413100986773\\
-0.78	1.66132962418644\\
-0.775182198554584	1.66\\
-0.77	1.65856317232651\\
-0.76	1.65583231385784\\
-0.75	1.65314321873023\\
-0.74	1.65049588694368\\
-0.738096818586579	1.65\\
-0.73	1.64788048315321\\
-0.72	1.64530472557745\\
-0.71	1.64277092604365\\
-0.7	1.64027908455181\\
-0.698860825188134	1.64\\
-0.69	1.63781903342559\\
-0.68	1.63539982967852\\
-0.67	1.63302278049819\\
-0.66	1.63068788588461\\
-0.656999721422602	1.63\\
-0.65	1.62838759358289\\
-0.64	1.6261264171123\\
-0.63	1.62390759358289\\
-0.62	1.62173112299465\\
-0.611888343189337	1.62\\
-0.61	1.61959509993553\\
-0.6	1.61749344508919\\
-0.59	1.61543434343434\\
-0.58	1.61341779497099\\
-0.57	1.61144379969912\\
-0.562524757983754	1.61\\
-0.56	1.60951004102786\\
-0.55	1.60761217879508\\
-0.54	1.60575707190672\\
-0.53	1.60394472036277\\
-0.52	1.60217512416325\\
-0.51	1.60044828330814\\
-0.507338120271831	1.6\\
-0.5	1.59875829898026\\
-0.49	1.59710913430245\\
-0.48	1.59550292905185\\
-0.47	1.59393968322847\\
-0.46	1.59241939683228\\
-0.45	1.59094206986331\\
-0.443432158523672	1.59\\
-0.44	1.5895053411816\\
-0.43	1.58810725964683\\
-0.42	1.58675234357968\\
-0.41	1.58544059298016\\
-0.4	1.58417200784827\\
-0.39	1.582946588184\\
-0.38	1.58176433398736\\
-0.37	1.58062524525834\\
-0.364294808036606	1.58\\
-0.36	1.57952705366922\\
-0.35	1.57846922234392\\
-0.34	1.5774547645126\\
-0.33	1.57648368017525\\
-0.32	1.57555596933187\\
-0.31	1.57467163198247\\
-0.3	1.57383066812705\\
-0.29	1.57303307776561\\
-0.28	1.57227886089814\\
-0.27	1.57156801752464\\
-0.26	1.57090054764513\\
-0.25	1.57027645125958\\
-0.245239532251981	1.57\\
-0.24	1.56969425489764\\
-0.23	1.56915430332379\\
-0.22	1.56865793528505\\
-0.21	1.56820515078142\\
-0.2	1.5677959498129\\
-0.19	1.56743033237948\\
-0.18	1.56710829848118\\
-0.17	1.56682984811798\\
-0.16	1.5665949812899\\
-0.15	1.56640369799692\\
-0.14	1.56625599823905\\
-0.13	1.56615188201629\\
-0.12	1.56609134932864\\
-0.11	1.56607440017609\\
-0.1	1.56610103455866\\
-0.0900000000000001	1.56617125247634\\
-0.0800000000000001	1.56628505392912\\
-0.0700000000000001	1.56644243891701\\
-0.0600000000000001	1.56664340744002\\
-0.05	1.56688795949813\\
-0.04	1.56717609509135\\
-0.03	1.56750781421968\\
-0.02	1.56788311688312\\
-0.01	1.56830200308166\\
0	1.56876447281532\\
0.00999999999999979	1.56927052608409\\
0.02	1.56982016288796\\
0.0230315398886849	1.57\\
0.0299999999999998	1.57041139101862\\
0.04	1.57104512595838\\
0.0499999999999998	1.57172223439211\\
0.0600000000000001	1.57244271631982\\
0.0699999999999998	1.57320657174151\\
0.0800000000000001	1.57401380065717\\
0.0899999999999999	1.57486440306681\\
0.1	1.57575837897043\\
0.11	1.57669572836802\\
0.12	1.57767645125958\\
0.13	1.57870054764513\\
0.14	1.57976801752464\\
0.142088345493988	1.58\\
0.15	1.58087464573795\\
0.16	1.58202332679311\\
0.17	1.58321517331589\\
0.18	1.5844501853063\\
0.19	1.58572836276433\\
0.2	1.58704970568999\\
0.21	1.58841421408328\\
0.22	1.58982188794419\\
0.221227648384673	1.59\\
0.23	1.59126665220221\\
0.24	1.59275352571057\\
0.25	1.59428335864613\\
0.26	1.5958561510089\\
0.27	1.59747190279887\\
0.28	1.59913061401606\\
0.285109014407754	1.6\\
0.29	1.60082833081408\\
0.3	1.60256467285683\\
0.31	1.60434377024401\\
0.32	1.6061656229756\\
0.33	1.60803023105161\\
0.34	1.60993759447204\\
0.340320008858376	1.61\\
0.35	1.61187878787879\\
0.36	1.61386223941543\\
0.37	1.61588824414356\\
0.38	1.61795680206318\\
0.389678306016493	1.62\\
0.39	1.62006759358289\\
0.4	1.62221112299465\\
0.41	1.62439700534759\\
0.42	1.62662524064171\\
0.43	1.62889582887701\\
0.434773883288634	1.63\\
0.44	1.63120310836704\\
0.45	1.63354737066212\\
0.46	1.63593378752395\\
0.47	1.63836235895252\\
0.476628177509695	1.64\\
0.48	1.64082920110193\\
0.49	1.64333036660309\\
0.5	1.64587349014622\\
0.51	1.6484585717313\\
0.515867548600468	1.65\\
0.52	1.65108057371862\\
0.53	1.65373718624763\\
0.54	1.6564355621177\\
0.55	1.65917570132883\\
0.552963075290015	1.66\\
0.56	1.66194856183078\\
0.57	1.66475918538736\\
0.58	1.66761137938274\\
0.588254371326997	1.67\\
0.59	1.67050282131661\\
0.6	1.67342466039707\\
0.61	1.67638787878788\\
0.62	1.67939247648903\\
0.621994511149229	1.68\\
0.63	1.68242729353027\\
0.64	1.68550052007489\\
0.65	1.68861493655086\\
0.654389214846067	1.69\\
0.66	1.69176247670325\\
0.67	1.69494470904949\\
0.68	1.69816794367364\\
0.685612510308952	1.7\\
0.69	1.70142568542569\\
0.7	1.70471593485879\\
0.71	1.70804700061843\\
0.715792015650792	1.71\\
0.72	1.71141247691361\\
0.73	1.71480976810999\\
0.74	1.71824769136056\\
0.745037460916761	1.72\\
0.75	1.72171848825332\\
0.76	1.72522185903984\\
0.77	1.72876567926456\\
0.77344371615845	1.73\\
0.78	1.7323394346146\\
0.79	1.73594793573317\\
0.8	1.73959670530811\\
0.801093224543802	1.74\\
0.81	1.74327110751164\\
0.82	1.74698380238915\\
0.828037226868088	1.75\\
0.83	1.75073331989518\\
0.84	1.75450937311026\\
0.85	1.75832533763354\\
0.854343143917612	1.76\\
0.86	1.7621715833835\\
0.87	1.76605017057997\\
0.88	1.76996849287578\\
0.88007960249455	1.77\\
0.89	1.77390929070929\\
0.9	1.77788951048951\\
0.905250260947363	1.78\\
0.91	1.78190093495126\\
0.92	1.7859425104436\\
0.929942481111382	1.79\\
0.93	1.79002337096455\\
0.94	1.79412576747871\\
0.95	1.79826737967914\\
0.954144203894074	1.8\\
0.96	1.8024375862749\\
0.97	1.80663932163281\\
0.977924668681702	1.81\\
0.98	1.81087630080503\\
0.99	1.8151376398979\\
1	1.81943785588062\\
1.00129553373456	1.82\\
1.01	1.82376070381232\\
1.02	1.82811984359726\\
1.0242751722605	1.83\\
1.03	1.83250691064824\\
1.04	1.83692446953475\\
1.04690183915076	1.84\\
1.05	1.84137468501648\\
1.06	1.84585016476061\\
1.0691935414609	1.85\\
1.07	1.85036247828605\\
1.08	1.85489538699093\\
1.09	1.85946651225632\\
1.09115740546878	1.86\\
1.1	1.86405862002691\\
1.11	1.86868652700365\\
1.11281501009186	1.87\\
1.12	1.87333837320574\\
1.13	1.87802258373206\\
1.13418757346087	1.88\\
1.14	1.88273318086526\\
1.15	1.88747322279398\\
1.15528860345447	1.89\\
1.16	1.89224160182198\\
1.17	1.8970370089201\\
1.1761307677204	1.9\\
1.18	1.90186221886222\\
1.19	1.90671253071253\\
1.19672595800626	1.91\\
1.2	1.91159363824581\\
1.21	1.91649840015057\\
1.21708534866893	1.92\\
1.22	1.92143448922212\\
1.23	1.92639325210872\\
1.23721944996811	1.93\\
1.24	1.93138342355796\\
1.25	1.9363957438865\\
1.25713815667862	1.94\\
1.26	1.94143911507715\\
1.27	1.94650455474995\\
1.27685079249408	1.95\\
1.28	1.95160025921126\\
1.29	1.95671838548417\\
1.29636615064114	1.96\\
1.3	1.96186557256131\\
1.31	1.96703595795685\\
1.31569253107625	1.97\\
1.32	1.97223379247016\\
1.33	1.97745601469238\\
1.33483777459575	1.98\\
1.34	1.98270367660508\\
1.35	1.98797731845619\\
1.35380929415412	1.99\\
1.36	1.99327400255056\\
1.37	1.99859865184915\\
1.37261410365336	2\\
1.38	2.00394356741063\\
1.39	2.00931881691163\\
1.39125884443848	2.01\\
1.4	2.01471118742093\\
1.40974815949898	2.02\\
1.41	2.02013609360936\\
1.42	2.02557569756976\\
1.42808055235903	2.03\\
1.43	2.03104679935449\\
1.44	2.03653595122826\\
1.44627016325338	2.04\\
1.45	2.04205250937667\\
1.46	2.04759081978925\\
1.46432242766046	2.05\\
1.47	2.05315210816581\\
1.48	2.05873919231453\\
1.4822425086226	2.06\\
1.49	2.06434449760766\\
1.5	2.0699799751905\\
1.50003531360355	2.07\\
1.51	2.07562859664607\\
1.51769139981986	2.08\\
1.52	2.08130701600141\\
1.53	2.08700334095305\\
1.53522873009542	2.09\\
1.54	2.0927239446488\\
1.55	2.09846768260641\\
1.55265179302192	2.1\\
1.56	2.10422980282673\\
1.56996444390876	2.11\\
1.57	2.11002051103772\\
1.58	2.11582357031114\\
1.58715451269988	2.12\\
1.59	2.12165471861472\\
1.6	2.12750424242424\\
1.60424173754378	2.13\\
1.61	2.1333751940659\\
1.62	2.13927082973952\\
1.62122967272727	2.14\\
1.63	2.14518095892765\\
1.63811095363434	2.15\\
1.64	2.15111813045711\\
1.65	2.15707104947783\\
1.65489233320941	2.16\\
1.66	2.16304639263176\\
1.67	2.16904451645915\\
1.67158405202884	2.17\\
1.68	2.17505743415463\\
1.68817900627056	2.18\\
1.69	2.18109632639242\\
1.7	2.18715033011681\\
1.70468116465975	2.19\\
1.71	2.19322583909597\\
1.72	2.19932416933716\\
1.72110218676936	2.2\\
1.73	2.20543572508822\\
1.73743085382869	2.21\\
1.74	2.21157224175456\\
1.75	2.21772509626653\\
1.75367750142088	2.22\\
1.76	2.22389674728941\\
1.76984978598541	2.23\\
1.77	2.23009273724447\\
1.78	2.23629931859731\\
1.78593106571132	2.24\\
1.79	2.24252955787382\\
1.8	2.24877910250041\\
1.8019433827987	2.25\\
1.81	2.25504306220096\\
1.81787791383696	2.26\\
1.82	2.26133042906461\\
1.83	2.2676324182147\\
1.83373757558456	2.27\\
1.84	2.27395266175266\\
1.84953215770313	2.28\\
1.85	2.28029574016648\\
1.86	2.28664942059736\\
1.86524676054898	2.29\\
1.87	2.29302455683851\\
1.88	2.29941990567572\\
1.8809025124611	2.3\\
1.89	2.30582660832928\\
1.89648372397472	2.31\\
1.9	2.31225528822864\\
1.91	2.31870111416115\\
1.91200513498018	2.32\\
1.92	2.32516057924376\\
1.92746049857373	2.33\\
1.93	2.33164149430816\\
1.94	2.33813708513709\\
1.94285401979809	2.34\\
1.95	2.34464802684135\\
1.95818841819649	2.35\\
1.96	2.35117990765802\\
1.97	2.35772456615189\\
1.97346011378768	2.36\\
1.98	2.36428573695066\\
1.98867816784329	2.37\\
1.99	2.37086735177866\\
2	2.37746039525692\\
55.6262545454545	516\\
-2	2.44323012810619\\
-1.99438942655693	2.44\\
-1.99	2.43746430230757\\
-1.98	2.43171813535698\\
-1.97699390326514	2.43\\
-1.97	2.42598896658897\\
-1.96	2.42028469308469\\
-1.95949820592183	2.42\\
-1.95	2.41459270232341\\
-1.9418886838699	2.41\\
-1.94	2.40892692849319\\
-1.93	2.40327632608355\\
-1.92416984546847	2.4\\
-1.92	2.3976486104569\\
-1.91	2.3920406657246\\
-1.90634083166756	2.39\\
-1.9	2.38645155191429\\
-1.89	2.3808865605798\\
-1.88839791589557	2.38\\
-1.88	2.37533660079051\\
-1.87033528990694	2.37\\
-1.87	2.36981421545296\\
-1.86	2.36430461684912\\
-1.85214225723305	2.36\\
-1.85	2.35882232128642\\
-1.84	2.35335647189938\\
-1.83382357249583	2.35\\
-1.83	2.34791484262662\\
-1.82	2.34249305000799\\
-1.81537481103833	2.34\\
-1.81	2.33709267275934\\
-1.8	2.33171524771525\\
-1.79679133799226	2.33\\
-1.79	2.32635671761866\\
-1.78	2.32102397425583\\
-1.77806829564426	2.32\\
-1.77	2.31570789601163\\
-1.76	2.31042015178427\\
-1.75920058987987	2.31\\
-1.75	2.30514713984767\\
-1.74018175635993	2.3\\
-1.74	2.29990437469507\\
-1.73	2.29467539437307\\
-1.72100328587076	2.29\\
-1.72	2.28947674228823\\
-1.71	2.28429361841031\\
-1.70166418454324	2.28\\
-1.7	2.27913972153972\\
-1.69	2.27400278460278\\
-1.68215832878734	2.27\\
-1.68	2.26889429557784\\
-1.67	2.26380387966464\\
-1.66247927974778	2.26\\
-1.66	2.25874146180498\\
-1.65	2.2536979046362\\
-1.64262026275065	2.25\\
-1.64	2.24868223215764\\
-1.63	2.24368587514489\\
-1.62257414512093	2.24\\
-1.62	2.23871763337211\\
-1.61	2.23376882167193\\
-1.60233341221813	2.23\\
-1.6	2.22884870725605\\
-1.59	2.22394778982485\\
-1.58189014152075	2.22\\
-1.58	2.21907651096601\\
-1.57	2.21422384061611\\
-1.56123597457186	2.21\\
-1.56	2.20940211729121\\
-1.55	2.20459805074777\\
-1.54036208657673	2.2\\
-1.54	2.1998266149435\\
-1.53	2.19507151290268\\
-1.52	2.19034980603812\\
-1.51925387631759	2.19\\
-1.51	2.18564533604198\\
-1.5	2.18097308278314\\
-1.49790226634065	2.18\\
-1.49	2.17632064570943\\
-1.48	2.17169821580289\\
-1.47629920385114	2.17\\
-1.47	2.16709858434249\\
-1.46	2.16252635169708\\
-1.45443346236236	2.16\\
-1.45	2.15798031159048\\
-1.44	2.15345865434001\\
-1.43229313699309	2.15\\
-1.43	2.14896700463997\\
-1.42	2.14449630520708\\
-1.41	2.14005963223922\\
-1.40986455365159	2.14\\
-1.4	2.13564050370881\\
-1.39	2.13125530446783\\
-1.38711493478175	2.13\\
-1.38	2.12689246753247\\
-1.37	2.12255913419913\\
-1.36404720667016	2.12\\
-1.36	2.11825343299148\\
-1.35	2.11397236224578\\
-1.34064590069993	2.11\\
-1.34	2.109724655383\\
-1.33	2.10549624847322\\
-1.32	2.10130239050776\\
-1.31686873348156	2.1\\
-1.31	2.09713207216675\\
-1.3	2.09299141706078\\
-1.29271447463845	2.09\\
-1.29	2.08888113240724\\
-1.28	2.08479409178829\\
-1.27	2.08074186741692\\
-1.26815336805708	2.08\\
-1.26	2.07671173874669\\
-1.25	2.07271368049426\\
-1.24315264353481	2.07\\
-1.24	2.06874570264044\\
-1.23	2.06480223285486\\
-1.22	2.06089385078859\\
-1.21769227249851	2.06\\
-1.21	2.05700889521437\\
-1.2	2.05315566625156\\
-1.19173477470761	2.05\\
-1.19	2.04933505983211\\
-1.18	2.04553741739596\\
-1.17	2.04177513841757\\
-1.16523697704509	2.04\\
-1.16	2.03804052357899\\
-1.15	2.03433440918056\\
-1.14	2.03066379774072\\
-1.13817392591131	2.03\\
-1.13	2.02701692169217\\
-1.12	2.02340306030603\\
-1.11048951048951	2.02\\
-1.11	2.01982414603289\\
-1.1	2.01626748599313\\
-1.09	2.01274661124164\\
-1.08211896489135	2.01\\
-1.08	2.00925857376157\\
-1.07	2.00579549990927\\
-1.06	2.00236835420069\\
-1.05301621274547	2\\
-1.05	1.99897303698306\\
-1.04	1.99560429950811\\
-1.03	1.99227163417745\\
-1.02310914617298	1.99\\
-1.02	1.98897091640754\\
-1.01	1.98569727455643\\
-1	1.98245985000915\\
-0.992315867664705	1.98\\
-0.99	1.97925564738292\\
-0.98	1.97607786960514\\
-0.97	1.97293645546373\\
-0.960542970367305	1.97\\
-0.96	1.96983072100313\\
-0.95	1.9667495851005\\
-0.94	1.96370496035405\\
-0.93	1.96069684676378\\
-0.927654979832454	1.96\\
-0.92	1.9577159785225\\
-0.91	1.95476893167932\\
-0.9	1.95185854471394\\
-0.893532633206624	1.95\\
-0.89	1.94898066555122\\
-0.88	1.94613199479457\\
-0.87	1.94332013385388\\
-0.86	1.94054508272913\\
-0.858009369271505	1.94\\
-0.85	1.93779783460892\\
-0.84	1.93508530894157\\
-0.83	1.93240974425985\\
-0.820867350548286	1.93\\
-0.82	1.9297701968135\\
-0.81	1.92715782567948\\
-0.8	1.92458256794752\\
-0.79	1.92204442361762\\
-0.781825676384621	1.92\\
-0.78	1.91954150197628\\
-0.77	1.91706738189347\\
-0.76	1.9146305288914\\
-0.75	1.91223094297007\\
-0.740556130985579	1.91\\
-0.74	1.90986807786808\\
-0.73	1.90753335853336\\
-0.72	1.90523606123606\\
-0.71	1.90297618597619\\
-0.7	1.90075373275373\\
-0.696550471412508	1.9\\
-0.69	1.89856272537483\\
-0.68	1.89640614917442\\
-0.67	1.8942871512621\\
-0.66	1.89220573163788\\
-0.65	1.89016189030176\\
-0.64919307539495	1.89\\
-0.64	1.88814789403469\\
-0.63	1.88617095483133\\
-0.62	1.88423175147703\\
-0.61	1.88233028397179\\
-0.6	1.88046655231561\\
-0.597444943116584	1.88\\
-0.59	1.87863483253589\\
-0.58	1.8768390430622\\
-0.57	1.87508114832536\\
-0.56	1.87336114832536\\
-0.55	1.8716790430622\\
-0.54	1.87003483253589\\
-0.539783152627191	1.87\\
-0.53	1.86842187199692\\
-0.52	1.8668468191428\\
-0.51	1.86530982125697\\
-0.5	1.86381087833942\\
-0.49	1.86234999039016\\
-0.48	1.86092715740919\\
-0.473304649548924	1.86\\
-0.47	1.85954043620923\\
-0.46	1.85818799459564\\
-0.45	1.85687376954256\\
-0.44	1.85559776104999\\
-0.43	1.85435996911793\\
-0.42	1.85316039374638\\
-0.41	1.85199903493534\\
-0.4	1.85087589268481\\
-0.391926703433553	1.85\\
-0.39	1.84979007559605\\
-0.38	1.84873890288816\\
-0.37	1.84772610971118\\
-0.36	1.84675169606513\\
-0.35	1.84581566194999\\
-0.34	1.84491800736577\\
-0.33	1.84405873231246\\
-0.32	1.84323783679008\\
-0.31	1.8424553207986\\
-0.3	1.84171118433805\\
-0.29	1.84100542740841\\
-0.28	1.84033805000969\\
-0.274625577812019	1.84\\
-0.27	1.83970780611252\\
-0.26	1.83911465836091\\
-0.25	1.83856005450652\\
-0.24	1.83804399454935\\
-0.23	1.83756647848939\\
-0.22	1.83712750632665\\
-0.21	1.83672707806113\\
-0.2	1.83636519369282\\
-0.19	1.83604185322172\\
-0.18	1.83575705664785\\
-0.17	1.83551080397119\\
-0.16	1.83530309519175\\
-0.15	1.83513393030952\\
-0.14	1.83500330932451\\
-0.13	1.83491123223671\\
-0.12	1.83485769904614\\
-0.11	1.83484270975277\\
-0.1	1.83486626435663\\
-0.0900000000000001	1.8349283628577\\
-0.0800000000000001	1.83502900525599\\
-0.0700000000000001	1.83516819155149\\
-0.0600000000000001	1.83534592174421\\
-0.05	1.83556219583414\\
-0.04	1.8358170138213\\
-0.03	1.83611037570566\\
-0.02	1.83644228148725\\
-0.01	1.83681273116605\\
0	1.83722172474207\\
0.00999999999999979	1.8376692622153\\
0.02	1.83815534358575\\
0.0299999999999998	1.83867996885342\\
0.04	1.8392431380183\\
0.0499999999999998	1.8398448510804\\
0.0524232289449658	1.84\\
0.0600000000000001	1.84048303934871\\
0.0699999999999998	1.84115894553208\\
0.0800000000000001	1.84187323124637\\
0.0899999999999999	1.84262589649157\\
0.1	1.84341694126769\\
0.11	1.84424636557472\\
0.12	1.84511416941268\\
0.13	1.84602035278155\\
0.14	1.84696491568133\\
0.15	1.84794785811204\\
0.16	1.84896918007366\\
0.169727455642948	1.85\\
0.17	1.85002875892685\\
0.18	1.85112217718587\\
0.19	1.8522538120054\\
0.2	1.85342366338545\\
0.21	1.854631731326\\
0.22	1.85587801582706\\
0.23	1.85716251688863\\
0.24	1.85848523451071\\
0.25	1.8598461686933\\
0.251099461994758	1.86\\
0.26	1.86124005381511\\
0.27	1.8626713434557\\
0.28	1.86414068806458\\
0.29	1.86564808764174\\
0.3	1.8671935421872\\
0.31	1.86877705170094\\
0.317541780253647	1.87\\
0.32	1.87039693779904\\
0.33	1.87204956937799\\
0.34	1.87374009569378\\
0.35	1.87546851674641\\
0.36	1.87723483253588\\
0.37	1.8790390430622\\
0.375216623376623	1.88\\
0.38	1.88087745378311\\
0.39	1.88274957118353\\
0.4	1.88465942443301\\
0.41	1.88660701353154\\
0.42	1.88859233847913\\
0.42695807819124	1.89\\
0.43	1.89061282975897\\
0.44	1.89266502182577\\
0.45	1.89475479218068\\
0.46	1.89688214082369\\
0.47	1.89904706775479\\
0.474326583369237	1.9\\
0.48	1.90124437724438\\
0.49	1.90347514647515\\
0.5	1.90574333774334\\
0.51	1.90804895104895\\
0.518327014600307	1.91\\
0.52	1.91039036325993\\
0.53	1.91276096367401\\
0.54	1.91516883116883\\
0.55	1.9176139657444\\
0.559611797710213	1.92\\
0.56	1.92009597000937\\
0.57	1.92260524835989\\
0.58	1.92515164011246\\
0.59	1.9277351452671\\
0.59864244331593	1.93\\
0.6	1.93035430278141\\
0.61	1.93300112002987\\
0.62	1.93568489826395\\
0.63	1.93840563748367\\
0.63578149326474	1.94\\
0.64	1.94115857966165\\
0.65	1.94394181074549\\
0.66	1.94676185164529\\
0.67	1.94961870236103\\
0.671317699967877	1.95\\
0.68	1.95250212923533\\
0.69	1.95542066284021\\
0.7	1.9583758563229\\
0.705428553747138	1.96\\
0.71	1.96136216116541\\
0.72	1.96437838834593\\
0.73	1.96743112668265\\
0.73831552557751	1.97\\
0.74	1.97051827364555\\
0.75	1.97363140495868\\
0.76	1.97678089990817\\
0.77	1.97996675849403\\
0.770103163294386	1.98\\
0.78	1.98317614779587\\
0.79	1.98642162063289\\
0.8	1.98970331077373\\
0.800894205854789	1.99\\
0.81	1.99300910912735\\
0.82	1.99634979049007\\
0.83	1.99972654399709\\
0.830801259808893	2\\
0.84	2.00312683723462\\
0.85	2.00656196697514\\
0.8599048565006	2.01\\
0.86	2.0100328935478\\
0.87	2.01352593529731\\
0.88	2.01705476233508\\
0.888262434720885	2.02\\
0.89	2.02061692169217\\
0.9	2.02420306030603\\
0.91	2.02782484248425\\
0.915947236304572	2.03\\
0.92	2.03147642101488\\
0.93	2.03515492200108\\
0.94	2.03886892594585\\
0.943016594137057	2.04\\
0.95	2.04260814431148\\
0.96	2.04637828183604\\
0.969517060121087	2.05\\
0.97	2.05018306351183\\
0.98	2.05400889521437\\
0.99	2.05786995196584\\
0.995466873658737	2.06\\
1	2.06175934786461\\
1.01	2.06567552720184\\
1.02	2.06962679425837\\
1.02093620804623	2.07\\
1.03	2.07359911738747\\
1.04	2.07760494263019\\
1.04592722030492	2.08\\
1.05	2.08163935291015\\
1.06	2.08569931422543\\
1.07	2.08979409178829\\
1.07049861613796	2.09\\
1.08	2.09390856542302\\
1.09	2.09805692765808\\
1.09464511536368	2.1\\
1.1	2.10223137323329\\
1.11	2.10643290874193\\
1.11842072743749	2.11\\
1.12	2.11066643490353\\
1.13	2.11492073700678\\
1.14	2.11920945593603\\
1.14182863576052	2.12\\
1.15	2.12351913419913\\
1.16	2.12786008658009\\
1.16489096449915	2.13\\
1.17	2.13222684146973\\
1.18	2.13661963084354\\
1.1876358960371	2.14\\
1.19	2.1410426190067\\
1.2	2.14548685341124\\
1.21	2.1499651142808\\
1.21007731271661	2.15\\
1.22	2.15446053757918\\
1.23	2.15898972778634\\
1.23221400967996	2.16\\
1.24	2.16353948490534\\
1.25	2.16811922224117\\
1.25407667566269	2.17\\
1.26	2.17272251486831\\
1.27	2.17735242141037\\
1.27567717252687	2.18\\
1.28	2.1820084645336\\
1.29	2.18668816658202\\
1.2970266872598	2.19\\
1.3	2.19139618822736\\
1.31	2.1961253162422\\
1.31813577929667	2.2\\
1.32	2.20088455721727\\
1.33	2.20566274575702\\
1.33901442391646	2.21\\
1.34	2.21047245940064\\
1.35	2.21529934706178\\
1.35967205208585	2.22\\
1.36	2.22015879899917\\
1.37	2.22503402835696\\
1.38	2.2299422852377\\
1.38011680113425	2.23\\
1.39	2.23486571381087\\
1.4	2.23982183812531\\
1.40035710716546	2.24\\
1.41	2.24479334326875\\
1.42	2.249796986256\\
1.42040309058031	2.25\\
1.43	2.25481587196832\\
1.44	2.25986668866524\\
1.44026224400376	2.26\\
1.45	2.26493227026138\\
1.45994130736238	2.27\\
1.46	2.27002981162981\\
1.47	2.27514152334152\\
1.4794446744149	2.28\\
1.48	2.28028464174963\\
1.49	2.28544263097764\\
1.49878054149241	2.29\\
1.5	2.29063067165393\\
1.51	2.29583460725321\\
1.51795508898345	2.3\\
1.52	2.30106692594393\\
1.53	2.30631648031113\\
1.53697419691345	2.31\\
1.54	2.3115924430809\\
1.55	2.31688729210399\\
1.55584346297251	2.32\\
1.56	2.32220627514079\\
1.57	2.32754609814964\\
1.57456821926015	2.33\\
1.58	2.33290748757415\\
1.59	2.33829196729197\\
1.59315354785234	2.34\\
1.6	2.34369515897108\\
1.61	2.34912398146669\\
1.6116042952863	2.35\\
1.62	2.35456838083108\\
1.62992465454545	2.36\\
1.63	2.36004109154371\\
1.64	2.36552625733778\\
1.64810963158954	2.37\\
1.65	2.37103920948617\\
1.66	2.37656790513834\\
1.666172832485	2.38\\
1.67	2.38212052938396\\
1.68	2.38769245312746\\
1.68411832522566	2.39\\
1.69	2.39328418904067\\
1.7	2.39889904223583\\
1.70194999860953	2.4\\
1.71	2.40452933813175\\
1.71966977348784	2.41\\
1.72	2.41018618431311\\
1.73	2.41585513800094\\
1.73727190654666	2.42\\
1.74	2.42154965034965\\
1.75	2.42726076146076\\
1.754770629787	2.43\\
1.76	2.43299241133653\\
1.77	2.43874539259718\\
1.77216923282903	2.44\\
1.78	2.44451365951536\\
1.78946807660549	2.45\\
1.79	2.45030718351023\\
1.8	2.45611259806184\\
1.80666121932576	2.46\\
1.81	2.46194189789974\\
1.82	2.46778844090143\\
1.82376314073301	2.47\\
1.83	2.47365301757066\\
1.84	2.47954041252865\\
1.84077663886809	2.48\\
1.85	2.48543977463073\\
1.85769272264893	2.49\\
1.86	2.49136318105934\\
1.87	2.49730141144331\\
1.8745215511761	2.5\\
67.4545363636364	411\\
-1.73237302895531	2.5\\
-1.73	2.49884200940962\\
-1.72	2.49399225982698\\
-1.71171678685014	2.49\\
-1.71	2.48916978833562\\
-1.7	2.48436409319324\\
-1.69086157966902	2.48\\
-1.69	2.4795871657754\\
-1.68	2.47482582123759\\
-1.67	2.47009472880061\\
-1.66979848539019	2.47\\
-1.66	2.46537835351832\\
-1.65	2.46069201287751\\
-1.64851371374008	2.46\\
-1.64	2.45602261190586\\
-1.63	2.45138132594985\\
-1.62700417014179	2.45\\
-1.62	2.44675953079179\\
-1.61	2.44216360549468\\
-1.60526082693803	2.44\\
-1.6	2.43759005730215\\
-1.59	2.43303980176553\\
-1.58327416646678	2.43\\
-1.58	2.42851515151515\\
-1.57	2.42401087801088\\
-1.56103414735818	2.42\\
-1.56	2.4195357866833\\
-1.55	2.41507781069702\\
-1.54	2.41065070949634\\
-1.53851984535168	2.41\\
-1.53	2.40624158973557\\
-1.52	2.40186121107808\\
-1.51572076123323	2.4\\
-1.51	2.39750321871565\\
-1.5	2.39316988538232\\
-1.49263202072917	2.39\\
-1.49	2.38886371514101\\
-1.48	2.38457775326926\\
-1.47	2.38032298723807\\
-1.46923527436864	2.38\\
-1.46	2.37608584980237\\
-1.45	2.37187889328063\\
-1.44550035969861	2.37\\
-1.44	2.36769522449627\\
-1.43	2.36353641123275\\
-1.42143186623102	2.36\\
-1.42	2.35940694156981\\
-1.41	2.35529660882025\\
-1.4	2.35121779971342\\
-1.39699107037489	2.35\\
-1.39	2.34716056878096\\
-1.38	2.3431306918038\\
-1.37216983016983	2.34\\
-1.37	2.33912938912939\\
-1.36	2.33514878948212\\
-1.35	2.3311999358666\\
-1.34693667880971	2.33\\
-1.34	2.32727320997586\\
-1.33	2.32337409493162\\
-1.3212752236322	2.32\\
-1.32	2.31950508638786\\
-1.31	2.31565606329727\\
-1.3	2.3118390117875\\
-1.29514141888145	2.31\\
-1.29	2.30804699400421\\
-1.28	2.30428050559067\\
-1.27	2.30054610273862\\
-1.26852497045564	2.3\\
-1.26	2.29683249308831\\
-1.25	2.29314912993983\\
-1.24137499443232	2.29\\
-1.24	2.28949616451771\\
-1.23	2.28586420760568\\
-1.22	2.28226456667211\\
-1.21365191929359	2.28\\
-1.21	2.27869254709255\\
-1.2	2.2751447993448\\
-1.19	2.27162948402948\\
-1.18532145040681	2.27\\
-1.18	2.26813989807661\\
-1.17	2.26467696860102\\
-1.16	2.26124658885418\\
-1.15633122066863	2.26\\
-1.15	2.25784095033823\\
-1.14	2.25446345487543\\
-1.13	2.25111862728923\\
-1.12662266500623	2.25\\
-1.12	2.24779847656897\\
-1.11	2.244507037589\\
-1.1	2.24124838549429\\
-1.09613007545814	2.24\\
-1.09	2.23801529001163\\
-1.08	2.23481053681236\\
-1.07	2.23163869037726\\
-1.06477947794779	2.23\\
-1.06	2.22849424520434\\
-1.05	2.22537681401168\\
-1.04	2.22229241034195\\
-1.0324872902203	2.22\\
-1.03	2.219238238741\\
-1.02	2.21620877281098\\
-1.01	2.21321245605224\\
-1	2.21024928846476\\
-0.999149191474773	2.21\\
-0.99	2.20730935977147\\
-0.98	2.20440178121324\\
-0.97	2.20152747437405\\
-0.964623528715917	2.2\\
-0.96	2.19868156518806\\
-0.95	2.19586338336988\\
-0.94	2.19307859672795\\
-0.93	2.19032720526227\\
-0.928796152652807	2.19\\
-0.92	2.18760030472321\\
-0.91	2.18490570509565\\
-0.9	2.18224462502116\\
-0.891457380323433	2.18\\
-0.89	2.17961563296517\\
-0.88	2.17701189464741\\
-0.87	2.17444180118946\\
-0.86	2.17190535259133\\
-0.852387127435671	2.17\\
-0.85	2.16940030701006\\
-0.84	2.16692188299505\\
-0.83	2.16447723008699\\
-0.82	2.16206634828586\\
-0.811307311473058	2.16\\
-0.81	2.1596880671118\\
-0.8	2.15733590138675\\
-0.79	2.15501763396679\\
-0.78	2.15273326485191\\
-0.77	2.15048279404212\\
-0.767821889240751	2.15\\
-0.76	2.14825966660938\\
-0.75	2.14606874033339\\
-0.74	2.14391184052243\\
-0.73	2.14178896717649\\
-0.72143562320033	2.14\\
-0.72	2.13969898223219\\
-0.71	2.13763636363636\\
-0.7	2.13560790063826\\
-0.69	2.13361359323788\\
-0.68	2.13165344143523\\
-0.671415136587549	2.13\\
-0.67	2.12972640692641\\
-0.66	2.12782735930736\\
-0.65	2.1259625974026\\
-0.64	2.12413212121212\\
-0.63	2.12233593073593\\
-0.62	2.12057402597403\\
-0.616677357923224	2.12\\
-0.61	2.11884199548062\\
-0.6	2.11714218668521\\
-0.59	2.1154767947158\\
-0.58	2.1138458195724\\
-0.57	2.112249261255\\
-0.56	2.1106871197636\\
-0.55550233246103	2.11\\
-0.55	2.10915616820799\\
-0.54	2.10765712790089\\
-0.53	2.10619263653813\\
-0.52	2.1047626941197\\
-0.51	2.10336730064561\\
-0.5	2.10200645611586\\
-0.49	2.10068016053045\\
-0.484734567067405	2.1\\
-0.48	2.09938605710282\\
-0.47	2.09812401471361\\
-0.46	2.09689665440533\\
-0.45	2.09570397617796\\
-0.44	2.09454598003153\\
-0.43	2.09342266596602\\
-0.42	2.09233403398143\\
-0.41	2.09128008407777\\
-0.4	2.09026081625504\\
-0.397351005159222	2.09\\
-0.39	2.08927343063126\\
-0.38	2.08831985229471\\
-0.37	2.08740109020573\\
-0.36	2.08651714436434\\
-0.35	2.08566801477053\\
-0.34	2.0848537014243\\
-0.33	2.08407420432566\\
-0.32	2.08332952347459\\
-0.31	2.08261965887111\\
-0.3	2.08194461051521\\
-0.29	2.08130437840689\\
-0.28	2.08069896254616\\
-0.27	2.08012836293301\\
-0.267604200853299	2.08\\
-0.26	2.07959099735216\\
-0.25	2.0790880847308\\
-0.24	2.07862012356575\\
-0.23	2.07818711385702\\
-0.22	2.07778905560459\\
-0.21	2.07742594880847\\
-0.2	2.07709779346867\\
-0.19	2.07680458958517\\
-0.18	2.07654633715799\\
-0.17	2.07632303618711\\
-0.16	2.07613468667255\\
-0.15	2.0759812886143\\
-0.14	2.07586284201236\\
-0.13	2.07577934686673\\
-0.12	2.07573080317741\\
-0.11	2.0757172109444\\
-0.1	2.0757385701677\\
-0.0900000000000001	2.07579488084731\\
-0.0800000000000001	2.07588614298323\\
-0.0700000000000001	2.07601235657546\\
-0.0600000000000001	2.07617352162401\\
-0.05	2.07636963812886\\
-0.04	2.07660070609003\\
-0.03	2.0768667255075\\
-0.02	2.07716769638129\\
-0.01	2.07750361871139\\
0	2.07787449249779\\
0.00999999999999979	2.07828031774051\\
0.02	2.07872109443954\\
0.0299999999999998	2.07919682259488\\
0.04	2.07970750220653\\
0.0453607246845686	2.08\\
0.0499999999999998	2.08025215403552\\
0.0600000000000001	2.08083049059258\\
0.0699999999999998	2.08144364339722\\
0.0800000000000001	2.08209161244945\\
0.0899999999999999	2.08277439774925\\
0.1	2.08349199929664\\
0.11	2.08424441709161\\
0.12	2.08503165113417\\
0.13	2.0858537014243\\
0.14	2.08671056796202\\
0.15	2.08760225074732\\
0.16	2.0885287497802\\
0.17	2.08949006506066\\
0.175119152691966	2.09\\
0.18	2.09048432299877\\
0.19	2.0915112979506\\
0.2	2.09257295498336\\
0.21	2.09366929409704\\
0.22	2.09480031529165\\
0.23	2.09596601856717\\
0.24	2.09716640392363\\
0.25	2.09840147136101\\
0.26	2.09967122087931\\
0.26252047804485	2.1\\
0.27	2.10097190717152\\
0.28	2.10230588030012\\
0.29	2.10367440237306\\
0.3	2.10507747339033\\
0.31	2.10651509335195\\
0.32	2.1079872622579\\
0.33	2.10949398010818\\
0.333283142760103	2.11\\
0.34	2.11103128802364\\
0.35	2.11260107769859\\
0.36	2.11420528419955\\
0.37	2.11584390752651\\
0.38	2.11751694767947\\
0.39	2.11922440465844\\
0.394452649436185	2.12\\
0.4	2.1209625974026\\
0.41	2.12273212121212\\
0.42	2.12453593073593\\
0.43	2.12637402597403\\
0.44	2.12824640692641\\
0.449197166469892	2.13\\
0.45	2.13015249266862\\
0.46	2.13208607900638\\
0.47	2.13405382094187\\
0.48	2.13605571847507\\
0.49	2.138091771606\\
0.4992175652029	2.14\\
0.5	2.14016136793263\\
0.51	2.14225777625021\\
0.52	2.14438821103282\\
0.53	2.14655267228046\\
0.54	2.14875115999313\\
0.545593872681087	2.15\\
0.55	2.15097996918336\\
0.56	2.15323797294984\\
0.57	2.1555298750214\\
0.58	2.15785567539805\\
0.589087281433651	2.16\\
0.59	2.16021456592188\\
0.6	2.1625991813065\\
0.61	2.16501756779806\\
0.62	2.16746972539655\\
0.63	2.169955654102\\
0.630175996750828	2.17\\
0.64	2.17246610025489\\
0.65	2.17501002548853\\
0.66	2.17758759558199\\
0.669238628229323	2.18\\
0.67	2.18019807008634\\
0.68	2.18283307939733\\
0.69	2.18550160826138\\
0.7	2.18820365667852\\
0.706566619221486	2.19\\
0.71	2.19093573958509\\
0.72	2.19369455220105\\
0.73	2.19648675999325\\
0.74	2.19931236296171\\
0.742405167836706	2.2\\
0.75	2.20216333389346\\
0.76	2.20504503444799\\
0.77	2.20796000672156\\
0.776919350242235	2.21\\
0.78	2.21090490540767\\
0.79	2.21387543947765\\
0.8	2.2168791227189\\
0.81	2.21991595513142\\
0.810273763429133	2.22\\
0.82	2.22297497914929\\
0.83	2.22606672226856\\
0.84	2.22919149291076\\
0.842560350747452	2.23\\
0.85	2.23234070134619\\
0.86	2.23551986039555\\
0.87	2.23873192620907\\
0.873907810499361	2.24\\
0.88	2.2419696969697\\
0.89	2.24523563503891\\
0.9	2.24853435999338\\
0.904399324022068	2.25\\
0.91	2.25185909915855\\
0.92	2.25521118627289\\
0.93	2.25859594126382\\
0.934108530874329	2.26\\
0.94	2.26200608252507\\
0.95	2.26544369554496\\
0.96	2.26891385829361\\
0.963100858872671	2.27\\
0.97	2.27240786240786\\
0.98	2.27593038493039\\
0.99	2.27948533988534\\
0.991434637687777	2.28\\
1	2.28306169414069\\
1.01	2.28666851640281\\
1.01915459478854	2.29\\
1.02	2.29030655391121\\
1.03	2.29396487233697\\
1.04	2.29765539112051\\
1.04629810842689	2.3\\
1.05	2.30137319721277\\
1.06	2.3051147301896\\
1.07	2.30888834872792\\
1.07292101341282	2.31\\
1.08	2.31268448248022\\
1.09	2.31650863878573\\
1.09905405971274	2.32\\
1.1	2.32036347546259\\
1.11	2.32423781174578\\
1.12	2.32814400643604\\
1.12471297242084	2.33\\
1.13	2.33207471540805\\
1.14	2.33603062369729\\
1.14995416348357	2.34\\
1.15	2.34001821377217\\
1.16	2.3440234861799\\
1.17	2.34806039303403\\
1.17476732770469	2.35\\
1.18	2.35212147747174\\
1.19	2.35620729183251\\
1.19921155407757	2.36\\
1.2	2.36032349674758\\
1.21	2.36445787720133\\
1.22	2.36862367126765\\
1.22327915327915	2.37\\
1.23	2.37281106719368\\
1.24	2.37702498023715\\
1.24700793266545	2.38\\
1.25	2.38126579486372\\
1.26	2.38552749330392\\
1.27	2.38982038758469\\
1.27041537620696	2.39\\
1.28	2.39413016172084\\
1.29	2.39847040351704\\
1.29349915592112	2.4\\
1.3	2.40283195118135\\
1.31	2.40721921452042\\
1.31629386974537	2.41\\
1.32	2.41163184157181\\
1.33	2.41606580383596\\
1.33881151119338	2.42\\
1.34	2.42052882672883\\
1.35	2.42500916860917\\
1.36	2.42952027972028\\
1.36105621514353	2.43\\
1.37	2.43404831965309\\
1.38	2.43860538949977\\
1.38303986767039	2.44\\
1.39	2.44318228121624\\
1.4	2.44778499768483\\
1.40478063892868	2.45\\
1.41	2.45241009075527\\
1.42	2.45705814490078\\
1.42628801578169	2.46\\
1.43	2.46173079871225\\
1.44	2.46642388471562\\
1.44757099737107	2.47\\
1.45	2.47114346829641\\
1.46	2.47588128342246\\
1.46863812606146	2.48\\
1.47	2.48064717527029\\
1.48	2.4854294198264\\
1.4894975160586	2.49\\
1.49	2.49024100774017\\
1.5	2.49506738503567\\
1.51	2.49992381241463\\
1.51015591514737	2.5\\
79.2828181818182	271\\
-1.25751247295686	2.5\\
-1.25	2.49741766580665\\
-1.24	2.49401032023069\\
-1.23	2.4906330247382\\
-1.22810881886194	2.49\\
-1.22	2.48727668646262\\
-1.21	2.48394837825491\\
-1.2	2.48065022080097\\
-1.19801034434556	2.48\\
-1.19	2.47737341482047\\
-1.18	2.47412467532468\\
-1.17	2.47090618792972\\
-1.16715771313557	2.47\\
-1.16	2.46771025601717\\
-1.15	2.46454162195309\\
-1.14	2.46140334202054\\
-1.13548463473586	2.46\\
-1.13	2.4582896477465\\
-1.12	2.45520166128288\\
-1.11	2.45214413167205\\
-1.10291681487881	2.45\\
-1.1	2.44911406081185\\
-1.09	2.44610726964038\\
-1.08	2.44313103874055\\
-1.07	2.44018536811236\\
-1.06936411288188	2.44\\
-1.06	2.43726095710082\\
-1.05	2.43436657890661\\
-1.04	2.43150286510763\\
-1.03469523861587	2.43\\
-1.03	2.42866526806527\\
-1.02	2.4258533022533\\
-1.01	2.42307210567211\\
-1	2.42032167832168\\
-0.998817210445118	2.42\\
-0.99	2.41759379385623\\
-0.98	2.41489568064868\\
-0.97	2.41222844222673\\
-0.961547288105519	2.41\\
-0.96	2.40959067438585\\
-0.95	2.40697621655453\\
-0.94	2.40439273979033\\
-0.93	2.40184024409326\\
-0.922701830592616	2.4\\
-0.92	2.39931637619721\\
-0.91	2.39681723975506\\
-0.9	2.39434919139582\\
-0.89	2.39191223111948\\
-0.882051817529205	2.39\\
-0.88	2.38950464786513\\
-0.87	2.38712163226721\\
-0.86	2.38476981250985\\
-0.85	2.38244918859304\\
-0.84	2.38015976051678\\
-0.839292541687016	2.38\\
-0.83	2.37789422924901\\
-0.82	2.37565944664032\\
-0.81	2.37345596837945\\
-0.8	2.3712837944664\\
-0.794003397090319	2.37\\
-0.79	2.36913993336506\\
-0.78	2.36702300491829\\
-0.77	2.36493749008409\\
-0.76	2.36288338886245\\
-0.75	2.36086070125337\\
-0.745677635248188	2.36\\
-0.74	2.35886546728228\\
-0.73	2.3568987422385\\
-0.72	2.35496354083745\\
-0.71	2.35305986307913\\
-0.7	2.35118770896354\\
-0.69354727099732	2.35\\
-0.69	2.34934478351174\\
-0.68	2.34752931778239\\
-0.67	2.34574548649944\\
-0.66	2.34399328966289\\
-0.65	2.34227272727273\\
-0.64	2.34058379932897\\
-0.63647739323243	2.34\\
-0.63	2.33892271925605\\
-0.62	2.33729132595799\\
-0.61	2.33569167869168\\
-0.6	2.33412377745711\\
-0.59	2.33258762225429\\
-0.58	2.33108321308321\\
-0.572644529123573	2.33\\
-0.57	2.32960917135961\\
-0.56	2.3281631536605\\
-0.55	2.32674899436846\\
-0.54	2.32536669348351\\
-0.53	2.32401625100563\\
-0.52	2.32269766693483\\
-0.51	2.32141094127112\\
-0.5	2.32015607401448\\
-0.498723852124722	2.32\\
-0.49	2.31892927498789\\
-0.48	2.31773389310512\\
-0.47	2.31657048280316\\
-0.46	2.31543904408203\\
-0.45	2.31433957694171\\
-0.44	2.31327208138221\\
-0.43	2.31223655740352\\
-0.42	2.31123300500565\\
-0.41	2.3102614241886\\
-0.407217735006018	2.31\\
-0.4	2.30931939718036\\
-0.39	2.30840852374007\\
-0.38	2.30752973586129\\
-0.37	2.306683033544\\
-0.36	2.3058684167882\\
-0.35	2.30508588559391\\
-0.34	2.30433543996111\\
-0.33	2.30361707988981\\
-0.32	2.30293080538\\
-0.31	2.3022766164317\\
-0.3	2.30165451304489\\
-0.29	2.30106449521958\\
-0.28	2.30050656295576\\
-0.270366718027734	2.3\\
-0.27	2.29998064725972\\
-0.26	2.29948511953163\\
-0.25	2.29902179216133\\
-0.24	2.2985906651488\\
-0.23	2.29819173849406\\
-0.22	2.29782501219711\\
-0.21	2.29749048625793\\
-0.2	2.29718816067653\\
-0.19	2.29691803545292\\
-0.18	2.29668011058709\\
-0.17	2.29647438607904\\
-0.16	2.29630086192877\\
-0.15	2.29615953813628\\
-0.14	2.29605041470158\\
-0.13	2.29597349162465\\
-0.12	2.29592876890551\\
-0.11	2.29591624654415\\
-0.1	2.29593592454058\\
-0.0900000000000001	2.29598780289478\\
-0.0800000000000001	2.29607188160677\\
-0.0700000000000001	2.29618816067653\\
-0.0600000000000001	2.29633664010408\\
-0.05	2.29651731988941\\
-0.04	2.29673020003253\\
-0.03	2.29697528053342\\
-0.02	2.2972525613921\\
-0.01	2.29756204260855\\
0	2.29790372418279\\
0.00999999999999979	2.29827760611482\\
0.02	2.29868368840462\\
0.0299999999999998	2.2991219710522\\
0.04	2.29959245405757\\
0.0481074086056294	2.3\\
0.0499999999999998	2.30009479824988\\
0.0600000000000001	2.30062777507697\\
0.0699999999999998	2.30119283746556\\
0.0800000000000001	2.30178998541565\\
0.0899999999999999	2.30241921892724\\
0.1	2.30308053800032\\
0.11	2.3037739426349\\
0.12	2.30449943283098\\
0.13	2.30525700858856\\
0.14	2.30604666990763\\
0.15	2.3068684167882\\
0.16	2.30772224923027\\
0.17	2.30860816723384\\
0.18	2.3095261707989\\
0.18498720791404	2.31\\
0.19	2.31047456806071\\
0.2	2.3114532536735\\
0.21	2.31246391086711\\
0.22	2.31350653964153\\
0.23	2.31458113999677\\
0.24	2.31568771193283\\
0.25	2.3168262554497\\
0.26	2.31799677054739\\
0.27	2.3191992572259\\
0.276486592544147	2.32\\
0.28	2.32043218020917\\
0.29	2.32169412711183\\
0.3	2.32298793242156\\
0.31	2.32431359613837\\
0.32	2.32567111826227\\
0.33	2.32706049879324\\
0.34	2.3284817377313\\
0.35	2.32993483507643\\
0.350438834109873	2.33\\
0.36	2.33141478274812\\
0.37	2.33292624659291\\
0.38	2.33446945646946\\
0.39	2.33604441237775\\
0.4	2.33765111431778\\
0.41	2.33928956228956\\
0.414253623884035	2.34\\
0.42	2.34095638280876\\
0.43	2.34265234062949\\
0.44	2.34437993289663\\
0.45	2.34613915961016\\
0.46	2.34793002077009\\
0.47	2.34975251637642\\
0.471334769495909	2.35\\
0.48	2.35160101894603\\
0.49	2.35348017831555\\
0.5	2.35539086132781\\
0.51	2.3573330679828\\
0.52	2.35930679828053\\
0.523456927352124	2.36\\
0.53	2.36130747263208\\
0.54	2.36333714104395\\
0.55	2.36539822306838\\
0.56	2.36749071870538\\
0.57	2.36961462795494\\
0.571788001472213	2.37\\
0.58	2.3717637944664\\
0.59	2.37394292490119\\
0.6	2.37615335968379\\
0.61	2.37839509881423\\
0.617060582875426	2.38\\
0.62	2.38066582637467\\
0.63	2.38296218685993\\
0.64	2.38528974318576\\
0.65	2.38764849535213\\
0.659839145625949	2.39\\
0.66	2.39003831056681\\
0.67	2.39245109122311\\
0.68	2.39489495996232\\
0.69	2.39736991678442\\
0.7	2.39987596168943\\
0.700488891639334	2.4\\
0.71	2.4024047879831\\
0.72	2.40496416836176\\
0.73	2.40755452980754\\
0.739329075389483	2.41\\
0.74	2.41017526898487\\
0.75	2.4128184936847\\
0.76	2.41549259317012\\
0.77	2.41819756744114\\
0.776588201766886	2.42\\
0.78	2.42093022533022\\
0.79	2.42368749028749\\
0.8	2.42647552447552\\
0.81	2.42929432789433\\
0.812476413808148	2.43\\
0.82	2.43213659594239\\
0.83	2.43500712405142\\
0.84	2.43790831655568\\
0.847134329935028	2.44\\
0.85	2.44083732057416\\
0.86	2.44378978237382\\
0.87	2.44677280444513\\
0.88	2.44978638678808\\
0.880701718805457	2.45\\
0.89	2.45282095062298\\
0.9	2.45588524842332\\
0.91	2.45898000307645\\
0.913263769257273	2.46\\
0.92	2.46209811436456\\
0.93	2.46524313965967\\
0.94	2.46841851908631\\
0.944933288699729	2.47\\
0.95	2.47161879297173\\
0.96	2.47484400305577\\
0.97	2.47809946524064\\
0.975784236224134	2.48\\
0.98	2.48138053905893\\
0.99	2.48468539668037\\
1	2.48802040505558\\
1.00588261912304	2.49\\
1.01	2.49138093792685\\
1.02	2.49476491121566\\
1.03	2.49817893458795\\
1.03528753360067	2.5\\
};
\addplot[only marks, mark=o, mark options={}, mark size=1.1180pt, draw=mycolor1] table[row sep=crcr]{%
x	y\\
0.100675514368443	0.327186362745972\\
};

\addplot[only marks, mark=o, mark options={}, mark size=1.5811pt, draw=mycolor2] table[row sep=crcr]{%
x	y\\
0.329422835070244	0.273094573509283\\
};

\addplot[only marks, mark=o, mark options={}, mark size=1.9365pt, draw=mycolor3] table[row sep=crcr]{%
x	y\\
0.549835631909033	0.259831597907541\\
};

\addplot[only marks, mark=o, mark options={}, mark size=2.2361pt, draw=mycolor4] table[row sep=crcr]{%
x	y\\
0.761285167848238	0.253859098176738\\
};

\addplot[only marks, mark=o, mark options={}, mark size=2.5000pt, draw=mycolor5] table[row sep=crcr]{%
x	y\\
0.968371263986741	0.250407606680042\\
};

\addplot[only marks, mark=o, mark options={}, mark size=2.7386pt, draw=mycolor6] table[row sep=crcr]{%
x	y\\
1.17327616009909	0.248140194892022\\
};

\addplot[only marks, mark=o, mark options={}, mark size=2.9580pt, draw=mycolor7] table[row sep=crcr]{%
x	y\\
1.37686448017189	0.246531124987353\\
};

\addplot[only marks, mark=o, mark options={}, mark size=3.1623pt, draw=mycolor1] table[row sep=crcr]{%
x	y\\
1.57960195587074	0.245327665213382\\
};

\addplot[only marks, mark=o, mark options={}, mark size=3.3541pt, draw=mycolor2] table[row sep=crcr]{%
x	y\\
1.78232531639941	0.244390192930824\\
};

\addplot[only marks, mark=o, mark options={}, mark size=3.5355pt, draw=mycolor3] table[row sep=crcr]{%
x	y\\
1.98350555057352	0.243644405222063\\
};

\end{axis}

\end{tikzpicture}%
            \caption{The contour lines of the quadratic model $m$ at point $x = (0, 0.5)^T$ and the family of solutions of trust region subproblem with $\Delta=0.2, 0.4, \dotsc, 2$, where a larger circle symbolizes a greater $\Delta$.}
            \label{fig:2}
        \end{figure}
        We have used Newton iteration method to solve the trust region subproblem. To make the algorithm pracicable, we select proper $\lambda_0$ to ensure that $\B +\lambda_0 \matr{I}$ is a diagonally dominant matrix.
        Please refer to \ref{fig:1} and \ref{fig:2} for results.
    \end{solution}

    \begin{problem}
        When $\B$ is positive definite, the {\itshape double-dogleg method} constructs a path with three line segments from the origin to the full step. The four points that define the path are
        \begin{itemize}
            \item the origin;
            \item the unconstrained Cauchy step $\pC=-\left.\left(g^Tg\right)\middle/\left(g^T\B g\right)\right.g$;
            \item a fraction of the full step $\bar{\gamma}\pB=-\bar{\gamma}\Bi g$, for some $\bar{\gamma}\in(\gamma, 1]$, where
            \begin{equation}
                \gamma \doteq \frac{\norm{g}^4}{\left(g^T\B g\right)\left(g^T\Bi g\right)}\leq -1;
            \end{equation}
            and
            \item the full step $\pB = -\Bi g$.
        \end{itemize}
        Show that $\norm{p}$ increases monotonically along this path.
    \end{problem}
    \begin{proof}
        Since the first, the third and the fourth points lay on the same line, it suffices to prove that $\norm{p}$ increases monotonically between the second point and the third point.

        Let $\alpha\in[0,1]$, and define $d(\alpha)$ by
        \begin{equation}
            d(\alpha)\doteq\frac{1}{2}\norm{\pC+\alpha\left(\bar{\gamma}\pB-\pC\right)}^2=\frac{1}{2}\norm{\pC}^2+\alpha\left(\pC\right)^T\left(\bar{\gamma}\pB-\pC\right)+\frac{1}{2}\alpha^2\norm{\bar{\gamma}\pB-\pC}^2.
        \end{equation}
        Hence the derivative
        \begin{equation}
            d'(\alpha)=\left(\pC\right)^T\left(\bar{\gamma}\pB-\pC\right)+\alpha\norm{\bar{\gamma}\pB-\pC}^2=\left((1-\alpha)\pC+\alpha\bar{\gamma}\pB\right)^T\left(\bar{\gamma}\pB-\pC\right)
        \end{equation}
        is a convex combination of $d'(0)$ and $d'(1)$. Thus to show $d'>0$, it suffices to show that $d'(0)>0$ and $d'(1)>0$ respectively.

        When $\alpha=0$, it is easy to check
        \begin{equation}
            \begin{aligned}
                d'(0)&= \left(\pC\right)^T\left(\bar{\gamma}\pB-\pC\right)\\
                &= \bar{\gamma}\frac{g^Tg}{g^T\B g}g^T \Bi g-\left(\frac{g^T g}{g^T\B g}\right)^2 g^T g>0,
            \end{aligned}
        \end{equation}
        since $\bar{\gamma}>\gamma$.

        When $\alpha=1$, we have
        \begin{equation}
            \begin{aligned}
                d'(1)&= \left(\bar{\gamma}\pB\right)^T\left(\bar{\gamma}\pB-\pC\right)
                = \bar{\gamma}\left(\bar{\gamma}g^T\Bii g- g^T\Bi g\frac{g^Tg}{g^T\B g}\right)\\
                &> \frac{\gamma}{\left(g^T\B g\right)\left(g^T\Bi g\right)}\left(g^Tgg^T\Bii g-\left(g^T\Bi g\right)^2\right)\geq 0,
            \end{aligned}
        \end{equation}
        by Cauchy-Schwarz inequality.
    \end{proof}

    \begin{problem}
        Show that the two formula
        \begin{equation}
            \lambda^{(l+1)}=\lambda^{(l)}-\frac{\varphi\left(\lambda^{(l)}\right)}{\varphi'\left(\lambda^{(l)}\right)}
            \quad\text{and}\quad
            \lambda^{(l+1)}=\lambda^{(l)}+\left(\frac{\norm{p_l}}{\norm{q_l}}\right)^2\left(\frac{\norm{p_l}-\Delta}{\Delta}\right)
        \end{equation}
        are equivalent.
    \end{problem}
    \begin{proof}
        Since this topic has been discussed in the course, we will only prove the derivative of inverse matrix.

        By taking derivative at the both side of the equation
        \begin{equation}
            \B(\lambda)\Bi(\lambda)=\matr{I},
        \end{equation}
        we have
        \begin{equation}
            \B'(\lambda)\Bi(\lambda)+\B(\lambda)\left(\Bi(\lambda)\right)'=0.
        \end{equation}
        It follows that
        \begin{equation}
            \left(\Bi(\lambda)\right)'=-\Bi(\lambda)\B'(\lambda)\Bi(\lambda),
        \end{equation}
        as needed.
    \end{proof}
    \begin{problem}
        The following example shows that the reduction in the model function $m$ achieved by the two-dimensional minimization strategy can be much smaller than achieved by the exact solution of \ref{eq:TrsModelFunc}
        \begin{equation}\label{eq:TrsModelFunc}
            \min_{p\in\BR^n}m(p)\doteq f+g^Tp+\frac{1}{2}p^T\B p\quad\text{s.t. }\norm{p}\leq\Delta.
        \end{equation}

        In \ref{eq:TrsModelFunc}, set
        \begin{equation}
            g = \left(-\frac{1}{\varepsilon}, -1, -\varepsilon^2\right),
        \end{equation}
        where $\varepsilon$ is a small positive number. Set
        \begin{equation}
            \B = \diag\left(\frac{1}{\varepsilon^3}, 1, \varepsilon^3\right),\quad \Delta=0.5.
        \end{equation}
        Show that the solution of \ref{eq:TrsModelFunc} has components $\left(O(\varepsilon), 1/2+O(\varepsilon), O(\varepsilon)\right)^T$ and that the reduction in the model $m$ is $3/8 + O(\varepsilon)$. For the two-dimensional minimization strategy, show that the solution is a multiple of $\Bi g$ and that the reduction in $m$ is $O(\varepsilon)$.
    \end{problem}
    \begin{proof}
        Theorem of global minimizer yields the equations
        \begin{equation}
            d(\lambda)=-(\B+\lambda\matr{I})^{-1}g=\left(\frac{\varepsilon^2}{\lambda\varepsilon^3+1},\frac{1}{1+\lambda},\frac{\varepsilon}{\varepsilon^3+\lambda}\right)^T
            \quad\text{and}\quad
            \norm{d(\lambda)}=0.5.
        \end{equation}
        Since $\lambda>1$ shown by second component, then it is easy to show that
        \begin{equation}
            \lambda=1+O(\varepsilon)
            \quad\text{and}\quad
            d(\lambda)=\left(O(\varepsilon), \frac{1}{2}+O(\varepsilon), O(\varepsilon)\right).
        \end{equation}
        Then, it naturally follows the reduction in the model $m$
        \begin{equation}
            \begin{aligned}
                &\phantom{{}={}}m(0)-m(p)\\
                &=\left(\frac{\varepsilon}{\lambda\varepsilon^3+1}+\frac{1}{1+\lambda}+\frac{\varepsilon^3}{\lambda\varepsilon^3+1}\right)-\frac{1}{2}\left(\frac{\varepsilon}{\left(\lambda\varepsilon^3+1\right)^2}+\frac{1}{\left(1+\lambda\right)^2}+\frac{\varepsilon^5}{\left(\lambda\varepsilon^3+1\right)^2}\right)\\
                &=\frac{3}{8}+O(\varepsilon).
            \end{aligned}
        \end{equation}

        For the two-dimensional minimization strategy, I have no idea how to solve that. If we only search along the line parallel to $-\Bi g=\left(\varepsilon^2, 1, 1/\varepsilon\right)^T$, then the minimizer must be $O(\varepsilon)\Bi g$. Hence the reduction is $O(\varepsilon)$.

        As for the reason why the solution must be a multiple of $\Bi g$, my thought is as follows. Let $\alpha = (\varepsilon^3, \varepsilon, 1)$ and $\beta = (\varepsilon, 1, \varepsilon^2)^T$, where $\alpha$ and $\beta$ have similar length and are parallel to $-\Bi g$ and $-g$ respectively. Without loss of generality let's assume that the minimizer lies at the boundary of the trust region, thus the component on $\alpha$ and the one on $\beta$ are closedly related. It is easy to find out that every unit length on direction $a$ provides a reduction of $O(\varepsilon)$, while every unit length on direction $b$ actually provides a increase of $O(1/\varepsilon)$. Hence the less component on $-g$ we choose, the better reduction it will make. It might be the reason why the solution is a multiple of $\Bi g$.
    \end{proof}

    \clearpage\appendix
    \section{Codes}

    \matlabinputlisting[caption={Newton Iteration Method for a Trust Region Subproblem}]{trs_iteration.m}
    \matlabinputlisting[caption={Contour Lines of a Quadratic Model}]{rosenbrock_quadratic_contour.m}
    \matlabinputlisting[caption={Main Script of \ref{prob:1}}]{main.m}
\end{document}

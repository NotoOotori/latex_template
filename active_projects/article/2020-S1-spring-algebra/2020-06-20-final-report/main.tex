% -------------------- Packages --------------------

\documentclass{assignment}[2019/10/15]
\usepackage[lineno]{packages}[2019/11/14]
\usepackage{ytableau}

% -------------------- Settings --------------------

% Title

\title{Representation Theory of Symmetric Group}
\author{Chen Xuyang, Chen Xuyang}
\date{\today}
\institute{School of Mathematical Science}
\professor{Song linliang}
\course{Algebra}
\subject{Algebra}
\keywords{}

% -------------------- New commands --------------------

\newcommand{\me}{\symup{e}}
\newcommand{\BR}{\symbb{R}}
\newcommand{\BZ}{\symbb{Z}}
\newcommand{\BN}{\symbb{N}}
\newcommand{\diag}{\mathop{}\!\symup{diag}}
\newcommand{\pr}{\mathop{}\!\symup{Pr}}
\newcommand{\expect}{\mathop{}\!\symup{E}}
\newcommand{\cov}{\mathop{}\!\symup{Cov}}
\newcommand{\var}{\mathop{}\!\symup{Var}}

\def\multiset#1#2{\ensuremath{\left(\kern-.3em\left(\genfrac{}{}{0pt}{}{#1}{#2}\right)\kern-.3em\right)}}

\newcommand{\lr}[3]{\left#1#3\right#2}
\newcommand{\lmr}[5]{\left#1#4\middle#2#5\right#3}

\theoremstyle{plain}
\theoremheaderfont{\upshape\bfseries}
\theorembodyfont{\upshape}
\theoremseparator{}
\theoremsymbol{}
\newtheorem{theorem}{Theorem}[section]
\newtheorem{definition}[theorem]{Definition}
\newtheorem{proposition}[theorem]{Proposition}
\newtheorem{lemma}[theorem]{Lemma}
\newtheorem{corollary}[theorem]{Corollary}
\newtheorem{example}[theorem]{Example}

\newcommand{\SC}{\mathscr{C}}
\newcommand{\BC}{\symbb{C}}

\newcommand{\Kernal}{\mathop{}\!\symup{Ker}}
\newcommand{\Image}{\mathop{}\!\symup{Im}}
\newcommand{\Hom}{\mathop{}\!\symup{Hom}}
\newcommand{\Aut}{\mathop{}\!\symup{Aut}}
\newcommand{\Type}{\mathop{}\!\symup{Type}}
\newcommand{\GL}{\mathop{}\!\symup{GL}}
\newcommand{\sgn}{\mathop{}\!\symup{sgn}}
\newcommand{\id}{\text{id.}}
\newcommand{\blank}{{\cdot}}
\newcommand{\tr}{\mathop{}\!\symup{tr}}
\newcommand{\Cl}{\mathop{}\!\symup{Cl}}

\numberwithin{equation}{section}

% -------------------- Document --------------------

\begin{document}
    \maketitle
    \tableofcontents
    \clearpage

    \setcounter{section}{-1}

    \section{Introduction}

    The main topic of this report is the symmetric group. We will discuss about the conjugacy classes, the class equation, and the irreducible representations of the symmetric group.

    To unify the definition of group actions on sets and group representations, and the definition of morphisms and equivalences (of group representations), we introduce in section one the definition of categories, though we are not using the language of categories to develop the theories. We also give some examples of category that we will encounter in the sequel.

    In section two, we introduce the definition of a group action as the a map from a group to the automorphism group of a category with single object. The concept of orbits and stablizers is referred while the class equation is established for finite groups, which is rather trivial.

    In section three, we focus on conjugacy classes of the symmetric group. Inspired by an exercise, we find an explicit expression for the conjugacy of a permutation. For a permutation $\sigma\in S_n$, by researching carefully on the orbits of the group action of the subgroup $\lr<>\sigma$ generated by $\sigma$ on the set of $n$-letters, we describe the conjugacy classes and compute their cardinals.

    In section four, we start to focus only on the representations of finite groups, and are devoted to decompose any representation of a finite group into irreducible ones. To achieve this, we introduce a technique named ``averaging trick". By averaging an existance inner product, we construct an inner product such that the representation preserves it, namely any image of the representation is a unitary linear map. Then by routine compution, we prove that every representation is either decomposable or irreducible, which leads to what we want.

    In section five, we establish the uniqueness of decomposition and the upper bound of inequivalent irreducible representations, through character theory. First we study the morphisms between representations, of which the collection forms a linear subspace of all linear maps between linear spaces. Again by the averaging trick, we discover a linear function which maps a morphism between linear spaces to a morphism between representations. By the Schur's observation on the dimension of the spaces between representations, we establish so-called Schur orthogonality relation on the entries of two group representations. Next we introduce the concept of a character, as the trace of a representation. It turns out that the characters of inequivalent irreducible representations to some extent are linear independent and span a subspace of a linear space with finite dimension, thanks to the first orthogonality relations established by the previous orthogonality relation.

    In section six, we start to construct the irreducible representations of the symmetric group, where we need to associate each partition with an irreducible representation and to prove that any two of the representations are not equivalent. First we develop the concept of a Young tableau, and assign each partition with a permutation representation. Though these representations are in general not irreducible, we for the third time construct a linear operator with the averaging trick, by which we find a invariant space which does not have a nontrivial invariant space. Finally, by the dominance lemma of tableaux and some tricks on the morphisms, we prove that the representations are irreducible and not equivalent.

    In section seven, we try to link the dimension of irreducible representations of symmetric group to the number of standard tableaux, which has a famous combinatorial formula to compute. We shall prove that the set of all polytabloids related to the standard tableaux forms a basis of the space of polytabloids. To achieve this, we have introduced several orders, either partial ones or linear ones. By a linear order we show that the polytabloids associated with the standard tableaux are linear independent. Then by the Garnir element we prove that any polytabloid can be written as a linear combination of polytabloids that are closer to the one corresponding to the standard tableaux, under a partial order named the dominance order, where another dominance lemma comes to help.

    \section{Categories}

    In this section, we will introduce the definition of a category. Category is an abstract mathematical concept, which focuses on not only mathematical objects, but also morphisms between them.

    \begin{definition}[Category]
        A \emph{category} is a class $\mathscr{C}$ of objects (denoted by $A, B, C, \dotsc$) together with
        \begin{enumerate}[(i)]
            \item a class of disjoint sets, denoted by $\Hom(A, B)$, one for each pair of objects in $\mathscr{C}$; (an element of $\Hom(A, B)$ is called a \emph{morphism} from $A$ to $B$ and is denoted by $f\colon A\to B$);
            \item for each triple $(A, B, C)$ of objects of $\mathscr{C}$ a function
            \begin{equation}
                \Hom(B, C)\times \Hom(A, B)\to \Hom(A, C);
            \end{equation}
            (for morphisms $f\colon A\to B, g\colon B\to C$, this function is written $(g, f)\mapsto g\circ f$ and $g\circ f\colon A\to C$ is called the \emph{composite} of $f$ and $g$); all subject to the two axioms:
            \begin{enumerate}[\hspace{-1em}(I)]
                \item Associativity. If $f\colon A\to B, g\colon B\to C, h\colon C\to D$ are morphisms of $\SC$, then $h\circ (g\circ f) = (h\circ g)\circ f$.
                \item Identity. For each object $B$ of $\SC$ there exists a morphism $1_B\colon B\to B$ such that for any $f\colon A\to B, g\colon B\to C$,
                \begin{equation}
                    1_B \circ f = f\quad\text{and}\quad g\circ 1_B = g.
                \end{equation}
            \end{enumerate}
        \end{enumerate}

        In a category $\SC$ a morphism $f\colon A\to B$ is called an \emph{equivalence} if there is in $\SC$ a morphism $g\colon B\to A$ such that $g\circ f = 1_A$ and $f\circ g = 1_B$.
    \end{definition}

    Note that in the definition of a category, an object is not necessarily a set and a morphism is not necessarily a map between sets. We may exclude these cases and consider all categories to be concrete in the sequel.

    \begin{definition}[Concrete Category]
        A category $\SC$ is called a \emph{concrete category} provided that
        \begin{enumerate}[(i)]
            \item every object is a set;
            \item every morphism of $\Hom(A, B)$ is a map on the sets $A\to B$ for each pair of objects $(A, B)$;
            \item composition of morphisms agrees with composition of maps on the sets;
            \item the identity morphism of each object $A$ is the identity map on the set $A$.
        \end{enumerate}
    \end{definition}

    \begin{example}[Category of Finite Dimensional Linear Spaces]
        Let $\mathscr{V}_\text{\it finite}$ denote the class of all finite dimensional linear spaces over $\BC$. Let $\Hom(V, W)$ be the set of all linear maps from $V$ to $W$. Then $\mathscr{V}_\text{\it finite}$ is a concrete category and is called the \emph{category of finite dimensional linear spaces}.
    \end{example}

    Sometimes we may be interested in those concrete categories each of which contains exactly one object.

    \begin{definition}[Structure]
        A concrete category is called a \emph{structure} provided that it contains exactly one object $C$. We usually denote a structure by the same symbol of its letter but with a different font $\mathcal{C}$. The set of all equivalences in $\mathcal{C}$ forms a group with the composite operation which is called the \emph{automorphism group} of the structure and is denoted by $\Aut(\mathcal{C})$.
    \end{definition}

    \begin{example}[Set Structure]
        Let $\mathcal{X}$ be the set of a given set $X$. Let $\Hom(X, X)$ be the set of all maps on $X$. Then $\mathcal{X}$ is easily seen to be a structure, which is called a \emph{set structure}.
    \end{example}

    \begin{example}[Group Structure]
        Let $\mathcal{G}$ be the set of a given group $G$. Let $\Hom(G, G)$ be the set of all endomorphisms of groups $G\to G$. Then $\mathcal{G}$ is easily seen to be a structure, which is called a \emph{group structure}.
    \end{example}

    \begin{example}[Linear Space Structure]
        Let $\mathcal{V}\subseteq \mathscr{V}_{\text{\it finite}}$ be the set of a given finite dimensional linear space $V$ over $\BC$. Let $\Hom(V, V)$ be the set of all morphisms $V\to V$ in $\mathscr{V}_{\text{\it finite}}$. Then $\mathcal{V}$ is easily seen to be a structure, which is called a \emph{group structure}. An equivalence in $\mathcal{V}$ is an linear transformation on $V$. Thus the automorphism group of a linear space structure $\mathcal{V}$ is also called a \emph{general linear group} of $V$ and is denoted by $\GL(V)$.
    \end{example}

    \section{Group Actions}

    In this section, we will introduce the definition of group actions and derive the class equation of any finite group.

    \begin{definition}[Action]
        An \emph{action} of a group $G$ on a structure $\mathcal{C}$ is a group homomorphism $\phi\colon G\to \Aut(\mathcal{C})$. We usually write $\phi_g$ for $\phi(g)$ and $\phi_g(x)$ for $\phi(g)(x)$ where $x\in C\in \mathcal{C}$.
    \end{definition}

    In particular, an action of the group $G$ on a set $X$ is a group homomorphism from $G$ to the group $S_X$ of all bijections on $X$.

    \begin{example}[Conjugation]
        Let $\mathcal{G}$ be a group structure. The action $\phi\colon G\to \Aut(\mathcal{G})$ given by $\phi_g(x) = gxg^{-1}$ is called \emph{conjugation} and the element $gxg^{-1}$ is said to be a \emph{conjugate} of $x$.
    \end{example}

    A group action on a structure $\mathcal{C}$ can derive an equivalence class on $C$, which is essential in counting theory.

    \begin{theorem}
        Let $G$ be a group that acts on a structure $\mathcal{C}$. The relation on $C$ defined by
        \begin{equation}
            x\sim x' \Longleftrightarrow \phi_g(x) = x'\text{ for some }g\in G
        \end{equation}
        is an equivalence relation.
    \end{theorem}

    The equivalence classes of the equivalence relation are called the \emph{orbits} of $G$ on $\mathcal{C}$; the orbit of $x\in C$ is denoted by $\overline{c}$.

    \begin{proof}
        It follows from the fact that $\phi$ is a homomorphism from group $G$. (transitivity follows from associativity, reflexivity follows from the existance of an identity element, and symmetry follows from the existance of an inverse element).
    \end{proof}

    \begin{theorem}
        Let $G$ be a group that acts on a structure $\mathcal{C}$. For each $x\in C$, $G_x=\{g\in G|\ \phi_g(x) = x\}$ is a subgroup of $G$.
    \end{theorem}

    The subgroup $G_x$ is called the \emph{stablizer} of $x$.

    \begin{example}
        Let a group $G$ acts on the group structure $\mathcal{G}$ by conjugation. The orbit $\{gxg^{-1}|\ g\in G\}$ of $x\in G$ is called the \emph{conjugacy class} of $x$ and the stablizer $G_x=\{g\in G|\ gxg^{-1} = x\}$ is called the \emph{centralizer} of $x$ and is denoted by $C_G(x)$.
    \end{example}

    The next theorem counts the cardinal number of any orbit, which will directly induce the group equation of a finite group.

    \begin{theorem}
        If a group $G$ acts on a structure $\mathcal{C}$, then the cardinal number of the orbit of $x\in C$ is the index $[G:G_x]$.
    \end{theorem}

    \begin{proof}
        The map given by $gG_x\mapsto \phi_g(x)$ from the set of all cosets of $G_x$ in $G$ to the orbit $\overline{x}$ is a well-defined bijection since $x$ runs through $G$ and
        \begin{equation}
            \phi_g(x)=\phi_{g'}(x)\Longleftrightarrow \phi_{g^{-1}g'}(x)=x\Longleftrightarrow g^{-1}g'\in G_x\Longleftrightarrow gG_x = g'G_x.
        \end{equation}
        It completes the proof.
    \end{proof}

    Apply the previous theorem to the conjugacy action of a finite group $G$ on $\mathcal{G}$ and notice that the cardinal of a finite set equals to the sum of the cardinals of all equivalence classes, we can derive the class equation, as desired.

    \begin{corollary}[Class Equation]
        If $G$ is a finite group and $\{\overline{x_1}, \dotsc, \overline{x_n}\}$ is the set of all conjugacy classes of $G$, then the equation
        \begin{equation}
            |G| = \sum_{i=1}^n[G: C_G(x_i)]
        \end{equation}
        is called the \emph{class equation} of the finite group $G$.
    \end{corollary}

    Given a group $G$ and a category $\mathscr{C}$. The class of all actions of $G$ on a arbitrary structure $\mathcal{C}$ of the category $\mathscr{C}$ can form a category by defining the morphism between actions.

    \begin{definition}[Category of Group Actions]
        Let $G$ be a group and $\mathscr{C}$ be a concrete category with $\mathcal{C}, \mathcal{C}'\subseteq \mathscr{C}$ to be two structures. Let $\phi\colon G\to \Aut(\mathcal{C})$ and $\phi'\colon G\to\Aut(\mathcal{C}')$ be group actions. A morphism from $\phi$ to $\phi'$ is by definition a morphism $f\colon C\to C'$ in $\mathscr{C}$ such that $f\phi_g = \phi'_gf$ for all $g\in G$. Then the class of all actions of $G$ on any structure $\mathcal{C}$ of $\mathscr{C}$ is a category and is denoted by $\mathscr{C}_G$. The set of all morphisms from $\phi$ to $\phi'$ is denoted by $\Hom_G(\phi, \phi')$. Notice that $\Hom_{G}(\phi, \phi')\subseteq \Hom(V, W)$.
    \end{definition}

    \section{Symmetric Groups: Conjugacy Classes and Class Equations}

    In this section, we will study the conjugacy classes of the symmetric group $S_n$, in order to give an explicit form of the class equation on $S_n$ $(n\geq 3)$.

    We will assume $n\geq 3$ when talking about the symmetric group $S_n$.

    Recall that an element $\sigma$ of the symmetric group $S_n$ is a bijection on $I_n = \{1, 2, \dotsc, n\}$ and is called a \emph{permutaion}.
    A \emph{cycle} denoted by $(i_1 i_2 \dotsb i_k)$ is a permutation which maps $i_1\mapsto i_2, \dotsc, i_{k-1}\mapsto i_k$ and $i_k\mapsto i_1$ and maps every other element to itself.

    It is widely known that every permutation $\tau$ can be uniquely (up to the order of the factors) written as a product of disjoint cycles, where each cycle corresponds to an orbit of the group action on the set $I_n$ by the group $\lr<>\tau < S_n$ with the length of the cycle to be equal to the cardinal of the orbit. Since the sum of the cardinals of all orbits equals to $n$, thus $\tau$ corresponds to a partition of $n$, which is called the \emph{cycle type} of the permutation $\tau$ and denoted by $\Type(\tau)$. Namely, $\Type(\tau)=(\lambda_1, \dotsc, \lambda_r)$ in decreasing order with multiplicity.

    The next lemma is one of the exercises in the Algebra course and is proved to be useful to identify the conjugacy classes of $S_n$.

    \begin{lemma}
        If $\sigma = (i_1 i_2 \dotsb i_r)\in S_n$ and $\tau\in S_n$, then $\tau\sigma\tau^{-1}$ is the $r$-cycle $\left(\tau(i_1) \tau(i_2) \dotsb \tau(i_r)\right)$.
    \end{lemma}

    \begin{theorem}
        Let $\sigma, \sigma'\in S_n$ be two permutations. If there is some $\tau\in S_n$ such that $\sigma'=\tau\sigma\tau^{-1}$, then $\Type(\sigma)=\Type(\tau\sigma\tau^{-1}) = \Type(\sigma')$. Conversely, if $\Type(\sigma) = \Type(\sigma')$, then there exists such permutation $\tau$ that $\sigma' = \tau\sigma\tau^{-1}$.
    \end{theorem}

    \begin{proof}
        For the first part of the theorem, suppose $\Type(\sigma) = (\lambda_1, \dotsc, \lambda_r)$ and $\sigma=\sigma_1\dotsb\sigma_r$ be the product of disjoint cycles where $\lambda_i$ is the length of $\sigma_i$. Then $\sigma' = \tau\sigma\tau^{-1} = \sigma'_1\dotsb \sigma'_r$ is also a product of disjoint cycles where $\sigma'_i = \tau\sigma_i\tau^{-1}$. Notice that $\sigma_i'$ and $\sigma_i$ share the same length for $i= 1, \dotsc, r$. Thus $\Type(\sigma') = (\lambda_1, \dotsc, \lambda_r) = \Type(\sigma)$.

        For the second part of the theorem, let $\sigma=\sigma_1\dotsb\sigma_r$ and $\sigma'=\sigma'_1\dotsb\sigma_r'$ be the products of disjoint cycles where corresponding cycles share the same length. Then we can choose such $\tau_i$ that $\tau\sigma_i\tau^{-1}=\sigma_i'$, and let $\tau = \tau_1\dotsb\tau_r$, which satisfies the desired property.
    \end{proof}

    \begin{corollary}
        The center of $S_n$ is a trivial group (for $n \geq 3$).
    \end{corollary}

    \begin{proof}
        Suppose $\sigma\in S_n$ is not the identity permutation. Consider the orbits of the group action of $\lr<>\sigma$ on $I_n$.

        First, suppose that there are multiple orbits of which at least one has more than one element, that is, $i_1, i_2\in O$ but $j\notin O$. Let $\tau = (i_1 j)$. Then $i_2, j$ are in the same orbit of the group action of $\lr<>{\tau\sigma\tau^{-1}}$ on $I_n$, which follows that $\tau\sigma\tau^{-1}\neq\sigma$. Thus $\sigma\notin C(S_n)$.

        Otherwise, suppose that there is exactly one orbit of $n$ elements with $n\geq 3$, i.e., $\sigma$ is a $n$-cycle. Assume that $\sigma=(i_1\dotsb i_n)$, then $(i_1i_2)\sigma(i_1i_2)\neq \sigma$. It follows that $\sigma\notin C(S_n)$.

        Since a non-identity permutation for $n>3$ must satisfy one of the two conditions, the proof is completed.
    \end{proof}

    In the next theorem we will compute the cardinal of each conjugacy class of $S_n$ by means of multiplication principle from combinatorics.

    \begin{theorem}
        If $\sigma\in S_n$, then the cardinal of the centralizer of $\sigma$
        \begin{equation}
            |C_G(\sigma)| = \prod_{i=1}^ni^{c_i(\sigma)}c_i(\sigma)!,
        \end{equation}
        where $c_i(\sigma)$ stands for the number of $i$-orbits of the group action on $I_n$ by $\lr<>\sigma$, namely the number of $i$-cycles in the cyclic form of $\sigma$.
    \end{theorem}

    \begin{proof}
        Any $\tau\in S_n$ can either permute the cycles of length $i$ among themselves or perform a cyclic rotation on each of the individual cycles (or both). The number of ways to do the former operation equals to the order of a $c_i(\sigma)$-permutation group, that is $c_i(\sigma)!$. Meanwhile, since the value of $\tau$ on a cycle is uniquely determined by the image of one element in this cycle, which can take $i$ values, and there are $c_i(\sigma)$ cycles on which $\tau$ need to be determined, then there are $i^{c_i(\sigma)}$ ways to do the latter operation. Since these operations are independent, the conclusion follows from the rule of product.
    \end{proof}

    \begin{corollary}
        The group equation of $S_n$ is
        \begin{equation}
            n! = \sum_{c_1+2c_2+\dotsb+nc_n=n}\frac{n!}{\prod_{i=1}^ni^{c_i}c_i!},
        \end{equation}
        where $c_1, \dotsc, c_n$ are nonnegative integers.
    \end{corollary}

    \section{Group Representations: Basic Concepts and Maschke Theorem}

    In this section, we will study the definition of a group representation and a irreducible representation, and state the existence of decomposing a representation of a finite group into irreducible ones.

    \begin{definition}[Representation]
        A \emph{representation} of a group is a group action on some linear space structure $\mathcal{V}$, that is a homomorphism $\phi\colon G\to\GL(V)$ where $V$ is a finite dimensional linear space over $\BC$.
    \end{definition}

    We shall tacitly assume in this text that representations are non-zero, although this is not formally part of the definition.

    Recall that we have defined the category of group actions and hence the one of group representations.

    \begin{definition}[Equivalence]
        Two representations $\phi\colon G\to \GL(V)$ and $\psi\colon G\to\GL(W)$ are said to be equivalent if there exists an isomorphism $T\colon V\to W$ in such that $T\phi_g = \psi_gT$ for all $g\in G$. In this case, we write $\phi\sim\psi$.
    \end{definition}

    Let $\phi\colon G\to\GL(V)$ be a representation of degree $n$. To a basis $B$ of $V$, we can associate a linear space isomorphism $T\colon V\to \BC^n$ by taking coordinates. We can then define a representation $\psi\colon V\to\GL_n(\BC)$ by setting $\psi_g = T\phi_gT^{-1}$ for all $g\in G$. It turns out that for different bases of $B$, the corresponding representations are equivalent.

    \begin{definition}[$G$-invariant Subspace]
        Let $\phi\colon G\to \GL(V)$ be a representation. A subspace $W\leq V$ is $G$-\emph{invariant} if, for all $g\in G$ and $w\in W$, one has $\phi_gw\in W$.
    \end{definition}

    Sometimes one may say $\phi|_W$ is a subrepresentation of $\phi$.

    \begin{definition}[Direct Sum]
        Gven the representations $\phi\colon G\to \GL(V)$ and $\psi\colon G\to\GL(W)$. Then their \emph{direct sum}
        \begin{equation}
            \phi\oplus\psi\colon G\to\GL(V\oplus W)
        \end{equation}
        is given by
        \begin{equation}
            (\phi\oplus\psi)_g(v, w) = (\phi_g(v), \psi_g(w)).
        \end{equation}
    \end{definition}

    \begin{definition}[Irreducible Representation]
        A non-zero representation $\phi\colon G\to \GL(V)$ of a group $G$ is said to be \emph{irreducible} if the only $G$-invariant subspace of $V$ are $\{0\}$ and $V$.
    \end{definition}

    Our goal is to decompose a representation into irreducible ones, which leads to the following definition.

    \begin{definition}[Completely Reducible Representation]
        Let $G$ be a group. A representation $\phi\colon G\to \GL(V)$ is said to be \emph{completely reducible} if $\phi$ can be written as the direct sum of irreducible representations, that is, $V=V_1\oplus \dotsb \oplus V_n$ where the $V_i$ are $G$-invariant subspaces and $\phi|_{V_i}$ is irreducible for all $i=1, \dotsc, n$.
    \end{definition}

    \begin{definition}[Decomposable Representation]
        A non-zero representation $\phi$ of a group $G$ is \emph{decomposable} if $V = V_1\oplus V_2$ with $V_1, V_2$ non-zero $G$-invariant subspaces.
    \end{definition}

    We will precede by decomposing the representation into ones whose dimension is lower. Hence if we can prove that all representations are either docomposable or irreducible, we are done. First we need some facts about the equivalence of representations.

    \begin{lemma}
        Suppose that $\phi\colon G\to \GL(V)$ and $\psi\colon G\to\GL(W)$ are two representations and $T$ is an equivalence from $\phi$ to $\psi$. If $U$ is a $G$-invariant subspace of $V$, then $T(U)$ is a $G$-invariant subspace of $W$. Moreover, if $\phi$ is irreducible [resp. decomposable or completely reducible], then so is $\psi$.
    \end{lemma}

    \begin{proof}
        Suppose that $U$ is a $G$-invariant subspace of $V$, then
        \begin{equation}
            \psi(T(U))\ni\psi_gT(u)=T\phi_g(u)=T(u')\in T(U),
        \end{equation}
        where $u' = \phi_g(u)\in U$. Hence $T(U)$ is a $G$-invariant subspace of $W$.

        For the second part, recall that $T$, by definition, is an isomorphism $V\to W$ in the category of finite dimensional linear spaces. The conclusion follows naturally.
    \end{proof}

    \begin{definition}[Unitary Representation]
        Let $V$ be an inner product space. A representation $\phi\colon G\to\GL(V)$ is said to be \emph{unitary} if $\phi_g$ is unitary for all $g\in G$, that is,
        \begin{equation}
            \lr<>{\phi_g(v), \phi_g(w)} = \lr<>{v, w}
        \end{equation}
        for all $v, w\in W$.
    \end{definition}

    \begin{lemma}\label{lem: unitary}
        Suppose $\phi\colon G\to \GL(V)$ is a representation. Then there exists some inner product on $V$ such that $\phi$ is unitary.
    \end{lemma}

    We shall prove this lemma using the crucial ``averaging trick".

    \begin{proof}
        Let $\lr<>{\blank, \blank}$ be any inner product on $V$. Since $G$ is a finite group by assumption, we can define a new operator $(\blank, \blank)$ by
        \begin{equation}
            (v, w) = \sum_{g\in G}\lr<>{\phi_g(v), \phi_g(w)}.
        \end{equation}
        It is easy to verify that $(\blank, \blank)$ is an inner product. We shall show that $\phi$ is unitary under the inner product. We compute
        \begin{equation}
            \left(\phi_h(v), \phi_h(w)\right) = \sum_{g\in G}\lr<>{\phi_g\phi_h(v), \phi_g\phi_h(w)} = \sum_{gh\in G}\lr<>{\phi_{gh}(v), \phi_{gh}(w)} = \lr(){v, w}.
        \end{equation}
        It completes the proof.
    \end{proof}

    \begin{theorem}\label{thm: decom}
        Let $\phi\colon G\to \GL(V)$ be a unitary representation of a group. Then $\phi$ is either irreducible or decomposable.
    \end{theorem}

    \begin{proof}
        Suppose $\phi$ is not irreducible. Then there is a non-zero proper $G$-invariant subspace $W$ of $V$. We claim that $W^\perp$ is also a $G$-invariant subspace of $V$. Let $v\in W^\perp$ and $w\in W$. We compute
        \begin{equation}
            \lr<>{\phi_g(v), w} = \lr<>{\phi_{g^{-1}}\phi_g(v), \phi_{g^{-1}}w} = \lr<>{v, \phi_{g^{-1}}(w)} = 0,
        \end{equation}
        where the last equality is becauce $W$ is $G$-invariant. It follows that $\phi$ is decomposable.
    \end{proof}

    The next theorem tells that every representation of a finite group can be written as the direct sum of irreducible ones.

    \begin{corollary}[Maschke]
        Every representation of a finite group is completely reducible.
    \end{corollary}

    \begin{proof}
        By Lemma \ref{lem: unitary} and Theorem \ref{thm: decom}, every representation of a finite group is either irreducible or decomposable. We can prove by induction on the dimension that every representation is completely reducible.
    \end{proof}

    \section{Character Theory: the Number of Irreducible Representations}

    Let's start with some facts on the morphisms between representations.

    \begin{lemma}
        Let $T\colon V\to W$ be in $\Hom_G(\phi, \psi)$, Then $\Kernal(T)$ is a $G$-invariant subspace of $V$ and $\Image(T)$ is a $G$-invariant subspace of $W$.
    \end{lemma}

    \begin{proof}
        If $v\in \Kernal(T)$, then $T\phi_g(v)=\psi_gT(v)=0$, which follows that $\phi_g(v)\in\Kernal(T)$. We conclude that $\Kernal(T)$ is $G$-invariant.

        Now let $w\in\Image(T)$, say $w=Tv$ with $v\in V$. Then $\psi_g(w) =\psi_gT(v) = T\phi_g(v)\in \Image(T)$. It follows that $\Image(T)$ is $G$-invariant.
    \end{proof}

    The set of morphisms from $\phi$ to $\psi$ has the additional structure of a vector space.

    \begin{lemma}\label{lem: hom-linear}
        Let $\phi\colon G\to \GL(V)$ and $\psi\colon G\to \GL(W)$ be representations. Then $\Hom_G(\phi, \psi)$ is a subspace of $\Hom(V, W)$.
    \end{lemma}

    \begin{proof}
        It suffices to verify that any linear combination of two morphisms is still a morphism. Let $T_1, T_2\in \Hom_G(\phi, \psi)$ and $c_1, c_2\in \BC$. Then
        \begin{equation}
            (c_1T_1+c_2T_2)\phi_g = c_1T_1\phi_g + c_2T_2\phi_g = c_1\psi_gT_1 + c_2\psi_gT2 = \psi_g(c_1T_1+c_2T_2),
        \end{equation}
        and hence $c_1T_1+c_2T_2\in\Hom_G(\phi, \psi)$, as desired.
    \end{proof}

    \begin{lemma}[Schur's Lemma]
        Let $\phi\colon G\to\GL(V), \psi\colon G\to\GL(W)$ be irreducible representations of $G$, and $T \in\Hom_G(\phi, \psi)$. Then either $T$ is invertible or $T=0$. Consequently:
        \begin{enumerate}[(a)]
            \item If $\phi\nsim\psi$, then $\Hom_G(\phi, \psi) = 0$;
            \item If $\phi= \psi$, then $T$ is multiplication by a scalar.
        \end{enumerate}
    \end{lemma}

    \begin{proof}
        First suppose that $T\neq 0$. Since kernal and image of $T$ are $G$-invariant subspaces and $\phi, \psi$ are irreducible, then there must be $\Kernal(T) = 0$ and $\Image(T) = W$, that is, $T$ is invertible.

        (a) is directly followed by the fact that all nonzero $T\in\Hom_G(\phi, \psi)$ is an isomorphism between linear spaces. Hence $\phi\sim\psi$ if such $T$ exists.

        For (b), let $\lambda\in \BC$ be an eigenvalue of $T$. Then by definition of an eigenvalue, $\lambda I-T$ is not invertible, while by Lemma \ref{lem: hom-linear} we have $\lambda I-T\in Hom_G{\phi, \phi}$, and hence $\lambda I - T=0$. It follows that $T$ is a multiplication by a scalar.
    \end{proof}

    From now on, let's assume the group $G$ to be finite. Let $\phi\colon G\to \GL_n(\BC)$ be a representation. Then $\phi_g=\lr(){\phi_{ij}(g)}$ where $\phi_{ij}(g)\in\BC$ for $1\leq i, j\leq n$. Thus there are $n^2$ functions $\phi_{i, j}\colon G\to \BC$.

    \begin{definition}[Group Algebra]
        Let $G$ be a group and define
        \begin{equation}
            L(G) = \BC^G = \lr\{\}{f\colon G\to\BC}.
        \end{equation}
        Then $L(G)$ is an inner product space with the inner product defined by
        \begin{equation}
            \lr<>{f_1, f_2} = \frac{1}{|G|}\sum_{g\in G}f_1(g)\overline{f_2(g)}.
        \end{equation}
    \end{definition}

    The following lemma is our second usage of averaging trick.

    \begin{lemma}\label{lem: sharp}
        Let $\phi\colon G\to \GL(V)$ and $\psi\colon G\to\GL(W)$ be representations and suppose that $T\colon V\to W$ is a linear map. Define a map $T^\sharp\colon V\to W$ by
        \begin{equation}
            T^\sharp = \frac{1}{|G|}\sum_{g\in G}\psi_{g^{-1}}T\phi_g\in\Hom_G(\phi, \psi).
        \end{equation}
        Then
        \begin{enumerate}[(a)]
            \item $T^\sharp\in\Hom_G(\phi, \psi)$;
            \item If $T\in\Hom_G(\phi, \psi)$, then $T^\sharp = T$;
            \item The map $T\in\Hom(V, W)\to T^\sharp\in\Hom_G(\phi, \psi)$ is an onto linear map.
        \end{enumerate}
    \end{lemma}

    \begin{proof}
        To establish (a), we compute
        \begin{equation}
            \begin{aligned}
                T^\sharp\phi_h &= \frac{1}{|G|}\sum_{g\in G}\psi_{g^{-1}}T\phi_g\phi_h\\
                &= \frac{1}{|G|}\sum_{x\in G}\psi_{hx^{-1}}T\phi_x\\
                &= \psi_h\frac{1}{|G|}\sum_{x\in G}\psi_{x^{-1}}T\phi_x = \psi_hT^\sharp,
            \end{aligned}
        \end{equation}
        where $x=gh$ iterates through $G$. It follows that $T^\sharp\in\Hom_G(\phi, \psi)$.

        Next, since $\psi_{g^{-1}}T\phi_g$ = $\psi_{g^{-1}}\psi_gT = T$ if $T\in \Hom_G(\phi, \psi)$, (b) is proved.

        Finally, it suffices to show that $(c_1T_1+c_2T_2)^\sharp=c_1T_1^\sharp + c_2T_2^\sharp$ for $T_1, T_2\in \Hom(V, W)$ and $c_1, c_2\in\BC$. It can be shown by a direct compution.
        \begin{equation}
            \begin{aligned}
                (c_1T_1+c_2T_2)^\sharp &= \frac{1}{|G|}\sum_{g\in G}\psi_{g^{-1}}(c_1T_1+c_2T_2)\phi_g\\
                &= c_1\frac{1}{|G|}\sum_{g\in G}\psi_{g^{-1}}T_1\phi_g + c_2\frac{1}{|G|}\sum_{g\in G}\psi_{g^{-1}}T_2\phi_g=c_1T_1^\sharp + c_2T_2^\sharp
            \end{aligned}
        \end{equation}
    \end{proof}

    The next lemma is a variant of Schur's lemma. We shall most commonly use this form.

    \begin{lemma}
        Let $\phi\colon G\to\GL(V)$ and $\psi\colon G\to\GL(W)$ be irreducible representations of $G$ and let $T\colon V\to W$ be a linear map. Then
        \begin{enumerate}[(a)]
            \item If $\phi\nsim\psi$, then $T^\sharp = 0$;
            \item If $\phi\sim\psi$, then $T^\sharp = \left(\tr(T)\big/\deg(\phi)\right)I$.
        \end{enumerate}
    \end{lemma}

    \begin{proof}\label{lem: schur-var}
        If $\phi\nsim\psi$, then by Schur's lemma we have $\Hom_G(\phi, \psi)=0$ and thus $T^\sharp = 0$. Next suppose $\phi = \psi$, we compute
        \begin{equation}
            \tr(T^\sharp) = \tr(\phi_{g^{-1}}T\phi_g) = \tr(T)
        \end{equation}
        using $\tr(AB)=\tr(BA)$. Since $T^\sharp$ is a multiplicity by a scalar, then
        \begin{equation}
            T^\sharp = \frac{\tr(T^\sharp)}{\deg(\phi)}I = \frac{\tr(T)}{\deg(\phi)}I.
        \end{equation}
        It ends the proof.
    \end{proof}

    Now we turn to compute the matrix form of the operator $T\mapsto T^\sharp$. Denote by $E_{ij}\in M_{mn}(\BC)$ the $m\times n$ matrix with 1 in position $(i, j)$ and 0 elsewhere.

    The following lemma is purely linear algebraic.

    \begin{lemma}
        Let $A\in M_{rm}(\BC)$, $B\in M_{ns}(\BC)$ and $E_{ki}\in M_{mn}(\BC)$. Then the formula $(AE_{ki}B)_{lj}$ holds where $A = (a_{ij})$ and $B=(b_{ij})$.
    \end{lemma}

    \begin{proof}
        By definition
        \begin{equation}
            (AE_{ki}B)_{lj} = \sum_{x, y}a_lx(E_{ki})_{xy}b_{yj} = a_{lk}b_{ij},
        \end{equation}
        as desired.
    \end{proof}

    Now we are ready to calculate the matrix form of the map $T\to T^\sharp$ with $\phi, \psi$ to be unitary.

    \begin{lemma}
        Let $\phi\colon G\to \GL_n(\BC)$ and $\psi\colon\ G\to\GL_m(\BC)$ be unitary representations. Let $A = E_{ki}\in M_{mn}(\BC)$. Then $A_{lj}^\sharp = \lr<>{\phi_{ij}, \psi_{kl}}$.
    \end{lemma}

    \begin{proof}
        Since $\psi$ is unitary, $\psi_{g^{-1}} = \psi_g^{-1} = \psi_{g}^*$. Thus $\psi_{lk}(g^{-1}) = \overline{\psi_{kl}(g)}$. Then we compute
        \begin{equation}
            \begin{aligned}
                A_{lk}^\sharp &= \frac{1}{|G|}\sum_{g\in G}(\psi_{g^{-1}}E_{ki}\phi_g)_{lj}\\
                &= \frac{1}{|G|}\sum_{g\in G}\psi_{lk}(g^{-1})\phi_{ij}(g)\\
                &= \frac{1}{|G|}\sum_{g\in G}\overline{\psi_{kl}(g)}\phi_{ij}(g)\\
                &= \lr<>{\phi_{ij}, \psi_{kl}},
            \end{aligned}
        \end{equation}
        as desired.
    \end{proof}

    \begin{theorem}[Schur Orthogonality Relations]
        Let $\phi\colon G\to \GL(\BC)$ and $\psi\colon G\to \GL(\BC)$ be inequivalent irreducible unitary representations. Then:
        \begin{enumerate}[(a)]
            \item $\lr<>{\phi_{ij}, \psi_{kl}} = 0$;
            \item $\lr<>{\phi_{ij}, \psi_{kl}} = 1/n$ if $i=k$ and $j=l$, and otherwise $\lr<>{\phi_{ij}, \psi_{kl}} = 0$.
        \end{enumerate}
    \end{theorem}

    \begin{proof}
        Suppose that $A = E_{ki}\in M_{mn}(\BC)$.

        For (a), since $\phi\nsim\psi$, by Lemma \ref{lem: sharp} we have $A^\sharp = 0$. Thus by Lemma \ref{lem: schur-var} we have $\lr<>{\phi_{ij}, \psi_{kl}} = A^\sharp_{ij}= 0$.

        Next suppose that $\phi=\psi$, by Lemma \ref{lem: sharp} we have
        \begin{equation}
            A^\sharp = \frac{\tr(E_{ki})}{n}I,
        \end{equation}
        while Lemma \ref{lem: schur-var} yields that $A_{ij}^\sharp =  \lr<>{\phi_{ij}, \psi_{kl}}$. The conclusion follows by simple discussion about whether $l=j$ and whether $i=k$.
    \end{proof}

    Now we are prepared to introduce the character theorey, which computes the number of irreducible representations by researching on their characters, i.e., mappings from $G$ to $\BC$.

    \begin{definition}[Character]
        Let $\phi\colon G\to \GL(V)$ be a representation. Define the \emph{character} $\chi_{\phi}\colon G\to \BC$ of $\phi$ by setting $\chi_\phi(g)=\tr(\phi_g)$. The character of an irreducible representation is called an \emph{irreducible character}.
    \end{definition}

    The first question is that whether the concept ``character" is well-defined. Since a representation itself is not necessarily a matrix representation, but only is equivalent to a matrix representation, we shall show that equivalent representations have the same character.

    \begin{lemma}\label{lem: char-wd}
        If $\phi\colon G\to \GL(V)$ is a representation, then corresponding matrix representations of $\phi$ share the same character. Moreover, if $\phi\sim\psi$, then $\chi_\phi=\chi_\psi$.
    \end{lemma}

    \begin{proof}
        Suppose $T_1, T_2\colon V\to \BC^n$ be two linear space isomorphisms and $\phi_1, \phi_2\colon V\to\GL_n(C)$ be the corresponding matrix representations where $\phi_1 = T_1\phi T_1^{-1}$ and $\phi_2 = T_2\phi T_2^{-1}$. Then we compute
        \begin{equation}
            \tr(\phi_1(g)) = \tr(T_1\phi_g T_1^{-1}) = \tr\left((T_2T_1^{-1})T_1\phi_g T_1^{-1}(T_1T_2^{-1})\right) = \tr(T_2\phi_g T_2^{-1}) = \tr(\phi_2(g)),
        \end{equation}
        by $\tr(AB)=\tr(BA)$. The second claim is followed by similar approach, which we shall omit here.
    \end{proof}

    \begin{lemma}
        Let $\phi$ be a representation of $G$. Then, for all $g, h\in G$, the equality $\chi_\phi(g)=\chi_\phi(hgh^{-1})$ holds.
    \end{lemma}

    \begin{proof}
        We compute directly
        \begin{equation}
            \chi_\phi(hgh^{-1}) = \tr(\phi_{hgh^{-1}}) = \tr(\phi_h\phi_g\phi_h^{-1}) = \tr(\phi_g) = \chi_\phi(g).
        \end{equation}
        It ends the proof.
    \end{proof}

    Functions which are constant on conjugacy classes play an important role in representation theorey.

    \begin{definition}[Class Function]
        A function $f\colon G\to \BC$ is called a \emph{class function} if $f(g)=f(ghg^{-1})$ for all $g, h\in G$, or equivalently if $f$ is constant on conjugacy classes of $G$. The space of class function is denoted $Z(L(G))$.
    \end{definition}

    In particular, characters are class functions. We shall show that the set of class functions is a subspace of the group algebra, whose dimension equals to the number of conjugacy classes of $G$. And the irreducible characters form an orthonomoral set of class functions.

    \begin{lemma}
        $Z(L(G))$ is a subspace of $L(G)$.
    \end{lemma}

    \begin{proof}
        It is easy to verify that any linear combination of two class functions is also a class function.
    \end{proof}

    Denote by $\Cl(G)$ the set of conjugacy classes of $G$, and by $\delta_C\colon G\to\BC$ the function
    \begin{equation}
        \delta_C(g)=
        \begin{cases}
            1, &g\in C;\\
            0, &g\notin C.
        \end{cases}
    \end{equation}
    where $C\in \Cl(G)$.

    \begin{theorem}
        The set $B = \{\delta_C|\ C\in\Cl(G)\}$ is a basis for $Z(L(G))$. Consequently, the dimension $\dim(Z(L(G))) = |\Cl(G)|$.
    \end{theorem}

    \begin{proof}
        $B$ spans $Z(L(G))$ since any $f$ in $Z(L(G))$ can be written as
        \begin{equation}
            f = \sum_{C\in\Cl(G)}f(C)\delta_C,
        \end{equation}
        and $B$ is linear independent by computing the inner product
        \begin{equation}
            \frac{1}{G}\sum_{g\in G}\delta_C(g)\overline{\delta_{C'}(g)}=
            \begin{cases}
                |C|/|G|, &C=C'\\
                0, &C\neq C',
            \end{cases}
        \end{equation}
        where $C, C'\in \Cl(G)$.
    \end{proof}

    The next theorem is one of the fundamental results in group representation theory, for which we have been well prepared.

    \begin{theorem}[First Orthogonality Relations]
        Let $\phi, \psi$ be irreducible representations of $G$. Then
        \begin{equation}
            \lr<>{\chi_\phi, \chi_\psi}=
            \begin{cases}
                1, &\phi\sim\psi;\\
                0, &\phi\nsim\psi.
            \end{cases}
        \end{equation}
        Thus the irreducible characters of $G$ form an orthonormal set of class functions.
    \end{theorem}

    \begin{proof}
        Suppose with out loss of generality that $\phi, \psi$ are unitary matrix representations. Then we compute directly
        \begin{equation}
            \begin{aligned}
                \lr<>{\chi_\phi, \chi_\psi}
                &= \frac{1}{|G|}\chi_\phi(g)\overline{\chi_\psi(g)}\\
                &= \sum_{i=1}^n\sum_{j=1}^m\frac{1}{|G|}\sum_{g\in G}\phi_{ii}(g)\overline{\psi_{ii}(g)}\\
                &= \sum_{i=1}^n\sum_{j=1}^m \lr<>{\phi_{ii}(g), \psi_{jj}(g)}.
            \end{aligned}
        \end{equation}
        If $\phi\nsim\psi$, by Schur orthogonality relations we have $\lr<>{\phi_{ii}(g), \psi_{jj}(g)}=0$, and thus $\lr<>{\chi_\phi, \chi_\psi}=0$. Otherwise assume that $\phi\sim\psi$. By Lemma \ref{lem: char-wd} we may assume that $\phi=\psi$. Again by Schur orthogonality relations we have
        \begin{equation}
            \lr<>{\phi_{ii}(g), \phi_{jj}(g)}=
            \begin{cases}
                1/n, &i=j;\\
                0, &i\neq j.
            \end{cases}
        \end{equation}
        It follows that
        \begin{equation}
            \lr<>{\chi_\phi, \chi_\psi}=\sum_{i=1}^n\lr<>{\phi_{ii}(g), \phi_{ii}(g)}=1,
        \end{equation}
        which ends the proof.
    \end{proof}

    \begin{corollary}
        There are at most $|\Cl(G)|$ equivalence classes of irreducible representations of $G$.
    \end{corollary}

    \begin{proof}
        Since inequivalent irreducible representations have distinct characters, and these characters are linear independent in a linear space of dimension $|\Cl(G)|$.
    \end{proof}

    \begin{lemma}
        Let $\phi = \psi+\rho$. Then $\chi_\phi = \chi_\psi + \chi_\rho$.
    \end{lemma}

    \begin{proof}
        Suppose that all these representations are matrix ones. Then notice that $\phi_g$ is a block diagonal matrix where the first matrix is $\psi_g$ and the second matrix is $\rho_g$. Since the trace of a block diagonal matrix equals to the sum of traces of block matrices, the conclusion follows.
    \end{proof}

    The next theorem shows that the decomposition of a representation into irreducible ones is unique.

    \begin{theorem}
        Let $\phi^{(1)}, \dotsc, \phi^{(s)}$ be a complete set of representatives of the equivalence classes of irreducible representations of $G$ and let
        \begin{equation}
            \psi\sim m_1\phi^{(1)}\oplus m_2\phi^{(2)}\oplus\dotsb\oplus m_s\phi^{(s)}.
        \end{equation}
        Then $m_i=\lr<>{\chi_\psi, \chi_{\psi^{(i)}}}$. Consequently, the decomposition of $\psi$ into irreducible constituents is unique and $\psi$ is determined up to equivalence by its character.
    \end{theorem}

    \begin{proof}
        The theorem follows from the previous lemma and the first orthogonality relations.
    \end{proof}

    \section{Symmetric Groups: Young Tableaux and Irreducible Representations}

    We have shown that each conjugacy class of a symmetric group of $n$ letters corresponds to a positive partition of $n$, and that the number of inequivalent irreducible representations is at most the number of conjugacy classes. In this section, we will construct one irreducible representation of $S_n$ for each partition of $n$ and will prove that any two representations of them are not equivalent by means of Young tableaux.

    \begin{definition}[Partition]
        A \emph{partition} of $n$ is a tuple $\lambda = (\lambda_1, \dotsc, \lambda_l)$ of positive integers such that $\lambda_1\geq \lambda_2\geq \dotsb \geq \lambda_l$ and $\lambda_1+\dotsb + \lambda_l=n$. Denote $\lambda\vdash n$ to indicate that $\lambda$ is a partition of $n$.
    \end{definition}

    Now we are ready to introduce the definition of Young diagram. It is a variation of Ferrers diagram which we have studied in the Combinatorics courses, with the dots replaced by boxes where numbers can be placed.

    \begin{definition}[Young Diagram]
        If $\lambda = (\lambda_1, \dotsc, \lambda_l)$ is a partition of $n$, then the \emph{Young diagram} of $\lambda$ consists of $n$ boxes placed into $l$ rows where the $i$th row has $\lambda_i$ boxes.
    \end{definition}

    \begin{definition}[Conjugate Partition]
        If $\lambda\vdash n$, then the \emph{conjugate partition} $\lambda^T$ of $\lambda$ is the partition whose Young diagram is the transpose of the diagram of $\lambda$. An example is shown in \ref{fig: young-diagram}.
    \end{definition}

    \begin{figure}[htb]
        \begin{subfigure}[b]{0.49\textwidth}
            \centering
            \begin{ytableau}
                {} & {} & {} \\
                {} & {} \\
            \end{ytableau}
            \caption{$\lambda=(3, 2)$}
        \end{subfigure}
        \hfill
        \begin{subfigure}[b]{0.49\textwidth}
            \centering
            \begin{ytableau}
                {} & {} \\
                {} & {} \\
                {}
            \end{ytableau}
            \caption{$\lambda^T = (2, 2, 1)$}
        \end{subfigure}
        \caption{Young diagram of $\lambda=(3, 2)$ and its conjugate partition.}
        \label{fig: young-diagram}
    \end{figure}

    We can define a partial order, denoted $\unrhd$, on the set of all Young diagrams of given integer $n$, where $\lambda = (\lambda_1, \dotsc, \lambda_l)\unrhd \mu = (\mu_1, \dotsc, \mu_m)$ if and only of
    \begin{equation}
        \lambda_1 + \dotsb + \lambda_i \geq \mu_1 + \dotsb + \mu_i
    \end{equation}
    for all $i\geq 1$, where we take $\lambda_i = 0$ ($\mu_i = 0$ resp.) if $i>l$ ($i>m$ resp.). The partial order is called the \emph{domination order}. We write $\lambda\rhd\mu$ if $\lambda\unrhd\mu$ and $\lambda\neq\mu$.

    \begin{theorem}
        The dominance order satisfies:
        \begin{enumerate}
            \item Reflexivity: $\lambda\unrhd\lambda$;
            \item Anti-symmetry: $(\lambda\unrhd\mu)\wedge(\mu\unrhd\lambda)\Leftrightarrow(\lambda=\mu)$;
            \item Transitivity: $(\lambda\unrhd\mu)\wedge(\mu\unrhd\rho)\Leftrightarrow(\lambda\unrhd\rho)$;
        \end{enumerate}
        that is to say, the dominance order is a partial order.
    \end{theorem}

    Young tableaux are obtained from Young diagrams by placing the distinct integers $1, \dotsc, n$ into the boxes.

    \begin{definition}[Young Tableaux]
        If $\lambda\vdash n$, then a $\lambda$-\emph{tableau} is an array $t$ of integers, usually denoted by $t^\lambda$, obtained by placing $1, \dotsc, n$ into the boxes of the Young diagram for $\lambda$.
    \end{definition}

    There is a rather technical combinatorial fact which is useful in the sequel is that if $t^\lambda$ is a $\lambda$-tableau and $s^\mu$ is a $\mu$-tableau such that the integers in any given row of $s^\mu$ belong to distinct columns of $t^\lambda$, then $\lambda\unrhd\mu$. We shall prove the fact through a lemma.

    \begin{lemma}\label{lem: tableaux}
        Let $\lambda = (\lambda_1, \dotsc, \lambda_l)$ and $\mu = (\mu_1, \dotsc, \mu_m)$ be partitions of $n$. Suppose that $t^\lambda$ is a $\lambda$-tableau and $s^\mu$ is a $\mu$-tableau such that entries in the same row of $s^\mu$ are located in different columns of $t^\lambda$. Then we can find a $\lambda$-tableau $u^\lambda$ such that:
        \begin{enumerate}
            \item The $j$th columns of $t^\lambda$ and $u^\lambda$ contain the same elements for $1\leq j \leq l$;
            \item The entries of the first $i$ rows of $s^\mu$ belong to the first $i$ rows of $u^\lambda$ for each $1\leq i \leq m$.
        \end{enumerate}
    \end{lemma}

    \begin{proof}
        We will prove the lemma by constructing a $\lambda$-tableau $t_r^\lambda$ such that
        \begin{enumerate}[(a)]
            \item The $j$th columns of $t^\lambda$ and $t_r^\lambda$ contain the same elements for $1\leq j\leq l$;
            \item The entries of the first $i$ rows of $s^\mu$ belong to the first $i$ rows of $t_r^\lambda$ for $1\leq i \leq r$
        \end{enumerate}
        by induction. Setting $u^\lambda = t_m^\lambda$ will complete the proof.

        When $r=1$, let $k$ be an element in the first row of $s^\mu$ and denote by $c(k)$ the column of $t^\lambda$ containing $k$. Then switch the first entry of column $c(k)$ with $k$.

        Now suppose that $t_r^\lambda$ has been constructed for $r$, we shall construct $t_{r+1}^\lambda$ from $t_r^\lambda$ as follows. Again, let $k$ be an element in the row $r+1$ of $s^\mu$. If $k$ is in the first $r+1$ rows of $t_r^\lambda$, then it is done. Otherwise, since $\lambda\unrhd\mu$, then the column $c(k)$ must intersect the row $r+1$ of $t_{r+1}^\lambda$, that is, the row $r+1$ must have at least $c(k)$ entries. Then we can switch $k$ with the entry $c(k)$ of row $r+1$ for each such $k$ independently, in order to get the tableau $t_{r+1}^\lambda$ satisfying desired properties.
    \end{proof}

    It follows immediately from the lemma that $\lambda\unrhd\mu$ by restating the property (b) for $u^\lambda = t_m^\lambda$. Hence we have proved the dominance lemma, which is stated as follows.

    \begin{lemma}[Dominance Lemma]\label{lem: dom}
        Let $\lambda$ and $\mu$ be partitions of $n$ and suppose that $t^\lambda$ and $s^\mu$ are tableaux of respective shapes $\lambda$ and $\mu$. Moreover, supposes that integers in the same row of $s^\mu$ are located in different columns of $t^\lambda$. Then $\lambda\unrhd\mu$.
    \end{lemma}

    In the next part, we shall construct the irreducible representations for $S_n$. We need some definitions first.

    \begin{definition}
        Let $t$ be a Young tableau. Then the \emph{column stablizer} $C_t$ of $t$ is the subgroup of $S_n$ preserving the columns of $t$, that is, $\sigma\in C_t$ if and only if $\sigma(i)$ is in the same column as $i$ for every $i\in \{1, \dotsc, n\}$.
    \end{definition}

    In the previous proof of Lemma \ref{lem: tableaux}, the transformation from $t^\lambda$ to $u^\lambda$ was actually a column stablizer.

    \begin{definition}[Tabloid]
        Define an equivalence relation $\sim$ on the set of all $\lambda$-tableaux by putting $t_1\sim t_2$ only if they have the same entries in each row. Call a $\sim$-equivalence class of $\lambda$-tableaux a $\lambda$-\emph{tabloid}, which is denoted by $[t]$. Denote by $T^\lambda$ the set of all $\lambda$-tabloids and by $T_\lambda$ the tabloid which has $j$ in the $j$th box.
    \end{definition}

    The action of $S_n$ on $\lambda$-tableaux induces an action of $S_n$ on $\lambda$-tabloids since $t_1\sim t_2$ implies $\sigma t_1\sim \sigma t_2$. Suppose that $\lambda = (\lambda_1, \dotsc, \lambda_l)$. Let $S_\lambda$ the stablizer of $T_\lambda$, which is called the \emph{Young subgroup} associated to the partition $\lambda$.

    For a partition $\lambda$, set $M^\lambda = \BC T^\lambda$ and let $\phi^\lambda\colon S_n\to \GL(M^\lambda)$ be the associated permutation representation, that is
    \begin{equation}
        \phi_\sigma^\lambda\lr(){\sum_{[t]\in T^\lambda}c_{[t]}[t]} = \sum_{[t]\in T^\lambda}c_{[t]}\sigma[t].
    \end{equation}

    If $\lambda\neq (n)$, then $\phi^\lambda$ is not irreducible. Hence we need to isolate the irreducible constituent from it.

    \begin{definition}[Polytabloid]
        Let $\lambda, \mu\vdash n$. Let $t$ be a $\lambda$-tableau and define a linear operator $A_t\colon M^\mu\to M^\mu$ by
        \begin{equation}
            A_t = \sum_{\pi\in C_t}\sgn(\pi)\phi^\mu_\pi.
        \end{equation}
        In the case $\lambda=\mu$, the element
        \begin{equation}
            e_t = A_t[t] = \sum_{\pi\in C_t}\sgn(\pi)\pi[t]
        \end{equation}
        of $M^\lambda$ is called the polytabloid associated to $t$.
    \end{definition}

    We shall now show that the subspace of $M^\lambda$ spanned by all polytabloids $e_t$ with $t$ a $\lambda$-tableau is in fact a $S_n$-invariant subspace of $M^\lambda$.

    \begin{theorem}\label{thm: poly-commu}
        If $\sigma\in S_n$ and $t$ is a $\lambda$-tableau, then $\phi_\sigma^\lambda e_t = e_{\sigma t}$.
    \end{theorem}

    \begin{proof}
        Notice that $\tau(X) = X$ if and only if $\sigma\tau\sigma^{-1}(\sigma(X)) = \sigma(X)$ where $X\subseteq I_n$. Hence $C_{\sigma t}=\sigma C_t\sigma^{-1}$. Now we compute
        \begin{equation}
            \begin{aligned}
                \phi_\sigma^\lambda e_t &= \phi_\sigma^\lambda A_t[t]\\
                &= \sum_{\pi\in C_t}\sgn(\pi)\phi_\sigma^\lambda \pi [t]\\
                &= \sum_{\sigma\pi\sigma^{-1}\in C_{\sigma t}}\sgn(\sigma\pi\sigma^{-1})\sigma\pi\sigma^{-1}[\sigma t]\\
                &= e_{\sigma t}.
            \end{aligned}
        \end{equation}
        It ends the proof.
    \end{proof}

    We can now define our desired subrepresentation.
    \begin{definition}[Sprecht Representation]
        Let $\lambda$ be a partition of $n$. Define $S^\lambda$ to be the subspace of $M^\lambda$ spanned by the polytabloids $e_t$ with $t$ a $\lambda$-tableau. Let $\psi^\lambda\colon S_n\to \GL(S^\lambda)$ be the subrepresentation of $\phi^\lambda\colon S_n\to \GL(M^\lambda)$. It is called the \emph{Sprecht representation} associated to $\lambda$.
    \end{definition}

    We shall show that $\psi^\lambda$ are the irreducible representations of $S_n$ and each two of them are not equivalent.

    \begin{lemma}\label{lem: Ats}
        Let $\lambda, \mu\vdash n$ and suppose that $t^\lambda$ is a $\lambda$-tableau and $s^\mu$ is a $\mu$-tableau such that $A_{t^\lambda}[s^\mu]\neq 0$. Then $\lambda\unrhd\mu$. Moreover, if $\lambda=\mu$, then $A_{t^\lambda}[s^\mu]=\pm e_{t^\lambda}$.
    \end{lemma}

    \begin{proof}
        We shall use the Dominance Lemma to prove the first assertion. Suppose there are two elements $i, j$ that are in the same row of $s^\mu$ and in the same column of $t^\lambda$. Then $(i j)s^\mu = s^\mu$ and $(i j)\in C_{t^\lambda}$. Let $H=\{\id, (i j)\}$ be the subgroup of $C_{t^\lambda}$, and let $\sigma_1, \dotsc, \sigma_k$ be a complete set of left cosets for $H$ in $C_{t^\lambda}$, that is to say, $\sigma_1H\cup\dotsb \cup \sigma_kH = C_{t^\lambda}$. Then we have
        \begin{equation}
            \begin{aligned}
                A_{t^\lambda}[s^\mu] &= \sum_{\pi\in C_{t^\lambda}}\sgn(\pi)\phi_\pi^\mu[s^\mu]\\
                &= \sum_{r=1}^k \lr(){\sgn(\sigma_r)\phi_{\sigma r}^\mu + \sgn(\sigma_r(i j))\phi_{\sigma_r(i j)}^\mu}[s^\mu]\\
                &= \sum_{r=1}^k\sgn(\sigma_r)\phi_{\sigma_r}^\mu(\phi_\id^\mu-\phi_{(i j)}\mu)[s^\mu]\\
                &= 0.
            \end{aligned}
        \end{equation}
        It follows from Lemma \ref{lem: dom} that if $A_{t^\lambda}[s^\mu]\neq 0$, then $\lambda\vdash\mu$.

        Next suppose that $\lambda=\mu$. Lemma \ref{lem: tableaux} states that there is a $\lambda$-tableau $u^\lambda = \sigma t^\lambda$, where $\sigma\in C_{t^\lambda}$, such that the entries of the first $i$ rows of $s^\mu$ belong to the first $i$ rows of $u^\lambda$ for each $1\leq i\leq m$. Since $\lambda=\mu$, we can prove by induction that $[u^\lambda]=[s^\mu]$. It follows that
        \begin{equation}
            \begin{aligned}
                A_{t^\lambda}[s^\mu]
                &= \sum_{\pi\in C_{t^\lambda}}\sgn(\pi)\phi_\pi^\mu[s^\mu]\\
                &= \sum_{\tau\sigma^{-1}\in C_{t^\lambda}}\sgn(\tau\sigma^{-1})\phi_{\tau\sigma^{-1}}^\lambda[u^\lambda]\\
                &= \sgn(\sigma^{-1})\sum_{\tau\in C_{t^\lambda}}\sgn(\tau)\phi_\tau^\lambda[t^\lambda]\\
                &= \sgn(\sigma^{-1})e_{t^\lambda},
            \end{aligned}
        \end{equation}
        which completes the proof.
    \end{proof}

    \begin{lemma}\label{lem: At}
        Let $t$ be a $\lambda$-tableau. Then the image of the operator $A_t\colon M^\lambda\to M^\lambda$ is $\BC e_t$.
    \end{lemma}
    \begin{proof}
        Notice that $A_t[s]$ equals either $\pm e_t$ or 0, for $[s]$ being any $\lambda$-tabloid. It follows the assertion.
    \end{proof}

    Recall that $M^\lambda = \BC T^\lambda$ has an inner product structure where $T^\lambda$ is an orthonormal basis. Notice that $\phi^\lambda_\sigma$ is unitary, and its adjoint is $(\phi^\lambda_\sigma)^*=\phi^\lambda_{\sigma^{-1}}$, for all $\sigma\in S_n$. It follows that
    \begin{equation}
        A_t^* = \sum_{\pi\in C_t}\sgn(\pi)(\phi^\lambda_\pi)^* = \sum_{\pi^{-1}\in C_t}\sgn(\pi^{-1})(\phi^\lambda_{\pi^{-1}}) = A_t,
    \end{equation}
    which means that $A_t$ is self-adjoint.

    The following theorem is the key to prove that the $\psi^\lambda$ are the irreducible representations of $S_n$.

    \begin{theorem}[Subrepresentation Theorem]\label{thm: subrepr}
        Let $\lambda$ be a partition of $n$ and suppose that $V$ is an $S_n$-invariant subspace of $M^\lambda$. Then either $S^\lambda\subseteq V$ or $V\subseteq (S^\lambda)^\perp$.
    \end{theorem}

    \begin{proof}
        First suppose that there is some $\lambda$-tableau $t$ and some vector $v\in V$ such that $A_t(v)\neq 0$. Since $V$ is a $S_n$-invariant subspace, then we have
        \begin{equation}
            A_t(v)=\sum_{\pi\in C_t}\sgn(\pi)\phi_\pi^\lambda(v)\in V.
        \end{equation}
        By Lemma \ref{lem: At}, we have $A_t(v)\in V\cap \BC e_t$. Thus $e_t\in V$. Again by the fact that the $V$ is $S_n$-invariant, $e_{\sigma t} = \phi^\lambda_\sigma e_t\in V$. Thus $S^\lambda\subseteq V$.

        Next suppose that for any $\lambda$-tableau $t$ and any vector $v\in V$, we have $A_t(v) = 0$. Then we compute
        \begin{equation}
            \lr<>{v, e_t} = \lr<>{v, A_t[t]} = \lr<>{A_tv, [t]} = 0.
        \end{equation}
        Since $v$ and $e_t$ are arbitrary, we have $V\subseteq (S^\lambda)^\perp$. It ends the proof.
    \end{proof}

    \begin{corollary}
        Let $\lambda\vdash n$. Then $\psi^\lambda\colon S_n\to \GL(S^\lambda)$ is irreducible.
    \end{corollary}

    It remains to be shown that $\lambda\neq\mu$ implies that $\psi^\lambda\nsim\psi^\mu$. Let's start with some lemmas. Recall that $T\in \Hom_{S_n}(\phi^\lambda, \phi^\mu)$ if and only if $T\phi^\lambda_\sigma = \phi^\mu_\sigma$ for $\sigma\in S_n$.

    \begin{lemma}\label{lem: T-hom}
        Suppose that $\lambda, \mu\vdash n$ and let $T\in \Hom_{S_n}(\phi^\lambda, \phi^\mu)$. If $S^\lambda\nsubseteq \Kernal(T)$, then $\lambda\unrhd \mu$.
    \end{lemma}

    \begin{proof}
        Recall that $\Kernal(T)$ is a $S_n$-invariant subspace. It follows from Theorem \ref{thm: subrepr} that\\ $\Kernal(T)\subseteq(S^\lambda)^\perp$. Hence for any $\lambda$-tableau $t$, there is $0\leq Te_t = TA_t[t]=A_tT[t]$, where it should be noticed that the two $A_t$ in the last equation have different domains. Since $T[t]$ is a linear combination of $\mu$-tabloids, there exists a $\mu$-tabloid $[s]$ such that $A_t[s]\neq 0$. By Lemma \ref{lem: Ats} we have $\lambda\unrhd\mu$.
    \end{proof}

    \begin{lemma}\label{lem: hom-ker}
        If $\Hom_{S_n}(\psi^\lambda, \phi^\mu)\neq 0$, then $\lambda\unrhd \mu$.
    \end{lemma}

    \begin{proof}
        We can extend any non-zero morphism $T\colon S^\lambda\to M^\mu$ by putting $T(v+w)=Tv$ for elements $v\in S^\lambda$ and $w\in (S^\lambda)^\perp$. Notice that $(S^\lambda)^\perp$ is $S_n$-invariant since $(\phi_\sigma^\lambda)=\phi_{\sigma^{-1}}^\lambda$. Thus the extension is also a morphism. Clearly $S^\lambda\nsubseteq\Kernal(T)$ and so $\lambda\unrhd\mu$ by Lemma \ref{lem: T-hom}.
    \end{proof}

    \begin{theorem}
        The Sprecht representations $\psi^\lambda$ with $\lambda\vdash n$ form a complete set of inequivalent irreducible representations of $S_n$.
    \end{theorem}

    \begin{proof}
        Suppose that $\psi^\lambda\sim\psi^\mu$, then
        \begin{equation}
            0\neq \Hom_{S_n}(\psi^\lambda, \psi^\mu)\subseteq \Hom_{S_n}(\psi^\lambda, \phi^\mu).
        \end{equation}
        Thus by Lemma \ref{lem: hom-ker}, we have $\lambda\unrhd\mu$. A symmetric argument shows that $\mu\unrhd\lambda$. Thus we have $\lambda=\mu$ if $\psi^\lambda\sim\psi^\mu$, as desired.
    \end{proof}

    \section{Dimensions of the Irreducible Representations}

    In this section, we shall compute the dimension of $S^\lambda$ for each partition $\lambda\vdash n$ by the following procedure. First we shall rove the set of polytabloids associated with standard $\lambda$-tableaux is linear independent and generates $S^\lambda$. Then we shall compute the number of standard $\lambda$-tableaux by Hook length formula.

    \begin{definition}[Standard Young Tableaux]
        A young tableau is called \emph{standard} provided that the entries along the rows and down the columns are ordered in increasing values. In this case we call the corresponding tabloid and polytabloid standard.
    \end{definition}

    We shall prove by induction that the polytabloids associated with standard tableaux are linear independent, where we need a linear order first.

    \begin{lemma}
        Suppose $\lambda\vdash n$ and $[t], [u]$ are two $\lambda$-tabloid. Then we say $[t]>[u]$ if there exists some $i$ such that,
        \begin{enumerate}[(1)]
            \item for all $j>i$, $j$ is in the same row of both $[t]$ and $[u]$;
            \item $i$ is in a higher row of $[t]$ than $[u]$, where the first row is the highest row;
        \end{enumerate}
        and we say $[t]\geq [u]$ if $[t]>[u]$ or $[t]=[u]$. The relation $\leq$ is a linear order on the set of standard Young tableaux.
    \end{lemma}

    \begin{proof}
        Notice that if $[t]\neq [u]$, then there exists a greatest integer $i\leq n$ such that $i$ is in different row of $[t]$ and $[u]$. The proof is then a routine.
    \end{proof}

    \begin{lemma}\label{lem: leq}
        Suppose $t$ is a tableau and $\pi$ is a column stablizer of $t$. Then $\pi [t]\geq [t]$.
    \end{lemma}

    \begin{proof}
        Suppose $i$ is the greatest integer such that $\pi (i)\neq i$. Thus all $j>i$ are not permuted. Since $\pi$ is a column stablizer and the columns are in in ascending order. Then the entry of $t$ with value $i$ must be permutated to a higher row. It ends the proof.
    \end{proof}

    \begin{theorem}
        The polytabloids associated with the standard tableaux are linearly independent.
    \end{theorem}

    \begin{proof}
        Let $t_1> t_2 > \dotsc > t_r$ be all standard tableaux. Suppose there is a linear combination of the associated polytabloids that gives 0, i.e.,
        \begin{equation}
            c_1e_{t_1}+\dotsb + c_re_{t_r}=0.
        \end{equation}
        Let's check the coefficients of $[t_r]$. Since there is
        \begin{equation}
            e_{t_j} = \sum_{\pi\in C_{t_j}}\sgn(\pi)\pi[t_j],
        \end{equation}
        and by Lemma \ref{lem: leq} we have $\pi[t_j]\geq[t_j]\neq [t_r]$, unless $\pi=\id$ and $j=r$. Thus by taking the inner product with $[t_r]$, we have
        \begin{equation}
            \lr<>{c_1e_{t_1}+\dotsb + c_re_{t_r}, [t_r]} = c_r = 0.
        \end{equation}
        Since $r$ is arbitary, it completes the proof.
    \end{proof}

    To prove that the polytabloids associated with the Young tableaux spans $S^\lambda$, we need some techniques. Let's start with some definition.

    \begin{definition}[Composition]
        A \emph{composition} of $n$ is a tuple $\lambda=(\lambda_1, \dotsc, \lambda_l)$ of positive integers such that $\lambda_1 + \dotsb + \lambda_l=n$.
    \end{definition}

    The definition of a tableau, a tabloid, a polytabloid and the dominance order can be extended to composition in a natural way. These concepts introduced before are in a row manner, i.e., two tableaux are equivalent if and only if they have the same elements in each row, and we can get similar concepts by substituting rows by columns and columns by rows. Denote the column variant of the concepts by a subscript $c$, i.e. $[t]_c$ for a column tabloid. We shall now define the dominance order on column tabloids.

    Suppose that $[t^\lambda]_c$ is a column tabloid with $\lambda$ a composition. For each index $i$ with $1\leq i\leq n$, let $[t^{\lambda_i}]_c$ be the column tabloid formed by all elements no greater than $i$ in $[t^\lambda]_c$, and let $\lambda_i$ be the corresponding column composition of $i$.

    \begin{definition}[Dominance Order on Tabloids]
        Let $[t^\lambda]_c$ and $[s^\mu]_c$ be two column tabloids with composition sequences $\lambda_i$ and $\mu_i$, respectively. Then $[t^\lambda]_c$ dominates $[s^\mu]_c$, denoted by $[t^\lambda]_c\unrhd_c [s^\mu]_c$, if $\lambda_i\unrhd_c\mu_i$ for all $i$.
    \end{definition}

    Just as for partitions, there is a dominance lemma for column tabloids.

    \begin{lemma}[Dominance Lemma for Tabloids]\label{lem: dom-tab}
        If $k<l$ and $k$ appears on the right hand side of $l$ in $[t]$, then $(k l)[t]_c\rhd[t]_c$.
    \end{lemma}

    \begin{proof}
        The lemma follows from checking the composition sequences for both tabloids.
    \end{proof}

    \begin{corollary}
        If $t$ is standard and $\lr<>{[s]_c, (e_t)_c}\neq 0$, then $[t]_c\unrhd_c [s]_c$.
    \end{corollary}

    \begin{proof}
        By the definition of $(e_t)_c$, we can assume that $s=\pi t$, where $\pi\in R_t$ is a row stablizer of $t$. Induct on the number of row inversions in $s$, that is, the number of pairs $k<l$ in the same row of $s$ such that $k$ is on the right hand side of $l$. By Lemma \ref{lem: dom-tab} we have $(k l)[s]_c\rhd_c[s]_c$, and it is easy to find out that $(k l)[s]_c$ has fewer inversions than $[s]_c$. Since $[t]_c$ has no inversions, then by induction we have $[t]_c\unrhd_c[s]_c$.
    \end{proof}

    Now it is ready to introduce the \emph{straightening algorithm}, which takes any standard tableau $t$ and writes the associated polytabloids $e_t$ in terms of other polytabloids associated with standard Young tableaux.

    Take a $\lambda$-tableau $t$. There exist some $\sigma\in C_t$ such that all columns of $s = \sigma t$ are in increasing order. Then $e_{s}=\pm e_{t}$ by easy calculation from the definition. If the tableau $s$ is not standard, then there must be two adjacent entries in the same row where the left is greater than the right, which is called a row descent. Denote by $A$ the entries below the left row-descending entry (including itself) and by $B$ the entries above the right row-descending entry (including itself), which is illustrated in \ref{fig: straightening}. The \emph{Garnir element}, denoted by $g_{A, B}$, is the signed sum of all permutations of $A\cup B$ that keep both subsets $A$ and $B$ without column descents, that is,
    \begin{equation}
        g_{A, B}=\sum_{\pi\in\Pi}\sgn(\pi)\phi^\lambda_\pi,
    \end{equation}
    where $\Pi$ is the collection of such permutations.

    \begin{figure}[htb]
        \centering
        \begin{ytableau}
            {} & {a_1} & *(cyan) {b_1} & {} & {} & {}\\
            {} & {\vdots} & *(cyan) {\vdots} & {} & {}\\
            {} & *(yellow) a_i & *(cyan) b_i & {}\\
            {} & *(yellow) \vdots & {\vdots}\\
            {} & *(yellow) \vdots & {b_q}\\
            {} & *(yellow) {a_p}\\
            {}
        \end{ytableau}
        \caption{$s$ is a tableau such that all columns are in increasing order, where $a_i>b_i$ is a row descent. The set $A$ is the collection of yellow entries $a_i, \dotsc, a_p$, and the set $B$ is the collection of cyan entries $b_1, \dotsc, b_i$, where $b_1<\dotsb <b_i<a_i<\dotsb < a_p$.}
        \label{fig: straightening}
    \end{figure}

    \begin{lemma}\label{lem: Garnir}
        Let $s$ be a $\lambda$-tableau with no column descents and $g_{A, B}$ be a corresponding Garnir element. Then we have
        \begin{equation}
            g_{A, B}(e_t) = \sum_{\pi\in\Pi}(\sgn(\pi)\phi_\pi^\lambda(e_t))=0.
        \end{equation}
    \end{lemma}

    Moreover, it is true for $\lambda$-tableau $t$ which has column descents, since $e_t=\pm e_s$.

    \begin{proof}
        First, notice that
        \begin{equation}
            \begin{aligned}
                g_{A, B}(e_s)
                &= \sum_{\pi\in\Pi}\sgn(\pi)\phi_\pi^\lambda(e_s)\\
                &= \sum_{\pi\in\Pi}\sum_{\tau\in C_s}\sgn(\pi\tau)\pi\tau[s]\\
                &= \sum_{\pi\tau\in S_{A\cup B}}\sgn(\pi\tau)\pi\tau[s],
            \end{aligned}
        \end{equation}
        where $S_{A\cup B}$ stands for the set of all permutations which fix the elements outside $A\cup B$. For a given tableau $s$ and all $\sigma\in S_{A\cup B}$, there exist two adjacent elements $a$ and $b$ in $s$, and a transposition of the two elements denoted $(a b)$, where $(a b)\sigma[s] = \sigma[s]$. If we call $\sigma\sim (a b)\sigma$, then $\sim$ is an equivalence relation where every equivalence class has exactly two elements. Since
        \begin{equation}
            \sgn(\sigma)\sigma[s] + \sgn\left((a b)\sigma\right)(a b)\sigma[t] = \sgn(\sigma)\left(\sigma[s]-(a b)\sigma[t]\right) = 0,
        \end{equation}
        then the whole expression cancels to 0.
    \end{proof}

    \begin{theorem}
        The standard $\lambda$-polytabloids spans $S^\lambda$.
    \end{theorem}

    \begin{proof}
        It suffices to show that every polytabloid $e_s$ with no column descents can be represented as a linear combination of polytabloids each of whose corresponding tabloid is greater than $s$ under dominance order of tabloids.

        Take $A=\{a_i, \dotsc, a_p\}$ and $B = \{b_1, \dotsc, b_i\}$ with
        \begin{equation}
            b_1<\dotsb <b_i<a_i<\dotsb <a_p.
        \end{equation}
        as is illustrated in \ref{fig: straightening}.

        By Lemma \ref{lem: Garnir} and Theorem \ref{thm: poly-commu}, there is
        \begin{equation}
            e_s = \mp\sum_{\pi\in(\Pi-\{\id\})}\sgn(\pi)e_{\pi s}.
        \end{equation}
        It follows from the dominance lemma (Lemma \ref{lem: dom-tab}) that $[\pi s]_c\rhd_c[s]_c$ for all $\pi\neq \id$, which ends the proof.
    \end{proof}

    Up to now, we have shown that the dimension of $S^\lambda$ equals to the number of standard $\lambda$-tableaux, which can be calculated by the \emph{hook length formula}, stating as follows.

    \begin{theorem}[Hook Length Formula]
        Suppose $\lambda\vdash n$. Then the number of standard $\lambda$-tableaux
        \begin{equation}
            \dim(S^\lambda)=\frac{n!}{\prod_{i, j}h_{i, j}},
        \end{equation}
        where $(i, j)$ is the index of an entry in the tableau, i.e., the entry locates in row $i$ and column $j$, and the hook $h_{i, j} = r_i - i + c_j-j + 1$, where $r_i$ (resp., $c_j$) denotes the number of entries in row $i$ (resp., column $j$).
    \end{theorem}

    We shall omit the prove of the hook length formula, since time is limited.

    \section*{Reference}
    \begin{enumerate}[1]
        \item Thomas W. Hungerford. Algebra. \url{https://doi.org/10.1007/978-1-4612-6101-8}.
        \item B. Steinberg. Representation Theory of Finite Groups: An Introductory Approach. \\\url{https://doi.org/10.1007/978-1-4614-0776-8}.
        \item R. McNamara. Irreducible Representations of the Symmetric Group. \\\url{http://math.uchicago.edu/~may/REU2013/REUPapers/McNamara.pdf}.
        \item Bruce Sagan. The Symmetric Group. \url{https://doi.org/10.1007/978-1-4757-6804-6}
    \end{enumerate}
\end{document}

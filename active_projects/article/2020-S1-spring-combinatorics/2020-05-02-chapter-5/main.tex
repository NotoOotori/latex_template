% -------------------- Packages --------------------

\documentclass[chinese]{assignment}[2019/10/15]
\usepackage[lineno]{packages}[2019/11/14]

% -------------------- Settings --------------------

% Title

\title{Homework of Chapter 5}
\author{Chen Xuyang}
\date{\today}
\institute{School of Mathematical Science}
\professor{Shan Haiying}
\course{Combinatorics}
\subject{Combinatorics}
\keywords{}

% -------------------- New commands --------------------

\newcommand{\me}{\symup{e}}
\newcommand{\BR}{\symbb{R}}
\newcommand{\BZ}{\symbb{Z}}
\newcommand{\diag}{\mathop{}\!\symup{diag}}
\newcommand{\pr}{\mathop{}\!\symup{Pr}}
\newcommand{\expect}{\mathop{}\!\symup{E}}
\newcommand{\cov}{\mathop{}\!\symup{Cov}}
\newcommand{\var}{\mathop{}\!\symup{Var}}

\def\multiset#1#2{\ensuremath{\left(\kern-.3em\left(\genfrac{}{}{0pt}{}{#1}{#2}\right)\kern-.3em\right)}}

\newcommand{\lr}[3]{\left#1#3\right#2}
\newcommand{\lmr}[5]{\left#1#4\middle#2#5\right#3}

% -------------------- Document --------------------

\begin{document}
    \maketitle
    \begin{problem}
        求下列序列的生成函数:
        \begin{equation}
            0, 1\times 3, 2\times 4, \dotsc, n(n+2), \dotsc.
        \end{equation}
    \end{problem}
    \begin{solution}
        \begin{equation}
            \begin{aligned}
                G\{n(n+2)\}
                &=\sum_{n=0}^\infty n(n+2)x^n\\
                &= \sum_{n=0}^\infty (n+2)(n+1)x^n - \sum_{n=0}^{\infty}(n+1)x^n - \sum_{n=0}^{\infty}x^n\\
                &= \lr(){\frac{x^2}{1-x}}'' - 2\lr(){\frac{x}{1-x}}' - \frac{1}{1-x}\\
                &= \frac{x(3-x)}{(1-x)^3}
            \end{aligned}
        \end{equation}
        为生成函数的解析表达式.
    \end{solution}

    \begin{problem}
        利用生成函数计算下列和式:
        \begin{equation}
            1\times 2 + 2\times 3 + \dotsb + n(n+1)
        \end{equation}
    \end{problem}
    \begin{solution}
        设原和式为$s_n$. 考虑生成函数$G\{n(n+1)\}$, 有
        \begin{equation}
            \begin{aligned}
                G\{n(n+1)\}
                &=\sum_{n=0}^\infty n(n+1)x^n\\
                &= \sum_{n=0}^\infty (n+2)(n+1)x^n - 2\sum_{n=0}^{\infty}(n+1)x^n\\
                &= \lr(){\frac{x^2}{1-x}}'' - 2\lr(){\frac{x}{1-x}}'\\
                &= \frac{2x}{(1-x)^3}.
            \end{aligned}
        \end{equation}
        于是原和式的生成函数
        \begin{equation}
            \begin{aligned}
                G\{s_n\}
                &= \frac{G\{n(n+1)\}}{1-x}\\
                &= \frac{2x}{(1-x)^4}\\
                &= \frac{1}{3}\lr(){\frac{1}{1-x}}''' = \lr(){\frac{1}{1-x}}\\
                &= \frac{1}{3}\sum_{n=0}^\infty (n+1)(n+2)(n+3)x^n - \sum_{n=0}^\infty (n+1)(n+2)x^n
            \end{aligned}
        \end{equation}
        故
        \begin{equation}
            s_n = \frac{1}{3}(n+1)(n+2)(n+3) - (n+1)(n+2) = \frac{1}{3}n(n+1)(n+2)
        \end{equation}
        为原和式的值.
    \end{solution}

    \begin{problem}
        从$\{n\cdot a, n\cdot b, n\cdot c\}$中取出$n$个字母, 要求$a$的个数为偶数, 问有多少种取法?
    \end{problem}
    \begin{solution}
        设$r_k$表示从$\{n\cdot a, n\cdot b, n\cdot c\}$中取出$k$个字母使得$a$的个数为偶数的取法数. 记$m=\lfloor n/2\rfloor$, 计算$r_k$的生成函数, 有
        \begin{equation}
            \begin{aligned}
                G\{r_k\}
                &= (1 + x + \dotsb + x^n)^2(1 + x^2 + \dotsb + x^{2m})\\
                &= \frac{(1-x^{n+1})^2(1-x^{2m+1})}{(1-x)^2(1-x^2)}\\
                &= (1-x^{n+1})^2(1-x^{2m+1})\lr(){\sum_{k=0}^{\infty}(k+1)x^k}\lr(){\sum_{k=0}^\infty x^{2k}}.
            \end{aligned}
        \end{equation}

        因此当$n$为奇数, 即$n=2m+1$时, 有
        \begin{equation}
            r_n = 2 + 4 + \dotsb + (2m + 2) = (m+2)(m+1);
        \end{equation}
        当$n$为偶数, 即$n=2m$时, 有
        \begin{equation}
            r_n = 1 + 3 + \dotsb + (2m + 1) = (m+1)^2,
        \end{equation}
        其中$m = \lfloor n/2 \rfloor$.
    \end{solution}

    \begin{problem}
        由字母$a, b, c, d, e$组成的长为$n$的字中, 要求$a$与$b$的个数之和为偶数, 问这样的字有多少个?
    \end{problem}
    \begin{solution}
        从$\{\infty\cdot c, \infty\cdot d, \infty\cdot e\}$中取出$n-2k$个字母进行排列的排列数为$3^{n-2k}$. 记$m=\lfloor n/2\rfloor$, 记$s_n$为全部可能的字的个数, 则
        \begin{equation}
            s_n = \sum_{k=0}^{m}3^{n-2k}2^{2k}
            %= \lmr./.{3^n\lr(){1-\lr(){\frac{4}{9}}^{m+1}}}{\lr(){1-\frac{4}{9}}} = \frac{3^n\lr(){1-\lr(){4/9}^{m+1}}}{1-4/9}
            = \frac{1}{5}3^{n-2m}\left(9^{m+1}-4^{m+1}\right),
        \end{equation}
        其中$m=\lfloor n/2\rfloor$.
    \end{solution}

    \begin{problem}
        用生成函数法证明下列等式:
        \begin{equation}
            \binom{n+2}{r} - 2\binom{n+1}{r} + \binom nr = \binom{n}{r-2}.
        \end{equation}
    \end{problem}
    \begin{solution}
        将$n$看做任意固定常量, 计算等式两边序列的生成函数. 根据线性性, 等式左边序列的生成函数
        \begin{equation}
            G[\text{l.h.s.}] = (1+x)^{n+2} - 2(1+x)^{n+1} + (1+x)^n = x^2(1+x)^n.
        \end{equation}
        与等式右边序列的生成函数相同, 因此等式对任意的$r$都成立.
    \end{solution}

    \begin{problem}
        设多重集合$S=\{\infty\cdot e_1, \dotsc, \infty\cdot e_k\}$, $a_n$表示$S$满足下列条件的$n$排列数, 分别求数列$\{a_n\}$的指数型生成函数:
        \begin{enumerate}[(1)]
            \item $S$的每个元素出现奇数次;
            \item $S$的每个元素至少出现4次;
            \item $e_i$至少出现$i$次($i=1,2,\dotsc, k$);
            \item $e_i$至多出现$i$次($i=1,2,\dotsc, k$).
        \end{enumerate}
    \end{problem}
    \begin{solution}
        设$S$中每个元素出现奇数次的情况对应的指数型生成函数为$G_1$, 则
        \begin{equation}
            G_1 = \left(\frac{x}{1!} + \frac{x^3}{3!} + \dotsb\right)^k = \lr(){\frac{1}{2}(\me^x+\me^{-x})}^k.
        \end{equation}

        设$S$中每个元素至少出现4次的情况对应的指数型生成函数为$G_2$, 则
        \begin{equation}
            G_2 = \lr(){\frac{x^4}{4!}+\frac{x^5}{5!} + \dotsb}^k = \lr(){\me^x-1-x-\frac{x^2}{2!}-\frac{x^3}{3!}}^k.
        \end{equation}

        设$S$中$e_i$至少出现$i$次的情况对应的指数型生成函数为$G_3$, 则
        \begin{equation}
            G_3 = (\me^x - 1)\left(\me^x - 1 - \frac{x}{1!}\right)\dotsb\left(\me^x-1-\dotsb-\frac{x^{k-1}}{(k-1)!}\right).
        \end{equation}

        设$S$中$e_i$至多出现$i$次的情况对应的指数型生成函数为$G_4$, 则
        \begin{equation}
            G_4 = \lr(){1+\frac{x}{1!}}\lr(){1+\frac{x}{1!}+\frac{x^2}{2!}}\dotsb\lr(){1+\frac{x}{1!}+\dotsb +\frac{x^k}{k!}}.
        \end{equation}

        即为所求.
    \end{solution}

    \begin{problem}
        设有砝码重为$1\symup{g}$的3个, 重为$2\symup{g}$的4个, 重为$4\symup{g}$的2个, 问能称出多少种重量? 各有几种方案?
    \end{problem}
    \begin{solution}
        设$a_k$表示称出$k\symup{g}$的方案数, 计算$a_k$的生成函数, 有
        \begin{equation}
            \begin{aligned}
                G[a_k]
                &= (1+x)^3(1+x^2)^4(1+x^4)^2\\
                &= 1+3x+7x^2+13x^3+20x^4+28x^5+36x^6+44x^7+50x^8+54x^9+54x^{10}\\
                &\phantom{{}={}} +50x^{11}+44x^{12}+36x^{13}+28x^{14}+20x^{15}+13x^{16}+7x^{17}+3x^{18}+x^{19}
            \end{aligned}
        \end{equation}
    \end{solution}

    \begin{problem}
        如果要把棋盘中偶数个方格涂成红色, 试确定用红色, 白色和蓝色对$1\times n$棋盘的方格涂色的方法数.
    \end{problem}
    \begin{solution}
        设$m=\lfloor n/2\rfloor$, 则方法数为
        \begin{equation}
            \sum_{k = 0}^m2^{n-2k} = 2^{n+1} - 2^{n-2m} = \frac{1}{3}\lr(){2^{n+2}-2^{n-2m}},
        \end{equation}
        其中$m=\lfloor n/2\rfloor$.
    \end{solution}
\end{document}

这本书利用概型和上同调的方法介绍了抽象代数几何. 书中的主要研究对象是代数闭域上的仿射空间或射影空间上的代数簇, 为了提供一些基础概念和例子, 本书在第一章中介绍了这些内容. 随后在第二和第三章中, 分别介绍了概型和上同调的方法, 内容撰写放弃了一定的一般性而是强调了这些方法的应用. 本书的最后两章(第四, 第五章)运用这些方法研究了古典代数曲线和代数曲面论.

这种研究代数几何的方式需要读者有交换代数的基础知识, 交换代数中的内容如果需要会在书中进行陈述, 一些初等的拓知识也是需要的; 而复分析或者微分几何的知识是不必要的. 书中有超过四百道习题, 不仅提供了一些特定的例子, 也研究了一些正文当中没有提到的更加特殊的话题. 书中还有三章附录提供了一些现代研究领域的概况.

这本书可以被用来当作代数几何的介绍课程, 开在一门基础研究生代数课程之后. 我最近在Berkeley以五季度的形式教授了这本教材, 以一个差不多一季度一章的速度. 或者可以单独拿出第一张来当成一门段课程. 第三种使用这本书的方式是先学第一章, 然后再直接看第四章, 只从第二第三章中看一些概念, 并且承认曲线版本的Riemann-Roch定理成立. 这使得读者能快速接触到有趣的内容, 并且可能在他们回过头来啃第二第三章的时候能够提供一个很好的启发.

书中涉及到的内容可以为阅读一些更专业的著作提供准备.

\subsection*{致谢}

在写这本书时, 我尝试呈现一门代数几何的基础课程中的必要内容. 我想给非专业人士呈现一个现在已经高度分化, 仅有一些"流言"所联系的数学领域. 尽管我重新整理了资料并且重写了证明, 这本书依然是一个我从老师们, 同事们和学生们学到的东西的集合体. 他们在许多层面上帮助了我, 这些帮助实在是多到无法列举. 我特别感谢Oscar Zariski, J.-P. Serre, David Mumford和Arthur Ogus对我的支持和鼓励.

除了难以溯源的古典内容之外, 我最大的智库是A. Grothendieck, 他的代表作EGA是概型和上同调的权威引用资料. 他的研究成果虽然没有标注但是遍布本书的第二第三章. 除此之外的情况我尽量标注了内容的来源.

在写这本书时开设的课程中, 我分发了这本书的草稿给了很多人, 也收到了很多有价值的评论. 感谢所有这些人尤其是J.-P. Serre, H. Matsumura和Joe Lipman, 仔细阅读了草稿并且提供了非常细致的建议.

我用这本书在Harvard和Berkeley教授过课程, 我也感谢这些学生参加了课程并且提了许多有启发性的问题.

我还感谢Richard Bassein, 他运用数学家和艺术家的天分绘制了这本书中的插图.

仅仅几句话难以表达我对妻子Edie Churchill Hartshorne的感谢. 当我在专心写作的时候, 她为我和我们的儿子Jonathan和Benjamin提供了一个温馨的家, 她一直以来对我的支持和关爱为我的生活也增添了人文色彩.

关于这本书成书过程中的经济支持, 我感谢the Research Institute for Mathematical Sciences of Kyoto University, the National Science Foundation和the University of California at Berkeley.

% !TeX root = ../main.tex

\section{代数}

这一部分将从环的定义开始, 发展本文中所需要用到的代数基础, 主要包括局部化, 准素分解, 维数理论, 素谱等内容.

\subsection{环与理想}

这一小节中我们将要给出环和理想的基本概念, 以及

设$A$为非空集合, 若在$A$中有两个二元运算(加法和乘法), 满足
\begin{enumerate}
    \item $A$关于加法构成Abel群, 即
    \begin{enumerate}
        \item $A$的加法满足结合律与交换律,
        \item 存在$0\in A$, 使得对任意的$x\in A$都有$0+x=x$成立,
        \item 对于任意的$x\in A$, 存在$-x\in A$使得$x+(-x)=0$成立;
    \end{enumerate}
    \item $A$的乘法满足结合律与交换律;
    \item 存在$1\in A$, 使得对于任意的$x\in A$都有$1x=x$成立,
\end{enumerate}
则称$A$为\emph{交换幺环}. 后文中\emph{环}指的都是交换幺环. 注意如果$1=0$, 则环中有且仅有一个元素, 称这样的环为\emph{零环}.

% 研究保持代数结构的映射为代数理论的核心, 而保持环结构的映射即为环同态, 定义如下.

设$A, B$是两个环, 若映射$f\colon A\to B$满足
\begin{enumerate}
    \item $f(x+y)=f(x)+f(y)$(即$f$为群同态, 同时还有$f(0)=0$与$f(-x)=-f(x)$成立);
    \item $f(xy)=f(x)f(y)$;
    \item $f(1)=1$,
\end{enumerate}
则称$f$为从$A$到$B$的\emph{环同态}.

% 理想在环论中的作用类似于正规子群在群论中的作用, 一个环可以与其理想做商得到商环, 商环中的理想与原来环中的理想有对应关系, 可以简化问题

设$A$为环, 如果$A$的子集$\ideal{a}$满足
\begin{enumerate}
    \item $\ideal{a}$为$A$的加法子群;
    \item 若$x\in A, y\in\ideal{a}$, 则$xy\in\ideal{a}$,
\end{enumerate}
则称$\ideal{a}$为$A$的\emph{理想}.

设$B$为环$A$的子集, 则定义\emph{$B$生成的理想}为$A$中包含$B$的(在包含意义下)最小的理想, 该理想$=\{\sum_{j=1}^n x_jy_j\vert n\in\ZZ_+, x_j\in A, y_j\in B\}$. 由有限个元素$x_1, \dotsc, x_n$生成的理想记为$(x_1, \dotsc, x_n)$, 由一个元素$x$生成的理想记为$(x)$, 称可以由有限个元素生成的理想\emph{有限生成理想}, 称可以由一个元素生成的理想为\emph{主理想}.

商群$A/\ideal{a}$有从$A$诱导的自然的环结构, 其中乘法由$(x+\ideal{a})(y+\ideal{a})=xy+\ideal{a}$定义. 称环$A/\ideal{a}$为\emph{$A$对$\ideal{a}$的商环}. 有一个从环到其商环的自然的\emph{投影映射}$\pi\colon A\to A/\ideal{a}$, 满足$\pi(x)=x+\ideal{a}$, 为满环同态. 下一个命题我们将在之后经常用到.

\begin{proposition}[{{\parencite{atiyah_introduction_1969}命题1.1, P2}}]\label{prop:quotidealcorr}
    设$\ideal{a}$为环$A$的理想, 则$A$中包含$\ideal{a}$的理想全体与$A/\ideal{a}$中的理想全体有由投影映射$\pi$诱导出的保序双射.
\end{proposition}

\begin{remark}
    由于\thref{prop:quotidealcorr}的成立, 在不致引起混淆的情况下我们会将$A$中包含$\ideal{a}$的某一个理想同时看作$A/\ideal{a}$的理想而不另加说明, 反之亦然.
\end{remark}

对于环$A$的元素$x$, 如果存在$y\neq 0$使得$xy=0$, 则称$x$为\emph{零因子}. 如果非零环$A$中没有非零零因子, 则称环$A$为\emph{整环}. 对于$x\in A$, 如果存在整数$n>0$使得$x^n = 0$, 则称$x$为\emph{幂零元}. 幂零元为零因子, 反过来则不一定成立. 如果$x\in A$关于乘法可逆, 即存在$y\in A$使得$xy=1$, 则称$x$为\emph{单位}, 此时这样的$y$唯一, 记为$x^{-1}$. 环$A$的单位全体关于乘法有Abel群的结构. 如果环$A$非零, 且每个非零元都是单位, 则称$A$为\emph{域}.

如果环$A$的理想$\ideal{p}$满足如果$xy\in\ideal{p}$那么$x\in \ideal{p}$或者$y\in\ideal{p}$成立, 则称$\ideal{p}$为\emph{素理想}; 如果环$A$的理想$\ideal{m}$为真包含于$A$的理想全体(在包含意义下)的极大元, 则称$\ideal{m}$为\emph{极大理想}. 有等价定义
\begin{equation}
    \begin{aligned}
        \ideal{p}\text{为素理想}&\Leftrightarrow A/\ideal{p}\text{是整环};\\
        \ideal{m}\text{为极大理想}&\Leftrightarrow A/\ideal{m}\text{是域}.
    \end{aligned}
\end{equation}
因此素理想为极大理想, 而下面一个著名的Zorn引理的引用保证了非零环中极大理想的存在性.

\begin{theorem}[{{\parencite{atiyah_introduction_1969}定理1.3, P3}}]\label{thm:maxideal}
    每个非零环都存在至少一个极大理想.
\end{theorem}

\begin{proofsketch}
    真包含于$A$的理想全体构成偏序集结构, 证明对于任意的全序子链$\{\ideal{a}_\lambda\}$, 都有$\bigcup_{\lambda}\ideal{a}_\lambda$是真包含于$A$的理想, 为该全序子链的上界, 故由Zorn引理可得该偏序集中存在极大元, 即环$A$中存在极大理想.
\end{proofsketch}

接下来介绍一些理想的运算. 两个理想$\ideal{a}, \ideal{b}$的\emph{和}$\ideal{a}+\ideal{b}=\{x+y\vert x\in\ideal{a}, y\in\ideal{b}\}$是理想. 这是包含$\ideal{a}$和$\ideal{b}$的最小的理想. 两个理想$\ideal{a}, \ideal{b}$的\emph{积}$\ideal{a}\ideal{b}=\{\sum_{j=1}^n x_jy_j\vert n\in\ZZ_+, x_j\in\ideal{a}, y_j\in\ideal{b}\}$是理想. 两个理想$\ideal{a}, \ideal{b}$的\emph{交}$\ideal{a}\cap\ideal{b}$是理想. 定义两个理想$\ideal{a}, \ideal{b}$的\emph{理想商}为$(\ideal{a}{:}\ideal{b}):=\{x\in A\vert xb\subseteq a\}$, 也是一个理想, 这将在\ref{subsec:primdecom}有重要的应用. 一个理想$\ideal{a}$的\emph{根基}定义为$\sqrt{\ideal{a}}:=\{x\in A\vert \exists n\in\ZZ_+\colon x^n\in\ideal{a}\}$, 也是一个理想.

理想在环同态作用下的表现也是研究的重点. 给定环同态$f\colon A\to B$, 设$\ideal{a}$为$A$的理想, $\ideal{b}$为$B$的理想. $\ideal{a}$在$f$下的像$f(\ideal{a})$不一定为理想, 定义$\ideal{a}$的\emph{扩张}为$f(\ideal{a})$在$B$中生成的理想, 记作$\ideal{a}^e$. $\ideal{b}$在$f$下的原像$f^{-1}(\ideal{b})$一定是理想, 定义$\ideal{b}$的\emph{收缩}即为$f^{-1}(\ideal{b})$, 记作$\ideal{b}^c$. 若$B$的某个理想可以写成$A$的一个理想的扩张, 则称这个$B$的理想为\emph{扩张理想}, 类似地, 如果$A$的某个理想可以写成$B$的一个理想的收缩, 则称这个$A$的理想为\emph{收缩理想}. 要注意扩张与收缩是相对于$f$而言的, 只不过一般在语境下$f$的选取不会有混淆, 故没有显式提及$f$.

\begin{proposition}[{{\parencite{atiyah_introduction_1969}命题1.17, P10}}]\label{prop:extencontr}
    令$A, B, f, \ideal{a}, \ideal{b}$与上一段的定义相同, 则
    \begin{enumerate}
        \item $\ideal{a}\subseteq\ideal{a}^{ec}, \ideal{b}\supseteq\ideal{b}^{ce}$,
        \item $\ideal{a}=\ideal{a}^{ece}, \ideal{b}=\ideal{b}^{cec}$,
        \item $A$的收缩理想全体与$B$的扩张理想全体有保序的一一对应关系.
    \end{enumerate}
\end{proposition}

\subsection{分式环}



\subsection{准素分解}\label{subsec:primdecom}

% !TeX root = ../ag.tex

\chapter{Varieties}

In this chapter, a variety always means a quasi-projective variety.

\section{Projective Varieties}

\subsection{Image of Regular Maps}

\begin{theorem}
  Let $Y$ be a projective variety and $X$ be any variety. Then the projection  $\pi\colon X\times Y\to X$ is a closed map.
\end{theorem}

\begin{corollary}
  Let $X$ be a projective variety. Then any regular map $f\colon X\to \PP^n$ is closed.
\end{corollary}

\begin{corollary}
  A regular function on a projective variety is constant.
\end{corollary}

\begin{corollary}
  Let $X\subseteq \PP^n$ be a connected projective variety and $X$ is not a point. Then any hypersurface $H$ of $\PP^n$ intersects $X$.
\end{corollary}

\section{Dimension}

\subsection{Dimension and Defining Equations}

\begin{theorem}
  Suppose that $X$ is a quasi-projective variety of $\PP^N$. Let $F_1, \dotsc, F_r$ be homogeneous polynomials.
  \begin{enumerate}
    \item If $X$ is of pure dimension $n$, then each nonempty components of $X\cap V(F_1, \dotsc, F_r)$ has dimension greater than or equal to $n-r$.
    \item If $X$ is projective of dimension $n$ and $r\leq n$, then $X\cap V(F_1, \dotsc, F_r)$ is nonempty.
  \end{enumerate}
\end{theorem}

The key is the Krull's Hauptidealsatz.

\begin{corollary}
    The hypersurfaces of $\AAA^n$ (resp. $\PP^n$) are the subvarieties of pure dimension $n-1$.
\end{corollary}

\subsection{Generally Finite Maps}

\begin{theorem}
  Let $X$ and $Y$ be irreducible varieties and $f\colon X\to Y$ be a dominant morphism. Then a general fiber of $f$ is finite if and only if $f^*\colon K(Y)\to K(X)$ induces a finite extension of fields.

  Moreover, we have $\DIM X=\DIM Y$, and for a general $y\in Y$, we have $\#f^{-1}(y)\leq [K(X):K(Y)]$, where the equality holds if $K(X):K(Y)$ is separable.
\end{theorem}

\begin{corollary}
    Let $X$ be an irreducible variety of dimension $n$. Then there exists a dominant morphism $f\colon X\to \PP^n$ with finite fibers.
\end{corollary}

\begin{corollary}
    Let $X, Y$ be varieties. Then $\DIM(X\times Y)=\DIM(X)+\DIM(Y)$.
\end{corollary}

\subsection{Morphisms and Dimension}

Let $\varphi\colon X\to Y$ be a morphism between varieties. For every $x\in X$, write $X_x\coloneq \varphi^{-1}(\varphi(x))$ the fiber of $\varphi$ containing $x$.

\begin{theorem}
  Let $\varphi\colon X\to \PP^n$ be a morphism. Then the function $\delta\colon x\in X\mapsto \DIM_x (X_x)\in \NN$ is upper semi-continuous, i.e. the set $X(r)\coloneq \DIM_x (X_x)\geq r$ is closed. Moreover, if $X$ is irreducible, then
  \begin{equation*}
    \DIM (X)=\DIM (\overline{\varphi(X)}) + \MIN_{x\in X}\delta(x).
  \end{equation*}
\end{theorem}

\begin{corollary}
    Let $X$ be an irreducible variety, and let $\varphi\colon X\to \PP^n$ be a morphism with $Y=\overline{\varphi(X)}$. Then
    \begin{enumerate}
      \item for any $y\in \varphi(X)$, any component of $\varphi^{-1}(y)$ has dimension no less than $\DIM(X) - \DIM(Y)$;
      \item for a general point $y\in Y$, $\varphi^{-1}(y)$ is of pure dimension $\DIM(X)-\DIM(Y)$.
    \end{enumerate}
\end{corollary}

Let $X'$ be an irreducible component of $\varphi^{-1}(y)$. If $x\in X'$ such that $x$ is not contained in any other irreducible component of $\varphi^{-1}(y)$, then $\DIM_x(X_x)=\MIN_{x\in X} \delta(x)$.

\begin{corollary}
    Let $\varphi\colon X\to Y$ be a closed morphism between varieties. Then the function $y\mapsto \DIM(\varphi^{-1}(Y))$ is upper semi-continuous.
\end{corollary}

\begin{proposition}
    Let $\varphi\colon X\to Y$ be a closed surjective morphism. Assume that $Y$ is irreducible and all fibers of $\varphi$ are irreducible of dimension $r$. Then $X$ is irreducible of dimension $\DIM(Y)+r$.
\end{proposition}

This is an important proposition to show a variety is irreducible.

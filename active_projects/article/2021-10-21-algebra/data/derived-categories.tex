% !TeX root = ../algebra.tex

\chapter{Derived Categories}

In this chapter all complexes are assumed to be cochain ones, and $\CHX$ denotes the category of cochain complexes. Chain maps are referred as morphisms between chain complexes.

Recall that if $X$ is a complex, then the $p$\textsuperscript{th} \emph{translate} $X[p]$ is a complex with degree $n$ part to be $X[p]^n=X^{n+p}$, and with differential $d_{X[p]}^n=(-1)^pd_X^{n+p}$. The sign convention is designed to simplify notation later on. Also for a chain map $f\colon X\to Y$, its \emph{translate} $f[p]$ is a chain map $X[p]\to Y[p]$ given by $f[p]^n=f^{n+p}$.

\section{Mapping Cones and Cylinders}

Let $\CAT C$ be an additive category, and let $f\colon X\to Y$ be a morphism in $\CHX(\CAT C)$. The \emph{mapping cone} $\CONE(f)$ of $f$ is defined as a complex whose degree $n$ part is $Y^n\oplus X^{n+1}$ and whose differential is given by the matrix
\begin{equation*}
  \begin{tikzcd}[ampersand replacement=\&,column sep=large,]
    Y^n\oplus X^{n+1} \& Y^{n+1}\oplus X^{n+2}.
    \arrow[from=1-1,]{1-2}[]{\begin{BSM}d^n_Y&-f^{n+1}\\0&-d^{n+1}_X\end{BSM}}
  \end{tikzcd}
\end{equation*}
% \begin{equation*}
%   Y^n\oplus X^{n+1}\xrightarrow{\begin{BSM}d^n_X&-f^{n+1}\\0&-d^{n+1}_Y\end{BSM}}Y^{n+1}\oplus X^{n+2}.
% \end{equation*}
% There is a short exact sequence
% \begin{equation*}
%   \begin{tikzcd}[ampersand replacement=\&,]
%     0 \& Y \& \CONE(f) \& X[1] \& 0,
%     \arrow[from=1-1,]{1-2}[]{}
%     \arrow[from=1-2,]{1-3}[]{\begin{BSM}1\\0\end{BSM}}[swap,]{\alpha(f)}
%     \arrow[from=1-3,]{1-4}[]{\begin{BSM}0&-1\end{BSM}}[swap,]{\beta(f)}
%     \arrow[from=1-4,]{1-5}[]{}
%   \end{tikzcd}
% \end{equation*}
% where the nonzero chain maps are denoted by $\alpha(f)$ and $\beta(f)$ respectively.

One may also construct the \emph{mapping cylinder} $\CYL(f)$ from a chain map $f\colon X\to Y$. The degree $n$ part of $\CYL(f)$ is $X^n\oplus Y^n\oplus X^{n+1}$, and the differential is
\begin{equation*}
  \begin{tikzcd}[ampersand replacement=\&,column sep=huge,]
    X^n\oplus Y^n\oplus X^{n+1} \& X^{n+1}\oplus Y^{n+1}\oplus X^{n+2}.
    \arrow[from=1-1,]{1-2}[]{\begin{BSM}d^n_X&0&1\\0&d^n_Y&-f^{n+1}\\0&0&-d^{n+1}_X\end{BSM}}
  \end{tikzcd}
\end{equation*}
It's routine to verify that the cylinder is a complex, and that it is actually the cone of $\beta(f)[-1]\colon \CONE(f)[-1]\to X$.

The main reason of studying mapping cones and cylinders is shown in the following commutative diagram with exact rows,
\begin{equation*}
  \begin{tikzcd}[ampersand replacement=\&,column sep=large,row sep=huge]
    \& 0 \& Y \& \CONE(f) \& X[1] \& 0\\
    0 \& X \& \CYL(f) \& \CONE(f)\& 0\\
    0 \& X \& Y \& Z \& 0
    \arrow[from=1-2,]{1-3}[]{}
    \arrow[from=1-3,]{1-4}[]{\begin{BSM}1\\0\end{BSM}}[swap,]{\alpha(f)}
    \arrow[from=1-4,]{1-5}[]{\begin{BSM}0&-1\end{BSM}}[swap,]{\beta(f)}
    \arrow[from=1-5,]{1-6}[]{}
    \arrow[from=2-1,]{2-2}[]{}
    \arrow[from=2-2,]{2-3}[]{\begin{BSM}1\\0\\0\end{BSM}}
    \arrow[from=2-3,]{2-4}[]{\begin{BSM}0&1&0\\0&0&1\end{BSM}}
    \arrow[from=2-4,]{2-5}[]{}
    \arrow[from=3-1,]{3-2}[]{}
    \arrow[from=3-2,]{3-3}[]{f}
    \arrow[from=3-3,]{3-4}[]{g}
    \arrow[from=3-4,]{3-5}[]{}
    \arrow[from=1-3,]{2-3}[]{\begin{BSM}0\\1\\0\end{BSM}}[swap,]{\mu(f)}
    \arrow[from=1-4,equal,]{2-4}[]{}
    \arrow[from=2-2,equal,]{3-2}[]{}
    \arrow[from=2-3,]{3-3}[]{\begin{BSM}f&1&0\end{BSM}}[swap,]{\nu(f)}
    \arrow[from=2-4,]{3-4}[]{\begin{BSM}g&0\end{BSM}}[swap,]{\varphi}
  \end{tikzcd}
\end{equation*}
where
$\begin{tikzcd}[cramped, sep=small]
  0 & X & Y & Z & 0
  \arrow[from=1-1,]{1-2}[]{}
  \arrow[from=1-2,]{1-3}[]{f}
  \arrow[from=1-3,]{1-4}[]{g}
  \arrow[from=1-4,]{1-5}[]{}
\end{tikzcd}$
is an arbitrary short exact sequence. Observe that $\nu(f)\mu(f)=\ID_Y$, and that a chain homotopy equivalence $\mu(f)\nu(f)$ and $\ID_{\CYL(f)}$ is given by the matrix
\begin{equation*}
  \begin{tikzcd}[ampersand replacement=\&,column sep=huge,]
    X^{n}\oplus Y^{n}\oplus X^{n+1} \& X^{n-1}\oplus Y^{n-1}\oplus X^{n}.
    \arrow[from=1-1,]{1-2}[]{\begin{BSM}0&0&0\\0&0&0\\1&0&0\end{BSM}}
  \end{tikzcd}
\end{equation*}
This gives a pair of chain homotopy equivalences between $Y$ and $\CYL(Y)$. Moreover, since $\nu(f)$ is a quasi-isomorphism, it follows from the five lemma that $\varphi$ is also a quasi-isomorphism (though it is not necessarily a chain homotopy equivalence (Weibel p. 23)).

\section{Triangulated Categories}

definition
\begin{enumerate}[label={(TR \arabic*)},start=0,labelindent=!,]
  \item every triangle isomorphic to a distinguished triangle is itself a distinguished triangle;
  \item 1
  \item 2
  \item 3
  \item 4
  \item 5
\end{enumerate}
over

(TR 3)
\begin{equation*}
  \begin{tikzcd}[ampersand replacement=\&,row sep=large]
    Y \& \CONE(f) \& X[1] \& Y[1]\\
    Y \& \CONE(f) \& \CONE(\alpha(f)) \& Y[1]
    \arrow[from=1-1,]{1-2}[]{\begin{BSM}1\\0\end{BSM}}
    \arrow[from=1-2,]{1-3}[]{\begin{BSM}0&-1\end{BSM}}
    \arrow[from=1-3,]{1-4}[]{-f[1]}
    \arrow[from=2-1,]{2-2}[]{\begin{BSM}1\\0\end{BSM}}
    \arrow[from=2-2,]{2-3}[]{\begin{BSM}1&0\\0&1\\0&0\end{BSM}}
    \arrow[from=2-3,]{2-4}[]{\begin{BSM}0&0&-1\end{BSM}}
    \arrow[from=1-1,equal]{2-1}[]{}
    \arrow[from=1-2,equal]{2-2}[]{}
    \arrow[from=1-3,]{2-3}[]{\begin{BSM}0\\-1\\f\end{BSM}}
    \arrow[from=1-4,equal]{2-4}[]{}
  \end{tikzcd}
\end{equation*}

differential of $\CONE(\alpha(f))$: $\begin{BSM}d&-f&-1\\0&-d&0\\0&0&-d\end{BSM}$. the second square commutes only in homotopy category, where the chain homotopy is given by $\begin{BSM}0&0\\0&0\\1&0\end{BSM}$

% !TeX root = ../algebra.tex

\chapter{Homological Algebra}

Three important lemmas: long exact sequence, comparison lemma, horseshoe lemma

\begin{itemize}\color{red}
  \item hom complex, tensor complex, and 2-complex
  \item mapping cone motivation (P273 Munkres); mapping cylinder
  \item Universal Coefficient Theorems
  \item Koszul complex, Hilbert's Syzygy
  \item Dimension (projective dimension and global projective dimension)
  \item $\{a\vert b\}$ typography
\end{itemize}

Size issue --- universe (Gabriel)

\section{Basic Concepts}

\subsection{Abelian Categories}

A \emph{pointed category} is a category with a zero object, i.e. an object 0 which is both an initial object and a terminal object. In a pointed category, one usually denote $0_{0, B}$ to be the unique morphism $0\to B$, $0_{A, 0}$ the unique morphism $A\to 0$, and $0_{A, B}$ the composition $0_{0, B}0_{A, 0}$.

An \emph{$\AB$-enriched category} is a locally small category such that every hom-set is equipped with the structrue of an abelian group; and such that the composition operation is bilinear. In an $\AB$-enriched category, a initial object is also a terminal object, and any finite product is also a coproduct, and dually.

A \emph{pre-additive category} is an pointed $\AB$-enriched category. One can prove that $0_{A,B}$ is the zero element of the abelian group $\HOM(A, B)$.

An \emph{additive category} is a pre-additive category which admits finite biproducts.

A \emph{pre-abelian category} is an additive category such that every morphism has a kernel and a cokernel. Equivalently, a pre-abelian category is an additive category which admits all finite limits and finite colimits. In a pre-abelian category, every morphism $f\colon A\to B$ induces a morphism $\BAR{f}\colon\COKER\KER f\to \KER\COKER f$ as shown in the diagram.
\begin{equation*}
  \begin{tikzcd}
    0 & \KER f & A & B & \COKER f & 0\\
    & & \COKER\KER f & \KER\COKER f
    \arrow[from=1-1,]{1-2}[]{}
    \arrow[from=1-2,]{1-3}[]{}
    \arrow[from=1-3,]{1-4}[]{f}
    \arrow[from=1-4,]{1-5}[]{}
    \arrow[from=1-5,]{1-6}[]{}
    \arrow[from=1-3,two heads,]{2-3}[]{}
    \arrow[from=1-4,hookleftarrow,]{2-4}[]{}
    \arrow[from=2-3,dashed,]{1-4}[]{}
    \arrow[from=2-3,dashed,]{2-4}[]{\BAR{f}}
  \end{tikzcd}
\end{equation*}

\begin{proposition}
  Every morphism $f\colon A\to B$ in a pre-abelian category has a canonical decomposition
  \begin{equation*}
    \begin{tikzcd}[sep=small]
      A & \COKER\KER f & \KER\COKER f & B
      \arrow[from=1-1,]{1-2}[]{p}
      \arrow[from=1-2,]{1-3}[]{\BAR{f}}
      \arrow[from=1-3,]{1-4}[]{i}
    \end{tikzcd}
  \end{equation*}
  where $p$ is a cokernel, hence epic, and $i$ is a kernel, hence monic.
\end{proposition}

An \emph{abelian category} is a pre-abelian category such that for every morphism $f$, the induced map $\BAR{f}$ is an isomorphism. Equivalently, an abelian category is a pre-abelian category such that every monomorphism is the cokernel of its kernel, and that every epimorphism is the kernel of its cokernel.

\begin{remark}
  The dual of an additive (abelian, resp.) category is additive (abelian, resp).
\end{remark}

\begin{example}
  The category $\CH(\CAT{C})$ of chain complexes over an abelian category $\CAT{C}$ is also an abelian category, with projective objects to be split exact complexes of projectives.
\end{example}

\begin{example}
  Let $\CAT{C}$ be an abelian category and $\CAT{J}$ a category. Then the functor category $\CAT{C}^{\CAT{J}}$ is an abelian category with finite biproducts, kernels and cokernels defined element-wise. If $\CAT{C}$ is cocomplete (complete, resp.) and has enough projectives (injectives, resp.), then $\CAT{C}^{\CAT{J}}$ has enough projectives (injectives, resp.).

  For example, let $\CAT{J}$ be the arrow category $1\to 2$. Then projective objects of $\CAT{C}^{\CAT{J}}$ are split monic morphisms $f\colon U\to V$ between projectives.
\end{example}

\begin{proposition}[Gabriel]
  Let $\CAT{C}$ be an abelian category and $\CAT{D}$ an abelian category with exact filtered colimits. Then the full subcategory of left exact functors of the functor category $\CAT{D}^{\CAT{C}}$ is indeed a reflexive subcategory, and hence is an abelian category.
\end{proposition}

\begin{theorem}[Freyd-Mitchell Embedding Theorem]
  Every small abelian category admits a full, faithful and exact functor to the category $\MODR$ for some ring $R$.
\end{theorem}

The Freyd-Mitchell Embedding Theorem implies that proofs about exactness of sequences in an abelian category can always be obtained by a diagram chase.

\subsection{Homology and Diagram Chasing Lemmas}

A \emph{split chain complex} is a chain complex with a map $s$ of degree 1 such that $dsd=d$. A \emph{split exact chain complex} is a split chain complex that is exact at each term.

\begin{proposition}\hspace*{\fill}
  \begin{enumerate}
    \item A chain complex $C_{\CCCIR}$ splits if and only if there are decompositions $C_n\cong Z_n\oplus B_n'$ and $Z_n\cong B_n\oplus H_n'$ for each $n$.
    \item A split chain complex $C_{\CCCIR}$ is exact if and only if $H_n'=0$.
  \end{enumerate}
\end{proposition}

\begin{proposition}\hspace*{\fill}
  \begin{enumerate}
    \item A bounded-below acyclic complex of free $R$-modules is split exact.
    \item An acyclic complex of finitely generated free abelian groups is split exact.
  \end{enumerate}
\end{proposition}

\begin{proposition}
  A chain complex is split exact if and only if the identity chain map is null-homotopic. Therefore, additive functors sends a split exact chain complex to a split exact chain complex.
\end{proposition}

The homology in degree $n$ defines a functor $H_n\colon\CH(\MODR)\to\MODR$.

\begin{proposition}
  $H_n$ commutes with exact functors. In particular, $H_n$ commutes with direct limits.
\end{proposition}

\begin{proof}
  Consider the exact sequence $0\to B_n\to Z_n\to H_n\to 0$ and recall that an exact functor preserves all finite limits and finite colimits.
\end{proof}

Let $C_\CCCIR$ be a complex. Then $\bigoplus C_n$ can be viewed as a graded $R$-module with a square-zero degree $+1$ endomorphism, or as an $R$-module with a square-zero endomorphism $d$ given by $(x_n)_n\mapsto (d_{n+1}(x_{n+1}))_n$.

positive chain complex; negative chain complex

exactness of a sequence of chain complexes

\begin{lemma}[Long Exact Sequence]
  Let $0\to C_\CCCIR'\xrightarrow{i} C_\CCCIR \xrightarrow{p} C_\CCCIR''\to 0$ be an exact sequence in $\CH(\CAT{C})$ for an abelian category $\CAT{C}$, then there is a long exact sequence in $\CAT{C}$.
  \begin{equation*}
    \begin{tikzcd}[sep=small]
      \dotsb & H_{n+1}(C_\CCCIR'') & H_n(C_\CCCIR') & H_n(C_\CCCIR) & H_n(C_\CCCIR'') & H_{n-1}(C_\CCCIR') & \dotsb
      \arrow[from=1-1,]{1-2}[]{}
      \arrow[from=1-2,]{1-3}[]{\partial_{n+1}}
      \arrow[from=1-3,]{1-4}[]{i_*}
      \arrow[from=1-4,]{1-5}[]{p_*}
      \arrow[from=1-5,]{1-6}[]{\partial_n}
      \arrow[from=1-6,]{1-7}[]{}
    \end{tikzcd}
  \end{equation*}
  Moreover, the long exact sequence is natural.
\end{lemma}

\begin{proof}
  For naturality, let $\CAT{J}$ denote the category $\BCIR\to\BCIR$. Then the functor category $\CAT{C}^{\CAT{J}}$ is also an abelian category. By applying the long exact sequence to the category $\CAT{C}^{\CAT{J}}$ one can obtain naturality of the long exact sequence.
\end{proof}

\begin{lemma}[Snake Lemma]
  \begin{equation*}
    \begin{tikzcd}[row sep=large]
      0 & M' & M & M'' & 0\\
      0 & N' & N & N'' & 0
      \arrow[from=1-1,dashed,]{1-2}[]{}
      \arrow[from=1-2,]{1-3}[]{}
      \arrow[from=1-3,]{1-4}[]{}
      \arrow[from=1-4,]{1-5}[]{}
      \arrow[from=2-1,]{2-2}[]{}
      \arrow[from=2-2,]{2-3}[]{}
      \arrow[from=2-3,]{2-4}[]{}
      \arrow[from=2-4,dashed,]{2-5}[]{}
      \arrow[from=1-2,]{2-2}[swap]{f'}
      \arrow[from=1-3,]{2-3}[]{f}
      \arrow[from=1-4,]{2-4}[]{f''}
    \end{tikzcd}
  \end{equation*}
  \begin{equation*}
    \begin{tikzcd}[sep=small, cramped]
      0 & \KER(f') & \KER(f) & \KER(f'') & \COKER(f') & \COKER(f) & \COKER(f'') & 0
      \arrow[from=1-1,dashed,]{1-2}[]{}
      \arrow[from=1-2,]{1-3}[]{}
      \arrow[from=1-3,]{1-4}[]{}
      \arrow[from=1-4,]{1-5}[]{\partial}
      \arrow[from=1-5,]{1-6}[]{}
      \arrow[from=1-6,]{1-7}[]{}
      \arrow[from=1-7,dashed,]{1-8}[]{}
    \end{tikzcd}
  \end{equation*}
\end{lemma}

\begin{lemma}[Five Lemma]
  Given a commutative diagram with exact rows.
  \begin{equation*}
    \begin{tikzcd}[row sep=large]
      A_1 & A_2 & A_3 & A_4 & A_5\\
      B_1 & B_2 & B_3 & B_4 & B_5
      \arrow[from=1-1,]{1-2}[]{}
      \arrow[from=1-2,]{1-3}[]{}
      \arrow[from=1-3,]{1-4}[]{}
      \arrow[from=1-4,]{1-5}[]{}
      \arrow[from=2-1,]{2-2}[]{}
      \arrow[from=2-2,]{2-3}[]{}
      \arrow[from=2-3,]{2-4}[]{}
      \arrow[from=2-4,]{2-5}[]{}
      \arrow[from=1-1,]{2-1}[swap]{h_1}
      \arrow[from=1-2,]{2-2}[swap]{h_2}
      \arrow[from=1-3,]{2-3}[]{h_3}
      \arrow[from=1-4,]{2-4}[]{h_4}
      \arrow[from=1-5,]{2-5}[]{h_5}
    \end{tikzcd}
  \end{equation*}
  \begin{enumerate}
    \item If $h_2, h_4$ are epic and $h_5$ is monic, then $h_3$ is epic.
    \item If $h_2, h_4$ are monic and $h_1$ is epic, then $h_3$ is monic.
  \end{enumerate}
\end{lemma}

One can prove the Snake Lemma, the Five Lemma and many other lemmas in homological algebra in an elementary way by applying the Freyd-Mitchel embedding theorem and then by proceeding a diagram chase. Or one can prove the five lemma and the long exact sequence by the snake lemma. A more illuminating approach is the salamander lemma.

Further Reading: \href{https://ncatlab.org/nlab/show/salamander+lemma}{The Salamander Lemma}, \url{https://math.stackexchange.com/a/2511909}, \url{https://math.stackexchange.com/a/74872}.

(Motivation: homotopic maps should define the same map of homology)

homotopy: $s$ of degree $+1$

homotopy defines an equivalence relation on chain maps.

homotopy category

\subsection{Derived Functors}

The idea is to replace every module by a deleted resolution of it; given a short exact sequence of modules, we shall see that this replacement gives a short exact sequence of complexes. (Rotman)

A \emph{projective resolution} $P_\CCCIR$ of a module $M$ is a long exact sequence,
\begin{equation*}
  \begin{tikzcd}[sep=small]
    \dotsb & P_2 & P_1 & P_0 & M & 0
    \arrow[from=1-1,]{1-2}[]{}
    \arrow[from=1-2,]{1-3}[]{}
    \arrow[from=1-3,]{1-4}[]{}
    \arrow[from=1-4,]{1-5}[]{\varepsilon}
    \arrow[from=1-5,]{1-6}[]{}
  \end{tikzcd}
\end{equation*}
where each module $P_n$ is projective. Its \emph{deleted resolution} is the complex deleting $M$,
\begin{equation*}
  \begin{tikzcd}[sep=small]
    \dotsb & P_2 & P_1 & P_0 & 0
    \arrow[from=1-1,]{1-2}[]{}
    \arrow[from=1-2,]{1-3}[]{}
    \arrow[from=1-3,]{1-4}[]{}
    \arrow[from=1-4,]{1-5}[]{}
  \end{tikzcd}
\end{equation*}
denoted by $P_M$. A projective resolution of $M$ can be viewed as a chain map from a projective chain complex $P_\CCCIR$ to $M$, which is a weak equivalence. The initial segment of a free resolution,
\begin{equation*}
  \begin{tikzcd}[sep=small]
    F_1 & F_0 & M & 0
    \arrow[from=1-1,]{1-2}[]{}
    \arrow[from=1-2,]{1-3}[]{\varepsilon}
    \arrow[from=1-3,]{1-4}[]{}
  \end{tikzcd}
\end{equation*}
is simply a presentation of $M$ by generators and relations.

\begin{example}
  If $R$ is a principal ideal domain, then any $R$-module $M$ has a length 1 projective resolution. This is the key fact behind the classifying theorem of finitely generated modules over a principal ideal domain.
\end{example}

Dually, an \emph{injective resolution} of a module $N$ is a long exact sequence,
\begin{equation*}
  \begin{tikzcd}[sep=small]
    0 & N & Q_0 & Q_1 & Q_2 & \dotsb
    \arrow[from=1-1,]{1-2}[]{}
    \arrow[from=1-2,]{1-3}[]{\eta}
    \arrow[from=1-3,]{1-4}[]{}
    \arrow[from=1-4,]{1-5}[]{}
    \arrow[from=1-5,]{1-6}[]{}
  \end{tikzcd}
\end{equation*}
where each module $Q_n$ is injective.

Usually, projective resolutions are easier to compute, while injective resolutions occur more naturally. (not enough projectives in the category of sheaves for some topological space)

\begin{lemma}[Comparison Lemma]
  Let $P_\CCCIR$ be a projective complex over a module $M$ and $P_\CCCIR'$ be a resolution over a module $M'$. Suppose there is a map $f\colon M\to M'$, then there is a chain map $\check f\colon P_\CCCIR\to P_\CCCIR'$ making the following diagram commutes.
  \begin{equation*}
    \begin{tikzcd}[row sep=large]
      \dotsb & P_2 & P_1 & P_0 & M & 0\\
      \dotsb & P_2' & P_1' & P_0' & M' & 0
      \arrow[from=1-1,]{1-2}[]{}
      \arrow[from=1-2,]{1-3}[]{}
      \arrow[from=1-3,]{1-4}[]{}
      \arrow[from=1-4,]{1-5}[]{\varepsilon}
      \arrow[from=1-5,]{1-6}[]{}
      \arrow[from=2-1,]{2-2}[]{}
      \arrow[from=2-2,]{2-3}[]{}
      \arrow[from=2-3,]{2-4}[]{}
      \arrow[from=2-4,]{2-5}[]{\varepsilon'}
      \arrow[from=2-5,]{2-6}[]{}
      \arrow[from=1-2,dashed,]{2-2}[]{\check{f}_2}
      \arrow[from=1-3,dashed,]{2-3}[]{\check{f}_1}
      \arrow[from=1-4,dashed,]{2-4}[]{\check{f}_0}
      \arrow[from=1-5,]{2-5}[]{f}
    \end{tikzcd}
  \end{equation*}
  Moreover, any two such chain maps are equivalent under homotopy.
\end{lemma}

\begin{proof}
  We need frequently the following equivalent condition for a module to be projective: A module $P$ is projective if and only if for any module $M$, $N$ and any map $M\to N$, whenever there is a map $g\colon P\to N$ such that $\IM(g)\subseteq\IM(f)$, then $g$ can be lifted to a map $\tilde g\colon P\to M$ (not necessarily unique) such that $g=f\tilde g$.
\end{proof}

The dual of comparison lemma is also true, which is stated as follows.

\begin{lemma}[Comparison Lemma]
  Let $Q_\CCCIR$ be an injective complex under a module $N$ and $Q_\CCCIR'$ be a resolution under a module $N'$. Suppose there is a map $g\colon N'\to N$, then there is a chain map $\check g\colon Q_\CCCIR'\to Q_\CCCIR$ making the following diagram commutes.
  \begin{equation*}
    \begin{tikzcd}[row sep=large]
      \dotsb & Q_2 & Q_1 & Q_0 & N & 0\\
      \dotsb & Q_2' & Q_1' & Q_0' & N' & 0
      \arrow[from=1-1,leftarrow]{1-2}[]{}
      \arrow[from=1-2,leftarrow]{1-3}[]{}
      \arrow[from=1-3,leftarrow]{1-4}[]{}
      \arrow[from=1-4,leftarrow]{1-5}[]{\eta}
      \arrow[from=1-5,leftarrow]{1-6}[]{}
      \arrow[from=2-1,leftarrow]{2-2}[]{}
      \arrow[from=2-2,leftarrow]{2-3}[]{}
      \arrow[from=2-3,leftarrow]{2-4}[]{}
      \arrow[from=2-4,leftarrow]{2-5}[]{\eta'}
      \arrow[from=2-5,leftarrow]{2-6}[]{}
      \arrow[from=1-2,dashed,leftarrow]{2-2}[]{\check{g}_2}
      \arrow[from=1-3,dashed,leftarrow]{2-3}[]{\check{g}_1}
      \arrow[from=1-4,dashed,leftarrow]{2-4}[]{\check{g}_0}
      \arrow[from=1-5,leftarrow]{2-5}[]{g}
    \end{tikzcd}
  \end{equation*}
\end{lemma}

\begin{lemma}[Horseshoe Lemma]
  \begin{equation*}
    \begin{tikzcd}[row sep=large]
      & \vdots & \color{cyan}\vdots & \vdots\\
      0 & P_1' & \color{cyan}P_1 & P_1'' & 0\\
      0 & P_0' & \color{cyan}P_0 & P_0'' & 0\\
      0 & M' & M & M'' & 0\\
      & 0 & 0 & 0
      \arrow[from=4-1,]{4-2}[]{}
      \arrow[from=4-2,]{4-3}[]{}
      \arrow[from=4-3,]{4-4}[]{}
      \arrow[from=4-4,]{4-5}[]{}
      \arrow[from=1-2,]{2-2}[]{}
      \arrow[from=2-2,]{3-2}[]{}
      \arrow[from=3-2,]{4-2}[]{}
      \arrow[from=4-2,]{5-2}[]{}
      \arrow[from=1-4,]{2-4}[]{}
      \arrow[from=2-4,]{3-4}[]{}
      \arrow[from=3-4,]{4-4}[]{}
      \arrow[from=4-4,]{5-4}[]{}
      \arrow[from=1-3,cyan,]{2-3}[]{}
      \arrow[from=2-3,cyan,]{3-3}[]{}
      \arrow[from=3-3,cyan,]{4-3}[]{}
      \arrow[from=4-3,]{5-3}[]{}
      \arrow[from=3-1,]{3-2}[]{}
      \arrow[from=3-2,cyan,]{3-3}[]{}
      \arrow[from=3-3,cyan,]{3-4}[]{}
      \arrow[from=3-4,]{3-5}[]{}
      \arrow[from=2-1,]{2-2}[]{}
      \arrow[from=2-2,cyan,]{2-3}[]{}
      \arrow[from=2-3,cyan,]{2-4}[]{}
      \arrow[from=2-4,]{2-5}[]{}
    \end{tikzcd}
  \end{equation*}
\end{lemma}

\begin{definition}
  Let $F\colon \CAT{C}\to \CAT{D}$ be an additive functor between abelian categories where $\CAT{C}$ has enough projectives. We shall construct the \emph{left derived functors} $L_nF$ of $F$.

  Choose a projective resolution $P_\CCCIR$ for each object $M\in\CAT{C}$. Then the image $FP_M$ of the deleted resolution $P_M$ is a complex over $\CAT{D}$. Define $L_nF(M)$ to be $H_n(FP_M)$.

  Let $f\colon M\to M'$ be a morphism in $\CAT{C}$. By the comparison lemma, there is a induced chain map $\check f\colon P_M\to P_{M'}'$ between deleted resolutions. Then $F\check f\colon FP_M\to FP_{M'}'$ is also a chain map, and we define $L_nF(f)$ to be $H_n(F\check f)$.
\end{definition}

\begin{remark}
  $L_nF$ is invariant under natural isomorphisms with the choice of resolutions.
\end{remark}

\begin{remark}
  If $P$ is projective, then $L_nF(P)=0$ for $i\neq 0$.
\end{remark}

\begin{remark}
  If $F$ is right exact, then the sequence
  \begin{equation*}
    \begin{tikzcd}[sep=small]
      FP_1 & FP_0 & FM & 0
      \arrow[from=1-1,]{1-2}[]{Fd_1}
      \arrow[from=1-2,]{1-3}[]{}
      \arrow[from=1-3,]{1-4}[]{}
    \end{tikzcd}
  \end{equation*}
  is exact, and hence $L_0F(M)=FP_0\DIV\IM Fd_1 \cong F(M)$. Moreover, we have a natural isomorphism $L_0F\cong F$.
\end{remark}

\begin{proposition}
  An exact functors preserve left derived functors.
\end{proposition}

\begin{proposition}
  $L_nF$ defines a homological $\delta$-functor. In particular, every short exact sequence $0\to M'\to M\to M''\to 0$ in $\CAT{C}$ induces a long exact sequence
  \begin{equation*}
    \begin{tikzcd}[sep=small]
      \dotsb & L_1FM'' & L_0FM' & L_0FM & L_0FM'' & 0
      \arrow[from=1-1,]{1-2}[]{}
      \arrow[from=1-2,]{1-3}[]{}
      \arrow[from=1-3,]{1-4}[]{}
      \arrow[from=1-4,]{1-5}[]{}
      \arrow[from=1-5,]{1-6}[]{}
    \end{tikzcd}
  \end{equation*}
  in $\CAT{D}$.
\end{proposition}

\begin{proof}
  Use the horseshoe lemma, apply by $F$, and then use the long exact sequence.

  The diagram commutes after applying the homology functor.
  \begin{equation*}
    \begin{tikzcd}[row sep=large]
      \dotsb & H_n(FP_M') & H_n(F\tilde{P}_M) & H_n(FP_M'') & \dotsb\\
      & & H_n(FP_M)
      \arrow[from=1-1,]{1-2}[]{}
      \arrow[from=1-2,]{1-3}[]{}
      \arrow[from=1-3,]{1-4}[]{}
      \arrow[from=1-4,]{1-5}[]{}
      \arrow[from=1-2]{2-3}[]{}
      \arrow[from=1-3,shift right,]{2-3}[]{}
      \arrow[from=1-3,shift left,leftarrow]{2-3}[]{}
      \arrow[from=1-4]{2-3}[]{}
    \end{tikzcd}
  \end{equation*}
\end{proof}

\begin{remark}[Dimension Shifting]
  If $0\to K\to P\to M\to 0$ is exact with $P$ projective (or $F$-acyclic), then $L_nF(M)\cong L_{n-1}F(K)$ for $n\geq 2$, and $L_1F(A)$ is the kernel of $F(K)\to F(P)$.
\end{remark}

\begin{theorem}
  Assume $\CAT{C}$ has enough projectives. Then for any right exact functor $F\colon\CAT{C}\to\CAT{D}$, the derived functors $L_nF$ form a universal $\delta$-functor.
\end{theorem}

\begin{proof}
  Dimension shifting.
\end{proof}

\begin{remark}
  An additive functor $F\colon\CAT{C}\to\CAT{D}$ is called \emph{coeffaceable} if for every $M$ here is a sujection $p\colon P\to A$ such that $F(p)=0$. The above proof actually shows that if $T_*$ is a homological $\delta$-functor such that each $T_n$ is coeffaceable except $T_0$, then $T_*$ is universal.

  Dually, an additive functor $F\colon\CAT{C}\to\CAT{D}$ is called \emph{effaceable} if for every $M$ here is injection $i\colon Q\to M$ such that $F(i)=0$. If $T^*$ is a cohomological $\delta$-functor such that each $T^n$ is effaceable except $T^0$, then $T^*$ is universal.

  Specifically, take $P, Q$ to be the mapping cone $\CONE(A)$, one can show that homology $H_*\colon \CHH(\CAT{C})\to \CAT{C}$ and cohomology $H^*\colon \CHC(\CAT{C})\to \CAT{C}$ are universal $\delta$-functors.
\end{remark}

\begin{definition}
  Let $G\colon \CAT{C}\to \CAT{D}$ be an additive functor between abelian categories where $\CAT{C}$ has enough injective. We shall construct the \emph{covariant right derived functors} $R^nG$ of $G$.

  Choose an injective resolution $Q_\CCCIR$ for each object $N\in\CAT{C}$. Then the image $GQ_N$ of the deleted resolution $Q_N$ is a complex over $\CAT{D}$. Define $R^nG(N)$ to be $H^n(GQ_N)$.

  Let $g\colon N\to N'$ be a morphism in $\CAT{C}$. By the comparison lemma, there is a induced chain map $\check g\colon Q_N\to Q_{N'}'$ between deleted resolutions. Then $G\check g\colon GQ_N\to GQ_{N'}'$ is also a chain map, and we define $R^nG(g)$ to be $H^n(G\check g)$.
\end{definition}

\begin{remark}
  $R^nG$ is invariant under natural isomorphisms with the choice of resolutions.
\end{remark}

\begin{remark}
  If $G$ is left exact, then the sequence
  \begin{equation*}
    \begin{tikzcd}[sep=small]
      0 & GN & GQ_0 & GQ_1
      \arrow[from=1-1,]{1-2}[]{}
      \arrow[from=1-2,]{1-3}[]{}
      \arrow[from=1-3,]{1-4}[]{Gd^1}
    \end{tikzcd}
  \end{equation*}
  is exact, and hence $R^0G(N)=\KER Gd^1 \cong G(N)$. Moreover, we have a natural isomorphism $R^0G\cong G$.
\end{remark}

\begin{proposition}
  $R^nG$ defines a homological $\delta$-functor. In particular, every short exact sequence $0\to N'\to N\to N''\to 0$ in $\CAT{C}$ induces a long exact sequence
  \begin{equation*}
    \begin{tikzcd}[sep=small]
      0 & R^0GN' & R^0GN & R^0GN'' & R^1GN' & \dotsb
      \arrow[from=1-1,]{1-2}[]{}
      \arrow[from=1-2,]{1-3}[]{}
      \arrow[from=1-3,]{1-4}[]{}
      \arrow[from=1-4,]{1-5}[]{}
      \arrow[from=1-5,]{1-6}[]{}
    \end{tikzcd}
  \end{equation*}
  in $\CAT{D}$.
\end{proposition}

\begin{definition}
  Let $H\colon \CAT{C}^{\OP}\to \CAT{D}$ be an contravariant additive functor between abelian categories where $\CAT{C}$ has enough projectives. We shall construct the \emph{contravariant right derived functors} $R^nH$ of $H$.

  Choose a projective resolution $P_\CCCIR$ for each object $M\in\CAT{C}$. Then the image $HP_M$ of the deleted resolution $P_M$ is a complex over $\CAT{D}$. Define $R^nF(M)$ to be $H^n(HP_M)$.

  Let $h\colon M\to M'$ be a morphism in $\CAT{C}$. By the comparison lemma, there is a induced chain map $\check h\colon P_M\to P_{M'}'$ between deleted resolutions. Then $H\check h\colon HP_{M'}'\to HP_{M}$ is also a chain map, and we define $R^nG(h)$ to be $H^n(\check h)$.
\end{definition}

\begin{remark}
  $R^nH$ is invariant under natural isomorphisms with the choice of resolutions.
\end{remark}

\begin{remark}
  If $H$ is left exact, then the sequence
  \begin{equation*}
    \begin{tikzcd}[sep=small]
      0 & HM & HP_0 & HP_1
      \arrow[from=1-1,]{1-2}[]{}
      \arrow[from=1-2,]{1-3}[]{}
      \arrow[from=1-3,]{1-4}[]{Hd_1}
    \end{tikzcd}
  \end{equation*}
  is exact, and hence $R^0H(M)=\KER Hd_1 \cong HM$. Moreover, we have a natural isomorphism $R^0H\cong H$.
\end{remark}

\begin{proposition}
  $R^nH$ defines a homological $\delta$-functor. In particular, every short exact sequence $0\to M'\to M\to M''\to 0$ in $\CAT{C}$ induces a long exact sequence
  \begin{equation*}
    \begin{tikzcd}[sep=small]
      0 & R^0HM'' & R^0HM & R^0HM' & R^1HM'' & \dotsb
      \arrow[from=1-1,]{1-2}[]{}
      \arrow[from=1-2,]{1-3}[]{}
      \arrow[from=1-3,]{1-4}[]{}
      \arrow[from=1-4,]{1-5}[]{}
      \arrow[from=1-5,]{1-6}[]{}
    \end{tikzcd}
  \end{equation*}
  in $\CAT{D}$.
\end{proposition}

The $\EXT$ functors are the covariant right derived functors of $\HOM_R(M,
\BLANK)$, and are also the contravariant right derived functors of $\HOM_R(\BLANK, N)$. By dimension shifting, one can prove that the two definitions give the same $\EXT_R^n(M, N)$.

\begin{remark}
  $\EXT^n_R(M, N)=\EXT_R^1(\IM d_{n-1}, N).$
\end{remark}

\begin{proposition}
  For a module $M\in \MODR$, the following are equivalent.
  \begin{enumerate}
    \item $M$ is projective.
    \item $\EXT_R^n(M, N)=0$ for any $n\geq 1$ and any $N\in\MODR$.
    \item $\EXT_R^n(M, N)=0$ for any $N\in\MODR$.
  \end{enumerate}
\end{proposition}

\begin{proof}
  Dimension shifting.
\end{proof}

\begin{remark}
  The first $\EXT$ group of $M$ and $N$ is isomorphic (as sets) to the extensions of $M$ by $N$, i.e. exact sequences $0\to M\to E\to N\to 0$.
\end{remark}

The $\TOR$ functors are the left derived functors of $M\otimes \BLANK$, and are also the left derived functors of $\BLANK\otimes N$. By dimension shifting, one can prove that the two definitions give the same $\TOR_n^R(M, N)$.

\begin{proposition}
  For a module $M\in\MODR$, the following are equivalent.
  \begin{enumerate}
    \item $M$ is flat.
    \item $\TOR_n^R(M, N)=0$ for any $n\geq 1$ and any $N\in\RMOD$.
    \item $\TOR_1^R(M, N)=0$ for any $N\in\RMOD$.
  \end{enumerate}
\end{proposition}

\section{Tor and Ext}

\subsection{Change of Rings}

We study whether projective resolution is preserved by extension of scalars of modules and of bimodules.

Model problem: bar resolution of $\ZG$ (over $\ZG$-Mod-$\ZG$) induces bar resolution of $\ZZ$ (over $\ZG$), and such relation induces an isomorphism of $\EXT$ groups.

To show the image is projective, it suffices to show that the right adjunction of the functor is exact. To show the image complex is exact, one usually shall examine the $\TOR$ group.

Reference: Cartan-Eilenberg: II.6.1, P30; VIII.3.1, P150; X.2.1, P185.

Let $R\to S$ and $R'\to S'$ be ring augmentations, and there is a morphism between augmented rings as shown in the diagram.
\begin{equation*}
  \begin{tikzcd}[row sep=large]
    R & S\\
    R' & S'
    \arrow[from=1-1,]{1-2}[]{\varepsilon}
    \arrow[from=2-1,]{2-2}[]{\varepsilon'}
    \arrow[from=1-1,]{2-1}[swap,]{\varphi}
    \arrow[from=1-2,]{2-2}[]{\psi}
  \end{tikzcd}
\end{equation*}
Let $M$ be a right $R'$-module and $N$ be a left $R'$-module. We shall define maps between $\TOR$ groups and between $\EXT$ groups.
\begin{equation*}
  \begin{aligned}
    F^{\varphi}\colon&\TOR^R(M, S)\to\TOR^{R'}(M, S'),\\
    F_\varphi\colon&\EXT_{R'}(S', N)\to\EXT_{R}(S, N).
  \end{aligned}
\end{equation*}

Let $P_{\CCCIR}$ ($P_{\CCCIR}'$, resp.) be a projective resolution of $S$ ($S'$, resp.), then $R'\otimes_{R}P_{\CCCIR}$ is a projective complex. By the comparison lemma, there exists chain maps $R'\otimes_{R}P_{\CCCIR}\to P_{\CCCIR}'$. Hence we may define $F^\varphi$ and $F_\varphi$ by the following maps.
\begin{equation*}
  \begin{aligned}
    F^{\varphi}\quad\leftwavearrow\quad& M\otimes_R P_{\CCCIR}=M\otimes_{R'}(R'\otimes_{R}P_{\CCCIR})\to M\otimes_{R'}P_{\CCCIR}';\\
    F_{\varphi}\quad\leftwavearrow\quad& \HOM_{R'}(P_{\CCCIR}', N)\to\HOM_{R'}(R'\otimes_{R}P_{\CCCIR})=\HOM_{R}(P_{\CCCIR}, N).
  \end{aligned}
\end{equation*}

\begin{theorem}[Mapping Theorem]\hspace*{\fill}
  \begin{enumerate}
    \item $F^\varphi$ are isomorphisms for any $M$, if and only if $R'\otimes_RS\cong S'$ and $\TOR_n^{R}(R', S)=0$ for $n\geq 0$.
    \item If (1) holds, then $F_\varphi$ are isomorphisms for any $C$.
    \item If (1) holds, then for any $R$-projective resolution $P_{\CCCIR}$ of $S$, the complex $R'\otimes_R P_{\CCCIR}$ with $R'\otimes_R P_{\CCCIR}\to R'\otimes_R S\cong S'$ is a $R'$-projective resolution of $S'$.
  \end{enumerate}
\end{theorem}

\begin{lemma}
  Suppose $A, B, C$ are $K$-algebras. Let $M$ be a projective ($A$,$B$)-bimodule, and $N$ a ($B$,$C$)-bimodule which is projective as a left $B$-module. Then $M\otimes_B N$ is a projective ($A$,$C$)-bimodule.
\end{lemma}

\begin{corollary}
  If $M$ is a projective ($A$,$B$)-bimodule, and if $A$ is $K$-projective, then $M$ is projective as a right $B$-module.
\end{corollary}

Now let $K$ be a commutative ring, $R$ a $K$-algebra, $M$ a right $R$-module, and $N$ a left $R$-module. Then we have the following morphism of augmented rings
\begin{equation*}
  \begin{tikzcd}[row sep=large]
    R^{\ENV} & R\\
    R & K
    \arrow[from=1-1,]{1-2}[]{\eta}
    \arrow[from=2-1,]{2-2}[]{\varepsilon}
    \arrow[from=1-1,]{2-1}[swap,]{\eta}
    \arrow[from=1-2,]{2-2}[]{\varepsilon}
  \end{tikzcd}
\end{equation*}
and the following maps between $\TOR$ groups and between $\EXT$ groups.
\begin{equation*}
  \begin{aligned}
    F^{\varphi}\colon&\TOR^{R^{\ENV}}({}_{\varepsilon}M, R)\to\TOR^{R}(M, K),\\
    F_\varphi\colon&\EXT_{R}(K, N)\to\EXT_{R^{\ENV}}(R, N_{\varepsilon}).
  \end{aligned}
\end{equation*}

\begin{lemma}
  If $M$ is a ($R$,$R$)-bimodule, then there is an $R$-module isomorphism as follows.
  \begin{equation*}
    {}_{\varepsilon}R\otimes_{R^{\ENV}}M\cong M\otimes_{R}K.
  \end{equation*}
\end{lemma}

\begin{theorem}
  If the supplemented $K$-algebra $R$ is $K$-projective, then $F^\varphi$ and $F_\varphi$ are isomorphisms, and for each $R^{\ENV}$-projective resolution $P_{\CCCIR}$ of $R$, the complex $P_{\CCCIR}\otimes_{R} K$ is an $R$-projective resolution of $K=R\otimes_R K$ as a left $R$-module.
\end{theorem}

Suppose $P_\CCCIR$ is a projective resolution of $\ZG$ over $\ZG^{\ENV}$. Then $P_\CCCIR\otimes_{\ZG}\ZZ$ is projective, and also resolution since $P_\CCCIR$ is projective as a right $\ZG$-module.

\section{Hochschild Theory}

Let $K$ be a commutative ring, and $R$ a $K$-algebra. (One can always view a ring as an algebra over its center.) For an ($R$, $R$)-bimodule $N$, define the $n^{\text{th}}$ \emph{Hochschild homology group} of $R$ with coefficient $N$ to be $\TOR_n^{R^{\ENV}}(R, N)$, which automatically has a structure of $K$-module. Dually, define the $n^{\text{th}}$ \emph{Hochschild cohomology group} of $R$ with coefficient $N$ to be $\EXT^n_{R^{\ENV}}(R, N)$.

There is a canonical resolution, named \emph{bar resolution}, of $R$ over $R^{\ENV}$, defined as follows,
\begin{equation*}
  \begin{tikzcd}[sep=small]
    \dotsb & R^{\otimes(n+2)} & R^{\otimes(n+1)} & \dotsb & R\otimes R & R & 0
    \arrow[from=1-1,]{1-2}[]{}
    \arrow[from=1-2,]{1-3}[]{d_n}
    \arrow[from=1-3,]{1-4}[]{}
    \arrow[from=1-4,]{1-5}[]{d_1}
    \arrow[from=1-5,]{1-6}[]{}
    \arrow[from=1-6,]{1-7}[]{}
  \end{tikzcd}
\end{equation*}
where $d_n\colon R^{\otimes(n+2)} \to R^{\otimes(n+1)}$ sends $x_0\otimes\dotsb\otimes x_{n+1}$ to $\sum_{0}^n (-1)^ix_0\otimes\dotsb \otimes x_ix_{i+1}\otimes\dotsb\otimes a_{n+1}$, and where $(R, R)$-module structure of $R^{\otimes(n+2)}$ is defined by outer action.

If $R$ is free as $K$-module, then we obtain a free resolution of $R$ over $R^{\ENV}$, by observing ${}_{R}R\otimes R^{\otimes n}\otimes R_{R}\cong {}_{R^{\ENV}}R^{\ENV}\otimes R^{\otimes n}$.

If there is an augmentation $R\to K$, and if $R$ is projective as $K$-module, then by applying $\BLANK\otimes_{R}K$ we obtain a projective resolution of $K$ over $R$, also named as the bar resolution of $K$ over $R$.

In the original paper of Hochschild, he defines the cochain complex $C^n(R, M)$ and the chain complex $C_n(R, N)$ as follows. $C^n(R, N)=\HOM_K(R^{\otimes n}, N)$.
\begin{equation*}
  \begin{aligned}
    \delta^n(f)(x_1\otimes\dotsb x_{n+1})
    =&\phantom{{}+{}}x_1f(x_2\otimes\dotsb\otimes x_{n+1})\\
    &+\sum_{i=1}^n(-1)^if(x_1\otimes\dotsb x_ix_{i+1}\otimes\dotsb\otimes x_{n+1})\\
    &+(-1)^{n+1}f(x_1\otimes\dotsb\otimes x_n)x_{n+1}.
  \end{aligned}
\end{equation*}% TODO: REVIEW TYPOGRAPHY

The zeroth Hochschild cohomology group is the center of $N$.
\begin{equation*}
  H^0(R, N)=\{y\in N\vert\ ry=yr, \forall r\in R\}.
\end{equation*}

The first Hochschild cohomology group is related to derivatives.
\begin{equation*}
  Z^1(R, N)=\{f\vert\ f(x_1\otimes x_2)=f(x_1)x_2+x_1f(x_2), \forall x_1, x_2\in R\}.
\end{equation*}

If $R$ is free as $K$-module, then the two definitions of Hochschild (co)homology coincide.

Additional Topics:

\subsection{Relation Between Hochschild Cohomology and Group Cohomology}

\begin{proposition}
  We have the following isomorphisms between $\EXT$ groups,
  \begin{equation*}
    \begin{aligned}
      \EXT_{\BIMOD{\ZG}{\ZG}}^\BCIR(\ZG, \ZG) &\cong \EXT_{\ZG{\textnormal-}\MOD}^\BCIR(\ZZ, {}_{\BCIR}\ZG),\\
      \EXT^\BCIR_{\BIMOD{\ZG}{\ZG}}(\ZG, \ZG_{\varepsilon}) &= \EXT^\BCIR_{\ZG{\textnormal-}\MOD}(\ZZ, \ZG),
    \end{aligned}
  \end{equation*}
  where the left $\ZG$-module structure of ${}_{\BCIR}\ZG$ is defined by conjugation action.
\end{proposition}

% !TeX root = ../algebra.tex

\chapter{Homological Algebra}

\begin{itemize}\color{red}
  \item hom complex, tensor complex, and 2-complex
  \item mapping cone motivation (P273 Munkres)
\end{itemize}

\section{Basic Concepts}

\subsection{Abelian Categories}

A \emph{pointed category} is a category with a zero object, i.e. an object 0 which is both an initial object and a terminal object. In a pointed category, one usually denote $0_{0, B}$ to be the unique morphism $0\to B$, $0_{A, 0}$ the unique morphism $A\to 0$, and $0_{A, B}$ the composition $0_{0, B}0_{A, 0}$.

An \emph{$\AB$-enriched category} is a locally small category such that every hom-set is equipped with the structrue of an abelian group; and such that the composition operation is bilinear. In an $\AB$-enriched category, a initial object is also a terminal object, and any finite product is also a coproduct, and dually.

A \emph{pre-additive category} is an pointed $\AB$-enriched category. One can prove that $0_{A,B}$ is the zero element of the abelian group $\HOM(A, B)$.

An \emph{additive category} is a pre-additive category which admits finite biproducts.

A \emph{pre-abelian category} is an additive category such that every morphism has a kernel and a cokernel. Equivalently, a pre-abelian category is an additive category which admits all finite limits and finite colimits. In a pre-abelian category, every morphism $f\colon A\to B$ induces a morphism $\overline{f}\colon\COKER\KER f\to \KER\COKER f$ as shown in the diagram.
\begin{equation*}
  \begin{tikzcd}
    0 & \KER f & A & B & \COKER f & 0\\
    & & \COKER\KER f & \KER\COKER f
    \arrow[from=1-1,]{1-2}[]{}
    \arrow[from=1-2,]{1-3}[]{}
    \arrow[from=1-3,]{1-4}[]{f}
    \arrow[from=1-4,]{1-5}[]{}
    \arrow[from=1-5,]{1-6}[]{}
    \arrow[from=1-3,two heads,]{2-3}[]{}
    \arrow[from=1-4,hookleftarrow,]{2-4}[]{}
    \arrow[from=2-3,dashed,]{1-4}[]{}
    \arrow[from=2-3,dashed,]{2-4}[]{\overline{f}}
  \end{tikzcd}
\end{equation*}

\begin{proposition}
  Every morphism $f\colon A\to B$ in a pre-abelian category has a canonical decomposition
  \begin{equation*}
    \begin{tikzcd}[sep=small]
      A & \COKER\KER f & \KER\COKER f & B
      \arrow[from=1-1,]{1-2}[]{p}
      \arrow[from=1-2,]{1-3}[]{\overline{f}}
      \arrow[from=1-3,]{1-4}[]{i}
    \end{tikzcd}
  \end{equation*}
  where $p$ is a cokernel, hence epic, and $i$ is a kernel, hence monic.
\end{proposition}

An \emph{abelian category} is a pre-abelian category such that for every morphism $f$, the induced map $\overline{f}$ is an isomorphism. Equivalently, an abelian category is a pre-abelian category such that every monomorphism is the cokernel of its kernel, and that every epimorphism is the kernel of its cokernel.

\begin{remark}
  The dual of an additive (abelian, resp.) category is additive (abelian, resp).
\end{remark}

Chain complex abelian, functor category abelian.

\begin{theorem}[Freyd-Mitchell Embedding Theorem]
  Every small abelian category admits a full, faithful and exact functor to the category $\MODR$ for some ring $R$.
\end{theorem}

The Freyd-Mitchell Embedding Theorem implies that proofs about exactness of sequences in an abelian category can always be obtained by a diagram chase.

The category of chain complexes of an abelian category $\CAT{A}$, denoted by $\CH(\CAT{A})$, is an abelian category.

\subsection{Homology and Diagram Chasing Lemmas}

The homology in degree $n$ defines a functor $H_n\colon\CH(\MODR)\to\MODR$.

\begin{proposition}
  $H_n$ commutes with exact functors. In particular, $H_n$ commutes with direct limits.
\end{proposition}

\begin{proof}
  Consider the exact sequence $0\to B_n\to Z_n\to H_n\to 0$ and recall that an exact functor preserves all finite limits and finite colimits.
\end{proof}

Let $C_\CCCIR$ be a complex. Then $\bigoplus C_n$ can be viewed as a graded $R$-module with a square-zero degree $+1$ endomorphism, or as an $R$-module with a square-zero endomorphism $d$ given by $(x_n)_n\mapsto (d_{n+1}(x_{n+1}))_n$.

positive chain complex; negative chain complex

exactness of a sequence of chain complexes

\begin{lemma}[Long Exact Sequence]
  Let $0\to C_\CCCIR'\xrightarrow{i} C_\CCCIR \xrightarrow{p} C_\CCCIR''\to 0$ be an exact sequence in $\CH(\CAT{A})$ for an abelian category $\CAT{A}$, then there is a long exact sequence in $\CAT{A}$.
  \begin{equation*}
    \begin{tikzcd}[sep=small]
      \dotsb & H_{n+1}(C_\CCCIR'') & H_n(C_\CCCIR') & H_n(C_\CCCIR) & H_n(C_\CCCIR'') & H_{n-1}(C_\CCCIR') & \dotsb
      \arrow[from=1-1,]{1-2}[]{}
      \arrow[from=1-2,]{1-3}[]{\partial_{n+1}}
      \arrow[from=1-3,]{1-4}[]{i_*}
      \arrow[from=1-4,]{1-5}[]{p_*}
      \arrow[from=1-5,]{1-6}[]{\partial_n}
      \arrow[from=1-6,]{1-7}[]{}
    \end{tikzcd}
  \end{equation*}
  Moreover, the long exact sequence is natural.
\end{lemma}

The lemma is the first of three fundamental lemmas of homological algebra.

\begin{proof}
  For naturality, let $\CAT{J}$ denote the category $\BCIR\to\BCIR$. Then the functor category $\CAT{A}^{\CAT{J}}$ is also an abelian category. By applying the long exact sequence to the category $\CAT{A}^{\CAT{J}}$ one can obtain naturality of the long exact sequence.
\end{proof}

\begin{lemma}[Snake Lemma]

\end{lemma}

\begin{lemma}[Five Lemma]
  Given a commutative diagram with exact rows.
  \begin{equation*}
    \begin{tikzcd}
      A_1 & A_2 & A_3 & A_4 & A_5\\
      B_1 & B_2 & B_3 & B_4 & B_5
      \arrow[from=1-1,]{1-2}[]{}
      \arrow[from=1-2,]{1-3}[]{}
      \arrow[from=1-3,]{1-4}[]{}
      \arrow[from=1-4,]{1-5}[]{}
      \arrow[from=2-1,]{2-2}[]{}
      \arrow[from=2-2,]{2-3}[]{}
      \arrow[from=2-3,]{2-4}[]{}
      \arrow[from=2-4,]{2-5}[]{}
      \arrow[from=1-1,]{2-1}[swap]{h_1}
      \arrow[from=1-2,]{2-2}[swap]{h_2}
      \arrow[from=1-3,]{2-3}[]{h_3}
      \arrow[from=1-4,]{2-4}[]{h_4}
      \arrow[from=1-5,]{2-5}[]{h_5}
    \end{tikzcd}
  \end{equation*}
  \begin{enumerate}
    \item If $h_2, h_4$ are epic and $h_5$ is monic, then $h_3$ is epic.
    \item If $h_2, h_4$ are monic and $h_1$ is epic, then $h_3$ is monic.
  \end{enumerate}
\end{lemma}

One can prove the Snake Lemma, the Five Lemma and many other lemmas in homological algebra in an elementary way by applying the Freyd-Mitchel embedding theorem and then by proceeding a diagram chase. Or one can prove the five lemma and the long exact sequence by the snake lemma. A more illuminating approach is the salamander lemma.

Further Reading: \href{https://ncatlab.org/nlab/show/salamander+lemma}{The Salamander Lemma}, \url{https://math.stackexchange.com/a/2511909}, \url{https://math.stackexchange.com/a/74872}.

(Motivation: homotopic maps should define the same map of homology)

homotopy: $s$ of degree $+1$

homotopy defines an equivalence relation on chain maps.

homotopy category

\subsection{Derived Functors}

resolution, left derived functors, right derived functors

Ext, Tor

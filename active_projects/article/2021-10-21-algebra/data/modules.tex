% !TeX root = ../algebra.tex

\chapter{Modules}

\section{Basic Structure}

\subsection{Wedderburn-Remak-Krull-Schmidt Theorem}

Every group with operators with composition series can be written uniquely as a finite direct product of indecomposable groups. Hence one can classify all finte groups if one knows all finite indecomposable groups.

We need the following properties of a normal endomorphism:
\begin{enumerate}
  \item A normal endomorphism preserves normal subgroups.
  \item The composition of two normal endomorphisms is normal.
  \item The sum of two normal endomorphisms is normal.
  \item The canonical idempotents of a biproduct are normal.
\end{enumerate}

\begin{proposition}[Fitting's Lemma]
  Suppose a group $G$ admit composition series with respect to its normal subgroups and let $f\colon G\to G$ be a normal endomorphism. Then $G$ is the biproduct of $f^{\infty}(G)$ and $f^{-\infty}(1)$, where $f$ is an automorphism on $f^{\infty}(G)$ while $f$ is a nilpotent endomorphism on $f^{\infty}(G)$.
\end{proposition}

\begin{corollary}
  Let $f$ be an endomorphism on an indecomposable group $G$. Then $f$ is either isomorphic or nilpotent.
\end{corollary}

\begin{corollary}
  Let $G$ be a group and $H$ be an indecomposable group and let $f\colon G\to H$ and $g\colon H\to G$ be morphisms such that $gf$ is an isomorphism on $G$ and $fg$ is a normal endomorphism on $H$. Then both $f$ and $g$ are isomorphisms.
\end{corollary}

\begin{proof}
  Observe that $(gf)^{n+1}=g(fg)^{n}f$ implies that $fg$ cannot be nilpotent.
\end{proof}

\begin{definition}
  A group is called strongly indecomposable only if that the sum $f+g$ of normal endomorphisms of $G$ is an isomorphism implies that either $f$ or $g$ is an isomorphism.
\end{definition}

\begin{proposition}
  A strongly indecomposable group is indecomposable.
\end{proposition}

\begin{proof}
  Suppose $G\cong G_1\times G_2$, then the canonical idempotent maps $i_1p_1$ and $i_2p_2$ are normal endomorphisms and their sum is an isomorphism.
\end{proof}

\begin{proposition}
  An indecomposable group with composition series is strongly indecomposable.
\end{proposition}

\begin{proof}
  If $f$, $g$ is nilpotent and $f+g$ is an endomorphism, then $f+g$ is nilpotent. (Also corollary of Fitting's lemma.)
\end{proof}

\begin{theorem}[Wedderburn-Remak-Krull-Schmidt Theorem]
  Let $G$ be a group with composition series and $g$ be a group isomorphism between $G$ and $H$. Then $G, H$ admits decompositions into indecomposable groups respectively. And after possibly change of indices of $H_j$, we have \begin{equation}
    G\cong g^{-1}(H_1)\times \dotsb g^{-1}(H_r) \times G_{r+1}\times \dotsb \times G_m
  \end{equation}
  for each $r$.
\end{theorem}

\begin{proof}
  Let $e_j$, $f_j$ denote the canonical idempotents of $G$ and $H$. Define $k_j=f_jge_1$ and $h_j=e_1g^{-1}f_j$. Then the restriction of $h_jk_j$ on $G_1$ is a normal endomorphism for all $j$, and their sum is an automorphism. Assume the restriction of $h_1k_1$ is an automorphism. Then both $h_1$ and $k_1$ are isomorphisms.

  We claim that $G\cong g^{-1}(H_1)\times G_2\times \dotsb\times G_m$. $G_1 = h_1(H_1)=e_1(g^{-1}(H_1))$ implies that $G$ is the product of these subgroups. If $x$ in the intersection of $g^{-1}(H_1)$ and the rest product, then $e_1(x)=e_1(g(y))=0$. It implies from the fact that $k_1$ is an isomorphism that $y=0$ and hence $x=g(y)=0$.
\end{proof}

\section{Special Modules}

\subsection{Projective Modules}

\section{Nakayama's Lemma}

\begin{lemma}
  Let $R$ be a ring, $J$ the Jacobson radical of $R$, and $M$ a non-zero finitely generated $R$-module. Then $JM\subsetneq M$.
\end{lemma}

\begin{proof}
  Suppose $x_1, \dotsc, x_n$ is a minimal set of generators of $M$. Assume $JM=M$. Then we have $x_n=\sum r_ix_i$, which implies that
  \begin{equation*}
    (1-r_n)x_n=r_1x_1+\dotsb+r_{n-1}x_{n-1}.
  \end{equation*}
  Since $1-r_n$ is a unit, we have derived a contradiction.
\end{proof}

\begin{proof}[Local Algebra, Serre, P1]
  By Zorn's lemma, every non-zero finitely generated module $M$ admits a non-zero simple quotient module $M\DIV N$, which is isomorphic to $R/I$ for a maximal left ideal $I$ of $R$. Since $J\subseteq I$, we have $J$ kills $M\DIV N$. It implies that $JM\subseteq N\subsetneq M$.
\end{proof}

\begin{corollary}
  Commutative; Commutative local ring.
\end{corollary}

\begin{lemma}
  Let $R$ be a (nonnegatively) graded ring, and $M$ be a non-zero finitely generated graded $R$-module with a lower bound. Let $R_+$ denote the invariant maximal ideal of $R$. Then $R_+M\subsetneq M$.
\end{lemma}

\begin{proof}
  Let $d$ be the lowest degree of $M$. Then $R_+M\cap M_d=0$.
\end{proof}

\begin{corollary}
  $M\otimes_R R\DIV R_+=0$ implies that $M=0$.
\end{corollary}

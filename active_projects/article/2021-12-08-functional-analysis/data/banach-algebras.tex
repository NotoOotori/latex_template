% !TeX encoding = UTF-8

\chapter{Banach Algebras}

Here are some typical examples of Banach algebras: algebra of functions, operator algebra and group algebra. In this chapter, we are concerning about complex Banach algebras (with identity) and their spectrums.

\begin{definition}
  A Banach space $\BA{A}$ (over $\CC$) is called a Banach algebra only if there is a multiplication operation on $\BA{A}$, making it into an algebra (with identity), with $\NORM 1=1$ and $\NORM{ab}\leq \NORM{a}\,\NORM{b}$.
\end{definition}

If there is a Banach space meeting all requirement to be a Banach algebra except $\NORM 1=1$, we may define an equivalent norm on it by letting
\begin{equation*}
  \NORM{x}'=\sup_{y\neq 0}\frac{\NORM{xy}}{\NORM{y}}.
\end{equation*}
Observe we have $\NORM{x}\DIV\NORM{1}\leq\NORM{x}'\leq\NORM{x}$ and $\NORM{1}'=1$.

Here are some examples of a Banach algebra.

\begin{itemize}
  \item Operator algebra: $M_n(\CC)$; $B(X)$, where $X$ is a Banach space;
  \item Algebra of functions (with multiplication induced by the field): $C(X)$, where $X$ is a compact Hausdorff space; $C(\overline\DD)$; $A(\overline\DD)$; $H^{\infty}(\DD)$: the space of bounded analytic functions on $\DD$;
  \item Algebra of functions (with multiplication defined by convolution): $l^1(\ZZ)$; $L^1(\RR)$.
\end{itemize}

\begin{example}
  $L^1(\RR)$ does not admit an identity. Consider $f = \chi_{[0, 1]}$, and recall the fact that $L^1(\RR)$ functions are continuous with respect to translation.
\end{example}

\begin{proposition}
  Let $\BA{A}$ be a Banach algebra, then
  \begin{enumerate}
    \item If $x\in \BA{A}$ and $\NORM{x}<1$, then $1-x$ is invertible with inverse $(1-x)^{-1}=\sum_{n=0}^\infty x^n$, and we have $\NORM{(1-x)^{-1}}\leq (1-\NORM{x})^{-1}$;
    \item Suppose $x\in \BA{A}^\times$ and $h\in \BA{A}$ satisfies $\NORM{h}\leq\NORM{x^{-1}}^{-1}\DIV 2$. Then $x+h\in \BA{A}^{\times}$, with $\NORM{(x+h)^{-1}-x^{-1}}\leq 2\NORM{x^{-1}}^2\,\NORM{h}$;
    \item $\BA{A}^\times$ is open, and the map $x\mapsto x^{-1}$ defines a homeomorphism on $\BA{A}^\times$.
  \end{enumerate}
\end{proposition}

\begin{hint}
  For (2), observe that
  \begin{equation}
    (x+h)^{-1}-x^{-1}=(x+h)^{-1}(x-(x+h))x^{-1}=-(1+x^{-1}h)x^{-1}hx^{-1}.
  \end{equation}
\end{hint}

Recall that if $X$ is a Banach space and $A\in B(X)$, then $A-\lambda$ is invertible if and only if $A-\lambda$ is bijective by the inverse mapping theorem.

\begin{definition}
  Let $\BA A$ be a Banach algebra and let $x\in \BA A$. Then we define the spectrum of $x$ to be the set
  \begin{equation*}
    \sigma(x)=\{\lambda\in \CC\VERT x-\lambda\notin \BA{A}^{\times}\},
  \end{equation*}
  and the spectral radius of $x$ to be
  \begin{equation*}
    r(x)=\sup\{\ABS{\lambda}\VERT \lambda\in \sigma(x)\}.
  \end{equation*}
\end{definition}

\begin{proposition}
  Let $\BA A$ be a Banach algebra and let $x\in \BA A$. Then
  \begin{enumerate}
    \item $\sigma(x)$ is not empty;
    \item $\sigma(x)$ is compact (or is both closed and bounded) and $r(x)\leq \NORM x$;
    \item $r(x)=\max\{\ABS{\lambda}\VERT \lambda\in \sigma(x)\}=\lim_{n\to\infty}\NORM{x^n}^{1\DIV n}$.
  \end{enumerate}
\end{proposition}

\begin{hint}
  For (1), assume $\sigma(x)$ is empty. Define $F\colon \lambda\in\CC\mapsto (f-\lambda)^{-1}\in \BA A$. Show by direct calculation that $F$ is differentialable (by the fact that inverse is continuous) and that $\lim_{\lambda\to\infty} \NORM{F(\lambda)}=0$. Then derive the contradiction using the Liouville's theorem (if $fF$ is constant for every $f\in B(\BA A)$, then $F$ must be constant).

  For (2), use the continuous function $\lambda\in\CC\to f-\lambda\in\BA A$ to show $\sigma(x)$ is closed, and use the equality $x-\lambda=\lambda(x\DIV\lambda-1)$ to show $r(x)\leq \NORM{x}$.

  For (3), first show that $\lambda\in\sigma(x)$ implies that $\lambda^n\in\sigma(x)$ by using the equality
  \begin{equation*}
    x^n-\lambda^n = (x-\lambda)(x^{n-1}+x^{n-2}\lambda+\dotsb+\lambda^{n-1}) = (x^{n-1}+x^{n-2}\lambda+\dotsb+\lambda^{n-1})(x-\lambda).
  \end{equation*}
  It follows that
  \begin{equation*}
    r(x)\leq \liminf \NORM{x^n}^{1\DIV n}
  \end{equation*}
  Next consider the series $-x^n\DIV \lambda^{n+1}$, which converges to $(x-\lambda)^{-1}$ when $\ABS{\lambda}>\NORM{x}$. It implies that for every $\varphi\in\BA A^*$ (and for such $\lambda$), the series $\varphi(-x^n\DIV \lambda^{n+1})$ converges. It follows from Banach-Steinhaus theorem that there is some $M_\lambda$ such that $\NORM{x^n\DIV \lambda^{n+1}}<M_\lambda$ for all $n$, which implies that
  \begin{equation*}
    r(x)\geq \limsup\NORM{x^n}^{1\DIV n}.
  \end{equation*}
\end{hint}

\begin{example}
  Let $V\colon C[a, b]\to C[a, b]$ sending $f$ to the function $x\mapsto \int_a^x f(t)\DIFF t$. Then $V\in B(C[a, b])$ and
  \begin{equation*}
    \ABS{V^nf(x)}\leq\frac1{n!}(x-a)^n\NORM{f}.
  \end{equation*}
  Hence we have $r(V)=0$, and thus $\sigma(V)=\{0\}$.
\end{example}

\begin{proposition}
  Let $\BA B$ be a Banach algebra and let $\BA A\subseteq \BA B$ be a closed subalgebra. Suppose $x\in\BA A$. Then
  \begin{enumerate}
    \item $r_{\BA A}(x) = r_{\BA B}(x) = \lim_{n\to\infty}\NORM{x^n}^{1\DIV n}$;
    \item $\partial\sigma_{\BA A}(x)\subseteq\sigma_{\BA B}(x)\subseteq \sigma_{\BA A}(x)$;
    \item If $\Omega$ is a connected component of $\CC - \sigma_{\BA B}(x)$, then either $\Omega\cap \sigma_{\BA A}(x)=\varnothing$ or $\Omega\cap \sigma_{\BA B}(x)=\varnothing$;
    \item Let $\{\Omega_n\}$ be the collection of connected components of $\CC - \sigma_{\BA B}(x)$ that intersects $\sigma_{\BA A}(x)$ (which is at most countable since $\CC - \sigma_{\BA B}(x)$ is locally path-connected and second-countable). Then
    \begin{equation}
      \sigma_{\BA A}(x)=\sigma_{\BA B}(x)\cup\bigcup \Omega_n.
    \end{equation}
  \end{enumerate}
\end{proposition}

We shall remark that (4) is a good tool to calculate the spectrum.

\begin{hint}
  For (2), suppose $\lambda\in\partial\sigma_{\BA A}(x)$, which is equivalent to the existence of a sequence $\lambda_n\notin \sigma_{\BA A}(x)$ such that $\lambda_n\to \lambda\in \sigma_{\BA A}(x)$. If $\lambda\notin \sigma_{\BA B}(x)$, then $(x-\lambda_n)^{-1}\to (x-\lambda)^{-1}$ in $\BA B$, and hence in $\BA A = \CL \BA A$, which contradicts to the fact that $\lambda\in\sigma_{\BA A}(x)$.

  For (3), let $X = \Omega\cap \sigma_{\BA A}(x)$. Observe that $(\Omega-X)\cap \sigma_{\BA A}(x)=\varnothing$, and apply
  \begin{equation*}
    \partial_\Omega X\subseteq\partial\sigma_{\BA A}(x)\subseteq\sigma_{\BA B}(x)\subseteq \CC-\Omega
  \end{equation*}
  to derive $\partial_\Omega X = \varnothing$.
\end{hint}

\begin{example}
  Let $K = \{z\in\CC\VERT 1\leq\ABS{z}\leq 2\}$, $C(K)$ be the Banach algebra of (complex valued) continous functions on $K$ with the uniform norm, $\BA A$ be the closed subalgebra generated by $\{1, z\}$, $\BA B$ be the closed subalgebra generated by $\{1, z, 1\DIV z\}$, and $f\in C(K)$ sending $z\in K$ to $z\in\CC$.

  Since we have $(z-\lambda)^{-1}=\sum_{n=0}^\infty \lambda^n\DIV z^{n+1}$ when $\ABS\lambda<\ABS z$, then $f-\lambda$ is invertible if $\ABS\lambda < 1$. Hence it's not hard to show $\sigma_{\BA B}(f)=K$.

  To show $\sigma_{\BA A}(f)=\{z\in\CC\VERT \ABS{z}\leq 2\}$, it suffices to show that $1\DIV z\notin \BA A$. Find a loop of $K$ such that the integral of $1\DIV z$ on the loop is nonzero while the integral of any polynomial is 0.
\end{example}

Let $x\in \BA A$ be an element of a Banach algebra, and let $f$ be a analytic function on a neighbourhood of $\{z\in\CC\VERT \ABS{z}\leq r(x)\}$ (containing $\{z\in\CC\VERT \ABS{z}\leq R\}$ for some $R>r(x)$). Then since
\begin{equation*}
  \sum_{n=0}^\infty \NORM{a_nx^n}\leq\sum_{n=0}^\infty\NORM{a_n}\,\NORM{x^n}=\sum_{n=0}^\infty \NORM{a_n}R^n(\frac{\NORM{x^n}^{1\DIV n}}{R})^n,
\end{equation*}
we may define an element $f(x)=\sum_{n=0}^\infty a_nx^n\in \BA A$. We have $(f\cdot g)(x)=f(x)g(x)$. Specifically, if $f$ is nowhere zero, then $f(x)$ is invertible in $\BA A$.

\begin{theorem}[Spectral Mapping Theorem]
  Let $x\in \BA A$ be an element of a Banach algebra, and let $f$ be a analytic function on a neighbourhood $\Omega$ of $\{z\in\CC\VERT \ABS{z}\leq r(x)\}$. Then $\sigma(f(x))=f(\sigma(x))$.
\end{theorem}

\begin{hint}
  For "$\supseteq$", observe that $g(z)=(f(z)-f(\lambda))\DIV (z-\lambda)$ is analytic on $\Omega$. And hence if $x-\lambda$ is not invertible, then $f(x)-f(\lambda)$ is not invertible.

  For "$\subseteq$", we suppose $\mu\notin f(\sigma(x))$ and are to show $f(x)-\mu$ is invertible. Observe that $f(z)-\mu$ is nowhere zero on $\sigma(x)$. Hence we may write
  \begin{equation*}
    f(z)-\mu=g(z)(z-\mu_1)\dotsb(z-\mu_n),
  \end{equation*}
  where $g$ is analytic on $\Omega'$, and $\mu_1, \dotsc, \mu_n\in\Omega'-\sigma(x)$, with $\Omega'=\INT \{z\in\CC\VERT \ABS{z}\leq r(x)+\epsilon\}$ satisfying $\CL\Omega'\subseteq\Omega$. It follows that $f(x)-\mu$ is invertible, being a finite multiplication of invertible elements.
\end{hint}

\begin{theorem}[Gelfand-Mazur]
  If $\BA A$ is a Banach algebra which is also a division algebra, then there is a unique isomorphism $\BA A\to \CC$.
\end{theorem}

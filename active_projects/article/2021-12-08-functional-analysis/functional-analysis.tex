% !TeX encoding = UTF-8

\documentclass{note}

\newcommand{\BA}[1]{\symscr{#1}} % Banach Algebra
\newcommand{\CC}{\symbb{C}}
\newcommand{\DD}{\symbb{D}}
\newcommand{\ZZ}{\symbb{Z}}
\newcommand{\RR}{\symbb{R}}
\newcommand{\NN}{\symbb{N}}
\newcommand{\DIV}{{\divslash}}
\newcommand{\ID}{\textup{id}}
\newcommand{\VERT}{\,\vert\,}
\newcommand{\ABS}[1]{\vert#1\vert}
\newcommand{\NORM}[1]{\Vert#1\Vert}
\newcommand{\DIFF}{\mathop{}\!\symup{d}}
\newcommand{\CAT}[1]{\symsfup{#1}}
\newcommand{\WS}{w^*}
\newcommand{\IDL}[1]{\symfrak{#1}}
\newcommand{\CONE}[1]{\symcal{#1}}
\DeclareMathOperator{\CL}{cl}
\DeclareMathOperator{\INT}{int}
\DeclareMathOperator{\DIST}{dist}
\DeclareMathOperator{\EXT}{ext}
\DeclareMathOperator{\CLCO}{\overline{co}}
\DeclareMathOperator{\RAD}{rad}
% \newcommand{\PP}{\symbb{P}}
% \newcommand{\NN}{\symbb{N}}
% \DeclareMathOperator*{\MAX}{max}
% \DeclareMathOperator*{\MIN}{min}
% \DeclareMathOperator{\DIM}{dim}

\title{Functional Analysis}

\begin{document}
  \maketitle
  % !TeX encoding = UTF-8

\chapter{Linear Operators}

\section{Topological Linear Spaces}

$0\in \CL S$ under the weak topology.

\begin{proposition}
  Let $X$ be a locally convex topological space defined by a family $\FML P$ of separating seminorms. Then a linear functional $l$ on $X$ is continuous if and only if there exist $p_1, \dotsc, p_n\in \FML P$ and $M>0$ that satisfies
  \begin{equation*}
    \ABS{l(x)}\leq M(p_1(x)+\dotsb+p_n(x)).
  \end{equation*}
\end{proposition}

\begin{proof}
  Suppose $l$ is continuous at 0. Observe that if we let
  \begin{equation*}
    y=\frac{x}{p_1(x)+\dotsb+p_n(x)+t},
  \end{equation*}
  where $t>0$, then $p_j(y)<1$ for every $x\in X$. The other direction is obvious.
\end{proof}

\begin{proposition}
  Let $X$ be a topological linear space such that $X^*$ separates $X$. Then
  \begin{equation*}
    (X_w)^*= X^*.
  \end{equation*}
\end{proposition}

\begin{proof}
  It suffices to show that if $l\colon X_w\to\CC$ is a continuous linear functional, then $l\in X^*$. Observe that since $l$ can be controlled by a finite number of seminorms, say $p_1=\ABS{l_1}, \dotsc, p_n=\ABS{l_n}$, then
  \begin{equation*}
    \ker l\supseteq\bigcap_{j=1}^n\ker l_j.
  \end{equation*}
  By some linear algebra arguments we can show that $l$ is a linear combination of $l_1, \dotsc, l_n$, and hence $l\in X^*$.
\end{proof}

The next proposition is a nice sufficient condition of a compact space being metrizable.

\begin{proposition}
  Let $X$ be a compact Hausdorff space. If there exists a separating sequence of functions $\{f_n\}\subseteq C(X)$, then $X$ is metrizable.
\end{proposition}

\begin{proof}
  Assume these $f_n$ are uniformly bounded, and define
  \begin{equation*}
    d(x, y)=\sup_{n} \frac{1}{2^n}\ABS{f_n(x)-f_n(y)}.
  \end{equation*}
\end{proof}

\begin{corollary}
  Let $X$ be a Banach space. If $X$ is separable, then the unit closed ball of $X^*$ is metrizable equipped with the $\WS$ topology.
\end{corollary}

We remark here that the converse on the previous corollary is also true, where a proof can be found in Conway's book.

\begin{example}
  The closed unit sphere of $(l^\infty)^*$ is unmetrizable equipped with the $\WS$ topology.
\end{example}

Canonical isometry

\begin{proposition}[Goldstein?]
  Let $X$ be a Banach space, and $J$ the canonical isometry from $X$ to $X^{**}$. Let $B$ denote the unit ball of $X$. Then the $\WS$ closure of $J(B)$ contains the unit ball of $X^{**}$.
\end{proposition}

\section{Hahn-Banach Theorem}

In this section, we may assume all vector spaces to be over $\RR$. To show the corresponding theorems of vector spaces over $\CC$, one may observe that a complex linear functional $l$ can be expressed by its real part $u$, which is a real linear functional, as $l(x)=u(x)-iu(ix)$.

Amenable semigroup actions are fairly important since some forms of Hahn-Banach theorem can preserve this kind of action. Note that in functional analysis, a semigroup is always required to admit an identity element. For amenable semigroups, we admit the fact that
\begin{itemize}
  \item any commutative semigroup is amenable, and that
  \item the group of rigid motions of 2-dimensional space is amenable.
\end{itemize}

\subsection{Hahn-Banach Theorem (Analytic Form)}

\begin{definition}
  Let $X$ be a real linear space and let $p$ be a function $X\to \RR$. $p$ is called to be sublinear only if it satisfies
  \begin{equation*}
    \begin{aligned}
      p(tx)=tp(x) &\text{\enskip for all } t>0 \text{ and all }x\in X,\\
      p(x+y)\leq p(x)+p(y)&\text{\enskip for all }x, y\in X.
    \end{aligned}
  \end{equation*}
  $p$ is called to be a seminorm only if it satisfies
  \begin{equation*}
    \begin{aligned}
      p(x)\geq 0 &\text{\enskip for all x\in X},\\
      p(tx)=\ABS tp(x) &\text{\enskip for all } t\in\RR \text{ and all }x\in X,\\
      p(x+y)\leq p(x)+p(y)&\text{\enskip for all }x, y\in X.
    \end{aligned}
  \end{equation*}
\end{definition}

\begin{theorem}[Hahn-Banach Theorem (Analytic Form)]
  Let $X$ be a real linear space and let $p$ be a sublinear functional on $X$. Let $Y$ be a subspace of $X$ and let $l$ be a linear function on $Y$ that satisfies
  \begin{equation*}
    l(y)\leq p(y) \text{\enskip for all } y\in Y.
  \end{equation*}
  Then there exists a linear functional $\tilde l\colon X\to\RR$ that satisfies
  \begin{equation*}
    \tilde l(y)=l(y)\text{\enskip for all }y\in Y\text{\quad and\quad} \tilde l(x)\leq p(x) \text{\enskip for all } x\in X.
  \end{equation*}
\end{theorem}

\begin{corollary}
  Let $X$ be a real normed linear space. Let $Y$ be a subspace of $X$ and let $l$ be a continuous linear functional on $Y$. Then there exists a continuous linear functional $\tilde l\colon X\to\RR$ that satisfies
  \begin{equation*}
    \tilde l(y)=l(y)\text{\enskip for all }y\in Y\text{\quad and\quad} \NORM{\tilde l}_{X'}=\NORM{l}_{Y'}.
  \end{equation*}
\end{corollary}

\begin{proof}
  Take $p(x)=\NORM{l}_{Y'}\,\NORM{x}_X$ and then apply the analytic form of Hahn-Banach theorem.
\end{proof}

\begin{example}
  We may show the existence of a finite-additive measure $\mu$ on $\RR$ such that $\mu((a, b])=b-a$ for all $a\leq b$. Let
  \begin{equation*}
    X = \{f\colon\RR\to\RR\VERT f\text{ is bounded with compact support}\}\subseteq\CAT{Set}(\RR, \RR),
  \end{equation*}
  and let
  \begin{equation*}
    Y = \{f=\sum_{j=1}^nc_j\chi_{(a_j, b_j]}\VERT n\in\ZZ_+, c_j\in\RR, a_j\leq b_j\}\subseteq X.
  \end{equation*}
  Define
  \begin{equation*}
    l\colon f=\sum_{j=1}^n\chi_{(a_j, b_j]}\in Y\mapsto \sum_{j=1}^n c_j(b_j-a_j)
  \end{equation*}
  to be a linear functional on $Y$ and define
  \begin{equation*}
    p\colon f\in X\mapsto \inf_{g\geq f, g\in Y}l(g)
  \end{equation*}
  to be a sublinear functional such that $l\leq p\vert_Y$. Apply the analytic form of Hahn-Banach theorem, we obtain a positive linear functional $\tilde l$ extending $l$. Finally, we can define $\mu(E)=\tilde l(\chi_E)$ for any bounded set $E$ and define
  \begin{equation*}
    \mu(F)=\lim_{n\to\infty}\mu(E\cap (-n, n])
  \end{equation*}
  for any subset $F$ of $\RR$.
\end{example}

\begin{theorem}{(\bf Hahn-Banach Theorem (Analytic Form with Amenable Semigroup Action))}
  Let $X$ be a real linear space, let $G$ be an amenable semigroup that acts on $X$, and let $p$ be a sublinear functional on $X$ that satisfies
  \begin{equation*}
    p(gx)\leq p(x) \text{\enskip for all } g\in G \text{ and all } x\in X.
  \end{equation*}
  Let $Y$ be a subspace of $X$ such that $GY\subseteq Y$ and let $l$ be a linear function on $Y$ that satisfies
  \begin{equation*}
    \begin{aligned}
      l(y)\leq p(y) &\text{\enskip for all } y\in Y,\\
      l(gy)=l(y)&\text{\enskip for all } g\in G \text{ and all }y\in Y.
    \end{aligned}
  \end{equation*}
  Then there exists a linear functional $\tilde l\colon X\to\RR$ that satisfies
  \begin{equation*}
    \tilde l(y)=l(y)\text{\enskip for all }y\in Y\text{\quad and\quad} \tilde l(x)\leq p(x) \text{\enskip for all } x\in X.
  \end{equation*}
\end{theorem}

\begin{example}
  We may show the existence of a translation-invariant finite-additive measure on $\RR$ and $\RR^2$ compatible with the Lebesgue measure.
\end{example}

\begin{example}
  Banach limit of $l^\infty(\NN)$. $Y=\{\lim x_n \text{ exists}\}$, $l\colon Y\to\RR$ sends $x_n$ to $\lim x_n$.
\end{example}

\subsection{Hahn-Banach Theorem (Ordered Form), alias Krein-Riesz Theorem}

Ordered linear space; positive cone; Krein-Riesz theorem.

\begin{definition}
  Let $X$ be a real linear space, a subset $\CONE P\subset X$ is called a positive cone only if it satisfies
  \begin{equation*}
    \begin{aligned}
      x+y\in\CONE P&\text{\enskip for all }x, y\in\CONE P,\\
      tx\in\CONE P&\text{\enskip for all }t>0 \text{ and for all }x\in\CONE P.
    \end{aligned}
  \end{equation*}
  A positive cone $\CONE P$ induces a natural partial order on $X$, by defining
  \begin{equation*}
    x\leq y \text{\enskip if and only if \enskip}y-x\in\CONE P.
  \end{equation*}
  The partial order satisfies that if $x\leq y$ then
  \begin{equation*}
    \begin{aligned}
      x+z\leq y+z&\text{\enskip for all }z\in X,\\
      tx\leq ty&\text{\enskip for all }t>0.
    \end{aligned}
  \end{equation*}
  Conversely, such a partial order can define a positive cone, which shows the equivalent of these two concepts.
\end{definition}

The following theorem is the ordered form of the Hahn-Banach theorem, which is discovered first by Krein and Riesz.

\begin{theorem}[Krein-Riesz]
  Let $X$ be an ordered linear space, and $Y$ a linear subspace that satisfies
  \begin{equation*}
    \text{for every }x\in X\text{, there exists some }y\in Y\text{ such that }y\geq x.
  \end{equation*}
  Let $l$ be a positive linear functional on $Y$, i.e., $l(y)\geq 0$ whenever $y\geq 0$. Then $l$ can be extended to a positive linear functional on $X$.
\end{theorem}

\begin{remark}
  We are to show that the Krein-Riesz theorem is equivalent to (the analytic form of) the Hahn-Banach theorem.

  First assume the Krein-Riesz theorem. Let $X$ be the Banach space, $p$ the sublinear functional, $Y$ the subspace, and $l$ be the linear functional on $Y$ controlled by $p$. Then the positive cone on $X\oplus \RR$ is defined to be
  \begin{equation*}
    \CONE P=\{(x,t)\in X\oplus \RR\VERT t-p(x)\geq 0\},
  \end{equation*}
  and the linear functional on $Y\oplus \RR$ is defined to be
  \begin{equation*}
    l'\colon (y, t)\in Y\oplus \RR\mapsto t-l(y)\in\RR,
  \end{equation*}
  which is positive.

  Conversely, let $X$ be the Banach space, $\CONE P$ the positive cone, $Y$ the subspace, and $l$ the positive linear functional on $Y$. Then a sublinear functional $p$ on $X$ is defined by
  \begin{equation*}
    p\colon x\in X\mapsto \inf_{y\geq x, y\in Y}\NORM{y-x}.
  \end{equation*}
\end{remark}

\subsection{Hahn-Banach Theorem (Geometric Form), and Hahn-Banach Separation Theorem (Geometric Form)}

We are to define the Minkowski functional, which connects geometry and analysis.

\begin{definition}
  Let $V$ be an absorbing convex subset of a real linear space $X$. Then we define the Minkowski functional $p$ of $V$ by
  \begin{equation*}
    p\colon x\in X\mapsto \inf_{t\geq 0}\{x\in tV\}\in\RR.
  \end{equation*}
\end{definition}

\begin{proposition}
  Let $V$ be an absorbing convex subset of a real linear space $X$, and let $p$ be the Minkowski function of $V$.
  \begin{enumerate}
    \item $p$ satisfies
    \begin{equation*}
      \begin{aligned}
        p(x)\geq 0,&\\
        p(tx)=tp(x) &\text{\enskip for all }t\geq 0,\\
        p(x+y)\leq p(x)+p(y) &\text{\enskip for all } x, y\in X.
      \end{aligned}
    \end{equation*}
    \item If $-V=V$ (or in particular if $V$ is balanced), then $p$ also satisfies
    \begin{equation*}
      p(tx)=\ABS tp(x)\text{\enskip for all }t\in\RR,
    \end{equation*}
    \item If $X$ is a topological linear space and $V$ contains the origin as its interior point, then $p$ is continuous.
  \end{enumerate}
\end{proposition}

\begin{theorem}[Hahn-Banach Theorem (Geometric Form)]
  Let $V$ be a absorbing convex subset of a real linear space $X$, and let $p$ be the Minkowski functional of $V$. Suppose $x_o\notin V$. Then there is a linear functional $l$ on $X$ that satisfies
  \begin{equation*}
    l(x)\leq p(x)\text{\enskip for all }x\in X\text{\quad and\quad}l(x_0)=1.
  \end{equation*}
  In particular, $l$ satisfies
  \begin{equation*}
    l(x)\leq 1\text{\enskip for all }x\in V.
  \end{equation*}
\end{theorem}

\begin{proof}
  Take $Y$ to be the linear subspace spanned by $x_0$, and define a linear functional $l$ on $Y$ such that $l(x_0)=1$.
\end{proof}

\begin{theorem}{(\bf Hahn-Banach Separation Theorem (Geometrical Form))}
  Let $V$ and $W$ be two disjoint nonempty convex subsets of a real linear space $X$.
  \begin{enumerate}
    \item If $V$ is absorbing after a possible translation, then there is a nonzero linear functional $l$ on $X$ that satisfies
    \begin{equation*}
        \sup_{x\in V}l(x)\leq\inf_{y\in W}l(y).
      \end{equation*}
    \item If $X$ is normed and $V$ contains an interior point, then there is a continuous linear functional $l$ on $X$that satisfies
    \begin{equation*}
      \sup_{x\in V}l(x)\leq\inf_{y\in W}l(y).
    \end{equation*}
    \item If $X$ is normed and $\DIST(V, W)>0$, then there is a continuous linear functional $l$ on $X$ that satisfies
    \begin{equation*}
      \sup_{x\in V}l(x)<\inf_{y\in W}l(y).
    \end{equation*}
  \end{enumerate}
\end{theorem}

\begin{proof}
  For (1), by a possible translation, assume $V$ to be absorbing (and hence to contain the origin). Then fix a point $y_0\in W$ and apply the geometric form of the Hahn-Banach theorem to the absorbing convex subset $V+(-W)+y_0$, which yields the existence of a linear functional that satisfies
  \begin{equation*}
    l(x-y)\leq 0\text{\enskip for all }x\in V\text{ and all }y\in W.
  \end{equation*}

  For (2), recall that if a convex subset contains the origin as its interior point, then its Minkowski functional is continuous, and hence is bounded.

  For (3), first, denote $d=\DIST(V, W)$, and let $\tilde V = V + \{x\in X\VERT \NORM x< d\DIV 2\}$, where $\tilde V$ is open convex and doesn't intersect $W$. Next, find an element $x_0\in \{x\in X\VERT \NORM x< d\DIV 2\}$ such that $l(x_0)>0$. Then there is the inequality
  \begin{equation*}
    \sup_{x\in V}l(x)<\sup_{x\in V}l(x+x_0)\leq\sup_{x\in\tilde V}l(x)\leq \inf_{y\in W}l(y).
  \end{equation*}
\end{proof}

\begin{corollary}
  Let $X$ be a Banach space and $V$ a nonempty closed convex subset of $X$. Then $V$ is closed under $w$ topology.
\end{corollary}

\begin{example}
  Consider $X=L^2([0, 1]; \RR)$. For any $a\in\RR$, define
  \begin{equation*}
    V_a = \{f\in C[0, 1]\VERT f(0)=a\}.
  \end{equation*}
  these $V_a$ are disjoint convex subsets of $X$. Since $V_0$ is a dense linear subspace of $X$ with every $V_a$ being a translation of $V_0$, then there doesn't exist a continuous linear functional that separates $V_a$ and $V_b$.
\end{example}

\subsection{Hahn-Banach Theorem (Topological Form), and Hahn-Banach Separation Theorem (Topological Form)}

\begin{theorem}{\bf (Hahn-Banach Separation Theorem (Topological Form))}
  Let $X$ be a topological linear space, and let $V$ and $W$ be two disjoint nonempty convex subsets of $X$.
  \begin{enumerate}
    \item If $V$ contains an interior point, then there exists a continuous linear functional $l$ on $X$ and a real number $\gamma$ that satisfies
    \begin{equation*}
      \begin{aligned}
        l(x)\leq\gamma\leq l(y)&\text{\enskip for all }x\in V\text{ and all }y\in W,\\
        l(x)<\gamma\leq l(y)&\text{\enskip for all }x\in \INT V\text{ and all }y\in W.
      \end{aligned}
    \end{equation*}
    In particular, if $V$ is open, then $l$ and $\gamma$ satisfies
    \begin{equation*}
      l(x)<\gamma\leq l(y)\text{\enskip for all }x\in V\text{ and all }y\in W.
    \end{equation*}
    \item If $X$ is locally convex, $V$ is compact, and $V$ and $W$ are separated (i.e., $V\cap \CL W=\varnothing$), then there exists a continous linear functional $l$ on $X$ and two real numbers $\gamma_1<\gamma_2$ that satisfies
    \begin{equation*}
      l(x)<\gamma_1<\gamma_2 < l(y)\text{\enskip for all }x\in V\text{ and all }y\in W.
    \end{equation*}
    \item If $X'$ separates $X$, and $V$ and $W$ are compact, then there exists a continous linear functional $l$ on $X$ and two real numbers $\gamma_1<\gamma_2$ that satisfies
    \begin{equation*}
      l(x)<\gamma_1<\gamma_2 < l(y)\text{\enskip for all }x\in V\text{ and all }y\in W.
    \end{equation*}
  \end{enumerate}
\end{theorem}

\begin{proof}
  For (1), continuity of $l$ follows from the continuity of the Minkowski functional, while the second part follows from the fact that any nonzero linear functional is an open map (proved by the fact that the scalar multiplication is continuous).

  For (2), we claim the existence of a convex neighbourhood $U$ of the origin such that the convex open subset $U+V$ doesn't intersect $W$. Apply (1) and note that $l(V)$ is compact.

  For (3), equip $X$ with its weak topology, denoted by $X_w$, and observe that
  \begin{itemize}
    \item $X_w$ is a locally convex topological linear space, that
    \item $V$ and $W$ are compact as subspaces of $X_w$, and that
    \item $(X_w)'\cong X'$ as topological linear spaces.
  \end{itemize}
\end{proof}

\begin{corollary}[Mazur]
  Let $X$ be a locally convex topological linear space, $M$ a linear subspace, and $V$ a convex subset disjoint from $M$ with nonempty interior. Then there exists a continuous linear functional $l$ on $X$ that satisfies
  \begin{equation*}
    l(x) \leq 0\text{\enskip for all }x\in V\text{,\quad}l(x) < 0\text{\enskip for all }x\in \INT V\text{,\quad and \quad}l(y) = 0\text{\enskip for all }y\in M.
  \end{equation*}
\end{corollary}

\begin{proof}
  Apply part (2) of the previous theorem.
\end{proof}

\begin{theorem}{\bf (Hahn-Banach Theorem (Topological Form))}
  Let $X$ be a locally convex topological linear space, $M$ a linear subspace, and $l$ a continuous linear functional on $M$. Then there exists a continuous linear functional $\tilde l$ on $X$ that satisfies
  \begin{equation*}
    \tilde l(y)=l(y)\text{\enskip for all }y\in M.
  \end{equation*}
\end{theorem}

\begin{proof}
  Pick a point $x_0\in M$ such that $l(x_0)=1$, and then apply part (2) of the topological form of the Hahn-Banach separation theorem with $V$ to be $x_0$ and with $W$ to be the zero locus of $f$. Observe
  \begin{equation*}
    x=l(x)x_0+(x-l(x)x_0)\text{\enskip lor all }x\in M.
  \end{equation*}
\end{proof}

The theory of extreme points is closedly related to optimization theory. The next theorem states about the relationship between a compact convex set and its extreme points in a manner of combinatorical topology.

\begin{theorem}[Krein-Milman]
  Let $X$ be a topological linear space such that $X'$ separates $X$, and $K$ a nonempty compact subset of $X$.
  \begin{enumerate}
    \item $\EXT K$ is nonempty.
    \item If $K$ is convex, then $K=\CLCO\EXT K$.
    \item If $X$ is locally convex, then $K\subseteq \CLCO\EXT K$.
    \item If $X$ is locally convex, and $S\subseteq K$ satisfies $\CLCO S=\CLCO \EXT K$, then $\EXT K\subseteq\CL S$.
  \end{enumerate}
\end{theorem}

\begin{example}
  Let $X$ be a complex normed linear space, and $x\in X$. Calculating the distance
  \begin{equation*}
    \DIST (x, K)=\inf_{y\in K}\NORM{y-x}
  \end{equation*}
  is a nonlinear optimization problem on $K$. If $X$ is reflexive, and $K$ is a non-empty convex subset of $X$, then the optimization problem is always solvable on $K$, i.e., there exists $y\in K$ such that
  \begin{equation*}
    \DIST(x, K)=\NORM{y-x}.
  \end{equation*}
  Moreover, if $X$ is a Hilbert space, then such $y$ is unique, following from the parallelogram identity.

  Consider the $w$ topology of $X$. The function $f\colon y\in K\to\NORM{y-x}\RR$ is lower semi-continuous, where $K$ is closed convex, and $M=K\cap\{z\in X\VERT \NORM{z-x}\leq 2d\}$ is compact. Hence $f$ admits a minimum on $M$.
\end{example}

% 优化问题
% 拓扑线性空间 Banach-Alaogue定理
% 锥版本 Krein-Riesz定理

  % !TeX encoding = UTF-8

\chapter{Banach Algebras}

Here are some typical examples of Banach algebras: algebra of functions, operator algebra and group algebra (that comes from harmony analysis). In this chapter, we are concerning about complex Banach algebras (with identity) and their spectrums.

\begin{definition}
  A Banach space $\BA{A}$ (over $\CC$) is called a Banach algebra only if there is a multiplication operation on $\BA{A}$, making it into an algebra (with identity), with $\NORM 1=1$ and $\NORM{ab}\leq \NORM{a}\,\NORM{b}$.
\end{definition}

If there is a Banach space meeting all requirement to be a Banach algebra except $\NORM 1=1$, we may define an equivalent norm on it by letting
\begin{equation*}
  \NORM{x}'=\sup_{y\neq 0}\frac{\NORM{xy}}{\NORM{y}}.
\end{equation*}
Observe we have $\NORM{x}\DIV\NORM{1}\leq\NORM{x}'\leq\NORM{x}$ and $\NORM{1}'=1$.

Here are some examples of a Banach algebra.

\begin{itemize}
  \item Operator algebra: $M_n(\CC)$; $B(X)$, where $X$ is a Banach space;
  \item Algebra of functions: $C(X)$, where $X$ is a compact Hausdorff space; $C(\overline\DD)$; $A(\overline\DD)$; $H^{\infty}(\DD)$: the space of bounded analytic functions on $\DD$;
  \item Group algebras (with multiplication defined by convolution): $l^1(\ZZ)$; $L^1(\RR)$.
\end{itemize}

\begin{example}
  $L^1(\RR)$ does not admit an identity. Consider $f = \chi_{[0, 1]}$, and recall the fact that $L^1(\RR)$ functions are continuous with respect to translation. Similarly, $L^1(\TT)$ is a commutative Banach algebra without a identity.
\end{example}

\begin{proposition}
  Let $\BA{A}$ be a Banach algebra, then
  \begin{enumerate}
    \item If $x\in \BA{A}$ and $\NORM{x}<1$, then $1-x$ is invertible with inverse $(1-x)^{-1}=\sum_{n=0}^\infty x^n$, and we have $\NORM{(1-x)^{-1}}\leq (1-\NORM{x})^{-1}$;
    \item Suppose $x\in \BA{A}^\times$ and $h\in \BA{A}$ satisfies $\NORM{h}\leq\NORM{x^{-1}}^{-1}\DIV 2$. Then $x+h\in \BA{A}^{\times}$, with $\NORM{(x+h)^{-1}-x^{-1}}\leq 2\NORM{x^{-1}}^2\,\NORM{h}$;
    \item $\BA{A}^\times$ is open, and the map $x\mapsto x^{-1}$ defines a homeomorphism on $\BA{A}^\times$.
  \end{enumerate}
\end{proposition}

\begin{proof}
  For (2), observe that
  \begin{equation}
    (x+h)^{-1}-x^{-1}=(x+h)^{-1}(x-(x+h))x^{-1}=-(1+x^{-1}h)^{-1}x^{-1}hx^{-1}.
  \end{equation}
\end{proof}

Recall that if $X$ is a Banach space and $A\in B(X)$, then $A-\lambda$ is invertible if and only if $A-\lambda$ is bijective by the inverse mapping theorem.

\begin{definition}
  Let $\BA A$ be a Banach algebra and let $x\in \BA A$. Then we define the spectrum of $x$ to be the set
  \begin{equation*}
    \sigma(x)=\{\lambda\in \CC\VERT x-\lambda\notin \BA{A}^{\times}\},
  \end{equation*}
  and the spectral radius of $x$ to be
  \begin{equation*}
    r(x)=\sup\{\ABS{\lambda}\VERT \lambda\in \sigma(x)\}.
  \end{equation*}
\end{definition}

\begin{proposition}
  Let $\BA A$ be a Banach algebra and let $x\in \BA A$. Then
  \begin{enumerate}
    \item $\sigma(x)$ is not empty;
    \item $\sigma(x)$ is compact (or is both closed and bounded) and $r(x)\leq \NORM x$;
    \item $r(x)=\max\{\ABS{\lambda}\VERT \lambda\in \sigma(x)\}=\lim_{n\to\infty}\NORM{x^n}^{1\DIV n}$.
  \end{enumerate}
\end{proposition}

\begin{proof}
  For (1), assume $\sigma(x)$ is empty. Define $F\colon \lambda\in\CC\mapsto (x-\lambda)^{-1}\in \BA A$. Show by direct calculation that $F$ is differentialable (by the fact that inverse is continuous) and that $\lim_{\lambda\to\infty} \NORM{F(\lambda)}=0$. Then derive the contradiction using the Liouville's theorem (if $fF$ is constant for every $f\in B(\BA A)$, then $F$ must be constant).

  For (2), use the continuous function $\lambda\in\CC\to f-\lambda\in\BA A$ to show $\sigma(x)$ is closed, and use the equality $x-\lambda=\lambda(x\DIV\lambda-1)$ to show $r(x)\leq \NORM{x}$.

  For (3), first show that $\lambda\in\sigma(x)$ implies that $\lambda^n\in\sigma(x)$ by using the equality
  \begin{equation*}
    x^n-\lambda^n = (x-\lambda)(x^{n-1}+x^{n-2}\lambda+\dotsb+\lambda^{n-1}) = (x^{n-1}+x^{n-2}\lambda+\dotsb+\lambda^{n-1})(x-\lambda).
  \end{equation*}
  It follows that
  \begin{equation*}
    r(x)\leq \liminf \NORM{x^n}^{1\DIV n}
  \end{equation*}
  Next consider the series $-x^n\DIV \lambda^{n+1}$, which converges to $(x-\lambda)^{-1}$ when $\ABS{\lambda}>\NORM{x}$. It implies that for every $\varphi\in\BA A^*$ (and for such $\lambda$), the series $\varphi(-x^n\DIV \lambda^{n+1})$ converges. It follows from Banach-Steinhaus theorem that there is some $M_\lambda$ such that $\NORM{x^n\DIV \lambda^{n+1}}<M_\lambda$ for all $n$, which implies that
  \begin{equation*}
    r(x)\geq \limsup\NORM{x^n}^{1\DIV n}.
  \end{equation*}
\end{proof}

\begin{example}
  Let $V\colon C[a, b]\to C[a, b]$ sending $f$ to the function $x\mapsto \int_a^x f(t)\DIFF t$. Then $V\in B(C[a, b])$ and
  \begin{equation*}
    \ABS{V^nf(x)}\leq\frac1{n!}(x-a)^n\NORM{f}.
  \end{equation*}
  Hence we have $r(V)=0$, and thus $\sigma(V)=\{0\}$.
\end{example}

\begin{proposition}
  Let $\BA B$ be a Banach algebra and let $\BA A\subseteq \BA B$ be a closed subalgebra. Suppose $x\in\BA A$. Then
  \begin{enumerate}
    \item $r_{\BA A}(x) = r_{\BA B}(x) = \lim_{n\to\infty}\NORM{x^n}^{1\DIV n}$;
    \item $\partial\sigma_{\BA A}(x)\subseteq\sigma_{\BA B}(x)\subseteq \sigma_{\BA A}(x)$;
    \item If $\Omega$ is a connected component of $\CC - \sigma_{\BA B}(x)$, then either $\Omega\cap \sigma_{\BA A}(x)=\varnothing$ or $\Omega\subseteq \sigma_{\BA A}(x)$;
    \item Let $\{\Omega_n\}$ be the collection of connected components of $\CC - \sigma_{\BA B}(x)$ that intersects $\sigma_{\BA A}(x)$ (which is at most countable since $\CC - \sigma_{\BA B}(x)$ is locally path-connected and second-countable). Then
    \begin{equation}
      \sigma_{\BA A}(x)=\sigma_{\BA B}(x)\cup\bigcup \Omega_n.
    \end{equation}
  \end{enumerate}
\end{proposition}

We shall remark that (4) is a good tool to calculate the spectrum.

\begin{proof}
  For (2), suppose $\lambda\in\partial\sigma_{\BA A}(x)$, which is equivalent to the existence of a sequence $\lambda_n\notin \sigma_{\BA A}(x)$ such that $\lambda_n\to \lambda\in \sigma_{\BA A}(x)$. If $\lambda\notin \sigma_{\BA B}(x)$, then $(x-\lambda_n)^{-1}\to (x-\lambda)^{-1}$ in $\BA B$, and hence in $\BA A = \CL \BA A$, which contradicts to the fact that $\lambda\in\sigma_{\BA A}(x)$.

  For (3), let $X = \Omega\cap \sigma_{\BA A}(x)$. Observe that $(\Omega-X)\cap \sigma_{\BA A}(x)=\varnothing$, and apply
  \begin{equation*}
    \partial_\Omega X\subseteq\partial\sigma_{\BA A}(x)\subseteq\sigma_{\BA B}(x)\subseteq \CC-\Omega
  \end{equation*}
  to derive $\partial_\Omega X = \varnothing$.
\end{proof}

\begin{example}
  Let $K = \{z\in\CC\VERT 1\leq\ABS{z}\leq 2\}$, $C(K)$ be the Banach algebra of (complex valued) continous functions on $K$ with the uniform norm, $\BA A$ be the closed subalgebra generated by $\{1, z\}$, $\BA B$ be the closed subalgebra generated by $\{1, z, 1\DIV z\}$, and $f\in C(K)$ sending $z\in K$ to $z\in\CC$.

  Since we have $(z-\lambda)^{-1}=\sum_{n=0}^\infty \lambda^n\DIV z^{n+1}$ when $\ABS\lambda<\ABS z$, then $f-\lambda$ is invertible if $\ABS\lambda < 1$. Hence it's not hard to show $\sigma_{\BA B}(f)=K$.

  To show $\sigma_{\BA A}(f)=\{z\in\CC\VERT \ABS{z}\leq 2\}$, it suffices to show that $1\DIV z\notin \BA A$. Find a loop of $K$ such that the integral of $1\DIV z$ on the loop is nonzero while the integral of any polynomial is 0.
\end{example}

Let $x\in \BA A$ be an element of a Banach algebra, and let $f$ be a analytic function on a neighbourhood of $\{z\in\CC\VERT \ABS{z}\leq r(x)\}$ (containing $\{z\in\CC\VERT \ABS{z}\leq R\}$ for some $R>r(x)$). Then since
\begin{equation*}
  \sum_{n=0}^\infty \NORM{a_nx^n}\leq\sum_{n=0}^\infty\NORM{a_n}\,\NORM{x^n}=\sum_{n=0}^\infty \NORM{a_n}R^n(\frac{\NORM{x^n}^{1\DIV n}}{R})^n,
\end{equation*}
we may define an element $f(x)=\sum_{n=0}^\infty a_nx^n\in \BA A$. We have $(f\cdot g)(x)=f(x)g(x)$. Specifically, if $f$ is nowhere zero, then $f(x)$ is invertible in $\BA A$.

\begin{theorem}[Spectral Mapping Theorem]
  Let $x\in \BA A$ be an element of a Banach algebra, and let $f$ be a analytic function on a neighbourhood $\Omega$ of $\{z\in\CC\VERT \ABS{z}\leq r(x)\}$. Then $\sigma(f(x))=f(\sigma(x))$.
\end{theorem}

We present a proof here, and we will provide a clearer proof using Gelfand transformation.

\begin{proof}
  For "$\supseteq$", observe that $g(z)=(f(z)-f(\lambda))\DIV (z-\lambda)$ is analytic on $\Omega$. And hence if $x-\lambda$ is not invertible, then $f(x)-f(\lambda)$ is not invertible.

  For "$\subseteq$", we suppose $\mu\notin f(\sigma(x))$ and are to show $f(x)-\mu$ is invertible. Observe that $f(z)-\mu$ is nowhere zero on $\sigma(x)$. Hence we may write
  \begin{equation*}
    f(z)-\mu=g(z)(z-\mu_1)\dotsb(z-\mu_n),
  \end{equation*}
  where $g$ is analytic on $\Omega'$, and $\mu_1, \dotsc, \mu_n\in\Omega'-\sigma(x)$, with $\Omega'=\INT \{z\in\CC\VERT \ABS{z}\leq r(x)+\epsilon\}$ satisfying $\CL\Omega'\subseteq\Omega$. It follows that $f(x)-\mu$ is invertible, being a finite multiplication of invertible elements.
\end{proof}

\begin{theorem}[Gelfand-Mazur]
  If $\BA A$ is a Banach algebra which is also a division algebra, then there is a unique isomorphism $\BA A\to \CC$.
\end{theorem}

Introduction of Gelfand transformation
% multiplicative linear functional <-> maximal ideal

\begin{definition}
  Let $\BA A$ be a Banach algebra and $I$ be a subset of $\BA A$. $I$ is called to be an ideal of $\BA A$ only if it is a two-sided ideal of the ring $\BA A$ and is also a linear subspace of $\BA A$.
\end{definition}

Here are some elementry properties of ideals and quotients of a Banach algebra.

\begin{proposition}
  Let $\BA A$ be a Banach algebra.
  \begin{enumerate}
    \item If $I$ is a proper ideal of a Banach algebra $\BA A$, and $y\in I$, then $\NORM{1-y}\geq 1$, and hence $\CL I$ is also a proper ideal of $\BA A$.
    \item Every maximal ideal is closed, and every proper ideal is contained in some maximal ideal.
  \end{enumerate}

  Now let $I$ be a closed proper ideal of $\BA A$, and define a norm on $\BA A\DIV I$ by
  \begin{equation*}
    \NORM{\tilde x}_{\BA A\DIV I}=\inf_{y\in I}\NORM{x-y}_{\BA A}.
  \end{equation*}
  \begin{enumerate}[resume*]
    \item $\BA A\DIV I$ is also a Banach algebra under the norm just defined.
    \item The canonical projection map $p\colon\BA A\to \BA A\DIV I$ satisfies $\NORM{p}_{\symcal L(\BA A; \BA A\DIV I)}=1$.
  \end{enumerate}
\end{proposition}

\begin{proof}
  For (3), observe that
  \begin{equation*}
    \NORM{\tilde x_1\tilde x_2}=\inf_{y\in I}\NORM{x_1x_2-y}\leq\inf_{y_1, y_2\in I}\NORM{(x_1-y_1)(x_2-y_2)}\leq \NORM{\tilde x_1}\,\NORM{\tilde x_2},
  \end{equation*}
  and show that $\NORM{1}_{\BA A\DIV I}=1$.
\end{proof}

\begin{definition}
  Let $\BA A$ be a Banach algebra. Then a multiplicative linear functional $\varphi$ on $\BA A$ is a nonzero algebra morphism $\varphi\colon\BA A\to \CC$. Denote the set of multiplicative linear functionals of $\BA A$ by $\Delta$. If $\Delta$ is nonempty, then it has a topological structure inherited from $\BA A^*$ equipped with the $\WS$ topology.
\end{definition}

\begin{proposition}
  Let $\BA A$ be a Banach algebra and assume $\Delta$ to be nonempty. Let $\varphi\in \Delta$ be a multiplicative linear functional.
  \begin{enumerate}
    \item $\varphi$ is continuous, with $\NORM{\varphi}=1$.
    \item The closed ideal $\ker\varphi$ is a maximal ideal of $\BA A$.
    \item If $\varphi_1, \varphi_2\in\Delta$ are distinct multiplicative linear functionals, then $\ker\varphi_1\neq \ker\varphi_2$.
    \item $\Delta$ is compact, where $\Delta$ is equipped with the $\WS$ topology.
    \item If $x\in\BA A$, then
    \begin{equation*}
      \{\varphi(x)\VERT \varphi\in\Delta\}\subseteq \sigma(x).
    \end{equation*}
  \end{enumerate}
\end{proposition}

\begin{proof}
  For (1), observe that $1-x\DIV \varphi(x)\in\ker\varphi$ cannot be invertible.

  For (2), note that $\varphi$ is a epimorphism onto a field. Thus $\ker\varphi$ is maximal among the left (or right) ideals.

  For (3), consider the element $x-\varphi_1(x)$.

  For (4), by the Banach-Alaogue theorem, it suffices to show that $\Delta$ is closed, which is a routine verification.

  For (5), observe that $x-\varphi(x)\in\ker\varphi$.
\end{proof}

\begin{proposition}
  Let $\BA A$ be a commutative Banach algebra.
  \begin{enumerate}
    \item For every maximal ideal $M$, there exists a unique multiplicative functional $\varphi$ that satisfies $M=\ker\varphi$.
    \item There is a bijection between $\Delta$ and the set of all maximal ideals of $\BA A$.
    \item If $\varphi\in\Delta$ and $x\in\BA A$, then
    \begin{equation*}
      \{\varphi(x)\VERT \varphi\in\Delta\}= \sigma(x).
    \end{equation*}
  \end{enumerate}
\end{proposition}

\begin{proposition}
  Let $\BA A$ be a Banach algebra such that $\Delta$ is nonempty. For $x\in \BA A$, define a function $\hat x$ on $\Delta$ that maps $\varphi\in\Delta$ to $\varphi(x)\in\CC$.
  \begin{enumerate}
    \item $\hat x$ is always continuous. Thus we've defined a map $x\in\BA A\mapsto \hat x\in C(\Delta)$. The map is denoted by $\Lambda$, and is called the Gelfand transform on $\BA A$.
    \item $\Lambda$ is a continuous algebra morphism.
    \item $\ker\Lambda=\RAD\BA A$ is the intersection of all maximal ideals of $\Delta$.
    \item $x$ in invertible in $\BA A$ if and only if $\hat x$ is invertible in $C(\Delta)$.
    \item $\Lambda$ preserves the norm if and only if for every $x\in \BA A$ there is $\NORM{x^2}=\NORM{x}^2$.
  \end{enumerate}
\end{proposition}

\begin{proof}
  For (2), observe that $\NORM{\hat x}_\infty=r(x)\leq \NORM{x}$.
\end{proof}

\begin{example}
  $C(X)$, $l^1(\ZZ)$.
\end{example}

\end{document}
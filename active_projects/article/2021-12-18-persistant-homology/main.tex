% -------------------- Packages --------------------

\PassOptionsToPackage{quiet}{fontspec}
\documentclass{assignment}[2019/10/15]
% \usepackage{packages}[2019/11/14]
\usepackage{amsmath}
\usepackage{hyperref}
\usepackage{ntheorem}
\usepackage{enumitem}
\usepackage[%
  backend=biber,%
  % style=gb7714-2015,gbpub=false,gbnamefmt=lowercase,%
  sortcites=true,doi=false,%
  defernumbers=true,%
  ]{biblatex} % 需要 biblatex-gb7714-2015
\bibliography{refs}

% -------------------- Settings --------------------

\RequirePackage{unicode-math}
\setmainfont{TeX Gyre Pagella}
\setmathfont{Asana Math}
\unimathsetup{%
  math-style = ISO,
  bold-style = ISO,
  nabla = upright,
  partial = upright,
}
\setmathfont{texgyretermes-regular}[
  Extension = .otf,
  range = up/{num},
  BoldFont = texgyretermes-bold,
]
\setmathfont{texgyretermes-bold}[
  Extension = .otf,
  range = bfup/{num},
  BoldFont = texgyretermes-bold,
]
\setmathfont{Asana Math}[
  % Extension = .otf,
  % BoldFont = \tju@font@name@xits@math@bf,
  StylisticSet = 8,
]
\setmathfont{Asana Math}[
  % Extension = .otf,
  % BoldFont = \tju@font@name@xits@math@bf,
  Style = Alternate,
  range = {cal, bfcal},
]
\setmathfont{Asana Math}[
  % Extension = .otf,
  % BoldFont = \tju@font@name@xits@math@bf,
  range = {scr, bfscr},
]
\setmathfont{Asana Math}[
  % Extension = .otf,
  % BoldFont = \tju@font@name@xits@math@bf,
  range = it/{greek,Greek,latin,Latin},
]
\setmathfont{Asana Math}[
  % Extension = .otf,
  % BoldFont = \tju@font@name@xits@math@bf,
  range = up/{greek,Greek,latin,Latin},
]
\setmathfont{Asana Math}[
  % Extension = .otf,
  % BoldFont = \tju@font@name@xits@math@bf,
  range = {
    "2212,"002B,"003D,"0028,"0029,"005B,%
    "005D,"221A,"2211,"2248,"222B,"007C,"2026,"2202,"00D7,"0302,%
    "2261,"0025,"22C5,"00B1,"2194,"21D4%
  },
]


% Title

\title{Persistent Homology}
\author{Chen Xuyang}
\date{\today}
\institute{School of Mathematical Science}
\professor{L\"u Zhi}
\course{Algebraic Topology}
\subject{Algebraic Topology}
\keywords{}

% -------------------- New commands --------------------

% \newcommand{\me}{\symup{e}}
% \newcommand{\BR}{\symbb{R}}
% \newcommand{\BZ}{\symbb{Z}}
% \newcommand{\BN}{\symbb{N}}
% \newcommand{\diag}{\mathop{}\!\symup{diag}}
% \newcommand{\pr}{\mathop{}\!\symup{Pr}}
% \newcommand{\expect}{\mathop{}\!\symup{E}}
% \newcommand{\cov}{\mathop{}\!\symup{Cov}}
% \newcommand{\var}{\mathop{}\!\symup{Var}}

% \def\multiset#1#2{\ensuremath{\left(\kern-.3em\left(\genfrac{}{}{0pt}{}{#1}{#2}\right)\kern-.3em\right)}}

\newcommand{\lr}[3]{\left#1#3\right#2}
\newcommand{\lmr}[5]{\left#1#4\middle#2#5\right#3}

\theoremstyle{plain}
\theoremheaderfont{\upshape\bfseries}
\theorembodyfont{\upshape}
\theoremseparator{}
\theoremsymbol{}
\newtheorem{theorem}{Theorem}
\newtheorem{definition}[theorem]{Definition}
\newtheorem{proposition}[theorem]{Proposition}
\newtheorem{lemma}[theorem]{Lemma}
\newtheorem{corollary}[theorem]{Corollary}
\newtheorem{example}[theorem]{Example}

% \newcommand{\SC}{\mathscr{C}}
% \newcommand{\BC}{\symbb{C}}

% \newcommand{\Kernal}{\mathop{}\!\symup{Ker}}
% \newcommand{\Image}{\mathop{}\!\symup{Im}}
% \newcommand{\Hom}{\mathop{}\!\symup{Hom}}
% \newcommand{\Aut}{\mathop{}\!\symup{Aut}}
% \newcommand{\Type}{\mathop{}\!\symup{Type}}
% \newcommand{\GL}{\mathop{}\!\symup{GL}}
% \newcommand{\sgn}{\mathop{}\!\symup{sgn}}
% \newcommand{\id}{\text{id.}}
% \newcommand{\blank}{{\cdot}}
% \newcommand{\tr}{\mathop{}\!\symup{tr}}
% \newcommand{\Cl}{\mathop{}\!\symup{Cl}}

\newcommand{\RR}{\symbb{R}}
\renewcommand{\emph}[1]{\textbf{#1}}
\newcommand{\CECH}{\u{C}ech}
\newcommand{\DIV}{{\divslash}}
\DeclareMathOperator{\CL}{cl}

% \numberwithin{equation}{section}

% -------------------- Document --------------------

\begin{document}
  \maketitle
  % \tableofcontents

  Persistent homology is a method to extract global information from cloud point data, by calculating homology groups of a finite sequence of simplicial complexes, and by examining how the free generator of the homology group transform along the sequence.

  A \emph{cloud of data} is a finite set of points $S$ in an Euclidean space, i.e., $S\subseteq \RR^n$. When $n$ is small (e.g. $n=2, 3$), we may easily find the pattern of a cloud of data from our naked eyes. But when $n$ is large, which is the most common case, we need some tools to distinguish patterns and noises, where homology theories come to help.

  We want to approximate the data using simplicial complexes. Here are two natural way to do this. Let $\varepsilon>0$ be the scale that we are looking at the data.

  \begin{definition}
    A \emph{{\CECH} complex} $C_\varepsilon$ is an abstract simplicial complex that can be described as follows: a subset $\sigma\subseteq S$ is a face of $C_\varepsilon$ if and only if the closed $\varepsilon\DIV 2$ neighbourhoods of the points of $\sigma$ have a common intersection, i.e.,
    \begin{equation*}
      \bigcap_{x\in\sigma} \CL B_{\varepsilon\DIV 2}(x)\neq \varnothing.
    \end{equation*}
  \end{definition}

  \begin{definition}
    A \emph{Vietoris-Rips complex} $R_\varepsilon$ is an abstract simplicial complex such that its $k$-faces are exactly the $(k+1)$-subsets $\sigma$ of $S$ that are pairwise within distance $\varepsilon$, i.e.,
    \begin{equation*}
      \CL B_{\varepsilon\DIV 2}(x)\cap \CL B_{\varepsilon\DIV 2}(y)\neq \varnothing
    \end{equation*}
    for every $x, y\in\sigma$.
  \end{definition}

  Though it may only have a geometric realization in higher dimensional, a {\CECH} complex preserves some intuitive properties of $S$, as it has the same homotopy type of the union of all $\varepsilon\DIV 2$-radius closed ball of the points of $S$. But a {\CECH} is generally hard to compute.

  A Vietoris-Rips complex is a flag complex, i.e., it is maximal with respect to its 1-dimensional skeleton. So it's much easier to compute. In general, the homology type of a Vietoris-Rips complex is hard to tell.

  \begin{lemma}
    There is a chain of inclusion maps
    \begin{equation}
      R_\varepsilon\to C_{\sqrt{2}\varepsilon}\to R_{\sqrt{2}\varepsilon}
    \end{equation}
  \end{lemma}

  The lemma shows us if a topological property of a Vietoris-Rips complex is persistent under the inclusion map $R_\varepsilon\to R_{\varepsilon'}$ for $\varepsilon'\geq \sqrt{2}\varepsilon$, then it's a topological property of a {\CECH} complex. Hence it suffices to compute the homology of Vietoris-Rips complexes.

  If $\varepsilon$ is very small, the complex is nothing but a disjoint union of the points. If $\varepsilon$ is very large, the complex is just a single simplex. When $\varepsilon$ goes from small to large, some ``holes" emerge and then disappear. These holes are the key topological feature we want to extract by calculating the homology groups for different $\varepsilon$, which is the key idea of persistent homology. We'll track the generators of the homology group. If it is quickly nullified, then it may be recognized as a noise. If it lasts significantly long, then it's possible that this is an important global property of the data.

  \begin{definition}
    Let $0<\varepsilon_1<\varepsilon_2<\dotsb$ be a strictly increasing sequence, and let $R_j$ denote the Vietoris-Rips complex $R_{\varepsilon_j}$ and $C(R_j)$ the chain complex consisting of the simplicial chain groups of $R_j$. Then we have the homology groups $H_p(C(R_j))$, and the natural map $H_p(C(R_j))\to H_p(C(R_k))$ for $j<k$ induced by the inclusion map. The image of $H_p(C(R_j))$ in $H_p(C(R_k))$ is called the $(j, k)$-\emph{persistent homology group}.
  \end{definition}

  Denote $H_p(C(R))$ to be the direct sum of $H_p(C(R_j))$ for $j=1, 2, \dotsc$. If the coefficient of the chain complexes is a field $F$, then there is a natural graded $F[x]$-module structure on $H_p(C(R; F))$, where $\sigma\in H_p(C(R_j; F))$ is sended to $i_*(\sigma)\in H_p(C(R_{j+1}; F))$. Apply the structure theorem of finitely-generated modules over a P.I.D., we have the following structure theorem for persistent homology.

  \begin{theorem}[Structure Theorem]
    \begin{equation*}
      H_p(C(R; F))\cong \bigoplus_{j}x^{t_j}\cdot F[x]\oplus\bigoplus_kx^{r_k}\cdot (F[x]\DIV (x^{s_k}\cdot F[x])).
    \end{equation*}
  \end{theorem}

  The structure theorem has a natural interpretation: A free generator $x^{t_j}$ represents a homology generator of $H_p(C(R_{t_j}; F))$ that persists for all future value of $\epsilon$, while a torsion generator $x^{r_k}$ represents a homology generator of $H_p(C(R_{r_k}); F)$ that disappear after $s_k-r_k$ steps.

  The birth and death of generators can reflect many properties of the data, say if there are about 2 significant generators in $p$-th persistent homology group, then we'll guess the $p$-th Betti number of the data to be about 2. By gathering all these numbers extracted from the data, we may have a good guess of the structure of the data.

  \nocite{*}
  \sloppy\hbadness 10000\relax%
  \printbibliography[heading=bibintoc]%
  \fussy%
  % \section*{Reference}
  % \begin{enumerate}[1]
  %     \item Thomas W. Hungerford. Algebra. \url{https://doi.org/10.1007/978-1-4612-6101-8}.
  %     \item B. Steinberg. Representation Theory of Finite Groups: An Introductory Approach. \\\url{https://doi.org/10.1007/978-1-4614-0776-8}.
  %     \item R. McNamara. Irreducible Representations of the Symmetric Group. \\\url{http://math.uchicago.edu/~may/REU2013/REUPapers/McNamara.pdf}.
  %     \item Bruce Sagan. The Symmetric Group. \url{https://doi.org/10.1007/978-1-4757-6804-6}
  % \end{enumerate}
\end{document}

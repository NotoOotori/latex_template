% !Mode::"TeX:UTF-8"
\documentclass[a4paper,12pt]{ctexart}
\usepackage{Constants}
\usepackage{amsmath}
%\usepackage{amsthm} %定理格式 由ntheorem代替
\usepackage{amssymb}
\usepackage[thmmarks, amsmath, thref]{ntheorem}
\usepackage{DefaultTheoremStyle}
\usepackage{lastpage}
\usepackage{makecell} %表格线加粗 \Xhline{1.2pt}
\usepackage{boldline} %长表格表格线加粗
\usepackage{multirow} %合并单元格
\usepackage{array}
\usepackage{longtable} %长表格
\usepackage[dvipsnames]{xcolor} %颜色声明
\usepackage{varioref} %For Cross References
\renewcommand{\reftextbefore}
    {on the \reftextvario{preceding page}{page before}}
\renewcommand{\reftextafter}
    {on the \reftextvario{following}{next} page}
\renewcommand{\reftextfacebefore}
    {on the \reftextvario{facing}{preceding} page}
\renewcommand{\reftextfaceafter}
    {on the \reftextvario{facing}{next}{page}}
\renewcommand{\reftextfaraway}[1]
    {在第\pageref{#1}页}
\usepackage{caption} %题注
\captionsetup{margin    =   6pt,
              font      =   small,
              labelfont =   bf}
\usepackage{fancyhdr} %脚注
\setlength{\headheight}{15pt}
\lhead{Copyright \copyright\ \AUTHOR}
\rhead{Page \thepage\ of \pageref{LastPage}}
\usepackage[square, numbers, sort&compress]{natbib} %引用
\renewcommand{\bibsection}{} %不显示"Reference"
\usepackage{hyperref}
\hypersetup{linktoc             =   all,
            colorlinks          =   true,
            linkcolor           =   TealBlue,
           %anchorcolor         =   Black,
            citecolor           =   Black,
           %filecolor           =   Cyan,
           %menucolor           =   Red,
           %runcolor            =   filecolor,
            urlcolor            =   magenta,
            pdfinfo             =   {
                Title           =   {\TITLE},
                Author          =   {\AUTHOR},
                Subject         =   {\SUBJECT}},
            bookmarksnumbered   =   true,
            pdfstartview        =   FitH,
            pdfpagelayout       =   OneColumn}
\usepackage[section]{placeins} % 使图像不会显示在别的部分 若过于严格则换成[below]\
\usepackage{graphicx}
\graphicspath{{figures/}} %图像文件目录
\usepackage[section]{placeins} % 使图像不会显示在别的部分 若过于严格则换成[below]
%\usepackage{fontspec} % 字体
\usepackage{titlesec} %Section标题格式
\usepackage{SUBSubsubsection}
\usepackage{authblk} %作者
\usepackage{stackrel} %上下写
\usepackage{mathtools} %\xleftrightarrow
%\usepackage{enumitem} 用enumerate包代替
\usepackage{listings} %排版程序语言
\usepackage{enumerate}
\usepackage{flafter} %不让float出现在定义之前的地方
\usepackage{float} %你们这帮float给我乖乖听话 HHHHHHHHHHH
\usepackage{pgfplots}
\pgfplotsset{width=7cm}
\usepackage{bigfoot} % to allow verbatim in footnote
\usepackage[numbered, framed]{matlab-prettifier}
\usepackage{filecontents}
\usepackage[all,cmtip]{xy} % Commutive diagram.
\usepackage{lineno} % Line numbers.
\newcommand*\diff{\mathop{}\!\mathrm{d}}
\newcommand*\Beta{\mathrm{B}}

%===============TITLE===============
\title{\TITLE}
\author{\AUTHOR}
\date{\today}

\begin{document}
    \maketitle
    \thispagestyle{empty}
    \newpage

    \pagestyle{fancy}
    \pagenumbering{arabic}
    \linenumbers
    
    \begin{problem}
        记$\displaystyle a_n=\int_{0}^{1}{\frac{x^n}{\sqrt[3]{1-x}}\diff x}$,
        $\displaystyle b_n=\int_{\delta}^{1}{\frac{x^n}{\sqrt[3]{1-x}}\diff x}$.
        证明:
        \[\lim_{n\to \infty}{\frac{b_n}{a_n}}=
        \lim_{n\to \infty}{\frac{a_{n+1}}{a_n}}=
        \lim_{n\to \infty}{\sqrt[n]{a_n}}=1.\]
    \end{problem}

    \begin{solution}
        分别证明三个极限等于1.
        
        \begin{enumerate}[Step i.]
        \item 证明$\sqrt[n]{a_n}\to 1$.\\
            根据Beta函数定义, 可知$a_n=\Beta(n+1, 2/3)$, 故有通项公式
            \begin{equation}
                a_n = \frac{3}{2}\prod_{j=1}^{n}{\frac{3j}{3j+2}}.
            \end{equation}

            首先有
            \begin{equation}
                \label{angeq}
                a_n 
                \geq \frac{3}{2}\prod_{j=1}^{n}{\frac{3j}{3j+3}}
                = \frac{3}{2(n+1)}
            \end{equation}

            其次我们有
            \begin{equation}
            \begin{aligned}
                a_n &= \frac{\Gamma(n+1)\Gamma(2/3)}{\Gamma(n+5/3)}\\
                &= \frac{n!\cdot\Gamma(2/3)}{(2/3)\cdot(5/3)\cdot\dotsm
                    \cdot(n+2/3)\cdot\Gamma(2/3)}\\
                &\leq \frac{3}{2}\cdot n!.
            \end{aligned}
            \end{equation}
            故
            \begin{equation}
                1\leftarrow \sqrt[n]{3/(2(n+1))} \leq a_n \leq \sqrt[n]{2/3\cdot n!} \to 1.
            \end{equation}
        \item 证明$b_n/a_n\to 1.$\\
            记$\displaystyle f_n(x)=\frac{x^n}{\sqrt[3]{1-x}}$, $x\in [0, 1)$.
            因为$f_n>0$, 由单调收敛定理和黎曼瑕积分定义可知
            \begin{equation}
                (L)\int_{0}^{1}f_n = (R)\int_{0}^{1}f_n.
            \end{equation}
            
            故
            \begin{equation}
            \begin{aligned}
                a_n-b_n
                &=(L)\int_{0}^{\delta}{f_n(x)\diff x}\\
                &\leq (L)\int_{0}^{\delta}{f_n(\delta)\diff x}\\
                &= (R)\int_{0}^{\delta}{f_n(\delta)\diff x}\\
                &=\frac{\delta^{n+1}}{\sqrt[3]{1-\delta}}.
            \end{aligned}
            \end{equation}

            根据式(\ref{angeq})有
            \begin{equation}
                \frac{a_n-b_n}{a_n}
                \leq\frac{\delta^{n+1}/(\sqrt[3]{1-\delta})}{3/(2(n+1))}
                \to 0
            \end{equation}

            即$b_n/a_n\to 1$.
        \item 证明$a_{n+1}/a_n\to 1$.\\
            因为$a_n=\Beta(n+1, 2/3)$, 所以
            \begin{equation}
            \begin{aligned}
                \frac{a_{n+1}}{a_n}
                &= \frac{\Beta(n+2, 2/3)}{\Beta(n+1, 2/3)}\\
                &= \frac{n+1}{n+1+2/3}\\
                &\to 1.
            \end{aligned}
            \end{equation}
        \end{enumerate}
    \end{solution}

\end{document}

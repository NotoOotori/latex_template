\subsection{问题重述}
    某学科大型国际学术会议及其附属卫星会议今年7-8月在中国召开,
    具体日程, 参会基本要求以及相关费用如下表(按照会议开始时间排序).
    \begin{table}[htb]\scriptsize
        \begin{center}
        \caption{会议日程, 基本要求和相关费用}
            \begin{tabular}{ccccccccc}
                \Xhline{1.2pt}
                \multicolumn{1}{m{1.2cm}}{\centering 会议时间}
                    & \multicolumn{1}{m{0.6cm}}{\centering 地点}
                    & \multicolumn{1}{m{1.5cm}}{\centering 住宿费用 \\ (元/人/天)}
                    & \multicolumn{1}{m{1.2cm}}{\centering 注册费用 \\ (元/人)}
                    & \multicolumn{1}{m{1.0cm}}{\centering 会场 \\ 交通费 \\ (元/人)}
                    & \multicolumn{1}{m{1.0cm}}{\centering 最低 \\ 总人数}
                    & \multicolumn{1}{m{1.2cm}}{\centering 最低(副)教授人数}
                    & \multicolumn{1}{m{1.2cm}}{\centering 最低 \\ 教授人数}
                    & \multicolumn{1}{m{1.0cm}}{\centering 会议 \\ 影响力} \\
                \hline
                7.20-7.25 & 北京 & 650 & 1000 & 200 & 3 & 2 & 2 & 5\\
                7.21-7.26 & 上海 &   0 &  900 &  80 & 3 & 1 & 1 & 5\\
                7.22-7.26 & 广州 & 550 &  500 & 150 & 2 & 1 & 0 & 4\\
                7.26-7.28 & 兰州 & 400 &  400 & 100 & 2 & 1 & 0 & 3\\
                7.26-7.28 & 成都 & 460 &  400 & 100 & 2 & 1 & 0 & 4\\
                7.29-7.31 & 昆明 & 480 &  400 & 100 & 2 & 1 & 0 & 3\\
                8.01-8.03 & 南京 & 490 &  400 & 150 & 2 & 1 & 0 & 3\\
                8.02-8.04 & 厦门 & 500 &  400 & 150 & 2 & 1 & 0 & 4\\
                8.03-8.06 & 杭州 & 500 &  400 & 200 & 2 & 1 & 0 & 3\\
                8.06-8.08 & 济南 & 450 &  400 & 100 & 2 & 1 & 0 & 4\\
                8.07-8.09 & 天津 & 480 &  400 & 100 & 2 & 1 & 0 & 3\\
                8.07-8.10 & 咸阳 & 320 &  300 & 100 & 2 & 1 & 1 & 3\\
                8.08-8.10 & 大连 & 490 &  500 & 150 & 2 & 1 & 1 & 4\\
                \Xhline{1.2pt}
            \end{tabular}
        \end{center}
    \end{table}
    
    为了了解国际最新的研究动态以及提升同济大学影响力,
    学院要求教研室组织教师积极报名参加这次会议.
    教研室中包含5名教授(包括主任和副主任), 8名副教授和5名讲师,
    其中主任和副主任至多参加三个会议, 其余老师没有限制.
    城市之间可用的交通方式为火车和飞机, 价格仅与里程相关, 并且与里程数成正比.
    
    在论文中, 我们需要一步步地给出不但总费用较小, 而且较能彰显我校在该学科影响力的参会安排.
    建模过程分为如下四个步骤(每步内的假设相互独立):
    \begin{enumerate}
        \item 在外加各教师最低参会数量的约束下, 仅考虑总费用最低, 给出此时最佳的参会安排.
        \item 在不考虑每个会议最低人员要求, 但是总费用不超过50000元的前提下,
                给出学校影响力尽可能大的参会安排.
                即多派职称高的教师去影响力大的会议.
        \item 以被选为大会报告的期望作为衡量在会议中我校学科影响力的关键因素, 简单考虑经费因素,
                给出最优方案.(参加同一会议的教师, 有至少一人的报告被选为大会报告的概率见下表)
            \begin{table}[htb]\footnotesize
                \begin{center}
                    \caption{被选为大会报告的概率}
                    \begin{tabular}{cc}
                        \Xhline{1.2pt}
                        情况 & 概率\\
                        \hline
                        有两位教授 & 0.75\\
                        有一位教授和一位副教授 & 0.50\\
                        有一位教授或两位副教授 & 0.35\\
                        其它 & 0.10\\
                        \Xhline{1.2pt}
                    \end{tabular}
                \end{center}
            \end{table}
        \item 根据之前三个步骤的结果, 全面考虑经费以及影响力因素, 得出科学的参会安排,
                并且给每一位教师打印出行日程和经费预算.
    \end{enumerate}

\subsection{解题途径}
    经检索, 题目给出的火车费用约为高铁一等座和二等座的平均值,
    因此, 我们假定题目中的火车即为高铁,
    并且统计了城市之间是否有高铁以及高铁的实际里程
    (见下表, 其中$+\infty$的里程表示高铁不可达) \citep{Huochepiao}.
    
    \begin{table}[htb]\scriptsize
        \begin{center}
            \caption{高铁里程表}
            \scalebox{0.95}{
            \begin{tabular}{cccccccccccccc}
                \Xhline{1.2pt}
                & 北京& 上海& 广州& 兰州& 成都& 昆明& 南京& 厦门& 杭州& 济南& 天津& 咸阳& 大连\\
                \hline
                北京\\
                上海& 1318\\
                广州& 2298& 1790\\
                兰州& 1784& 2185& 2687\\
                成都& 1874& 1976& $+\infty$& $+\infty$\\
                昆明& 2760& 2252& 1330& $+\infty$& 1112\\
                南京& 1023&  295& 1581& 1782& 1872& 2349\\
                厦门& 2053& 1085& $+\infty$& $+\infty$& $+\infty$& 2337& 1182\\
                杭州& 1279&  159& 1631& 2038& $+\infty$& 2093& 256& 993\\
                济南&  406&  912& 2251& 1737& 1830& 2713& 608& 1647& 873\\
                天津&  120& 1196& 2309& $+\infty$& 1885& $+\infty$& 901& 1931& 1157& 284\\
                咸阳& 1246& 1539& $+\infty$& 538& $+\infty$& $+\infty$& 1244& $+\infty$& $+\infty$
                    & 1199& $+\infty$\\
                大连&  963& 2054& $+\infty$& $+\infty$& $+\infty$& $+\infty$& 1759& $+\infty$& $+\infty$
                    & 1142&  836& $+\infty$\\
                \Xhline{1.2pt}
            \end{tabular}}
        \end{center}
    \end{table}
    
    飞机方面, 因为未能检索到航班的具体里程, 并且考虑到飞机飞行的高度相对于地球半径而言可以忽略不计,
    所以我们在百度地图上$1:20,000,000$的比例尺下,
    用工具箱中的测距工具测得了两个城市之间的球面距离,
    记录在下表(并假设任意两城市间都存在密集的航班) \citep{Baidu_Map}.
    
    \begin{table}[htb]\scriptsize
        \begin{center}
            \caption{飞机里程表}
            \scalebox{0.95}{
            \begin{tabular}{cccccccccccccc}
                \Xhline{1.2pt}
                & 北京& 上海& 广州& 兰州& 成都& 昆明& 南京& 厦门& 杭州& 济南& 天津& 咸阳& 大连\\
                \hline
                北京\\
                上海& 1070\\
                广州& 1893& 1216\\
                兰州& 1182& 1716& 1697\\
                成都& 1528& 1669& 1232&  620\\
                昆明& 2106& 1955& 1068& 1245& 645\\
                南京&  898&  270& 1130& 1451& 1402& 1746\\
                厦门& 1726&  828&  513& 1878& 1536& 1545&  845\\
                杭州& 1121&  174& 1043& 1650& 1540& 1807&  235& 678\\
                济南&  360&  733& 1550& 1196& 1380& 1889&  535& 1357& 762\\
                天津&  107&  963& 1811& 1222& 1518& 2080&  793& 1629& 1014&  271\\
                咸阳&  921& 1250& 1321&  481&  598& 1191&  968& 1417& 1167&  801&  922\\
                大连&  461&  853& 1928& 1598& 1842& 2346&  799& 1631&  967&  472&  380& 1259\\
                \Xhline{1.2pt}
            \end{tabular}}  
        \end{center}
    \end{table}
    \clearpage
    
    本题的各个任务均为规划问题, 要求我们在满足一定约束的情况下,
    寻求既影响力高, 又能照顾到经费的最佳方案.
    考虑到路费是一个离散量, 因此零一整数规划模型在本题并不适用,
    于是我们想到了另一个常用的优化方法---动态规划模型.
    现代意义的动态规划由Richard E. Bellman在1953年提出,
    指将大决策分解成嵌套在其中的小决策, 并从小到大递推求解的方法 \citep{R.Bellman_2002}.
    动态规划中, 有如下的一些基本概念:\citep{H.W.Zhang_2008}
    \begin{itemize}
        \item \textbf{阶段}为系统需要做出决策的步骤.
                我们需要把系统顺序地向前发展划分为若干个相互联系的阶段,
                使能按阶段的次序求解.
                描述阶段的变量称为\textbf{阶段变量}.
                阶段变量的主要作用是按顺序编出所研究过程划分的编号.
        \item \textbf{状态}为每个阶段开始时面临的自然状况或客观条件.
                过程的状态通常可以用一个或一组变数描述,称为\textbf{状态变量}.
                状态变量取值的集合称为\textbf{状态集合}.
                此外, 状态具有\textbf{无后效性}: 未来的收益仅取决于当前的状态,
                并不依赖于过去的状态和决策的历史.
        \item \textbf{决策}为从当前阶段的状态到下一阶段的状态时所做的决定.
                描述决策的变量称为\textbf{决策变量}.
                所以, 系统的状态必须包含在某个给定的阶段上, 
                并且确定全部允许决策所需的全部信息.
        \item \textbf{策略}是由每一阶段的决策组成的决策函数序列.
                对于每个实际的多阶段决策过程,
                可供选取的策略有一定的范围限制,
                这个范围称为\textbf{允许策略集合},
                允许策略集合中达到最优效果的策略称为\textbf{最优策略}.
        \item 利用动态规划解决优化问题时, 所研究的是逐阶段的决策过程.
                给定第$k$阶段状态变量的值后, 如果这一阶段的决策变量一经确定,
                第$k+1$阶段的状态变量也就完全确定,
                即$k+1$阶段的状态变量随$k$阶段的状态变量和策略的变化而变化,
                可以把这一关系看成$k$阶段的状态变量和策略与$k+1$阶段的状态变量确定的对应关系.
                这是从$k$阶段到$k+1$阶段的状态转移规律,
                称为\textbf{状态转移方程}.
    \end{itemize}
    
    任务一和任务二中, 我们均采用了动态规划, 分别建立了费用最优化模型和影响力最优化模型.
    并在任务二中, 我们通过对问题的抽象,
    将问题转化为了动态规划中的分组背包模型(不同于常见的分组01背包模型).
    
    任务三中, 我们定义了星级期望, 建立了影响力期望最优化模型,
    并采用贪心算法, 从被选为大会报告的概率高的老师组合先考虑,
    再以概率从大到小考虑其余的老师组合.
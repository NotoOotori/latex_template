% !Mode::"TeX:UTF-8"

% -------------------- Information --------------------

\newcommand{\TITLE}{幻灯片模板}
\newcommand{\AUTHOR}{Jason}
\newcommand{\SUBJECT}{幻灯片}
\newcommand{\INSTITUTE}{同济大学数学科学学院}

% -------------------- Packages --------------------

\documentclass{ctexbeamer}
\usepackage{amsmath}
\usepackage{amssymb}
\usepackage{commath} % abs, norm
\usepackage[mathscr]{euscript}
\usepackage{float} % 你们这帮float给我乖乖听话 HHHHHHHHHHH.
\usepackage{graphicx}
\usepackage{hyperref}
\usepackage{mathtools} % \xleftrightarrow.
\usepackage{tikz}
\usepackage[printwatermark]{xwatermark} % Foreground Watermarks.
\usepackage[all, cmtip]{xy}

% -------------------- Settings --------------------

% Title

\title{\TITLE}
\author{\AUTHOR}
\date{\today}
\institute{\INSTITUTE}
\titlegraphic{\includegraphics[width=0.1\paperwidth]{wallpaper.jpg}}

% Package: beamer

\transpush<2>[duration=2]
\usetheme{Berlin}

% Package: ctex

\setCJKfamilyfont{fzstk}{FZShuTi} % 方正舒体
\newcommand{\fzstk}{\CJKfamily{fzstk}}

% Package: graphicx

\graphicspath{{resources/}} %图像文件目录

% Package: hyperref

\hypersetup{
    linktoc             =   all,
    colorlinks          =   true,
    linkcolor           =   cyan,
    anchorcolor         =   black,
    citecolor           =   green,
    filecolor           =   cyan,
    menucolor           =   red,
    runcolor            =   filecolor,
    urlcolor            =   magenta,
    pdfinfo             =   {
        Title           =   {\TITLE},
        Author          =   {\AUTHOR},
        Subject         =   {\SUBJECT}},
    bookmarksnumbered   =   true,
    pdfstartview        =   Fit,
    pdfpagelayout       =   OneColumn
}

% Package: xwatermark

\newsavebox\mybox
\savebox\mybox{\tikz[color=cyan, opacity=0.2]\node{\fzstk\SUBJECT};}
\newwatermark*[
    allpages,
    angle=45,
    scale=6,
    xpos=-20,
    ypos=15
]{\usebox\mybox}

% -------------------- General new commands --------------------

\DeclareMathAlphabet{\mathsfsl}{OT1}{cmss}{m}{sl}

\newcommand{\diff}{\mathop{}\!\mathrm{d}}
\newcommand{\matr}[1]{\ensuremath{\mathsfsl{#1}}} % italic sans serif
\newcommand{\me}{\mathrm{e}}
\newcommand{\mi}{\mathrm{i}}
\newcommand{\restrict[1]}{\raisebox{-.5ex}{$|$}_{#1}}

% -------------------- Specific new commands --------------------

\newcommand{\mahjong}{\mathscr{M}}
\newcommand{\point}{\mathrm{point}}
\newcommand{\Hand}{\mathscr{H}}
\newcommand{\hand}{h}
\newcommand{\base}{\omega}
\newcommand{\prob}{\mathrm{Pr}}

% -------------------- Document --------------------

\begin{document}

	% -------------------- Title Page --------------------

    \begin{frame}
        \titlepage
	\end{frame}
	
	% -------------------- Contents --------------------
	
	\begin{frame}
		\frametitle{目录}
		\tableofcontents[pausesections]
	\end{frame}

	% -------------------- Body --------------------
	
	\section{背景}

	\begin{frame}
		\frametitle{背景}
		麻将运动在我国有广泛的群众基础, 并且在2017年4月成为了继桥牌,
		国际象棋, 围棋, 中国象棋和国际跳棋以后第六个世界智力运动会项目.
		不过从数学角度研究麻将的论文很少,
		在arXiv上搜索`Mahjong'关键词仅有两条相关结果,
		其中一篇研究麻将牌中的组合问题, 与麻将博弈无关, 所以真正相关的文献仅有一篇.
	\end{frame}

	\section{基本定义}

	\begin{frame}
		\frametitle{局中人}
		\begin{definition}[局中人]
			局中人就是在麻将博弈中独自决策的个体, 最后获得相应结果.
		\end{definition}
		
		\begin{note}
			假设局中人做的决策与局中人追求的目的是一致的.
		\end{note}

		\begin{note}
			假设局中人的目的是最大化收益的期望值.
		\end{note}

		\begin{note}[完全理性假设]
			假设局中人有一切博弈所需的分析能力, 并且在行为上不会犯失误等任何错误.
		\end{note}

		% 我们讨论的麻将有以下几种共同规则:
		% \begin{itemize}
		% 	\item 存在和牌(胡牌)的规则, 可以荣和可以自摸.
		% 		若有局中人和牌, 则一局博弈结束.
		% 	\item 暂不考虑吃碰杠.
		% \end{itemize}
	\end{frame}

	\begin{frame}
		\frametitle{策略}
		\begin{definition}[策略]
			一个策略是一个局中人的完整的相机抉择的行动计划,
			在本博弈问题中即局中人对吃碰杠和(胡)切牌的选择.
		\end{definition}
		\begin{definition}[策略集]
			一个局中人的所有的策略的全体称为这个局中人的策略集.
		\end{definition}
	\end{frame}

	收益是策略组合的函数, 即一局博弈结束时的点数得失.

	信息是与博弈有关的消息和知识. 消息是指消息库balabala 知识是麻将规则.

	假设: 
	
	行动顺序

	\begin{frame}
		\frametitle{麻将规则的严格定义}
		\begin{definition}[麻将规则]
			麻将规则由以下元素构成
			\begin{itemize}
				\item 麻将牌全体为广义集合$\mahjong$,
					合法手牌全集为$\Hand\subset\mahjong$,
					且有$\abs{\hand}=14, \forall{\hand\in\Hand}$.
				\item 手牌点数为映射$\point: \Hand\rightarrow\mathbb{N}$.
				\item 和牌手牌(winning hands)全体$\mathscr{W}=\{\point>0\}$.
			\end{itemize}
		\end{definition}
	\end{frame}
	
	\begin{frame}
		\frametitle{其它定义}
		\begin{definition}[知识库]%TODO 消息库
			知识库$\base$记录了每个局中人在之前打出的牌以及打出它们的次序.
			知识库同时反映了其它局中人手牌的可能组合以及牌堆中牌的可能组合.
		\end{definition}
		\begin{definition}[收益]
			博弈的收益即为点数的收益, 每个局中人的目标为最大化收益的期望.
		\end{definition}
	\end{frame}

	\section{纯进攻}

	\begin{frame}
		\frametitle{纯进攻游戏}
		\begin{definition}[纯进攻游戏]
			该游戏仅有一位玩家, 该玩家手中有十四张牌$T$,
			他的知识库$\base$为牌堆中牌的所有可能组合及其概率.
			每回合该玩家打出一张牌并随机摸进一张牌. 给定限定回合数$k$,
			若该玩家在$k$回合内和牌, 则收益记为该和牌型的点数,
			若该玩家未能在$k$回合内和牌, 则收益为0.
		\end{definition}
		记知识库为$\base$时
		切出手牌$i$在$k$回合内完成点数为$p$的和牌的概率为
		$\mathrm{val}_{k}^{p}(\hand, \base, i)$,
		则纯进攻游戏优化的目标即为
		$\sum_{p}{p\cdot\mathrm{val}_{k}^{p}(\hand, \base, i)}$.
	\end{frame}

	\begin{frame}
		\frametitle{求解纯进攻游戏思路}
		定义
		\[\mathrm{val}_{k}^{p}(\hand, \base, i)=
		\max_{i\in\mathbb{Z}_{14}}{\mathrm{val}_{k}^{p}(\hand, \base, i)}.\]
		假设我们定义了摸进牌$t$时知识库的更新函数
		$\base\mapsto\mathrm{renew}_t{\base}$,
		那么对于每一个$p$, 都应有
		\[\mathrm{val}_{k}^{p}(\hand, \base, i)=
		\sum_{t}{\prob_{\base}}(t)\mathrm{val}_{k-1}^{p}(T[i/t]),
		\mathrm{renew}_{t}^{-1}{\base},\]
		其中$\prob_{\base}(t)$为摸进$t$的概率, $T[i/t]$为$T$打出$i$并摸进$T$后的手牌.
		于是我们可递归求解$\mathrm{val}_{k}^{p}(\hand, \base, i)$,
		从而解得纯进攻游戏各决策的期望收益.
	\end{frame}

\end{document}
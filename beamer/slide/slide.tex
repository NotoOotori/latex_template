% !Mode::"TeX:UTF-8"

% -------------------- Information --------------------

\newcommand{\TITLE}{幻灯片模板}
\newcommand{\AUTHOR}{Jason}
\newcommand{\SUBJECT}{幻灯片}
\newcommand{\INSTITUTE}{同济大学数学科学学院}

% -------------------- Packages --------------------

\documentclass{ctexbeamer}
\usepackage{amsmath}
\usepackage{amssymb}
\usepackage{float} % 你们这帮float给我乖乖听话 HHHHHHHHHHH.
\usepackage{graphicx}
\usepackage{hyperref}
\usepackage{mathtools} % \xleftrightarrow.
\usepackage{tikz}
\usepackage[printwatermark]{xwatermark} % Foreground Watermarks.
\usepackage[all, cmtip]{xy}

% -------------------- Settings --------------------

% Title

\title{\TITLE}
\author{\AUTHOR}
\institute{\INSTITUTE}
\titlegraphic{\includegraphics[width=0.1\paperwidth]{wallpaper.jpg}}

% Package: beamer

\transpush<2>[duration=2]
\usetheme{Berlin}

% Package: ctex

\setCJKfamilyfont{fzstk}{FZShuTi} % 方正舒体
\newcommand{\fzstk}{\CJKfamily{fzstk}}

% Package: graphicx

\graphicspath{{resources/}} %图像文件目录

% Package: hyperref

\hypersetup{
    linktoc             =   all,
    colorlinks          =   true,
    linkcolor           =   cyan,
    anchorcolor         =   black,
    citecolor           =   green,
    filecolor           =   cyan,
    menucolor           =   red,
    runcolor            =   filecolor,
    urlcolor            =   magenta,
    pdfinfo             =   {
        Title           =   {\TITLE},
        Author          =   {\AUTHOR},
        Subject         =   {\SUBJECT}},
    bookmarksnumbered   =   true,
    pdfstartview        =   Fit,
    pdfpagelayout       =   OneColumn
}

% Package: xwatermark

\newsavebox\mybox
\savebox\mybox{\tikz[color=cyan, opacity=0.2]\node{\fzstk\SUBJECT};}
\newwatermark*[
    allpages,
    angle=45,
    scale=6,
    xpos=-20,
    ypos=15
]{\usebox\mybox}

% -------------------- Document --------------------

\begin{document}

	% -------------------- Title Page --------------------

    \begin{frame}
        \titlepage
	\end{frame}
	
	% -------------------- Contents --------------------
	
	\begin{frame}
		\frametitle{目录}
		\tableofcontents[pausesections]
	\end{frame}

	% -------------------- Body --------------------
	
	\section{什么是线性代数?}
	\begin{frame}
		\frametitle{什么是线性代数?}
		走进大学数学\quad  \pause  两门基础学科:  高等数学 \: 线性代数\pause
		\begin{itemize}
			\item 高数\\ 分析学:以微积分学与无穷级数理论为基础建立的较为系统的一门学科.\pause
			\item 线代\\ 代数学:研究数、数量、关系、结构与代数方程(组)的通用解法与性质.
		\end{itemize}
	\end{frame}

	\begin{frame}
	\frametitle{什么是线性代数?}
	代数学:研究数、数量、关系、结构与代数方程(组)的通用解法与性质.\pause
	\begin{itemize}
		\item 线性: 量与量之间成比例, 成直线的关系. \pause
		\item 研究线性关系的意义:简单直观, 计算方便, 复杂的实际问题的简化.\pause
		\item 线性代数: 代数学的基础, 以向量, 向量空间(线性空间), 线性变换与有限维线性方程组为主要研究对象. 
	\end{itemize}
	\end{frame}

	\section{为什么要学线性代数?}
	\begin{frame}
	\frametitle{为什么要学线性代数?}
		线性代数具有理论性: 元素的表达空间化, 思维上升维度, 从初等的符号代数中脱离出来, 使得运算变得立体. 掌握线、面等高维中的运算方法, 培养严密的空间思维、逻辑思维, 理论性极强.
	\end{frame}

	\begin{frame}
	\frametitle{为什么要学线性代数?}
		线性代数具有广泛性:
		许多同学因为不知晓自己学习线性代数的理由而失去动力. 事实上线性代数的使用无处不在.\pause
		\begin{itemize}
			\item 小学: 开始接触解方程(组), 小学奥数中著名的鸡兔同笼问题;\pause
			\item 中学: 计算几何体面积、体积;\pause 物理中的矢量方程式;\pause 化学的氧化还原反应方程式配平;\pause 生物中的生态系统中的能量传递; \pause
			\item 大学:  今后课程中的抽象代数、泛函分析、数值分析等课程均以线代为基础;\pause 高效算法的产出;\pause 强大的逻辑思维能力\pause 
			\item 未来: 电气工程 复杂的电路网络;\pause 机械工程 大型动力系统的计算;\pause 土木工程 桥梁建筑应力分析;\pause 经济金融 多边经济均衡问题等等等等\pause 
		\end{itemize}
	\end{frame}
	
	\begin{frame}
	\frametitle{为什么要学线性代数?}
	线性代数还具有实用性:\\\pause
	现实的需求推动着理论的发展, 面临越来越复杂的数值计算问题, 由线性代数推导出的理论工具越来越完备(高斯消元, 克拉默法则, 矩阵分解) \\\pause
	另外, 算法的高效性则更是决定着计算机的计算水平(正是因为线性代数, 现代计算机才能算那么多) \\\pause
	当然, 线性代数作为大学的一门基础必修课, 它直接关系着同学们的期末成绩, 长远来看, 对线性代数的掌握程度也影响着今后各个学科的学习情况
	\end{frame}
	
	\section{怎么理解线性代数?}
	\begin{frame}
		\frametitle{怎么理解线性代数?}
		我们来大致厘清一下线性代数究竟学些什么. 这一切都要从多元线性方程组谈起:\pause
		\begin{equation*}
			\begin{aligned}
				a_{11}x_1+a_{12}x_2+\cdots+a_{1n}x_n&=b_1,\\
				a_{21}x_1+a_{22}x_2+\cdots+a_{2n}x_n&=b_2,\\
				\cdots&\cdots\\
				a_{n1}x_1+a_{n2}x_2+\cdots+a_{nn}x_n&=b_n.
			\end{aligned}
		\end{equation*}
	\end{frame}
	
	\begin{frame}
		其实我们应最先引入矩阵的概念. 运用矩阵与向量的概念, 线性方程组可以表示为:\pause
		$$
			\begin{pmatrix}
				a_{11} & a_{12} & \cdots & a_{1n}\\
				a_{21} & a_{22} & \cdots & a_{2n}\\
				\vdots & \vdots & \ddots & \vdots\\
				a_{n1} & a_{n2} & \cdots & a_{nn}
			\end{pmatrix}\begin{pmatrix}
			x_1\\x_2\\\vdots\\x_n
			\end{pmatrix}=\begin{pmatrix}
			b_1\\b_2\\\cdots\\b_n
			\end{pmatrix}$$
			\pause
			$$\Leftrightarrow x_1\begin{pmatrix}
			a_{11}\\a_{21}\\\vdots\\a_{n1}
			\end{pmatrix}+x_2\begin{pmatrix}
			a_{12}\\a_{22}\\\vdots\\a_{n2}
			\end{pmatrix}+\cdots+x_n\begin{pmatrix}
			a_{1n}\\a_{2n}\\\vdots\\a_{nn}
			\end{pmatrix}=\begin{pmatrix}
			b_1\\b_2\\\cdots\\b_n
			\end{pmatrix}
			\pause
		$$$$\Leftrightarrow x_1\alpha_1+x_2\alpha_2+\cdots+x_n\alpha_n=\beta(\makebox{向量方程})$$
	\end{frame}
	
	\begin{frame}
		进一步可以将上述简记为$$Ax=b,$$其中$A$为$n\times n$矩阵, $x,b$为$n$维向量. 
		由此, 线性代数的所有基础内容都将由此展开:  \pause
		\begin{itemize}
			\item 若$A,b$已知, 如何求解$x$?\pause\\$\left\{\begin{array}{ll}
				\makebox{解是不是唯一?}\Rightarrow\makebox{Chapter 1}\\
				\makebox{如果解唯一 怎么求解?}\Rightarrow\makebox{Chapter 2}\\
				\makebox{如果解不唯一 怎么求解?}\Rightarrow\makebox{Chapter 3}
			\end{array}\right.$\pause
			\item 研究$A$列向量组之间的关系与向量方程$\Rightarrow$Chapter 4
			\item 研究$A$本身的性质以及矩阵间的特殊关系$\Rightarrow$Chapter 3, 5\pause
			\item 将$A$视作作用在向量上一种变换(或研究向量$x,b$之间的变换关系)$\Rightarrow$Chapter 6
		\end{itemize}
	\end{frame}

\begin{frame}
\frametitle{参考文献}
\begin{itemize}
\item 同济大学数学系. 工程数学线性代数(第六版)[M]. 北京: 高等教育出版社, 2014.
\item 线性代数的本质. \url{https://www.bilibili.com/} \\ 搜索: 3BLUE1BROWN(三蓝一土)
\end{itemize}
\end{frame}

\begin{frame}
\begin{center}
\Large Thanks for your attention!
\end{center}
\begin{flushright}
---Made by Hu Yukuan\\
---Revised by Zhang Yuanhang
\end{flushright}
\end{frame}
\end{document}
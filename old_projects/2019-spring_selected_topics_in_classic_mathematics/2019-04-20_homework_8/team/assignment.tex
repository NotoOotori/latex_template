% !Mode::"TeX:UTF-8"

% -------------------- Information --------------------

\newcommand{\TITLE}{小组作业8}
\newcommand{\AUTHOR}{Jason}
\newcommand{\SUBJECT}{经典数学专题选讲}
\newcommand{\KEYWORDS}{}

% -------------------- Packages --------------------

\documentclass[a4paper, 12pt]{ctexart}
\usepackage{amsmath}
\usepackage{amssymb}
% \usepackage{amsthm} % 定理格式 由ntheorem代替.
\usepackage{authblk} % 作者 (见校赛论文).
\usepackage{array}
\usepackage{bigfoot} % to allow verbatim in footnote.
\usepackage{bm} % \bm for bold symbols.
\usepackage{boldline} % 长表格表格线加粗.
\usepackage{caption} % 题注.
\usepackage{commath} % abs, norm
\usepackage{enumerate}
% \usepackage{enumitem} 用enumerate包代替.
\usepackage{fancyhdr} % 脚注.
\usepackage{filecontents}
\usepackage{flafter} % 不让float出现在定义之前的地方.
\usepackage{float} % 你们这帮float给我乖乖听话 HHHHHHHHHHH.
\usepackage[T1]{fontenc} % Bera Mono Font
\usepackage{fontspec} % 字体.
\usepackage{graphicx}
\usepackage{hyperref}
\usepackage{lastpage}
\usepackage{letltxmacro} % \let
\usepackage{lipsum}
\usepackage{listings} % 排版程序语言.
\usepackage{longtable} % 长表格.
\usepackage{makecell} % 表格线加粗 \Xhline{1.2pt}.
\usepackage{mathtools} % \xleftrightarrow.
\usepackage{mathrsfs} % \mathscr
\usepackage{multirow} % 合并单元格.
\usepackage[square, numbers, sort&compress]{natbib} % 引用.
\usepackage[thmmarks, amsmath, thref]{ntheorem} % 定理格式.
\usepackage[section]{placeins} % 使图像不会显示在别的部分 若过于严格则换成[below].
\usepackage{stackrel} % 上下写 见校赛论文.
\usepackage{subcaption} % subcaption and subfigure
% \usepackage{SUBSubsubsection}
\usepackage{titlesec} % Section标题格式.
\usepackage{varioref} % For Cross References.
\usepackage[dvipsnames]{xcolor} % 颜色声明.
\usepackage{xfrac} %\sfrac{}{}
\usepackage[all, cmtip]{xy} % Commutive diagram.

% Require `ntheorem'

\usepackage[mathlines, edtable]{lineno} % Line numbers.
    %\begin{edtable}{tabular}[<args>] <entries> \end{edtable}

% Require `xcolor'

\usepackage[numbered, framed]{matlab-prettifier}
\usepackage{pgfplots}
\usepackage{tikz}

% Incompatible with `matlab-prettifier'

\usepackage[printwatermark]{xwatermark} % Foreground Watermarks.

% -------------------- Settings --------------------

% Title

\title{\TITLE}
\author{\AUTHOR}
\date{\today}

% Package: caption

\captionsetup{
    margin    =   6pt,
    font      =   small,
    labelfont =   bf
}

% Package: ctex

\setCJKfamilyfont{fzstk}{FZShuTi} % 方正舒体
\newcommand{\fzstk}{\CJKfamily{fzstk}}

% Package: fancyhdr

\setlength{\headheight}{15pt}
\lhead{Copyright \copyright\ \AUTHOR}
\rhead{Page \thepage\ of \pageref{LastPage}}

% Package: graphicx

\graphicspath{{resources/}} % 图像文件目录

% Package: hyperref

\hypersetup{
    linktoc             =   all,
    colorlinks          =   true,
    linkcolor           =   cyan,
    anchorcolor         =   black,
    citecolor           =   green,
    filecolor           =   cyan,
    menucolor           =   red,
    runcolor            =   filecolor,
    urlcolor            =   magenta,
	pdftitle           	=   {\TITLE},
	pdfauthor          	=   {\AUTHOR},
	pdfsubject         	=   {\SUBJECT},
	pdfcreator			=	{Visual Studio Code},
	pdfproducer			=	{XeLaTeX with documentclass ctexart},
	pdfkeywords        	=   {\KEYWORDS},
    bookmarksnumbered   =   true,
    pdfstartview        =   FitH,
    pdfpagelayout       =   OneColumn
}

% Package: lineno

\renewcommand{\linenumberfont}{\normalfont\scriptsize\sffamily}

\let\oldlstinputlisting\lstinputlisting
\renewcommand{\lstinputlisting}[2][\empty]{
    \par\nolinenumbers\oldlstinputlisting[#1]{#2}\linenumbers\par
}

\let\oldlstlisting\lstlisting
\let\oldendlstlisting\endlstlisting
\renewenvironment{lstlisting}
    {\par\nolinenumbers\oldlstlisting}
    {\oldendlstlisting\endnolinenumbers\par}

\let\oldtable\table
\let\oldendtable\endtable
\renewenvironment{table}
    {\par\nolinenumbers\oldtable}
    {\oldendtable\endnolinenumbers\par}

% Package: listings

%% Title

\renewcommand\lstlistingname{代码}
\renewcommand\lstlistlistingname{代码}

%% Lstinline with color box

\LetLtxMacro{\oldlstinline}{\lstinline}
\renewcommand{\lstinline}[2][]{\colorbox{lightgray}{\oldlstinline[#1]{#2}}}
\newcommand{\matlabinline}[1]{
    \lstinline[style=MATLAB-editor, basicstyle=\mlttfamily]{#1}}

\lstset{
    breaklines=true,
    backgroundcolor=\color{lightgray},
    basicstyle=\scriptsize,
    inputpath=resources/,
    numbers=left,
    numberstyle={\color{black!33}\scriptsize\sffamily},
    xleftmargin=2em,
    xrightmargin=2em
}

% Package: ntheorem

%% Theorem
\newtheorem{theorem}{Theorem}[section]
\newtheorem{lemma}[theorem]{Lemma}
\newtheorem{corollary}[theorem]{Corollary}
%% Problem
\theoremstyle{plain}
\newtheorem{problem}{Problem}[section]
%% Definition
\theoremstyle{plain}
\theoremheaderfont{\bfseries}
\theorembodyfont{\rmfamily}
\newtheorem{definition}{Definition}[section]
%% Note
\theoremstyle{plain}
\theoremheaderfont{\itshape}
\theorembodyfont{\itshape}
\newtheorem{note}{Note}[section]
%% Proof
\theoremstyle{nonumberplain}
\theoremheaderfont{\itshape}
\theorembodyfont{\upshape}
\theoremseparator{.}
\theoremsymbol{\ensuremath{\square}}
\newtheorem{proof}{Proof}
%% Solution
\theoremsymbol{\ensuremath{\blacksquare}}
\newtheorem{solution}{Solution}

% Package: pgfplot

\pgfplotsset{width=7cm, compat=1.16}

% Package: varioref

\renewcommand{\reftextbefore}
    {on the \reftextvario{preceding page}{page before}}
\renewcommand{\reftextafter}
    {on the \reftextvario{following}{next} page}
\renewcommand{\reftextfacebefore}
    {on the \reftextvario{facing}{preceding} page}
\renewcommand{\reftextfaceafter}
    {on the \reftextvario{facing}{next} page}
\renewcommand{\reftextfaraway}[1]
    {on page \pageref{#1}}

%% Label formats

\labelformat{lstlisting}{代码#1}
\labelformat{equation}{式(#1)}
\labelformat{figure}{图#1}
\labelformat{table}{表#1}

% Package: xwatermark

\newsavebox\mybox
\savebox\mybox{\tikz[color=cyan, opacity=0.2]\node{\fzstk\SUBJECT};}
\newwatermark*[
    allpages,
    angle=45,
    scale=6,
    xpos=-20,
    ypos=15
]{\usebox\mybox}

% -------------------- General new commands --------------------

\DeclareMathAlphabet{\mathsfsl}{OT1}{cmss}{m}{sl}

\DeclareMathOperator{\arcosh}{arcosh}
\DeclareMathOperator{\Arcosh}{Arcosh}
\DeclareMathOperator*{\Beta}{B}
\DeclareMathOperator*{\diff}{d}
\DeclareMathOperator{\Log}{Log}

% Expectation

\newcommand{\expect}{\operatorname{E}\expectarg}
\DeclarePairedDelimiterX{\expectarg}[1]{(}{)}{
    \ifnum\currentgrouptype=16 \else\begingroup\fi
    \activatebar#1
    \ifnum\currentgrouptype=16 \else\endgroup\fi
}

\newcommand{\innermid}{\nonscript\;\delimsize\vert\nonscript\;}
\newcommand{\activatebar}{
    \begingroup\lccode`\~=`\|
    \lowercase{\endgroup\let~}\innermid
    \mathcode`|=\string"8000
}

\newcommand{\BR}{\mathbb{R}}
\newcommand{\matr}[1]{\ensuremath{\mathsfsl{#1}}} % italic sans serif
\newcommand{\me}{\mathrm{e}}
\newcommand{\mi}{\mathrm{i}}
\newcommand{\restrict}[1]{\raisebox{-.5ex}{$\vert$}_{#1}}
\newcommand{\vect}[1]{\bm{#1}}

% -------------------- Specific new commands --------------------



% -------------------- Document --------------------

\begin{document}

    % -------------------- Title Page --------------------

    \maketitle
    \thispagestyle{empty}
    \pagenumbering{roman}

    % -------------------- Abstract Page --------------------

    % -------------------- Contents --------------------

    % \newpage
    % \tableofcontents

    % -------------------- Body --------------------

    \newpage
    \pagestyle{fancy}
    \pagenumbering{arabic}
    \linenumbers

    \begin{problem}
        将一个级数$\sum{b_{n}}$和一个人对应起来. 称一个平方收敛的级数
        $(\sum{b_{n}^{2}}<\infty)$为\textbf{正常人}, 一个平方发散的级数
        为\textbf{不正常人}, 一个只有有限项非零的级数为\textbf{凡人}.
        路人甲$\sum{a_{n}}$对路人乙的感觉通过$\sum{a_{n}b_{n}}$体现,
        若$\sum{a_{n}b_{n}}$收敛则\textbf{甲觉得乙正常}, 反之为不正常.
        证明:
        \begin{enumerate}
            \item 如果每个正常人都觉得某人正常, 则某人是正常人;
            \item 如果所有人都觉得某人正常, 则某人是凡人.
        \end{enumerate}
    \end{problem}

    \section{第一小问}

    \begin{problem}
        若$\sum{b_{n}}$不平方收敛, 则存在平方收敛$\sum{a_{n}}$,
        使得$\sum{a_{n}b_{n}}$发散.
    \end{problem}

    事实上我们可以假设$a_{n}$与$b_{n}$均恒为正.
    \begin{itemize}
        \item

        可以假设它们不为零是因为$\sum{b_{n}}$有无限项不为零,
        所以取出所有非零项能构成子列, 并且子列求和的敛散性与原级数相同.

        \item

        可以假设它们恒正是因为我们可以选取$a_{n}$使得$a_{n}b_{n}>0$,
        这样问题中所讨论的三个级数
        ($\sum{a_{n}^{2}}$, $\sum{b_{n}^{2}}$和$\sum{a_{n}b_{n}}$)
        对于$a_{n}$, $b_{n}$是否同时取绝对值无关.
    \end{itemize}

    \begin{lemma}
        设$\sum{p_{n}}$, $\sum{q_{n}}$为正项级数, 其中$\sum{p_{n}}$收敛,
        且有
        \begin{equation}
            \label{equation_limsup}
            \limsup_{n}{\frac{q_{n}}{p_{n}}} = C \in \BR,
        \end{equation}
        那么$\sum{q_{n}}$收敛.
    \end{lemma}

    \begin{proof}
        由推广Stolz-Ces\`aro定理可知
        \begin{equation}
            \limsup_{n}{\frac{\sum_{1}^{n}{q_{k}}}{\sum_{1}^{n}{p_{k}}}}
            \leq
            \limsup_{n}{\frac{{q_{n}}}{{p_{n}}}}
            =
            C,
        \end{equation}
        又$\sum{p_{n}}$收敛, 所以$\limsup_{n}{\sum_{1}^{n}{q_{k}}}$存在.
        由于$\sum{q_{n}}$为正项级数, 所以$\sum{q_{n}}$存在, 即级数收敛.
    \end{proof}

    \begin{lemma}
        设$\sum{q_{n}}$为正项级数, 若存在正实数$m$使得$\sum{q_{n}^{m}}$收敛,
        那么对于一切的实数$r>m$都有$\sum{q_{n}^{r}}$收敛.
    \end{lemma}

    \begin{proof}
        只须注意到
        \begin{equation}
            \limsup_{n}{\frac{q_{k}^{r}}{q_{k}^{m}}}
            =
            \limsup_{n}{q_{n}^{r-m}}
            =
            0,
        \end{equation}
        再利用前一个引理即可得证.
    \end{proof}

    根据以上引理及$\sum{b_{n}}$不平方收敛的条件,
    我们可以将$\sum{b_{n}}$分为两种情况分别讨论:
    \begin{itemize}
        \item
    
        集合$\{r\in \BR, r > 2: \sum{b_{n}^{r}}<+\infty\}$非空.

        \item

        对于任意的正实数$m$都有$\sum{b_{n}^{m}}$发散.
    \end{itemize}

    \subsection{第一种情况}

    事实上我们暂时不能完全证明第一种情况下$a_{n}$的存在性,
    所以给出一个加强条件后的证明.

    \begin{theorem}
        设$\sum{b_{n}}$不平方收敛,
        并设集合$\{r\in \BR, r > 2: \sum{b_{n}^{r}}<+\infty\}$非空,
        且下确界为$m>2$,
        则存在平方收敛级数$\sum{a_{n}}$使得
        $\sum{a_{n}b_{n}}$发散.
    \end{theorem}

    \begin{proof}
        令$m=\inf\{r\in \BR, r > 2: \sum{b_{n}^{r}}<+\infty\}>2$,
        取$\varepsilon = \sfrac{(m-2)}{2}$,
        并令$a_{n}=b_{n}^{\sfrac{(m+\varepsilon)}{2}}$,
        则由题意有$\sum{a_{n}}$平方收敛.
        但是
        \begin{equation}
            a_{n}b_{n} = b_{n}^{\sfrac{(m+\varepsilon+2)}{2}},
        \end{equation}
        而
        $\sfrac{(m+\varepsilon+2)}{2}
        =
        \sfrac{(m+\varepsilon+m-2\varepsilon)}{2}
        < m$.
        由$m$的定义知$\sum{a_{n}b_{n}}$发散.
    \end{proof}

    \subsection{第二种情况}

    \begin{lemma}
        设$\sum{p_{n}}$, $\sum{q_{n}}$为正项级数, 其中$\sum{p_{n}}$发散,
        且有
        \begin{equation}
            \label{equation_liminf}
            \liminf_{n}{\frac{q_{n}}{p_{n}}} = +\infty,
        \end{equation}
        那么$\sum{q_{n}}$发散.
    \end{lemma}

    \clearpage

    \begin{proof}
        由推广Stolz-Ces\`aro定理可知
        \begin{equation}
            \liminf_{n}{\frac{\sum_{1}^{n}{q_{k}}}{\sum_{1}^{n}{p_{k}}}}
            \geq
            \liminf_{n}{\frac{{q_{n}}}{{p_{n}}}}
            =
            +\infty,
        \end{equation}
        又$\sum{p_{n}}$发散, 所以$\liminf_{n}{\sum_{1}^{n}{q_{k}}}=+\infty$.
        由于$\sum{q_{n}}$为正项级数, 所以$\sum{q_{n}}=+\infty$, 即级数发散.
    \end{proof}

    \begin{theorem}
        设对于任意的正实数$m$都有$\sum{b_{n}^{m}}$发散,
        那么存在平方收敛级数$\sum{a_{n}}$使得
        $\sum{a_{n}b_{n}}$发散.
    \end{theorem}

    \begin{proof}
        取$a_{n}=n^{-0.6}$, $c_{n}=n^{-1}$, 则有$\sum{c_{n}}$发散, 且
        \begin{equation}
            \liminf_{n}{\frac{a_{n}b_{n}}{c_{n}}} =
            \liminf_{n}{\frac{b_{n}}{n^{-0.4}}} =
            \liminf_{n}{\sqrt[5]{\frac{b_{n}^{5}}{n^{-2}}}} = +\infty,
        \end{equation}
        否则$\sum{b_{n}^{5}}$收敛, 与假设矛盾.
        由引理得$\sum{a_{n}b_{n}}$, 又显然有$\sum{a_{n}}$平方收敛, 即得证.
    \end{proof}

    \section{第二小问}

    \begin{theorem}
        设级数$\sum{b_{n}}$满足对于任意的级数$\sum{a_{n}}$都有
        $\sum{a_{n}b_{n}}$收敛, 那么$\sum{b_{n}}$只有有限项非零.
    \end{theorem}

    \begin{proof}
        按如下方法构造$a_{n}$.
        \begin{equation}
            a_{n} =
            \begin{cases}
                {\displaystyle\frac{1}{b_{n}}}&, b_{n}\neq 0,\\
                114514810893 &, b_{n} = 0.
            \end{cases}
        \end{equation}
        故
        \begin{equation}
            a_{n}b_{n} =
            \begin{cases}
                1&, b_{n}\neq 0,\\
                0&, b_{n} = 0.
            \end{cases}
        \end{equation}
        又因为$\sum{a_{n}b_{n}}$收敛, 即为有限值, 所以$\sum{b_{n}}$只有有限项非零.
    \end{proof}

    % -------------------- Bibliography --------------------

    % \newpage
    % \bibliography{Principles_of_Mathematical_Analysis}
    % \bibliographystyle{plain}

\end{document}

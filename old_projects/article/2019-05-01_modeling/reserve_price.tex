\subsection{模型假设}

出于简化模型的考虑, 本文认为所有的广告视频长度是固定的,
这样就可以把广告总时长的限制转化为广告数量的限制,
可以将拍卖规则设计为前$K$高价者赢得广告播放时段,
其中$K$为该时段能承载的最大广告数量.

假设卖方对广告播放时段的估价为$V_{0}$,
这对买方们来说是公开信息.
记$V_{j}$表示第$j$位买方对广告播放时段的估价.
本文假定本拍卖环境有以下性质:
\begin{enumerate}[({A}1)]
    \item (私有价值) $V_{j}$为各个买方的私人信息,
        对于其余买方而言$V_{j}$为随机变量,
        他们能且仅能知道$V_{j}$的分布, 记累计分布函数为$F_{j}(v_{j})$.
    \item (风险中性) 每个买方的目标是最大化他的期望收益.
    \item (非合作行为) 所有买方之间不存在任何具有约束力的合作性协议.
    \item 以上均为各个买方的公共知识.
\end{enumerate}

\subsection{模型建立}

\subsubsection{估计卖方估价$V_{0}$}

假设广告播放时段的市场价$\tilde{V}_{0}$
与且仅与频道收视率$a_{0}$和频道特征$u_{0}$有关,
那么通过统计回归就可以得出市场价与频道收视率和频道特征的关系,
从而估计某个频道的广告播放时段的市场价.

市场价参数不仅体现了频道收视情况, 也体现了该频道对于广告总体的匹配程度.
比如说少儿频道中受众大多为少年儿童, 与大部分广告内容匹配度低,
因此在收视率相近的情况下,
少儿频道的广告播放时段的市场价会低于卫视广告播放时段的市场价.

因此本文将以市场价$\tilde{V}_{0}$来估计卖方估价$V_{0}$.

\subsubsection{估计买方估价$V_{j}$}

买方估价$V_{j}$取决于他对于播放广告后销售量增加的预期,
而销售量增加的预期取决于收视率以及匹配度. 当卖方固定的时候收视率也固定了,
因此可认为$V_{j}$独立服从于分布$F_{m_{j}}$, 其中参数$m_{j}$为匹配度.

\subsubsection{确定合理保留价}

根据拍卖理论, 如果拍卖机制为第二价格拍卖, 那么每个买方的最优策略即为报出
自己真实愿意付出的价钱$V_{j}$. 赢得广告播放时段之后,
他只需付出报价中低于他报价的最高价即可.

在设计保留价的时候, 卖家需要考虑到保留价过低而造成的广告播放时段以过低价格卖出的风险,
也要考虑到保留价过高而造成的一个广告播放时段中的广告位未能全部卖出的风险.

先考虑一个买方的情况, 简记其私人价值为$V$, 累计分布函数为$F(v)$,
概率密度函数为$f(v)$.
如果卖方定一个保留价$r$, 那么只有在买方的私人价值$V\geq r$
时他才购买其物品, 这一事件发生的概率为$1-F(r)$. 因此, 卖方的期望利润为
\begin{equation}
    \pi(r) = (r-v_{0})(1-F(r)),
\end{equation}
卖方选择最大化$\pi(r)$, 因此最优保留价$r^{*}$满足
\begin{equation}
    r^{*}-\frac{1-F(r^{*})}{f(r^{*})} = v_{0}.
\end{equation}

可以证明如果假设各个买方的私人价值独立同分布,
那么最优保留价仍满足上方程, 并且最优保留价$r^{*}$高于卖方的真实价值$V_{0}$.
本题中各个买方的私人价值不尽相同, 所以本文认为卖方的真实价值$V_{0}$是合理的保留价.

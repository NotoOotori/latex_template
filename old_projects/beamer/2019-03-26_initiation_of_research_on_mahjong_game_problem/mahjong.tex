% !Mode::"TeX:UTF-8"

% -------------------- Information --------------------

\newcommand{\TITLE}{麻将博弈问题的研究}
\newcommand{\AUTHOR}{郑博文, 张赫, 陈旭阳}
\newcommand{\SUBJECT}{大学生创新创业训练计划项目立项答辩}
\newcommand{\KEYWORDS}{}
\newcommand{\INSTITUTE}{同济大学数学科学学院}

% -------------------- Packages --------------------

\documentclass[xcolor=dvipsnames]{ctexbeamer}
\usepackage{amsmath}
% \usepackage{amsthm} % loaded by beamer.
\usepackage{amssymb}
\usepackage{commath} % abs, norm
\usepackage[mathscr]{euscript}
\usepackage{float} % 你们这帮float给我乖乖听话 HHHHHHHHHHH.
% \usepackage{fontspec}
\usepackage{graphicx}
% \usepackage{hyperref} % loaded by beamer.
\usepackage{mathtools} % \xleftrightarrow.
\usepackage[timeinterval=1, font=Times, resetatpages=1]{tdclock}
\usepackage[absolute, overlay]{textpos}
\usepackage{tikz}
% \usepackage{xcolor} % loaded by beamer.
% \usepackage[printwatermark]{xwatermark} % Foreground Watermarks.
\usepackage[all, cmtip]{xy}

% -------------------- Settings --------------------

% Package: beamer

\transpush<2>[duration=2]
\usetheme{Berlin}

% Package: ctex

\setCJKfamilyfont{fzstk}{FZShuTi} % 方正舒体
\newcommand{\fzstk}{\CJKfamily{fzstk}}

% Package: graphicx

\graphicspath{{resources/}} % 图像文件目录

% Package: hyperref

\hypersetup{
    linktoc             =   all,
    colorlinks          =   true,
    linkcolor           =   cyan,
    anchorcolor         =   black,
    citecolor           =   green,
    filecolor           =   cyan,
    menucolor           =   red,
    runcolor            =   filecolor,
    urlcolor            =   magenta,
    pdftitle           	=   {\TITLE},
    pdfauthor          	=   {\AUTHOR},
    pdfsubject         	=   {\SUBJECT},
    pdfcreator			=	{Visual Studio Code},
    pdfproducer			=	{XeLaTeX with documentclass ctexbeamer},
    pdfkeywords        	=   {\KEYWORDS},
    bookmarksnumbered   =   true,
    pdfstartview        =   Fit,
    pdfpagelayout       =   OneColumn,
    pdfpagemode			=   FullScreen
}

% Package: tdclock

\newcommand{\FrameTextCrono}[1]{
    \begin{textblock*}{\paperwidth}(\textwidth + 1em, \textheight + 1em)
        #1
    \end{textblock*}
}

\newcommand{\FrameTextResetCrono}[1]{
    \begin{textblock*}{\paperwidth}(\textwidth + 1.5em, \textheight - 0.5em)
        #1
    \end{textblock*}
}

\newcommand{\ResetCronoBox}{\resetcrono{\fbox{reset}}}

\let\oldframe\frame
\let\oldendframe\endframe
\renewenvironment{frame}
    {\oldframe\FrameTextCrono{\small\color{blue}{\crono}}}
    {\oldendframe}

\let\oldtitlepage\titlepage
\renewcommand{\titlepage}{\oldtitlepage\FrameTextResetCrono{\ResetCronoBox}}

% Package: xwatermark

% \newsavebox\mybox
% \savebox\mybox{\tikz[color=cyan, opacity=0.2]\node{\fzstk\SUBJECT};}
% \newwatermark*[
%     allpages,
%     angle=45,
%     scale=6,
%     xpos=-20,
%     ypos=15
% ]{\usebox\mybox}

% Title

\title{\TITLE}
\author{\AUTHOR}
\date{\tdyear\dateseparator\tdmonth\dateseparator\tdday\hspace{1em}\tdtime}
\institute{\INSTITUTE}
\titlegraphic{\includegraphics[width=0.1\paperwidth]{wallpaper.jpg}}

% Theorem Environments

\let\note\undefined
\newtheorem{note}{注}[section]

% -------------------- General new commands --------------------

\DeclareMathAlphabet{\mathsfsl}{OT1}{cmss}{m}{sl}

\newcommand{\diff}{\mathop{}\!\mathrm{d}}
\newcommand{\matr}[1]{\ensuremath{\mathsfsl{#1}}} % italic sans serif
\newcommand{\me}{\mathrm{e}}
\newcommand{\mi}{\mathrm{i}}
\newcommand{\restrict[1]}{\raisebox{-.5ex}{$|$}_{#1}}

% -------------------- Specific new commands --------------------

\newcommand{\mahjong}{\mathscr{M}}
\newcommand{\point}{\mathrm{point}}
\newcommand{\Hand}{\mathscr{H}}
\newcommand{\hand}{h}
\newcommand{\base}{\omega}

% -------------------- Document --------------------

\begin{document}

    % -------------------- Title Page --------------------

    \begin{frame}
        \initclock
        \titlepage
    \end{frame}
    
    % -------------------- Contents --------------------
    
    \begin{frame}
        \frametitle{目录}
        \tableofcontents
    \end{frame}

    % -------------------- Body --------------------
    
    \section{背景}

    \begin{frame}
        \frametitle{背景}
        \begin{itemize}
            \item 麻将运动在我国有广泛的群众基础, 并且在2017年4月成为了继桥牌,
                国际象棋, 围棋, 中国象棋和国际跳棋以后第六个世界智力运动会项目.
            \item 不过从数学角度研究麻将的论文很少,
                在arXiv上搜索`Mahjong'关键词仅有两条相关结果.
            \item 其中一篇
                "Mathematical aspects of the combinatorial game Mahjong"
                [arXiv:1707.07345v2]研究麻将牌中的组合问题,
                与麻将博弈无关.
            \item 另一篇"Let's Play Mahjong!"[arXiv:1903.03294v1]
                花了主要篇幅定义并计算了牌型的好坏程度,
                而只在一章里面提到了一点点具体决策的方法.
        \end{itemize}
    \end{frame}

    \section{基本定义}

    \begin{frame}
        \frametitle{博弈的基本要素}
        博弈的五要素分别为局中人, 策略集, 收益, 信息和行动顺序.
        我们将从这五个要素来介绍本麻将博弈的基本要素.
    \end{frame}

    \begin{frame}
        \frametitle{局中人}
        \begin{definition}[局中人]
            局中人就是在麻将博弈中独自决策的个体, 最后获得相应结果.
        \end{definition}
        
        \begin{block}{假设}
            \begin{itemize}
                \item 假设局中人做的决策与局中人追求的目的是一致的.
                \item 假设局中人的目的是最大化收益的期望值.
                \item {[\alert{完全理性假设}]}假设局中人有一切博弈所需的分析能力,
                    并且在行为上不会犯失误等任何错误.
            \end{itemize}
        \end{block}
    \end{frame}

    \begin{frame}
        \frametitle{策略}
        \begin{definition}[策略]
            一个策略是一个局中人的完整的相机抉择的行动计划,
            在本博弈问题中即局中人对鸣牌(吃碰杠)和(胡)切牌的选择.
        \end{definition}
        \begin{definition}[策略集]
            一个局中人的所有的策略的全体称为这个局中人的策略集.
        \end{definition}
    \end{frame}

    \begin{frame}
        \frametitle{收益}
        \begin{definition}[收益]
            收益是策略组合的函数, 在本博弈中即为一局博弈结束时的点数得失.
        \end{definition}
    \end{frame}

    \begin{frame}
        \frametitle{信息}
        \begin{definition}[信息]
            信息是与博弈有关的消息和知识.
        \end{definition}
        \begin{definition}[消息]
            消息, 之后我们称为消息库$\base$, 是局中人之间的共同知识.
            在本博弈中为每个局中人切出的牌和切牌顺序, 以及鸣(吃碰杠)的牌.
        \end{definition}
        \begin{note}[消息库]
            消息库反映了其它局中人手牌的可能组合以及牌堆中牌的可能组合.
        \end{note}
    \end{frame}

    \begin{frame}
        \frametitle{信息(续)}
        \begin{definition}[知识]
            知识在本博弈中指的是博弈的相关定义以及麻将的规则等.
        \end{definition}
        \begin{note}[私人信息]
            所有局中人的私人信息有且仅有自己当前手牌.
        \end{note}
    \end{frame}
    
    \begin{frame}
        \frametitle{行动顺序}
        \begin{definition}[行动顺序]
            \begin{itemize}
                \item 设$X$为有$n$种取值的随机变量, 记为
                    $x_{1}, x_{2}, \dotsc, x_{n}$,
                    且满足$0<\Pr(X)=x_{i}<1$, 以及
                    \begin{equation}
                        \sum_{i}{x_{i}} = 1.
                    \end{equation}
                \item 定义随机变量序列$\{X_{i}\}_{1}^{N}$, 其中$N$为正整数,
                    且$X_{i}$之间独立并且与$X$同分布.
                \item 我们称$\{X_{i}\}$的一个观测值为一个行动顺序.
            \end{itemize}
        \end{definition}
    \end{frame}

    \begin{frame}
        \frametitle{行动顺序流程图}
        \begin{figure}[H]
            \centering
            \includegraphics[scale=0.18]{flowchart_order.png}
            \caption{行动顺序流程图}
        \end{figure}
    \end{frame}

    \begin{frame}
        \frametitle{研究的问题}
        \begin{itemize}
            \item 在麻将博弈是公平博弈的假设下, 给出麻将博弈的概率测度,
                即给出在任意行动顺序下轮到行为的局中人的各种策略的期望收益.
            \item {[\alert{公平博弈假设}]}局中人在初始时刻具有点数$P_{0}$为常随机变量,
                经过$k$回合后点数变为$P_{k}$, 且满足$P_{0} = E(P_{k})$,
                i.e.
                \begin{equation}
                    E(P_{k}|\sigma(P_{0}, P_{1}, \dotsc, P_{k-1})) = P_{k-1},
                \end{equation}
                这里$\sigma(P_{0}, P_{1}, \dotsc, P_{k-1})$
                是随机理论中的$\sigma$代数,
                实际意义为局中人直到第$k-1$次行为点数
                $P_{0}, P_{1}, \dotsc, P_{k-1}$所有可能发生的信息.
                换句话说, 点数$\{P_{j}: 0\leq j\leq N\}$是鞅(martingale).
        \end{itemize}
    \end{frame}

    \begin{frame}
        \frametitle{麻将规则的定义}
        \begin{definition}[麻将规则]
            麻将规则由麻将牌全体和每种14张手牌的点数构成, 我们定义了如下记号:
            \begin{itemize}
                \item 麻将牌全体为广义集合$\mahjong$,
                    合法手牌全集为$\Hand\subset\mahjong$,
                    且有$\abs{\hand}=14, \forall{\hand\in\Hand}$.
                \item 手牌点数为映射$\point: \Hand\rightarrow\mathbb{N}$.
                \item 和牌手牌(winning hands)全体$\mathscr{W}=\{\point>0\}$.
            \end{itemize}
        \end{definition}
    \end{frame}

    \section{求解思路}

    \subsection{纯进攻游戏}

    \begin{frame}
        \frametitle{纯进攻游戏引言}
        \begin{itemize}
            \item 麻将博弈中局中人只有和牌才可能得到可观的正收益,
                因此尽快和牌的能力也就是做牌的能力十分重要.
            \item 所以我们将引入纯进攻游戏,
                通过将消息库中其余局中人的私人消息转化为牌山的消息,
                将博弈问题分解为最优化问题和计算牌山可能组合概率的问题.
        \end{itemize}
    \end{frame}

    \begin{frame}
        \frametitle{纯进攻游戏定义}
        \begin{definition}[纯进攻游戏]
            该游戏仅有一位玩家, 该玩家手中有十四张牌$T$,
            他的消息库$\base$为牌堆中牌的所有可能组合及其概率.
            每回合该玩家打出一张牌$i$并随机摸进一张牌$t$, 消息库也随之更新,
            更新函数为$\base\mapsto\mathrm{renew}_t{\base}$.
            给定限定回合数$k$,
            若该玩家在$k$回合内和牌, 则收益记为该和牌型的点数,
            若该玩家未能在$k$回合内和牌, 则收益为0.
        \end{definition}
        记消息库为$\base$时
        局中人选择切出手牌$i$在$k$回合内完成点数为$p$的和牌的概率为
        $\mathrm{val}_{k}^{p}(\hand, \base, i)$,
        则纯进攻游戏优化的目标即为期望
        \[\sum_{p}{p\cdot\mathrm{val}_{k}^{p}(\hand, \base, i)}.\]
    \end{frame}

    \begin{frame}
        \frametitle{求解纯进攻游戏思路}
        定义
        \[\mathrm{val}_{k}^{p}(\hand, \base)=
        \max_{i\in\mathbb{Z}_{14}}{\mathrm{val}_{k}^{p}(\hand, \base, i)}.\]
        那么对于每一个$p$, 都应有
        \[\mathrm{val}_{k}^{p}(\hand, \base, i)=
        \sum_{t}{\Pr_{\base}}(t)\mathrm{val}_{k-1}^{p}(\hand[i/t],
        \mathrm{renew}_{t}^{-1}{\base}),\]
        其中$\Pr_{\base}(t)$为摸进$t$的概率,
        $T[i/t]$为$T$打出$i$并摸进$t$后的手牌.
        于是我们可递归求解$\mathrm{val}_{k}^{p}(\hand, \base, i)$,
        从而解得纯进攻游戏各决策的期望收益.
    \end{frame}

    \subsection{利用隐马尔科夫模型推断条件分布}

    \begin{frame}
        \frametitle{引言}
        在纯进攻游戏中, 我们假定了局中人的知识库$\base$
        即牌堆中牌的所有可能组合及其概率是已知的.
        但事实上知识库$\base$本身是一个在公共信息确定下的条件分布,
        我们需要进一步的工作来求解.
    \end{frame}

    \begin{frame}
        \frametitle{隐马尔科夫模型的介绍}
        \begin{itemize}
            \item 隐马尔科夫模型(Hidden Markov Model, HMM)
                是结构最简单的动态贝叶斯网.
                这是一种著名的有向图模型.
            \item 隐马尔科夫模型中的变量可以分为两组:
            \begin{itemize}
            \item 第一组是状态变量$\{y_{1}, y_{2}, \dotsc, y_{n}\}$,
                其中$y_{i}\in Y$表示第$i$时刻的系统状态,
                $Y$是有$N$个取值的离散空间.
                由于在这个问题中我们考虑的$y_{i}$是对手的牌型,
                所以我们认为状态变量是不可被观测的.
            \item 第二组变量是观测变量$\{x_{1}, x_{2}, \dotsc, x_{n}\}$,
                其中$x_{i}\in X$表示第$i$时刻的观测值.
            \end{itemize}
        \end{itemize}
    \end{frame}

    \begin{frame}
        \frametitle{隐马尔科夫模型的介绍(续)}
        \begin{itemize}
            \item {[\alert{马尔科夫性}]}
                在任意时刻$t$观测变量的取值仅依赖于状态变量,
                即$x_{t}$由$y_{t}$决定,
                与其它观测变量及状态变量的取值无关.
                同时$t$时刻$y_{t}$的状态仅依赖于$y_{t-1}$.
            \item 所有变量的联合概率分布为
            \begin{equation}
                \Pr(x_{1}, y_{1}, \dotsc, x_{n}, y_{n}) =
                \Pr(y_{1})\Pr(x_{1}|y_{1})
                \prod_{i=2}^{n}\Pr(y_{i}|y_{i-1})\Pr(x_{i}|y_{i}).
            \end{equation}
        \end{itemize}
    \end{frame}

    \begin{frame}
        \frametitle{隐马尔科夫模型可以解决的问题}
        在实际应用中, 隐马尔科夫模型可以解决如下问题:
        \begin{itemize}
            \item 给定模型$\lambda=[A, B, \pi]$和观测序列$x=(x_{1}, \dotsc, x_{n})$,
                如何找到与此观测序列最匹配的状态序列$y=(y_{1}, \dotsc, y_{n})$?
                换言之, 如何根据观测序列推断出隐藏的模型状态.
            \item 上述$A$为状态转移矩阵, $B$为输出观测概率:
                模型根据当前状态获得各个观测值的概率, $\pi$为初始概率:
                模型在初始时刻各状态出现的概率.
        \end{itemize}
    \end{frame}

    \section*{}
    
    \begin{frame}
        \FrameTextResetCrono{\ResetCronoBox}
        \centering\huge{谢谢!}
    \end{frame}
\end{document}